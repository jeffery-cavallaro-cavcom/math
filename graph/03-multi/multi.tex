\documentclass[letterpaper,12pt,fleqn]{article}
\usepackage{matharticle}
\usepackage{graphtheory}
\pagestyle{empty}
\newcommand{\E}{\mathscr{E}}
\begin{document}

\section*{Multigraphs}

\begin{definition}[Multigraph]
  A \emph{multigraph} \(M=(V,E,\E,\ldots)\) is a graph with a non-empty and finite set of vertices \(V(M)\), a
  possibly empty and finite set of edges \(E(M)\), and a function \(\E\) that associates each edge with a two-element
  subset of \(V(M)\):
  \[\E:E(M)\to\ps_2(V(M))\]
\end{definition}

\begin{example}
  \begin{minipage}[t]{3in}
    \vspace{0cm}
    \begin{tikzpicture}[node distance=2]
      \begin{scope}[every node/.style=labeled node]
        \node (E) at (0,0) {\(e\)};
        \node (A) [above left=of E] {\(a\)};
        \node (B) [above right=of E] {\(b\)};
        \node (C) [below right=of E] {\(c\)};
        \node (D) [below left=of E] {\(d\)};
      \end{scope}
      \draw (A) edge node [auto] {\(e_1\)} (B);
      \draw (B) edge node [auto] {\(e_2\)} (C);
      \draw (C) edge node [auto] {\(e_3\)} (D);
      \draw (D) edge [bend left] node [auto] {\(e_4\)} (A);
      \draw (D) edge [bend right] node [auto,swap] {\(e_5\)} (A);
    \end{tikzpicture}
  \end{minipage}
  \begin{minipage}[t]{3.25in}
    \vspace{0.1in}
    \(V=V(M)=\set{a,b,c,d,e}\)

    \(E=E(M)=\set{e_1,e_2,e_3,e_4,e_5}\)
    \begin{gather*}
      e_1\mapsto\set{a,b} \\
      e_2\mapsto\set{b,c} \\
      e_3\mapsto\set{c,d} \\
      e_4\mapsto\set{a,d} \\
      e_5\mapsto\set{a,d}
    \end{gather*}
  \end{minipage}
\end{example}

\begin{definition}[Isolated Vertex]
  Let \(M\) be a multigraph and let \(u\in V(M)\).  To say that \(u\) is an \emph{isolated} vertex means that
  it is not and endpoint for any edge in \(E(M)\):
  \[\forall\,e\in E(M),u\notin\E(e)\]
\end{definition}

In the above example, \(e\) is an isolated vertex.

\begin{definition}[Parallel Edges]
  Let \(M\) be a multigraph and let \(e,f\in E(M)\).  To say that \(e\) and \(f\) are \emph{parallel} edges
  means that \(\E\) associates \(e\) and \(f\) with the same two endpoints:
  \[\E(e)=\E(f)\]
\end{definition}

In the above example, \(e_4\) and \(e_5\) are parallel edges.

\begin{definition}[Pseudograph]
  A \emph{pseudograph} \(P\) is a multigraph such that each edge is associated with either a one-element or a
  two-element subset of \(V(P)\):
  \[\E:E(P)\to\ps_1(V(P))\cup\ps_2(V(P))\]
  The single endpoint case is referred to as a \emph{loop} edge.
\end{definition}

\begin{example}
  \begin{minipage}[t]{3.25in}
    \vspace{0cm}
    \begin{tikzpicture}[node distance=2]
      \begin{scope}[every node/.style=labeled node]
        \node (E) at (0,0) {\(e\)};
        \node (A) [above left=of E] {\(a\)};
        \node (B) [above right=of E] {\(b\)};
        \node (C) [below right=of E] {\(c\)};
        \node (D) [below left=of E] {\(d\)};
      \end{scope}
      \draw (A) edge node [auto] {\(e_1\)} (B);
      \draw (B) edge node [auto] {\(e_2\)} (C);
      \draw (C) edge node [auto] {\(e_3\)} (D);
      \draw (D) edge [bend left] node [auto] {\(e_4\)} (A);
      \draw (D) edge [bend right] node [auto,swap] {\(e_5\)} (A);
      \draw (C) edge [out=345,in=285,min distance=2cm] node [auto] {\(e_6\)} (C);
    \end{tikzpicture}
  \end{minipage}
  \begin{minipage}[t]{3in}
    \vspace{0.1in}
    \(V=V(M)=\set{a,b,c,d,e}\)

    \(E=E(M)=\set{e_1,e_2,e_3,e_4,e_5,e_6}\)
    \begin{gather*}
      e_1\mapsto\set{a,b} \\
      e_2\mapsto\set{b,c} \\
      e_3\mapsto\set{c,d} \\
      e_4\mapsto\set{a,d} \\
      e_5\mapsto\set{a,d} \\
      e_6\mapsto\set{c}
    \end{gather*}
  \end{minipage}
\end{example}

In the above example, \(e_6\) is a loop edge.

\begin{definition}[Adjacent Vertices]
  Let \(M\) be a multigraph or pseudograph and let \(u,v\in V(M)\).  To say that \(u\) and \(v\) are
  \emph{adjacent} vertices (\emph{neighbors}) means that they are the endpoints of some edge \(e\in E(M)\):
  \[\exists\,e\in E(M),\E(e)=\set{u,v}\]
  The edge \(e\) is said to \emph{join} \(u\) and \(v\).  Furthermore, the edge \(e\) is said to be \emph{incident}
  to \(u\) and \(v\).
\end{definition}

Note that in the pseudograph case, a vertex \(v\in V(M)\) is adjacent to itself when:
\[\exists\,e\in E(M),\E(e)=\set{v,v}=\set{v}\]

\begin{definition}[Adjacent Edges]
  Let \(M\) be a multigraph or pseudograph and let \(e,f\in E(M)\).  To say that \(e\) and \(f\) are
  \emph{adjacent} edges means that they share an endpoint:
  \[\exists\,v\in V(M),\E(e)\cap\E(f)=\set{v}\]
  or
  \[\abs{\E(e)\cap\E(f)}=1\]
\end{definition}

\end{document}
