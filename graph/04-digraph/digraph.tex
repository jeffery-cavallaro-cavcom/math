\documentclass[letterpaper,12pt,fleqn]{article}
\usepackage{matharticle}
\usepackage{graphtheory}
\newcommand{\E}{\mathscr{E}}
\pagestyle{empty}
\begin{document}

\section*{Directed Graphs}

\begin{definition}[Directed Graph]
  A \emph{directed graph} (\emph{digraph}) \(D=(V,E,\ldots)\) is a graph with a non-empty and finite set of
  vertices \(V(D)\) and a possibly empty and finite set of \emph{directed edges} \(E(D)\) such that each directed
  edge is represented by an ordered pair of vertices from \(V(D)\):
  \[E(D)\subseteq V(D)\times V(D)\]
\end{definition}

\begin{example}
  Digraphs are often portrayed visually using filled or labeled circles for the vertices and directed lines from
  the first endpoint in the ordered pair to the second endpoint in the ordered pair.

  \begin{minipage}[t]{2.6in}
    \vspace{0cm}
    \begin{tikzpicture}[node distance=2,>=directed]
      \node (E) [coordinate] at (0,0) {\(e\)};
      \begin{scope}[every node/.style=labeled node]
        \node (A) [above left=of E] {\(a\)};
        \node (B) [above right=of E] {\(b\)};
        \node (C) [below right=of E] {\(c\)};
        \node (D) [below left=of E] {\(d\)};
      \end{scope}
      \draw (A) edge [->] (B);
      \draw (B) edge [->] (C);
      \draw (C) edge [->] (D);
      \draw (A) edge [->] [bend right] (D);
      \draw (D) edge [->] [bend right] (A);
      \draw (C) edge [out=345,in=285,min distance=2cm,->] (C);
    \end{tikzpicture}
  \end{minipage}
  \begin{minipage}[t]{3.7in}
    \vspace{0.5in}
    \(V=V(D)=\set{a,b,c,d}\)

    \(E=E(D)=\set{(a,b),(a,d),(b,c),(c,d),(d,a),(c,c)}\)
  \end{minipage}
\end{example}

\begin{notation}
  The directed edge \((u,v)\) is usually represented by just \(uv\).
\end{notation}

Note that directed multigraphs (multidigraphs) are also possible by adding an endpoint function:
\[\E:E(D)\to V(D)\times V(D)\]
to the digraph tuple.

\begin{definition}[Adjacent Vertices]
  Let \(D\) be a digraph such that \(u,v\in V(D)\) and \(e=uv\in E(D)\):
  \begin{itemize}
    \item \(u\) is said to be \emph{adjacent to} \(v\).
    \item \(v\) is said to be \emph{adjacent from} \(u\).
  \end{itemize}
  The directed edge \(e\) is said to be \emph{incident} to \(u\) and \(v\).
\end{definition}

\begin{definition}[Orientation]
  Let \(D\) be a digraph.  To say that \(D\) is an \emph{oriented} graph means that \(D\) has no loops and each
  pair of vertices in \(D\) has at most one directed edge between them:
  \[\forall\,u,v\in V(D),uv\in E(D)\implies vu\notin E(D)\]
  Furthermore, \(D\) is said to be an \emph{orientation} for the underlying simple graph.
\end{definition}

\begin{example}
  \(D\) is an oriented digraph and is an orientation of the simple graph \(G\):

  \bigskip

  \begin{minipage}[t]{3in}
    \begin{center}
      \begin{tikzpicture}[every node/.style={labeled node},node distance=2cm,>=directed]
        \node (D) at (0,0) {\(d\)};
        \node (A) [above=of D] {\(a\)};
        \node (B) [right=of A] {\(b\)};
        \node (C) [right=of D] {\(c\)};
        \draw [->] (A) edge (B);
        \draw [->] (A) edge (C);
        \draw [->] (A) edge (D);
        \draw [->] (C) edge (B);
      \end{tikzpicture}

      \bigskip

      \(D\)
    \end{center}
  \end{minipage}
  \begin{minipage}[t]{3in}
    \begin{center}
      \begin{tikzpicture}[every node/.style={labeled node},node distance=2cm,>=directed]
        \node (D) at (0,0) {\(d\)};
        \node (A) [above=of D] {\(a\)};
        \node (B) [right=of A] {\(b\)};
        \node (C) [right=of D] {\(c\)};
        \draw (A) edge (B);
        \draw (A) edge (C);
        \draw (A) edge (D);
        \draw (C) edge (B);
      \end{tikzpicture}

      \bigskip

      \(G\)
    \end{center}
  \end{minipage}
\end{example}

Note that the digraph in the first example is not an oriented digraph because \(cc\in E(D)\) and \(ad,da\in E(D)\).

\end{document}
