\documentclass[letterpaper,12pt,fleqn]{article}
\usepackage{matharticle}
\usepackage{graphtheory}
\pagestyle{empty}
\begin{document}

\section*{Walks}

\begin{definition}[Walk]
  A \(u-v\) \emph{walk} \(W\) in a graph \(G\) is a finite sequence of vertices \(w_i\in V(G)\) starting with
  \(u=w_0\) and ending with \(v=w_k\):
  \[W=(u=w_0,w_1,\ldots,w_k=v)\]
  such that \(w_iw_{i+1}\in E(G)\) for \(0\le i<k\).

  To say that \(W\) is \emph{open} means that \(u\ne v\).  To say that \(W\) is \emph{closed} means that \(u=v\).
  The \emph{length} \(k\) of \(W\) is the number of edges traversed: \(k=\abs{W}\).

  A \emph{trivial} walk is a walk of zero length --- i.e, a single vertex: \(W=(u)\).
\end{definition}

\begin{example}
  \begin{minipage}{3in}
    \begin{center}
      \begin{tikzpicture}[every node/.style={labeled node}]
        \cycleVnodes{\(a\),\(b\),\(c\),\(d\),\(e\)}{(0,0)}{0.75in}{90}{}
        \draw (1) edge (2) edge (3) edge (4) edge (5);
        \draw (2) edge (3) edge (5);
        \draw (3) edge (4);
      \end{tikzpicture}
    \end{center}
  \end{minipage}
  \begin{minipage}{3in}
    \(W_1=(a,b,e,a,c)\ \text{is open}\)

    \(W_2=(a,e,b,c,a)\ \text{is closed}\)

    \bigskip

    \(\abs{W_1}=\abs{W_2}=4\)
  \end{minipage}
\end{example}

Note that in the general case, vertices and edges are allowed to be repeated during a walk.

\begin{definition}[Special Walks]
  \begin{tabular}{lll}
    \emph{trail} & An open walk with no repeating edges & \((a,b,c,a,e)\) \\
    \\
    \emph{path} & A trail with no repeating vertices & \((a,e,b,c)\) \\
    \\
    \emph{circuit} & A closed trail & \((a,b,e,a,c,d,a)\) \\
    \\
    \emph{cycle} & A closed path & \((a,e,b,c,a)\)
  \end{tabular}
\end{definition}

\begin{notation}[Concatenation]
  Let \(G\) be a graph and let \(u,v,w\in V(G)\) such that \(W_1=(u=u_0,u_1,\ldots,u_k=v)\) is a \(u-v\) walk of
  length \(k\) in \(G\) and \(W_2=(v=v_0,v_1,\ldots,v_{\ell}=w)\) is a \(v-w\) walk of length \(\ell\) in \(G\)
  with common endpoint \(v\).  The \emph{concatenation} of these two walks \(W\) given by:
  \[W=W_1\cup W_2=(u,\ldots,v,\ldots,w)\]
  is a \(u-w\) walk in \(G\) of length \(k+\ell\).
\end{notation}

Note that the concatenation of two paths is a walk, but not necessarily a path.  In the above example, let
\(P_1=(a,e,b)\) and \(P_2=(b,a,c,d)\):
\[P_1\cup P_2=(a,e,b,a,c,d)\]
which is not a path due to vertex \(a\) being traversed twice.

\begin{theorem}
  Let \(G\) be a graph and let \(u,v\in V(G)\):
  \begin{quote}
    \(G\) contains a \(u-v\) walk of length \(k\implies G\) contains a \(u-v\) path of length \(\ell\le k\).
  \end{quote}
\end{theorem}

\begin{proof}
  Assume \(G\) contains a \(u-v\) walk of length \(k\).

  Consider the set of all \(u-v\) walks in \(G\).  Their lengths form a non-empty set of positive integers.  By the
  well-ordering principle, there exists a \(u-v\) walk \(P\) of minimal length \(\ell\le k\):
  \[P=(u=w_0,\ldots,w_{\ell}=v)\]
  Claim: \(P\) is a path.

  ABC: \(P\) is not a path, and thus \(P\) has at least one repeating vertex.

  Assume \(w_i=w_j\) for some \(0\le i<j\le\ell\):
  \begin{description}
  \item Case 1: \(j=\ell\)

    \(P'=(u=w_0,\ldots,w_i=v)\) is a \(u-v\) walk in \(G\) of length \(i<\ell\).

  \item Case 2: \(j<\ell\)
    
    \(P'=(u=w_0,\ldots,w_i,w_{j+1},\ldots,w_{\ell}=v)\) is a \(u-v\) walk in \(G\) of length \(\ell-(j-i)<\ell\)
    
  \end{description}

  Both cases contradict the minimality of the length of \(P\).

  \(\therefore P\) is a \(u-v\) path in \(G\) of length \(\ell\le k\).
\end{proof}

\begin{definition}[Connected]
  Let \(G\) be a graph and let \(u,v\in V(G)\).  To say that \(u\) and \(v\) are \emph{connected} means that \(G\)
  contains a \(u-v\) path.
\end{definition}

\begin{definition}[Cycles]
  Let \(C\) be a cycle in a graph \(G\):
  \begin{itemize}
  \item To say that \(C\) is a \emph{k-cycle} means that \(\abs{C}=k\).
  \item To say that \(C\) is an \emph{even} cycle means that \(\abs{C}\) is even.
  \item To say that \(C\) is an \emph{odd} cycle means that \(\abs{C}\) is odd.
  \end{itemize}
\end{definition}

Note that in simple graphs, circuits and cycles must have length \(\ge3\).

\end{document}
