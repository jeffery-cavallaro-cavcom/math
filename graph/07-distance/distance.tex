\documentclass[letterpaper,12pt,fleqn]{article}
\usepackage{matharticle}
\usepackage{graphtheory}
\usepackage{adjustbox}
\pagestyle{empty}
\begin{document}

\section*{Distance}

\begin{definition}[Distance]
  Let \(G\) be a graph and let \(u,v\in V(G)\) such that \(u\) and \(v\) are connected:
  \begin{itemize}
  \item The \emph{distance} between \(u\) and \(v\), denoted by \(d_G(u,v)\) or \(d(u,v)\), is the length of the
    shortest \(u-v\) path:
    \[d_G(u,v)=d(u,v)=\min\setb{\abs{P}}{P\ \text{is a}\ u-v\ \text{path in}\ G}\]
  \item To say that a \(u-v\) path \(P\) in \(G\) is \emph{geodesic} means that:
    \[\abs{P}=d_G(u,v)\]
  \item The \emph{diameter} of \(G\) is the greatest distance between any two vertices in \(G\):
    \[\diam(G)=\max\setb{d_G(u,v)}{u,v\in V(G)}\]
  \end{itemize}
\end{definition}

For a graph \(G\) or order \(n\):
\begin{itemize}
\item \(d_G(u,v)=0 \iff u=v\)
\item \(d_G(u,v)=1 \iff uv\in E(G)\)
\item \(0\le d(u,v)<n\)
\end{itemize}

\begin{example}
  \begin{minipage}{4in}
    \begin{center}
      \begin{tikzpicture}[every node/.style={labeled node}]
        \cycleVnodes{\(a\),\(b\),\(c\),\(d\),\(e\)}{(0,0)}{0.75in}{90}{}
        \draw (1) edge (2) edge (4) edge (5);
        \draw (2) edge (4) edge (5);
        \draw (3) edge (4);
      \end{tikzpicture}
      
      \(G\)
    \end{center}
  \end{minipage}
  \begin{minipage}{2in}
    \(\diam(G)=3\)

    \bigskip

    \(*=\) geodesic
  \end{minipage}

  \bigskip

  \begin{adjustbox}{width=\textwidth}
    \begin{tabular}{c|c|c|c|c|c|c|c|c|c}
      \(a-b\) & \(a-c\) & \(a-d\) & \(a-e\) &\(b-c\) & \(b-d\) & \(b-e\) & \(c-d\) & \(c-e\) & \(d-e\) \\
      \hline
      \((a,b)^*\) & \((a,b,d,c)\) & \((a,b,d)\) & \((a,b,e)\) & \((b,a,d,c)\) & \((b,a,d)\) & \((b,a,e)\) &
      \((c,d)^*\) &
      \((c,d,a,b,e)\) & \((d,a,b,e)\) \\
      \((a,d,b)\) & \((a,d,c)^*\) & \((a,d)^*\) & \((a,d,b,e)\) & \((b,d,c)^*\) & \((b,d)^*\) & \((b,d,a,e)\) & &
      \((c,d,a,e)^*\) & \((d,a,e)^*\) \\
      \((a,e,b)\) & \((a,e,b,d,c)\) & \((a,e,b,d)\) & \((a,e)^*\) & \((b,e,a,d,c)\) & \((b,e,a,d)\) & \((b,e)^*\) & &
      \((c,d,b,a,e)\) & \((d,b,a,e)\) \\
      & & & & & & & & \((c,d,b,e)^*\) & \((d,b,e)^*\) \\
      \hline
      \(d=1\) & \(d=2\) & \(d=1\) & \(d=1\) & \(d=2\) & \(d=1\) & \(d=1\) & \(d=1\) & \(d=3\) & \(d=2\) \\
    \end{tabular}
  \end{adjustbox}
\end{example}

\bigskip

\begin{theorem}
  Let \(G\) be a graph with \(u,v\in V(G)\) and let \(P=(u=v_0,v_1,\ldots,v_k=v)\) be a \(u-v\) geodesic in \(G\).
  For all \(i\) such that \(0\le i\le k\):
  \[d(u,v_i)=i\]
\end{theorem}

\begin{proof}
  Assume \(0\le i\le k\).

  Since \((u=v_0,v_1,\ldots,v_i)\) is a \(u-v_i\) path of length \(i\) in \(G\), it must be that case that
  \(d(u,v_i)\le i\).

  ABC: There exists a shorter \(u-v_i\) path in \(G\): \((u=w_0,w_1,\ldots,w_{\ell}=v_i)\) for \(\ell<i\).

  Let \(W=(u=w_0,w_1,\ldots,w_{\ell}=v_i,\ldots v_k=v)\).  \(W\) is a \(u-v\) walk in \(G\) of length:
  \[\ell+(k-i)=k-(i-\ell)<k\]
  So there exists a \(u-v\) path in \(G\) of length \(<k\), contradicting the minimality of \(P\).

  \(\therefore d(u,v_i)=i\)
\end{proof}

\begin{theorem}
  Let \(G\) be a graph with \(u,v\in V(G)\) such that \(d(u,v)=\diam(G)\).  \(\forall\,w\in V(G),w\ne u,v\):
  \begin{quote}
    \(v\) appears in no \(u-w\) geodesic in \(G\).
  \end{quote}
\end{theorem}

\begin{proof}
  Assume \(w\in V(G),w\ne u,v\) and \(P\) is a \(u-w\) geodesic in \(G\).

  ABC: \(v\) is contained in \(P\).

  Then \(P=(u,\ldots,v,\ldots,w)\) and the \(u-v\) subpath must be geodesic.  But \(u,v\ne w\), and so
  \(d(u,w)>d(u,v)\), contradicting the maximality of \(d(u,v)\).

  \(\therefore v\) cannot appear in any \(u-w\) geodesic in \(G\).
\end{proof}

\end{document}

