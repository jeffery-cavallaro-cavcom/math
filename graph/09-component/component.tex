\documentclass[letterpaper,12pt,fleqn]{article}
\usepackage{matharticle}
\usepackage{graphtheory}
\pagestyle{empty}
\newcommand{\G}{\mathscr{G}}
\begin{document}

\section*{Components}

\begin{definition}[Component]
  Let \(G\) be a graph and let \(\G\) be the set of all connected subgraphs of \(G\).  To say that a graph \(H\in\G\)
  is a \emph{component} of a \(G\) means that \(H\) is not a subgraph of any other connected subgraph of \(G\):
  \[\forall\,F\in\G-\set{H},H\not\subset F\]
  The number of distinct components in \(G\) is denoted by:
  \[k=k(G)\]
  For a connected graph: \(k(G)=1\).
\end{definition}

\begin{notation}[Union]
  Let \(G\) be a graph and let \(\G=\set{G_i:1\le i\le k}\) for some \(k\in\N\) be a set of subgraphs of \(G\) such
  that each vertex and each edge in \(G\) is present in exactly one \(G_i\):
  \[G=\bigcup_{1\le i\le k}G_i\]
  Note that the components of a graph are suitable choices for the \(G_i\).
\end{notation}

\begin{lemma}
  Let \(G\) be a graph and let \(G_i\) be a component of \(G\):
  \begin{quote}
    \(G_i\) is an induced subgraph of \(G\).
  \end{quote}
\end{lemma}

\begin{proof}
  By definition, \(G_i\) is a maximal connected subgraph of \(G\).

  ABC: \(G_i\) is not an induced subgraph of \(G\).

  Thus, \(G_i\) is missing some edges that when added would result in a connected induced subgraph \(H\) of \(G\).
  But then \(G_i\subset H\), contradicting the maximality of \(G_i\).

  \(\therefore G_i\) is an induced subgraph of \(G\).
\end{proof}

\begin{theorem}
  Let \(G\) be a graph and define a relation \(\sim\) on \(V(G)\) as follows:
  \[\forall\,u,v\in V(G),u\sim v\iff u\ \text{and}\ v\ \text{are connected}\]
  \(\sim\) is an equivalence relation.
\end{theorem}

\begin{proof}
  Assume \(u,v,w\in V(G)\):
  \begin{description}
  \item{R:} \(u\) is connected to \(u\) by the trivial path \((u)\).

    \(\therefore u\sim u\)

  \item{S:} Assume \(u\sim v\):

    There exists a \(u-v\) path in \(G\). \\
    And so, going in reverse order, there exists a \(v-u\) path in \(G\).

    \(\therefore v\sim u\)
    
  \item{T:} Assume \(u\sim v\) and \(v\sim w\):

    There exists a \(u-v\) path and a \(v-w\) path in \(G\). \\
    So there exists a \(u-w\) walk in \(G\). \\
    And thus there must exist a \(u-w\) path in \(G\).

    \(\therefore u\sim w\)
  \end{description}
\end{proof}

\begin{theorem}
  Let \(G\) be a graph and let \(G_i\) be a subgraph of \(G\).  TFAE:
  \begin{enumerate}
  \item \(G_i\) is a component of \(G\).
  \item \(G_i\) is induced by an equivalence class of the connectedness relation.
  \end{enumerate}
\end{theorem}

\begin{proof}
  \begin{description}
  \item[]
  \item[\(\implies\)] Assume \(G_i\) is a component of \(G\).

    So \(G_i\) is a maximal connected induced subgraph of \(G\).

    ABC: \(V(G_i)\) is not an equivalence class of the connectedness relation.

    Thus, \(V(G_i)\) must be a proper subset of some equivalence class \(V_i\) and \(G[V_i]\) is an connected
    induced subgraph of \(G\) such that \(G_i\subset G[V_i]\), contradicting the maximality of \(G_i\).

    \(\therefore G_i\) is induced by an equivalence class of the connectedness relation.

  \item[\(\impliedby\)] Assume \(G_i\) is induced by an equivalence class of the connectedness
    relation.

    By definition, \(G_i\) is a connected subgraph of \(G\).

    ABC: \(G_i\) is not maximal.

    Thus, \(G_i\) is a proper subgraph of some connected subgraph \(H\) of \(G\) and \(V(G_i)\subset V(H)\),
    contradicting the definition of \(V(G_i)\) as an equivalence class.

    \(\therefore G_i\) is a component of \(G\).
  \end{description}
\end{proof}

\begin{corollary}
  Let \(G\) be a graph with \(k\) components.  Each vertex and each edge in \(G\) belong to exactly one component
  of \(G\).
\end{corollary}

\begin{proof}
  Each \(v\in V(G)\) is in exactly one equivalence class and hence in exactly one component.  Furthermore, the
  endpoints of each edge are also in the same equivalence class and thus must exist in the same component, forcing
  the edge into that component as well.
\end{proof}

\begin{corollary}
  Let \(G\) be a graph with \(k\) components and let \(u,v\in V(G)\) such that \(u\in G_i\) and \(v\in G_j\) for
  \(1\le i,j\le k\):
  \[i\ne j\implies uv\notin E(G)\]
\end{corollary}

\begin{proof}
  Assume \(uv\in E(G)\).

  This means that \(u\) and \(v\) are connected and thus must be in the same component.

  \(\therefore i=j\)
\end{proof}

\end{document}
