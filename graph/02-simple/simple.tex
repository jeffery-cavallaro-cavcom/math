\documentclass[letterpaper,12pt,fleqn]{article}
\usepackage{matharticle}
\usepackage{graphtheory}
\pagestyle{empty}
\begin{document}

\section*{Simple Graphs}

\begin{definition}[Simple Graph]
  A \emph{simple} graph \(G=(V,E,\ldots)\) is a graph with a non-empty and finite set of vertices \(V(G)\) and a
  possibly empty and finite set of edges \(E(G)\) such that each edge is represented by a two-element subset of
  \(V(G)\):
  \[E(G)\subseteq\ps_2(V(G))\]
\end{definition}

In particular, a simple graph is never the null graph, has no loops, and has no multiple edges.

\begin{examples}
  \begin{minipage}[t]{2.5in}
    \(V=V(G)=\set*{1,2,3,4}\) \\
    \(E=E(G)=\set[\big]{\set{1,2},\set{2,3}}\)
    
    \bigskip

    \begin{center}
      \begin{tikzpicture}[every node/.style={labeled node}]
        \cycleVnodes{\(1\),\(2\),\(3\),\(4\)}{(0,0)}{1.75cm}{90}{}
        \draw (1) -- (2) -- (3);
      \end{tikzpicture}
    \end{center}
  \end{minipage}
  \hspace{0.5in}
  \begin{minipage}[t]{3in}
    \(V=\set{a,b,c,d,e}\) \\
    \(E=\set[\big]{\set{a,b},\set{a,c},\set{a,d},\set{a,e},\set{b,e},\set{c,d}}\)
      
    \bigskip
      
    \begin{center}
      \begin{tikzpicture}[every node/.style={labeled node}]
        \cycleVnodes{\(a\),\(b\),\(c\),\(d\),\(e\)}{(0,0)}{2cm}{90}{}
        \draw (1) -- (2) -- (5) -- (1);
        \draw (1) -- (3) -- (4) -- (1);
      \end{tikzpicture}
    \end{center}
  \end{minipage}
\end{examples}

\begin{notation}
  The edge \(\set{u,v}\) is usually represented by just \(uv\).
\end{notation}

\begin{definition}[Isolated Vertex]
  Let \(G\) be a simple graph and let \(u\in V(G)\).  To say that \(v\) is an \emph{isolated} vertex means that
  it is not an endpoint for any edge in \(E(G)\):
  \[\forall\,e\in E(G),v\notin e\]
\end{definition}

In the above example, vertex \(4\) is an isolated vertex.

\begin{definition}[Adjacent Vertices]
  Let \(G\) be a simple graph and let \(u,v\in V(G)\).  To say that \(u\) and \(v\) are \emph{adjacent} vertices
  (\emph{neighbors}) means that they are the endpoints of some edge \(e\in E(G)\):
  \[\exists\,e\in E(G),e=uv\]
  The edge \(e\) is said to \emph{join} \(u\) and \(v\).  Furthermore, the edge \(e\) is said to be \emph{incident}
  to \(u\) and \(v\).
\end{definition}

\begin{definition}[Adjacent Edges]
  Let \(G\) be a simple graph and let \(e,f\in E(G)\).  To say that \(e\) and \(f\) are \emph{adjacent} edges means
  that they share an endpoint:
  \[\exists\,v\in V(G),e\cap f=\set{v}\]
  or
  \[\abs{e\cap f}=1\]
\end{definition}

\begin{example}
  Let \(G\) be a simple graph; \(u,v,w\in V(G)\); and \(e,f\in E(G)\) such that \(e=uv\) and \(f=vw\):

  \begin{minipage}{2.5in}
    \vspace{0in}
    \begin{tikzpicture}
      \draw (0,0) ellipse [x radius=3,y radius=2];
      \node [labeled node] (V) at (0,-0.75) {v};
      \node [labeled node] (U) at (-1.5,0.5) {u};
      \node [labeled node] (W) at (1.5,0.5) {w};
      \draw (U) to node [auto,swap] {e} (V) to node [auto,swap] {f} (W);
      \node at (-1.5,-1.25) {G};
    \end{tikzpicture}
  \end{minipage}
  \begin{minipage}{3.5in}
    \begin{itemize}
    \item \(u\) and \(v\) are adjacent vertices (neighbors).
    \item \(u\) and \(v\) are joined by \(e\).
    \item \(u\) and \(e\) are incident.
    \item \(e\) and \(f\) are adjacent edges.
    \end{itemize}
  \end{minipage}
\end{example}

\begin{definition}[Equality]
  To say that two simple graphs \(G\) and \(H\) are \emph{equal}, denoted by \(G=H\), means that \(V(G)=V(H)\) and
  \(E(G)=E(H)\).
\end{definition}

\begin{theorem}
  Let \(G\) be a simple graph of order \(n\) and size \(m\):
  \[ m\le\frac{n(n-1)}{2} \]
\end{theorem}

\begin{proof}
  Since the graph is simple, each pair of distinct vertices has at most one edge joining them, and so the maximum
  number of possible edges is \(\binom{n}{2}\).  Hence:
  \[m\le\binom{n}{2}=\frac{n!}{2!(n-2)!}=\frac{n(n-1)}{2}\]
\end{proof}

\end{document}
