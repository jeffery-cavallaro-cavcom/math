\documentclass[letterpaper,12pt,fleqn]{article}
\usepackage{matharticle}
\usepackage{graphtheory}
\pagestyle{empty}
\renewcommand{\d}{\delta}
\newcommand{\D}{\Delta}
\begin{document}
\section*{2.1: The Degree of a Vertex}

\begin{enumerate}
\item Give an example of the following or explain why no such example exists:
  \begin{enumerate}
  \item a graph of order 7 whose vertices have degrees 1, 1, 1, 2, 2, 3, 3.

    Not possible due to an odd number of odd vertices.

  \item a graph of order 7 whose vertices have degrees 1, 2, 2, 2, 3, 3, 7.

    Not possible because \(\D(G)\le7-1=6\), so a vertex with degree 7 cannot exist.

  \item a graph of order 4 whose vertices have degrees 1, 3, 3, 3.

    Not possible because 3 of the 4 vertices are universal, and thus are all adjacent to the remaining vertex, which
    must also have degree \(3\ne1\).
  \end{enumerate}

\item Give an example of the following or explain why no such example exists:
  \begin{enumerate}
  \item a graph that has no odd vertices.
    \[C_n\]
  \item a non-complete graph, all of whose vertices have degree 3.
    \[K_{3,3}\]
  \item a graph \(G\) of order 5 or more with the property that \(\deg(u)\ne\deg(v)\) for every pair \(u,v\) of
    adjacent vertices of \(G\).
    \begin{quote}
      \begin{tikzpicture}[every node/.style={unlabeled node}]
        \cycleNnodes{4}{(0,0)}{0.5in}{135}{c};
        \node (v) at (0,0) {};
        \foreach \i in {1,2,3,4}{
          \draw (v) edge (c\i);
        }
      \end{tikzpicture}
    \end{quote}

  \item A non-complete graph \(H\) of order 5 or more with the property that \(\deg(u)\ne\deg(v)\) for every pair
    \(u,v\) of non-adjacent vertices in \(H\).
    \begin{quote}
      \begin{tikzpicture}[every node/.style={unlabeled node}]
        \cycleNnodes{4}{(0,0)}{0.5in}{135}{c};
        \node (v) at (0,0) {};
        \draw (c1) edge (c2) edge (v) edge (c4);
        \draw (c2) edge (v) edge (c3);
      \end{tikzpicture}
    \end{quote}
  \end{enumerate}

\item The degree of each vertex of a certain graph of order 12 and size 31 is either 4 or 6.  How many vertices of
  degree 4 are there?

  Let \(x=\) the number of vertices with degree 4:
  \begin{gather*}
    4x+6(12-x)=2\cdot31 \\
    4x+72-6x=62 \\
    2x=10 \\
    x=5
  \end{gather*}

\item Give an example of a graph \(G\) of order 6 and size 10 such that \(\d(G)=3\) and \(\D(G)=4\).
  \begin{quote}
    \begin{tikzpicture}[every node/.style={unlabeled node}]
      \cycleNnodes{6}{(0,0)}{0.5in}{90}{};
      \draw (1) edge (2) edge (3) edge (5) edge (6);
      \draw (4) edge (2) edge (3) edge (5) edge (6);
      \draw (2) edge (6);
      \draw (3) edge (5);
    \end{tikzpicture}
  \end{quote}

\item The degree of every vertex of a graph \(G\) of order 25 and size 62 is 3, 4, 5, or 6.  There are two vertices
  of degree 4 and 11 vertices of degree 6.  How many vertices of \(G\) have degree 5?

  Let \(x=\) the number of vertices with degree 5:
  \begin{gather*}
    3(25-2-x-11)+4\cdot2+5x+11\cdot6=2\cdot62 \\
    3(12-x)+8+5x+66=124 \\
    36-3x+5x+74=124 \\
    2x=14 \\
    x=7
  \end{gather*}

\item Prove that if a graph of order \(3n\) (\(n\ge1\)) has \(n\) vertices each of the degrees \(n-1\),\(n\), and
  \(n+1\), then \(n\) is even.

  \begin{proof}
    \begin{gather*}
      n(n-1)+n(n)+n(n+1)=2m \\
      n^2-n+n^2+n^2+n=2m \\
      3n^2=2m
    \end{gather*}
    Thus, \(3n^2\) must be even, and so \(n^2\) must be even.

    \(\therefore n\) must be even.
  \end{proof}
\end{enumerate}

\end{document}
