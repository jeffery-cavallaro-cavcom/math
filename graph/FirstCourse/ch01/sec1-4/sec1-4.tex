\documentclass[letterpaper,12pt,fleqn]{article}
\usepackage{matharticle}
\usepackage{graphtheory}
\pagestyle{empty}
\begin{document}
\section*{1.4: Multigraphs and Digraphs}

\begin{enumerate}[start=29]
\item Let \(S=\set{2,3,4,7,11,13}\).
  \begin{enumerate}
    \item Construct the multigraph \(M\) whose vertex set is \(S\) and where \(ij\) is an edge for distinct
      elements \(i\) and \(j\) in \(S\) whenever \(i+j\) and \(ij\) is an edge whenever \(\abs{i-j}\in S\).  In
      other words, \(i\) and \(j\) are joined by two edges if both \(i+j\in S\) and \(\abs{i-j}\in S\).

      \bigskip

      \begin{center}
        \begin{tikzpicture}[every node/.style={labeled node}]
          \cycleVnodes{\(2\),\(3\),\(4\),\(7\),\(11\),\(13\)}{(0,0)}{1in}{90}{};
          \draw (1) edge (3) edge (5) edge (6);
          \draw (2) edge (3) edge (4);
          \draw (3) edge (4) edge [out=240,in=0] (4) edge (5);
          \draw (4) edge (5);
          \draw (5) edge (6);
        \end{tikzpicture}
      \end{center}

    \item How are the problem and solution in (a) affected if we remove the word ``distinct.''

      This allows for loop edges, so add a loop on vertex \(2\).

      \bigskip

      \begin{center}
        \begin{tikzpicture}[every node/.style={labeled node}]
          \cycleVnodes{\(2\),\(3\),\(4\),\(7\),\(11\),\(13\)}{(0,0)}{1in}{90}{};
          \draw (1) edge (3) edge (5) edge (6);
          \draw (2) edge (3) edge (4);
          \draw (3) edge (4) edge [out=240,in=0] (4) edge (5);
          \draw (4) edge (5);
          \draw (5) edge (6);
          \draw (1) edge [out=45,in=135,min distance=1in] (1);
        \end{tikzpicture}
      \end{center}
  \end{enumerate}

  \bigskip

\item Consider the twelve configurations \(c_i,1\le i\le12\), in Figure 1.38.  Draw the digraph \(D\), where
  \(V(D)=\set{c_1,c_2,\ldots,c_{12}}\) and where \((c_i,c_j)\) is a directed edge of \(D\) if it is possible to obtain
  \(c_j\) by rotating the configuration \(c_i\) either \(90^{\circ}\) or \(180^{\circ}\) clockwise about the midpoint of
  the checkerboard.
  
  \bigskip

  \begin{center}
    \begin{tikzpicture}[every node/.style={labeled node,minimum size=1cm},>=directed]
      \cycleVnodes{
        \(c_1\),\(c_2\),\(c_3\),\(c_4\),\(c_5\),\(c_6\),\(c_7\),\(c_8\),\(c_9\),\(c_{10}\),\(c_{11}\),\(c_{12}\)
      }{(0,0)}{2in}{90}{c};
      \draw [->] (c1) edge (c4) edge [bend right=20] (c7);
      \draw [->] (c2) edge (c5) edge [bend right=20] (c9);
      \draw [->] (c3) edge (c6) edge [bend right=20] (c8);
      \draw [->] (c4) edge (c7) edge [bend right=20] (c10);
      \draw [->] (c5) edge (c9) edge [bend right=20] (c12);
      \draw [->] (c6) edge (c8) edge [bend right=20] (c11);
      \draw [->] (c7) edge (c10) edge [bend right=20] (c1);
      \draw [->] (c8) edge (c11) edge [bend right=20] (c3);
      \draw [->] (c9) edge (c12) edge [bend right=20] (c2);
      \draw [->] (c10) edge (c1) edge [bend right=20] (c4);
      \draw [->] (c11) edge (c3) edge [bend right=20] (c6);
      \draw [->] (c12) edge (c2) edge [bend right=20] (c5);
    \end{tikzpicture}
  \end{center}

  \bigskip

\item Using the twelve configurations in Figure 1.38, define a transformation different from the one described in
  Exercise 1.30 which can be modeled by a digraph but not by a graph.

  First, move any coin in the upper row to the right (if possible), and then move the leftmost coin in the lower
  row up (if possible).
  
  \bigskip

  \begin{center}
    \begin{tikzpicture}[every node/.style={labeled node,minimum size=1cm},>=directed]
      \cycleVnodes{
        \(c_1\),\(c_2\),\(c_3\),\(c_4\),\(c_5\),\(c_6\),\(c_7\),\(c_8\),\(c_9\),\(c_{10}\),\(c_{11}\),\(c_{12}\)
      }{(0,0)}{2in}{90}{c};
      \draw [->] (c1) edge (c3);
      \draw [->] (c2) edge (c5);
      \draw [->] (c3) edge [out=30,in=330,min distance=1in] (c3);
      \draw [->] (c4) edge (c5);
      \draw [->] (c5) edge [out=30,in=330,min distance=1in] (c5);
      \draw [->] (c6) edge (c7);
      \draw [->] (c7) edge (c9);
      \draw [->] (c8) edge (c11);
      \draw [->] (c9) edge [out=150,in=210,min distance=1in] (c9);
      \draw [->] (c10) edge (c11);
      \draw [->] (c11) edge [out=150,in=210,min distance=1in] (c11);
      \draw [->] (c12) edge (c1);
    \end{tikzpicture}
  \end{center}

  \bigskip

\item Let \(S\) and \(A\) be two finite nonempty sets of integers.  Define a digraph \(D\) with \(V(D)=A\), where
  \((x,y)\) is an arc of \(D\) if \(x\ne y\) and \(y-x\in S\).
  \begin{enumerate}
  \item Draw the digraph \(D\) for \(A=\set{0,1,2,3,4}\) and \(S=\set{-2,1,2,4}\).

    \bigskip

    \begin{center}
      \begin{tikzpicture}[every node/.style={labeled node,minimum size=1cm},>=directed]
        \cycleVnodes{\(0\),\(1\),\(2\),\(3\),\(4\)}{(0,0)}{1.5in}{90}{n};
        \draw [->] (n1) edge (n2) edge [bend right=10] (n3) edge (n5);
        \draw [->] (n2) edge (n3) edge [bend right=10] (n4);
        \draw [->] (n3) edge [bend right=10] (n1) edge (n4) edge [bend right=10] (n5);
        \draw [->] (n4) edge [bend right=10] (n2) edge (n5);
        \draw [->] (n5) edge [bend right=10] (n3);
      \end{tikzpicture}
    \end{center}

    \bigskip

  \item What can be said about \(D\) if \(A\) and \(S\) consist only of odd integers?

    \(D\) will be empty because the difference of two odds is always even.

  \item How can the question in (b) be generalized?

    \(D\) will be empty whenever \(S\) is all odd integers and \(A\) is either all odd or all even integers.

  \item If \(\abs{A}=\abs{S}=5\), how large can the size of \(D\) be?

    \(m=14\)

    For example: \(A=\set{0,1,2,3,4}\) and \(S=\set{-1,1,2,3,4}\).

    \bigskip

    \begin{center}
      \begin{tikzpicture}[every node/.style={labeled node,minimum size=1cm},>=directed]
        \cycleVnodes{\(0\),\(1\),\(2\),\(3\),\(4\)}{(0,0)}{2in}{90}{n};
        \draw [->] (n1) edge [bend right=10] (n2) edge (n3) edge (n4) edge (n5);
        \draw [->] (n2) edge [bend right=10] (n1) edge [bend right=10] (n3) edge (n4) edge (n5);
        \draw [->] (n3) edge [bend right=10] (n2) edge [bend right=10] (n4) edge (n5);
        \draw [->] (n4) edge [bend right=10] (n3) edge [bend right=10] (n5);
        \draw [->] (n5) edge [bend right=10] (n4);
      \end{tikzpicture}
    \end{center}
  \end{enumerate}

  \bigskip

\item A digraph \(D\) has vertex set \(\set{-3,3,6,12}\) and \((i,j)\in D\) if \(i\ne j\) and \(i\mid j\), that is,
  \(j\) is a multiple of \(i\).  Draw the digraph \(D\).
    \bigskip

    \begin{center}
      \begin{tikzpicture}[every node/.style={labeled node,minimum size=1cm},>=directed]
        \cycleVnodes{\(-3\),\(3\),\(6\),\(12\)}{(0,0)}{2in}{90}{n};
        \draw [->] (n1) edge [bend right=10] (n2) edge (n3) edge (n4);
        \draw [->] (n2) edge [bend right=10] (n1) edge (n3) edge (n4);
        \draw [->] (n3) edge (n4);
      \end{tikzpicture}
    \end{center}
\end{enumerate}

\end{document}
