\documentclass[letterpaper,12pt,fleqn]{article}
\usepackage{matharticle}
\usepackage{siunitx}
\pagestyle{plain}
\begin{document}

\begin{center}
  \large
  Math-13 Sections 01 and 02

  \Large
  Homework \#5

  \large
  \textbf{Due: Midnight 9/25}
\end{center}

Consider the following function:

\bigskip

\begin{center}
  \begin{tikzpicture}
    \draw [very thin, gray] (-5,-5) grid (5,5);
    \draw [very thick] (-5,0) -- (5,0) node [right] {\(5\)};
    \draw [very thick] (0,-5) -- (0,5) node [above] {\(5\)};
    \node [left] at (-5,0) {\(-5\)};
    \node [below] at (0,-5) {\(-5\)};
    \node [closed point] (A) at (-4,0) {};
    \node [closed point] (B) at (-2,4) {};
    \node [open point] (C) at (0,0) {};
    \node [closed point] (D) at (0,-2) {};
    \node [open point] (E) at (4,0.7) {};
    \node [closed point] (F) at (4,2) {};
    \draw (A) to (B);
    \draw (C) parabola (B);
    \draw (D) parabola (1.9,5);
    \draw (5,0.1) parabola (2.1,5);
  \end{tikzpicture}
\end{center}

\bigskip

Answer the following for the points at \(c=-4, -2, 0, 2, 4\):

\begin{enumerate}[label={\alph*)}]
  \item Determine \(\displaystyle\lim_{x\to c^-}f(x)\)
  \item Determine \(\displaystyle\lim_{x\to c^+}f(x)\)
  \item Determine \(\displaystyle\lim_{x\to c}f(x)\)
  \item State whether or not the function is continuous at the point.  If not, then state why.
\end{enumerate}

\end{document}
