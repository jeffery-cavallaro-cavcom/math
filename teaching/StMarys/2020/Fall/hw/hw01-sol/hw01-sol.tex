\documentclass[letterpaper,12pt,fleqn]{article}
\usepackage{matharticle}
\pagestyle{plain}

\newcommand{\cm}{\checkmark}

\begin{document}

\begin{center}
  \large
  Math-13 Sections 01 and 02

  \Large
  Homework \#1 Solutions
\end{center}

\begin{enumerate}
  \setlength\itemsep{5ex}
\item Indicate which subsets of the real numbers that the following values are members of:

  \bigskip

  \begin{center}
    \renewcommand{\arraystretch}{1.2}
    \begin{tabular}{|c||c|c|c|c|c|}
      \hline
      VALUE & \(\N\) & \(\Z\) & \(\Q\) & \(\R-\Q\) & \(\R\) \\
      \hline
      \hline
      \(0\) & & \cm & \cm & & \cm \\
      \hline
      \(\sqrt{2}\) & & & & \cm & \cm \\
      \hline
      \(\sqrt{9}\) & \cm & \cm & \cm & & \cm \\
      \hline
      \(123.4\) & & & \cm & & \cm \\
      \hline
      \(-123\) & & \cm & \cm & & \cm \\
      \hline
      \(\pi\) & & & & \cm & \cm \\
      \hline
      \(12.34\overline{56}\) & & & \cm & & \cm \\
      \hline
    \end{tabular}
  \end{center}

  \bigskip

\item Convert the rational value \(12.34\overline{56}\) to integer ratio form using the algorithm that we learned
  in class.
  \begin{enumerate}
  \item \(x=12.34\overline{56}\).  The goal is to find the alternate syntax for \(x\).
  \item Capture all non-repeating digits: \(100x=1234.\overline{56}\).
  \item Capture one set of repeated digits: \(10000x=123456.\overline{56}\).
  \item Subtract the two equations (note that the repeating part cancels) and solve for \(x\):
    \begin{gather*}
      (10000-100)x=123456-1234 \\
      9900x=122222 \\
      x=\frac{122222}{9900}
    \end{gather*}
  \end{enumerate}
  Therefore, \(12.34\overline{56}=\frac{122222}{9900}\).

\item Convert \(24.57\overline{9}\) to integer ratio form \emph{without} using the algorithm and justify your
  answer.  (Hint: look for an alternate syntax for the value.)

  Note that \(24.57\overline{9}\) becomes arbitrarily close to \(24.58\), and is thus equal to it.  Now, use the
  algorithm to convert this finite decimal value to an integer ratio:
  \[24.58\left(\frac{100}{100}\right)=\frac{2458}{100}\]

\item Is \(\displaystyle\frac{123.4}{56.99}\) a rational number?  If so, then explain why \emph{without} using the
  algorithm.

  Note that:
  \[\frac{123.4}{56.99}=\frac{\frac{1234}{10}}{\frac{5699}{100}}=
  \left(\frac{1234}{10}\right)\left(\frac{100}{5699}\right)=\frac{12340}{5699}\]
  This is an integer ratio where the denominator is not zero, and is therefore a rational number.
\end{enumerate}

\end{document}
