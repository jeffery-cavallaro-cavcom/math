\documentclass[letterpaper,12pt,fleqn]{article}
\usepackage{matharticle}
\usepackage{siunitx}
\pagestyle{plain}
\begin{document}

\begin{center}
  \large
  Math-13 Sections 01 and 02

  \Large
  Homework \#3 Solutions
\end{center}

\begin{enumerate}
\item Determine the smallest estimate for \(\pi+e\) that is within \(0.000005\) of the exact value.

  Using a calculator: \(\pi+e=5.859874482\ldots\)
  \begin{gather*}
    (\pi+e)-5=0.859874482\ldots>0.000005 \\
    (\pi+e)-5.8=0.059874482\ldots>0.000005 \\
    (\pi+e)-5.85=0.009874482\ldots>0.000005 \\
    (\pi+e)-5.859=0.000874482\ldots>0.000005 \\
    (\pi+e)-5.8598=0.000074482\ldots>0.000005 \\
    (\pi+e)-5.85987=0.000004482\ldots<0.000005 \\
    \\
    \pi+e\approx5.85987
  \end{gather*}

\item Using a table of values, determine the following:
  \[\lim_{x\to0}\frac{\cos x-1}{x}\]
  Be sure to approach \(0\) from both sides in your table.

  Make sure that your calculator is in radians mode!
  \[\begin{array}{|c|c|}
  \hline
  1 & -0.459698 \\
  0.1 & -0.049958 \\
  0.01 & -0.005000 \\
  0.001 & -0.000500 \\
  0.0001 & -0.000050 \\
  \hline
  0 & ??? \\
  \hline
  -0.0001 & 0.000050 \\
  -0.001 & 0.000500 \\
  -0.01 & 0.005000 \\
  -0.1 & 0.049958 \\
  -1 & 0.459698 \\
  \hline
  \end{array}\]
  So it appears that:
  \[\lim_{x\to0}\frac{\cos x-1}{x}=0\]
\end{enumerate}

\end{document}
