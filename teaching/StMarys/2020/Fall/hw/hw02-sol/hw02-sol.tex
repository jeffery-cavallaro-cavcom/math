\documentclass[letterpaper,12pt,fleqn]{article}
\usepackage{matharticle}
\usepackage{siunitx}
\pagestyle{plain}
\begin{document}

\begin{center}
  \large
  Math-13 Sections 01 and 02

  \Large
  Homework \#2 Solutions
\end{center}

\begin{enumerate}
\item You throw a ball straight up into the air using an airgun.  The ball eventually slows down due to gravity,
  stops, and then falls back to earth.  Let \(h\) be the height of the ball (in feet) at time \(t\) (in seconds)
  such that the height of the ball is given by \(48t-16t^2\).
  \begin{enumerate}
  \item Represent this situation using function notation.
    \[h(t)=48t-16t^2\]
  \item Identify the independent and dependent variables.

    Independent: \(x\)

    Dependent: \(h\)

  \item Where is the ball at \(t=2\) seconds?
    \[h(2)=48(2)-16(2)^2=96-64=\SI{32}{ft}\]
  \item How long does it take the ball to reach its maximum height of \SI{36}{ft}?
    \begin{gather*}
      48t-16t^2=36 \\
      12t-4t^2=9 \\
      4t^2-12t+9=0 \\
      (2t-3)^2=0 \\
      2t-3=0 \\
      2t=3 \\
      t=\frac{3}{2}=\SI{1.5}{sec}
    \end{gather*}
  \end{enumerate}

\item The popular hamburger chain Bun-N-Burger has finally decided to go public.  Their stock opens on the NASDAQ
  at \$25 per share and increases in price according to the function \(p(t)=25+5t\), where \(t\) is the number of
  hours that the market has been open.
  \begin{enumerate}
  \item Represent the function using a table with values from \(t=0\) to \(t=5\).
    \[\begin{array}{c|c}
    t & p(t) \\
    \hline
    0 & 25 \\
    1 & 30 \\
    2 & 35 \\
    3 & 40 \\
    4 & 45 \\
    5 & 50
    \end{array}\]
    
  \item Interpret the statement: \(p(3)=40\).

    \bigskip

    After \SI{3}{hours}, the stock price is \$\(40\).

    \bigskip

  \item Sketch a graph of the function.

    \bigskip

    \begin{center}
      \begin{tikzpicture}
        \draw (0,0) grid (5,5);
        \draw [very thick] (-1,0) -- (5,0) node [right] {\(5\)};
        \draw [very thick] (0,-1) -- (0,5) node [above] {\(50\)};
        \node [left] at (0,2.5) {\(25\)};
        \node [closed point] (A) at (0,2.5) {};
        \node [closed point] (B) at (1,3) {};
        \node [closed point] (C) at (2,3.5) {};
        \node [closed point] (D) at (3,4) {};
        \node [closed point] (E) at (4,4.5) {};
        \node [closed point] (F) at (5,5) {};
        \draw [blue] (A) to (F);
      \end{tikzpicture}
    \end{center}

    \bigskip

  \item From your graph, estimate the price of the stock at \(t=\SI{2.5}{hrs}\) and show that point on the graph.

    \bigskip

    \begin{center}
      \begin{tikzpicture}
        \draw (0,0) grid (5,5);
        \draw [very thick] (-1,0) -- (5,0) node [right] {\(5\)};
        \draw [very thick] (0,-1) -- (0,5) node [above] {\(50\)};
        \node [left] at (0,2.5) {\(25\)};
        \node [closed point] (A) at (0,2.5) {};
        \node [closed point] (B) at (1,3) {};
        \node [closed point] (C) at (2,3.5) {};
        \node [closed point] (D) at (3,4) {};
        \node [closed point] (E) at (4,4.5) {};
        \node [closed point] (F) at (5,5) {};
        \draw [blue] (A) to (F);
        \node [closed point, red] (G) at (2.5,3.75) {};
        \draw [dashed,red] (G) to (2.5,0) node [below] {\(2.5\)};
        \draw [dashed,red] (G) to (0,3.75) node [left] {\(37.5\)};
      \end{tikzpicture}
    \end{center}

    \bigskip

  \item What does the point \((4,45)\) on the graph represent?
    \[f(4)=45\]
  \end{enumerate}

\item Let \(f(x)=2x+\sqrt{x+1}\).  Solve for \(f(x)=8\).
  \begin{gather*}
    2x+\sqrt{x+1}=8 \\
    \sqrt{x+1}=8-2x \\
    x+1=(8-2x)^2 \\
    x+1=64-32x+4x^2 \\
    4x^2-33x+63=0 \\
    (4x-21)(x-3)=0 \\
    x=3,\frac{21}{4}
  \end{gather*}
  Check for extraneous solutions:
  \begin{gather*}
    2(3)+\sqrt{3+1}=6+\sqrt{4}=6+2=8 \\
    \\
    2\left(\frac{21}{4}\right)+\sqrt{\frac{21}{4}+1}=\frac{21}{2}+\sqrt{\frac{25}{4}}=
    \frac{21}{2}+\frac{5}{2}=\frac{26}{2}=13\ne8
  \end{gather*}
  So \(x=\frac{21}{4}\) is extraneous and the only solution is \(x=3\).

\item Determine the implicit domain for the function:
  \[f(x)=\sqrt{\frac{x^2+x-2}{x^2-4}}\]
  First, turn this into an inequality:
  \[\frac{x^2+x-2}{x^2-4}\ge0\]
  Now, factor:
  \[\frac{(x+2)(x-1)}{(x+2)(x-2)}\ge0\]
  Next, cancel the common factor, but remember that \(x\ne-2\).
  \[\frac{x-1}{x-2}\ge0\]
  Note that there is a zero at \(x=1\) and a pole at \(x=2\).  Using test points, we get the following:

  \bigskip

  \begin{center}
    \begin{tikzpicture}
      \draw [<->] (0,0) -- (5,0);
      \node [closed point] (A) at (3,0) {};
      \node [below] at (A) {\(1\)};
      \node [open point] (B) at (4,0) {};
      \node [below] at (B) {\(2\)};
      \draw [very thick,->,blue] (A) to (0,0);
      \draw [very thick,->,blue] (B) to (5,0);
    \end{tikzpicture}
  \end{center}

  \bigskip

  Finally, remember to leave the hole at \(x=-2\).

  \bigskip

  \begin{center}
    \begin{tikzpicture}
      \draw [<->] (0,0) -- (5,0);
      \node [open point] (C) at (1,0) {};
      \node [below] at (C) {\(-2\)};
      \node [closed point] (A) at (3,0) {};
      \node [below] at (A) {\(1\)};
      \node [open point] (B) at (4,0) {};
      \node [below] at (B) {\(2\)};
      \draw [very thick,->,blue] (A) to (C) to (0,0);
      \draw [very thick,->,blue] (B) to (5,0);
    \end{tikzpicture}
  \end{center}

  \bigskip

  So the final domain is:
  \[(-\infty,-2)\cup(-2,1]\cup(2,\infty)\]
\end{enumerate}

\end{document}
