\documentclass[letterpaper,12pt,fleqn]{article}
\usepackage{matharticle}
\usepackage{siunitx}
\pagestyle{plain}
\newcommand{\cm}{\checkmark}
\begin{document}

\begin{center}
  \large
  Math-13 Sections 01 and 02

  \Large
  Homework \#10 Solutions
\end{center}

\vspace{0.5in}

Consider the function:
\[f(x)=x^{\frac{2}{3}}-x\]
on the closed interval \([0,8]\).
\begin{enumerate}[left=0in]
\item Determine \(f'(x)\).
  \[f'(x)=\frac{2}{3}x^{-\frac{1}{3}}-1=\frac{2}{3\sqrt[3]{x}}-1=\frac{2-3\sqrt[3]{x}}{3\sqrt[3]{x}}\]
\item Determine the critical points on the interval.

  To find the zeros:
  \begin{gather*}
    2-3\sqrt[3]{x}=0 \\
    3\sqrt[3]{x}=2 \\
    \sqrt[3]{x}=\frac{2}{3} \\
    x=\left(\frac{2}{3}\right)^3=\frac{8}{27}
  \end{gather*}

  Therefore there is a zero at \(x=\frac{8}{27}\) and a poll at \(x=0\).

\item Calculate \(f(x)\) at each endpoint and critical point.
  \begin{align*}
    f(0) &= 0^{\frac{2}{3}}-0=0 \\
    f\left(\frac{8}{27}\right) &= \left(\frac{8}{27}\right)^{\frac{2}{3}}-\frac{8}{27}=\frac{4}{9}-\frac{8}{27}=
    \frac{4}{27} \\
    f(8) &= 8^{\frac{2}{3}}-8=4-8=-4
  \end{align*}
  Therefore, the endpoints are \((0,0)\) and \((8,-4)\) and there is one critical point at
  \(\left(\frac{8}{27},\frac{4}{27}\right)\).
\item Determine where \(f(x)\) is increasing and decreasing over the interval.  You must prove your result by
  evaluating the derivative at proper test points.  Summarize this information with a real number graph.

  \begin{center}
    \begin{tikzpicture}
      \draw [<->] (-1,0) -- (6,0);
      \node [closed point] (A) at (0,0) {};
      \node [below] at (A) {\(0\)};
      \node [closed point] (B) at (1.5,0) {};
      \node [below] at (B) {\(\frac{8}{27}\)};
      \node [closed point] (C) at (5,0) {};
      \node [below] at (C) {\(8\)};
      \node [above] at (0.75,0) {\(+\)};
      \node [above] at (3.25,0) {\(-\)};
      \node [below] at (0.75,0) {\(\nearrow\)};
      \node [below] at (3.25,0) {\(\searrow\)};
    \end{tikzpicture}
  \end{center}
  \begin{align*}
    f'\left(\frac{1}{27}\right) &= \frac{2}{3}\left(\frac{1}{27}\right)^{-\frac{1}{3}}-1=\frac{2}{3}(3)-1=2-1=1>0 \\
    f'(1) &= \frac{2}{3}(1)^{-\frac{1}{3}}-1=\frac{2}{3}-1=-\frac{1}{3}<0
  \end{align*}
  
\item Classify each endpoint and derivative critical point as either a relative or absolute minimum or maximum or
  point of inflection.

  Since \(f(x)\) is increasing on \((0,\frac{8}{27})\) and decreasing on \((\frac{8}{27},8)\), the critical point
  at \((\frac{8}{27},\frac{4}{27})\) is a relative maximum.  Based on the function values, we have the following:
  \[\begin{array}{|c|c|c|c|c|c|}
  \hline
  point & rmin & rmax & amin & amax & poi \\
  \hline
  (0,0) & \cm & & & & \\
  \hline
  \left(\frac{8}{27},\frac{4}{27}\right) & & \cm & & \cm & \\
  \hline
  (8,-4) & \cm & & \cm & & \\
  \hline
  \end{array}\]

\item Sketch the graph on the interval.  Be very specific near \(x=0\).

  \begin{center}
    \begin{tikzpicture}
      \begin{axis}[
          axis lines=middle,
          xmin=0,
          xmax=8,
          ymin=-4,
          ymax=1,
          ticks=none,
          xlabel={\(x\)},
          ylabel={\(y\)},
          x label style={at={(axis cs:8,0)},anchor=west},
          y label style={at={(axis cs:0,1)},anchor=south},
          clip=false
        ]
        \addplot [domain=0:8,blue,samples=1000,smooth] {x^(2/3)-x};
        \node [closed point, red] (M) at (8/27,4/27) {};
        \node [above] at (M) {\((\frac{8}{27},\frac{4}{27})\)};
      \end{axis}
    \end{tikzpicture}
  \end{center}

\end{enumerate}

\end{document}
