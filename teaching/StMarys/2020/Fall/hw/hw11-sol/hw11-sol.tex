\documentclass[letterpaper,12pt,fleqn]{article}
\usepackage{matharticle}
\usepackage{polynom}
\pagestyle{plain}
\newcommand{\no}{\textcolor{red}{\times}}
\newcommand{\yes}{\textcolor{green}{\checkmark}}
\begin{document}

\begin{center}
  \large
  Math-13 Sections 01 and 02

  \Large
  Homework \#11 Solutions
\end{center}

\vspace{0.5in}

Consider the function:
\[f(x)=x^3-4x^2-4x+16\]
\begin{enumerate}
\item Using the rational roots theorem, completely factor \(f(x)\).
  \begin{gather*}
    p=\pm1,\pm2,\pm4,\pm8,\pm16 \\
    q=\pm1 \\
    \\
    c=\pm1,\pm2,\pm4,\pm8,\pm16 \\
    \\
    f(1)=1^3-4(1)^2-4(1)+16=1-4-4+16\ne0\qquad\no \\
    f(-1)=(-1)^3-4(-1)^2-4(-1)+16=-1-4+4+16\ne0\qquad\no \\
    f(2)=2^3-4(2)^2-4(2)+16=8-16-8+16=0\qquad\yes \\
  \end{gather*}
  \polylongdiv{x^3-4x^2-4x+16}{x-2}
  \[f(x)=(x-2)(x^2-2x-8)=(x-2)(x+2)(x-4)\]
\item What are the critical points of \(f(x)\)?

  The critical numbers are \(x=\pm2,4\), so the critical points are \((\pm2,0)\) and \((4,0)\).

\item What is the \(y\)-intercept of \(f(x)\)?
  \[f(0)=0^3-4(0)^2-4(0)=16=0+0+0+16=16\]
  Therefore, the \(y\)-intercept is \((0,16)\).
  
\item What are the critical points of \(f'(x)\)?
  \[f'(x)=3x^2-8x-4\]
  This does not factor nicely, so:
  \begin{align*}
    x &= \frac{8\pm\sqrt{(-8)^2-4(3)(-4)}}{2(3)} \\
    &= \frac{8\pm\sqrt{64+48}}{6} \\
    &= \frac{8\pm\sqrt{112}}{6} \\
    &= \frac{8\pm4\sqrt{7}}{6} \\
    &= \frac{4\pm2\sqrt{7}}{3} \\
    &\approx -0.43,3.10
  \end{align*}
  \begin{gather*}
    f\left(\frac{4-2\sqrt{7}}{3}\right)\approx16.90 \\
    f\left(\frac{4+2\sqrt{7}}{3}\right)\approx-5.05
  \end{gather*}
  Therefore, the first derivative critical points are \((-0.43,16.90)\) and \((3.10,-5.05)\).

\item Using the first derivative test, determine the relative extrema of \(f(x)\).

  \begin{center}
    \begin{tikzpicture}
      \draw [<->] (-2,0) -- (6,0);
      \node [closed point,red] (X) at (-0.43,0) {};
      \node [below] at (X) {\(-0.43\)};
      \node [closed point,red] (Y) at (3.10,0) {};
      \node [below] at (Y) {\(3.10\)};
      \node [above] at (-1.5,0) {\(\textcolor{blue}{+}\)};
      \node [below] at (-1.5,0) {\(\textcolor{blue}{\nearrow}\)};
      \node [above] at (1.5,0) {\(\textcolor{blue}{-}\)};
      \node [below] at (1.5,0) {\(\textcolor{blue}{\searrow}\)};
      \node [above] at (4.5,0) {\(\textcolor{blue}{+}\)};
      \node [below] at (4.5,0) {\(\textcolor{blue}{\nearrow}\)};
    \end{tikzpicture}

    Using test points:
    \begin{gather*}
      f'(-1)>0 \\
      f'(0)<0 \\
      f'(4)>0
    \end{gather*}
  \end{center}

  Therefore \((-0.43,16.90)\) is a relative maximum and \((3.10,-5.05)\) is a relative minimum.
  
\item Using the second derivative test, verify the relative extrema of \(f(x)\).
  \[f''(x)=6x-8\]
  So \(f''(-0.43)<0\), so the function is concave down, verifying that \((-0.43,16.90)\) is a relative maximum.
  Likewise, \(f''(3.10)>0\), so the function is concave up, verifying that \((3.10,-5.05)\) is a relative minimum.
  
\item What are the critical points of \(f''(x)\)?
  \begin{gather*}
    x=\frac{8}{6}=\frac{4}{3}\approx1.33 \\
    \\
    f\left(\frac{4}{3}\right)\approx5.93
  \end{gather*}
  Therefore, the second derivative has one critical point at \((1.33,5.93)\).

\item Using the second derivative, prove that the critical point of \(f''(x)\) is a point of inflection.
  \begin{center}
    \begin{tikzpicture}
      \draw [<->] (0,0) -- (6,0);
      \node [closed point,red] (X) at (3,0) {};
      \node [below] at (X) {\(\frac{4}{3}\)};
      \node [above] at (1.5,0) {\(\textcolor{blue}{-}\)};
      \node [below] at (1.5,0) {\textcolor{blue}{CCD}};
      \node [above] at (4.5,0) {\(\textcolor{blue}{+}\)};
      \node [below] at (4.5,0) {\textcolor{blue}{CCU}};
    \end{tikzpicture}
  \end{center}

  Using test points:
  \begin{gather*}
    f''(0)<0 \\
    f''(2)>0
  \end{gather*}
  Therefore, the function is concave down to the left of the critical point and concave up to the right of the
  critical point.  The change in concavity proves that the critical point is a point of inflection.

\item What is the end behavior of \(f(x)\)?

  Using the leading term test, the end behavior is like \(x^3\):

  As \(x\to-\infty\), \(f(x)\to-\infty\).

  As \(x\to+\infty\), \(f(x)\to+\infty\).

\item Sketch \(f(x)\).  For full credit, all intercepts, extrema, and points of inflection must be labeled with
  their coordinate values.

  \begin{center}
    \begin{tikzpicture}
      \begin{axis}[
          axis lines=middle,
          xmin=-5,
          xmax=6,
          ymin=-10,
          ymax=20,
          ticks=none,
          xlabel={\(x\)},
          ylabel={\(y\)},
          x label style={at={(axis cs:6,0)},anchor=west},
          y label style={at={(axis cs:0,20)},anchor=south},
          clip=false
        ]
        \addplot [domain=-2.35:5,blue,smooth] {x^3-4*x^2-4*x+16};
        \node [closed point,red] (X1) at (-2,0) {};
        \node [below left] at (X1) {\(-2\)};
        \node [closed point,red] (X2) at (2,0) {};
        \node [below left] at (X2) {\(2\)};
        \node [closed point,red] (X3) at (4,0) {};
        \node [below right] at (X3) {\(4\)};
        \node [closed point,red] (Y1) at (0,16) {};
        \node [above right] at (Y1) {\(16\)};
        \node [closed point,red] (E1) at (-0.43,16.90) {};
        \node [above left] at (E1) {\((-0.43,16.90)\)};
        \node [closed point,red] (E2) at (3.10,-5.05) {};
        \node [below] at (E2) {\((3.10,-5.05)\)};
        \node [closed point,red] (E3) at (1.33,5.93) {};
        \node [right] at (E3) {\((1.33,5.93)\)};
      \end{axis}
    \end{tikzpicture}
  \end{center}

\end{enumerate}

\end{document}
