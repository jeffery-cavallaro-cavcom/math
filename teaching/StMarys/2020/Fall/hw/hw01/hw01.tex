\documentclass[letterpaper,12pt,fleqn]{article}
\usepackage{matharticle}
\pagestyle{plain}
\begin{document}

\begin{center}
  \large
  Math-13 Sections 01 and 02

  \Large
  Homework \#1

  \large
  \textbf{Due: Midnight 9/1}
\end{center}

\begin{enumerate}
  \setlength\itemsep{5ex}
\item Indicate which subsets of the real numbers that the following values are members of:

  \bigskip

  \begin{center}
    \renewcommand{\arraystretch}{1.2}
    \begin{tabular}{|c||c|c|c|c|c|}
      \hline
      VALUE & \(\N\) & \(\Z\) & \(\Q\) & \(\R-\Q\) & \(\R\) \\
      \hline
      \hline
      \(0\) & & & & & \\
      \hline
      \(\sqrt{2}\) & & & & & \\
      \hline
      \(\sqrt{9}\) & & & & & \\
      \hline
      \(123.4\) & & & & & \\
      \hline
      \(-123\) & & & & & \\
      \hline
      \(\pi\) & & & & & \\
      \hline
      \(12.34\overline{56}\) & & & & & \\
      \hline
    \end{tabular}
  \end{center}

  \bigskip

\item Convert the rational value \(12.34\overline{56}\) to integer ratio form using the algorithm that we learned
  in class.

\item Convert \(24.57\overline{9}\) to integer ratio form \emph{without} using the algorithm and justify your
  answer.  (Hint: look for an alternate syntax for the value.)

\item Is \(\displaystyle\frac{123.4}{56.99}\) a rational number?  If so, then explain why \emph{without} using the
  algorithm.
\end{enumerate}

\end{document}
