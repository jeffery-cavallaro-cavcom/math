\documentclass[letterpaper,12pt,fleqn]{article}
\usepackage{matharticle}
\usepackage{siunitx}
\pagestyle{plain}

\begin{document}

\begin{center}
  \large
  Math-13 Sections 01, 02

  \Large
  Exam \#2

  Due: 11/8/2020 at 11:59pm
\end{center}

\vspace{0.5in}

This exam is open book and notes.  You may use a calculator.  No collaboration or other web access is allowed. All
answers must be in exact form unless stated otherwise (i.e., no decimal answers allowed).  You \emph{must} show all
work and that work must be logical and complete; there is \emph{no} credit for guessed answers or answers without
supporting work.

You must work the exam problems, in order, on separate sheets of paper; camscan your results into a single PDF
file; and then submit your PDF file back to Moodle (just like the written homeworks).  This \emph{must} be done by
the deadline; late exams, multiple page or non-PDF submissions, and exams sent by email will not be accepted.

Good luck!

\vspace{0.5in}

\begin{enumerate}[left=0pt,itemsep=0.5in]

\item What are the three characterizations of the derivative that were discussed in class?

\item Let \(f(x)\) be continuous on \([1,5]\) and differentiable on \((1,5)\).  State the theorem that supports
  the following conclusions:
  \begin{enumerate}
  \item If \(f(1)=1\) and \(f(5)=9\) then there exists some \(c\in(1,5)\) such that \(f'(c)=2\).
  \item If \(f(1)=-1\) and \(f(5)=1\) then there exists some \(c\in[1,5]\) such that \(f(c)=0\).
  \item \(f(x)\) has an absolute minimum and an absolute maximum on \([1,5]\).
  \item If \(f(1)=1\) and \(f(5)=1\) then there exists some \(c\in(1,5)\) such that \(f'(c)=0\).
  \end{enumerate}

\item Let \(f(x)=\sqrt{x^2+1}\).
  \begin{enumerate}
  \item Using the definition of the derivative (not the formulas!), determine \(f'(x)\).
  \item Using the chain rule, determine \(f'(x)\).
  \end{enumerate}

\item Let \(f(t)=\frac{1}{3}t^3-\frac{1}{4}t^2+3t-100\).  Determine \(f'(t)\).

\item Let \(f(x)=2x^2-3x+5\).
  \begin{enumerate}
  \item Determine the equation of the tangent line at \(x=2\).
  \item Determine the equation of the normal line at \(x=2\).
  \end{enumerate}

\item Let \(f(x)=(2x+3)^2\sqrt{x^2+1}\).  Determine \(f'(x)\).  Your answer must be fully simplified for full credit.

\item Let \(\displaystyle f(x)=\frac{x^2+3x-2}{2x-1}\).  Using the quotient rule, determine \(f'(x)\).  Your answer
  must be fully simplified for full credit.

\item Let \(y^2+y-5x^2=100\).  Determine \(y'\).

\item A home-made model rocket is launched from the ground at an initial speed of \SI{256}{ft/s}.  The height \(h\)
  of the rocket (in feet) at time \(t\) (in seconds) is given by \(h(t)=256t-16t^2\).  What is the maximum height of
  the rocket and how long does it take to achieve that height?

\item Let \(f(x)=x^3+x^2-4x-4\).
  \begin{enumerate}
  \item Using the rational roots theorem, completely factor \(f(x)\).  If you don't think that you can do this
    then you can ask Wolfram Alpha to help you; however, you will receive no credit for this part.
  \item What are the critical points of \(f(x)\)?
  \item What is the \(y\)-intercept of \(f(x)\)?
  \item What are the critical points of \(f'(x)\)?
  \item Using the first derivative test, determine the relative extrema of \(f(x)\).
  \item Using the second derivative test, verify the relative extrema of \(f(x)\).
  \item What are the critical points of \(f''(x)\)?
  \item Using the second derivative, prove that the critical point of \(f''(x)\) is a point of inflection.
  \item What is the end behavior of \(f(x)\)?
  \item Sketch \(f(x)\).  For full credit, all intercepts, extrema, and points of inflection must be labeled with
    their coordinate values.
  \end{enumerate}

\end{enumerate}

\end{document}
