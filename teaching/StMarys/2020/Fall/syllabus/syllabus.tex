\documentclass[letterpaper,12pt,fleqn]{article}
\usepackage[margin=1in]{geometry}
\usepackage{libertine}
\usepackage{parskip}
\usepackage{url}
\usepackage{fancyhdr}
\usepackage{lastpage}
\usepackage{enumitem}
\lhead{}
\chead{}
\rhead{}
\lfoot{Math-13; Sections 01,02; Fall 2020 --- Cavallaro}
\cfoot{}
\rfoot{Page \thepage\ of \pageref{LastPage}}
\setlength{\footskip}{0.5in}
\renewcommand{\headrulewidth}{0pt}
\renewcommand{\footrulewidth}{1pt}
\pagestyle{fancy}
\begin{document}

\begin{center}
  \emph{St Mary's College of California}

  Department of Mathematics and Computer Science

  {\large Fall 2020}
  
  \begin{Large}
    \bfseries
    Math-31: Calculus with Elementary Functions

    Sections 01 and 02
  \end{Large}
\end{center}

\vspace{0.25in}

\subsection*{Course and Contact Information}

\begin{tabular}{|p{2in}|p{4.5in}|}
  \hline
  Instructor: & Jeffery Cavallaro \\
  \hline
  Office Location: & Galileo 103-D \\
  \hline
  Email: & {\small\url{jac51@stmarys-ca.edu}} \\
  \hline
  Office Hours: & Th 2-5pm \\
  \hline
  Class Days/Time: & \begin{minipage}{4.5in}
    \vspace{0.1cm}
    Section 01: MWF 9:15am--10:20am

    Section 02: MWF 10:30am--11:35am
    \vspace{0.1cm}
  \end{minipage} \\
  \hline
  Classroom: & Online (Zoom) \\
  \hline
  Prerequisites: & \begin{minipage}{4.5in}
    \vspace{0.1cm}
    Math Level 003 or 004; or \\
    Math-002; or \\
    Math-12
    \vspace{0.1cm}
  \end{minipage} \\
  \hline
  Corequisites: & Math-13T (all sections meet on Tuesday) \\
  \hline
\end{tabular}

\subsection*{Course Description}

Math-13 is the first semester in the Math-13/Math-14 calculus with precalculus series.  The course covers function
basics, limits and the limit laws, continuity, derivatives of polynomial and rational functions, the derivative
formulas, rate of change and optimization problems, and curve sketching.  The course concludes with an introduction
to integral calculus and the fundamental theorem of calculus (FTC), time permitting.  A review of the necessary
material from precalculus precedes each new calculus subject.

This course satisfies the \emph{Mathematical Understanding} requirement of the \emph{Core Curriculum}.

\subsection*{Course Learning Outcomes}

Upon successful completion of this course, students will be able to:
\begin{itemize}
\item Describe how functions on the real numbers are defined mathematically and how they are used to model
  phenomena in the real world.
\item Describe how the concept of \emph{arbitrarily close} leads to the definition of the limit.
\item Identify cases where limits exists and where they fail, both graphically and analytically.
\item Describe the formal concept of continuity and where continuity fails, both graphically and analytically.
\item Use the limit laws to determine the limits of polynomials and rational functions.
\item State the interpretations of the derivative in one variable as the limit of the difference quotient, the
  slope of the tangent line to a curve at a point, and the instanteous rate of change of a function at a point.
\item Use the derivative formulas to determine the derivatives of polynomial and rational functions.
\item Solve rate of change and optimization problems.
\item Use precalculus and optimization techniques to sketch polynomial and rational functions.
\item Explain the connection between the area under a curve and the integral of the corresponding function.
\item Describe the concept of the Riemann Sum and how it leads to the definition of the integral of a function.
\item Distinguish between definite and indefinite integrals.
\item Understand antiderivatives and applications of the fundamental theorem of calculus (FTC) to solve integrals.
\item Identify and solve integrals where substitution is required.
\end{itemize}

\subsection*{Course Requirements}

\subsubsection*{Texts}

We will use the following textbooks:

\begin{itemize}
\item \emph{Precalculus: An Investigation of Functions}, Lippman and Rasmussen, Edition 2.1.
\item \emph{Apex Calculus}, Hartman, et al, Version 4.0.
\end{itemize}

Both are freely available online and are accessible via Moodle under the \emph{Resources} topic.

\subsubsection*{Web}

This is a fully online class that will meet via Zoom.  The meeting link and invite for both class and office hours
is available on Moodle under the \emph{Resources} topic.  Please connect at least five minutes before the class
start time with video on and audio muted so that we can start immediately on time.  Please connect to the class in
a stable environment such as a desk, since you will not be able to adequately participate if you are in your car or
on the move.

All class communications, including announcements, reading assignments, written homework assignments, quizzes,
exams, and grades are available via Moodle.  Sections 01 and 02 will be tracked under a combined course called:

\begin{quote}
  MATH 013-01/02: Calculus with Elem Functions '[20-FA]' (combined)
\end{quote}

Of course, you can communicate with me directly via email.

Online homework assignments are performed in WebWork.  The link to the WebWork course is available on Moodle
under the \emph{Online Homework} topic.

\subsubsection*{Technology}

Since this is to be a fully online course, there are some extra technology requirements:
\begin{itemize}
\item A network-accessible computing device with a video camera and microphone.  A desktop or laptop is strongly
  suggested; however, a smart phone may be workable.
\item You should have the Zoom application installed on your device; however, you can also access Zoom from MySMC.
\item You must have the ability to scan your written homework and exams, each to a \emph{single} PDF file, so that you
  can submit them via Moodle.  The CamScanner application seems to work best for this.  Other file formats or multiple
  files per assignment will \emph{not} be accepted.
\item Your will need the ability to collaborate with the class and breakout rooms, preferrably with some sort of
  tablet or pad device with a stylus/pen.  Using your finger on a smartphone may be workable.  The minimum
  requirement is a personal whiteboard that you can hold in front of your camera.
\item A scientific (TI-30X) or graphing calculator (TI-84 CE).  A graphing calculator is preferred since we will
  use it during class to investigate certain graphs visually prior to describing them analytically.
\end{itemize}

\subsubsection*{Time}

You will need to spend a \emph{minimum} of 10 hours per week outside of class doing homework and studying. This
class is intensive and requires disciplined study habits.

\subsection*{Assignments}

\subsubsection*{Reading and Quizzes}

Reading from the textbooks will be assigned on a regular basis on Moodle under the \emph{Reading and Quizzes} topic.
Each reading assignment includes a short (and easy) quiz in Moodle taken directly from the material.  Each reading
assignment and its corresponding quiz must be completed by the stated due date, since that material will be
discussed during the next class meeting.

\subsubsection*{Online Homework}

Online homework will be assigned each Monday in WebWork and will be due the following Tuesday at midnight.  There
are no extensions and no assignments will be dropped.

\subsubsection*{Written Homework}

Written homework assignments will be assigned each Monday in Moodle under the \emph{Written Homework} topic and
will be due the following Tuesday at midnight.  The assignment will be a PDF file attached to the corresponding
activity in Moodle.  Written homeworks must be neat, organized, scanned to a single PDF file, and then uploaded to
the corresponding Moodle activity.  There are no extensions; however, your three lowest scores will be dropped.

\subsubsection*{Exams}

There are two regular exams and a comprehensive final exam.  The tentative regular exam schedule is as follows:

\begin{enumerate}
\item \textbf{September 26}
\item \textbf{October 31}
\end{enumerate}

Exams are assigned and submitted like written homework assignments, but are available under the \emph{Exams} topic.
Each exam is available Saturday morning at midnight and is due Sunday evening by midnight.  Exams are open
book/notes and you may use your calculator; however, collaboration and web searches (especially Chegg) are strictly
forbidden.

Exam problems \emph{must} be solved according to the techniques that are learned in class and \emph{must} be
supported by complete and logical work.  The use of alternate methods, answers that do not follow from work, and
guessed answers are assumed to be the results of cheating and will receive no credit.  Anyone caught cheating or
collaborating on an exam will receive an automatic score of \(0\) for that exam.

\subsubsection*{Final}

The final exam is comprehensive and is scheduled (for both sections) on:

\begin{quote}
  \textbf{Thursday, December 3, 7-9pm}
\end{quote}

The final exam will be available under the \emph{Exams} topic at 7pm and must be completed by 9pm.  You then have
an additional 30 minutes to scan and submit it.  Final exams will not be accepted after 9:30pm.

\newpage

\subsection*{Determination of Grades}

Your semester grade is determined as follows:

\bigskip

\begin{center}
  \begin{minipage}{2.5in}
    \centering
    \begin{tabular}{|c|c|}
      \hline
      Online Homework & 20\% \\
      \hline
      Written Homework & 20\% \\
      \hline
      Quizzes & 10\% \\
      \hline
      Regular Exams & 30\% \\
      \hline
      Final Exam & 20\% \\
      \hline
    \end{tabular}
  \end{minipage}
  \begin{minipage}{2.5in}
    \centering
    \begin{tabular}{|l|c||l|c|}
      \hline
      A+ & 100--97 & D+ & 67--69 \\
      \hline
      A & 96--93 & D & 63--66 \\
      \hline
      A- & 92--90 & D- & 60--62 \\
      \hline
      B+ & 89--87 & F & \(<\)60 \\
      \hline
      B & 86--83 & & \\
      \hline
      B- & 82--80 & & \\
      \hline
      C+ & 79--77 & & \\
      \hline
      C & 76--73 & & \\
      \hline
      C- & 72--70 & & \\
      \hline
    \end{tabular}
  \end{minipage}
\end{center}

\subsection*{Classroom Protocol}
  
\subsubsection*{Attendance}

I will only take attendance on the first two Mondays; however, it is important that you come (on time) to every
class.  I will record each Section 02 class meeting to the cloud, so if you miss a class then it is your
responsibility to watch the recording.  I will post the recording link in a Moodle announcement.  Note that
recordings are purged after 180 days, but you can download a personal copy.

\subsubsection*{Moodle and WebWork}

Check Moodle and WebWork at least three times a day (morning, afternoon, and evening) to make sure that you are
current on all assignment due dates.

\subsubsection*{Holidays}

Class will not meet on Monday, September 7 (Labor Day).

\subsection*{College Policies}

\subsubsection*{STEM Center}

Online tutoring is available this semester from the STEM center.  Drop-in tutoring hours are:

\begin{itemize}
\item Monday--Thursday, 1--9pm
\item Sunday, 6--9pm
\end{itemize}

For more information, see:

\begin{quote}
  {\small\url{https://www.stmarys-ca.edu/school-of-science/stem-center}}
\end{quote}

\subsubsection*{Student Disability Services (SDS)}

The College strives to make all learning experiences as accessible as possible.  Students who anticipate or
experience academic barriers based on a disability are encouraged to contact Student Disability Services (SDS) to
set up a confidential appointment to discuss available services and options.  The Student Disability Services
office can be reached by emailing {\small\url{sds@stmarys-ca.edu}}; calling 925-631-4358; or visiting the office
located in Filippi Academic Hall FAH190.

\subsubsection*{Honor Code}

Saint Mary’s College expects every member of its community to abide by the Academic Honor Code.  According to the Code,

\begin{quote}
  Academic dishonesty is a serious violation of College policy because, among other things, it undermines the bonds of
  trust and honesty between members of the community.
\end{quote}

Violations of the Code include but are not limited to acts of plagiarism.  For more information, please consult the
Student Handbook at:

\begin{quote}
  {\small\url{http://www.stmarys-ca.edu/your-safety-resources/student-handbook}}
\end{quote}

If a reasonable suspicion arises that you have violated academic honor code, you will be referred to the Academic
Honor Council for further review and or necessary sanctions.

\end{document}
