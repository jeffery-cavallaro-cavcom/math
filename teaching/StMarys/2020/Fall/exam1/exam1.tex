\documentclass[letterpaper,12pt,fleqn]{article}
\usepackage{matharticle}
\usepackage{tikz}
\pagestyle{plain}

\begin{document}

\begin{center}
  \large
  Math-13 Sections 01, 02

  \Large
  Exam \#1

  Due: 10/4/2020 at 11:59pm
\end{center}

\vspace{0.5in}

This exam is open book and notes.  You may use a calculator.  No collaboration or other web access is allowed. All
answers must be in exact form unless stated otherwise (i.e., no decimal answers allowed).  You \emph{must} show all
work and that work must be logical and complete; there is \emph{no} credit for guessed answers or answers without
supporting work.

You must work the exam problems, in order, on separate sheets of paper; camscan your results into a single PDF
file; and then submit your PDF file back to Moodle (just like the written homeworks).  This \emph{must} be done by
the deadline; late exams, multiple page or non-PDF submissions, and exams sent by email will not be accepted.

Good luck!

\vspace{0.5in}

\begin{enumerate}[left=0pt,itemsep=0.5in]

\item Explain why \(\pi\) is an irrational number.  Note: do not just say, ``because it is not rational.''
  Be more specific.

\item In class, we discussed the fact that every function is a relation, but not every relation is a function.
  Explain what makes a relation a function.

\item The follow function shows the population of bacteria \(p\) in a petri dish at time \(t\) (in days):

  \bigskip

  \begin{center}
    \begin{tikzpicture}[scale=0.8]
      \begin{axis}[
          axis lines=middle,
          xmin=-1,
          xmax=4,
          ymin=-100,
          ymax=2000,
          ticks=none,
          xlabel={\(t\)},
          ylabel={\(p(t)\)},
          x label style={at={(axis cs:4,0)},anchor=west},
          y label style={at={(axis cs:0,2000)},anchor=south},
          clip=false
        ]
        \addplot [domain=0:4,blue] {100*2^x+400};
        \node [left] at (0,500) {\(100\)};
        \node [closed point,red] (P) at (3,1200) {};
        \node [right] at (P) {\((3,800)\)};
        \node [below left] at (0,0) {\(0\)};
        \node [above right] at (4,2000) {\(p(t)\)};
      \end{axis}
    \end{tikzpicture}
  \end{center}

  \bigskip

  \begin{enumerate}
  \item What is the initial \((t=0)\) number of bacteria in the dish?
  \item Describe using words the meaning of the point \((3,800)\) on the graph.
  \item What is the alternate functional syntax for the point \((3,800)\)?
  \item What is \(\displaystyle\lim_{t\to3}p(t)\)?
  \item Is \(p(t)\) continuous at \(t=3\)?  If not then indicate which requirement for continuity fails.
  \end{enumerate}

  \bigskip

\item Determine the implicit domain for the function:
  \[f(x)=\sqrt{6x^2+5x-4}\]
  Your work must include a real number line graph containing the critical points and an indication of the sign in
  each interval between the critical points.  You may use test points or factor multiplicity to determine the sign
  changes across the critical points.  Your final answer must be in interval notation.

\item Determine the smallest approximation for \(e^{\pi}\) that is with \(0.000005\) of the exact value.  Make sure
  that your final answer is the \emph{approximation} and not the error.

\item Determine the following limit:
  \[\lim_{x\to1}\frac{2x^2-x-1}{x^2-1}\]

\item Let \(f(x)=\sqrt{x+1}\).  Determine the following limit:
  \[\lim_{h\to0}\frac{f(x+h)-f(x)}{h}\]

\item Consider the following piecewise function:
  \[f(x)=\begin{cases}
  x^2, & x<2 \\
  3, & x=2 \\
  2x, & x>2
  \end{cases}\]
  Is \(f(x)\) continuous at \(x=2\)?  If not, then state the specific reason why not.

\item Consider the following function:

  \bigskip

  \begin{center}
    \begin{tikzpicture}
      \begin{axis}[
          axis lines=middle,
          xmin=-5,
          xmax=5,
          ymin=-5,
          ymax=5,
          ticks=none,
          minor tick num=1,
          xlabel={\(5\)},
          ylabel={\(5\)},
          axis line style={very thick},
          x label style={at={(axis cs:5,0)},anchor=west},
          y label style={at={(axis cs:0,5)},anchor=south},
          grid=both,
          clip=false
        ]
        \addplot [<-,domain=-5:-1,blue] {-2};
        \addplot [->,domain=-1:0.9,blue] {(x+1)^2+1};
        \addplot [<->,domain=1.2:5,blue] {-1/(x-1)};
        \node [closed point] (A) at (-1,-2) {};
        \node [open point] (B) at (-1,1) {};
      \end{axis}
    \end{tikzpicture}
  \end{center}

  \bigskip

  Determine the following.  If a value does not exists then write DNE.

  \begin{enumerate}
  \item\(\displaystyle\lim_{x\to-\infty}f(x)\)
  \item\(\displaystyle\lim_{x\to-1^-}f(x)\)
  \item\(\displaystyle\lim_{x\to-1^+}f(x)\)
  \item\(\displaystyle\lim_{x\to-1}f(x)\)
  \item\(\displaystyle f(-1)\)
  \item\(\displaystyle\lim_{x\to1^-}f(x)\)
  \item\(\displaystyle\lim_{x\to1^+}f(x)\)
  \item\(\displaystyle\lim_{x\to1}f(x)\)
  \item\(\displaystyle f(1)\)
  \item\(\displaystyle\lim_{x\to\infty}f(x)\)
  \end{enumerate}

\item Sketch the graph for the following function:
  \[f(x)=\frac{x^2-x-6}{(x+1)(x+2)(x+3)}\]
  Your work must clearly show how you determined the critical points.  Your sketch must clearly show and label any
  and all zeros, vertical asymptotes, horizontal asymptotes, and holes.

\end{enumerate}

\end{document}
