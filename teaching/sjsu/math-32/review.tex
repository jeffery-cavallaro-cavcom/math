\documentclass[letterpaper,12pt,fleqn]{article}
\usepackage{libertine}
\usepackage{parskip}
\usepackage{enumitem}
\usepackage{amsfonts}
\usepackage{amssymb}
\usepackage{amsthm}
\usepackage{mathtools}
\usepackage{mathrsfs}
\usepackage{esvect}
\usepackage{diagbox}
\newcommand{\uv}[1]{\hat{#1}}
\newcommand{\iv}{\uv{\text{\i}}}
\newcommand{\jv}{\uv{\text{\j}}}
\newcommand{\kv}{\uv{\text{k}}}
\renewcommand{\O}{\theta}
\newcommand{\p}{\rho}
\newcommand{\setb}[2]{\left\{{#1}\,\middle|\,{#2}\right\}}
\DeclarePairedDelimiter{\abs}{\lvert}{\rvert}
\DeclareMathOperator{\comp}{comp}
\DeclareMathOperator{\proj}{proj}
\pagestyle{plain}
\begin{document}
\section*{Math-32 Spring 2020 Final Review Problems}

\subsection*{Vectors and the Geometry of Space}

\begin{enumerate}
\item Let \(\vec{a}=2\iv-3\jv+\kv\) and \(\vec{b}=2\jv+5\kv\).  Calculate the following:
  \begin{enumerate}
  \item \(\vec{a}+\vec{b}\)
  \item \(\vec{b}-\vec{a}\)
  \item \(\vec{a}\cdot\vec{b}\)
  \item \(\vec{b}\times\vec{a}\)
  \item \(\abs{\vec{a}}\)
  \item \(\uv{a}\)
  \item \(\comp_{\vec{a}}(\vec{b})\)
  \item \(\proj_{\vec{b}}(\vec{a})\)
  \item Are \(\vec{a}\) and \(\vec{b}\) orthogonal?  Give your reason.
  \item Are \(\vec{a}\) and \(\vec{b}\) parallel?  Give your reason.
  \item Assuming that \(\vec{a}\) and \(\vec{b}\) are in the standard position, what is the measure of the
    angle between them (in radians).
  \end{enumerate}

\item Consider the four points:
  \begin{gather*}
    A(-1,-1,-1) \\
    B(2,2,0) \\
    C(-2,2,1) \\
    D(1,1,3)
  \end{gather*}
  What is the volume of the parallelepiped defined by \(\vv{AB}\), \(\vv{AC}\), and \(\vv{AD}\)?

\item Consider the tetrahedron with vertices \((0,0,2)\), \((0,1,0)\), \((0,0,0)\), and \(1,2,0\).
  \begin{enumerate}
  \item Sketch the tetrahedron.
  \item Determine the volume of the tetrahedron.
  \end{enumerate}

\item Consider the line through the points \(A(-1,3,2)\) and \(B(5,0,1)\):
  \begin{enumerate}
  \item Determine the vector equation for the line.
  \item Determine the parametric equations for the line.
  \item Determine the symmetric equations for the line.
  \item Find a third point on the line not the same as \(A\) or \(B\).
  \item Determine the equation of the line orthogonal to this line through the third point that you found.
  \end{enumerate}

\item Consider the points \(A(1,1,1)\), \(B(5,0,6)\), and \(C(-1,-2,-3)\):
  \begin{enumerate}
  \item Determine the scalar equation of the plane containing these three points.
  \item Determine the \(x\), \(y\), and \(z\) intercepts of the plane.
  \item Sketch the plane.
  \item Determine the distance from the point \((10,10,10)\) to the plane.
  \end{enumerate}

\item Determine the scalar equation for the plane through the point \((1,2,-2)\) that contains the line with the
  parametric equations:
  \begin{gather*}
    x=2t \\
    y=3-t \\
    z=1+3t
  \end{gather*}

\item Consider the following linear equations for two planes:
  \begin{gather*}
    3x+5y-z=3 \\
    2x-3y+5z=4
  \end{gather*}
  \begin{enumerate}
  \item Show that these planes are neither parallel nor orthogonal.
  \item Determine the measure of the angle between the two planes (in radians, to 2 decimal places).
  \item Determine the parametric equations for the line of intersection between the two planes.
  \item Determine the symmetric equations for the line of intersection between the two planes.
  \end{enumerate}

\item Consider the following linear equations for two planes:
  \begin{gather*}
    3x+y-4z=2 \\
    3x+y-4z=24
  \end{gather*}
  \begin{enumerate}
  \item Show why these planes are parallel.
  \item Determine the distance between the two planes.
  \end{enumerate}
\end{enumerate}

\subsection*{Vector Functions}

\begin{enumerate}
\item Consider the following parameterized equation for a curve:
  \[\left.\begin{array}{l}
  x = \sin\O \\
  y = \cos\O
  \end{array}\right\}\quad0\le\O\le\pi\]
  \begin{enumerate}
  \item Determine the Cartesion equation for the curve by eliminating the parameter.
  \item Sketch the curve and indicate with an arrow the direction in which the curve is traced as the parameter
    increases.
  \end{enumerate}

\item Determine the speed of a particle with the position function:
  \[\vec{r}(t)=\frac{\sqrt{5}}{2}t\iv+e^{5t}\jv-e^{-5t}\kv\]

\item Consider the curve \(\vec{r}(t)=\langle t^2,0,t\rangle\).
  \begin{enumerate}
  \item Determine the length of the curve for \(1\le t\le4\) (to four decimal places).
  \item Determine the curvature.
  \end{enumerate}

\item Determine the curvature of the curve \(\vec{r}(t)=\langle\sqrt{15}t,e^t,e^{-t}\rangle\) at the point
  \((0,1,1)\).  Simplify your answer.

\item Consider the following parameterized equation for a curve:
  \begin{gather*}
    x = 2\sin(3t) \\
    y = t \\
    z=2\cos(3t)
  \end{gather*}
  at the point \((0,\pi,-2)\).
  \begin{enumerate}
  \item Determine the unit vector \(\vec(T)(t)\).
  \item Determine the unit normal vector \(\vec{N}(t)\).
  \item Determine the unit binormal vector \(\vec{B}(t)\).
  \item Determine the equation of the normal plane to the curve at the point.
  \item Determine the equation of the osculating plane to the curve at the point.
  \end{enumerate}

\item Two particles travel along the space curves:
  \begin{gather*}
    \vec{r}_1(t)=\langle t,t^2,t^3\rangle \\
    \vec{r}_2(s)=\langle 1+2s,1+6s,1+14s\rangle \\
  \end{gather*}
  \begin{enumerate}
  \item Determine all point of intersection.
  \item Do the two particles collide?  Explain why or why not.
  \item Determine the angle between the curves at any one point of intersection (in radians, rounded to two decimal
    places).
  \end{enumerate}

\item Reparameterize the curve:
  \[\vec{r}(t)=e^{7t}\cos(7t)\iv+6\jv+e^{7t}\sin(7t)\kv\]
  with respect to arc length \(s\) measured from \(t=0\) in the direction of increasing \(t\).
\end{enumerate}

\subsection*{Partial Derivatives}

\begin{enumerate}
\item Consider the function \(f(x,y)=\ln(x+y+1)\).
  \begin{enumerate}
  \item Determine an expression for the domain.
  \item Sketch the domain.
  \end{enumerate}

\item Let \(\displaystyle N(u,v,w)=\frac{p+q}{p+r}\) where:
  \begin{gather*}
    p=u+vw \\
    q=v+uw \\
    r=w+uv
  \end{gather*}
  \begin{enumerate}
  \item Using proper chain rule notation, determine all three first order partial derivatives.
  \item Evaluate the partial derivatives at \(u=2\), \(v=3\), and \(w=4\).
  \end{enumerate}

\item How many critical points does the following function have?
  \[g(x,y)=-32xy+(x+y)^4\]

\item Find all extrema and saddle points for the function:
  \[f(x,y)=x^2+x^2y+2y^2+3\]

\item Use the second partial derivative test to determine all extrema for the function:
  \[f(x,y)=\frac{1}{3}x^3+y^2+2xy-6x-3y+4\]

\item Find the absolute minimum and maximum value of the function:
  \[f(x,y)=x^2-2xy+4y^2-4x-2y+24\]
  over the region \(R=[0,4]\times[0,2]\).

\item Determine the absolute maximum value of \(f(x,y)=2x^3+y^4\) on the disk
  \(D=\setb{(x,y)}{0\le x\le3,0\le y\le2}\).

\item Determine the gradient of the function \(f(x,y,z)=z^2e^{x\sqrt{y}}\).

\item Consider the function \(f(x,y)=xe^{xy}\).
  \begin{enumerate}
  \item Determine the linearization \(L(x,y)\) of the function  at the point \((1,0)\).
  \item Use \(L(x,y)\) to approximate \(f(1.1,-0.1)\).
  \end{enumerate}

\item Find three positive numbers whose sum is \(12\) and whose squares is as small as possible.

\item At what point on the paraboloid \(y=x^2+z^2\) is the tangent plane parallel to the plane
  \(7x+4y+5z=4\)?

\item The electric potential \(V\) at any point \((x,y)\) is given by \(V=\ln\sqrt{x^2+y^2}\).
  \begin{enumerate}
  \item Determine the rate of change in \(V\) at the point \((3,4)\) in the direction of the point \((2,6)\).
  \item What is the direction of maximum change in \(V\) from the point \((3,4)\).
  \item What is the maximum rate of change?
  \end{enumerate}

\item The temperature in \(^{\circ}C\) of a flat plate is a function of the distances \(x\) [mm]  and \(y\) [mm]
  from the center of the plate.  The following table samples some values of the function \(T(x,y)\) in the
  vicinity of the point \((-2,1)\).  Assume that \(T(x,y)\) is continuous and differentiable everywhere (i.e., is
  smooth).

  \begin{tabular}{|c||c|c|c|c|c|}
    \hline
    1.2 & 36.12 & 35.45 & 34.80 & 34.17 & 33.56 \\
    \hline
    1.1 & 35.18 & 34.53 & 33.90 & 33.29 & 32.70 \\
    \hline
    1.0 & 34.24 & 33.61 & 33.00 & 32.41 & 31.84 \\
    \hline
    0.9 & 33.30 & 32.69 & 32.10 & 31.53 & 30.98 \\
    \hline
    0.8 & 32.36 & 31.77 & 31.20 & 30.65 & 30.12 \\
    \hline
    \hline
    \slashbox{y}{x} & -2.2 & -2.1 & -2.0 & -1.9 & -1.8 \\
    \hline
  \end{tabular}

  Select the best estimate for \(\left\lVert\nabla T(-2,1)\right\rVert\):
  \begin{enumerate}
  \item \(10.8\)
  \item \(15.0\)
  \item \(4.0\)
  \item \(12.6\)
  \item \(20.8\)
  \end{enumerate}
  and select the best estimate for the direction of \(\nabla T(-2,1)\):
  \begin{enumerate}
  \item \(\langle-3,-2\rangle\)
  \item \(\langle3,3\rangle\)
  \item \(\langle-3,2\rangle\)
  \item \(\langle2,-3\rangle\)
  \item \(\langle-2,3\rangle\)
  \end{enumerate}

\item The temperature over a particular surface is given by:
  \[T(x,y,z)=300e^{-x^2-3y^2-7z^2}\]
  where \(T\) is measured in \(^{\circ}C\) and \(x,y,x\) is measured in meters.  Consider an ant on the surface at
  position \(P(2,-1,5)\) who starts crawling toward the point \(Q(5,-2,6)\).  For each of the following questions,
  all answers must be in exact form (i.e., no decimals).
  \begin{enumerate}
  \item Determine the rate of change of temperature in the indicated direction.
  \item Is the temperature getting hotter or cooler in the indicated direction.
  \item In what direction does the temperature increase the fastest?
  \item What is the maximum rate of change at \(P\).
  \end{enumerate}

\item Use the Lagrange multiplier technique to determine the extreme values of the function \(f(x,y)=xe^y\) subject
  to the constraint \(x^2+y^2=3\).
\end{enumerate}

\subsection*{Multiple Integrals}

\begin{enumerate}
\item Evaluate the integral \(\displaystyle\iint_Dxy\,dA\) where \(D\) is the triangle with vertices
  \((0,2)\), \((3,-1)\), and \((3,2)\).
  
\item Consider the integral \(\displaystyle\iint_D(x^2+2y)dA\) where \(D\) is the region bounded by \(y=x\),
  \(y=x^3\), and \(x\ge0\).
  \begin{enumerate}
  \item Set up (but do not evaluate) the integral as a type 1 region.
  \item Set up (but do not evaluate) the integral as a type 2 region.
  \item Evaluate once using either order of integration.
  \end{enumerate}
  
\item Evaluate \(\displaystyle\iint_D(x+y)\,dA\) where \(D\) is the region that lies to the left of the \(y\)-axis
  between the circles \(x^2+y^2=1\) and \(x^2+y^2=4\).

\item Determine the area of the surface \(z=xy\) that lies within the cylinder \(x^2+y^2=4\).

\item Consider the lamina that occupies the region \(D\) bounded by the parabolas \(y=x^2\) and \(x=y^2\) with
  density \(\p(x,y)=\sqrt{x}\).
  \begin{enumerate}
  \item Determine the total mass of the lamina.
  \item Determine the position of the lamina's center of mass.
  \end{enumerate}

\item Determine the volume of the solid under the paraboloid \(z=x^2+4y^2\) and above the rectangle
  \(R=[0,2]\times[1,4]\).

\item Evaluate \(\displaystyle\iiint_Ez\,dV\) where \(E\) is the region enclosed by the paraboloid \(z=x^2+y^2\)
  and the plane \(z=4\).

\item Which of the following describes the boundary of \(E\) in the following integral:
  \[\iiint_Ef(x,y,z)dV=\int_{-1}^1\int_{-\sqrt{1-x^2}}^{\sqrt{1-x^2}}\int_{\sqrt{x^2+y^2}}^{2-x^2-y^2}\]
  \begin{enumerate}
  \item A paraboloid and a plane
  \item A sphere
  \item A cone and a half-sphere
  \item A paraboloid and a half-sphere
  \item A cone and a plane
  \item An ellipsoid
  \item Two cones
  \item A cone and a paraboloid
  \end{enumerate}

\item Evaluate the integral \(\displaystyle\iint_D(x^2+y^2)y\,dA\) where \(D\) is the region bounded by the
  circle \(x^2+y^2=2x\) using polar coordinates.

\item Use cylindrical coordinates to determine the volume of the solid that is enclosed by the cone
  \(z=\sqrt{x^2+y^2}\) and the sphere \(x^2+y^2+z^2=50\).  Your answer must be in exact form (i.e., no decimals).

\item Use cylindrical coordinates to evaluate \(\displaystyle\iint_E(x^3+xy^2)dV\) where \(E\) is the solid in the
  first octant the lies beneath the paraboloid \(z=1-x^2-y^2\).

\item Use spherical coordinates to determine the volume of the solid that lies within the sphere \(x^2+y^2+z^2=36\),
  above the \(xy\)-plane, and before the cone \(z=\sqrt{x^2+y^2}\).  Your answer must be in exact form (i.e., no
  decimals).

\item Evaluate \(\displaystyle\iiint_Exe^{x^2+y^2+z^2}\,dV\) where \(E\) is the portion of the unit ball
  \(x^2+y^2+z^2\le1\) that lies in the first octant.

\item Change \(\displaystyle\int_0^1\int_0^{\sqrt{1-x^2}}\int_{\sqrt{x^2+y^2}}^{\sqrt{2-x^2-y^2}}xy\,dzdydx\) to spherical
  coordinates and evaluate it.
\end{enumerate}

\end{document}
