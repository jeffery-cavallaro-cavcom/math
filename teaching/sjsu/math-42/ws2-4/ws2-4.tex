\documentclass[letterpaper,12pt,fleqn]{article}
\usepackage{matharticle}
\usetikzlibrary{positioning}
\pagestyle{plain}

\begin{document}

\begin{center}
  \large Math-42 Worksheet \#12

  \textbf{Sequences and Summations}
\end{center}

\vspace{0.5in}

\begin{enumerate}[left=0in,itemsep=0.5in]
\item Determine the first \(5\) terms (starting from \(a_0\)) of each of the following sequences:
  \begin{enumerate}
  \item \(a_n=n\)
  \item \(a_n=\frac{1}{n}\)
  \item \(a_n=2n-3\)
  \item \(a_n=(-1)^n\frac{1}{2^n}\)
  \item \(a_n=(n+1)^2-n^2\)
  \end{enumerate}

\item Determine a closed form for each of the following sequences:
  \begin{enumerate}
  \item \(1,4,7,10,13,\ldots\)
  \item \(-2,3,8,13,18,\ldots\)
  \item \(1,4,9,16,25,\ldots\)
  \item \(1,8,27,81,243,\ldots\)
  \item \(2,6,18,54,162,\ldots\)
  \end{enumerate}

\item Determine the first \(5\) terms of each of the following sequences:
  \begin{enumerate}
  \item \(a_n=2a_{n-1}+1, a_0=3\)
  \item \(a_n=a_{n-1}^2-4, a_0=0\)
  \item \(a_n=2a_{n-1}+3a_{n-2}, a_0=1, a_1=-1\)
  \item \(a_n=\frac{a_{n-2}}{a_{n-1}}, a_0=1, a_1=2\)
  \end{enumerate}

\item Calculate the following sums:
  \begin{enumerate}
  \item \(\displaystyle\sum_{k=3}^7k\)
  \item \(\displaystyle\sum_{k=-1}^4k^2\)
  \item \(\displaystyle\sum_{k=1}^{10}(-1)^n\)
  \item \(\displaystyle\sum_{k=0}^{5}(-1)^n\frac{1}{2^n}\)
  \item \(\displaystyle\sum_{k=0}^{10}3(2^k)\)
  \item \(\displaystyle\sum_{k=10}^{20}(-2)^k\)
  \item \(\displaystyle\sum_{k=1}^n(3k^2+5k-2)\)
  \item \(\displaystyle\sum_{k=1}^{\infty}\frac{2}{3^k}\)
  \item \(\displaystyle\sum_{k=0}^{\infty}\frac{2}{(-3)^k}\)
  \item \(\displaystyle\sum_{k=3}^{\infty}\left(-\frac{1}{2}\right)^k\)
  \end{enumerate}

\item A \emph{telescoping} sum is a special type of sum where most of the terms cancel each other out.  For
  example, consider the sum:
  \[\sum_{k=1}^{100}[(k+1)^2-k^2]\]
  Note that the terms of this sum are:
  \[(2^2-2^1)+(2^3-2^2)+(2^4-2^3)+\cdots+(2^{100}-2^{99})+(2^{101}-2^{100})\]
  Due to cancellation, the only terms left are:
  \[2^{101}-2^1\]
  The general form of a telescoping sum is:
  \[\sum_{k=m}^n(a_{k+1}-a_k)=a_{n+1}-a_m\]
  Calculate each of the following telescoping sums:
  \begin{enumerate}
  \item \(\displaystyle\sum_{k=1}^{10}\left(\frac{1}{k+1}-\frac{1}{k}\right)\)
  \item \(\displaystyle\sum_{k=5}^{49}\left(\frac{1}{k}-\frac{1}{k-1}\right)\)
  \item \(\displaystyle\sum_{k=1}^{100}[(k+1)^2-k^2]\)
  \item \(\displaystyle\sum_{k=1}^{100}[k^3-(k-1)^3]\)
  \end{enumerate}

\item An important identity is Gauss's Formula:
  \[\sum_{k=1}^nk=\frac{n(n+1)}{2}\]
  Prove this formula starting with sum:
  \[\sum_{k=1}^n[(k+1)^2-k^2]\]

\item Prove the following:
  \[\sum_{k=1}^nk^2=\frac{n(n+1)(2n+1)}{6}\]
  Which telescoping sum do you start with this time?
\end{enumerate}

\end{document}
