\documentclass[letterpaper,12pt,fleqn]{article}
\usepackage{matharticle}
\usetikzlibrary{positioning}
\pagestyle{plain}

\begin{document}

\begin{center}
  \large Math-42 Worksheet \#11

  \textbf{Functions}
\end{center}

\vspace{0.5in}

\begin{enumerate}[left=0in,itemsep=0.5in]
\item Determine if each of the following is a function.  If so then identify the domain and range.  If not then
  indicate why not.
  \begin{enumerate}
  \item \(\set{(1,a),(2,b),(3,a),(4,b)}\)

    \bigskip

  \item
    \begin{tikzpicture}
      \draw (0,0) ellipse (1 and 2);
      \draw (4,0) ellipse (1 and 2);
      \node (1) [closed point] at (0,1) {};
      \node [left=1ex of 1] {\(1\)};
      \node (2) [closed point] at (0,0) {};
      \node [left=1ex of 2] {\(2\)};
      \node (3) [closed point] at (0,-1) {};
      \node [left=1ex of 3] {\(3\)};
      \node (a) [closed point] at (4,0.5) {};
      \node [right=1ex of a] {\(a\)};
      \node (b) [closed point] at (4,-0.5) {};
      \node [right=1ex of b] {\(b\)};
      \draw (1) to (b);
      \draw (2) to (a);
    \end{tikzpicture}

    \bigskip

  \item \(f(x)=\sqrt{4-x^2}\)
  \item \(f(x)=\frac{1}{x}\)

    \bigskip

  \item
    \begin{tikzpicture}
      \draw [<->] (-3,0) -- (3,0) node [right] {\(x\)};
      \draw [<->] (0,-3) -- (0,3) node [above] {\(y\)};
      \draw (0,0) circle [radius=1.5];
      \node [below right] at (1.5,0) {\(2\)};
    \end{tikzpicture}
  \end{enumerate}

\item Consider the function \(f:\R\to\R\) defined by \(f(x)=x^3\).
  \begin{enumerate}
  \item Prove that \(f(x)\) is injective (one-to-one) using a standard injection proof.
  \item Prove that \(f(x)\) is surjective (onto) using a standard surjection proof.
  \item Prove that \(f(x)\) is bijective.  (Hint: only one line is required).
  \end{enumerate}

\item Let \(g:A\to B\) and \(f:B\to C\).  Prove each of the following:
  \begin{enumerate}
  \item If \(f\) and \(g\) are injective then \(f\circ g\) is injective.
  \item If \(f\) and \(g\) are surjective then \(f\circ g\) is surjective.
  \item If \(f\circ g\) is surjective then \(f\) is surjective.
  \item If \(f\circ g\) is injective then \(g\) is injective.
  \end{enumerate}

\item Let \(g:A\to B\) and \(f:B\to C\).  Use a function diagram to provide a counterexample to each of the
  following false propositions:
  \begin{enumerate}
  \item If \(f\circ g\) is surjective then \(g\) is surjective.
  \item If \(f\circ g\) is injective then \(f\) is injective.
  \end{enumerate}

\item Let \(f:A\to B\) and \(S,T\subseteq A\).  Prove that \(f(S\cup T)=f(S)\cup f(T)\).

\item Let \(f:A\to B\) and \(S,T\subseteq A\):
  \begin{enumerate}
  \item Prove that \(f(S\cap T)\subseteq f(S)\cap f(T)\)
  \item Which step in your proof is non-reversible, thus precluding equality, and why?
  \item What is required of \(f\) so that equality holds and why?
  \end{enumerate}

\item Let \(f:A\to B\) and \(S,T\subseteq B\).  Prove each of the following:
  \begin{enumerate}
  \item \(f^{-1}(S\cup T)=f^{-1}(S)\cup f^{-1}(T)\)
  \item \(f^{-1}(S\cap T)=f^{-1}(S)\cap f^{-1}(T)\)
  \end{enumerate}

\item Let \(f:A\to B\) and \(S\subset A\).  Draw a function diagram to demonstrate why \(f^{-1}(f(S))\supseteq S\).
  What is required of \(f\) so that equality holds?

\item Let \(f:A\to B\) and \(T\subset A\).  Draw a function diagram to demonstrate why \(f(f^{-1}(T))\subseteq T\).
  What is required of \(f\) so that equality holds?

\item Evaluate the following:
  \begin{enumerate}
  \item \(\floor{\pi}\)
  \item \(\ceil{\pi}\)
  \item \(\floor{-\pi}\)
  \item \(\ceil{-\pi}\)
  \end{enumerate}
\end{enumerate}

\end{document}
