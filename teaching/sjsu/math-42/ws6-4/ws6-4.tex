\documentclass[letterpaper,12pt,fleqn]{article}
\usepackage{matharticle}
\pagestyle{plain}

\begin{document}

\begin{center}
  \large Math-42 Worksheet \#21

  \textbf{Binomial Coefficients and Identities}
\end{center}

\vspace{0.5in}

\begin{enumerate}[left=0in,itemsep=0.5in]
\item Expand: \((x+y)^6\)

\item Expand: \((2x+3y)^4\)

\item Expand: \((x^2-2y)^3\)

\item What is the \(x^4y^6\) coefficient of \((x+y)^{10}\).

\item What is the \(x^6y^4\) coefficient of \((x+y)^{10}\).

\item What is the \(x^4y^6\) coefficient of \((x-2y)^{10}\).

\item We know that \(\displaystyle2^n=\sum_{k=1}^n\binom{n}{k}\).  Construct similar expressions for \(3^n\) and
  \(4^n\).  Can you generalize this for \(m^n\)?

\item Prove Pascal's Identity analytically (i.e., by manipulating the factorials).

\item Construct a clear committee selection argument for Pascal's Identity.

\item For this last exercise we will prove the binomial theorem using induction:
  \[(x+y)^n=\sum_{k=1}^n\binom{n}{k}x^{n-k}y^k\]
  \begin{enumerate}
  \item Start with the base case for \(n=0\).
  \item Write the inductive hypothesis.
  \item Rewrite \((x+y)^{n+1}\), apply the inductive hypothesis, and then distribute.
  \item Bring the extra \(x\) and \(y\) into their respective sums (since they are not dependent on the index).
  \item Move the index on the sum with the extra \(y\) from \(0\) to \(1\).
  \item Pull out the \(k=0\) term from the sum with the extra \(x\).
  \item Combine the two sums.
  \item Apply Pascal's Identity.
  \item Fold the \(k=0\) term back into the sum.
  \end{enumerate}
\end{enumerate}

\end{document}
