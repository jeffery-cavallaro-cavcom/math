\documentclass[letterpaper,12pt,fleqn]{article}
\usepackage{matharticle}
\pagestyle{plain}

\renewcommand{\implies}{\rightarrow}

\begin{document}

\begin{center}
  \large Math-42 Worksheet \#4

  \textbf{Predicates and Qualifiers}
\end{center}

\vspace{0.5in}

\begin{enumerate}[left=0in,itemsep=0.5in]
\item Let \(P(x)\) be the predicate ``\(x\) is a rational number.''  Determine the truth values of the following
  propositions:
  \begin{enumerate}
  \item \(P(0)\)
  \item \(P\left(\frac{1}{2}\right)\)
  \item \(P(\pi)\)
  \item \(P(0.125)\)
  \item \(P(-5)\)
  \item \(P(123.45\overline{67})\)
  \item \(P(\sqrt{2})\)
  \end{enumerate}

\item Let \(P(x,y)\) be the predicate ``\(xy\) is an even number.''  Determine the truth values of the following
  propositions:
  \begin{enumerate}
    \item \(P(2, 4)\)
    \item \(P(3, 1)\)
    \item \(P(5, -2)\)
    \item \(P(0, \pi)\)
    \item \(P\left(\pi, \frac{1}{\pi}\right)\)
  \end{enumerate}

\item Let \(x\in\set{-1,0,1,2,3,5,9}\) and let \(P(x)\coloneqq x\le5\).  Determine the truth values of the following
  quantified propositions.  State a counterexample for false universal quantifiers and an example for true
  existential quantifiers.
  \begin{enumerate}
  \item \(\forall\,xP(x)\)
  \item \(\lnot\forall\,xP(x)\)
  \item \(\forall\,x\lnot P(x)\)
  \item \(\exists\,xP(x)\)
  \item \(\lnot\exists\,xP(x)\)
  \item \(\exists\,x\lnot P(x)\)
  \end{enumerate}

\item Recall that \(p\implies q\) is true whenever \(p\) is false.  Thus, constructs like:
  \[\forall\,x(P(x)\implies Q(x))\]
  let \(P(x)\) act like a filter to only select particular values of \(x\).  If \(P(x)\) is false then the
  implication is true and the truth value of the universal quantifier is unaffected.  So let \(x\) and \(P(x)\)
  be the same as the previous problem and let \(Q(x)\coloneqq x>0\).  Determine the truth values of the following
  quantified propositions.  State a counterexample for false universal quantifiers.
  \begin{enumerate}
  \item \(\forall\,x(P(x)\implies Q(x))\)
  \item \(\forall\,x(P(x)\implies\lnot Q(x))\)
  \item \(\forall\,x(\lnot P(x)\implies Q(x))\)
  \item \(\forall\,x(P(x)\implies\lnot Q(x))\)
  \end{enumerate}

\item Let \(R(x,y)\coloneqq ``x\ \text{is related to}\ y\text{''}\) and \(T(x)\coloneqq ``x\ \text{is a teenager''}\),
  where \(x\) and \(y\) are taken from a particular set of people.  Convert each of the following to logic
  expressions:
  \begin{enumerate}
  \item George is related to Mary.
  \item Everyone is related to Mary.
  \item Someone is related to Mary.
  \item Mary is related to someone.
  \item No one is related to Mary.
  \item Someone is not related to Mary.
  \item No one is related to themselves.
  \item If someone is a teenager then they are related to Mary.
  \item If someone is related to Mary then they are a teenager.
  \item Everyone is a teenager and related to Mary.
  \end{enumerate}

\item Negate each expression in the previous problem (hint: deMorgan).

\item An important concept that we will use in our first proofs is the concept of \emph{closure}.  Closure states
  that if you perform an operation on one or more operands (e.g., addition or multiplication) then the result is
  the same type of object as the two operands.  For example, if you add any two natural numbers then the result
  must be a natural number.  If \(\N\) is the set of natural numbers, then:
  \[\forall\,n,m\in\N,n+m\in\N\]
  is a statement of the closure of natural numbers under addition.
  \begin{enumerate}
  \item Convince yourself that the natural numbers are in fact closed under addition.
  \item Construct a similar logical expression for the closure of natural numbers under multiplication.  Is it
    true?
  \item If \(\Z\) is the set of integers, write similar logical expressions for the closure of integers under
    addition and multiplication.  Are they true?
  \item If \(\Q\) is the set of rational numbers, write similar logical expressions for the closure of rational
    numbers under addition and multiplication.  Are they true?
  \item If \(\R\) is the set of real numbers, write similar logical expressions for the closure of real numbers
    under addition and multiplication.  Are they true?
  \end{enumerate}

\item The set of irrational numbers is \emph{not} closed under addition or multiplication.  Find counterexamples
  where adding or multiplying two irrational numbers results in a rational number.
\end{enumerate}

\end{document}
