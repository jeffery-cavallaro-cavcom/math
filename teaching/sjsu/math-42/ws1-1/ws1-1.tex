\documentclass[letterpaper,12pt,fleqn]{article}
\usepackage{matharticle}
\usepackage{numprint}
\npthousandsep{,}
\renewcommand{\implies}{\longrightarrow}
\renewcommand{\iff}{\longleftrightarrow}
\pagestyle{plain}

\begin{document}

\begin{center}
  \large Math-42 Worksheet \#1

  \textbf{Propositional Logic}
\end{center}

\vspace{0.5in}

\begin{enumerate}[left=0in,itemsep=0.5in]
\item Determine whether or not the following sentences are statements of propositions.  If a sentence
  is a proposition then determine whether it is true or false.  If a sentence is not a proposition then indicate
  whether it is subjective or ambiguous.
  \begin{enumerate}
  \item Microsoft Windows is an operating system.
  \item Linux is an operating system.
  \item Linux is a better operating system than Windows.
  \item \(1+1=3\)
  \item \(\numprint{1000000}\) is a big number.
  \item \(x+1=4\)
  \item The letter `y' is a vowel.
  \item \(0\) is an even integer.
  \item \(0\) is a rational number.
  \item \(\pi\) is a rational number.
  \end{enumerate}

\item Negate the following propositions.  State whether the original or negated proposition is true.
  \begin{enumerate}
  \item It is raining in San Jose.
  \item \(7\) is a prime number.
  \item \(1+1=3\)
  \item \(2>3\)
  \item \(0\) is an even number.
  \end{enumerate}

\item Consider the following propositions:
  \begin{align*}
    p &\coloneqq\text{Skiing in Lake Tahoe is fun.} \\
    q &\coloneqq\text{Driving to Lake Tahoe is boring.} \\
    r &\coloneqq\text{Ann hates going to Lake Tahoe.}
  \end{align*}
  Represent the following statements using logical operators:
  \begin{enumerate}
  \item Skiing in Lake Tahoe is fun but it is boring to drive there.
  \item If driving to Lake Tahoe is fun then Ann likes going there.
  \item Driving to Lake Tahoe is entertaining and skiing there is fun.
  \item Ann likes going to Lake Tahoe if and only if driving to there is not boring and skiing there is fun.
  \item Driving to or skiing in Lake Tahoe is boring.
  \end{enumerate}

\item Rewrite each implication in \(p\implies q\) form:
  \begin{enumerate}
  \item There are clouds in the sky if it is raining.
  \item You will pass the exam only if you receive a score of \(70\) or better.
  \item \(x\) is an integer is sufficient to conclude that \(x\) is a rational number.
  \item \(x\) is an integer is a necessary condition for \(x\) to be a rational number.
  \item \(x\) is irrational unless \(x\) can be written as a ratio of integers \(\frac{p}{q}\) where \(q\ne0\).
  \end{enumerate}

\item Determine whether the following implications are true or false.
  \begin{enumerate}
  \item If \(0\) is an even number then \(\sqrt{2}\) is rational.
  \item If \(0\) is an even number then \(\sqrt{2}\) is irrational.
  \item If \(0\) is an odd number then \(\sqrt{2}\) is rational.
  \item If \(0\) is an odd number then \(\sqrt{2}\) is irrational.
  \end{enumerate}

\item Consider the proposition: if \(x=\sqrt{2}\) then \(x\) is a irrational number.
  \begin{enumerate}
  \item Construct the inverse, converse, and contrapositive.
  \item Of the four forms of the implication, which are true and which are false?
  \end{enumerate}

\item Determine whether the following equivalences are true or false.
  \begin{enumerate}
  \item Sacramento is the capital of California if and only if Carson City is the capital of Nevada.
  \item \(x\) is an even integer is equivalent to there exists an integer \(k\) such that \(x=2k\).
  \item Pigs can grow wings and fly iff the moon is made of green cheese.
  \item An integer is composite if and only if it is not prime.
  \item An integer \(x\) is a perfect square iff there exists an integer \(k\) such that \(x=k+k\).
  \end{enumerate}

\item Let \(p\), \(q\), and \(r\)  be propositions and consider the compound proposition:
  \[p\implies p\land\lnot q\iff q\lor r\]
  \begin{enumerate}
    \item Use parentheses to indicate the correct order of operation with respect to operator precedence.
    \item Construct a truth table for this 3-variable proposition.  Be sure to show each intermediary result and then
      the final result.
    \item Consider the following propositions:
      \begin{align*}
        p &\coloneqq\sqrt{2}\ \text{is a rational number.} \\
        q &\coloneqq0\ \text{is an even number.} \\
        r &\coloneqq x^2=1\implies x=1
      \end{align*}
      Using your truth table, indicate whether the compound proposition is true or false.  Be sure to clearly
      indicate the row that gives you the correct answer.
  \end{enumerate}

\end{enumerate}

\end{document}
