\documentclass[letterpaper,12pt,fleqn]{article}
\usepackage{matharticle}
\pagestyle{plain}

\begin{document}

\begin{center}
  \large Math-42 Worksheet \#8

  \textbf{Proof Methods and Strategy}
\end{center}

\vspace{0.5in}

\begin{enumerate}[left=0in,itemsep=0.5in]
\item Prove that \(63\) is not a perfect square.

\item Prove (by exhaustion) that for all positive integers \(n\le5\), \(n^2<25\).

\item Prove or disprove: \(a,b\in\Q\implies a^b\in\Q\).

\item Prove or disprove: \(a,b\in\R-\Q\implies a+b\in\R-\Q\)

\item Prove by cases: if \(\displaystyle\frac{x^2-4}{x+1}\ge0\) then \(x\in[-2,-1)\cup[2,\infty)\).  Be sure that
    any values not in the domain are separate cases.

\item Use the rational roots theorem and proof by exhaustion to prove that \(\sqrt{3}\) is irrational.

\item Prove by contradiction that \(\sqrt{3}\) is irrational.  (Hint: start the same way that you would to prove
  that \(\sqrt{2}\) is irrational.)

\item Prove that for all \(x,y\in\R\):
  \begin{gather*}
    \min\set{x,y}=\frac{x+y-\abs{x-y}}{2} \\
    \max\set{x,y}=\frac{x+y+\abs{x-y}}{2}
  \end{gather*}
  Is it necessary to uses cases? (Hint: AWLOG)

\item The real number axiom for the additive identity says:
  \[\exists\,0\in\R,\forall\,x\in\R,x+0=x\]
  Note that the axiom establishes existence, but says nothing about uniqueness.  Prove that there is no other real
  number that is an additive identity.  (Hint: start by assuming that there are at least two: \(0\) and \(0'\), and
  then show that \(0=0'\).

\item Likewise, the real number axiom for the additive inverse says:
  \[\forall\,x\in\R,\exists\,y\in\R,x+y=0\]
  Prove that for any real number \(x\), its additive inverse is unique.
\end{enumerate}

\end{document}
