\documentclass[letterpaper,12pt,fleqn]{article}
\usepackage{matharticle}
\pagestyle{plain}

\renewcommand{\implies}{\rightarrow}
\renewcommand{\iff}{\longleftrightarrow}

\begin{document}

\begin{center}
  \large Math-42 Worksheet \#6

  \textbf{Rules of Inference}
\end{center}

\vspace{0.5in}

\begin{enumerate}[left=0in,itemsep=0.5in]
\item Identify each rule of inference, or state that an argument is a fallacy:
  \begin{enumerate}
  \item \(n\in\N\) or \(n<0\).  \(n\notin\N\) or \(n=1\).  Therefore \(n<0\) or \(n=1\).
  \item If \(n\) is odd then \(n^2\) is odd.  \(n\) is odd.  Therefore \(n^2\) is odd.
  \item  For all \(n\in\Z\), if \(n\) is even then there exists \(k\in\Z\) such that \(n=2k\).  \(100\) is even.
    Therefore there exists \(k\in\Z\) such that \(100=2k\).
  \item \(n\in\N\) and \(n\in\Z\).  Therefore \(n\in\Z\).
  \item \(a<x\). \(x<b\).  Therefore \(a<x<b\).
  \item If \(n<n^2\) then \(n\ne1\).  If \(n\ne1\) then \(n+5\ne6\).  Therefore if \(n<n^2\) then \(n+5\ne 6\).
  \item If \(n=1\) then \(n^2=1\).  \(n^2=1\).  Therefore \(n=1\).
  \item If \(n\) is even then \(n^2\) is even.  \(n^2\) is odd.  Therefore \(n\) is odd.
  \item \(a\le b\).  \(a\ne b\).  Therefore \(a<b\).
  \item \(a<b\).  Therefore \(a\le b\).
  \item If \(n=1\) then \(n^2=1\).  \(n\ne1\).  Therefore \(n^2\ne1\).
  \item For all \(n\in\Z\), if \(n\) is even then \(n^2\) is even.  \(225=15^2\) is odd.  Therefore \(15\) is
    odd.
  \end{enumerate}

\item Consider the following two predicates:
  \begin{gather*}
    C(x)\coloneqq x\ \text{is an calculus problem} \\
    H(x)\coloneqq x\ \text{is a hard problem} \\
  \end{gather*}
  and the following premises:
  \begin{itemize}
  \item If \(x\) is a calculus problem then it is hard.
  \item Integration is a calculus problem.
  \item Solving a linear equation in one variable is not hard.
  \item Solving a system of \(100\) linear equation in \(100\) unknowns is not a calculus problem.
  \end{itemize}
  \begin{enumerate}
  \item Write each premise as a logical expression.
  \item What conclusions can you make from the premises?
  \item Which premise does not lead to any conclusion and why?
  \end{enumerate}

\item Consider the following three predicates:
  \begin{gather*}
    Z(x)\coloneqq x\ \text{is an integer} \\
    O(x)\coloneqq x\ \text{is odd} \\
    E(x)\coloneqq x\ \text{is even}
  \end{gather*}
  and the following premises:
  \begin{itemize}
  \item If \(x\) is an integer then it is either even or odd.
  \item \(a\) is an integer or \(b\) is an integer.
  \item \(a\) is neither even nor odd.
  \item \(b\) is not even.
  \end{itemize}
  \begin{enumerate}
  \item Write each premise as a logical expression.
  \item What three conclusions can you make from the premises?
  \item What is the rule of inference used for each conclusion?
  \end{enumerate}

\item One of the most important uses of rules of inference is modus ponens (or modus tolens) on a definition.  A
  definition is an equivalence: \(p\iff q\), meaning \(p\) is the same thing as \(q\).  Thus, if \(p\) is true then
  we can conclude that \(q\) is true and if \(q\) is true then we can conclude that \(p\) is true.  Also, by modus
  tolens, if \(q\) is false then \(p\) is false and if \(p\) is false then \(q\) is false.  Some common ways to
  state definitions are:
  \begin{itemize}
  \item \(p\iff q\)
  \item \(p\) if and only if \(q\)
  \item \(p\) iff \(q\)
  \item To say that \(p\) means \(q\)
  \item If \(p\) then \(q\).
  \end{itemize}
  Note that in the last case a simple implication is used; however, if it is understood that what is being stated
  is a definition then an equivalence is assumed.

  So consider the following definition of a rational number:
  \[x\in\Q\iff\exists\,p,q\in\Z,q\ne0\land x=\frac{p}{q}\]
  Use modus pollens or modus tollens to make conclusions from the following premises:
  \begin{enumerate}
  \item \(0.5\) is a rational number.
  \item \(0.5=\frac{1}{2}\)
  \item \(\pi\) is an irrational number.
  \item \(\forall\,p,q\in\Z,\sqrt{2}\ne\frac{p}{q}\)
  \end{enumerate}

\item Consider the following argument:
  \begin{itemize}
  \item For all candy, if the candy contains chocolate then it is good.
  \item A Hersey bar contains chocolate.
  \item A Hershey bar is good.
  \item There exists a good candy.
  \end{itemize}
  \begin{enumerate}
  \item Convert the propositions in the argument to logical expressions.
  \item If the first two propositions are premises, what are the rules of inference that result in the third and fourth
    propositions?
  \end{enumerate}

\item Consider the following argument:
  \begin{gather*}
    \exists\,x(P(x)\land Q(x)) \\
    \forall\,x(Q(x)\implies R(x)) \\
    P(a)\land Q(a) \\
    P(a) \\
    Q(a) \\
    R(a) \\
    P(a)\land R(a) \\
    \exists\,x(P(x)\land R(x))
  \end{gather*}
  Assuming that the first two lines are premises, what are the rules of inference that result in each of the
  remaining lines.
\end{enumerate}

\end{document}
