\documentclass[letterpaper,12pt,fleqn]{article}
\usepackage{matharticle}
\pagestyle{plain}

\begin{document}

\begin{center}
  \large Math-42 Worksheet \#7

  \textbf{Introduction to Proofs}
\end{center}

\vspace{0.5in}

\begin{enumerate}[left=0in,itemsep=0.5in]
\item Write the formal definition for \(x\in\Q\).

\item Prove that \(Q\) is closed under addition (hint: see text).

\item Prove that \(Q\) is closed under multiplication.

\item In class, you probably saw a proof by contradiction that \(\sqrt{2}\) is irrational.  Here is a direct proof:

  Let \(x=\sqrt{2}\).  This means that \(x^2=2\) and \(x^2-2=0\).  So, by the rational roots theorem, any rational
  root of \(x^2-2=0\) must be of the form \(\frac{p}{q}\) where \(p\) is a factor of \(-2\) and \(q\) is a factor of
  \(1\).  Thus, the only possible rational roots are \(\pm1\) and \(\pm2\).  But none of these four values is a
  solution to \(x^2-2=0\).  Therefore \(x\) is not rational, and is thus irrational.

  Repeat this proof for \(\sqrt{7}\).  Make sure that you understand each step.

\item Write the formal definition for \(n\in\Z\) is an even number.

\item Write the formal definition for \(n\in\Z\) is an odd number.

\item Study the definitions and convince yourself they do not preclude a value from being both even and odd.  Forget
  about what you think you know, just consider the content of the definitions.  Using the definitions, write down
  what it would mean for an integer to be both odd and even.  Be sure to use a different variable in each part.

\item We are used to assuming that ``not even'' means ``odd''; however, the definitions of even and odd do not
  support this immediately---it must be proved.  This exercise provides an outline of this proof.
  \begin{enumerate}
  \item First, prove that \(0\) is even (based on the definition of even).
  \item Next, prove that if \(n\) is even then \(n+1\) and \(n-1\) are odd.  Note that you will need to make use of
    the closure principle.
  \item Next, prove that if \(n\) is odd then \(n+1\) and \(n-1\) are even.  Note that you will need to make use of
    the closure principle.
  \item These first three steps prove that each integer can be labeled as even or odd.  Why did we need to start
    with showing that \(0\) is even?
  \item Next, prove that if \(n\) is even then it is not odd (by contradiction), using your equations from the
    previous problem.
  \item Finally, prove that if \(n\) is odd then it is not even.  This should be a one line proof!  Do not repeat
    the proof from the previous step.
  \end{enumerate}
\end{enumerate}

\end{document}
