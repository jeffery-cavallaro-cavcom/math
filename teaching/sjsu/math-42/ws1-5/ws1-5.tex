\documentclass[letterpaper,12pt,fleqn]{article}
\usepackage{matharticle}
\pagestyle{plain}

\renewcommand{\implies}{\rightarrow}
\newcommand{\e}{\epsilon}
\renewcommand{\d}{\delta}

\begin{document}

\begin{center}
  \large Math-42 Worksheet \#5

  \textbf{Nested Quantifiers}
\end{center}

\vspace{0.5in}

\begin{enumerate}[left=0in,itemsep=0.5in]
\item An important definition that you learn in calculus is the definition of the limit \(L\) of a function
  \(f(x)\) at \(x=a\):
  \[\forall\,\e>0,\exists\,\d>0,\forall\,x\in\R,\abs{x-a}<\d\implies\abs{f(x)-L}<\e\]
  Negate this proposition to say that \(L\) is not the limit of \(f(x)\) at \(x=a\).

\item There is a similar definition of continuity of a function:
  \[(f(a)\ \text{exists})\land(\forall\,\e>0,\exists\,\d>0,\forall\,x\in\R,\abs{x-a}<\d\implies\abs{f(x)-f(a)}<\e)\]
  Negate this proposition to say that \(f(x)\) is not continuous at \(x=a\).

\item Rewrite each of the following real number axioms using quantified propositions.  For example, the
  commutative property of addition states that two real numbers can be added in any order:
  \[\forall\,a,b\in\R,a+b=b+a\]
  \begin{enumerate}
  \item Commutative multiplication: two real numbers can be multiplied in any order.
  \item Associative addition: when adding three numbers, either the first two can be added first or the last two
    can be added first.
  \item Associative multiplication: when multiplying three numbers, either the first two can be multiplied first or
    the last two can be multiplied first.
  \item Additive identity: there exists \(0\in\R\) such that for any real number \(a\), \(a+0=a\).
  \item Multiplicative identity: there exists \(1\in\R\) such that for any real number \(a\), \(a\cdot1=a\).
  \item Additive inverse: for every real number \(a\) there exists a real number \(-a\) such that when added
    together you get the additive identity.
  \item Multiplicative inverse: for every real number \(a\), if \(a\ne0\) then there exists a real number \(a^{-1}\)
    such that when multiplied together you get the multiplicative identity.
  \item Distributive: for all real numbers \(a\), \(b\), and \(c\), \(a(b+c)=ab+ac\).
  \end{enumerate}

\item What is the difference between these two quantified propositions?:
  \begin{gather*}
    \exists\,0\in\R,\forall\,a\in\R,a+0=a \\
    \forall\,a\in\R,\exists\,0\in\R,a+0=a
  \end{gather*}

\item Look at your definitions for the additive and multiplicative identities.  Convince yourself that these
  definitions say nothing about \(0\) and \(1\) being unique identities---in other words, is it possible that there
  exists some other \(z\in\R\) such that \(z\ne0\) and for all \(a\in\R, a+z=a\)?  Forget about what you think you
  know and only pay attention to what the definitions say (hint: consider the use of the existential quantifier).
  In fact, uniqueness is something that we must prove (and we will do so later).

\item Likewise, do your definitions for the additive and multiplicative inverses say anything about uniqueness?

\item You probably learned that there is a form of the existential quantifier that requires uniqueness: \(\exists!\).
  We tend not to use this quantifier because it does not negate nicely.
  \begin{enumerate}
  \item What would be the negation of:
    \[\exists!\,a\in\R,a^2=0\]
  \item Instead, uniqueness is expressed using a conjection (and) that indicates both \emph{existence} and
    \emph{uniqueness}.  First, write the proposition that says that there exists some real number \(a\) such that
    \(a^2=0\).
  \item Now, state the uniqueness of that \(a\) by saying that for any real number \(b\), if \(b^2=0\) then it
    must be the case that \(a\) and \(b\) are the same (equal).
  \item Put these two propositions together in a conjuction for a proposition equivalent to the one using the
    unique existential quantitier.
  \end{enumerate}

\item Let \(K(x,y)\) be the predicate, ``\(x\) knows \(y\),'' where the domain of \(x\) and \(y\) is a particular
  set of people \(P\).  Rewrite each of the following statements as quantified propositions.  Note that you will
  need to understand the previous problem on uniqueness in order to rewrite a couple of these statements:
  \begin{enumerate}
  \item Everybody knows Jeff.
  \item Everybody knows somebody.
  \item There is somebody whom everybody knows.
  \item Nobody knows everybody.
  \item There is somebody whom Jeff does not know.
  \item There is somebody whom no one knows.
  \item There is exactly one person whom everybody knows.
  \item There are exactly two people whom Jeff knows.
  \item Everyone knows themselves.
  \item There is someone who knows no one besides themselves.
  \end{enumerate}

\end{enumerate}

\end{document}
