\documentclass[letterpaper,12pt,fleqn]{article}
\usepackage{matharticle}
\pagestyle{plain}

\begin{document}

\begin{center}
  \large Math-42 Worksheet \#22

  \textbf{Generalized Permutations and Combinations}
\end{center}

\vspace{0.5in}

\begin{enumerate}[left=0in,itemsep=0.5in]
\item How many \(10\)-letter strings can be made from the \(26\)-letter alphabet when:
  \begin{enumerate}
  \item Repetition is allowed.
  \item Repetition is not allowed.
  \item The word must be a sequence of zero or more Z's, followed by a sequence of zero or more A's, followed by
    a sequence of zero or more T's.
  \end{enumerate}

\item How many distinct strings can be constructed by arranging all the letters in MISSISSIPPI?  Previously, you
  learned how to do this by placing letters via selection.  This time, use the formula for arrangement of
  indistinguishable objects.

\item A hot dog cart offers chili-cheese dogs, mustard dogs, and kraut dogs.  How many different ways are there to
  select a dozen hot dogs when:
  \begin{enumerate}
  \item The cart can make up to twenty of each type.
  \item You want to buy at least four chili-cheese dogs.
  \item You want to buy at least two of each type.
  \item The cart can only make three more chili-cheese dogs. You could partition this into cases of \(0\), \(1\),
    \(2\), or \(3\) chili-cheese dogs selected; however, there is an easier way using a complement.
  \end{enumerate}

\item How many solutions are there to the equation \(x_1+x_2+x_3+x_4=23\) where the \(x_i\) are nonnegative
  integers and:
  \begin{enumerate}
  \item \(x_1,x_2,x_3,x_4\ge0\)
  \item \(x_1,x_2,x_3,x_4>0\)
  \item \(x_1,x_3,x_5\ge2\) and \(x_2,x_4\ge1\).
  \end{enumerate}

\item The so-called \emph{ball-and-walls} method for selection with repetition is fairly easy when there are
  no bounds or just lower bounds for each box.  It is a bit harder when there are upper bounds.  Let's work the
  following problem step by step:

  You have three boxes in which you must place red balls in the first box, green balls in the second box, and
  blue balls in the third box.  The total number of balls is limited to \(19\) and the number of balls allowed in
  each box is:
  \begin{gather*}
    1\le\abs{b_1}\le4 \\
    2\le\abs{b_2}\le6 \\
    5\le\abs{b_3}
  \end{gather*}
  How many ways are there to arrange the \(19\) balls in the three boxes?
  \begin{enumerate}
  \item Start by allocating the lower bounds to each box and restate the problem with the lower bounds removed.
  \item Focusing on the new problem, how many ways total are there if the upper bounds are ignored?
  \item How many ways are there if there are \(4\) or more red balls in the first box, ignoring the upper
    bound for the second box?
  \item How many ways are there if there are \(5\) or more green balls in the second box, ignoring the upper
    bound for the first box?
  \item How many ways are there for \(4\) more red balls in the first box \emph{and} \(5\) or more green balls in
    the second box?
  \item Can you put these results together to obtain the answer for the original problem?  (Hint: PIE).
  \end{enumerate}
\end{enumerate}

\end{document}
