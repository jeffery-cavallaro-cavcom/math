\documentclass[letterpaper,12pt,fleqn]{article}
\usepackage{matharticle}
\pagestyle{plain}
\begin{document}
\section*{Quadratic Equations}

\begin{definition}
  The general form of a quadratic equation is given by:
  \[ax^2+bx+c=0\]
  where $a\ne0$
\end{definition}

Note that if $a=0$ we have a linear equation.

Before we tackle such equations, we need to be clear on two points:
\begin{enumerate}
\item $\abs{x}=c$

  Ask for a value whose absolute value is $c$. From either the definition:
  the high school one or the notion of distance, $x=\pm c$.

  $\abs{x}=2$ \\
  $x=\pm2$

  $\abs{x}=0$ \\
  $x=0$

  $\abs{x}=-1$ \\
  no solutions

\item $a=b\iff a^r=b^r$?

  It depends! Don't do it without thinking!

\item $x^n=c$, n odd

  $x^3=8$ \\
  $(x^3)^{\frac{1}{3}}=8^{\frac{1}{3}}$ \\
  $x=2$

  $x^3=-8$ \\
  $(x^3)^{\frac{1}{3}}=(-8)^{\frac{1}{3}}$ \\
  $x=-2$

\item $x^n=c$, n even

  $x^2=4$ \\
  $(x^2)^{\frac{1}{2}}=4^{\frac{1}{2}}$ \\
  $\abs{x}=2$ \\
  $\abs{x}=\pm2$ \\

  $x^2=-4$ \\
  $(x^2)^{\frac{1}{2}}=(-4)^{\frac{1}{2}}$ \\
  no solution in $\R$

  Don't mix up $\sqrt{a}$, which asks for the principle (positive) root,
  compared to $x^2=c$, asking for solutions - they are two different questions!

\item $(x+d)^2=c$

  $(x+1)^2=4$ \\
  $\abs{x+1}=2$ \\
  $x+1=\pm2$ \\
  $x+1=2$ and $x+1=-2$ \\
  $x=1,-3$
\end{enumerate}

Case 1: $b=0$

This is the case above:

\begin{eqnarray*}
  2x^2-8 &=& 0 \\
  2x^2 &=& 8 \\
  x^2 &=& 4 \\
  (x^2)^{\frac{1}{2}} &=& 4^{\frac{1}{2}} \\
  \abs{x} &=& 2 \\
  x &=& \pm2 \\
\end{eqnarray*}

Case 2: $b\ne0$

If we can factor by inspection, then we get an easy answer:

Two solutions:

$x^2-5x+6=0$ \\
$(x-2)(x-3)=0$ \\
$x=2,3$

Repeated solution:

$x^2+2x+1=0$ \\
$(x+1)^2=0$ \\
$x=-1,-1$

Otherwise, complete the square!

Start with $a=1$:
\[x^2+bx\]
Want to know what constant to add so that it can be rewritten as $(x+d)^2$,
because once we do that, we can use the rules above to get at the variable.
\[x^2+bx+c=(x+d)^2=x^2+2d+d^2\]
So $c=d^2$ and $d=\frac{b}{2}$, so divide by 2 and then square.

\begin{example}
  $x^2+6x$ \\
  $b=6$ \\
  $d=\frac{6}{2}=3$
  $c=3^2=9$
  $x^2+6x+9=(x+3)^2$

  $x^2-2x$ \\
  $b=-2$ \\
  $d=\frac{-2}{2}=-1$ \\
  $c=(-1)^2=1$ \\
  $x^2-2x+1=(x-1)^2$

  $x^2-x$ \\
  $b=-1$ \\
  $d=-\frac{1}{2}$ \\
  $c=\left(-\frac{1}{2}\right)^2=\frac{1}{4}$ \\
  $x^2-x+\frac{1}{4}=\left(x-\frac{1}{2}\right)^2$
\end{example}

When $a\ne1$, divide it out first:
\begin{example}
  $2x^2+3x=2\left(x^2+\frac{3}{2}x\right)$ \\
  $b=\frac{3}{2}$ \\
  $d=\frac{3}{4}$ \\
  $c=\frac{9}{16}$ \\
  $2\left(x^2+\frac{3}{2}x+\frac{9}{16}\right)=2\left(x+\frac{3}{4}\right)^2$
\end{example}

\begin{minipage}[t]{3in}
  \begin{eqnarray*}
    2x^2+4x-3 &=& 0 \\
    2x^2+4x &=& 3 \\
    2(x^2+2x) &=& 3 \\
    x^2+2x &=& \frac{3}{2} \\
    x^2+2x+1 &=& \frac{3}{2}+1 \\
    (x+1)^2 &=& \frac{5}{2} \\
    \abs{x+1} &=& \sqrt{\frac{5}{2}} \\
    x+1 &=& \pm\sqrt{\frac{5}{2}} \\
    x &=& -1\pm\sqrt{\frac{5}{2}} \\
  \end{eqnarray*}
\end{minipage}
\begin{minipage}[t]{3in}
  \begin{eqnarray*}
    ax^2+bx+c &=& 0 \\
    ax^2+bx &=& -c \\
    x^2+\frac{b}{a}x &=& -\frac{c}{a} \\
    x^2+\frac{b}{a}x+\frac{b^2}{4a^2} &=& -\frac{c}{a}+\frac{b^2}{4a^2} \\
    \left(x+\frac{b}{2a}\right)^2 &=& \frac{b^2-4ac}{4a^2} \\
    x+\frac{b}{2a} &=& \pm\frac{\sqrt{b^2-4ac}}{2a} \\
    x &=& -\frac{b}{2a}\pm\frac{\sqrt{b^2-4ac}}{2a} \\
    x &=& \frac{-b\pm\sqrt{b^2-4ac}}{2a} \\
  \end{eqnarray*}
\end{minipage}

And viola! We have the quadratic formula! So, we can solve quadratics in one
of three ways (in order of desiredness):
\begin{enumerate}
\item Inspection
\item Completing the square
\item Quadratic formula
\end{enumerate}

\begin{definition}
  The discriminant of the quadratic equation $ax^2+bx+c=0$ is given by:
  \[D=b^2-4ac\]
\end{definition}

4 cases:
\begin{enumerate}
\item $D>0$ and a perfect square:

  Two rational solutions (probably by inspection):

  $x^2+4x+3$ \\
  $a=1, b=4, c=3$ \\
  $x=\frac{-4\pm\sqrt{4^2-4(1)(3)}}{2(1)}=\frac{-4\pm2}{2}=-1,-3$

  Note that this provides a factoring: $(x-r_1)(x-r_2)=0$

  $(x+1)(x+3)=0$

  Another example:

  $3x^2+10x+3$ \\
  $a=3, b=10, c=3$ \\
  $x=\frac{-10\pm\sqrt{10^2-4(3)(3)}}{2(3)}=\frac{-10\pm8}{6}=-3,-\frac{1}{3}$

  $\left(x+\frac{1}{3}\right)(x+3)=0$

  But if you foil this, you don't quite get the original, need to adjust so
  that the F works:
  
  $3\left(x+\frac{1}{3}\right)(x+3)=3(0)$ \\
  $(3x+1)(x+3)=3(0)$

  which is what we would get if we factored by inspection.

\item $D>0$ and not a perfect square:

  Two matching irrational roots, like the previous example.

  $2x^2+4x-3=0$ \\
  $x=-1\pm\sqrt{\frac{5}{2}}$ \\
  $[x-(-1+\sqrt{\frac{5}{2}})][x-(-1-\sqrt{\frac{5}{2}})]=0$ \\
  $[2x-2(-1+\sqrt{\frac{5}{2}})][x-(-1-\sqrt{\frac{5}{2}})]=0$ \\
  $[\sqrt{2}x-(-\sqrt{2}+\sqrt{5})][\sqrt{2}x-(-\sqrt{2}-\sqrt{5})]=0$ \\

\item $D=0$

  One repeated rational root:

  $x^2-2x+1=0$ \\
  $a=1, b=2, c=1$ \\
  $x=\frac{2\pm\sqrt{(-2)^2-4(1)(1)}}{2(1)}=\frac{2}{2}=1$ \\
  $(x-1)^2=0$

\item $D<0$

  No real solutions:

  $x^2+x+1=0$ \\
  $a=1, b=1, c=1$ \\
  $1^2-4(1)(1)=-3$

  $x^2+x=-1$ \\
  $x^2+x+\frac{1}{4}=-\frac{3}{4}$ \\
  $\left(x+\frac{1}{2}\right)^2=-\frac{3}{4}$

  stuck!
\end{enumerate}

\end{document}
