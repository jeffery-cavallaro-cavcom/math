\documentclass[letterpaper,12pt,fleqn]{article}
\usepackage{matharticle}
\pagestyle{plain}
\begin{document}
\section*{Equations}

\begin{definition}
  An equation is a syntactic construct of the form $e_1=e_2$, where $e_1$ and
  $e_2$ are expressions.
\end{definition}

The usual case is that one or both of the expressions contain one or more
variables. Almost all of our equations will contain a single variable,
usually, but not necessarily, called $x$.

\begin{example}
  Note that the expressions can be as complicated as desired:
  \[\frac{x^2+1}{2x}=x^3-x^2+5x-1\]
\end{example}

\begin{definition}
  An equation that is true for all possible values of the variables is called
  an \emph{identity}.
\end{definition}

\begin{example}
  $\forall\,x\in\R,2x+3x=5x$
\end{example}

But more often, an equation is only true for a limited (possible 0) number of
values. The goal is to \emph{solve} the equation to determine those values.
Remember, expressions are \emph{evaluated} and equations are \emph{solved}.

\begin{example}
  $x^2=-1$ has no solutions in $\R$.

  $3x+1=4$ is only true for $x=1$

  $(x+1)(x-2)=0$ is only true for $x=-1$ and $x=2$

  $\sqrt{x}>1$ is true for $x\in(1,\infty)$
\end{example}

Our toolbox for solving equations contains only the rules from Chapter 0:
\begin{enumerate}
\item Arithmetic
\item Properties of equality
\item The substitution principle
\item Well-defined operators (do the same thing to both sides)
\item Closure and the 10 axioms (expand, factor, simplify)
\item Properties of zero
\item Properties of negatives
\item Properties of fractions
\item Exponent rules
\end{enumerate}

Do not make up your own rules!

\subsection*{Linear Equations}

\begin{minipage}[t]{2in}
  \begin{eqnarray*}
    3x-1 &=& 5 \\
    3x &=& 6 \\
    x &=& 2 \\
  \end{eqnarray*}
\end{minipage}
\begin{minipage}[t]{2in}
  \begin{eqnarray*}
    Ax+B &=& 0 \\
    Ax &=& -B \\
    x &=& -\frac{B}{A} \\
  \end{eqnarray*}
\end{minipage}

Note that the result is a sequence of reversible steps, each implying the
other (TFAE). In the final step, we can plug the found solution in to make
sure that it is a solution and complete the implication cycle.
\[3(2)-1=6-1=5\]

Goal: Isolate the variable so that a simplified for of $x=?$ is achieved.

\begin{example}[More Complex]
  \begin{eqnarray*}
    3(x+1)-5(2-x) &=& 2x-9 \\
    3x+3-10+5x &=& 2x-9 \\
    8x-7 &=& 2x-9 \\
    6x+2=0 \\
    6x=-2 \\
    x=-\frac{1}{3} \\
  \end{eqnarray*}
  \begin{eqnarray*}
    3\left(-\frac{1}{3}+1\right)-5\left(2+\frac{1}{3}\right) &=&
    2\left(-\frac{1}{3}\right)-9 \\
    3\left(\frac{2}{3}\right)-5\left(\frac{7}{3}\right) &=& -\frac{2}{3}-9 \\
    2-\frac{35}{3} &=& -\frac{29}{3} \\
    -\frac{29}{3} &=& -\frac{29}{3} \\
  \end{eqnarray*}
\end{example}

\begin{example}[No solution]
  \begin{eqnarray*}
    2(x+1)-2x &=& 1 \\
    2x+2-2x &=& 1 \\
    2 &\ne& 1 \\
  \end{eqnarray*}
\end{example}

\begin{example}[Identity]
  \begin{eqnarray*}
    2(x+1)-2x &=& 2 \\
    2x+2-2x &=& 2 \\
    2 &=& 2 \\
  \end{eqnarray*}
\end{example}

\subsection*{Rational Equations}

Recall:
\[\frac{a}{b}=\frac{c}{d}\iff ad=bc\]

Why does $\frac{1}{2}=\frac{2}{4}$? \\
Because $1\cdot4=2\cdot2$

Do not confuse this with the addition rule for fractions! That is for
expressions, this is for equations.

\begin{example}
  \listbreak
  \begin{eqnarray*}
    \frac{x-1}{x+2} &=& \frac{x+3}{x-4} \\
    (x-1)(x-4) &=& (x+2)(x+3) \\
    x^2-5x+4 &=& x^2+5x+6 \\
    10x &=& -2 \\
    x &=& \frac{1}{5} \\
  \end{eqnarray*}
\end{example}
\newpage
Sometimes, several fraction rules come into play:

\begin{example}
  \listbreak
  \begin{eqnarray*}
    \frac{1}{x}-\frac{1}{x-1} &=& \frac{1}{x-4} \\
    \frac{(x-1)-x}{x(x-1)} &=& \frac{1}{x-4} \\
    \frac{-1}{x(x-1)} &=& \frac{1}{x-4} \\
    -(x-4) &=& x(x-1) \\
    -x+4 &=& x^2-x \\
    x^2 &=& 4 \\
    x &=& \pm2 \\
  \end{eqnarray*}

  As an alternative method, multiply both sides by the common denominator. This
  is not a problem because none of the factors could have been 0:
  \begin{eqnarray*}
    \frac{1}{x}-\frac{1}{x-1} &=& \frac{1}{x-4} \\
    x(x-1)(x-4)\left(\frac{1}{x}-\frac{1}{x-1}\right) &=&
    x(x-1)(x-4)\left(\frac{1}{x-4}\right) \\
    (x-1)(x-4)-x(x-4) &=& x(x-1) \\
    x^2-5x+4-x^2+4x &=& x^2-x \\
    x^2 &=& 4 \\
    x &=& \pm2 \\
  \end{eqnarray*}
\end{example}

\subsection*{Pitfalls}

\begin{enumerate}
\item Multiplying both sides by a variable (0 is an annihilator):
  \begin{eqnarray*}
    1 &=& 2 \\
    1x &=& 2x \\
    x &=& 0 \\
  \end{eqnarray*}
  
\item Multiplying by the recipricol of a variable
  Solve $x^2=x$

  Incorrect:\\
  $\frac{1}{x}(x^2)=\frac{1}{x}(x)$ \\
  $x=1$

  Only one solution? But $x=0$ is also a solution!

  Correct: \\
  $x^2-x=0$ \\
  $x(x-1)=0$ \\
  $x=0$ or $x=1$

  Attempting to multiply both sides by $x^{-1}$ is not correct because $x$
  can be $0$ and $0$ has no multiplicative inverse.

\item Extraneous solutions (non-reversible steps)
  \begin{eqnarray*}
    \sqrt{2-x} &=& x-2 \\
    2-x &=& (x-2)^2 \\
    2-x &=& x^2-4x+4 \\
    x^2-3x+2 &=& 0 \\
    (x-1)(x-2) &=& 0 \\
    x &=& 1,2
  \end{eqnarray*}
  $x=2$ works; however, $x=1$ is extraneous.

\item Solutions not in the domain
  \begin{eqnarray*}
    \frac{\frac{1}{x}+\frac{1}{x-1}}{\frac{1}{x}-\frac{1}{x-1}} &=& -x \\
    \frac{(x-1)+x}{(x-1)-x} &=& -x \\
    \frac{2x-1}{-1} &=& -x \\
    2x-1 &=& x \\
    x &=& 1 \\
  \end{eqnarray*}
\end{enumerate}

\end{document}
