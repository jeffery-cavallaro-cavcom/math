\documentclass[letterpaper,12pt,fleqn]{article}
\usepackage{matharticle}
\pagestyle{plain}
\begin{document}
\section*{Exponential Expressions}

\begin{definition}
  An exponential expression is an expression of the form $a^b$, where $a$, the
  \emph{base} and $b$, the exponent, are real expressions.
\end{definition}

How an exponential expression is interpreted depends on what type of real
number $b$ is.

\subsection*{$b\in\N$}

\[a^1=a\]
\[a^2=aa\]
\[a^3=aaa\]
\[a^n=aa\cdots a\ (n\ \mbox{times})\]

\subsection*{$b\in\Z$}

\subsubsection*{$b=0$}

By definition:
\[a\ne0\implies a^0=1\]

\subsubsection*{$b<0$}

\[a^{-1}=\frac{1}{a}\]
\[a^{-2}=\frac{1}{a}\cdot\frac{1}{a}\]
\[a^{-3}=\frac{1}{a}\cdot\frac{1}{a}\cdot\frac{1}{a}\]
\[a^{-n}=\frac{1}{a}\cdot\frac{1}{a}\cdots\frac{1}{a}\ (n\ \mbox{times})\]

\subsection*{$b\in\Q$}

\begin{definition}
  Let $n\in\N,n>1$ and $a\in\R$:
  
  $\sqrt[n]{a}$ is the number that when raised to the $n^{th}$ power equals $a$
  \[b=\sqrt[n]{a}\iff a^n=b\]
  When $n=2$ the $n$ is omitted.

  The root sign is called a \emph{radical} and the expression inside the root
  is called the \emph{radicand}.
\end{definition}

\begin{example}
  \begin{minipage}[t]{3in}
    $\sqrt{1}=1$ because $1^2=1$ \\
    $\sqrt{4}=2$ because $2^2=4$ \\
    $\sqrt{9}=3$ because $3^2=9$ \\
    $\sqrt{16}=4$ because $4^2=16$
  \end{minipage}
  \begin{minipage}[t]{3in}
    $\sqrt[3]{1}=1$ because $1^3=1$ \\
    $\sqrt[3]{8}=2$ because $2^3=8$ \\
    $\sqrt[3]{27}=3$ because $3^3=27$ \\
    $\sqrt[3]{64}=4$ because $4^3=64$
  \end{minipage}
\end{example}

When $n$ is odd, there aren't any limitations because you can always take the
odd root of any real number:

$\sqrt[3]{-1}=-1$ because $(-1)^3=1$ \\
$\sqrt[3]{-8}=-2$ because $(-2)^3=8$ \\
$\sqrt[3]{-27}=-3$ because $(-3)^3=27$ \\
$\sqrt[3]{-64}=-4$ because $(-4)^3=64$

Note the placement of the parentheses: $(-a)^n\ne-a^n$ because the exponent has
higher precedence that the minus sign:

$(-2)^2=4$ but $-2^2=-4$

When $n$ is even, the radicand must be $\ge0$. For example, you cannot take
the square root of a negative number because a number squared is always
nonnegative:
\[\sqrt{-1}\notin\R\]
because $x^2=-1$ has no solutions in $\R$.

When $n$ is even, $\sqrt[n]{a}$ refers to the \emph{principle} root, which is
the nonnegative root. $-\sqrt[n]{a}$ refers to the negative root.

\begin{example}
  \[\sqrt{4}=2\]
  \[-\sqrt{4}=-2\]
\end{example}

\begin{definition}
  Let $p,q\in\Z$ such that $q>1$ and let $a\in\R$:
  \[\sqrt[q]{a}=a^{\frac{1}{q}}\]
  \[\sqrt[q]{a^p}=a^{\frac{p}{q}}\]
\end{definition}

\begin{example}
  \[\sqrt{100}=100^{\frac{1}{2}}=10\]
  \[\sqrt[3]{100}=\sqrt[3]{10^2}=10^{\frac{2}{3}}\]
  \[\sqrt[3]{\frac{1}{100}}=\sqrt[3]{\frac{1}{10}\cdot\frac{1}{10}}=
  \sqrt[3]{10^{-2}}=10^{-\frac{2}{3}}\]
\end{example}

\subsection*{$b\in\R$}

What in the heck does something like $2^{\pi}$ mean?

Using a calculator:
\[2^{\pi}=8.82498\]

Build a table of approximations using rational exponents that get arbitrarity
close to the exact value of $2^{\pi}$:

\begin{tabular}{c|c}
  $\pi$ & $2^{\pi}$ \\
  \hline
  3 & 8.00000\\
  3.1 & 8.57419 \\
  3.14 & 8.81524 \\
  3.141 & 8.82135 \\
  3.1415 & 8.82441 \\
  3.14159 & 8.82496 \\
\end{tabular}

\subsection*{Properties}

\begin{properties}
  $\forall\,a,b,c\in\R$ (not 0 when appropriate):
  \begin{enumerate}
  \item $a\ne0\implies a^0=1$
  \item $a^{-1}=\frac{1}{a}$
  \item $\left(\frac{a}{b}\right)^{-1}=\frac{b}{a}$
  \item $a^{-b}=\left(a^{-1}\right)^b=\left(\frac{1}{a}\right)^b=\frac{1}{a^b}$
  \item $a^{-b}=\left(a^b\right)^{-1}=\frac{1}{a^b}$
  \item $a^ba^c=a^{b+c}$
  \item $\frac{a^b}{a^c}=a^b\cdot\frac{1}{a^c}=a^ba^{-c}=a^{b-c}$
  \item $(ab)^c=a^cb^c$
  \item $\left(\frac{a}{b}\right)^c=\left(a\cdot\frac{1}{b}\right)^c=
    a^c\left(\frac{1}{b}\right)^c=a^c\left(\frac{1}{b^c}\right)=
    \frac{a^c}{b^c}$
  \item $\left(\frac{a}{b}\right)^{-c}=
    \left[\left(\frac{a}{b}\right)^{-1}\right]^c=
    \left(\frac{b}{a}\right)^c=\frac{b^c}{a^c}$
  \item $\left(a^b\right)^c=a^{bc}$
  \end{enumerate}
\end{properties}

We need to be careful when rational exponents cancel:

\begin{description}
\item Case 1: $(a^{\frac{1}{n}})^n$, $n$ odd

  No problems, sinced we can always take an odd root of any real number:
  \[(a^{\frac{1}{n}})^n=a\]

\item Case 2: $(a^{\frac{1}{n}})^n$, $n$ even

  We can assume that $a\ge0$, since we cannot take an even root of a negative
  number:
  \[(a^{\frac{1}{n}})^n=a\]

\item Case 3: $(a^n)^{\frac{1}{n}}$, $n$ odd

  No problems with odd roots:
  \[(a^n)^{\frac{1}{n}}=a\]

\item Case 4: $(a^n)^{\frac{1}{n}}$, $n$ even

  Always take the principle root:
  \[(a^n)^{\frac{1}{n}}=\abs{a}\]
\end{description}

\end{document}
