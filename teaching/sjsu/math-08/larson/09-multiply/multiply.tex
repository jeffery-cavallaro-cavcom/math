\documentclass[letterpaper,12pt,fleqn]{article}
\usepackage{matharticle}
\pagestyle{plain}
\begin{document}
\section*{Multiplying Terms}

Section 0.5 starts with a definition of a polynomial. Please read it; however,
I won't cover it formally until chapter 3, which is all about polynomials.

But what is important is the notion of combine like factors, which is an
application of the distributive rules:
\[ac+bc=(a+b)c\]

\begin{example}
  $2x+3x=(2+3)x=5x$

  $5\sqrt(x+1)+\pi\sqrt(x+1)=(5+\pi)\sqrt(x+1)$
\end{example}

In some problems you need to match up terms with like factors:

\begin{example}
  $(x^2+5x-3)-(2x^2-3x+5)=-x^2+8x-8$
\end{example}

And when going the other way, remember to apply the common factor to each and
every term in the parentheses:

\begin{example}
  $5x(x+1)=5x^2+5x$

  $2xy(3xz+2y+1)=6x^2yz+4xy^2+2xy$

  $x^{\frac{1}{2}}(x^2+x^{\frac{1}{3}}+1)=x^{\frac{5}{2}}+x^{\frac{5}{6}}+x^{\frac{1}{2}}$
\end{example}

A common problem is to multiply two terms by two terms:
\[(a+b)(c+d)=(a+b)c+(a+b)d=ac+bc+ad+bd\]

This is the so-called FOIL method (first, outer, inner, last).

\begin{example}
  $(x+2)(x-3)=x^2+2x-3x-6=x^2-x-6$
\end{example}

More generally, each term in the first gets multiplied by each term in the
second:

\begin{example}
  $(2xy+z)(3yz-2x-1)=6xy^2z-4x^2y-2xy+3xyz-2xz-z$

  $(x^2-3x+1)(x^2+2x+3)=$
  \begin{tabular}{ccccc}
    $x^4$ & $+2x^3$ & $+3x^2$ & & \\
    & $-3x^3$ & $-6x^2$ & $-9x$ & \\
    & & $x^2$ & $+2x$ & $+3$ \\
  \end{tabular}
  $=x^4-x^3-2x^2-7x+3$
\end{example}

Special Products:
\begin{enumerate}
\item Sum/Difference
  \[(a+b)(a-b)=a^2+ab-ab+b^2=a^2-b^2\]

  \begin{example}
    $(x+2)(x-2)=x^2-4$

    $(x+y)(x-y)=x^2-y^2$

    $(2x^2+3\sqrt{y})(2x^2-3\sqrt{y})=4x^4-9y$

    Note in the last example, the even root is on the inside, so y is assumed
    to be $\ge0$, so the absolute value is not needed.
  \end{example}

\item Two terms squared
  \[(a+b)^2=a^2+2ab+b^2\]

  \begin{example}
    $(x+3)^2=x^2+6x+9$

    $(2x-1)^2=(2x+(-1))^2=4x^2-4x+1$

    $(\sqrt{x}+\sqrt{y})^2=x+2\sqrt{xy}+y$
  \end{example}

\item Two terms cubed
  \[(a+b)^3=a^3+3a^2b+3ab^2+b^3\]

  \begin{example}
    $(x+1)^3=x^3+3x^2+3x+1$

    $(2x-3)^3=(2x+(-3))^3=8x^3-36x^2+54x-27$

    $(x-y)^3=(x+(-y))^3=x^2-x^2y+xy^2-y^3$
  \end{example}
\end{enumerate}

\end{document}
