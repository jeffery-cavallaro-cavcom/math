\documentclass[letterpaper,12pt,fleqn]{article}
\usepackage{matharticle}
\usepackage{centernot}
\newcommand{\Rp}{\R^+}
\pagestyle{plain}
\begin{document}
\section*{Inequalities}

\begin{definition}
  An inequality is two expressions separated by one of the four inequality
  signs:

  $e1 op e2$ where $op$ is one of $<,\le,\ge,>$
\end{definition}

Recall that we have two definitions of ``less than''. Definition 1 is
graphical: $a<b$ means that $a$ occurs to the left of $b$ on the real number
line:

\bigskip

\begin{tikzpicture}
  \draw [<->] (-3,0) -- (3,0);
  \draw [fill=black] (-1,0) circle [radius=0.05];
  \draw [fill=black] (1,0) circle [radius=0.05];
  \node [below] at (-1,0) {$a$};
  \node [below] at (1,0) {$b$};
\end{tikzpicture}

This leads us to all the graphical and interval-style notation that we have
already seen, which you can review on pp 126-7 in your textbook.

The second definition is more analytical. We start by defining the set of
positive real numbers:

\begin{definition}
  \[\Rp=\{x\in\R\mid x>0\}\]
\end{definition}

Note that this set is closed under addition and multiplication.

\begin{definition}
  To say that $a<b$ means $b-a\in\Rp$.
\end{definition}

As long as we are working on the \emph{same side} of an inequality then
we can use all of our previous rules regarding the manipulation of expressions.
But when we need to ``do something to both sides'', the rules are a little
different.

Recall the properties of equality:
\begin{properties}
  $\forall\,a,b,c\in\R$:
  \begin{enumerate}
  \item Reflexive: $a=a$
  \item Symmetric: $a=b\implies b=a$
  \item Transitive: $a=b$ and $b=c\implies a=c$
  \end{enumerate}
\end{properties}
\newpage
\begin{properties}
  $\forall\,a,b,c,d\in\R$:
  \begin{enumerate}
  \item Not reflexive

    Inequalities are not reflexive unless the ``or equals to'' part is
    included:
    
    $a\not<a$ \\
    $a\le a$

  \item Not symmetric
  
    Inequalities are not symmetric:

    $a<b\centernot\implies b<a$ \\
    $a\le b\centernot\implies b\le a$, unless $a=b$ \\

  \item Transitive

    $a<b$ and $b<c\implies a<c$

    This seems to make sense using the graphical definition. How can we show
    this using the analytical definition:

    Assume $a<b$ and $b<c$ \\
    $b-a\in\Rp$ and $c-b\in\Rp$ (definition) \\
    $(b-a)+(c-b)\in\Rp$ (closure) \\
    $c-a\in\Rp$ (axioms) \\
    $a<c$ (definition)

    The rest of the properties can be proved in this way.

  \item Addition of a constant

    $a<b\implies a+c<b+c$

    Thus, like equality, we can add the same thing to both sides.

    Using the graphical approach, this seems reasonable: if we translate $a$ and
    $b$ by the same amount then their relative positioning does not change.

  \item Addition of inequalities

    $a<b$ and $c<d\implies a+c<b+d$

    Again, this sounds reasonable: if we translate $a$ by some value but
    translate $b$ by even more, then the gap widens.

    Danger: This does not work with subtraction!

    $-5<0$ and $1<2$ but $1-(-5)=6\not<1-2=-1$

  \item Multiplication by a constant

    \begin{itemize}
    \item $a<b$ and $c>0\implies ac<bc$
    \item $a<b$ and $c<0\implies ac>bc$
    \end{itemize}

    As long as you multiply both sides by a positive number, the inequality
    stays the same; however, if you multiply both sides by a negative number
    then the inequality flips the other direction!

    $1<2$ \\
    $1(2)=2<4=2(2)$ \\
    $1(-2)=-2>-4=2(-2)$
  \end{enumerate}
\end{properties}

\subsection*{Linear Inequalities}

Remember: the answer is going to be a subset of the real number line, not
individual numbers!

\begin{example}
  $5x-1<2x+3$ \\
  $3x<4$ \\
  $x<\frac{4}{3}$ \\
  $(-\infty,\frac{4}{3})$
\end{example}

\begin{example}
  $1-5x\le2x+3$ \\
  $-7x\le2$ \\
  $x\ge-\frac{7}{2}$ \\
  $[\frac{7}{2},\infty)$
\end{example}

\subsection*(Polynomial Inequalities)

What do we do with something like:
\[x^2-3x-4>0\]
\begin{enumerate}
\item Put in a form that compares against 0.
\item Factor.
  \[(x-4)(x+1)>0\]
\item Identify the 0 points (like an equality) and graph them and mark them as
  either included or excluded (depending on equality allowed):
  \[x=-1,4\]
\item Note that these expressions can only change sign by passing through
  zero, so make a sign table with test points to see how the sign changes.

  \begin{tabular}{cccc}
    test & $x-4$ & $x+1$ & sign \\
    -2 & $-$ & $-$ & $+$ \\
    0 & $-$ & $+$ & $-$ \\
    5 & $+$ & $+$ & $+$ \\
  \end{tabular}

\item Choose the intervals with the proper sign, in this case $+$:
  \[(-\infty,-1)\cup (4,\infty)\]
\end{enumerate}

\begin{example}
  $2(x+1)(2-x)(x+3)\ge0$

  \begin{itemize}
  \item Beware of turned-around factors:
    \[-2(x+1)(x-2)(x+3)\ge0\]
      
  \item Divide out leading factors, especially negative ones! If negative
    then remember to turn the sign around.
    \[(x+1)(x-2)(x+3)\le0\]
      
  \item Solve:
    \[(-\infty,-3]\cup[-1,2]\]

  \item Beware of special conditions (p 141)

  \item Note that only odd factors will change sign:
    \[(x+4)(x-1)^2(x+2)^3\le0\]
    \[[-4,-2]\]
  \end{itemize}
\end{example}
  
\subsection*{Rational Inequalities}

Zeros vs poles/discontinuities.

\begin{itemize}
\item Can also change sign across a discontinuity.
\[\frac{x^2-4x-5}{x^2-7x+10}\ge0\]

\item Watch for holes caused by cancelled factors.
  Answer: $(-\infty,-1]\cup[2,5)\cup(5,\infty)$

\item Cannot cross multiply across an inequality!
\item Discontinuity points are never included; however zeros will be if
  equality is allowed.
  \[\frac{x+6}{x+1}\le2\]
  \[(-\infty,-1)\cup[4,\infty]\]
\end{itemize}

\end{document}
