\documentclass[letterpaper,12pt,fleqn]{article}
\usepackage{matharticle}
\pagestyle{plain}
\begin{document}
\section*{Factoring}

Factoring is the application of the distributive property to remove a set of
common factors from a collection of terms:
\[t_1+t_2+t_3+\ldots=(f_1+f_2)(f_3+f_4)\cdots\]
Sometimes the factors are obvious, sometimes not.

But why do we want to do this? Usually, because we want to compare the factors
to zero:

\begin{example}
  \listbreak
  \[x^2+3x-10=(x+5)(x-2)=0\]
  Due to our property of zero, we know that at least one of the factors must be
  zero.
\end{example}

Techniques:
\begin{enumerate}
\item Obvious common factors
  \begin{example}
    $3x+4x-5x=(3+4-5)x=2x$

    $2xy^2z+4xy-10xz+2x=2x(y^2z+2y-5z+1)$
  \end{example}

\item Difference of Squares
  \[a^2-b^2=(a+b)(a-b)\]

  \begin{example}
    $x^2-4=(x+2)(x-2)$

    $x^4-y^2=(x^2+y)(x^2-y)$

    $x-y=(\sqrt{x}+\sqrt{y})(\sqrt{x}-\sqrt{y})$
  \end{example}

\item Sum/Difference of Cubes
  \[a^3+b^3=(a+b)(a^2-ab+b^2)\]
  \[a^3-b^3=(a-b)(a^2+ab+b^2)\]

  \begin{example}
    $x^3+1=(x+1)(x^2-x+1)$

    $x^3-8=(x-2)(x^2+2x+4)$

    $x^3+y^3z^3=(x+yz)(x^2-xyz+y^2z^2)$
  \end{example}

\item Perfect Square
  \[a^2+2ab+b^2=(a+b)^2\]
  \[a^2=2ab+b^2=(a-b)^2\]

  \begin{example}
    $x^2+2x+1=(x+1)^2$

    $4x^2-4x+1=(2x-1)^2$

    $x^4+2x^2y^2+y^4=(x^2+y^2)^2$
  \end{example}

\item By inspection (backwards FOIL)
  $ax^2+bx+c=(nx+m)(rx+s)$

  \begin{enumerate}
  \item If the leading term is negative, factor out a (-1).
  \item Determine possible FIRST
  \item Determine possible LAST
  \item Try to match up possible LAST with OUTER+INNER
  \end{enumerate}

  \begin{example}
    $x^2+3x+2=(x+2)(x+1)$

    $-4x^2-16x+16=-(2x+4)^2$
  \end{example}

  \begin{description}
  \item Case 1: $c>0, b>0$

    Must be $+,+$

    \begin{example}
      $x^2+3x+2=(x+2)(x+1)$
    \end{example}

  \item Case 2: $c>0, b<0$

    Must be $-,-$

    $x^2-3x-2=(x-2)(x-1)$

  \item Case 3: $c<0, b>0$

    Must be $+,-$ (larger,smaller)

    $x^2+3x-10=(x+5)(x-2)$

    \item Case 4: $c<0, b<0$

      Must be $-,+$ (larger,smaller)

    $x^2-3x-10=(x-5)(x+2)$
  \end{description}

\item Grouping

  Read on your own. Comes up rarely.
\end{enumerate}

An important note: due to the way that the real numbers work, you can factor
anything from anything: multiply above and below by the thing that you want
to factor out.

\begin{example}
  $2=3\cdot\frac{2}{3}$

  $2x^2+2x+\frac{1}{2}=2(x^2+x+\frac{1}{4})=2(x+\frac{1}{2})^2$

  $x^{\frac{3}{2}}+2x^{\frac{1}{2}}+x^{-\frac{1}{2}}=x^{-\frac{1}{2}}(x^2+2x+1)=
  \frac{(x+1)^2}{\sqrt{x}}$

  $x^{\frac{1}{2}}+x^{\frac{1}{3}}=x^{\frac{1}{3}}(x^{\frac{1}{6}}+1)$
\end{example}

When factoring out rational exponents, always pick the smallest (negative)
exponent.

\end{document}
