\documentclass[letterpaper,12pt,fleqn]{article}
\usepackage{matharticle}
\usepackage{tikz}
\pagestyle{plain}
\renewcommand{\implies}{\rightarrow}
\renewcommand{\iff}{\leftrightarrow}
\newcommand{\e}{\epsilon}
\renewcommand{\d}{\delta}
\begin{document}

\begin{center}
  \large
  Math-42 Sections 01, 02, 05

  \Large
  Exam \#1
\end{center}

\vspace{0.5in}

Name: \rule{4in}{1pt}

\vspace{0.5in}

This exam is closed book and notes. You may use a calculator; however, no other electronics are allowed.  A
cheatsheet with a table of logical equivalences is provided on the last page.  Show all work; there is no credit
for guessed answers.

\vspace{0.5in}

\begin{enumerate}[left=0pt]
\item Prove the following logical equivalence using a truth table.  Be sure to show all intermediate steps and the
  final tautology.
  \[p\iff q\equiv p\land q\lor \lnot p\land\lnot q\]

  \newpage

\item You are in the land of knights and knaves.  Remember that knights always tell the truth (i.e., make a true
  statement) and knaves always lie (i.e., make a false statement).  You meet two men: \(A\) and \(B\).  \(A\) says,
  ``If I am a knight then he is a knave.''  \(B\) says, ``He is a knave.''  Determine whether each person is
  either a knight or a knave.  Be sure to explain why.

  \vspace{3in}

\item Prove using logical equivalences in a step by step fashion (i.e., do not skip steps) and justify each step:
  \[(p\implies q)\lor(p\implies r)\equiv p\implies(q\lor r)\]
  You are only allowed to use the equivalence of the implication and the rules stated in the cheatsheet on the
  last page.

  \newpage

\item Consider the eight types of rules of inference that we studied:
  \begin{enumerate}
  \item Modus Ponens
  \item Modus Tollens
  \item Hypothetical Syllogism
  \item Disjunctive Syllogism
  \item Addition
  \item Simplification
  \item Conjunction
  \item Resolution
  \end{enumerate}

  \vspace{0.5in}

  Identify the rule of inference used in each of the following arguments:
  \begin{itemize}
  \item[\underbar{\qquad}] \(n\in\N\) or \(n<0\).  \(n\notin\N\) or \(n=1\).  Therefore \(n<0\) or \(n=1\).
  \item[\underbar{\qquad}] If \(n\) is odd then \(n^2\) is odd.  \(n\) is odd.  Therefore \(n^2\) is odd.
  \item[\underbar{\qquad}] \(n\in\N\) and \(n\in\Z\).  Therefore \(n\in\Z\).
  \item[\underbar{\qquad}] \(a<x\) and \(x<b\).  Therefore \(a<x<b\).
  \item[\underbar{\qquad}] If \(n<n^2\) then \(n\ne1\).  If \(n\ne1\) then \(n+5\ne6\).  Therefore if \(n<n^2\) then
    \(n+5\ne 6\).
  \item[\underbar{\qquad}] If \(n\) is even then \(n^2\) is even.  \(n^2\) is odd.  Therefore \(n\) is odd.
  \item[\underbar{\qquad}] \(a\le b\).  \(a\ne b\).  Therefore \(a<b\).
  \item[\underbar{\qquad}] \(a<b\).  Therefore \(a\le b\).
  \end{itemize}

  \newpage

\item To say that a function \(f(x)\) is continuous at a point \(x=a\) means:
  \[\forall\,\e>0,\exists\,\d>0,\forall\,x\in\R,\abs{x-a}<\d\implies\abs{f(x)-f(a)}<\e\]
  State the definition that says a function \(f(x)\) is discontinuous (i.e., not continuous) at a point \(x=a\).

  \vspace{1.5in}

\item State the following definitions:
  \begin{enumerate}
  \item \(n\) is an even integer.

    \vspace{1.5in}

  \item \(n\) is an odd integer.
  \end{enumerate}

  \vspace{1.5in}

\item State the definition for \(x\in\Q\).

  \newpage

\item Prove by direct proof:
  \[\forall\,n,m\in\Z,(n,m\ \text{odd}\implies nm\ \text{odd})\]

  \vspace{5in}

\item Prove:
  \[\forall\,n,m\in\Z,(nm\ \text{even}\implies n\ \text{even or}\ m\ \text{even})\]
  
  \newpage

\item Prove:
  \begin{quote}
    The set of rational numbers is closed under multiplication.
  \end{quote}
\end{enumerate}

\newpage

Logical Equivalences:

\begin{tabular}{|l|l|}
  \hline
  EQUIVALENCE & NAME \\
  \hline
  \(\begin{array}{l}
  p\land T\equiv p \\
  p\lor F\equiv p
  \end{array}\) & Identity \\
  \hline
  \(\begin{array}{l}
  p\lor T\equiv T \\
  p\land F\equiv F
  \end{array}\) & Domination \\
  \hline
  \(\begin{array}{l}
  p\lor p\equiv p \\
  p\land p\equiv p
  \end{array}\) & Idempotent \\
  \hline
  \(\lnot(\lnot p)\equiv p\) & Double Negation \\
  \hline
  \(\begin{array}{l}
  p\lor q\equiv q\lor p \\
  p\land q\equiv q\land p \\
  \end{array}\) & Commutative \\
  \hline
  \(\begin{array}{l}
  (p\lor q)\lor r\equiv p\lor(q\lor r) \\
  (p\land q)\land r\equiv p\land(q\land r) \\
  \end{array}\) & Associative \\
  \hline
  \(\begin{array}{l}
  p\lor(q\land r)\equiv(p\lor q)\land(p\lor r) \\
  p\land(q\lor r)\equiv(p\land q)\lor(p\land r) \\
  \end{array}\) & Distributive \\
  \hline
  \(\begin{array}{l}
  \lnot(p\land q)\equiv\lnot p\lor\lnot q \\
  \lnot(p\lor q)\equiv\lnot p\land\lnot q \\
  \end{array}\) & DeMorgan \\
  \hline
  \(\begin{array}{l}
  p\lor(p\land q)\equiv p \\
  p\land(p\lor q)\equiv p \\
  \end{array}\) & Absorption \\
  \hline
  \(\begin{array}{l}
  p\lor\lnot p=T \\
  p\land\lnot p=F \\
  \end{array}\) & Negation \\
  \hline
\end{tabular}

\end{document}
