\documentclass[letterpaper,12pt,fleqn]{article}
\usepackage[margin=1in]{geometry}
\usepackage{libertine}
\usepackage{parskip}
\usepackage{url}
\usepackage{fancyhdr}
\usepackage{lastpage}
\usepackage{enumitem}
\lhead{}
\chead{}
\rhead{}
\lfoot{Math-42; Sections 01,02,05; Spring 2020 --- Cavallaro}
\cfoot{}
\rfoot{Page \thepage\ of \pageref{LastPage}}
\setlength{\footskip}{0.5in}
\renewcommand{\headrulewidth}{0pt}
\renewcommand{\footrulewidth}{1pt}
\pagestyle{fancy}
\begin{document}

\begin{center}
  \emph{San Jos\'{e} State University}

  Department of Mathematics and Statistics

  {\large Spring 2020}
  
  \begin{Large}
    \bfseries
    Math-42: Discrete Math

    Sections 01, 02, 05
  \end{Large}
\end{center}

\vspace{0.25in}

\subsection*{Course and Contact Information}

\begin{tabular}{|p{2in}|p{4.5in}|}
  \hline
  Instructor: & Jeffery Cavallaro \\
  \hline
  Office Location: & Duncan Hall 209 (the TA room) \\
  \hline
  Email: & \url{jeffery.cavallaro@sjsu.edu} \\
  \hline
  Office Hours: & TR 9am--11:30am \\
  \hline
  Class Days/Time: & \begin{minipage}{4.5in}
    \vspace{0.1cm}
    Section 01: MW 9:00am--10:15am

    Section 02: MW 10:30am--11:45am
    
    Section 05: TR 12:00pm--1:15pm
    \vspace{0.1cm}
  \end{minipage} \\
  \hline
  Classroom: & MacQuarrie Hall 424 \\
  \hline
  Prerequisites: & \begin{minipage}{4.5in}
    \vspace{0.1cm}
    A grade of B or higher in Math 19, or: \\
    A score of 80 or higher on the CPE.
    \vspace{0.1cm}
  \end{minipage} \\
  \hline
\end{tabular}

\subsection*{Course Description}

Discrete mathematics is a potpourri of topics involving the counting of a finite number of objects, studying the
relationships between the elements of countable collections of objects, and studying processes involving a finite
number of steps applied to such objects.  It is the gateway to classes in mathematics involving logic and proof,
number theory, combinatorics, probability, and graph theory.  It is also the gateway to classes in theoretical
computer science involving the study of computability and algorithms.

In this course we will concentrate on logic and methods of proof, na\"{\i}ve set theory, relations and functions,
sequences and summation, counting with permutations and combinations, discrete probability, and some basic integer
number theory.

The primary goal of this class is to train you to think mathematically and to develop some mathematical
sophistication so that you are prepared for later courses.

\subsection*{Course Learning Outcomes}

Upon successful completion of this course, students will be able to:
\begin{itemize}
\item Apply the rules of logic to analyze the truth value of simple, compound, and quantified mathematical
  statements.
\item Construct mathematical proofs using the methods of direct proof, proof by contradiction, contrapositive
  proof, and proof by induction.
\item Describe the basic tenets of na\"{\i}ve set theory and construct proper proofs related to subsets and
  set equality.
\item Understand functions and construct proper proofs involving function images and preimages.
\item Recognize sequences and series and determine closed forms in certain well-known cases.
\item Describe how relations differ from functions and identify equivalence relations.
\item Solve basic counting problems using permutations and combinations.
\item Determine the probabilities of discrete events.
\end{itemize}

\subsection*{Required Texts/Readings}

\subsubsection*{Textbook}

\emph{Discrete Mathematics and Its Applications}, Rosen, \textbf{8th edition}, ISBN: 978-1-259-67651-2.  Either the
physical book or the ebook is fine (your preference).  If using the ebook then make sure that you have a device on
which you can access it during class.

\subsubsection*{Web}

We will use both Canvas and Connect. All class communications, including written homework assignments and grades,
are available via Canvas (\url{sjsu.instructure.com}).  Connect (\url{connect.mheducation.com}) is used for the
major portion of the homework (see below).  Once you are registered for the course you should be able to see the
course listed on your Canvas account.  Each student must purchase a Connect license.

\subsubsection*{Calculator}

You should have a TI-84 calculator or equivalent to use on homework and exams; however, a calculator is generally
not helpful since the emphasis is on closed-form, descriptive solutions as opposed to numerical answers.

\subsection*{Course Requirements and Assignments}

\subsubsection*{Time}

You will need to spend a \emph{minimum} of 10 hours per week outside of class doing homework and studying. This
class is intensive and requires disciplined study habits.

\subsubsection*{Reading}

Reading from the textbook is assigned based on the course outline below.  Please read each section prior to the
corresponding lecture.  Read everything (not just the stuff in the boxes) and make sure that you understand and can
work all of the example problems.

\subsubsection*{Web Homework}

The web-based homework will be submitted via Connect.  Due dates are listed with the assignments and there are no
extensions.  The problems assigned on Connect are problems from the book; however, the software may randomize some
of the values involved.

\subsubsection*{Written Homework}

In addition to the web-based homework, there are ten small written homework sets.  Homework is assigned at the
beginning of each week and is due at the beginning of class on the first class day in the following week.  All
assignments must be submitted on paper.  Email and late submissions are not accepted.

\subsubsection*{Exams}

There are two regular exams and a comprehensive final exam.  The tentative regular exam schedule is as follows:

\bigskip

\begin{tabular}{|l|l|l|}
  \hline
  \textbf{EXAM} & \textbf{SECTIONS 01 and 02} & \textbf{SECTION 05} \\
  \hline
  1 & Monday, 3/2 & Tuesday, 3/3 \\
  \hline
  2 & Monday, 4/13 & Tuesday 4/14 \\
  \hline
\end{tabular}
  
\bigskip

Prior to an exam, I will post an announcement on canvas telling you exactly what to expect on the exam.  All exams
are closed book and closed notes.  A calculator (as described above) is allowed; however, any answers without
supporting work receive zero credit.

\subsubsection*{Final}

The final exam is comprehensive and is scheduled as follows:
\begin{description}
  \bfseries
\item{Section 01:} Monday, 5/18, 7:15am to 9:30am
\item{Section 02:} Friday, 5/15, 9:45am to 12:00pm
\item{Section 05:} Wednesday, 5/13, 9:45am to 12:00pm
\end{description}
\emph{Do not make any travel plans that occur prior to your exam date --- attendance is mandatory.}

\subsection*{Determination of Grades}

Your semester grade is determined as follows:

\bigskip

\begin{center}
  \begin{minipage}{2.5in}
    \centering
    \begin{tabular}{|c|c|}
      \hline
      Online Homework & 20\% \\
      \hline
      Written Homework & 10\% \\
      \hline
      Regular Exams & 40\% \\
      \hline
      Final Exam & 30\% \\
      \hline
    \end{tabular}
  \end{minipage}
  \begin{minipage}{2.5in}
    \centering
    \begin{tabular}{|l|c|}
      \hline
      A+ & 100--97 \\
      A & 96--93 \\
      A- & 92--90 \\
      B+ & 89--87 \\
      B & 86--83 \\
      B- & 82--80 \\
      C+ & 79--77 \\
      C & 76--70 \\
      C- & 69--65 \\
      D+ & 64--60 \\
      D & 59--55 \\
      D- & 54--50 \\
      F & \(<\)50 \\
      \hline
    \end{tabular}
  \end{minipage}
\end{center}

\subsection*{Course Content}

We will cover the following sections, tentatively scheduled on the following days:

\begin{center}
  \begin{tabular}{|c|c|c|c|}
    \hline
    \textbf{DAY} & \textbf{SECTION(s)} & \textbf{DAY} & \textbf{SECTION(s)}\\
    \hline
    1 & 1.1 & 16 & 4.1 \\
    \hline
    2 & 1.2 & 17 & 4.2--4.3* \\
    \hline
    3 & 1.3 & 18 & 5.1 \\
    \hline
    4 & 1.4--1.5 & 19 & 5.2 \\
    \hline
    5 & 1.6 & 20 & 6.1 \\
    \hline
    6 & 1.7 & 21 & EXAM 2 \\
    \hline
    7 & 12.1 & 22 & 6.2 \\
    \hline
    8 & 12.2 & 23 & 6.3--6.4 \\
    \hline
    9 & 2.1 & 24 & 6.5 \\
    \hline
    10 & 2.2 & 25 & 7.1--7.2 \\
    \hline
    11 & EXAM 1 & 26 & 9.1 \\
    \hline
    12 & 2.3 & 27 & 9.2--9.3 \\
    \hline
    13 & 2.4 & 28 & 9.4 \\
    \hline
    14 & 2.5* & 29 & 9.5--9.6* \\
    \hline
    15 & 2.6* & & \\
    \hline
  \end{tabular}
\end{center}

Sections marked with an asterisk (*) are optional and may be skipped depending on time.  Please read the assigned
sections prior to the corresponding class day.

\subsection*{Classroom Protocol}
  
\subsubsection*{Attendance}

I do not take attendance after the first week; however, it is important that you come (on time) to every class. The
book has more information than we could possibly cover, so I will highlight in class what is important. Bring your
book and calculator to every class meeting. If you miss a class, it is your responsibility to talk to your peers
and find out what you missed.

\subsubsection*{Holidays}

Class will not meet on the following days:

\begin{tabular}{cl}
  3/30--4/3 & Spring Break
\end{tabular}

\subsection*{University Policies}

Per University Policy S16-9 (\url{http://www.sjsu.edu/senate/docs/S16-9.pdf}), information relevant to all courses:
academic integrity, accommodations, dropping and adding, consent for recording of class, etc., is available on the
Office of Graduate and Undergraduate Programs’ Syllabus Information web page at
\url{http://www.sjsu.edu/gup/syllabusinfo}.  Please make sure to review these university policies and resources.

\end{document}
