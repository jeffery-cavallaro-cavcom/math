\documentclass[letterpaper,12pt,fleqn]{article}
\usepackage{matharticle}
\pagestyle{plain}
\begin{document}

\begin{center}
  \large
  Math-42 Sections 01, 02, 05

  \Large
  Homework \#12 Solutions
\end{center}

\subsection*{Problems}

How many possible 5-card poker hands from a standard deck are there with 3-of-a-kind (3 cards of the same rank and
2 additional cards).  You must NOT count 4-of-a-kind and full-house hands.  Thus the 2 additional cards cannot
match your 3-of-kind nor can they be a pair.  For full credit, you must explain each task in your procedure;
simply stating an answer receives no credit.

\begin{enumerate}
\item Select a rank for the 3-of-a-kind: \(C(13,1)\)
\item Select three cards from the selected rank: \(C(4,3)\)
\item Select two other ranks for the two remaining cards: \(C(12,2)\)
\item Select one card from the rank for the fourth card: \(C(4,1)\)
\item Select one card from the rank for the fifth card: \(C(4,1)\)
\end{enumerate}

\begin{align*}
  \text{Total number of ways} &= C(13,1)C(4,3)C(12,2)C(4,1)C(4,1) \\
  &= 13\cdot4\cdot\frac{12!}{10!2!}\cdot4\cdot4 \\
  &= 13\cdot4\cdot\frac{12\cdot11}{2}\cdot4\cdot4 \\
  &= 13\cdot4\cdot66\cdot4\cdot4 \\
  &= 54912
\end{align*}

\end{document}
