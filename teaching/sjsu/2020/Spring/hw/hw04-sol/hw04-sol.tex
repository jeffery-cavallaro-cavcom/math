\documentclass[letterpaper,12pt,fleqn]{article}
\usepackage{matharticle}
\pagestyle{plain}
\newcommand{\e}{\epsilon}
\renewcommand{\implies}{\rightarrow}
\begin{document}

\begin{center}
  \large
  Math-42 Sections 01, 02, 05

  \Large
  Homework \#4 Solutions
\end{center}

\subsection*{Problem}

One of the most important definitions in mathematics (calculus in particular) is that of the limit of a sequence.
Consider the infinite sequence: \(1,\frac{1}{2},\frac{1}{3},\frac{1}{4},\ldots\).  Such a sequence can be
represented as follows:
\[a_n=\frac{1}{n}, n\in\N\]
Note that the elements of the sequence get arbitrarily close to \(0\) as \(n\to\infty\).  We call such a point the
\emph{limit} of the sequence.

The formal definition for ``\(L\) is the limit of a sequence \(a_n\)'' is as follows:
\[\forall\,\e>0,\exists\,N\in\N,\forall\,n\in\N,(n>N\implies\abs{a_n-L}<\e)\]
Negate this proposition to obtain the definition of ``\(L\) is NOT the limit of a sequence \(a_n\).''

The following a step-by-step application of DeMorgan, but you could jump directly to the solution:

\begin{gather*}
\overline{\forall\,\e>0,\exists\,N\in\N,\forall\,n\in\N,(n>N\implies\abs{a_n-L}<\e})\equiv \\
\exists\,\e>0,\overline{\exists\,N\in\N,\forall\,n\in\N,(n>N\implies\abs{a_n-L}<\e})\equiv \\
\exists\,\e>0,\forall\,N\in\N,\overline{\forall\,n\in\N,(n>N\implies\abs{a_n-L}<\e})\equiv \\
\exists\,\e>0,\forall\,N\in\N,\exists\,n\in\N,\overline{n>N\implies\abs{a_n-L}<\e}\equiv \\
\exists\,\e>0,\forall\,N\in\N,\exists\,n\in\N,(n>N\ \text{and}\ \abs{a_n-L}\ge\e)
\end{gather*}

Remember that \(\overline{p\implies q}\equiv p\land\bar{q}\).  Also note that technically the parentheses are
required since quantifiers have the highest precedence.

\end{document}

