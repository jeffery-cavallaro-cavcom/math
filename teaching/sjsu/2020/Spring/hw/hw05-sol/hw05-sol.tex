\documentclass[letterpaper,12pt,fleqn]{article}
\usepackage{matharticle}
\pagestyle{plain}
\renewcommand{\implies}{\rightarrow}
\begin{document}

\begin{center}
  \large
  Math-42 Sections 01, 02, 05

  \Large
  Homework \#5 Solutions

\end{center}

Prove the following theorem:

\begin{theorem}
  \(\forall\,n\in\Z,n\) is either even or odd (but not both).
\end{theorem}

Your proof must address the following points:
\begin{enumerate}
\item \(n\) is even or odd (and nothing else).
\item \(n\) is odd \(\implies n\) is not even (hint: contradiction).
\item \(n\) is even \(\implies n\) is not odd (hint: contrapositive).
\end{enumerate}

The first point is a bit more difficult.  Start by making a statement about 0.  Then assuming that \(n\) is even,
what can you say about \(n-1\) and \(n+1\)?  Likewise, assuming that \(n\) is odd, what can you say about \(n-1\)
and \(n+1\).  Can you organize these facts into an argument that shows that you have accounted for all possible
\(n\in\Z\)?

\begin{proof}
  First, we need to prove that integers are either even or odd and nothing else.  So we start with \(0\):
  \[0=2(0)\]
  Therefore, \(0\) meets the definition of being an even number.

  Next, assume that \(n\in\Z\) is even.  This means that \(\exists\,k\in\Z,n=2k\).  Consider the next value:
  \(n+1=2k+1\).  This meets the definition of being an odd number.  Next, consider the previous value:
  \(n-1=2k-1=2k-2+1=2(k-1)+1\).  But \(k-1\in\Z\) by closure and so the previous value is also odd.  So both the
  next and previous values are odd.

  Now, assume that \(n\in\Z\) is odd.  This means that \(\exists\,k\in\Z,n=2k+1\).  Consider the next value:
  \(n+1=(2k+1)+1=2k+2=2(k+1)\).  But \(k+1\in\Z\) by closure and so the next value is even.  Next, consider the
  previous value: \((2k+1)-1=2k\).  This meets the definition of being an even number.  So both the next and
  previous values are even.

  In summary, we have showed that \(0\) is even, and every even is surrounded by odds, and every odd is surrounded
  by evens.  Thus, we have accounted for all the integers and shown that they are either even or odd and nothing
  else.

  Now, we need to show that a number cannot be both even an odd at the same time.  So assume that \(n\in\Z\) and
  assume that it is odd.  This means that \(\exists\,k\in\Z,n=2k+1\).  Now, ABC that \(n\) is also even.  This means
  that \(\exists\,\ell\in\Z,n=2\ell\).  Since these are both valid representations of \(n\):
  \begin{gather*}
    2k+1=2\ell \\
    2\ell-2k=1 \\
    2(\ell-k)=1 \\
    \ell-k=\frac{1}{2}
  \end{gather*}
  But this is a contradiction because it violates closure.  Therefore \(n\) is not even.

  Finally, we need to prove that if \(n\) is even then it is not odd.  But note that this is just the
  contrapositive of the previous proposition, and so we can conclude that this is a true statement.
\end{proof}

In summary, we have proved that every integer is either even or odd, but not both.  Thus, we can now say that
``not even'' means odd and ``not odd'' means even.

\end{document}
