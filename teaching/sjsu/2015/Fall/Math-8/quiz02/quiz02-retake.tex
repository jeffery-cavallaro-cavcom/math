\documentclass[letterpaper, 12pt]{article}

\setlength{\topmargin}{0in}
\setlength{\headheight}{0in}
\setlength{\headsep}{0in}
\setlength{\footskip}{0.5in}
\setlength{\textheight}{\paperheight}
\addtolength{\textheight}{-2in}
\addtolength{\textheight}{-\footskip}

\setlength{\oddsidemargin}{0in}
\setlength{\evensidemargin}{0in}
\setlength{\textwidth}{\paperwidth}
\addtolength{\textwidth}{-2in}

\pagestyle{empty}

\usepackage{amsfonts}

\begin{document}

\begin{center}
\bfseries
San Jos\'{e} State University \\
Fall 2015 \\
Math-8: College Algebra \\
Section 03: MW noon--1:15pm \\
Section 05: MW 4:30--5:45pm \\
\bigskip
Quiz \#2 (retake)
\end{center}

\bigskip

\begin{tabular}{|c|l|}
\hline
A1 & Commutative Addition \\
\hline
A2 & Associative Addition \\
\hline
A3 & Additive Identity (0) \\
\hline
A4 & Additive Inverse (-a) \\
\hline
M1 & Commutative Multiplication \\
\hline
M2 & Associative Multiplication \\
\hline
M3 & Multiplicative Identity \\
\hline
M4 & Multiplicative Inves (1/a) \\
\hline
RD & Right Distributive \\
\hline
LD & Left Distributive \\
\hline
CAN & Cancellation \\
\hline
SUB & Substitution \\
\hline
\end{tabular}

\newcommand{\fillin}{\rule{1in}{1pt}}

\vspace{1in}

1. Identify the parts of the term $-\frac{9a^2b^3}{2}$:

\bigskip

Coefficient: \fillin

\bigskip

Variables: \fillin

\bigskip

2. Identify the factors in the term $-\frac{x(x+1)^2y^3}{2}$: \fillin

\bigskip

3. If $ax=0$ and $a\ne0$, what do we know about $x$: \fillin

\newpage

4. A careful solution of $2(x-3)=1$ is given below. Give the rationale for each
step from the ten real number rules (A1--A4, M1--M4, LD, RD) and two additional
rules (SUB, CAN) that we discussed in lecture.  Note that some steps have two
things to identify.

\bigskip

\begin{tabular}{ll}
$2(x-3)=1$ & \\
$2x-6=1$ & \fillin, \fillin \\
$(2x-6)+6=1+6$ & \fillin \\
$(2x-6)+6=7$ & \fillin \\
$2x+(-6+6)=7$ & \fillin \\
$2x+0=7$ & \fillin, \fillin \\
$2x=7$ & \fillin, \fillin \\
$\frac{1}{2}(2x)=\frac{1}{2}(7)$ & \fillin \\
$\frac{1}{2}(2x)=\frac{7}{2}$ & \fillin \\
$(\frac{1}{2}2)x=\frac{7}{2}$ & \fillin \\
$1x=\frac{7}{2}$ & \fillin, \fillin \\
$x=\frac{7}{2}$ & \fillin, \fillin \\
\end{tabular}

\bigskip

5. Consider: $\frac{2(x+3)}{x}=-\frac{5}{2}$

\bigskip

a. Solve for x.

\vspace{2.5in}

b. Evaluate the left-hand side of the equation using your found solution to
prove that your solution is correct.

\end{document}
