\documentclass[letterpaper, 12pt]{article}

\setlength{\topmargin}{0in}
\setlength{\headheight}{0in}
\setlength{\headsep}{0in}
\setlength{\footskip}{0.5in}
\setlength{\textheight}{\paperheight}
\addtolength{\textheight}{-2in}
\addtolength{\textheight}{-\footskip}

\setlength{\oddsidemargin}{0in}
\setlength{\evensidemargin}{0in}
\setlength{\textwidth}{\paperwidth}
\addtolength{\textwidth}{-2in}

\pagestyle{empty}

\usepackage{amsfonts}

\begin{document}

\begin{center}
\bfseries
San Jos\'{e} State University \\
Fall 2015 \\
Math-8: College Algebra \\
Section 03: MW noon--1:15pm \\
Section 05: MW 4:30--5:45pm \\
\bigskip
Quiz \#2 (take-home)
\end{center}

\section*{Instructions:}

Print out this quiz, do it, and turn it in on Wednesday, Sept 9 at the start of
class. The quiz is open book and open notes with no time limit (other than the
due date), but do not use a calculator and do not work or discuss the quiz with
anyone else.

\bigskip

Good luck!

\bigskip

In problems 1--4, fill in the blanks.

\newcommand{\answer}[1]{\textbf{\underline{#1}}}

\bigskip

1. An algebraic expression is a sum of \answer{terms}.

\bigskip

Many of you tried to use the first definition in the first box of section 0.2
in the book. But if you look right under the box, you will see the correct
answer, and the answer is actually given in the next question!

\bigskip

2. An algebraic term is a product of constants (called \answer{coefficients})
and \answer{variables}.

\bigskip

Most of you got the variable part right, but many of you answered
``real number'' for the constant part.  But both the coefficients and the
variables are real numbers (remember - it's just a number).  Plus, the answer
is actually given in question 5!

\bigskip

3. An algebraic equation is two algebraic \answer{expressions} separated by
\answer{an equals sign}.

\bigskip

Most of you got this right.  Some answered ``terms'' separated by
``plus sign'', which is the definition for an expression (see problem 1), not
an equation.

\bigskip

4. We \answer{evaluate} an algebraic expression, but we \answer{solve} an
algebraic equation.

\bigskip

Most of you got this one correct also; however, some of you answered
``simplify'' instead of evaluate.  We simplify both expressions and equations,
depending on the problem, but we specifically evaluate expressions when we plug
in a value.  For equations, we solve to find that value.

\bigskip

5. Identify the parts of the term $-\frac{2xy^2z}{13}$:

\bigskip

Coefficient: \answer{$-\frac{2}{13}$}

\bigskip

I asked for the ``coefficient'', singular, which means that there is only one
answer, and that answer has to include all the constant values and most
definitely any minus sign.

\bigskip

Variables: \answer{$x, y, z$}

\bigskip

Many of you included the exponent on the $y$.  This is not correct.  The
exponent may be part of a ``variable term'', but the variable itself is just
$y$.

\bigskip

6. True or false: There exists $x,y\in\mathbb{R}$ such that $x,y\ne0$ and
$xy=0$.

\bigskip

FALSE!  The question is asking if we can multiply two non-zero numbers and get
zero. No! This is a very special property of the real numbers.  See the
properties of zero on page 14, property number 5.

\bigskip

7. Evaluate $5(3-(x-3))$ at $x=1$.

\bigskip

$5(3-((1)-3))=5(3-(-2))=5(3+(-1)(-2))=5(3+2)=5(5)=25$

\bigskip

A couple of you wrote that $3-(-2) = 6$.  But the problem is addition, not
multiplication.

\bigskip

8. A careful solution of $5x-3=12$ is given below. Give the rationale for each
step from the ten real number rules (A1--A4, M1--M4, LD, RD) and two additional
rules (SUB, CAN) that we discussed in lecture.  Note that some steps have two
things to identify.

\bigskip

\begin{tabular}{ll}
$5x-3=12$ & \\
$(5x-3)+3=12+3$ & \answer{CAN} \\
$(5x-3)+3=15$ & \answer{SUB} \\
$5x+(-3+3)=15$ & \answer{A2} \\
$5x+0=15$ & \answer{A4}, \answer{SUB} \\
$5x=15$ & \answer{A3}, \answer{SUB} \\
$\frac{1}{5}(5x)=\frac{1}{5}(15)$ & \answer{CAN} \\
$\frac{1}{5}(5x)=3$ & \answer{SUB} \\
$(\frac{1}{5}5)x=3$ & \answer{M2} \\
$1x=3$ & \answer{M4}, \answer{SUB} \\
$x=3$ & \answer{M3}, \answer{SUB} \\
\end{tabular}

\bigskip

A couple of you came close on this, but for the most part, the class just
didn't get this.  The idea is to take a step, look at the next step,
determine what changed, and then to match that to one of our rules.

When we ``do the same to both sides'', that is cancellation
($a+c=b+c\ \mbox{iff}\ a=b$ and {$ac=bc\ \mbox{iff}\ a=b$}.  When we simplify
an expression (on route to a solution) we are using substitution.  In fact, on
the lines that have two answers, substitution is just about always the second
answer.

Make sure that you understand how to do this.  There will be a similar question
on the end-of-Sept exam.

\bigskip

9. Solve: $\frac{3(x-5)}{4(x+1)}=\frac{9}{2}$

\bigskip

The correct way to do this is to cross multiply.  For an example, see problem
40 from the homework.  Here is a very careful solution.  Please note that the
numbers never need get big and decimals should not be used.  I know that many
of you dislike it when I show all the steps, but if you understood the
underlying steps, the solutions should be much simpler than what most of you
showed.

And, BTW - NO MORE DECIMALS, PLEASE! Keep everything in fractions unless
instructed otherwise, or the problem has decimal inputs.

\newcommand{\tag}[1]{\hspace{0.25in}\mbox{\textbf{(#1)}}}

\begin{eqnarray*}
\frac{3(x-5)}{4(x+1)} &=& \frac{9}{2} \\
2(3(x-5)) &=& 9(4(x+1))\tag{cross-multiply}\\
(2\cdot3)(x-5) &=& (9\cdot4)(x+1)\tag{M2} \\
6(x-5) &=& 36(x+1)\tag{SUB} \\
\frac{1}{6}(6(x-5)) &=& \frac{1}{6}(36(x+1))
    \tag{CAN - Do first to keep numbers small!}\\
(\frac{1}{6}6)(x-5) &=& (\frac{1}{6}36)(x+1)\tag{M2} \\
(1)(x-5) &=& (6)(x+1)\tag{M4,SUB} \\
x-5 &=& 6(x+1)\tag{M3,SUB} \\
x-5 &=& 6x+6\tag{LD} \\
-x+(x-5) &=& -x+(6x+6)\tag{CAN} \\
(-x+x)-5 &=& (-x+6x)+6\tag{A2} \\
0-5 &=& (-1+6)x+6\tag{A4,SUB,LD} \\
-5 &=& 5x+6\tag{A3,SUB} \\
-5+(-6) &=& (5x+6)+(-6)\tag{CAN} \\
-11 &=& 5x+(6+(-6))\tag{SUB,A2} \\
-11 &=& 5x+0\tag{A4,SUB} \\
-11 &=& 5x\tag{A3,SUB} \\
\frac{1}{5}(-11) &=& \frac{1}{5}(5x)\tag{CAN} \\
\frac{-11}{5} &=& (\frac{1}{5}5)x\tag{M2} \\
\frac{-11}{5} &=& (1)x\tag{M4,SUB} \\
\frac{-11}{5} &=& x\tag{M3,SUB} \\
x &=& -\frac{11}{5} \\
\end{eqnarray*}

\bigskip

10. Evaluate the left-hand-side expression in (9) at your answer in order to
prove that you have found a correct solution.

\bigskip

Notice that in the solution below, the numbers never get real big.  Normally,
do what is inside parenthesis first, don't distribute when evaluating.

\begin{eqnarray*}
\frac{3\left(\left(-\frac{11}{5}\right)-5\right)}
    {4\left(\left(-\frac{11}{5}\right)+1\right)} &=& \\
\frac{3\left(\left(-\frac{11}{5}\right)+\left(-\frac{25}{5}\right)\right)}
    {4\left(\left(-\frac{11}{5}\right)+\left(\frac{5}{5}\right)\right)} &=& \\
\frac{3\left(-\frac{36}{5}\right)}{4\left(-\frac{6}{5}\right)} &=& \\
\frac{3}{4}\left(\frac{36}{5}\right)\left(\frac{5}{6}\right) &=& \\
\frac{3}{4}(6) &=& \\
\frac{9}{2} \\
\end{eqnarray*}

\end{document}
