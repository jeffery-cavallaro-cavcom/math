\documentclass[letterpaper, 12pt]{article}

\setlength{\topmargin}{0in}
\setlength{\headheight}{0in}
\setlength{\headsep}{0in}
\setlength{\footskip}{0.5in}
\setlength{\textheight}{\paperheight}
\addtolength{\textheight}{-2in}
\addtolength{\textheight}{-\footskip}

\setlength{\oddsidemargin}{0in}
\setlength{\evensidemargin}{0in}
\setlength{\textwidth}{\paperwidth}
\addtolength{\textwidth}{-2in}

\pagestyle{empty}

\usepackage{amsfonts}

\begin{document}

\begin{center}
\bfseries
San Jos\'{e} State University \\
Fall 2015 \\
Math-8: College Algebra \\
Section 03: MW noon--1:15pm \\
Section 05: MW 4:30--5:45pm \\
\bigskip
Quiz \#2 (take-home)
\end{center}

\section*{Instructions:}

Print out this quiz, do it, and turn it in on Wednesday, Sept 9 at the start of
class. The quiz is open book and open notes with no time limit (other than the
due date), but do not use a calculator and do not work or discuss the quiz with
anyone else.

\bigskip

Good luck!

\bigskip

In problems 1--4, fill in the blanks.

\newcommand{\fillin}{\rule{1in}{1pt}}

\bigskip

1. An algebraic expression is a sum of \fillin.

\bigskip

2. An algebraic term is a product of constants (called \fillin) and \fillin.

\bigskip

3. An algebraic equation is two algebraic \fillin\ separated by \fillin.

\bigskip

4. We \fillin\ an algebraic expression, but we \fillin\ an algebraic equation.

\bigskip

5. Identify the parts of the term $-\frac{2xy^2z}{13}$:

\bigskip

Coefficient: \fillin

\bigskip

Variables: \fillin

\bigskip

6. True or false: There exists $x,y\in\mathbb{R}$ such that $x,y\ne0$ and
$xy=0$.

\vspace{0.5in}

7. Evaluate $5(3-(x-3))$ at $x=1$.

\newpage

8. A careful solution of $5x-3=12$ is given below. Give the rationale for each
step from the ten real number rules (A1--A4, M1--M4, LD, RD) and two additional
rules (SUB, CAN) that we discussed in lecture.  Note that some steps have two
things to identify.

\bigskip

\begin{tabular}{ll}
$5x-3=12$ & \\
$(5x-3)+3=12+3$ & \fillin \\
$(5x-3)+3=15$ & \fillin \\
$5x+(-3+3)=15$ & \fillin \\
$5x+0=15$ & \fillin, \fillin \\
$5x=15$ & \fillin, \fillin \\
$\frac{1}{5}(5x)=\frac{1}{5}(15)$ & \fillin \\
$\frac{1}{5}(5x)=3$ & \fillin \\
$(\frac{1}{5}5)x=3$ & \fillin \\
$1x=3$ & \fillin, \fillin \\
$x=3$ & \fillin, \fillin \\
\end{tabular}

\bigskip

9. Solve: $\frac{3(x-5)}{4(x+1)}=\frac{9}{2}$

\vspace{3in}

10. Evaluate the left-hand-side expression in (9) at your answer in order to
prove that you have found a correct solution.

\end{document}
