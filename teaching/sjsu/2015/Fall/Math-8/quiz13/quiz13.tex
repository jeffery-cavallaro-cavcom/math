\documentclass[letterpaper, 12pt]{article}

\setlength{\topmargin}{0in}
\setlength{\headheight}{0in}
\setlength{\headsep}{0in}
\setlength{\footskip}{0.5in}
\setlength{\textheight}{\paperheight}
\addtolength{\textheight}{-2in}
\addtolength{\textheight}{-\footskip}

\setlength{\oddsidemargin}{0in}
\setlength{\evensidemargin}{0in}
\setlength{\textwidth}{\paperwidth}
\addtolength{\textwidth}{-2in}

\pagestyle{empty}

\usepackage{amsfonts}

\begin{document}

\begin{center}
\bfseries
San Jos\'{e} State University \\
Fall 2015 \\
Math-8: College Algebra \\
Section 03: MW noon--1:15pm \\
Section 05: MW 4:30--5:45pm \\
\bigskip
Quiz \#13 (Take-home)
\end{center}

\bigskip

When a function $f(x)$ has an inverse, we denote the inverse function as
$f^{-1}(x)$. Note that this should note be confused with $\frac{1}{f(x)}$,
which we would denote by $[f(x)]^{-1}$.

\bigskip

Consider the function $f(x)=x^2-4x+3$.

\bigskip

1. Put $f(x)$ in standard form and sketch the graph.

\vspace{4in}

2. What is the domain of $f(x)$?

\vspace{1in}

3. By looking at the graph, does $f(x)$ have an inverse? Why or why not?

\vspace{1in}

4. How can the domain of $f(x)$ be adjusted such that it does have an inverse?

\vspace{1in}

5. Using the adjustment you found in (4), find $f^{-1}(x)$.

\end{document}
