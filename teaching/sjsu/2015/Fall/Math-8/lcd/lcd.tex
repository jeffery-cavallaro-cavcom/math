\documentclass[letterpaper]{article}

\setlength{\topmargin}{0in}
\setlength{\headheight}{0in}
\setlength{\headsep}{0in}
\setlength{\footskip}{0.5in}
\setlength{\textheight}{\paperheight}
\addtolength{\textheight}{-2in}
\addtolength{\textheight}{-\footskip}

\setlength{\oddsidemargin}{0in}
\setlength{\evensidemargin}{0in}
\setlength{\textwidth}{\paperwidth}
\addtolength{\textwidth}{-2in}

\pagestyle{empty}

\begin{document}

\begin{center}
\textbf{The Lowest Common Denominator Method}
\end{center}

\bigskip

When we want to add or subtract two fractions with different denominators, we
need to modify one or more of the fractions so that we have a common
denominator.  We will then be able to use the rule:

\[\frac{a}{c}\pm\frac{b}{c}=\frac{a\pm b}{c} \]

We could just multiple the denominators by each other in order to achieve such
a common denominator, but that typically leads to large, error-prone numbers---
especially when there are more than two fractions in our expression. Instead,
we want to pick the lowest common multiple (LCM) of all the denominators.

Let's take the example problem:

\[\frac{3}{10}+\frac{7}{12}\-\frac{3}{5}\]

1. Perform a prime factorization of each denominator.

\bigskip

\begin{tabular}{l}
$10=2\cdot5$ \\
$12=2\cdot6=2\cdot2\cdot3=2^2\cdot3$ \\
$5=5$ (already prime!) \\
\end{tabular}

\bigskip

2. Take the highest power of each prime across all the factorizations.

\bigskip

From $2^1$, $2^2$ get $2^2$.

From $3^1$ we get just $3$.

From $5^1$ and $5^1$ we get $5$.

\bigskip

So the LCM of the denominators, i.e., the LCD, is $2^2\cdot3\cdot5=60$.

\bigskip

3. Multiply each fraction above and below the achieve the common denominator
and then perform the operation.

\begin{eqnarray*}
\frac{3\cdot6}{10\cdot6}+\frac{7\cdot5}{12\cdot5}-
    \frac{3\cdot12}{5\cdot12} &=&
    \frac{18}{60}+\frac{35}{60}-\frac{36}{60} \\
&=& \frac{18+35-26}{60} \\
&=& \frac{27}{60} \\
\end{eqnarray*}

4. Use prime factorizations on the numerator and denominator to simply the
result.

\bigskip

\begin{tabular}{l}
$27=3\cdot9=3\cdot3\cdot3=3^3$ \\
$60=2^2\cdot3\cdot5$.
\end{tabular}

\bigskip

\begin{eqnarray*}
\frac{27}{60} &=& \frac{3^3}{2^2\cdot3\cdot5} \\
              &=& \frac{3^2}{2^2\cdot5} \\
              &=& \frac{9}{4\cdot5} \\
              &=& \frac{9}{20} \\
\end{eqnarray*}

\end{document}
