\documentclass[letterpaper, 12pt]{article}

\setlength{\topmargin}{0in}
\setlength{\headheight}{0in}
\setlength{\headsep}{0in}
\setlength{\footskip}{0.5in}
\setlength{\textheight}{\paperheight}
\addtolength{\textheight}{-2in}
\addtolength{\textheight}{-\footskip}

\setlength{\oddsidemargin}{0in}
\setlength{\evensidemargin}{0in}
\setlength{\textwidth}{\paperwidth}
\addtolength{\textwidth}{-2in}

\pagestyle{empty}

\usepackage{amsfonts}

\begin{document}

\begin{center}
\bfseries
San Jos\'{e} State University \\
Fall 2015 \\
Math-8: College Algebra \\
Section 03: MW noon--1:15pm \\
Section 05: MW 4:30--5:45pm \\
\bigskip
Exam 3
\end{center}

\vspace{0.5in}

Name: \rule{5in}{1pt}

\vspace{0.5in}

For problems 1-4, define the following:

\begin{eqnarray*}
f(x) &=& x^2-8x+12 \\
g(x) &=& \sqrt[4]{x} \\
h(x) &=& \sqrt[9]{x+1} \\
s(x) &=& \frac{h}{(g\circ f)}(x) \\
t(x) &=& \frac{f(x)}{f(x)} \\
\end{eqnarray*}

1. Determine $s(x)$

\vspace{2in}

2. Evaluate $s(0)$. Leave the answer in exact form.

\newpage

3. What is the domain of $s(x)$?

\vspace{4in}

4. Sketch the graph of $t(x)$. Be sure to label all important values.

\vspace{2in}

5. (True or false): $(f\circ g)$=$(g\circ f)$.

\vspace{0.5in}

6. Let $h(x)=\frac{2}{x+1}+x+1$. Find suitable $f(x)$ and $g(x)$ such that
$h(x)=(g\circ f)(x)$.

\newpage

7. Divide $x^4-x^2+2$ by $x^2+1$. State the answer in division algorithm form.

\vspace{4in}

8. Find all points of intersection for:

\begin{eqnarray*}
x^2+y-6x+7 &=& 0 \\
x-y &=& 3 \\
\end{eqnarray*}

\newpage

9. Find all solutions (if any):

\begin{eqnarray*}
x+4z &=& 1 \\
x+y+10z &=& 10 \\
2x-y+2z &=& -5 \\
\end{eqnarray*}

\newpage

10. Determine all x and y intercepts, use the leading coefficient test to
determine behavior as $x\to\pm\infty$, and sketch the following polynomial.
Be sure to label all important points and explain how you arrived at the
sign for each interval between the x intercepts.

\[f(x)=x^4+3x^3-9x^2-23x-12\]

\end{document}
