\documentclass[letterpaper, 12pt]{article}

\setlength{\topmargin}{0in}
\setlength{\headheight}{0in}
\setlength{\headsep}{0in}
\setlength{\footskip}{0.5in}
\setlength{\textheight}{\paperheight}
\addtolength{\textheight}{-2in}
\addtolength{\textheight}{-\footskip}

\setlength{\oddsidemargin}{0in}
\setlength{\evensidemargin}{0in}
\setlength{\textwidth}{\paperwidth}
\addtolength{\textwidth}{-2in}

\pagestyle{empty}

\usepackage{amsfonts}

\begin{document}

\begin{center}
\bfseries
San Jos\'{e} State University \\
Fall 2015 \\
Math-8: College Algebra \\
Section 03: MW noon--1:15pm \\
Section 05: MW 4:30--5:45pm \\
\bigskip
Quiz \#8
\end{center}

\bigskip

Closed book and notes, and no calculators. Show all work for full credit.

\bigskip

\newcommand{\answer}[1]{\textbf{\underline{#1}}}

1. A function maps a set of independent values called the \answer{domain}\ to a
set of dependent values called the \answer{codomain}. The subset of the
codomain actually used is called the \answer{range}.

\bigskip

2. What are the two requirements for values in the domain of a function?

\begin{enumerate}
\item{All values must be used.}
\item{Each value must map to exactly one value in the codomain/range.}
\end{enumerate}

3. Solve for $x$, giving your answer in interval notation.

\bigskip

First, we need to get the inequality in $|x-a|<b$ form:

\begin{eqnarray*}
2|5-3x|+7 &<& 21 \\
2|5-3x] &<& 14 \\
|5-3x| &<& 7 \\
\end{eqnarray*}

Now, since this is a less-than inequality, we use a 3-way expression:

\begin{eqnarray*}
-7 < 5-3x < 7 \\
-12 < -3x < 2 \\
4 > x > -\frac{2}{3} \\
-\frac{2}{3} < x < 4 \\
\end{eqnarray*}

Note that we turn around the inequality when we divide by $-3$. Thus, the
final solution in interval notation is: $(-\frac{2}{3},4)$.

\bigskip

4. You start a side-business manufacturing widgets. The variable costs are \$2
per widget. The fixed costs are \$500 per month. You would like to keep your
monthly costs between \$1000 and \$2000 per month. What are the minimum and
maximum number of widgets that you can make per month?

First, we build a cost function using the variable and fixed costs:

\[C(x)=2x+500\]

Now, we build a 3-way inequality and solve:

\begin{eqnarray*}
1000 \le C(x) \le 2000 \\
1000 \le 2x+500 \le 2000 \\
500 \le 2x \le 1500 \\
250 \le x \le 750 \\
\end{eqnarray*}

Thus, the minimum is 250 widgets and the maximum is 750 widgets.

\bigskip

5. Determine the domain of the following function:

\[f(x)=\frac{x-5}{\sqrt{x^2-9}}\]

The $x-5$ in the numerator is not under the square root and thus has no affect
on the domain. Remember, the numerator can be 0. But, the denominator may not
be 0 and the radicand under the square root must be positive. Thus, we have:

\begin{eqnarray*}
x^2-9 &>& 0 \\
(x+3)(x-3) &>& 0 \\
\end{eqnarray*}

which results in the critical points: $\pm3$. Picking test values, we can
build a sign table as follows:

\bigskip

\begin{tabular}{|c|c|c|c|}
\hline
 & $(x+3)$ & $(x-3)$ & \\
\hline
-4 & - & - & + \\
\hline
0 & + & - & - \\
\hline
4 & + & + & + \\
\hline
\end{tabular}

\bigskip

Thus, the final domain is: $(-\infty,-3)\cup(3,\infty)$.

\end{document}
