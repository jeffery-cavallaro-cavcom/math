\documentclass[letterpaper, 12pt]{article}

\setlength{\topmargin}{0in}
\setlength{\headheight}{0in}
\setlength{\headsep}{0in}
\setlength{\footskip}{0.5in}
\setlength{\textheight}{\paperheight}
\addtolength{\textheight}{-2in}
\addtolength{\textheight}{-\footskip}

\setlength{\oddsidemargin}{0in}
\setlength{\evensidemargin}{0in}
\setlength{\textwidth}{\paperwidth}
\addtolength{\textwidth}{-2in}

\pagestyle{empty}

\usepackage{amsfonts}

\begin{document}

\begin{center}
\bfseries
San Jos\'{e} State University \\
Fall 2015 \\
Math-8: College Algebra \\
Section 03: MW noon--1:15pm \\
Section 05: MW 4:30--5:45pm \\
\bigskip
Quiz \#15 (Solutions)
\end{center}

\bigskip

1. Assume that we know the following:

\begin{eqnarray*}
\log_{b}2 &=& 0.43068 \\
\log_{b}7 &=& 1.20906 \\
\end{eqnarray*}

a. Without determining $b$, what is $\log_{b}28$?

\begin{eqnarray*}
\log_{b}28 &=& \log_{b}(2^2\cdot7) \\
    &=& \log_{b}(2^2)+\log_{b}7 \\
    &=& 2\log_{b}2+\log_{b}7 \\
    &=& 2(0.43068)+1.20906 \\
    &=& 2.07042 \\
\end{eqnarray*}

b. What is $b$?

\begin{eqnarray*}
\log_{b}2 &=& 0.43068 \\
\frac{\ln 2}{\ln b} &=& 0.43068\ \mbox{(change of base formula)} \\
\frac{\ln b}{\ln 2} &=& \frac{1}{0.43068} \\
\ln b &=& \frac{\ln 2}{0.43068} \\
e^{\ln b} &=& e^{\frac{\ln 2}{0.43068}} \\
b &=& 5 \\
\end{eqnarray*}

2. Solve for $x$:

\[\log_2(2x)+\log_2(x-3)=3\]

\begin{eqnarray*}
log_2[2x(x-3)] &=& 3 \\
2^{log_2[2x(x-3)]} &=& 2^3 \\
2x(x-3) &=& 8 \\
2x^2-6x &=& 8 \\
2x^2-6x-8 &=& 0 \\
(x+1)(x-4) &=& 0 \\
x &=& -1,4 \\
\end{eqnarray*}

Before we can conclude that both solutions are correct, we must make sure that
we have not violated any domain restrictions.  Indeed, plugging in $x=-1$
results in a $\log$ or a negative value, and thus must be discarded.  The
other solution, $x=4$, is OK.  Therefore, the final answer is $x=4$.

\bigskip

3. Derive the change-of-base formula:

\[\log_{b}x=\frac{log_a x}{log_a b}\]

\begin{eqnarray*}
y &=& \log_{b}x \\
b^y &=& x \\
\log_{a}b^y &=& \log_{a}x \\
y\log_{a}b &=& \log_{a}x \\
y &=& \frac{\log_{a}x}{\log_{a}b} \\
\end{eqnarray*}

4. In class we talked about carbon dating and noted that the equation typically
used:

\[y=\frac{1}{10^{12}}e^{-\frac{t}{8223}}\]

is slightly different than the normal half-life-based equation:

\[y=\frac{1}{10^{12}}\left(\frac{1}{2}\right)^{\frac{t}{5700}}\]

where 5700 years is the half-life of $C_{14}$. Show that these two equations are
equivalent.

\bigskip

Let's equate the two expressions, replace 8223 with $x$, and then solve for
$x$:

\begin{eqnarray*}
\frac{1}{10^{12}}\left(\frac{1}{2}\right)^{\frac{t}{5700}} &=&
    \frac{1}{10^{12}}e^{-\frac{t}{x}} \\
\left(\frac{1}{2}\right)^{\frac{t}{5700}} &=& e^{-\frac{t}{x}} \\
\ln\left(\frac{1}{2}\right)^{\frac{t}{5700}} &=& \ln\left(e^{-\frac{t}{x}}\right) \\
\frac{t}{5700}\ln\left(\frac{1}{2}\right) &=& -\frac{t}{x} \\
\frac{1}{5700}\ln\left(\frac{1}{2}\right) &=& -\frac{1}{x} \\
\frac{5700}{\ln\left(\frac{1}{2}\right)} &=& -x \\
x &=& -\frac{5700}{\ln\left(\frac{1}{2}\right)} \\
x &=& 8223 \\
\end{eqnarray*}

\end{document}
