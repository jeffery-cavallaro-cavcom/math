\documentclass[letterpaper, 12pt]{article}

\setlength{\topmargin}{0in}
\setlength{\headheight}{0in}
\setlength{\headsep}{0in}
\setlength{\footskip}{0.5in}
\setlength{\textheight}{\paperheight}
\addtolength{\textheight}{-2in}
\addtolength{\textheight}{-\footskip}

\setlength{\oddsidemargin}{0in}
\setlength{\evensidemargin}{0in}
\setlength{\textwidth}{\paperwidth}
\addtolength{\textwidth}{-2in}

\pagestyle{empty}

\usepackage{amsfonts}

\begin{document}

\begin{center}
\bfseries
San Jos\'{e} State University \\
Fall 2015 \\
Math-8: College Algebra \\
Section 03: MW noon--1:15pm \\
Section 05: MW 4:30--5:45pm \\
\bigskip
Quiz \#14 (Take-home)
\end{center}

\bigskip

1. We saw in class that $a^x$ for $x\in\mathbb{R}$ is the value that $a^x$
approaches as we get closer and closer to $x$ with a sequence of rational
numbers. This works for the base as well. We were also introduced to the
special base $e=2.71828\ldots$, known as Euler's number. Calculate $e^2$ on
your calculator and show how $2^2$, $2.7^2$, $2.71^2$, \ldots approaches
$e^2$. Look at 6 such terms.

\bigskip

2. Sketch the graph: $y=e^{-x+2}+1$. (Hint: factor out the negative in the
exponent first).

\bigskip

3. Determine the amount of money in a savings account after 5 years at a yearly
interest rate of 2\% assuming that the original principle is \$10000 and the
compounding rate is: a) monthly, and b) continuous.

\bigskip

4. The half-life of Uranium-235 is about 700 million years. What percent of a
sample is left after only 100 million years?

\bigskip

5. Evaluate:

\bigskip

a. $\log_2 256$

\bigskip

b. $\log_{10} 10000$

\bigskip

c. $\ln 5$

\bigskip

6. Solve:

\bigskip

a. $\log_3(x+1)=\log_3(13)$.

\bigskip

b. $10\log_7(7^{x-2})=5^{\log_5(2x-1)}$

\bigskip

7. Determine the domain for $f(x)=\frac{\log_3(x^2+7x+12)}{\sqrt{x-1}}$

\end{document}
