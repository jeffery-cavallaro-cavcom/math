\documentclass[letterpaper]{article}

\setlength{\topmargin}{0in}
\setlength{\headheight}{0in}
\setlength{\headsep}{0in}
\setlength{\footskip}{0.5in}
\setlength{\textheight}{\paperheight}
\addtolength{\textheight}{-2in}
\addtolength{\textheight}{-\footskip}

\setlength{\oddsidemargin}{0in}
\setlength{\evensidemargin}{0in}
\setlength{\textwidth}{\paperwidth}
\addtolength{\textwidth}{-2in}

\pagestyle{empty}

\begin{document}

\begin{center}
\textbf{Interest Problems}
\end{center}

\bigskip

\section*{What is Interest?}

Interest is money that is paid on the outstanding (unpaid) portion of a loan
over time. For every loan there is a \emph{borrower}, the person who borrows
the money and is responsible for paying the money back with interest, and the
\emph{lender}, the person that loans the money and receives the payments and
interest.

\section*{Important Values}

Every loan has the following values:

\begin{description}
\item{\textbf{Principal:}} The original loan amount, as well as any unpaid
  balance over the course of the loan.
\item{\textbf{Rate}} A percentage of the principal that is paid by the borrower
  to the lender for the privilege of using the loaned funds.  Rates are usually
  stated as a particular percentage over time - e.g., 5\% per year.  Loans can
  have either a \emph{fixed} interest rate, one that stays the same over the
  life of the loan, or a \emph{variable} interest rate, one that can adjust
  over time.  For all of our problems, we will deal with fixed interest rates
  only.
\item{\textbf{Time}} The amount of time that there is unpaid principal.  Some
  loans have a set time period - all of the principal must be repaid within
  that period.  Other loans are open-ended and allow the borrow to make
  payments as long as there is unpaid principal.
\end{description}

\section*{Loan Types}

The first part of any problem is to determine who is the borrower
and who is the lender.  Here are a couple of common scenarios:

\begin{description}

\item{\textbf{Savings Account}} Although a savings account may not seem like
  loan, it really is one: the saver is lending money to the bank and the bank
  in turn pays interest for the privilege of using the money for its own
  investments.  This is a special type of loan, since the lender (saver) can
  demand full repayment (i.e., a total withdrawal) at any time.

\item{\textbf{Mortgage}} Since house prices are so high, people tend to need
  help purchasing a home. The type of loan used to buy a house is called a
  mortgage. In this case, the lender is the bank and the borrower is the home
  buyer. Mortgages tend to have a monthly repayment plan.  Sometimes, this
  monthly payment plan doesn't cover the full payment of the loan, leaving a
  large last payment, typically called a \emph{balloon} payment.  One special
  feature of a mortgage is that they can be renegotiated so that a borrower can
  add more principle to the loan for things like home improvements (e.g., a
  bathroom or kitchen remodel).

\item{\textbf{Credit Card}} Everyone is probably familiar with credit cards;
  they have become such an integral part of our everyday lives. With a credit
  card, the bank is the lender and the credit card holder is the borrower.
  Every time the card holder purchases something with the card, new principal
  is added to the account. Every time the card hold makes a payment to the
  bank, the principal is reduced. Payments are usually monthly and interest is
  paid on any unpaid amount past the statement due date.

\item{\textbf{Home Equity Line}} Some borrowers take out a second mortgage (on
  top of the first) that acts like a credit card and is used primarily for home
  improvements. Such loans are called home equity lines, and are a simpler
  alternative to first mortgage renegotiations.

\item{\textbf{Corporate and Government Bonds}} Corporations and governments
  issue bonds to raise money for expansion or special projects.  Government
  bonds can be issued by cities, counties, states, the federal government, and
  foreign governments. The corporation or government is the borrower and the
  bond buyer is the lender. There are all types of bonds, but a popular type is
  one that pays interest only over the life of the loan, followed by a balloon
  payment of the principal at the end of the loan.

\end{description}

\section*{Interest Models}

There are two types of interest: simple and compounded.

\subsection*{Simple Interest}

Simple interest is paid on the \emph{principle only} and follows the equation:

\[A=P\left(1+rt\right)\]

where $r$ is the annual interest rate, $t$ is the number of years, $P$ is the
outstanding principal, and $A$ is the total amount (value) of the loan.

\subsection*{Compounded Interest}

Compounded interest is paid on the \emph{principal} plus all unpaid, previously
accrued (earned) \emph{interest} and follows the equation:

\[A=P\left(1+\frac{r}{n}\right)^{nt}\]

The value $n$ is the number of times interest is paid over the period of the
interest rate. Thus, the $\frac{r}{n}$ represents the interest rate per
compoundin period and the exponent $nt$ represents the number of compounding
periods.  Since the period if most interest rates in one year, examples
of compounding periods are:

\bigskip

\begin{center}
\begin{tabular}{|c|c|}
\hline
daily & 365 \\
\hline
weekly & 52 \\
\hline
monthly & 12 \\
\hline
quarterly & 4 \\
\hline
semi-annually & 2 \\
\hline
yearly & 1 \\
\hline
\end{tabular}
\end{center}

\bigskip

\section*{Repayments}

Most loans involve some type of activity over time, such deposits and
withdrawals to a savings account, monthly mortgage payments, and monthly
purchases and payments on a credit card. For example, assume that you have a
savings account to which you make monthly deposits.

\end{document}
