\documentclass[letterpaper, 12pt]{article}

\setlength{\topmargin}{0in}
\setlength{\headheight}{0in}
\setlength{\headsep}{0in}
\setlength{\footskip}{0.5in}
\setlength{\textheight}{\paperheight}
\addtolength{\textheight}{-2in}
\addtolength{\textheight}{-\footskip}

\setlength{\oddsidemargin}{0in}
\setlength{\evensidemargin}{0in}
\setlength{\textwidth}{\paperwidth}
\addtolength{\textwidth}{-2in}

\pagestyle{empty}

\usepackage{amsfonts}

\newcommand{\fillin}{\rule{1in}{1pt}}

\begin{document}

\begin{center}
\bfseries
San Jos\'{e} State University \\
Fall 2015 \\
Math-8: College Algebra \\
Section 03: MW noon--1:15pm \\
Section 05: MW 4:30--5:45pm \\
\bigskip
Exam 1
\end{center}

\bigskip

\underline{Helpful Stuff}

\bigskip

\begin{tabular}{cc}
$A=P\left(1+rt\right)$ & $s=s_0+v_0t+\frac{1}{2}at^2$ \\
& \\
$A=P\left(1+\frac{r}{n}\right)^{nt}$ & $g=32 ft/s^2\ \mbox{and}\ g=9.8 m/s^2$ \\
& \\
$x=\frac{-b\pm\sqrt{b^2-4ac}}{2a}$ & \\
\end{tabular}

\bigskip

\begin{tabular}{|c|l|}
\hline
A1 & Commutative Addition \\
\hline
A2 & Associative Addition \\
\hline
A3 & Additive Identity (0) \\
\hline
A4 & Additive Inverse (-a) \\
\hline
M1 & Commutative Multiplication \\
\hline
M2 & Associative Multiplication \\
\hline
M3 & Multiplicative Identity (1) \\
\hline
M4 & Multiplicative Inverse (1/a) \\
\hline
RD & Right Distributive \\
\hline
LD & Left Distributive \\
\hline
CAN & Cancellation \\
\hline
SUB & Substitution \\
\hline
\end{tabular}

\vspace{1in}

Name: \rule{3in}{1pt}

\newpage

1. A careful solution of $4(x+2)=11$ is given below. Give the rationale for
each step from the ten real number rules (A1--A4, M1--M4, LD, RD) and two
additional rules (SUB, CAN).  Note that some steps have two things to identify.

\bigskip

\begin{tabular}{ll}
$4(x+2)=11$ & \\
$4x+8=11$ & \fillin, \fillin \\
$(4x+8)-8=11-8$ & \fillin \\
$(4x+8)-8=3$ & \fillin \\
$4x+(8-8)=3$ & \fillin \\
$4x+0=3$ & \fillin, \fillin \\
$4x=3$ & \fillin, \fillin \\
$\frac{1}{4}(4x)=\frac{1}{4}(3)$ & \fillin \\
$\frac{1}{4}(4x)=\frac{3}{4}$ & \fillin \\
$(\frac{1}{4}4)x=\frac{3}{4}$ & \fillin \\
$1x=\frac{3}{4}$ & \fillin, \fillin \\
$x=\frac{3}{4}$ & \fillin, \fillin \\
\end{tabular}

\vspace{1in}

2. Solve for x, but do not use the quadratric formula:

\[\frac{\left[2x(x+5)^3\right]^2}{2x^2(x+5)^7}=-\frac{x}{3}\]

\newpage

3. Simplify:

\[\frac{\sqrt{4x\sqrt[3]{(x+1)^4}}}{\sqrt[4]{x}(x+1)}\]

\vspace{4in}

4. Expand: $(3xy-4z)^2$

\newpage

5. Simplify: $\sqrt{200x^2y}-\sqrt{50x^2y}$

\vspace{3in}

6. Solve by completing the square. Do not use the quadratic formula.

\[2x^2+6x-10=0\]

\newpage

7. A man stands on a 100 ft cliff and throws a rock up in the air with a speed
of 12 ft/s. How long does it take for the rock hit the ground? (Hint: You
should set up your coordinate system so that 0 is the ground and up is the
positive direction. Use the constant acceleration formula given on the first
page and pick the appropriate value of g to plug in for a. Make sure that you
use the correct sign for g.)

\vspace{4in}

8. A woman is standing at the bottom of the cliff in problem (7). She throws a
rock upwards just hard enough so that on its way up it passes the man on the
cliff with a speed of 12 ft/s. How long does it take from the time that the
rock leaves the woman's hand to the time that it hits the ground? (Hint: all
the values that you need to solve this were calculated in problem 7).

\newpage

9. You ask your friend the photographer to take your picture and give you a
nice, glossy 8x12 square inch copy mounted in a nice frame. So that the frame
doesn't damage the photo, you ask your friend to include a border around the
picture that is the same width on all sides of the picture. The total area of
the frame is 117 square includes.  How wide is the border? (Hint: this is the
same as the pool problem on the quiz 4 retake). Please show all work - do not
guess!

\vspace{4in}

10. You open a savings account that pays 3\% annual interest, compounded
monthly, with an initial deposit of \$1000. You are a good saver, so you
auto-deposit \$500 from your paycheck every month. In the third month, you
need to buy a new car, so you withdraw \$750 to help with the down-payment.
What is your account balance at the start of the fifth month (right
after the fourth deposit)?

\end{document}
