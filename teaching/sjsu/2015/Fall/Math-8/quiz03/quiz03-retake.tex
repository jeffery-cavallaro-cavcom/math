\documentclass[letterpaper, 12pt]{article}

\setlength{\topmargin}{0in}
\setlength{\headheight}{0in}
\setlength{\headsep}{0in}
\setlength{\footskip}{0.5in}
\setlength{\textheight}{\paperheight}
\addtolength{\textheight}{-2in}
\addtolength{\textheight}{-\footskip}

\setlength{\oddsidemargin}{0in}
\setlength{\evensidemargin}{0in}
\setlength{\textwidth}{\paperwidth}
\addtolength{\textwidth}{-2in}

\pagestyle{empty}

\usepackage{amsfonts}

\begin{document}

\begin{center}
\bfseries
San Jos\'{e} State University \\
Fall 2015 \\
Math-8: College Algebra \\
Section 03: MW noon--1:15pm \\
Section 05: MW 4:30--5:45pm \\
\bigskip
Quiz \#3 Retake
\end{center}

\bigskip

\newcommand{\fillin}{\rule{1in}{1pt}}

\underline{Helpful Formulas}

\[A=P\left(1+rt\right)\hspace{1in}A=P\left(1+\frac{r}{n}\right)^{nt}\]

\bigskip

1. Use the lowest common denominator method to perform the following addition.
For full credit, find the LCD using the prime factorization method, perform
the operation, then simply using prime factorization (if necessary).

\[\frac{5}{51}+\frac{7}{34}-\frac{2}{9}\]

\vspace{3in}

2. For simple interest, interest is paid on the \fillin only.  For compound
interest, interest is paid on the \fillin and \fillin.

\newpage

3. Jill puts \$1,000 into savings account that has a yearly interest rate of
2\% and compounds monthly.  What is the value of the account after 5 years?

\vspace{3in}

4. Solve for x.

\[\frac{\left[3x\left(x-2\right)\left(x+1\right)^2\right]^3}{x^3(x-2)^2(x+1)^7}=-\frac{1}{2}\]

\end{document}
