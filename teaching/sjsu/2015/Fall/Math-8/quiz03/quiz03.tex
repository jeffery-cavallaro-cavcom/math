\documentclass[letterpaper, 12pt]{article}

\setlength{\topmargin}{0in}
\setlength{\headheight}{0in}
\setlength{\headsep}{0in}
\setlength{\footskip}{0.5in}
\setlength{\textheight}{\paperheight}
\addtolength{\textheight}{-2in}
\addtolength{\textheight}{-\footskip}

\setlength{\oddsidemargin}{0in}
\setlength{\evensidemargin}{0in}
\setlength{\textwidth}{\paperwidth}
\addtolength{\textwidth}{-2in}

\pagestyle{empty}

\usepackage{amsfonts}

\begin{document}

\begin{center}
\bfseries
San Jos\'{e} State University \\
Fall 2015 \\
Math-8: College Algebra \\
Section 03: MW noon--1:15pm \\
Section 05: MW 4:30--5:45pm \\
\bigskip
Quiz \#3
\end{center}

\bigskip

\newcommand{\fillin}{\rule{1in}{1pt}}

\underline{Helpful Formulas}

\[A=P\left(1+rt\right)\hspace{1in}A=P\left(1+\frac{r}{n}\right)^{nt}\]

\bigskip

1. Use the lowest common denominator method to perform the following addition.
For full credit, make sure that you find the LCD, perform the operation, and
then simplify the result (if necessary) using prime factorization.

\[\frac{5}{12}+\frac{7}{18}-\frac{11}{24}\]

\vspace{3in}

2. For simple interest, interest is paid on the \fillin only.  For compound
interest, interest is paid on the \fillin and \fillin.

\newpage

3. Jack puts \$5,000 into a savings account that has a yearly interest rate of
2\% and compounds monthly.  What is the value of the account after 10 years?

\vspace{3in}

4. Solve for x.

\[\frac{\left[5x\left(x+1\right)^2\right]^2}{x^2(x+1)^5}=-\frac{3}{x}\]

\end{document}
