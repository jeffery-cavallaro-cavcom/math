\documentclass[letterpaper, 12pt]{article}

\setlength{\topmargin}{0in}
\setlength{\headheight}{0in}
\setlength{\headsep}{0in}
\setlength{\footskip}{0.5in}
\setlength{\textheight}{\paperheight}
\addtolength{\textheight}{-2in}
\addtolength{\textheight}{-\footskip}

\setlength{\oddsidemargin}{0in}
\setlength{\evensidemargin}{0in}
\setlength{\textwidth}{\paperwidth}
\addtolength{\textwidth}{-2in}

\pagestyle{empty}

\usepackage{amsfonts}

\begin{document}

\begin{center}
\bfseries
San Jos\'{e} State University \\
Fall 2015 \\
Math-8: College Algebra \\
Section 03: MW noon--1:15pm \\
Section 05: MW 4:30--5:45pm \\
\bigskip
Quiz \#3 Solutions
\end{center}

\bigskip

\newcommand{\answer}[1]{\textbf{\underline{#1}}}

\underline{Helpful Formulas}

\[A=P\left(1+rt\right)\hspace{1in}A=P\left(1+\frac{r}{n}\right)^{nt}\]

\bigskip

1. Use the lowest common denominator method to perform the following addition.
For full credit, make sure that you find the LCD, perform the operation, and
then simplify the result (if necessary) using prime factorization.

\[\frac{5}{12}+\frac{7}{18}-\frac{11}{24}\]

\bigskip

We start by doing prime factorizations of the denominators:

\bigskip

$12=2^2\cdot3$

$18=2\cdot3^2$

$24=2^3\cdot3$

\bigskip

Next, we make a list of all the primes that occur in all the factorizations:
2 and 3.  We then take the highest exponent occurring in any factorization for
each prime: $2^3$ (from 24) and $3^2$ (from 18).  Finally, we multiply these
together to find the LCD:

\[LCD=2^3\cdot3^2=8\cdot9=72\]

We now adjust our fractions to the new denominator and perform the operation:

\begin{eqnarray*}
\frac{5\cdot6}{12\cdot6}+\frac{7\cdot4}{18\cdot4}-\frac{11\cdot3}{24\cdot3} &=&
\frac{30}{72}+\frac{28}{72}-\frac{33}{72} \\
&=& \frac{30+28-33}{72} \\
&=& \frac{25}{72} \\
\end{eqnarray*}

Finally, we check to see if we can reduce the answer.  Again, looking at the
prime factorizations:

\bigskip

$25=5^2$

$72=2^3\cdot3^2$

\bigskip

Since there are no common prime factors, the answer is irreducible.

\bigskip

2. For simple interest, interest is paid on the \answer{principal} only.  For
compound interest, interest is paid on the \answer{principle} and
\answer{interest}.

3. Jack puts \$5,000 into a savings account that has a yearly interest rate of
2\% and compounds monthly.  What is the value of the account after 10 years?

Use the compound interest formula above with:

\bigskip

$P=\$5000$

$r=2\%=0.02$

$n=12$

$t=10$

\bigskip

Note that for monthly compounding on a yearly interest rate of 2\%, $n=12$ for a
monthly interest rate of $\frac{0.02}{12}$.

\[A=\$5000\left(1+\frac{0.02}{12}\right)^{10\cdot12}=\$6106\]


4. Solve for x.

\[\frac{\left[5x\left(x+1\right)^2\right]^2}{x^2(x+1)^5}=-\frac{3}{x}\]

We start by resolving the numerator.  For this, we use our rule:
$(ab)^n=a^nb^n$.  Note that we have 3 factors in the numerator, so we want to
make sure and apply the exponent to all three of them.  Note that for the last
factor we use the rule $(a^m)^n=a^{mn}$:

\[\frac{5^2x^2\left[\left(x+1\right)^2\right]^2}{x^2(x+1)^5}=-\frac{3}{x}\]

\[\frac{5^2x^2\left(x+1\right)^4}{x^2(x+1)^5}=-\frac{3}{x}\]

Now, we group the factors in the numerator and denominator so that we can use
our rule $\frac{a^n}{a^m}=a^{n-m}$:

\[25\left[\frac{x^2}{x^2}\right]\left[\frac{(x+1)^4}{(x+1)^5}\right]=
-\frac{3}{x}\]

Now, apply the rule:

\begin{eqnarray*}
25x^{2-2}(x+1)^{4-5} &=& -\frac{3}{x} \\
25x^0(x+1)^{-1} &=& -\frac{3}{x} \\
\frac{25(1)}{x+1} &=& -\frac{3}{x} \\
\frac{25}{x+1} &=& -\frac{3}{x} \\
\end{eqnarray*}

Now, cross multiply and solve, being careful with the negative sign:

\begin{eqnarray*}
25x &=& -3(x+1) \\
25x &=& -3x-3 \\
28x &=& -3 \\
x &=& -\frac{3}{28} \\
\end{eqnarray*}

\end{document}
