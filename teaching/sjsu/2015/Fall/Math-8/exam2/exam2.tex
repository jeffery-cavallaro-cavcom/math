\documentclass[letterpaper, 12pt]{article}

\setlength{\topmargin}{0in}
\setlength{\headheight}{0in}
\setlength{\headsep}{0in}
\setlength{\footskip}{0.5in}
\setlength{\textheight}{\paperheight}
\addtolength{\textheight}{-2in}
\addtolength{\textheight}{-\footskip}

\setlength{\oddsidemargin}{0in}
\setlength{\evensidemargin}{0in}
\setlength{\textwidth}{\paperwidth}
\addtolength{\textwidth}{-2in}

\pagestyle{empty}

\usepackage{amsfonts}

\begin{document}

\begin{center}
\bfseries
San Jos\'{e} State University \\
Fall 2015 \\
Math-8: College Algebra \\
Section 03: MW noon--1:15pm \\
Section 05: MW 4:30--5:45pm \\
\bigskip
Exam 2
\end{center}

\vspace{0.5in}

\begin{tabular}{lll}
$d=\sqrt{(x_2-x_1)^2+(y_2-y_1)^2}$ &
    $\left(\frac{x_1+x_2}{2},\frac{y_1+y_2}{2}\right)$ & \\
$m=\frac{y_2-y_1}{x_2-x_1}$ & $y=mx+b$ & $y-y_1=m(x-x_1) $\\
$(x-h)^2+(y-k)^2=r^2$ & $y=a(x-h)^2+k$ & \\
\end{tabular}

\vspace{0.5in}

You have two dogs that just don't like each other. You tie each dog to a stake
in your backyard. Fido (dog 1) can cover a circular area bordered by the
equation:

\[(x+2)^2+(y+2)^2=4\]

Fluffy (dog 2) can cover a circular area bordered by the equation:

\[x^2+y^2-8x-2y+8=0\]

\vspace{0.5in}

1. What are the coordinates of Fido's stake (i.e., the center of his circle)?

\vspace{1in}

2. What are the coordinates of Fluffy's stake?

\vspace{3in}

3. What is the distance between the two stakes?

\vspace{3in}

4. Do you need to worry about Fido and Fluffy fighting? Why? (Hint: how do the
two radii compare to the distance?)

\vspace{3in}

5. What is the equation of the line connecting Fido's and Fluffy's stakes?
Please answer in point-slope form.

\vspace{3in}

6. You decide to put a wall in between the two dogs located halfway between the
two stakes.  What is the coordinate of the point on the line found in (5) at
which the wall should cross? (Hint: midpoint)

\vspace{3in}

7. Assuming that the wall is perpendicular to the line in (5), what is the
equation of the line on which the wall is built? (Hint: how are the slopes of
perpendicular lines related?)

\newpage

Consider the equation:

\[y=-x^2+4x-1\]

\vspace{0.5in}

8. What are the $x$-intercept(s) (if any)?

\vspace{3in}

9. What are the $y$-intercept(s) (if any)?

\vspace{2in}

10. Put the equation in standard form.

\vspace{3in}

11. Start with the basic graph and construct a list of rigid transformations to
get from the basic graph to the graph in (10).

\vspace{3in}

12. Sketch the graph in (10) using the list of transformations in (11).

\vspace{4in}

13. Is the graph in (12) that of a function? Why?

\vspace{3in}

14. What is the implied range of the graph?

\vspace{2in}

15. What is the axis of symmetry of the graph?

\vspace{2in}

16. Over what interval(s) is the graph increasing (if any)?

\vspace{2in}

17. Identify any maxima or minima.

\newpage

18. Solve for $x$. Give the result in \emph{interval notation}.

\[2|x-5|+10>14\]

\vspace{3in}

19. You will awarded up to 5 points from your week 9 homework.

\bigskip

20. Determine the domain for the following:

\[y=\frac{(x+1)\sqrt[3]{x+2}}{(x-5)\sqrt{(x^2-16)(x-1)}}\]

\end{document}
