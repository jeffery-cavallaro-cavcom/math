\documentclass[letterpaper, 12pt]{article}

\setlength{\topmargin}{0in}
\setlength{\headheight}{0in}
\setlength{\headsep}{0in}
\setlength{\footskip}{0.5in}
\setlength{\textheight}{\paperheight}
\addtolength{\textheight}{-2in}
\addtolength{\textheight}{-\footskip}

\setlength{\oddsidemargin}{0in}
\setlength{\evensidemargin}{0in}
\setlength{\textwidth}{\paperwidth}
\addtolength{\textwidth}{-2in}

\pagestyle{empty}

\usepackage{amsfonts}

\begin{document}

\begin{center}
\bfseries
San Jos\'{e} State University \\
Fall 2015 \\
Math-8: College Algebra \\
Section 03: MW noon--1:15pm \\
Section 05: MW 4:30--5:45pm \\
\bigskip
Quiz \#7
\end{center}

\bigskip

Closed book and notes. No calculator allowed. All work must be shown for full
credit.

\bigskip

1. The map of a town is laid out on a grid. The police station and the fire
station are both located on Main Street, with the police station at the
position (3, -7) and the fire station at the position (-2, 5).

\bigskip

a. How far is it between the two stations?

\begin{eqnarray*}
d &=& \sqrt{((x_2-x_1)^2+(y_2-y_1)^2} \\
  &=& \sqrt{(3+2)^2+(-7-5)^2} \\
  &=& \sqrt{5^2+(-12)^2} \\
  &=& \sqrt{25+144} \\
  &=& \sqrt{169} \\
d &=& 13 \\
\end{eqnarray*}

b. City Hall is also located on Main Street, exactly halfway between the two
stations. What is the postion of City Hall?

\[\left(\frac{3-2}{2},\frac{-7+5}{2}\right)=
    \left(\frac{1}{2},\frac{-2}{2}\right)=
    \left(\frac{1}{2},-1\right)\]

\newpage

2. Water boils at $212^{\circ} F$ and $100^{\circ} C$. It freezes at $32^{\circ}F$
and $0^{\circ}C$. Let $y$ be degrees Farenheit and $x$ be degrees Celsius.

\bigskip

a. Find an equation for converting from Celsius to Farenheit expressed in
slope-intercept form.

\bigskip

We need to build a line from the points $(100,212)$ and $(0,32)$. First, we
calculate the slope:

\[m=\frac{212-32}{100-0}=\frac{180}{100}=\frac{9}{5}\]

Next, we notice that the point $(0,32)$ gives us a $y$-intercept of 32. Thus,
the equation of the line in slope-intercept form is:

\[y=\frac{9}{5}x+32\]

b. Express the equation in point-slope form using the boiling point.

\bigskip

Using the boiling point of $(100,212)$, we have:

\[y-212=\frac{9}{5}(x-100)\]

c. Express the equation in general form.

Start with the slope-intercept form and do the algebra:

\begin{eqnarray*}
y &=& \frac{9}{5}x+32 \\
\frac{9}{5}x-y+32 &=& 0 \\
9x-5y+160 &=& 0 \\
\end{eqnarray*}

Note that multiplying by 5 in the last step is optional, but gets rid of the
ugly fraction.

\newpage

3. Consider the equation $y=\frac{8}{x^2-2}$.

\bigskip

a. Determine any x-intercept(s).

\[0=\frac{8}{x^2-2}\]

Note that it is not possible to divide 8 by some number and get 0. Thus, there
are no $x$-intercepts.

\bigskip

b. Determine any y-intercept(s).

\[y=\frac{8}{0^2-2}=\frac{8}{0-2}=\frac{8}{-2}=-4\]

Thus, there is a $y$-intercept at $(0,-4)$.

\bigskip

c. Determine any discontinuities.

\bigskip

Discontinuities occur when the denominator is 0. Thus:

\begin{eqnarray*}
x^2-2 &=& 0 \\
x^2 &=& 2 \\
x &=& \pm\sqrt{2} \\
\end{eqnarray*}

d. Determine any symmetries by plugging in (-x) and/or (-y) appropriately. Be
sure to correctly name any discovered symmetry.

\bigskip

Check for y-axis symmetry:

\begin{eqnarray*}
y=\frac{8}{(-x)^2-2}
y=\frac{8}{x^2-2}
\end{eqnarray*}

Yes. Now check for x-axis symmetry:

\begin{eqnarray*}
(-y)=\frac{8}{x^2-2}
y=-\frac{8}{x^2-2}
\end{eqnarray*}

No.  Now check for origin symmetry:

\begin{eqnarray*}
(-y)=\frac{8}{(-x)^2-2}
y=-\frac{8}{x^2-2}
\end{eqnarray*}

No.  So we only have y-axis symmetry.

\bigskip

e. Sketch the graph, showing all of the features found above.

\bigskip

In order to sketch this graph, we first plot the $y$-intercept and where the
discontinuities occur. We must then determine how the graph behaves on each
side of each discontinuity. Note that since there are no $x$-intercepts, the
graph must stay negative between the two discontinuities, and can only possibly
go positive across a discontinuity.

\bigskip

Start with the $x=+\sqrt{2}$ case. If we move a little to the right the value
is positive. If we move a little to the right it is negative. This is mirrored
at the other discontinuity due to y-axis symmetry. We also note that the graph
has $y=0$ as an asymptote. The final graph is thus as follows:

\end{document}
