\documentclass[letterpaper, 12pt]{article}

\setlength{\topmargin}{0in}
\setlength{\headheight}{0in}
\setlength{\headsep}{0in}
\setlength{\footskip}{0.5in}
\setlength{\textheight}{\paperheight}
\addtolength{\textheight}{-2in}
\addtolength{\textheight}{-\footskip}

\setlength{\oddsidemargin}{0in}
\setlength{\evensidemargin}{0in}
\setlength{\textwidth}{\paperwidth}
\addtolength{\textwidth}{-2in}

\pagestyle{empty}

\usepackage{amsfonts}
\usepackage{amsmath}

\begin{document}

\begin{center}
\bfseries
San Jos\'{e} State University \\
Fall 2015 \\
Math-8: College Algebra \\
Section 03: MW noon--1:15pm \\
Section 05: MW 4:30--5:45pm \\
\bigskip
Quiz \#12 (Solutions)
\end{center}

\bigskip

You may use your book, notes, and homework, but please do not work together or
ask for help from others.

\bigskip

\newcommand{\answer}[1]{\textbf{\underline{#1}}}

1. A system of linear equations can have \answer{zero}, \answer{one}, or
\answer{infinite}\ solutions.

\bigskip

2. Find all points of intersection:
\begin{eqnarray*}
x^2+y^2-6x-2y-6 &=& 0 \\
x-y &=& 0 \\
\end{eqnarray*}

This is the intersection between a circle and a line. This can happen in zero
places, one place (tangent), or two places (chord). Substitution is the best
way to solve this. Let $y=x$ and plug into the circle equation:

\begin{eqnarray*}
x^2+x^2-6x-2x-6 &=& 0 \\
2x^2-8x-6 &=& 0 \\
x^2-4x-3 &=& 0 \\
x &=& \frac{4\pm\sqrt{4^2-4(1)(-3)}}{2(1)} \\
  &=& \frac{4\pm\sqrt{16+12}}{2} \\
  &=& \frac{4\pm\sqrt{28}}{2} \\
  &=& \frac{4\pm2\sqrt{7}}{2} \\
x &=& 2\pm\sqrt{7} \\
\end{eqnarray*}

Thus, the points of intersection are $(2-\sqrt{7},2-\sqrt{7})$ and
$(2+\sqrt{7},2+\sqrt{7})$ (since y=x).

\bigskip

3. Why doesn't the answer in problem 2 contradict the statement in problem 1?

Because the system in (2) is not a linear system. Thus, we can have something
other than 0, 1, or $\infty$ solutions --- in this case two solutions.

\bigskip

4. Solve using substitution, elimination, row operations, or matrices. You must
show all steps for full credit:
\begin{eqnarray*}
x+y+z+w &=& 6 \\
2x+3y-w &=& 0 \\
-3x+4y+z+2w &=& 4 \\
x+2y-z+w &=& 0 \\
\end{eqnarray*}

Start by transferring the coefficients into an augmented matrix:

\newenvironment{amatrix}[1]{%
  \left(\begin{array}{@{}*{#1}{c}|c@{}}
}{%
  \end{array}\right)
}

\[\begin{amatrix}{4}
1 & 1 & 1 & 1 & 6 \\
2 & 3 & 0 & -1 & 0 \\
-3 & 4 & 1 & 2 & 4 \\
1 & 2 & -1 & 1 & 0 \\
\end{amatrix}\]

Now, use the 1 in the (1,1) pivot position to get rid of everything in the
column below it. The row operations are as follows:

\[-2*R1+R2\to R2\]

\[\begin{amatrix}{4}
1 & 1 & 1 & 1 & 6 \\
0 & 1 & -2 & -3 & -12 \\
-3 & 4 & 1 & 2 & 4 \\
1 & 2 & -1 & 1 & 0 \\
\end{amatrix}\]

\[3*R1+R3\to R3\]

\[\begin{amatrix}{4}
1 & 1 & 1 & 1 & 6 \\
0 & 1 & -2 & -3 & -12 \\
0 & 7 & 4 & 5 & 22 \\
1 & 2 & -1 & 1 & 0 \\
\end{amatrix}\]

\[-1*R1+R4\to R4\]

\[\begin{amatrix}{4}
1 & 1 & 1 & 1 & 6 \\
0 & 1 & -2 & -3 & -12 \\
0 & 7 & 4 & 5 & 22 \\
0 & 1 & -2 & 0 & -6 \\
\end{amatrix}\]

Now, use the 1 in the (2,2) pivot position to get rid of everything in the
column below it:

\[-7*R2+R3\to R3\]

\[\begin{amatrix}{4}
1 & 1 & 1 & 1 & 6 \\
0 & 1 & -2 & -3 & -12 \\
0 & 0 & 18 & 26 & 106 \\
0 & 1 & -2 & 0 & -6 \\
\end{amatrix}\]

\[-1*R2+R4\to R4\]

\[\begin{amatrix}{4}
1 & 1 & 1 & 1 & 6 \\
0 & 1 & -2 & -3 & -12 \\
0 & 0 & 18 & 26 & 106 \\
0 & 0 & 0 & 3 & 6 \\
\end{amatrix}\]

Now, do some clean-up:

\[-frac{1}{2}*R3\to R3\]

\[\begin{amatrix}{4}
1 & 1 & 1 & 1 & 6 \\
0 & 1 & -2 & -3 & -12 \\
0 & 0 & 9 & 13 & 53 \\
0 & 0 & 0 & 3 & 6 \\
\end{amatrix}\]

\[-frac{1}{3}*R4\to R4\]

\[\begin{amatrix}{4}
1 & 1 & 1 & 1 & 6 \\
0 & 1 & -2 & -3 & -12 \\
0 & 0 & 9 & 13 & 53 \\
0 & 0 & 0 & 1 & 2 \\
\end{amatrix}\]

The matrix is now in row eschelon form! The new system of equations is:

\begin{eqnarray*}
x+y+z+w &=& 6 \\
y-2z-3w &=& -12 \\
9z+13w &=& 53 \\
w &=& 2 \\
\end{eqnarray*}

Now, using back substitution:

\begin{eqnarray*}
9z+13(2) &=& 53 \\
9z+26 &=& 53 \\
9z &=& 27 \\
z &=& 3 \\
\end{eqnarray*}

\begin{eqnarray*}
y-2(3)-3(2) &=& -12 \\
y-6-6 &=& -12 \\
y-12 &=& -12 \\
y &=& 0 \\
\end{eqnarray*}

\begin{eqnarray*}
x+0+3+2 &=& 6 \\
x+5 &=& 6 \\
x &=& 1 \\
\end{eqnarray*}

So, the final answer is $(x,y,z,w)=(1,0,3,2)$.

\end{document}
