\documentclass[letterpaper, 12pt]{article}

\setlength{\topmargin}{0in}
\setlength{\headheight}{0in}
\setlength{\headsep}{0in}
\setlength{\footskip}{0.5in}
\setlength{\textheight}{\paperheight}
\addtolength{\textheight}{-2in}
\addtolength{\textheight}{-\footskip}

\setlength{\oddsidemargin}{0in}
\setlength{\evensidemargin}{0in}
\setlength{\textwidth}{\paperwidth}
\addtolength{\textwidth}{-2in}

\pagestyle{empty}

\usepackage{amsfonts}

\begin{document}

\begin{center}
\bfseries
San Jos\'{e} State University \\
Fall 2015 \\
Math-8: College Algebra \\
Section 03: MW noon--1:15pm \\
Section 05: MW 4:30--5:45pm \\
\bigskip
Quiz \#4 Retake
\end{center}

\bigskip

1. Expand: $(\frac{1}{2}xy+4y^2z)^2$

\vspace{3in}

2. Put in general form: $(3x-2)(x+4)-(5x+1)(5x-1)$

\newpage

3. You have a pool in your backyard and wish to build a fence around it to keep
out the neighborhood children when you are not at home.  The pool is 50 feet
long and 25 feet wide.  You want there to be an equal amount of space between
each edge of the pool and the fence; however, you want the total area to be
3750 square feet.  What is the distance between each edge of the pool and the
fence?

\vspace{3in}

4. You have a credit card that has a 30\% per year interest rate, compounded
monthly. A summary of your monthly statements appears in the table below:

\bigskip

\begin{tabular}{|c|c|c|}
\hline
Month & Purchases & Payments \\
\hline
Oct & \$500 & \$100 \\
\hline
Nov & \$750 & \$1000 \\
\hline
Dec & \$1000 & \$500 \\
\hline
\end{tabular}

\bigskip

Assuming no additional purchases or payments, what is the account balance on
your January statement?

\end{document}
