\documentclass[letterpaper, 12pt]{article}

\setlength{\topmargin}{0in}
\setlength{\headheight}{0in}
\setlength{\headsep}{0in}
\setlength{\footskip}{0.5in}
\setlength{\textheight}{\paperheight}
\addtolength{\textheight}{-2in}
\addtolength{\textheight}{-\footskip}

\setlength{\oddsidemargin}{0in}
\setlength{\evensidemargin}{0in}
\setlength{\textwidth}{\paperwidth}
\addtolength{\textwidth}{-2in}

\pagestyle{empty}

\usepackage{amsfonts}

\begin{document}

\begin{center}
\bfseries
San Jos\'{e} State University \\
Fall 2015 \\
Math-8: College Algebra \\
Section 03: MW noon--1:15pm \\
Section 05: MW 4:30--5:45pm \\
\bigskip
Quiz \#4
\end{center}

\bigskip

1. Expand: $(5xy+2z)^2$

\bigskip

This is just a FOIL problem: $(a+b)^2=a^2+2ab+b^2$, with $a=5xy$ and $b=2z$. So
doing the foil we get:

\begin{eqnarray*}
(5xy+2z)^2 &=& (5xy)^2+2(5xy)(2z)+(2z)^2 \\
           &=& 5^2x^2y^2+20xyz+2^2z^2 \\
           &=& 25x^2y^2+20xyz+4z^2 \\
\end{eqnarray*}

2. Put in general form: $(3x+2)(3x-2)-(2x+5)(x-1)$

Note that the left-hand size is a difference of squares.  The right-hand side
needs to be FOIL'd in the regular way. Be careful to keep the correct scope on
the intervening minus sign!

\begin{eqnarray*}
(3x+2)(3x-2)-(2x+5)(x-1) &=& ((3x)^2-2^2)-(2x^2+3x-5) \\
                         &=& 9x^2-4-2x^2-3x+5 \\
                         &=& (9x^2-2x^2)-3x+(5-4) \\
                         &=& 7x^2-3x+1 \\
\end{eqnarray*}

3. A box is 5 inches longer than it is wide and it is 5 inches deep. Find the
\emph{length} of the box if the total volume is 250 cubic inches.

\bigskip

The way the problem is worded suggests that we make $w$=width be our variable.
Since the length is 5 inches greater than the width, $l=w+5$.  The depth is
given as $d=5$ inches.  Since the formula for the volume of a box is
$width\times length\times depth$, and since the total volume is 250 cubic
inches:

\begin{eqnarray*}
5w(w+5)=250 \\
w(w+5)=50 \\
w^2+5w=50 \\
w^2+5w-50=0 \\
(w+10)(w-5)=0 \\
w=-10,5
\end{eqnarray*}

Since dimensions must be positive, we discard -10 and conclude that our width
is 5 inches. But, the problem asked for the length:

\[length=width+5=5+5=10\ \mbox{inches}\]

4. You have a credit card with an 30\% per year interest rate that compounds
monthly. Your first statement indicates that you made \$1000 in purchases, so
you make a \$100 payment.  The second statement indicates that you made \$500 in
new purchases, so you make another \$100 payment. The third statement says that
you didn't make any new purchases, so you pay \$200.  What is your account
balance after the last \$200 payment?

\bigskip

This problem involves the repeated application of the compound interest formula
over a couple of compounding periods.  Instead of the original principle
remaining the same, as in the previous problems, the principle now changes with
various deposit and withdraw events.

So here is how we determine the coefficients for the polynomial:

\bigskip

\begin{tabular}{|c|c|c|c|}
\hline
Start of Month & Purchases & Payments & Net \\
\hline
1 & 1000 & 100 & 900 \\
\hline
2 & 500 & 100 & 400 \\
\hline
3 & 0 & 200 & -200 \\
\hline
\end{tabular}

\bigskip

Since the problem asks for the balance as soon as the last \$200 dollar payment
is made, we don't wait another whole month. Thus, the problem lasts for only
2 months, or 2 compounding periods.  So, we use the coefficients to build our
polynomial:

\[900x^2+400x-200\]

where $x=1+\frac{0.30}{12}$. Plugging this into your calculator should yield
the answer: \$1155.56.

Note that each term in the polynomial is an instance of the compounded interest
formula, telling how a deposit or withdrawal affects the balance. Positive
values indicate an increase in the account value on which interest must be
paid.  Negative values indicate a decrease in the account value and adjust the
interest accordingly.

\end{document}
