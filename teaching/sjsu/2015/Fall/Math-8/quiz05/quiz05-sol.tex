\documentclass[letterpaper, 12pt]{article}

\setlength{\topmargin}{0in}
\setlength{\headheight}{0in}
\setlength{\headsep}{0in}
\setlength{\footskip}{0.5in}
\setlength{\textheight}{\paperheight}
\addtolength{\textheight}{-2in}
\addtolength{\textheight}{-\footskip}

\setlength{\oddsidemargin}{0in}
\setlength{\evensidemargin}{0in}
\setlength{\textwidth}{\paperwidth}
\addtolength{\textwidth}{-2in}

\pagestyle{empty}

\usepackage{amsfonts}

\begin{document}

\begin{center}
\bfseries
San Jos\'{e} State University \\
Fall 2015 \\
Math-8: College Algebra \\
Section 03: MW noon--1:15pm \\
Section 05: MW 4:30--5:45pm \\
\bigskip
Quiz \#5 Solutions
\end{center}

\bigskip

1. Simplify:
    $\frac{\left[3x\sqrt[5]{x^2+2x+1}\right]^3}{\sqrt{9x\sqrt[3]{x+1}}}$

\bigskip

The general procedure for simplifying an ugly thing like this is to:
\begin{enumerate}
\item{Convert all radicals to rational exponents.}
\item{Simplify the numerator and denominator separately using the various
exponent rules, mostly:
\begin{itemize}
\item{$a^ma^n=a^{m+n}$}
\item{$(ab)^n=a^nb^n$}
\item{$(a^m)^n=a^{mn}$}
\end{itemize}
Remember to apply exponents to all factors inside parentheses!
}
\item{Match up like bases in the numerator and denominator for cancellation
using the $\frac{a^m}{a^n}=a^{m-n}$}
\end{enumerate}

So, let's start by getting rid of the ugly radicals:

\[\frac{\left[3x(x^2+2x+1)^{\frac{1}{5}}\right]^3}
{(9x(x+1)^{\frac{1}{3}})^{\frac{1}{2}}}\]

The denominator is a bit tricky, since it contains a radical in a radical.
Just be sure to do the inside one first and then the outside one and don't
mix them.

Next, we see that the $x^2+2x+1$ in the numerator can be factored, so we will
do so:

\[\frac{\left[3x((x+1)^2)^{\frac{1}{5}}\right]^3}
{(9x(x+1)^{\frac{1}{3}})^{\frac{1}{2}}}\]

Note that we are careful to keep the $\frac{1}{5}$ outside the factor. Now
let's combine the $\frac{1}{5}$ and the $2$ using the $(a^m)^n=a^{mn}$ rule:

\[\frac{\left[3x(x+1)^{\frac{2}{5}}\right]^3}{(9x(x+1)^{\frac{1}{3}})^{\frac{1}{2}}}\]

Now it is time to apply the exponent in the numerator to \emph{all} of its
factors using the $(ab)^n=a^nb^n$ rule:

\[\frac{3^3x^3\left[(x+1)^{\frac{2}{5}}\right]^3}
{(9x(x+1)^{\frac{1}{3}})^{\frac{1}{2}}}\]

We will need to use the $(a^m)^n=a^{mn}$ rule again to combine the $\frac{2}{5}$
and the $3$:

\[\frac{3^3x^3(x+1)^{\frac{6}{5}}}{(9x(x+1)^{\frac{1}{3}})^{\frac{1}{2}}}\]

Now, let's work on the denominator. Start by applying the $\frac{1}{2}$ to all
of its factors:

\[\frac{3^3x^3(x+1)^{\frac{6}{5}}}
{9^{\frac{1}{2}}x^{\frac{1}{2}}\left[(x+1)^{\frac{1}{3}}\right]^{\frac{1}{2}}}\]

Next, resolve the layered exponent:

\[\frac{3^3x^3(x+1)^{\frac{6}{5}}}{3x^{\frac{1}{2}}(x+1)^{\frac{1}{6}}}\]

Now, group the like bases:

\[\left(\frac{3^3}{3^1}\right)
\left(\frac{x^3}{x^{\frac{1}{2}}}\right)
\left(\frac{(x+1)^{\frac{6}{5}}}{(x+1)^{\frac{1}{6}}}\right)\]

Now, apply the $\frac{a^m}{a^n}=a^{m-n}$ rule:

\[3^{3-1}x^{3-\frac{1}{2}}(x+1)^{\frac{6}{5}-\frac{1}{6}}\]

Finally, do the fractional arithmetic to get the answer:

\[3^2x^{\frac{5}{2}}(x+1)^{\frac{31}{30}}\]

or

\[9x^{\frac{5}{2}}(x+1)^{\frac{31}{30}}\]

\bigskip

2. Simplify: $\sqrt{28x}+\sqrt{63x}$

\begin{eqnarray*}
\sqrt{28x}+\sqrt{63x} &=& \sqrt{4\cdot7x}+\sqrt{9\cdot7x} \\
                      &=& 2\sqrt{7x}+3\sqrt{7x} \\
                      &=& (2+3)\sqrt{7x} \\
                      &=& 5\sqrt{7x} \\
\end{eqnarray*}

\bigskip

3. Put into general form and solve by completing the square:
\[2x^2+11x-2=x(x+8)\]

\bigskip

General form means that we have a polynomial on the left with terms listed in
order starting with highest (or lowest) degree term. In this case, we want it
to look like $ax^2+bx+c=0$:

\begin{eqnarray*}
2x^2+11x-2 &=& x^2+8x \\
x^2+3x-2 &=& 0 \\
\end{eqnarray*}

Move the constant to the other side:

\[x^2+3x=2\]

We now need to find something to add to both sides so that the left-hand side
becomes a perfect square. We do this by taking the ``b'' coefficient (3),
dividing it by 2 ($\frac{3}{2}$), squaring it ($frac{9}{4}$), then adding it to
both sides:

\[x^2+3x+\frac{9}{4}=2+\frac{9}{4}\]

The left-hand side is now a perfect square, using the $\frac{b}{2}$ value that
we calculated above:

\[\left(x+\frac{3}{2}\right)^2=\frac{17}{4}\]

Now, when we take the square root of both sides, we need to be careful since
we are dealing with an even power --- we need the absolute value!:

\[\left|x+\frac{3}{2}\right|=\frac{\sqrt{17}}{2}\]

When we remove the absolute value, we need to take plus or minus so that we
don't lose solutions:

\[x+\frac{3}{2}=\pm\frac{\sqrt{17}}{2}\]

We now solve for x:

\[x=-\frac{3}{2}\pm\frac{\sqrt{17}}{2}=\frac{-3\pm\sqrt{17}}{2}\]

Note the similarity of the solutions form to one that we would have received if
we had used the quadratic formula here.

\bigskip

4. A person stands on top of a 100ft cliff and releases a rock.  How long does
it take the rock to hit the ground if:

\begin{description}
\item{a.} The rock is simply dropped.
\item{b.} The rock is thrown downward at 10 ft/s.
\item{c.} the rock is thrown upward at 10 ft/s.
\item{d.} What does the discarded solution in each case represent?
\end{description}

For this problem we need free-fall equation using English units:

\[h=100+v_0t-16t^2\]

Parts a, b, and c are different in their choice of $v_0$. In part a we simply
drop the rock, so $v_0=0$, and we want to know when it hits the ground ($h=0$):

\begin{eqnarray*}
100-16t^2 &=& 0 \\
16t^2 &=& 100 \\
t^2 &=& \frac{100}{16} \\
|t| &=& \frac{10}{4} \\
t &=& \pm\frac{10}{4}=\pm2.5\ \mbox{seconds} \\
\end{eqnarray*}

We take the positive solution, so the answer is 2.5 seconds. But what about the
negative solution? Imagine someone back in time standing on the ground and
throwing a rock such that it stops when it reaches me and then starts falling
at the same instant that I drop my rock.  Our two rocks would be
indistinguishable. His rock would take 2.5 seconds to reach me, which accounts
for the -2.5 seconds solution. So the total roundtrip for his rock would be
$2.5+2.5=5.0$ seconds.

\bigskip

Now lets look at part b.  This time, I help gravity and throw my rock down
instead of just releasing it. Thus, $v_0=-10$ and the full equations is:

\[100-10t-16t^2=0\]

Let's clean this up a bit to make using the quadratic equation a bit easier:

\begin{eqnarray*}
100-10t-16t^2 &=& 0 \\
16t^2+10t-100 &=& 0 \\
8t^2+5t-50 &=& 0 \\
\end{eqnarray*}

We are now ready to use the quadratic formula with $a=8$, $b=5$, and $c=-50$:

\[\frac{-5\pm\sqrt{5^2-4(8)(-50)}}{2(8)}\]

If you punch this into your calculator, you should get 2.2 seconds and
-2.8 seconds, so 2.2 seconds is the answer.  But again, what about the negative
solution? Once again, imagine the other guy on the ground tossing his rock up,
but this time, he throws it hard enough so that it passes me, stops somewhere
above me, and then starts to fall such that when it passes me on the way down,
his rock's velocity matches how fast I throw my rock down. Once again, the
two rocks are indistinguishable.  The -2.8 seconds is the time it took for
his rock to go up, stop, and then pass me on the way down. Thus, the total
round trip for his rock is $2.8+2.2=5.0$ seconds.

\bigskip

For part c I throw my rock up against gravity, so $v_0=+10$ and the full
equation is:

\[100+10t-16t^2=0\]

I could clean this up and plug it into the quadratic formula as before, but
note that the only thing that changes is the sign on the ``b'' value. So, the
times in part b just switch - now my rock takes 2.8 seconds to rise, stop, and
then fall to the ground.  His rock takes 2.2 seconds to reach me such that when
it passes me, it has the same velocity as when I throw my rock up.  Thus, the
path of my rock in part c is the exact reverse of his rock in part b and takes
the same amount of time.

\bigskip

5. Simplify: $\frac{\left(\frac{x^2+6x+5}{x^2+3x+2}-\frac{x-2}{x-5}\right)}
    {\frac{x^2+5x+6}{x-5}}$

\bigskip

Problems such as these are based on our rules for fractions:

\begin{itemize}
\item{$\frac{a}{c}+\frac{b}{c}=\frac{a+b}{c}$}
\item{$\frac{a\cdot c}{b\cdot c}=\frac{a}{b}$}
\item{$\frac{a}{b}\pm\frac{c}{d}=\frac{ad\pm bc}{bd}$}
\item{$\frac{\frac{a}{b}}{\frac{c}{d}}=\frac{a}{b}\cdot\frac{d}{c}$}
\end{itemize}

Start by factoring the polynomials:

\[\frac{\frac{(x+5)(x+1)}{(x+2)(x+1)}-\frac{x-2}{x-5}}
{\frac{(x+2)(x+3)}{x-5}}\]

Note that there is a (x+1) factor in both the numerator and denominator that
cancel:

\[\frac{\frac{x+5}{x+2}-\frac{x-2}{x-5}}{\frac{(x+2)(x+3)}{x-5}}\]

Now, combine the numerator using the addition rule above:

\[\frac{\frac{(x+5)(x-5)-(x+2)(x-2)}{(x-5)(x+2)}}{\frac{(x+2)(x+3)}{x-5}}\]

Next, simplify the numerator a bit:

\[\frac{\frac{(x^2-25)-(x^2-4)}{(x-5)(x+2)}}{\frac{(x+2)(x+3)}{x-5}}=
\frac{\frac{-21}{(x-5)(x+2)}}{\frac{(x+2)(x+3)}{x-5}}\]

Now, apply the division rule above:

\[\frac{-21}{(x-5)(x+2)}\cdot\frac{x-5}{(x+2)(x+3)}\]

Now use the multiple rule above:

\[\frac{-21(x-5)}{(x-5)(x+2)(x+2)(x+3}\]

Finally, we see that the (x-5) cancels and we can combine the (x+2)'s:

\[\frac{-21}{(x+2)^2(x+3)}\]

\bigskip

6. Solve for x: $(x+1)^4=16$

\bigskip

The only trick here is to make sure that we realize that we have an even
exponent, so we will need the absolute value:

\begin{eqnarray*}
(x+1)^4 &=& 16 \\
\left[(x+1)^4\right]^{\frac{1}{4}} &=& (16)^{\frac{1}{4}} \\
|x+1| &=& 2 \\
x+1 &=& \pm2 \\
x=1, -3 \\
\end{eqnarray*}

Note that if had not used the absolute value then we would loose the negative
solution.

\bigskip

7. Solve for x: $4x^4+12x^2-3=0$

Every time we see something of this form where the highest degree (4) is twice
the lower degree (2), we can turn it into a quadratic in the lower degree:

$4(x^2)^2+12(x^2)-3=0$

We can then solve for $x^2$ using factoring, completing the square, or the
quadratic formula.  Let's use the latter with $a=4$, $b=12$, and $c=-3$, but
remember that this is for $x^2$!:

\begin{eqnarray*}
x^2 &=& \frac{-12\pm\sqrt{12^2-4(4)(-3)}}{2(4)} \\
    &=& \frac{-12\pm\sqrt{144+48}}{8} \\
    &=& \frac{-12\pm\sqrt{192}}{8} \\
    &=& \frac{-12\pm8\sqrt{3}}{8} \\
    &=& \frac{-3\pm2\sqrt{3}}{2} \\
\end{eqnarray*}

Before we run off and take the square root to find $x$, let's take a look at
what we have so far.  Notice that one of the solutions for $x^2$ is negative,
which is impossible, so we discard that solution first:

\[x^2=\frac{-3+2\sqrt{3}}{2}\]

Now, go ahead and solve for $x$, once again being careful to use the absolute
value:

\begin{eqnarray*}
\left(x^2\right)^{\frac{1}{2}} &=&
    \left(\frac{-3+2\sqrt{3}}{2}\right)^{\frac{1}{2}} \\
|x| &=& \left(\frac{-3+2\sqrt{3}}{2}\right)^{\frac{1}{2}} \\
x &=& \pm\left(\frac{-3+2\sqrt{3}}{2}\right)^{\frac{1}{2}} \\
\end{eqnarray*}

\end{document}
