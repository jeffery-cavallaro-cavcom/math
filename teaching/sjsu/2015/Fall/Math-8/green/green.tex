\documentclass[letterpaper, 12pt]{article}

\setlength{\topmargin}{0in}
\setlength{\headheight}{0in}
\setlength{\headsep}{0in}
\setlength{\footskip}{0.5in}
\setlength{\textheight}{\paperheight}
\addtolength{\textheight}{-2in}
\addtolength{\textheight}{-\footskip}

\setlength{\oddsidemargin}{0in}
\setlength{\evensidemargin}{0in}
\setlength{\textwidth}{\paperwidth}
\addtolength{\textwidth}{-2in}

\begin{document}

\begin{center}
\bfseries
San Jos\'{e} State University \\
Fall 2015 \\
Math-8: College Algebra \\
Section 03: MW noon--1:15pm \\
Section 05: MW 4:30pm-5:45pm \\
DH-416
\end{center}

\bigskip

\begin{description}
\setlength{\itemsep}{0in}

\item[Instructor:] Jeffery Cavallaro

\item[Office:] DH-209

\item[Office hours:] MWF 9am--10am

\item[E-mail:] {\texttt{jeffery.cavallaro@sjsu.edu}}

\item[Texts:] \emph{College Algebra and Calculus: An Applied Approach},
  Larson and Hodgkins, \textbf{2nd edition}.  Make sure that you have the
  second edition and not the first.  The webassign combo-pack is optional ---
  we are not be using webassign, but you may find the included ebook and
  teaching materials helpful.  Note that this is the same book that is used
  in Math-71.

\item[Calculator:] You should have a ``scientific calculator'' (e.g., TI-30X)
  for use on quizzes, exams, and the final. You are \emph{not} allowed to use
  either a graphing calculator (e.g., TI-83) or a calculator that can do
  symbolic algebra (e.g., TI-89 or TI-92) on quizzes, exams, or the final;
  however, you may use the more advanced calculators to check your homework
  answers (after you have attempted the problems!).

\end{description}

\bigskip

\begin{description}
\setlength{\itemsep}{0in}

\item[Learning Objectives.] In this course you will master the algebra skills
  required for later math classes, understand and apply fundamental ideas about
  functions, and study some specific types of functions (e.g., polynomials,
  exponentials, logarithms).  You will use mathematical methods to solve
  quantitative (word) problems and arrive at conclusions based on numerical and
  graphical data.  These learning objectives, as well as the minimum 500 word
  writing requirement (homework, quizzes, and exams), satisfy the Area B4
  (Mathematical Concepts) GE requirement.

\item[Math 8W.] You are strongly encouraged to register and participate in the
  workshop that corresponds to this class.  Math 8W is a 1 unit class (two
  sessions/week) where you will work in teams on problems, aided by a workshop
  facilitator.  Although you are not required to take Math 8W, experience has
  shown that students who take the workshop are more successful than those who
  do not.

\item[Canvas.] All class communications will be via canvas
  (sjsu.instructure.com). All SJSU students are assigned a canvas account and
  all Math-8 students should be assigned to a Math-8 section course.  All
  assignments and grades will be available in canvas. Please use canvas for any
  public discussions; however, feel free to email me directly for more personal
  issues.

\item[Class.] Bring your textbook and calculator to class every day.
  \emph{Anything} covered in class is fair game for quizzes and exams (unless
  otherwise stated), so don't miss class.  If you do miss a class then it is
  your responsibility to get with one of your classmates to determine what you
  missed.  The last day of class is Monday, 12/7.

\item[Time Expectations.] The standard expectation for a 3-unit class at SJSU
  is that you will spend at least 9 hours per week working on the class --- 3
  hours in the classroom and a minimum of 6 hours on homework and studying.

\item[Holidays.] Class will not meet on 9/7 (Labor Day) or 11/11
  (Veteran's Day).

\item[Reading.] Reading is assigned each Wednesday for the material that will
  be covered in lecture during the following week.  It will be announced in
  class and recorded on canvas.  The proper way the read a math text is to read
  it through once quickly, then go back and read it more carefully (as many
  times as needed), working through all examples.

\item[Homework.] Homework is assigned on each Monday and will be recorded on
  canvas, but it will not be collected.  Math is learned by doing, so you must
  do the homework if you expect to pass this class.  In fact, quiz and exam
  problems will strongly resemble the assigned homework problems.  You may work
  on homework in teams; however, it is vital that you individually know how to
  do the problems instead of leaning on your team.  Assigned homework problems
  may be discussed in class, in workshop, and during office hours.  Solutions
  will be posted on canvas on Friday so that you have the weekend to check your
  work.

\item[Quizzes.] A 20-30 minute quiz is given each Monday (except for exam
  weeks) at the start of class.  Quizzes cover the reading, lecture, and
  homework material from the previous week.  Quizzes are in-class, closed-book,
  and closed-notes unless specified otherwise; however, a simple scientific
  calculator is allowed.  Instead of a note card, I will supply any hints that
  I think are reasonable.  The first quiz is \textbf{Monday, Aug 31}.  There
  are no make-ups; however, only your top 10 quiz scores will be counted.

\item[Exams.] There will be 3 non-cumulative exams (tentatively) scheduled on
  the following Wednesdays: 9/30, 10/28, and 11/25.  Exams will last the entire
  class period, cover all new material since the last exam,  and follow the
  same rules as the quizzes.  \emph{You must plan to take the exams at their
  scheduled times}  A missed exam for anything other than an SJSU
  officially-recognized excuse will receive a 0.  In particular, the Monday,
  Tuesday, and Wednesday of Thanksgiving week are \emph{not} SJSU holidays and
  travel plans are not an acceptable excuse.  Missed exams due to valid excuses
  will be treated on a case-by-case basis at the instructor's descretion.

\item[Final.] The final exam is cumulative and is scheduled for
  \textbf{Saturday, 12/12, 9:45am--noon}.  All sections of Math-8 are taking the
  same final at the same time, and the exam \emph{must} be taken at that time.
  Any and all excuses must be discussed directly with the math office (MH-308).
  Once again, travel arrangements are not a valid excuse, so don't plan to leave
  town prior to noon on 12/12.  The final exam follows the same rules as the
  quizzes and exams; however, a $3\times 5$ notecard is allowed.  A sample
  final is available on canvas.

\item[Grading.]  Your semester grade is determined as follows: Quizzes 20\%;
  Exams 1, 2, and 3 20\% each; and Final 20\%.  Grades given: A+ (97--100),
  A (93--96), A- (90--92), B+ (87--89), B (83--86), B- (80--82), C+ (77--79),
  C (73--76), C- (70--72), D (60--69), otherwise F.  Note that you need to get
  a C or better to satisfy the Area B4 (Mathematical Concepts) GE requirement.

\item[Academic Integrity.] Your commitment to learning (as shown by your
  enrollment at SJSU) and SJSU's Academic Integrity Policy require you to be
  honest in all of your academic course work.  Faculty are required to report
  all infractions to the Office of Student Conduct and Ethical Development.
  See: \texttt{www.sjsu.edu/studentconduct}.

\item[Disabilities.] If you need course adaptations or accommodations due to a
  disability, or if you need special arrangements in case the building must be
  evacuated, please make an appointment with me as soon as possible, or see me
  during office hours.  Presidential Directive 97-03 requires that students
  with disabilities register with the Accessible Education Center to establish
  a record of their disability.

\item[Tutoring.] Peer tutoring is available to all SJSU students, free of
  charge, through:
  \begin{itemize}
  \item The College of Science Advising Center (\texttt{www.sjsu.edu/cosac})
  \item Peer Connections (\texttt{peerconnections.sjsu.edu})
  \end{itemize}

\end{description}

\end{document}
