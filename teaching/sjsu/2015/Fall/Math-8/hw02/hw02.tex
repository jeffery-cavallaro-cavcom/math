\documentclass[letterpaper]{article}

\setlength{\topmargin}{0in}
\setlength{\headheight}{0in}
\setlength{\headsep}{0in}
\setlength{\footskip}{0.5in}
\setlength{\textheight}{\paperheight}
\addtolength{\textheight}{-2in}
\addtolength{\textheight}{-\footskip}

\setlength{\oddsidemargin}{0in}
\setlength{\evensidemargin}{0in}
\setlength{\textwidth}{\paperwidth}
\addtolength{\textwidth}{-2in}

\pagestyle{empty}

\begin{document}

\newcommand{\?}{\stackrel{?}{=}}

\begin{center}
\bfseries
San Jos\'{e} State University \\
Fall 2015 \\
Math-8: College Algebra \\
Section 03: MW noon--1:15pm \\
Section 05: MW 4:30pm-5:45pm \\
\bigskip
Homework Week 2 Solutions
\end{center}

\underline{\textbf{Section 0.2}}

\bigskip

6.  $3x^4 + 2x^3 + x^2 - 1$ has 4 terms:

\bigskip

\begin{tabular}{|c|c|c|}
\hline
term & coefficient & variable \\
\hline
$3x^4$ & $3$ & $x$ \\
\hline
$2x^3$ & $2$ & $x$ \\
\hline
$x^2$ & $1$ & $x$ \\
\hline
$1$ & $-1$ & none \\
\hline
\end{tabular}

\bigskip

Note that the coefficient of $x^2$ is in fact 1.  Also note that the constant
term is $-1$, not just $1$ (remember, adding the negative!).  Also also note
that even though there is no variable with the term $-1$, when we talk about
polynomials, we consider the constant term to be associated with $x^0$.

\bigskip

10. \[ -5(-2-6) = -5(-8) = 40 \]
or to demonstrate proper use of the distributive law:
\[ -5(-2-6) = -5[(-2)+(-6)] = -5(-2)+-5(-6) = 10 + 30 = 40 \]

\bigskip

14. Evaluate: $-x^3 + 2x - 1$

\bigskip

a. At $x = 0$:

\[ -(0)^3 + 2(0) - 1 = 0 + 0 - 1 = -1 \]

b. At $x = 2$:

\[ -(2)^3 + 2(2) - 1 = -8 + 4 - 1 = -5 \]

22. Evaluate: $\frac{4z - 2y}{20x}$ at $x = 3$, $y = -2$, and $z = 4$.

\[ \frac{4(4) - 2(-2)}{20(3)} = \frac{16 + 4}{60}  = \frac{20}{60} =
  \frac{1}{3} \]

\begin{tabular}{cccl}
23. & $3+4=4+3$ & A1 & Commutative Addition \\
24. & $x+9=9+x$ & A1 & Commutative Addition \\
25. & $-15 + 15 = 0$ & A4 & Additive Inverse \\
26. & $(x + 2) - (x + 2)$ & A4 & Additive inverse \\
27. & $2(x+3)=2x+6$ & LD & Left Distributive \\
28. & $(5 + 9)6 = 5(6)+9(6)$ & RD & Right Distribution \\
29. & $2(\frac{1}{2}=1$ & M4 & Multiplicative Inverse\\
30. & $\frac{1}{h+6}(h+6)=1$ & M4 & Multiplicative Inverse \\
31. & $h+0=h$ & A3 & Additive Identity \\
32. & $(z-2)+0=z-2$ & A3 & Additive Identity \\
33. & $57 \cdot 1 = 57$ & M3 & Multiplicative Identity \\
34. & $1 \cdot (1+x) = 1 + x$ & M3 & Multiplicative Identity \\
35. & $6+(7+8)=(6+7)+8$ & A2 & Associate Addition \\
36. & $x+(y+10)=(x+y)+10$ & A2 & Associate Addition \\
\end{tabular}

\bigskip

37.

\bigskip

\begin{tabular}{rcl}
$x(3y)=(x \cdot 3)y$ & M2 & Associate Multiplication \\
$=(3x)y$ & M1 & Commutative Multiplication \\
\end{tabular}

\bigskip

38.

\bigskip

\begin{tabular}{rcl}
$\frac{1}{7}(7\cdot12)=(\frac{1}{7}\cdot7)12$ & M2 &
    Associate Multiplication \\
$=1\cdot12$ & M4 & Multiplicative Inverse \\
$=12$ & M3 & Multiplicative Identity \\
\end{tabular}

\bigskip

42. $150 = 5\cdot30 = 5\cdot5\cdot6 = 5\cdot5\cdot2\cdot3 = 2\cdot3\cdot5^2$

\bigskip

52.
\begin{eqnarray*}
\frac{10}{11}+\frac{6}{33}-\frac{13}{66} &=&
    \frac{10\cdot6}{11\cdot6}+\frac{6\cdot2}{33\cdot2}-\frac{13}{66} \\
&=& \frac{60}{66}+\frac{12}{66}-\frac{13}{66} \\
&=& \frac{60+12-13}{66} \\
&=& \frac{59}{66} \\
\end{eqnarray*}

54.
\begin{eqnarray*}
-\frac{2}{3}\cdot\frac{5}{8}\cdot\frac{3}{4} &=&
    -\frac{2\cdot5\cdot3}{3\cdot8\cdot4} \\
&=& -\frac{3\cdot2\cdot5}{3\cdot4\cdot8} \\
&=& -\frac{5}{2\cdot8} \\
&=& -\frac{5}{16} \\
\end{eqnarray*}

70.

\bigskip

Total people in study $=12857$

Percent with risk $=39.5\%$

People at risk $=12857\cdot0.395=5079$

Note that the answer should be a whole number of people, and in this case we
needed to round up.

\bigskip

\underline{\textbf{Section 1.1}}

\bigskip

2. $3(x+2)=3x+6$ is an identity; it is a statement of the left distributive
rule for all real numbers $x$.

\bigskip

4. $3(x+2)=2x+4$ is conditional (one solution):

\begin{eqnarray*}
3(x+2) &=& 2x+4 \\
3x+6 &=& 2x+4 \\
x &=& -2 \\
\end{eqnarray*}

\bigskip

12. $3+\frac{1}{x+2}=4$

\bigskip

a. $x=-1$
\begin{eqnarray*}
3+\frac{1}{(-1)+2} &\?& 4 \\
3+\frac{1}{1} &\?& 4 \\
3+1 &\?& 4 \\
4 &=& 4 \\
\mbox{YES!} \\
\end{eqnarray*}

b. $x=-2$
\begin{eqnarray*}
3+\frac{1}{(-2)+2} &\?& 4 \\
3+\frac{1}{0} &\?& 4 \\
\mbox{\emph{undefined}} &\ne& 4 \\
\mbox{NO!} \\
\end{eqnarray*}

c. $x=0$
\begin{eqnarray*}
3+\frac{1}{0+2} &\?& 4 \\
3+\frac{1}{2} &\?& 4 \\
\frac{7}{2} &\ne& 4 \\
\mbox{NO!} \\
\end{eqnarray*}

d. $x=5$
\begin{eqnarray*}
3+\frac{1}{5+2} &\?& 4 \\
3+\frac{1}{7} &\?& 4 \\
\frac{22}{7} &\ne& 4 \\
\mbox{NO!} \\
\end{eqnarray*}

\bigskip

14. $(3x+5)(2x-7)=0$

\bigskip

Remember the last property of 0 on page 14: if you multiple two numbers and get
0 then at least one of them has to be 0.  And also remember that $3x+5$ and
$2x-7$ are both \emph{just numbers}!.

\bigskip

a. $x = -\frac{5}{3}$

\bigskip

$3(-\frac{5}{3}) + 5 = -5 + 5 = 0$

YES!

\bigskip

b. $x = -\frac{2}{7}$

\bigskip

$3(-\frac{2}{7}) + 5 = -\frac{6}{7} + 5 = \frac{29}{7} \ne 0$

$2(-\frac{2}{7}) - 7 = -\frac{4}{7} - 7 = -\frac{53}{7} \ne 0$

NO!

\bigskip

c. $x = \frac{2}{3}$

\bigskip

$3(\frac{2}{3}) + 5 = \frac{6}{3} + 5 = 2 + 5 = 7 \ne 0$

$2(\frac{2}{3}) - 7 = \frac{4}{3} - 7 = -\frac{17}{3} \ne 0$

NO!

\bigskip

d. $x = \frac{3}{2}$

\bigskip

$3(\frac{3}{2}) + 5 = \frac{9}{2} + 5 = \frac{19}{2} \ne 0$

$2(\frac{3}{2}) - 7 = 3 - 7 = -4 \ne 0$

NO!

\bigskip

Note that in the following problems I am very careful that each syntactic
step employs at most one of our 10 rules (with possible substitution).  You
don't need to generally be this careful (unless I ask you to in a problem to
test your knowledge of the steps); however, you really need to watch for
pitfalls with subtraction and division.

\bigskip

18.
\begin{eqnarray*}
9 - x &=& 13 \\
(9 + (-x)) &=& 13 \\
(9 + (-x)) + x &=& 13 + x \\
9 + ((-x) + x) &=& 13 + x \\
9 + 0 &=& 13 + x \\
9 &=& 13 + x \\
9 &=& x + 13 \\
9 + (-13) &=& (x + 13) + (-13) \\
-4 &=& x + (13 + (-13)) \\
-4 &=& x + 0 \\
-4 &=& x \\
x &=& -4 \\
\end{eqnarray*}

20.
\begin{eqnarray*}
7x+2 &=& 16 \\
(7x+2)+(-2) &=& 16+(-2) \\
7x+(2+(-2)) &=& 14 \\
7x+0 &=& 14 \\
7x &=& 14 \\
\frac{1}{7}(7x) &=& \frac{1}{7}(14) \\
(\frac{1}{7}(7))x &=& 2 \\
1x &=& 2 \\
x &=& 2 \\
\end{eqnarray*}

22.
\begin{eqnarray*}
7x+3 &=& 3x-13 \\
7x+3 &=& 3x+(-13) \\
(-3x)+(7x+3) &=& (-3x)+(3x+(-13)) \\
(-3x+7x)+3 &=& ((-3x)+3x)+(-13) \\
((-3+7)x)+3 &=& 0 + (-13) \\
4x+3 &=& -13 \\
(4x+3)+(-3) &=& -13+(-3) \\
4x+(3+(-3)) &=& -16 \\
4x+0 &=& -16 \\
4x &=& -16 \\
\frac{1}{4}(4x) &=& \frac{1}{4}(-16) \\
(\frac{1}{4}(4))x &=& -4 \\
1x &=& -4 \\
x &=& -4
\end{eqnarray*}

28.
\begin{eqnarray*}
2(13t-15)+3(t-19) &=& 0 \\
(2(13t)+2(-15))+(3t+3(-19)) &=& 0 \\
((2\cdot13)t+(-30))+(3t+(-57)) &=& 0 \\
(26t+(-30))+(3t+(-57)) &=& 0 \\
(26t+(-30))+(-57+3t) &=& 0 \\
26t+((-30)+(-57+3t)) &=& 0 \\
26t+(((-30)+(-57))+3t) &=& 0 \\
26t+(-87+3t) &=& 0 \\
26t+(3t+(-87)) &=& 0 \\
(26t+3t)+(-87) &=& 0 \\
((26t+3t)+(-87))+87 &=& 0+87 \\
(26t+3t)+((-87)+87) &=& 87 \\
(26t+3t)+0 &=& 87 \\
26t+3t &=& 87 \\
(26+3)t &=& 87 \\
29t &=& 87 \\
\frac{1}{29}(29x) &=& \frac{1}{29}(87) \\
(\frac{1}{29}(29))x &=& 3 \\
1x &=& 3 \\
x &=& 3
\end{eqnarray*}

30.
\begin{eqnarray*}
3(2x-(x+7)) &=& 5(x-3) \\
3(2x+(-x+(-7)) &=& 5x+5(-3) \\
3((2x+(-x))+(-7) &=& 5x+(-15) \\
3((2+(-1)x+(-7)) &=& 5x+(-15) \\
3(x+(-7)) &=& 5x+(-15) \\
3x+3(-7) &=& 5x+(-15) \\
3x+(-21) &=& 5x+(-15) \\
(-3x)+(3x+(-21)) &=& (-3x)+(5x+(-15)) \\
(-3x+3x)+(-21) &=& ((-3x)+5x)+(-15) \\
0+(-21) &=& (-3+5)x+(-15) \\
-21 &=& 2x+(-15) \\
-21+15 &=& (2x+(-15))+15 \\
-6 &=& 2x+(-15+15) \\
-6 &=& 2x+0 \\
-6 &=& 2x \\
\frac{1}{2}(-6) &=& \frac{1}{2}(2x) \\
-3 &=& (\frac{1}{2}(2))x \\
-3 &=& 1x \\
-3 &=& x \\
x &=& -3 \\
\end{eqnarray*}

34.
\begin{eqnarray*}
\frac{x}{5}-\frac{x}{2} &=& 3 \\
\frac{x\cdot2}{5\cdot2}-\frac{x\cdot5}{2\cdot5} &=& 3 \\
\frac{2x}{10}-\frac{5x}{10} &=& 3 \\
\frac{2x+(-5x)}{10} &=& 3 \\
\frac{(2+(-5))x}{10} &=& 3 \\
\frac{-3x}{10} &=& 3 \\
\frac{1}{10}(-3x) &=& 3 \\
10(\frac{1}{10}(-3x)) &=& 10(3) \\
(10(\frac{1}{10}))(-3x) &=& 30 \\
1(-3x) &=& 30 \\
-3x &=& 30 \\
-\frac{1}{3}(-3x) &=& -\frac{1}{3}(30) \\
(-\frac{1}{3}(-3))x &=& -10 \\
1x &=& -10 \\
x &=& -10 \\
\end{eqnarray*}

36.
\begin{eqnarray*}
\frac{3x}{2}+\frac{1}{4}(x-2) &=& 10 \\
4(\frac{3x}{2}+\frac{1}{4}(x-2)) &=& 4(10) \\
4(\frac{3x}{2})+4(\frac{1}{4}(x-2)) &=& 40 \\
4(\frac{1}{2}(3x))+(4(\frac{1}{4}))(x-2) &=& 40 \\
(4(\frac{1}{2}))(3x)+(1)(x-2) &=& 40 \\
(2)(3x)+(x-2) &=& 40 \\
(2\cdot3)x+(x-2) &=& 40 \\
6x+(x+(-2)) &=& 40 \\
(6x+x)+(-2) &=& 40 \\
(6+1)x+(-2) &=& 40 \\
7x+(-2) &=& 40 \\
(7x+(-2))+2 &=& 40+2 \\
7x+((-2)+2) &=& 42 \\
7x+0 &=& 42 \\
7x &=& 42 \\
\frac{1}{7}(7x) &=& \frac{1}{7}(42) \\
(\frac{1}{7}(7))x &=& 6 \\
1x &=& 6 \\
x &=& 6 \\
\end{eqnarray*}

38.
\begin{eqnarray*}
\frac{17+y}{y}+\frac{32+y}{y} &=& 100 \\
y(\frac{17+y}{y}+\frac{32+y}{y}) &=& y\cdot100 \\
y(\frac{17+y}{y})+y(\frac{32+y}{y}) &=& 100y \\
y(\frac{1}{y}(y+17))+y(\frac{1}{y}(32+y)) &=& 100y \\
(y(\frac{1}{y}))(y+17)+(y(\frac{1}{y}))(32+y) &=& 100y \\
(1)(y+17)+(1)(32+y) &=& 100y \\
(y+17)+(32+y) &=& 100y \\
y+(17+(32+y)) &=& 100y \\
y+((17+32)+y) &=& 100y \\
y+(49+y) &=& 100y \\
y+(y+49) &=& 100y \\
(y+y)+49 &=& 100y \\
(1+1)y+49 &=& 100y \\
2y+49 &=& 100y \\
(-2y)+(2y+49) &=& (-2y)+100y \\
((-2y)+2y)+49 &=& (-2+100)y \\
0+49 &=& 98y \\
49+0 &=& 98y \\
49 &=& 98y \\
\frac{1}{98}(49) &=& \frac{1}{98}(98y) \\
\frac{1}{2} &=& (\frac{1}{98}(98))y \\
\frac{1}{2} &=& 1y \\
\frac{1}{2} &=& y \\
y &=& \frac{1}{2} \\
\end{eqnarray*}

40.
\begin{eqnarray*}
\frac{10x+3}{5x+6} &=& \frac{1}{2} \\
2(10x+3) &=& 1(5x+6) \\
2(10x)+2(3) &=& 5x+6 \\
(2\cdot10)x+6 &=& 5x+6 \\
20x+6 &=& 5x+6 \\
(20x+6)+(-6) &=& (5x+6)+(-6) \\
20x+(6+(-6)) &=& 5x+(6+(-6)) \\
20x+0 &=& 5x+0 \\
20x &=& 5x \\
20x+(-5x) &=& 5x+(-5x) \\
(20+(-5))x &=& 0 \\
15x &=& 0 \\
\frac{1}{15}(15x) &=& \frac{1}{15}(0) \\
(\frac{1}{15}(15))x &=& 0 \\
1x &=& 0 \\
x &=& 0 \\
\end{eqnarray*}

50.
\begin{eqnarray*}
0.60x+0.40(100-x) &=& 50 \\
0.60x+(0.40(100)+0.40(-x)) &=& 50 \\
0.60x+(40+(-0.40x)) &=& 50 \\
0.60x+((-0.40x)+40) &=& 50 \\
(0.60x+(-0.40x))+40 &=& 50 \\
0.20x+40 &=& 50 \\
(0.20x+40)+(-40) &=& 50+(-40) \\
0.20x+(40+(-40)) &=& 10 \\
0.20x+0 &=& 10 \\
0.20x &=& 10 \\
\frac{1}{0.20}(0.20x) &=& \frac{1}{0.20}(10) \\
(\frac{1}{0.20}(0.20))x &=& 50 \\
1x &=& 50 \\
x &=& 50 \\
\end{eqnarray*}

76. First, we let $S=18$ (for $\$18$ billion) and solve:
\begin{eqnarray*}
18 &=& 2.903t-1.98 \\
18+1.98 &=& (2.903t+(-1.98))+1.98 \\
19.98 &=& 2.903t+((-1.98)+1.98) \\
19.98 &=& 2.903t+0 \\
19.98 &=& 2.903t \\
\frac{19.98}{2.903} &=& \frac{2.903t}{2.903} \\
6.883 &=& t \\
t &=& 6.883 \\
\end{eqnarray*}
So somewhere near the end of ``year 6'', which is 2006.

\bigskip

78. First, we solve the male femur length equation to determine the man's
height:
\begin{eqnarray*}
21 &=& 0.449x-12.15 \\
21+12.15 &=& (0.449x+(-12.15))+12.15 \\
33.15 &=& 0.449x+((-12.15)+12.15) \\
33.15 &=& 0.449x+0 \\
33.15 &=& 0.449x \\
\frac{33.15}{0.449} &=& \frac{0.449x}{0.449} \\
73.83 &=& 1x \\
x &\approx& 74\ \mbox{inches} \\
x &\approx& 6\ \mbox{feet}\ 2\ \mbox{inches} \\
\end{eqnarray*}
Thus, it is possible that the femur belongs to the missing man.

\end{document}
