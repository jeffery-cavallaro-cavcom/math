\documentclass[letterpaper,12pt,fleqn]{article}
\usepackage{matharticle}
\usepackage{cancel}
\pagestyle{plain}
\begin{document}

\section*{Algebra Review}

\bigskip

\fbox{
  \parbox{\textwidth}{
    It is strongly suggested that you make use of the \emph{Personal Study Plan} in Webassign to at least review the
    material in Chapter 0, especially if you got a C in Math 8.
  }
}

\bigskip

Recall the real number axioms from Section 0.2, Basic Rules of Algebra, p12:

\fbox{
  \parbox{6in}{
    \(\forall\,a,b,c\in\R\):

    \bigskip

    \begin{tabular}{clc}
      1) & Commutative Addition & \(a+b=b+a\) \\
      2) & Commutative Multiplication & \(ab=ba\) \\
      3) & Associative Addition & \((a+b)+c=a+(b+c)\) \\
      4) & Associative Multiplication & \((ab)c=a(bc)\) \\
      5) & Additive Identity & \(a+0=a\) \\
      6) & Multiplicative Identity & \(a\cdot1=a\) \\
      7) & Additive Identity & \(a+(-a)=0\) \\
      8) & Multiplicative Identity (reciprocal) & \(a\cdot\frac{1}{a}=1\qquad (a\ne0)\) \\
      9) & Left Distributive & \(a(b+c)=ab+ac\) \\
      10) & Right Distributive & \((b+c)a=ba+ca\) \\
    \end{tabular}
  }
}

Note that the right distributive property is usually stated as a tenth axiom; however, it can be derived from the
others:
\[(b+c)a=a(b+c)=ab+ac=ba+ca\]

To these axioms, we added the axioms of equality from Section 0.2, p16:

\fbox{
  \parbox{6in}{
    \(\forall\,a,b,c\in\R\):

    \bigskip

    \begin{description}
    \item{Reflexive:} \(a=a\)
    \item{Symmetric:} \(a=b\implies b=a\)
    \item{Transitive:} \(a=b\) and \(b=c\implies a=c\)
    \end{description}
  }
}

And then, the substitution principle:

\fbox{
  \parbox{6in}{
    If two real value expressions are equal then one can be typographically replaced with the other.
  }
}

For example, consider the expression:
\[a+5\]
If it is true that \(a=-1\) then I can replace \(a\) with \(-1\) and evaluate:
\[(-1)+5=5\]

And finally, some notation:

\fbox{
  \parbox{6in}{
    \begin{tabular}{lc}
      Subtraction: & \(a-b=a+(-b)\) \\
      Division: & \(\frac{a}{b}=a\cdot\frac{1}{b}=\frac{1}{b}\cdot a\)
    \end{tabular}
  }
}

So note that subtraction and division don't really exist --- they are convenience notations for addition of an
additive inverse and multiplication of a multiplicative inverse (reciprocal), respectively.

\bigskip

\fbox{
  \parbox{6in}{
    \emph{\underline{EVERYTHING}} that we do with the real numbers must be traceable back to one of these axioms!
    We add to our algebra toolbox by making new definitions (like exponentiation) and using the axioms to prove new
    rules called \emph{theorems}.  We then use the definitions, axioms, and theorems that we have already proved to
    create new definitions and prove even more theorems.  This is how mathematics works.  Students make algebra
    mistakes when they try and make up their own rules.  Sometimes, these made-up rules resemble a true rule,
    adding to the confusion.
  }
}

\bigskip

\begin{examples}
  Let \(a,b\in\R\):
  \begin{enumerate}
  \item The exponent rules from Section 0.3, Integer Exponents, p21:
    \[(ab)^2=(ab)(ab)=a(ba)b=a(ab)b=(aa)(bb)=a^2b^2\]
    It looks like we are ``distributing'' the exponent, but that is not correct, we distribute factors:
    \(2(a+b)=2a+2b\); not exponents.

  \item Special products from Section 0.5, p43:
    
    NO:
    \[(a+b)^2=a^2+b^2\]
    Although it resembles the previous example, it is not the same (addition, not multiplication).

    YES:
    \[(a+b)^2=(a+b)(a+b)=(a+b)a+(a+b)b=aa+ba+ab+bb=a^2+ab+ab+b^2=a^2+2ab+b^2\]
    FOILing is a shortcut.

  \item Cancellation of factors:

    \[\frac{ab}{a}=\frac{1}{a}\cdot(ab)=\left(\frac{1}{a}\cdot a\right)b=1b=b\]

  \item \(\frac{a+b}{a}\)

    NO:
    \[\frac{\cancel{a}+b}{\cancel{a}}=1+b\]
    Although it resembles the previous example, it is not the same (addition, not multiplication).

    YES:
    \[\frac{a+b}{a}=\frac{1}{a}(a+b)=\frac{1}{a}\cdot a+\frac{1}{a}\cdot b=1+\frac{b}{a}\]
  \end{enumerate}
\end{examples}

So, if everything that we need is provided by these axioms, what in the heck is this \emph{calculus} thing and how
is it different from algebra?  To answer this question we introduce a new concept: \emph{arbitrarily close}.

\end{document}
