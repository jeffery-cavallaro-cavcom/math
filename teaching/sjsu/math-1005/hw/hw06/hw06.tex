\documentclass[letterpaper,12pt,fleqn]{article}
\usepackage{matharticle}
\usepackage{mathtools}
\pagestyle{plain}
\begin{document}

\begin{center}
  \Large Math-1005a Homework \#6
  
  Exponents: Multiplying/Dividing Common Bases
\end{center}

\vspace{0.5in}

\underline{Problems}

\begin{enumerate}
\item An exponential expression is an expression of the form $a^b$ where $a$
  is called the base and $b$ is called the exponent. We will begin our
  examination of exponential expressions by assuming that the base and the
  exponent are positive integers greater than 1. Recall that multiplication is
  a shorthand for repeated addition. For example:
  \[2\cdot3=2+2+2=6\]
  Similarly, exponentiation is a shorthand for repeated multiplication:
  \[2^3=2\cdot2\cdot2=8\]
  Evaluate each of the following:
  
  \begin{tabular}{p{1.5in}p{1.5in}p{1.5in}}
    $2^2=$ & $2^3=$ & $2^4=$ \\
    \\
    $3^2=$ & $3^3=$ & $3^4=$ \\
    \\
    $4^2=$ & $4^3=$ & $4^4=$ \\
    \\
    $5^2=$ & $5^3=$ & $5^4=$
  \end{tabular}

\item Now let's look at some of the cases where the either the exponent or base
  are either 0 or 1:
  \begin{itemize}
  \item We will not consider the case where the base and the exponent are
    both zero: $0^0$.

  \item Any non-zero value $a$ to the zero power is always $1$: $a^0=1$.

  \item Any value $a$ to the first power is just $a$ and we omit the $1$:
    $a^1=a$.

  \item Zero to any non-zero value $a$ is always zero: $0^a=0$.

  \item One to any value $a$ is always one: $1^a=1$.
  \end{itemize}

  Evaluate the following exponential expressions: 

  \begin{tabular}{p{1in}p{1in}p{1in}p{1in}}
    $1^0=$ & $1^1=$ & $0^1=$ & $0^{100}$ \\
    \\
    $5^1=$ & $0^5=$ & $5^0=$ & $1^5$ \\
    \\
    $0^{15}=$ & $15^0=$ & $1^{15}=$ & $15^1$
  \end{tabular}

\item When the base in an exponential expression is negative, we still have
  $a^0=1$ and $a^1=a$, but when the exponent is an integer greater than or
  equal to $2$ then the sign of the result is dependent on whether the exponent
  is even or odd.

  Consider an example with an even power:
  \[(−2)^4=(−2)(−2)(−2)(−2)=16\]
  Since we are multiplying an even number of negative values the result is
  positive value.

  Now consider an example with an odd power:
  \[(−2)^5=(−2)(−2)(−2)(−2)(−2)=−32\]
  Since we are multiplying an odd number of negative values the result is
  negative value.

  The parentheses in the above are important! In general:
  \[(-a)^n\ne-a^n\]
  This is because the exponent on the RHS is more binding (i.e., has higher
  precedence) than the minus sign. The expression on the LHS overrides this
  precedence. For example:
  \[(-2)^4=16\]
  but:
  \[-2^4=-(2^4)=-16\]
  Evaluate the following expressions:
  \begin{enumerate}
  \item $(-3)^2=$
  \item $(-3)^3=$
  \item $(-3)^0=$
  \item $(-3)^1=$
  \item $-3^0=$
  \item $-3^1=$
  \item $-3^2=$
  \item $-3^3=$
  \end{enumerate}

\item When we multiply exponential expressions with common bases, we add their
  exponents:
  \[a^na^m=a^{n+m}\]
  You can think of this as:
  \[(a\cdot a\cdot a\cdots a)(a\cdot a\cdot a\cdots a)\]
  where the first parentheses contain $n$ $a$'s and the second parentheses
  contain $m$ $a$'s for a total of $n+m$ $a$'s. For example:
  \[2^2\cdot2^3=2^{2+3}=2^5=32\]
  Note these special cases:
  \begin{itemize}
  \item $a^na^0=a^{n+0}=a^n$
  \item $a^na=a^na^1=a^{n+1}$
  \end{itemize}
  If there is more than two factors with a common base then combine all their
  exponents. For example:
  \[2^2\cdot2^5\cdot2^3\cdot2^2=2^{2+5+3+2}=2^{12}\]

  Simplify the following expressions. Leave your answers in exponent form:
  \begin{enumerate}
  \item $3^2\cdot3^3=$
  \item $7^3\cdot7^5=$
  \item $11\cdot11^5=$
  \item $0^2\cdot0^3=$
  \item $1^2\cdot1^3=$
  \item $5^2\cdot5^4\cdot5^3=$
  \item $13^{10}\cdot13^{10}\cdot13^5=$
  \item $2^3\cdot2\cdot2^0\cdot2^2=$
  \end{enumerate}

\item Sometimes there are other factors between the factors with common bases.
  For example:
  \[2^2\cdot 3^2\cdot2^3\]
  But remember, multiplication can be done in any order so we are free to
  rearrange the factors and then combine exponents:
  \[2^2\cdot 3^2\cdot2^3=2^2\cdot2^3\cdot3^2=2^{2+3}\cdot3^2=2^5\cdot3^2\]

  Simplify the following expressions. Leave your answers in exponent form:
  \begin{enumerate}
  \item $5^3\cdot7^2\cdot5^2=$
  \item $11^2\cdot13\cdot11^5=$
  \item $3\cdot5^2\cdot3=$
  \item $7^4\cdot17\cdot7^3$
  \item $2^2\cdot3^2\cdot2^5\cdot3\cdot2\cdot5^3=$
  \end{enumerate}

\item The next exponent rule is as follows:
  \[(a^n)^m=a^{nm}\]
  In this case, we multiply the exponents. You can think of this as:
  \[a^n\cdot a^n\cdot a^n\cdots a^n\]
  a total of $n$ times, where each $a^n$ contains $n$ $a$'s, for a total of
  $nm$ $a$'s. For example:
  \[(2^3)^2=2^{3\cdot2}=2^6\]
  Make sure that you can distinguish between these two rules:
  \[a^na^m=a^{n+m}\]
  \[(a^n)^m=a^{nm}\]
  Mixing up these two rules results in lots of algebra errors!

  Simplify the following expressions. Leave your answers in exponent form:
  \begin{enumerate}
  \item $(3^4)^2=$
  \item $(5^2)^3=$
  \item $(7^1)^2=$
  \item $(2^0)^1=$
  \item $(11^1)^0=$
  \item $(10^0)^0=$
  \item $(0^2)^3=$
  \item $(1^2)^3=$
  \end{enumerate}

\item Now let's look at the exponent rules for different bases. The first rule
  is as follows:
  \[(ab)^n=a^nb^n\]
  In other words, the exponent needs to be applied to \emph{every} factor
  inside the parens. You can think of this as:
  \[(ab)^n=(ab)(ab)(ab)\cdots(ab)\]
  $n$ times. Then, since multiplication can be done in any order, we group all
  $n$ $a$'a and all $n$ $b$'s:
  \[(ab)^n=(a\cdot a\cdot a\cdots a)(b\cdot b\cdot b\cdots b)=a^nb^n\]
  For example:
  \[(2\cdot3)^2=2^2\cdot3^2\]
  and:
  \[(2\cdot3\cdot5)^2=2^2\cdot3^2\cdot5^2\]
  Note that $(ab)^2$ is \emph{very} different from $(a+b)^2$. Many student mix
  up this rule and try to say $(a+b)^2=a^2+b^2$; however, this is very wrong
  --- you \emph{cannot} distribute an exponent across addition! But you can
  distribute it across multiplication.

  Simplify the following expressions. Leave your answers in exponent form:
  \begin{enumerate}
  \item $(5\cdot7)^2=$
  \item $(5\cdot(-7))^2=$
  \item $(3\cdot11)^5=$
  \item $((-3)\cdot11)^5=$
  \item $(11\cdot13)^0=$
  \item $(2\cdot17)^1=$
  \item $(5\cdot0)^2=$
  \item $(11\cdot19\cdot2)^3=$
  \item $(11\cdot(-19)\cdot2)^2=$
  \item $(11\cdot(-19)\cdot2)^3=$
  \end{enumerate}
  
\item Sometimes you might have two factors that look like they have the same
  base, but one is negative and one is positive. If you remember that:
  \[(-a)=(-1)a\]
  then you can combine the previous rules to simplify. For example:
  \[(-2)^2\cdot2^3=((-1)\cdot2)^2\cdot2^3=(-1)^2\cdot2^2\cdot2^3=
  1\cdot2^{2+3}=2^5\]
  and:
  \[(-2)^3\cdot2^3=((-1)\cdot2)^3\cdot2^3=(-1)^3\cdot2^2\cdot2^3=
  (-1)\cdot2^{2+3}=-2^5\]
  Note that the evenness or oddness of the exponent will make a difference in
  the final sign.

  Simplify the following expressions. Leave your answers in exponent form:
  \begin{enumerate}
  \item $((-5)\cdot7)^5=$
  \item $(5\cdot(-7))^5=$
  \item $((-5)\cdot7)^4=$
  \item $(5\cdot(-7))^4=$
  \item $((-5)\cdot(-7))^3=$
  \item $((-5)\cdot(-7))^2=$
  \item $((-5)\cdot7)^0=$
  \item $(5\cdot(-7))^1=$
  \end{enumerate}

\item Many problems require you to apply multiple rules, one at a time. One
  common pattern that you should know how to handle is something like this:
  \[(2\cdot3^2)^4\]
  Note that one of the factors is an exponential expression itself. So the
  exponent of $4$ needs to be applied to both $2$ and $3^2$ first:
  \[(2\cdot3^2)^4=2^4\cdot(3^2)^4=2^4\cdot3^8\]
  Simplify the following expressions. Leave your answers in exponent form:
  \begin{enumerate}
  \item $(2^2\cdot3\cdot5^3)^2=$
  \item $(2^2\cdot(-3)\cdot5^3)^2=$
  \item $(2^2\cdot(-3)\cdot5^3)^3=$
  \item $(2^2\cdot7^3)^2(5^3\cdot7)^2=$
  \item $((2^2\cdot(-11))^2((-2)\cdot5)^3)^3=$
  \end{enumerate}

\item The final two rules deal with division. The first rule is for a common
  base:
  \[\frac{a^n}{a^m}=a^{n-m}\]
  For now, we will assume that $n\ge m$; we will deal with $n<m$ in the next
  lesson. You can think of this a $n$ $a$'s in the numerator and $m$ $a$'s in
  the denominator, so the $m$ $a$'s below will all cancel, leaving $n-m$ $a$'s
  on top. For example:
  \[\frac{a^3}{a^2}=a^{3-1}=a^1=a\]
  Note that the technique of \emph{cancelling} factors in the numerator and
  denominator are just a consequence of this above rule:
  \[\frac{a^n}{a^n}=a^{n-n}=a^0=1\]
  Simplify the following expressions. Leave your answers in exponent form:
  \begin{enumerate}
  \item $\frac{3^4}{3^2}=$
  \item $\frac{5^3}{5^3}=$
  \item $\frac{7^1}{5^1}=$
  \item $\frac{2^2}{2^0}=$
  \item $\frac{3^2}{3}=$
  \item $\frac{11}{11}=$
  \item $\frac{13}{13^0}=$
  \item $\frac{(-3)^4}{3^2}=$
  \item $\frac{(-3)^3}{3^2}=$
  \item $\frac{5^9}{(-5)^5}=$
  \end{enumerate}

\item The next and final rule is for different bases:
  \[\left(\frac{a}{b}\right)^n=\frac{a^n}{b^n}\]
  Simplify the following expressions. Leave your answers in exponent form:
  \begin{enumerate}
    \item $\left(\frac{3}{2}\right)^3$
    \item $\left(\frac{3^2}{2}\right)^3$
    \item $\left(\frac{3^2}{2^4}\right)^3$
    \item $\left(\frac{(-3)}{2^4}\right)^2$
    \item $\left(\frac{(-3)}{2^4}\right)^3$
  \end{enumerate}

\item When composite numbers are involved, determine their prime
  factorizations first --- this may result in some unexpected cancelling.

  Simplify the following expression by putting everything in prime factored
  form first. Leave your answers in exponent form:
  \[\frac{81\cdot12}{2\cdot30}\]

\item Simplify the following expression. Leave the answer in exponent form:
  \[-4((-5)\cdot2^6)^2=\]

\item Simplify the following expression. Leave the answer in exponent form:
  \[\frac{2\cdot(-3)^2(-2)^3\cdot5}{2^2\cdot3}=\]

\end{enumerate}

\end{document}
