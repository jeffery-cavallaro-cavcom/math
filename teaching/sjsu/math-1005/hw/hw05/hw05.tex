\documentclass[letterpaper,12pt,fleqn]{article}
\usepackage{matharticle}
\usepackage{mathtools}
\pagestyle{plain}
\begin{document}

\begin{center}
  \Large Math-1005a Homework \#5
  
  Decimals and Percents
\end{center}

\vspace{0.5in}

\underline{Problems}

\begin{enumerate}
\item Normally you will be asked to supply your answers to math problems
  using exact, non-decimal values. This includes rational numbers, square
  roots, $\pi$, etc. However there are times when a decimal value is more
  appropriate. The main case is an answer to a word problem. For example,
  saying that something is $\frac{131}{16}$ feet away is not as meaningful as
  saying that it is about $8.2$ feet away.

  To convert from a rational number to a decimal number, simply divide the
  numerator by the denominator. Although you can use the long division that you
  learned in elementary school, it is much easier these days to just use a
  calculator when the numbers aren't nice.

  Furthermore, don't go wild on decimal digits. For example, saying that
  something is $1.2456789128273$ feet long is silly. Instead, round to a
  reasonable number of significant figures (usually 2 or 3). Most problems will
  state how many decimal digits are required.

  Convert the following rational numbers to decimal numbers. Round your answers
  to at most four decimal places:
  \begin{enumerate}
  \item $\frac{1}{2}$

  \item $-\frac{3}{8}$

  \item $\frac{1}{9}$
    
  \item $-\frac{135}{116}$

  \item $\frac{38}{178}$
  \end{enumerate}

\item When converting a decimal number to rational form, determine how many
  places the decimal point must be moved to the right and then multiply and
  divide by that power of 10. For example, 1.234 requires the decimal point to
  be moved to the right by 3 places, so: 
  \[1.234=1.234\cdot\frac{1000}{1000}=\frac{1234}{1000}=\frac{617}{500}\]
  Remember that whole numbers are already rational and thus don't require any
  conversion.

  Convert the following decimal numbers to rational numbers. Your answers
  should be reduced (improper) fractions or whole numbers - not mixed numbers:
  \begin{enumerate}
  \item $5.2$
  \item $4$
  \item $0.89$
  \item $38.7$
  \item $3.14$
  \end{enumerate}

\item The percent sign (\%) is a shorthand for "divide by 100". When the
  percent value is a whole number, it can be converted into a rational number
  by dividing by 100. For example: 
  \[12\%=\frac{12}{100}=\frac{3}{25}\]
  and:
  \[120\%=\frac{120}{100}=\frac{6}{5}\]
  Evaluate the following expressions. Your answers should be reduced (improper)
  fractions or whole numbers - not mixed numbers:
  \begin{enumerate}
  \item $17\%$
  \item $-130\%$
  \item $98\%$
  \item $-50\%$
  \item $112\%$
  \end{enumerate}

\item Since ``percent'' means divide by 100, to convert from percent to decimal
  just move the decimal point over two places. For example: 
  \[12\%=0.12\]
  and:
  \[123.4\%=1.234\]
  and:
  \[0.01\%=0.0001\]
  Convert the following percents to decimal numbers:
  \begin{enumerate}
  \item $66\%$
  \item $645\%$
  \item $-58.7\%$
  \item $0\%$
  \item $-0.08\%$
  \end{enumerate}

\item When comparing values in different forms, convert to a common form and
  then compare.

  For each of the following problems, determine whether the first value is less
  than, equal to, or greater than the second value: 
  \begin{enumerate}
  \item $52\%\,\framebox(20,20){}\,1$
  \item $3.4\,\framebox(20,20){}\,34\%$
  \item $90\%\,\framebox(20,20){}\,\frac{9}{10}$
  \item $0.75\%\,\framebox(20,20){}\,\frac{7}{8}$
  \item $11\%\,\framebox(20,20){}\,-11\%$
  \end{enumerate}

\item All percentage-related problems at some point come down to the following
  statement:
  \[a\%\ \mbox{of}\ b\ \mbox{is}\ c\]
  Note that this is the same as the equation:
  \[\frac{a}{100}\cdot b=c\]
  You are usually given two of the values and your job is to find the third.
  When $a$ and $b$ are given then multiply to find $c$. For example, $50\%$ of
  $25$ is:
  \[\frac{50}{100}\cdot25=12.5\]
  When either $a$ or $b$ is not known then divide $c$ by the known value to find
  the unknown value. For example, to find out what percent $2$ is of $16$:
  \[\frac{2}{16}=0.125=12.5\%\]
  Similarly, to determine that $2$ is $12.5\%$ of what number:
  \[\frac{2}{12.5\%}=\frac{2}{0.125}=16\]
  Fill in the blanks with the appropriate numbers. Round all decimal numbers to
  four decimal points:
  \begin{enumerate}
  \item $13\%$ of $200$ is \rule{1in}{0.1mm}
  \item $9\%$ of \rule{1in}{0.1mm} is $36$
  \item \rule{1in}{0.1mm}$\%$ of $150$ is $38.125$
  \item $63.1\%$ of $270$ is \rule{1in}{0.1mm}
  \item $98.3\%$ of \rule{1in}{0.1mm} is $322.5$
  \item \rule{1in}{0.1mm}$\%$ of $34$ is $2.89$
  \item $110.7\%$ of $\frac{110}{7}$ is \rule{1in}{0.1mm}
  \item $80.9\%$ of \rule{1in}{0.1mm} is $\frac{400}{3}$
  \item \rule{1in}{0.1mm}$\%$ of $363$ is $275.88$
  \end{enumerate}

\item Aja, Nella, and Cai are buying sandwiches at Lee's Sandwich Shop. Each
  sandwich normally sells for $\$9.00$; however, each girl has a discount
  coupon. Aja's coupon gives her $\$2.25$ off the normal price. Bella`s coupon
  gives her a $25\%$ discount. Cai's coupon lets her save $\frac{1}{3}$ of the
  normal price.
  \begin{enumerate}
  \item How much does each girl pay for her sandwich? Your answers should be
    in dollars and cents (i.e., rounded to 2 decimal places).
  \item Who pays the most for their sandwich?
  \end{enumerate}

\item Sometimes we need to take a percentage/fraction of a percentage/fraction
  of something (two steps). Just multiply everything. For example, $10\%$ of
  $50\%$ of $75$ is: 
  \[10\%\cdot50\%\cdot75=0.1\cdot0.5\cdot75=3.75\]
  Compute $40\%$ of $\frac{1}{9}$ of $4$.

\item The city of Fremont has a lake in its Central Park called Lake Elizabeth.
  This summer, the city estimated the following breakdown of the bird
  population around the lake:

  \begin{figure}[h]
    \setlength{\leftskip}{1in}
    \begin{tabular}{lc}
      ducks & $40\%$ \\
      geese & $20\%$ \\
      cormorants & $10\%$ \\
      herons & $30\%$
    \end{tabular}
  \end{figure}
  
  What percentage of the birds that are not geese are herons? Round your answer
  to one decimal place.

\item A common problem is calculating how much of something after a percent
  increase. Suppose that you invest $\$1000$ (called the principal) at $5\%$
  simple interest per year. The interest after one year would be:
  \[\$1000\cdot0.05=\$50\]
  So to determine how much you have after one year your would add the interest
  to the original principal:
  \[1000+1000\cdot0.05=1000(1+0.05)=1000(1.05)=1050\]
  for a total of $\$1050$.

  Note the $\left(1+\frac{p}{100}\right)$ factor; you can use this as a
  shortcut for percent growth problems.

  You buy a one-year CD (certificate of deposit) for $\$13000$ that pays
  $3.5\%$ interest. How much do you collect when you cash it in at the
  end of the year?

\item Percent decrease problems are similar to percent increase problems,
  except this time you subtract the percentage from the original and the key
  factor is $\left(1−\frac{p}{100}\right)$.

  The decrease in population in rural cities and towns has always been a
  problem as the children of residents seek better jobs in the city.
  Smallville, USA had a population of $1842$ people at the start of $2016$, but
  by the end of the year the population had dropped $6\%$. What was the
  population of Smallville at the start of 2017?

  Remember, people don't come in fractions, so round to a whole number.

\item Sometimes we have a situation that changes from time to time. For
  example, suppose that you have an $\$1000$ investment that pays $10\%$ in
  the first year and then $5\%$ in the second year. After the first year the
  value of your investment would be:
  \[\$1000(1+0.10)=\$1000(1.1)=\$1100\]
  The important thing to note is that this value is the starting value for the
  next year, so after year two you would have:
  \[\$1100(1+0.05)=\$1100(1.05)=\$1155\]
  As a short cut, note that this is the same as:
  \[\$1000(1+0.10)(1+0.05)=\$1000(1.1)(1.05)=\$1155\]
  So the multiple growth and decrease factors can be multiplied together to
  obtain the total growth/decrease factor.

  Suppose that you decide to play the stock market, so you invest $\$60000$ in
  Google, Inc. During the first month, Google stock increased $2\%$ in value;
  however, during the second month the stock decreased by $2\%$. What is the
  value of your investment after the second month?

  (Hint: the answer is not $\$60000$)

\item An important application of percentages in lab work and manufacturing is
  percent error. This is the amount that an actual measured value deviates from
  an estimated or required value and is calculated as follows:
  \[\%\mbox{error}=\frac{\mbox{desired}-\mbox{actual}}{\mbox{desired}}\cdot100\]
  Note that the percent error can be positive or negative depending on whether
  the actual value is less than or greater than the desired value.

  You are a quality control officer for a manufacturing firm that makes parts
  for the aircraft industry. Your plant is currently working on a bolt that is
  used in an aircraft wing. The design engineers specify that the bolt must be
  $4$ cm long with a percent error of $0.05\%$. You receive $5$ parts from the
  line and test them for compliance. Compute the percent error (to at most
  four decimal places) for each part and specify whether each part passes or
  fails inspection: 

  \begin{tabular}{|c|c|c|c|}
    \hline
    required (cm) & measured(cm) & $\%$ error & pass/fail \\
    \hline
    $4$ & $3.9984$ & & \\
    \hline
    $4$ & $4.0024$ & & \\
    \hline
    $4$ & $3.9976$ & & \\
    \hline
    $4$ & $4.0004$ & & \\
    \hline
    $4$ & $3.9992$ & & \\
    \hline
  \end{tabular}
\end{enumerate}

\end{document}
