\documentclass[letterpaper,12pt,fleqn]{article}
\usepackage{matharticle}
\pagestyle{plain}
\begin{document}

\begin{center}
  \large
  Math-71 Sections 9, 11, 12

  \Large
  Homework \#5

  \large
  \textbf{Due: 3/12/2019 5:45pm}
\end{center}

\subsection*{Reading}

Read sections 8.4 and 8.5

\subsection*{Problem}

You are a leader for your local Girl Scout troop and it has fallen upon you to plan next year's cookie sales.  It has become
common to adjust the price of a box of cookies to maximize profits based on the affluence of the community.  From past years,
you know that when the price was \$5.00 per box about 10,000 boxes were sold.  When the price was raised to \$6.00 per box,
only 7500 boxes were sold.  Assume that the demand function \(n(p)\) is linear.  The factory that makes the cookies reports
that the fixed costs are \$10,000 and the variable costs are \$2.00 per box.
\begin{enumerate}
\item At what sales price will your troop maximize its profits?
\item At that price, how many boxes is your troop projected to sell?
\item What is the expected profit?
\end{enumerate}

\end{document}
