\documentclass[letterpaper,12pt,fleqn]{article}
\usepackage{matharticle}
\pagestyle{plain}
\begin{document}

\begin{center}
  \large
  Math-71 Sections 9, 11, 12

  \Large
  Homework \#4

  \large
  \textbf{Due: 3/5/2019 5:45pm}
\end{center}

\subsection*{Reading}

Read section 7.7

\subsection*{Problem}

Your are the finance manager for a company that just had a great year.  Last year's income statement and this year's
expectations indicate that the company has a surplus of cash.  You decide to invest \$100,000 of this cash in a 5 year CD that
compounds monthly.  The total amount of the investment after the 5 years is given by:
\[A(r)=100000\left(1+\frac{r}{12}\right)^{60}\]
where \(r\) is the annual interest rate. Assuming that the interest rate is 3\% (\(r=0.03\)):
\begin{enumerate}
\item What is the total amount of the investment after 5 years?
\item How fast is the amount growing with respect to \(r\), in dollars per percent?
\end{enumerate}

\end{document}
