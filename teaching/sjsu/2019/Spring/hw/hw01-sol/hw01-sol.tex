\documentclass[letterpaper,12pt,fleqn]{article}
\usepackage{matharticle}
\pagestyle{plain}
\begin{document}

\begin{center}
  \large
  Math-71 Sections 9, 11, 12

  \Large
  Homework \#1 Solutions
\end{center}

\subsection*{Problem}

You are a business analyst for the city of Lakeside, Calfornia.  Lakeside is famous for its huge lake, which unfortunately
has become polluted with carbon compounds over the years by a nearby factory.  The factory is now closed, so the city
council wants to clean up the lake.  They have made you the head of the project.  You contact the EPA and find out three
important things:
\begin{enumerate}
\item The lake can be cleaned up to what is considered to be a naturally-occurring level of the carbon pollutants
  by removing \(80\%\) of the pollutants.
\item The cost, in \emph{millions} of dollars, to reduce the pollutants in the lake to a given percentage \(p\) is estimated
  by the following formula (function):
  \[C(p)=\frac{10p}{100-p}\]
\item EPA standards say that cleaning up \(75\%\) of the pollutants is good enough, because the extra \(5\%\) doesn't do any
  significant harm.
\end{enumerate}
You decide that the best way to raise the money for this project is to hold a special election for a local ballot initiative to
assess a special charge on next year's property tax bill.  You know that the population of Lakeside is 100,000 people, of
which \(80\%\) are property owners.
\begin{enumerate}[label={(\alph*)}]
\item Assuming that the property tax burden is to be distributed equally, what will be the resulting assessment on each
  property tax bill under the federal requirements?

  \bigskip

  First, determine the total cleanup bill:
  \[C(75)=\frac{10(75)}{100-75}=\frac{750}{25}=30\]
  So the total cost is \(\$30\) million.  Now, divide that amount by the number of property owners to get the per-bill
  assessment.  To determine the number of property owners (hence property bills):
  \[\frac{80}{100}\cdot100,000=80,000\]
  Now distribute the total cost equally among the property owners:
  \[\frac{30,000,000}{80,000}=375\]
  And so the assessment per property tax bill under the federal guidelines is \(\$375\).
  
  \bigskip

\item Governor Newsom visits the lake with a state EPA representative.  They tell you that California leads the nation in
  environmental policy, so he says that the state requirement is to bring the pollutant level all the way to the natural
  level.  What is the resulting property tax bill assessment now?

  \[C(80)=\frac{10(80)}{100-80}=\frac{800}{20}=40\]
  \[\frac{40,000,000}{80,000}=500\]
  
  And so the assessment per property tax bill under the state guidelines increases to \(\$500\).

  \bigskip

\item Some members from the group \emph{Save the Fish!} protest outside your office.  They claim that the lake is home to a
  rare species of fish that is being harmed by the pollutants.  They demand that you decrease the pollutant level by \(95\%\)
  and if you don't then they are going to sue you in federal court under the endangered species act.  What would be the
  resulting property bill assessment for that level of cleanup?

  \[C(95)=\frac{10(95)}{100-95}=\frac{950}{5}=190\]
  \[\frac{190,000,000}{80,000}=2375\]
  
  And so the assessment per property tax bill under the group's demand increases to \(\$2375\).

  \bigskip
  
\item What does the formula tell you about removing \emph{all} of the pollutants from the lake and why do you think this
  might be so?

  \bigskip

  Since \(\displaystyle\lim_{p\to100}C(p)=\infty\), this means that \(100\%\) cleanup is not possible.  This is probably because
  natural processes within the lake's ecosystem attempt to push the amount of pollutants in the lake to the natural level.
  
  \bigskip
  
\item Over what values of percentage \(p\), expressed in interval notation, is the function represented by the formula
  continuous?

  Note that \(C(p)\) is a rational function with a single vertical asymptote at \(p=100\) and no holes.  Thus, \(C(p)\) is
  continuous for all values of \(p<100\).  Furthermore, we have a one-sided limit at \(p=0\), which can be determined by
  substitution:
  \[\lim_{p\to0^+}C(p)=C(0)=0\]
  Thus, \(C(p)\) is considered continuous at \(p=0\).

  So, the interval of continuity is: \([0,\infty)\)
\end{enumerate}

\end{document}
