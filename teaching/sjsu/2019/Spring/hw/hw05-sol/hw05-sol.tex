\documentclass[letterpaper,12pt,fleqn]{article}
\usepackage{matharticle}
\usepackage{siunitx}
\pagestyle{plain}
\begin{document}

\begin{center}
  \large
  Math-71 Sections 9, 11, 12

  \Large
  Homework \#5 Solutions
\end{center}

\subsection*{Problem}

You are a leader for your local Girl Scout troop and it has fallen upon you to plan next year's cookie sales.  It has become
common to adjust the price of a box of cookies to maximize profits based on the affluence of the community.  From past years,
you know that when the price was \$5.00 per box about 10,000 boxes were sold.  When the price was raised to \$6.00 per box,
only 7500 boxes were sold.  Assume that the demand function \(n(p)\) is linear.  The factory that makes the cookies reports
that the fixed costs are \$10,000 and the variable costs are \$2.00 per box.
\begin{enumerate}
\item At what sales price will your troop maximize its profits?

  We start by recognizing that we what a profit function in terms of price: \(P(p)\).  Once we have this, we can calculate
  \(P'(p)\) and find critical points.  First of all, we know that:
  \[P(p)=R(p)-C(p)\]
  where \(R(p)\) is a revenue function and \(C(p)\) is a cost function.

  Let's start with \(R(p)\).  We also know that revenue is quantity (\(n\))times price (\(p\)):
  \[R(p)=np\]
  However, \(n\) is actually a function \(p\) via the demand function:
  \[R(p)=n(p)\cdotp\]
  So we need the construct the demand function.

  We are told to assume that the demand function is linear, and we are given two points:
  \begin{align*}
    n(\$5) &= \SI{10000}{boxes} \\
    n(\$6) &= \SI{7500}{boxes}
  \end{align*}
  This is sufficient information to construct a line:
  \begin{gather*}
    m=\frac{10000-7500}{5-6}=\SI{-2500}{boxes/\$} \\
    n-10000=-2500(p-5) \\
    n(p)=-2500p+22500
  \end{gather*}

  We can now construct the revenue function:
  \[R(p)=p\cdot n(p)=p(-2500p+22500)=-2500p^2+22500p\]

  Now onto the cost function.  We are given the fixed and variable costs, so:
  \begin{align*}
    C(p) &= 10000+2n(p) \\
    &= 10000+2(-2500p+22500) \\
    &= 10000-5000p+45000 \\
    C(p) &= 55000-5000p
  \end{align*}

  And so, the final profit function is:
  \begin{align*}
    P(p) &= R(p)-C(p) \\
    &= (-2500p^2+22500p)-(55000-5000p) \\
    P(p) &= -2500p^2+27500p-55000
  \end{align*}

  We can no differentiate with respect to \(p\):
  \[P'(p)=-5000p+27500\]

  To find all critical points we set this to 0:
  \begin{gather*}
    0=-5000p+27500 \\
    5000p=27500 \\
    p=\$5.50
  \end{gather*}

  Therefore, to maximize profits, boxes should be sold at \$5.50 per box.
  
\item At that price, how many boxes is your troop projected to sell?
  \[n(\$5.50)=-2500(5.50)+22500=\SI{8750}{boxes}\]
  
\item What is the expected profit?
  \[P(\$5.50)=-2500(5.50)^2+27500(5.50)-55000=\$20625\]
\end{enumerate}

\end{document}
