\documentclass[letterpaper,12pt,fleqn]{article}
\usepackage{matharticle}
\usepackage{tikz}
\pagestyle{plain}
\begin{document}

\begin{center}
  \large
  Math-71 Sections 9, 11, 12

  \Large
  Homework \#7 Solutions
\end{center}

\subsection*{Reading}

Read sections 10.1 and 10.2.

\subsection*{Problems}

\begin{enumerate}
\item Sketch the graph \(y=-2\left(\frac{3}{2}-e^{3-x}\right)\) by:
  \begin{enumerate}
  \item Performing the necessary algebra so that the function is in the proper form (i.e., the transformations are in the
    proper order).
    \[y=2e^{-(x-3)}-3\]
  \item Listing the transformations in the order that they are to be applied.
    \begin{enumerate}[label={\arabic*)},start=0]
    \item Basic graph: \(y=e^x\).
    \item Right by 3.
    \item Horizontal reflection.
    \item Vertical scale by 2.
    \item Down by 3.
    \end{enumerate}
  \item Marking the key point and horizontal asymptote.
    \begin{enumerate}[label={\arabic*)},start=0]
    \item \((0,1)\)
    \item \((3,1)\)
    \item \((3,1)\)
    \item \((3,2)\)
    \item \((3,-1)\)
    \end{enumerate}
    KP: \((3,-1)\) \\
    HA: \(y=-3\)
  \item Calculating and marking all existing intercepts.

    \(x\)-intercept:
    \begin{gather*}
      0=2e^{-(x-3)}-3 \\
      2e^{-(x-3)}=3 \\
      e^{-(x-3)}=\frac{3}{2} \\
      -(x-3)=\ln\left(\frac{3}{2}\right) \\
      x-3=-\ln\left(\frac{3}{2}\right) \\
      x=3-\ln\left(\frac{3}{2}\right) \\
      x\approx2.6 \\
      \\
      (2.6,0)
    \end{gather*}
    \(x\)-intercept:
    \begin{gather*}
      y=2e^{-(0-3)}-3=2e^3-3=37.2 \\
      \\
      (0,37.2)
    \end{gather*}
  \item Sketching the final graph.

    \begin{tikzpicture}
      \draw [<->] (-5,0) to (5,0);
      \draw [<->] (0,-5) to (0,5);
      \draw [dashed] (-5,-1) to (5,-1);
      \node (K) [draw,circle,fill,scale=0.5] at (2,-1/2) {};
      \node (Y) [draw,circle,fill,scale=0.5] at (0,3) {};
      \node (X) [draw,circle,fill,scale=0.5] at (1.4,0) {};
      \draw [domain=-0.4:5] plot ({\x},{4*e^(-\x)-1});
      \node at (X) [above right] {\((2.6,0)\)};
      \node at (Y) [above right] {\((0,37.2)\)};
      \node at (K) [right] {\((3,-1)\)};
      \node [below] at (-1,-1) {\(y=-3\)};
   \end{tikzpicture}
  \end{enumerate}

\item You open a new savings account on April 1 with an initial \$1000 at a local bank that pays 2\% interest, compounded
  monthly on the last day of the month.  You file the necessary direct deposit paperwork at you job, where you get paid
  monthly on the first of the month, so that \$500 will be auto-deposited into your new account each month, starting with
  your May paycheck.  In June, you withdraw \$250 to pay for a new mobile phone.  In July, you withdraw \$750 to help pay
  for your summer vacation.  There are no other transactions.  What is your account balance on August 2?

  First, calculate the compounding factor:
  \[x=1+\frac{r}{n}=1+\frac{0.02}{12}=1.001666667\]
  Note that there are lots of decimal places here, so store this result into a storage register on your calculator.

  Now, construct the transaction table:

  \begin{tabular}{|c|c|c|c|}
    \hline
    month & deposit & withdrawal & net \\
    \hline
    Apr & 1000 & & 1000 \\
    \hline
    May & 500 & & 500 \\
    \hline
    Jun & 500 & 250 & 250 \\
    \hline
    Jul & 500 & 750 & -250 \\
    \hline
    Aug & 500 & & 500 \\
    \hline
  \end{tabular}

  Next, note that the first deposit compounds over 4 months, so we build the compounding polynomial as follows:
  \[1000x^4+500x^3+250x^2-250x+500=\$2009.60\]
\end{enumerate}

\end{document}
