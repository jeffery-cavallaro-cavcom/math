\documentclass[letterpaper,12pt,fleqn]{article}
\usepackage{matharticle}
\usepackage{siunitx}
\usepackage{cancel}
\pagestyle{plain}
\begin{document}

\begin{center}
  \large
  Math-71 Sections 9, 11, 12

  \Large
  Homework \#2

  \large
  \textbf{Due: 2/12/2019 5:45pm}
\end{center}

\subsection*{Problem}

\textbf{Note: All numerical answers must contain correct units for full credit!}

A man stands on the edge of a \SI{100}{ft} cliff.  He throws a small red ball up in the air at a speed of \SI{10}{ft/s} so
that the ball rises to a peak height, stops, and then falls to the ground at the foot of the cliff.  The equation of motion
for this scenario is:
\[s(t)=100+10t-16t^2\]
where \(t\) is time (in seconds) and \(s\) is the height of the ball (in feet).
\begin{enumerate}
\item Use the definition of the derivative (i.e., the difference quotient from Section 7.3) to determine \(v(t)=s'(t)\),
  which gives the instantaneous velocity of the ball (in ft/sec) at a time \(t\).
  \begin{align*}
    s'(t) &= \lim_{h\to0}\frac{s(t+h)-s(t)}{h} \\
    &= \lim_{h\to0}\frac{\left[100+10(t+h)-16(t+h)^2\right]-(100+10t-16t^2)}{h} \\
    &= \lim_{h\to0}\frac{\left[100+10t+10h-16(t^2+2th+h^2)\right]-100-10t+16t^2}{h} \\
    &= \lim_{h\to0}\frac{\cancel{100}+\cancel{10t}+10h-\cancel{16t^2}-32th-16h^2-\cancel{100}-\cancel{10t}+\cancel{16t^2}}{h} \\
    &= \lim_{h\to0}\frac{10h-32th-16h^2}{h} \\
    &= \lim_{h\to0}\frac{\cancel{h}(10-32t-16h)}{\cancel{h}} \\
    &= \lim_{h\to0}(10-32t-16h) \\
    &= 10-32t-16(0) \\
    &= 10-32t
  \end{align*}
  
\item How fast is the ball moving at the instant it reaches its peak height?

  Note that gravity is pushing against the ball as it goes higher, eventually forcing it to \emph{stop} at its peak height,
  at which point the ball starts falling.  Since the ball is stopped, its speed is \SI{0}{ft/s}.
  
\item How long does it take for the ball to reach its peak height?

  Since we know that the ball is stopped at its peak height, we can use our formula for \(s'(t)\) to see how long it takes:
  \begin{align*}
    0 &= 10-32t \\
    32t &= 10 \\
    t &= \frac{10}{32} = \frac{5}{16}=0.3125
  \end{align*}
  Thus, the ball reaches its peak height after about \SI{0.3}{s}.
  
\item What is the ball's peak height?

  Now that we know how long it takes to reach its peak height, we can plug that time into the equation of motion to determine
  how high the ball travels:
  \begin{align*}
    s\left(\frac{5}{16}\right) &= 100+10\left(\frac{5}{16}\right)-16\left(\frac{5}{16}\right)^2 \\
    &= 100+\frac{50}{16}-\frac{25}{16} \\
    &= \frac{1600}{16}+\frac{50}{16}-\frac{25}{16} \\
    &= \frac{1625}{16} \\
    &= 101.5625
  \end{align*}
  Thus, the ball reaches a maximum height of about \SI{101.6}{ft}.
  
\item How fast is the ball going when it passes the cliff edge on the way down?

  You can determine this answer in one of three ways:
  \begin{enumerate}
  \item If you know physics (or if you have good intuition), then it is clear that the ball is going the same speed at
    which it was initially thrown up, but in the opposite direction, so \SI{10}{ft/s} (down).

  \item A slightly more complicated way that still uses some intuition is to note that the ball takes the same amount of time
    to travel from release point to its peak as it does from its peak to its release point.  Thus, the ball takes:
    \[2\left(\frac{5}{16}\right)=\frac{5}{8}\]
    seconds for the round trip.  We can then plug this into our velocity equation:
    \[s\left(\frac{5}{8}\right)=10-32\left(\frac{5}{8}\right)=10-20=-10\]
    The negative sign indicates that the ball is traveling down.  So once again, \SI{10}{ft/s}.

  \item The most complicated way is to first realize that the ball's height is \SI{100}{ft} when it passes the cliff edge and
    use the equation of motion to determine the travel time:
    \begin{align*}
      \cancel{100} &= \cancel{100}+10t-16t^2 \\
      0 &= 10t-16t^2 \\
      0 &= 2t(5-8t)
    \end{align*}
    This equation has two solutions: \(t=0\), which is the starting point when the ball is tossed up, and \(t=\frac{5}{8}\)
    when it passes the edge again.  Then as in the previous solution, use the velocity equation to determine that the speed
    is \SI{10}{ft/s} (down).
  \end{enumerate}
\end{enumerate}

\end{document}
