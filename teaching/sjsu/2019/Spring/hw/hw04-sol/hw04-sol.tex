\documentclass[letterpaper,12pt,fleqn]{article}
\usepackage{matharticle}
\pagestyle{plain}
\begin{document}

\begin{center}
  \large
  Math-71 Sections 9, 11, 12

  \Large
  Homework \#4 Solutions
\end{center}

\subsection*{Problem}

Your are the finance manager for a company that just had a great year.  Last year's income statement and this year's
expectations indicate that the company has a surplus of cash.  You decide to invest \$100,000 of this cash in a 5 year CD that
compounds monthly.  The total amount of the investment after the 5 years is given by:
\[A(r)=100000\left(1+\frac{r}{12}\right)^{60}\]
where \(r\) is the annual interest rate. Assuming that the interest rate is 3\% (\(r=0.03\)):
\begin{enumerate}
\item What is the total amount of the investment after 5 years?
  \[A(0.03)=100000\left(1+\frac{0.03}{12}\right)^{60}=\$116161.68\]
\item How fast is the amount growing with respect to \(r\), in dollars per percent?

  Getting the correct units on this is tricky.  Here are two methods.  I prefer the second method.

  \begin{enumerate}
    \item Using the chain rule:
      \begin{gather*}
        A'(r)=100000\left[60\left(1+\frac{r}{12}\right)^{59}\right]\left(\frac{1}{12}\right)=
        500000\left(1+\frac{r}{12}\right)^{59} \\
        A'(0.03)=500000\left(1+\frac{0.03}{12}\right)^{59}=\$579359.99\ \text{per percent}/100=579359.99 \$100/percent
      \end{gather*}
      Note the weird units: \(\text{percent}/100\).  This is because the equation for \(A(r)\) needs \(r\) to be a
      fractional number --- i.e., \(\text{percent}/100\).  To get rid of this weird unit, we need to divide the final
      answer by 100:
      \[A'(0.03)=\$5793.60/percent\]

    \item The first method is a bit of a kludge.  A much better way is to adjust the original \(A(r)\) so that it
      accepts whole percent values:
      \[A(r)=100000\left(1+\frac{r/100}{12}\right)^{60}=100000\left(1+\frac{r}{1200}\right)^{60}\]
      An now:
      \begin{gather*}
        A'(r)=100000\left[60\left(1+\frac{r}{1200}\right)^{59}\right]\left(\frac{1}{1200}\right)=
        5000\left(1+\frac{r}{1200}\right)^{59} \\
        A'(0.03)=5000\left(1+\frac{0.03}{1200}\right)^{59}=\$5793.60\ \text{per percent}
      \end{gather*}
  \end{enumerate}
\end{enumerate}

\end{document}
