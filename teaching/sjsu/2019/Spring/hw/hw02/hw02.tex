\documentclass[letterpaper,12pt,fleqn]{article}
\usepackage{matharticle}
\usepackage{siunitx}
\pagestyle{plain}
\begin{document}

\begin{center}
  \large
  Math-71 Sections 9, 11, 12

  \Large
  Homework \#2

  \large
  \textbf{Due: 2/12/2019 5:45pm}
\end{center}

\subsection*{Reading}

\begin{itemize}
\item Read section 7.3.
\end{itemize}

\subsection*{Problem}

A man stands on the edge of a \SI{100}{ft} cliff.  He throws a small red ball up in the air at a speed of \SI{10}{ft/s} so
that the ball rises to a peak height, stops, and then falls to the ground at the foot of the cliff.  The equation of motion
for this scenario is:
\[h(t)=100+10t-16t^2\]
where \(t\) is time (in seconds) and \(h\) is the height of the ball (in feet).
\begin{enumerate}
\item Use the definition of the derivative (i.e., the difference quotient from Section 7.3) to determine \(h'(t)\),
  which gives the instantaneous velocity of the ball (in ft/sec) at a time \(t\).
\item How fast is the ball moving at the instant it reaches its peak height?
\item How long does it take for the ball to reach its peak height?
\item What is the ball's peak height?
\item How fast is the ball going when it passes the cliff edge on the way down?
\end{enumerate}

\end{document}
