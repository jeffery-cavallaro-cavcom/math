\documentclass[letterpaper,12pt,fleqn]{article}
\usepackage{matharticle}
\usepackage{siunitx}
\pagestyle{plain}
\begin{document}

\begin{center}
  \large
  Math-71 Sections 9, 11, 12

  \Large
  Homework \#3

  \large
  \textbf{Due: 2/19/2019 5:45pm}
\end{center}

\subsection*{Reading}

\begin{itemize}
\item Read sections 7.3 and 7.4.
\end{itemize}

\subsection*{Problem}

A thin board is to be rested up against a large rock such that it touches the rock at exactly one point.  Let the ground be the
\(x\) direction and assume that the origin is at the start of the rock.  The surface of the rock follows the function:
\[s(x)=\sqrt{x}\]
where \(s(x)\) is the height of the surface of the rock (in feet) at position \(x\) (also in feet).
\begin{enumerate}
\item Use the definition of the derivative (i.e., the difference quotient from Section 7.3) to determine \(s'(t)\).
\item Assuming that the board intersects with the rock surface at \(x=4\), determine where one end of the board touches the
  ground.
\end{enumerate}

\end{document}
