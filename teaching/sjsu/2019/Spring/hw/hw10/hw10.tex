\documentclass[letterpaper,12pt,fleqn]{article}
\usepackage{matharticle}
\usepackage{tikz}
\renewcommand{\l}{\lambda}
\pagestyle{plain}
\begin{document}

\begin{center}
  \large
  Math-71 Sections 9, 11, 12

  \Large
  Homework \#10

  \large
  \textbf{Due: 4/30/2019 5:45pm}
\end{center}

\subsection*{Reading}

Read section 13.6.

\subsection*{Problem}

You work for a large conglomerate of 100 associated companies.  In order to avoid antitrust issues with the DOJ, you need
to divide the 100 companies into 4 different groups such that:
\begin{enumerate}
\item Companies within the same group cannot do business with each other.
\item Companies in different groups can do business with each other.
\end{enumerate}
What organization of the 100 companies into the 4 groups maximimizes the number of business opportunities?

\begin{enumerate}[label={\alph*)}]
\item Start by labeling the groups \(X\), \(Y\), \(Z\), and \(W\).  Let \(x=\) the number of companies assigned to group \(X\)
  and so on for the other groups.  Construct an equation in \(x\) for the number of business opportunities for a company in
  group \(X\) --- i.e., how many companies can that company do business with?
\item Now build an equation in \(x\) for the total number of business opportunities for all companies in group \(X\).
\item Do likewise for the remaining groups and construct a function \(f(x,y,z,w)\) that gives the total number of business
  opportunities across all the groups.
\item What is the constraint on \(x\), \(y\), \(z\), and \(w\)?
\item Introduce a Lagrange multiplier \(\l\) and determine \(f_x=\l g_x\), where \(g\) is the function constructed from the
  above constraint.
\item Do likewise for \(f_y\), \(f_z\), and \(f_w\), and combine them with the constraint so that you have 5 equations in
  5 unknowns.
\item Use substitution to determine a value for \(\l\).
\item Substitute the value for \(\l\) into the other equations to determine the optimal distribution of companies into the
  4 groups.
\end{enumerate}

\end{document}
