\documentclass[letterpaper,12pt,fleqn]{article}
\usepackage{matharticle}
\pagestyle{plain}
\begin{document}

\begin{center}
  {\Large Math Homework Rules}

  \vspace{0.25in}

  \emph{I reserve the right to deduct 10\% from your score if these rules are not followed.}
\end{center}

\vspace{0.25in}

\begin{enumerate}
\item Written homework must be done in pencil (no pen!) and must be submitted on \(8\times11\)-inch college rule, pad, or
  graph paper.  Do not print out the homework assignment sheet and attempt to cram all of your work onto it.

\item Treat your written Math homework like you would an English paper. In
  particular:
  \begin{enumerate}
  \item It should be neat, organized, and legible. In fact, it is suggested that you start with a rough draft and then make a
    final draft once you are satisfied with your answers.

  \item Do not rip your assignment out of a spiral notebook.
    
  \item Be sure that your name is on the first page, and that all pages are stapled, in order. Do not creatively corner-fold
    the sheets.
    
  \item Problems must be in order.
  \end{enumerate}
  
\item All work must be shown for full credit. Answers with no supporting work receive zero credit.

\item Always present answers in exact form; do not give decimal answers unless the problem specifically asks for answers in
  that form. The presence of decimal answers when not asked for will result in points off.

\item All answers should be presented in a reasonably-simplified form unless otherwise stated.  This means that all
  obvious evaluation, factoring, cancellation, and combining of terms should be performed.  For example, \(1x\) should just
  be \(x\) and \(\frac{4x^2}{2x}\) should be just \(2x\).

\item It is OK to work in teams; however, make sure that the work that you turn in reflects your ability to do the problems.
  Remember, your team will not be able to help you during exams.  Math is a very lonely subject!

\item When factoring a polynomial via inspection, just write down the factoring:
\begin{align*}
x^2+3x+2 &= 0 \\
(x+2)(x+1) &= 0
\end{align*}
Don't draw the little cross and arrange the numbers in the slots like you
learned in high school --- that is OK for your rough draft, but just show me
the factoring in the final draft.

\item When doing algebraic manipulation, each line in your answer should
  embody a single step. Don't combine steps like this:

\begin{tabular}{rcl}
3x+6 & = & 0 \\
-6 & = & -6 \\
\hline
 & 3 & \\
x & = & -2
\end{tabular}

Instead, do the following:
\begin{align*}
3x+6 &= 0 \\
3x &= -6 \\
x &= -2
\end{align*}

\item If I see anything resembling this:
\[\frac{\not{x}+y}{\not{x}} = 1+y\]
or this:
\[(a+b)^2=a^2+b^2\]
anywhere in one of your answers then you get an automatic zero for that problem.

\item I will always try to be available to you when you need my help; however, please do not tell me that you don't understand
  anything and expect me to repeat the previous lectures.  What you need to do is do the reading, try the examples and some
  odd numbered practice problems, then come to me and show me a particular problem that is giving you trouble. Tell me how you
  have tried to approach the problem.
\end{enumerate}

\end{document}
