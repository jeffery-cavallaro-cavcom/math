\documentclass[letterpaper,12pt,fleqn]{article}
\usepackage{matharticle}
\usepackage{tikz}
\pagestyle{plain}
\begin{document}

\begin{center}
  \large
  Math-71 Sections 9, 11, 12

  \Large
  Practice Exam \#3
\end{center}

\vspace{0.5in}

\begin{enumerate}[left=0pt]
\item Consider the follow function of two variables:

  \bigskip

  \begin{center}
    \scalebox{1.25}{\(\displaystyle f(x,y)=\ln\left[\frac{\sqrt{y^2+2}e^{(x^2+y^3+1)}}{4x^2(y^3-2)}\right]\)}
  \end{center}
  
  \bigskip

  Determine the following partials:
  \begin{enumerate}[label={\alph*)}]
  \item \(f_x\)
  \item \(f_y\)
  \item \(f_{xx}\)
  \item \(f_{yx}\)
  \end{enumerate}

\item{\label{pd}} Use the second partial derivative test to find the the \((x,y,z)\) coordinates \emph{and} type of the
  absolute extremum on the following surface:

  \bigskip

  \begin{center}
    \scalebox{1.25}{\(\displaystyle z=5-x^2-2x-y^2+4y\)}
  \end{center}

  \bigskip

\item{\label{lm}} Use the Lagrange multiplier technique to find the \((x,y,z)\) coordinates of the absolute minimum
  point on the surface in problem \ref{pd} given the following constraint:

  \bigskip

  \begin{center}
    \scalebox{1.25}{\(\displaystyle 2x-y+4=0\)}
  \end{center}

  \bigskip

\item Compare the answers in problems \ref{pd} and \ref{lm} and explain why they are either different or the same.

\item Evaluate the following definite integral:

  \bigskip

  \begin{center}
    \scalebox{1.25}{\(\displaystyle\int_{93}^{93}\frac{\sqrt{x+1}}{xe^x}dx\)}
  \end{center}

  \bigskip

\item The Uber IPO will happen soon.  Your work for a hedge fund that is considering purchasing a significant number of Uber
  shares on the first day of trading.  Your main analyst has determined that the change in rate of share purchases (in
  millions of shares per hour) on the market starting from the opening bell (t=0) will follow the model:

  \bigskip

  \begin{center}
    \scalebox{1.25}{\(\displaystyle\frac{dx}{dt}=500e^{-4t}\)}
  \end{center}

  \bigskip

  How many shares are sold in the first 2 hours?
\end{enumerate}

\end{document}
