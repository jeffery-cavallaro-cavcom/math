\documentclass[letterpaper,12pt,fleqn]{article}
\usepackage{matharticle}
\usepackage{tikz}
\pagestyle{plain}
\newcommand{\m}{\mu}
\renewcommand{\o}{\sigma}
\begin{document}

\begin{center}
  \large
  Math-71 Sections 9, 11, 12

  \Large
  Exam \#2
\end{center}

\vspace{0.5in}

Name: \rule{4in}{1pt}

\vspace{0.5in}

This exam is closed book and notes. You may use a scientific calculator; however, no other electronics are allowed.  You may
also use the instructor-provided cheatsheet.  Show all work; there is no credit for guessed answers.  Simplify your answers
unless told otherwise.  In particular, all answers should contain no negative or rational exponents.  All numerical answers
should be in exact form unless you are specifically asked for a decimal value.  

\vspace{0.5in}

\begin{enumerate}[left=0pt]

\item Determine \(f'(x)\).  You do not need to simplify.
  \[f(x)=e^{\sqrt{x^2+1}}\]

  \newpage

\item Let \(f(x)=x^3-3x^2-24x+1\)
  \begin{enumerate}
  \item What is the \(y\)-intercept?

    \vspace{2in}
    
  \item Find all critical points for the first derivative.

    \vspace{3in}
    
  \item Use the second derivative test to determine whether these critical points are minima, maxima, or points of
    inflection.

    \newpage
    
  \item Use the second derivative to find any points of inflection.

    \vspace{3in}
    
  \item Sketch the graph, showing and labeling all extrema, points of inflection, and the \(y\)-intercept.  You do \emph{not}
    need to determine the \(x\)-intercepts.
  \end{enumerate}

  \newpage

\item Let \(y=-2e^{x-1}+3\).
  \begin{enumerate}
  \item List the transformations in the proper order.

    \vspace{3in}
    
  \item What are the coordinates of the final key point?

    \vspace{2in}
    
  \item What is the equation of the final horizontal asymptote?

    \newpage
    
  \item What are the \(x\)-intercepts (if any)?

    \vspace{2in}
    
  \item What are the \(y\)-intercepts (if any)?

    \vspace{2in}

  \item Sketch the graph, showing and labeling the key point, horizontal asymptote, and all intercepts.
  \end{enumerate}

  \newpage

\item You buy a new home for \$500,000 on the first day of the month.  You put down \$50,000 and finance the rest with a
  mortgage at 6\% annual interest compounded monthly on the last day of the month.  Your monthly payments, including principal
  and interest, are \$2500.  Your payments are due on the first of the month, starting next month.  What is your loan balance
  after your third monthly payment?

  \newpage

\item You are testing the duration of certain fuses for a pyrotechnic company.  The manufacturer states that the duration
  (from ignition to explosion) follows a normal distribution as follows, where the mean and standard deviation are expressed
  in seconds:
  \[p(x)=\frac{1}{2\sqrt{2\pi}}e^{\frac{-(t-10)^2}{8}}\]
  \begin{enumerate}
  \item What is the mean of the fuse duration?

    \vspace{1in}
    
  \item What is the standard deviation of the fuse duration?

    \vspace{1in}
    
  \item At what \(t\) value does the corresponding bell curve have its absolute maximum?

    \vspace{1in}
    
  \item At what \(t\) values does the corresponding bell curve have its points of inflection?

    \vspace{1in}
    
  \item What is the probability that a fuse duration will be between 8 and 12 seconds?
  \end{enumerate}

\end{enumerate}

\end{document}
