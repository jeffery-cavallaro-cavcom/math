\documentclass[letterpaper,12pt,fleqn]{article}
\usepackage{matharticle}
\usepackage{tikz}
\pagestyle{plain}
\begin{document}

\begin{center}
  \large
  Math-19 Sections 1

  \Large
  Midterm Exam
\end{center}

\vspace{0.5in}

Name: \rule{4in}{1pt}

\vspace{0.5in}

This exam is closed book and notes. You may use a TI-84 graphing calculator; however, no other electronics are
allowed.  You may also use a double-sided \(8.5\times11\) cheatsheet of your making.  Show all work; there is no
credit for guessed answers.  Simplify your answers unless told otherwise.  In particular, all answers should
contain no negative or rational exponents.  All numerical answers should be in exact form unless you are
specifically asked for a decimal value.

\vspace{0.5in}

\begin{enumerate}[left=0pt]
\item Consider the expression:
  \[-\sqrt{\frac{7.\overline{9}}{2}}\]
  \begin{enumerate}
  \item Without using a calculator, evaluate the expression.

    \vspace{1in}
    
  \item Circle the subsets of the real numbers to which the value belongs.

    \vspace{0.25in}

    \begin{tabular}{p{1in}p{1in}p{1in}p{1in}p{1in}}
      \(\N\) & \(\Z\) & \(\Q\) & \(\R-\Q\) & \(\R\)
    \end{tabular}
  \end{enumerate}

  \bigskip

\item Complete the following:
  \begin{enumerate}
  \item \(\displaystyle\left(x^{\frac{1}{2}}\right)^2=\)
  \item \(\displaystyle\left(x^2\right)^{\frac{1}{2}}=\)
  \item \(\displaystyle\left(x^{\frac{1}{3}}\right)^3=\)
  \item \(\displaystyle\left(x^3\right)^{\frac{1}{3}}=\)
  \end{enumerate}

  \newpage

\item You own a store that sells candy and nuts.  You currently sell peanuts at \$5 per pound and cashews at \$7.50
  per pound.  The cashews aren't selling very well because their price is too high, so you decide to make a
  peanut/cashew mix.  You want to have 10 pounds of this mix to sell at \$6.50 per pound.  How many pounds of
  peanuts and how may pounds of cashews should be in the mix?

  \newpage

\item Consider the following general form equation for a circle:
  \[x^2-6x+y^2+2y+1=0\]
  \begin{enumerate}
  \item Convert the equation to standard form.

    \vspace{4in}
    
  \item What are the center and radius of the circle?
  \end{enumerate}

  \newpage

\item Let \(f(x)=2x^2+3x-1\).  Calculate the following:
  \[\frac{f(x+h)-f(x)}{h}\]

  \vspace{4in}

\item You decide to start your own penny tee business.  You estimate that the fixed costs per month are \$1000 and
  the variable costs are \$2.50 per shirt.  Assuming a linear cost model, how much does it cost to produce
  500 shirts each month?

  \newpage

\item Consider the function:
  \[y=-2\sqrt{x+1}+3\]
  \begin{enumerate}
  \item Starting with the basic function, list the transformations in the order that they should be applied.
    \vspace{2in}
  \item What are the final transformed coordinates of the key point?
    \vspace{2in}
  \item What are the \(x\)-intercepts, if any?
    \newpage
  \item What are the \(y\)-intercepts, if any?
    \vspace{1in}
  \item Sketch the graph.  The transformed key point and all intercepts must be clearly shown and labeled for
    full credit.

    \bigskip

    \begin{tikzpicture}
      \draw [help lines] (-5,0) -- (5,0);
      \draw [help lines] (0,-5) -- (0,5);
    \end{tikzpicture}
    
    \bigskip

  \item What is the domain (in interval notation)?
    \vspace{1in}
  \item What is the range (in interval notation)?
  \end{enumerate}

  \newpage

\item Consider the general form parabolic function:
  \[y=-2x^2+4x+3\]
  \begin{enumerate}
  \item Convert the equation to standard form.
    \vspace{4in}
  \item What are the coordinates of the vertex?
    \vspace{1in}
  \item What are the \(x\)-intercepts, if any?
    \newpage
  \item What are the \(y\)-intercepts, if any?
    \vspace{1in}
  \item Sketch the graph.  The vertex and all intercepts must be clearly shown and labeled for full credit.

    \bigskip

    \begin{tikzpicture}
      \draw [help lines] (-5,0) -- (5,0);
      \draw [help lines] (0,-5) -- (0,5);
    \end{tikzpicture}
    
    \bigskip

  \item What is the domain (in interval notation)?
    \vspace{1in}
  \item What is the range (in interval notation)?
  \end{enumerate}

  \newpage

\item Consider the polynomial:
  \[y=2x^4+3x^3-x\]
  \begin{enumerate}
  \item Completely factor by determining candidate zeros and using long or synthetic division.  There is no credit
    for factoring by grouping.
    \newpage
  \item Sketch the polynomial.  All intercepts must be clearly marked and labeled and the end behavior must be clearly
    visible.

    \bigskip

    \begin{tikzpicture}
      \draw [help lines] (-5,0) -- (5,0);
      \draw [help lines] (0,-5) -- (0,5);
    \end{tikzpicture}
    
    \bigskip

  \item Using your calculator, determine any extrema and mark them on your sketch (if any).
    \vspace{1in}
  \end{enumerate}

  \newpage

\item Consider the rational function:
  \[y=\frac{x^2-4x+3}{x^2-5x+6}\]
  \begin{enumerate}
  \item What are the zeros?
    \vspace{1in}
  \item What are the poles?
    \vspace{1in}
  \item Where are the \(x\)-intercepts (if any)?
    \vspace{1in}
  \item Where are the \(y\)-intercepts (if any)?
    \vspace{1in}
  \item Where are the vertical asymptotes (if any)?
    \vspace{1in}
  \item Where are the horizontal asymptotes (if any)?
    \newpage
  \item What is the end behavior as \(x\to\infty\) and \(x\to-\infty\)?  Be sure to specify direction - i.e., from
    above (\(+\)) or below (\(-\)).
    \vspace{2.5in}
  \item Sketch the graph.  All intercepts, asymptotes, and holes must be clearly marked and labeled for full credit.

    \bigskip

    \begin{tikzpicture}
      \draw [help lines] (-5,0) -- (5,0);
      \draw [help lines] (0,-5) -- (0,5);
    \end{tikzpicture}
    
    \bigskip

  \item Using your calculator, determine any extrema and mark them on your sketch (if any).
  \end{enumerate}
\end{enumerate}

\end{document}
