\documentclass[letterpaper,12pt,fleqn]{article}
\usepackage{matharticle}
\usepackage{tikz}
\pagestyle{plain}
\allowdisplaybreaks
\tikzset{dot node/.style={draw,circle,fill,inner sep=0pt,minimum size=5pt}}

\begin{document}

\begin{center}
  \large
  Math-19 Section 1

  \Large
  Homework \#6 Solutions
\end{center}

\subsection*{Problems}

Consider the transformed function:
\[y=2f\left(-\frac{1}{3}(x-1)\right)+1\]
For each of the following choices of \(f(x)\), determine the final coordinates of the key point, position of all
asymptotes, and any \(x\) and \(y\) intercepts and then sketch the final graph.

Here are the transformations in the order that they are applied and how the affect the key point:

\begin{tabular}{l|c|c}
  TRANSFORMATION & \(e^x\) & \(\ln(x)\) \\
  \hline
  Start with basic function & \((0,1)\) & \((1,0)\) \\
  Right \(1\) & \((1,1)\) & \((2,0)\) \\
  H scale \(\frac{1}{3}\) & \((1,1)\) & \((4,0)\) \\
  H reflect & \((1,1)\) & \((-2,0)\) \\
  V scale \(2\) & \((1,2)\) & \((-2,0)\) \\
  Up \(1\) & \((1,3)\) & \((-2,1)\)
\end{tabular}

\begin{enumerate}[left=0pt]
\item \(f(x)=e^x\)

  The vertical translation moves the HA up 1.  The \(x\)-intercept is found as follows:
  \begin{gather*}
    0=2e^{-\frac{1}{3}(x-1)}+1 \\
    e^{-\frac{1}{3}(x-1)}=-\frac{1}{2}
  \end{gather*}
  But the exponential is always greater than 0, as so there is no \(x\)-intercept.  For the \(y\)-intercept:
  \[y=2e^{-\frac{1}{3}(0-1)}+1=2e^{\frac{1}{3}}+1=3.791\]
  So the \(y\)-intercept is at \((0,3.791)\).

  \begin{center}
    \begin{tikzpicture}
      \draw [help lines,<->] (-5,0) -- (5,0);
      \draw [help lines,<->] (0,-5) -- (0,5);
      \draw [help lines,dashed] (-5,1) -- (5,1) node [right] {\(y=1\)};
      \draw [domain=-1:5] plot (\x,{2*exp(-(1/3)*(\x-1))+1});
      \node (A) [dot node] at (1,3) {};
      \node [below] at (A) {\((1,3)\)};
      \node (B) [dot node] at (0,3.791) {};
      \node [right] at (B) {\((0,2\sqrt[3]{e}+1)\)};
    \end{tikzpicture}
  \end{center}

\item \(f(x)=\ln(x)\)

  The horizontal translation move the VA right 1.  This causes the key point, which is at \((2,0)\) after the
  horizontal translation, to be scale \(3\) times away from the VA.  Thus, the distance of \(1\) is scaled to \(3\)
  and the key point moves to \((4,0)\). The horizontal reflection moves it 3 to the left of the VA to \(-2\).  We
  can test that we have the correct \(x\)-coordinate of the final key point by plugging it in to the horizontal
  transformations and making sure that we get back to the original value of \(1\):
  \[-\frac{1}{3}(-2-1)=-\frac{1}{3}(-3)=1\]
  The \(x\)-intercept is found as follows:
  \begin{gather*}
    0=2\ln\left[-\frac{1}{3}(x-1)\right]+1 \\
    \ln\left[-\frac{1}{3}(x-1)\right]=-\frac{1}{2} \\
    -\frac{1}{3}(x-1)=e^{-\frac{1}{2}} \\
    x-1=-3e^{-\frac{1}{2}} \\
    x=1-\frac{3}{\sqrt{e}} \\
    x=-0.820
  \end{gather*}
  And so the \(x\)-intercept is at \((-0.820,0)\).  For the \(y\)-intercept:
  \[y=2\ln\left[-\frac{1}{3}(0-1)\right]+1=2\ln\left(\frac{1}{3}\right)+1=1-2\ln(3)=-1.197\]
  So the \(y\)-intercept is at \((0,-1.197)\).

  \begin{center}
    \begin{tikzpicture}
      \draw [help lines,<->] (-5,0) -- (5,0);
      \draw [help lines,<->] (0,-5) -- (0,5);
      \draw [help lines,dashed] (1,-5) -- (1,5) node [above] {\(x=1\)};
      \node (A) [dot node] at (-2,1) {};
      \node [below] at (A) {\((-2,1)\)};
      \draw [domain=-5:0.85] plot (\x,{2*ln(-(1/3)*(\x-1))+1});
      \node (B) [dot node] at (-0.820,0) {};
      \node [below left] at (B) {\((1-\frac{3}{\sqrt{e}},0)\)};
      \node (B) [dot node] at (0,-1.197) {};
      \node [below left] at (B) {\((0,1-2\ln(3))\)};
    \end{tikzpicture}
  \end{center}
\end{enumerate}

\end{document}
