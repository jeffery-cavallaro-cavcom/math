\documentclass[letterpaper,12pt,fleqn]{article}
\usepackage{matharticle}
\usepackage{siunitx}
\pagestyle{plain}
\newcommand{\w}{\omega}
\begin{document}

\begin{center}
  \large
  Math-19 Section 1

  \Large
  Homework \#7 Solutions
\end{center}

\subsection*{Reading}

Sections 6.1

\subsection*{Problem}

A blu-ray disk has a diameter of $\SI{12}{cm}$.  The track of recorded information on the disk spirals in from the
outside toward the center.  A blu-ray player must make sure that the track passing under the optical reader stays
at a constant linear speed, so the motor will vary from $\SI{200}{}$ to $\SI{500}{rpm}$, depending on where the
disk is being read.
\begin{enumerate}
\item Which angular speed is used for the part of the track on the outer rim of the disk? Why?

  For constant angular speed, a point on the outer rim is going to have a faster linear speed than a point closer
  to the center. The equation $v=r\w$ tells us this. Thus, the player motor must start slow at the rim and speed up
  as the reader moves along the spiral toward the center. Therefore, the lowest speed of 200 rpm is used when
  reading from near the rim.

\item What is the linear speed of the outer part of the track in cm/s?
  \[v=r\w=\SI{6}{cm}\cdot\frac{\SI{200}{rev}}{\SI{1}{min}}\cdot
  \frac{2\pi}{\SI{1}{rev}}\cdot\frac{\SI{1}{min}}{\SI{60}{sec}}=
  \SI{40\pi}{cm/s}\approx\SI{126}{cm/s}\]
\item What is the distance from the center of the disk to the innermost part of the track?

  Since the linear speed is constant, we use the highest angular speed and solve for the radius. Note that we can
  keep the original units because they will cancel properly.
  \begin{gather*}
    v_{\text{outer}}=v_{\text{inner}} \\
    \SI{6}{cm}\cdot\SI{200}{rpm}=r\cdot\SI{500}{rpm} \\
    r=\SI{6}{cm}\left(\frac{\SI{200}{rpm}}{\SI{500}{rpm}}\right) \\
    r=\SI{2.4}{cm}
  \end{gather*}
\end{enumerate}

\end{document}
