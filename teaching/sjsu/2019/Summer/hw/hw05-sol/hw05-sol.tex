\documentclass[letterpaper,12pt,fleqn]{article}
\usepackage{matharticle}
\usepackage{polynom}
\usepackage{cancel}
\usepackage{tikz}
\pagestyle{plain}
\begin{document}

\begin{center}
  \large
  Math-19 Section 1

  \Large
  Homework \#5 Solutions
\end{center}

\subsection*{Problems}

Consider the rational function:
\[y=\frac{2x^3-3x^2-3x+2}{2x^3+x^2-2x-1}\]

\begin{enumerate}
\item Completely factor the numerator and the denominator and rewrite the function in simplified, factored form.

  Starting with the numerator, first apply the rational roots theorem:
  \begin{gather*}
    a_0=2: \pm1,\pm2 \\
    a_n=2: \pm1,\pm2 \\
    \frac{a_0}{a_n}=\pm1,\pm\frac{1}{2},\pm2
  \end{gather*}
  And now the remainder theorem in order to find our first root:
  \begin{gather*}
    2(1)^3-3(1)^2-3(1)+2=2-3-3+2\ne0 \\
    2(-1)^3-3(-1)^2-3(-1)+2=-2-3+3+2=0
  \end{gather*}
  And so \(x+1\) is a factor.  So do the long division:
  \[\polylongdiv{2x^3-3x^2-3x+2}{x+1}\]
  And so:
  \[2x^3-3x^2-3x+2=(x+1)(2x^2-5x+2)=(x+1)(2x-1)(x-2)\]
  Repeating the process for the denominator:
  \begin{gather*}
    a_0=1: \pm1 \\
    a_n=2: \pm1,\pm2 \\
    \frac{a_0}{a_n}=\pm1,\pm\frac{1}{2} \\
    \\
    2(1)^3+(1)^2-2(1)-1=2+1-2-1=0 \\
  \end{gather*}
  \[\polylongdiv{2x^3+x^2-2x-1}{x-1}\]
  \[2x^3+x^2-2x-1=(x-1)(2x^2+3x+1)=(x-1)(2x+1)(x+1)\]
  And so the factored rational function is:
  \[\frac{\cancel{(x+1)}(2x-1)(x-2)}{(x-1)(2x+1)\cancel{(x+1)}}=\frac{(x-2)(2x-1)}{(x-1)(2x+1)}\]
  Noting that: \(x\ne-1\)
  
\item Do the long division and rewrite the function in quotient/remainder form.
  \[\polylongdiv{2x^3-3x^2-3x+2}{2x^3+x^2-2x-1}\]
  Factoring out a (-1) from the remainder:
  \[1-\frac{4x^2+x-3}{2x^3+x^2-2x-1}=1-\frac{(4x-3)\cancel{(x+1)}}{(x-1)(2x+1)\cancel{(x+1)}}=
  1-\frac{4x-3}{(x-1)(2x+1)}\]
\item Where are the zeros?
  \[x=\frac{1}{2},2\]
\item Where are the poles?
  \[x=-\frac{1}{2},1\]
\item Where are the holes?
  \[x=-1\]
\item Where are the horizontal asymptotes?
  \[y=1\]
\item Where are the vertical asymptotes?
  \[x=-\frac{1}{2},1\]
\item Where is the \(y\)-intercept?
  \[(0,-2)\]
\item What is the end-behavior as \(x\to\infty\)? (be sure to specify above or below)

  The last zero/pole is at \(x=2\), so use \(x=3\) as a test point and plug it in to the remainder portion in
  part (2).  The result is \(1-c\) where \(c>0\) and so \(f(x)\to1^{-}\) (from below).

\item What is the end-behavior as \(x\to-\infty\)? (be sure to specify above or below)

  The first zero/pole is at \(x=-\frac{1}{2}\), so use \(x=-1\) as a test point and plug it in to the remainder
  portion in part (2).  The result is \(1-c\) where \(c<0\) and so \(f(x)\to1^{+}\) (from above).
  
\item Sketch the graph.  All intercepts, asymptotes, and holes must be clearly marked.

  \newcommand{\eq}{(2*(\x)^3-3*(\x)^2-3*(\x)+2)/(2*(\x)^3+(\x)^2-2*(\x)-1)}

  \begin{tikzpicture}
    \draw [<->,help lines] (-5,0) -- (5,0);
    \draw [<->,help lines] (0,-5) -- (0,5);
    \draw [dashed] (-5,1) -- (5,1) node [right] {\(y=1\)};
    \draw [dashed] (-1/2,-5) -- (-1/2,5) node [above] {\(x=-\frac{1}{2}\)};
    \draw [dashed] (1,-5) -- (1,5) node [above] {\(x=1\)};
    \node [draw,circle,fill,inner sep=0pt,minimum size=5pt] (A) at (1/2,0) {};
    \node [below] at (A) {\(\frac{1}{2}\)};
    \node [draw,circle,fill,inner sep=0pt,minimum size=5pt] (B) at (2,0) {};
    \node [below] at (B) {\(2\)};
    \node [draw,circle,fill,inner sep=0pt,minimum size=5pt] (C) at (0,-2) {};
    \node [right] at (C) {\(-2\)};
    \draw [domain=-5:-0.95] plot (\x,{\eq});
    \draw [domain=-0.23:0.935] plot (\x,{\eq});
    \draw [domain=1.067:5] plot (\x,{\eq});
    \node (D) [draw,circle,fill=white,inner sep=0pt,minimum size=5pt] at (-1,9/2) {};
    \node [below left] at (D) {\(\left(-1,\frac{9}{2}\right)\)};
  \end{tikzpicture}
\end{enumerate}

\end{document}
