\documentclass[letterpaper,12pt,fleqn]{article}
\usepackage[margin=1in]{geometry}
\usepackage{libertine}
\usepackage{parskip}
\usepackage{url}
\usepackage{fancyhdr}
\usepackage{lastpage}
\lhead{}
\chead{}
\rhead{}
\lfoot{Math-19; Section 1; Summer 2019 --- Cavallaro}
\cfoot{}
\rfoot{Page \thepage\ of \pageref{LastPage}}
\setlength{\footskip}{0.5in}
\renewcommand{\headrulewidth}{0pt}
\renewcommand{\footrulewidth}{1pt}
\pagestyle{fancy}
\begin{document}

\begin{center}
  \emph{San Jos\'{e} State University}

  Department of Mathematics and Statistics

  {\large Summer 2019}
  
  \begin{Large}
    \bfseries
    Math-19: Precalculus

    Section 01
  \end{Large}
\end{center}

\vspace{0.25in}

\subsection*{Course and Contact Information}

\begin{tabular}{|p{2in}|p{4.5in}|}
  \hline
  Instructor: & Jeffery Cavallaro \\
  \hline
  Office Location: & Duncan Hall 209 (the TA room) \\
  \hline
  Telephone: & (510)-697-7231 (cell) \\
  \hline
  Email: & \url{jeffery.cavallaro@sjsu.edu} \\
  \hline
  Office Hours: & F 9am--noon \\
  \hline
  Class Days/Time: & MTWR 9--10:40am \\
  \hline
  Classroom: & MacQuarrie Hall 320 \\
  \hline
  Prerequisites: & Math Enrollment Category M-I, M-II, or M-III; or MATH 001 with a grade of C- or better. \\
  \hline
  Corequisite: & Math-19W (the workshop) \\
  \hline
  GE/SJSU Studies Category: & Area B4 \\
  \hline
\end{tabular}

\subsection*{Course Description}

Preparation for STEM calculus: polynomial, rational, exponential, logarithmic and trigonometric functions; analytic geometry.

\subsection*{Course Learning Outcomes}

Upon successful completion of this course, students will be able to:
\begin{itemize}
\item Understand and explain the concept of a function.
\item Recognize and transform a set of standard functions.
\item Perform function arithmetic, including composition.
\item Understand the concept of and determine an inverse for a function.
\item Factor, analyze, and sketch polynomials.
\item Understand exponential functions and their applications to growth and decay.
\item Use the properties of logarithms.
\item Recognize, analyze, and sketch the basic trigonometric functions and their inverses.
\item Solve problems using right-triangle trigonometry, including the laws of sines and cosines.
\item Recognize and use trigonometric identities and double, half, and product-sum formulas.
\item Solve basic systems of equations using substitution, elimination, and matrices.
\item Recognize, analyze, and sketch conic sections, including circles, parabolas, ellipses, and hyperbolas.
\end{itemize}

\subsection*{Required Texts/Readings}

\subsubsection*{Corequisite}

The workshop (Math-19W) is a corequisite for this course.  Workshops are designed to help students succeed in Math courses.
In a typical workshop, students work together in small groups on problems and projects to help them better understand
concepts covered in the class.  The workshop will meet in Sweeney Hall 348 on MTW 11am--12:15pm.  Your workshop
coordinator will be Ishant Sharma.

\subsubsection*{Textbook}

\emph{Precalculus: Mathematics for Calculus}, Stewart/Redlin/Watson, \textbf{7th edition}, ISBN: 978-1-305-07175-9. Book and/or
ebook is fine (your preference) and we \emph{will} be using it in class and workshop.  If using the ebook then make sure that
you have a device on which you can access it during class.

\subsubsection*{Web}

We will use both canvas and webassign. All class communications, including written homework assignments and grades, are
via canvas (\url{sjsu.instructure.com}).  Webassign (\url{webassign.com}) will be used for the major portion of the
homework (see below).  Once you are registered for the course you should be able to see the course listed on your
canvas account.  Each student must purchase a webassign license (usually good for one year). The necessary webassign
class code is posted in a canvas announcement and will be announced on the first day of class. Once you register your
license, you will need this class code to access the class.

\subsubsection*{Calculator}

You must have a TI-84 graphing calculator for use on homework and exams.  I have no problem with you checking your
answers on homework and exams using your calculator; however, answers with no supporting work will receive zero
credit. \emph{No cell phones, tablets, or computers are allowed in lieu of a calculator!}

\subsection*{Course Requirements and Assignments}

\subsubsection*{Time}

You will need to spend a \emph{minimum} of 15 hours per week outside of class doing homework and studying.  This class
is \emph{very} intensive and condensed for the Summer, so it requires \emph{extremely} disciplined study habits.
Please, please, please do \emph{not} register for any other Summer classes or commit to a demanding Summer job; if you
do then your chances of passing this class drop dramatically.

\subsubsection*{Reading}

Reading from the textbook will be assigned each day in class and will prepare you for the next day's lecture and
workshop.  Please read everything, not just the stuff in the boxes, and make sure that you can work all of the example
problems prior to attempting any of the homework problems.

\subsubsection*{Web Homework}

The web-based homework will be submitted via Webassign.  Due dates, which occur frequently, are listed with the assignments.
Webassign requires that you format your answers with math symbols using their answer tool.  Don't get frustrated!  It may
take a couple times for you to get the hang of it; it will get easier the more you use it.  The problems assigned on
Webassign are problems from the book; however, the software may change some of the values involved.  Since you will spend
most of your time on this homework, it constitutes the largest percentage of your grade.  So don't fall behind because there
are \emph{NO} extensions.

\subsubsection*{Written Homework}

In addition to the web-based homework, there are 10 smaller written homework sets.  Whereas the web-based problems are
typically based on single concepts, the written homework problems will combine concepts and will need a little more
thought.  Homework will be assigned each Monday and is due on the following Monday by 9:00am (start of class).  Late
homework will not be accepted.  See \emph{Homework Rules} for more information.

\subsubsection*{Exams}

There will be one midterm and one comprehensive final exam.  The exam schedule is as follows:

\bigskip

\begin{tabular}{ll}
  Midterm & Wednesday, 7/3 \\
  Final & Friday, 8/9
\end{tabular}
  
\bigskip

Prior to an exam, I will post an announcement on canvas telling you exactly what to expect on the exam.  All exams are closed
book and closed notes.  A calculator (as described above) is allowed; however, any answers without supporting work receive
zero credit.  You are allowed one letter-size cheat sheet (front and back).

\subsection*{Determination of Grades}

Your semester grade is determined as follows:

\bigskip

\begin{minipage}{3in}
  \begin{tabular}{|c|c|}
    \hline
    Webassign Homework & 40\% \\
    \hline
    Written Homework & 20\% \\
    \hline
    Midterm Exam & 20\% \\
    \hline
    Final Exam & 20\% \\
    \hline
  \end{tabular}
\end{minipage}
\begin{minipage}{3in}
  \begin{tabular}{|l|c|}
    \hline
    A & 90--100 \\
    B & 80--89 \\
    C & 70--79 \\
    NC & \(<70\) \\
    \hline
  \end{tabular}
\end{minipage}

A grade of C or higher fulfills the Area B4 GE requirement.  Students receiving a C may register for Math 30P.  Students
receiving an A or B may register for Math 30.

\subsection*{Course Content}

We will cover materials from chapters 1--7, 10, and 11 as follows:

\begin{tabular}{|l|l|c|}
  \hline
  \textbf{Week} & \textbf{Sections} \\
  \hline
  1 & 1.1--1.5 \\
  \hline
  2 & 1.7--1.12 \\
  \hline
  3 & 2.1--2.7 \\
  \hline
  4 & 3.1--3.4, 1.6 and 3.5 (time permitting) \\
  \hline
  5 & 3.6--3.7, 10.1--10.3 \\
  \hline
  6 & 2.8, 4.1--4.7 \\
  \hline
  7 & 11.1--11.4 \\
  \hline
  8 & 5.1--5.6 \\
  \hline
  9 & 6.1--6.6 \\
  \hline
  10 & 7.1--7.5  \\
  \hline
\end{tabular}

Please note that this schedule may adjust due to time and class pace.

\subsection*{Classroom Protocol}
  
\subsubsection*{Attendance}

I will not take attendance after the first week; however, it is important that you come (on time) to every class. The book
has more information than we could possibly cover, so I will highlight in class what is important. Bring your book and
calculator to every class meeting. If you miss a class, it is your responsibility to talk to your peers and find out what you
missed.

\subsubsection*{Holidays}

Class will not meet on Independence Day (7/4).

\subsection*{University Policies}

Per University Policy S16-9 (\url{http://www.sjsu.edu/senate/docs/S16-9.pdf}), information relevant to all courses:
academic integrity, accommodations, dropping and adding, consent for recording of class, etc., is available on the Office of
Graduate and Undergraduate Programs’ Syllabus Information web page at \url{http://www.sjsu.edu/gup/syllabusinfo}.
Please make sure to review these university policies and resources.

\end{document}
