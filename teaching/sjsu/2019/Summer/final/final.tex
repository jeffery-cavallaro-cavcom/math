\documentclass[letterpaper,12pt,fleqn]{article}
\usepackage{matharticle}
\usepackage{tikz}
\usepackage{siunitx}
\pagestyle{plain}
\begin{document}

\begin{center}
  \large
  Math-19 Sections 1

  \Large
  Final Exam
\end{center}

\vspace{0.5in}

Name: \rule{4in}{1pt}

\vspace{0.5in}

This exam is closed book and notes. You may use a TI-84 graphing calculator; however, no other electronics are
allowed.  You may also use the trig cheatsheet distributed in class with your own notes added to the back.  Show
all work; there is no credit for guessed answers.  Simplify your answers unless told otherwise.  In particular, all
answers should contain no negative or rational exponents.  All numerical answers should be in exact form unless you
are specifically asked for a decimal value.

\vspace{0.5in}

\begin{center}
  \scalebox{1.25}{
    \begin{tabular}{|c|c|}
      \hline
      problem & score \\
      \hline
      1 & \hspace{0.5in} \\
      \hline
      2 & \\
      \hline
      3 & \\
      \hline
      4 & \\
      \hline
      5 & \\
      \hline
      6 & \\
      \hline
      7 & \\
      \hline
      8 & \\
      \hline
      9 & \\
      \hline
      10 & \\
      \hline
      TOTAL & \\
      \hline
    \end{tabular}}
\end{center}

\newpage

\begin{enumerate}[left=0pt]
\item Calculate the difference quotient for \(f(x)=x^2-3x+1\)

  \vspace{5in}

\item Consider the function:
  \[h(x)=(x+2)^3+\sqrt{x+2}-2e^{x+2}+\ln(x+2)\]
  Determine two functions \(f(x)\) and \(g(x)\) such that \(h=f\circ g\) and neither \(f(x)=x\) nor \(g(x)=x\).

  \newpage

\item A sample of bismuth-210 decays to \(33\%\) of its original mass after \(\SI{8}{days}\). Find the half-life of
  this isotope.

  \vspace{4in}

\item Fully expand the following logarithmic expression:
  \[\ln\left[\frac{x^3(x+1)e^x}{\sqrt[3]{x-2}(x-1)^2}\right]\]

  \newpage

\item Consider the ellipse with foci at \((-2,1)\) and \((6,1)\) and an eccentricity of \(\frac{1}{2}\).
  \begin{enumerate}
  \item Determine the standard-form equation of the ellipse.

    \vspace{4in}

  \item Sketch the ellipse.  You must show and label the center, the foci, and all four vertices.

    \bigskip
    
    \begin{center}
      \begin{tikzpicture}
        \draw (-5,0) -- (5,0);
        \draw (0,-5) -- (0,5);
      \end{tikzpicture}
    \end{center}
  \end{enumerate}

  \newpage

\item Sketch the graph for one full period of the following sinusoidal function.  You must show how you calculate
  the period and the five key points on the graph.  Be sure to show the amplitude and label the $t$ values for the
  five key points.
  \[y=-2\sin\left(\frac{\pi}{2}t-\frac{\pi}{6}\right)\]

  \begin{center}
    \begin{tikzpicture}
      \draw (-3,0) -- (10,0);
      \draw (0,-3) -- (0,3);
    \end{tikzpicture}
  \end{center}

  \newpage

\item You are standing about one quarter mile away from a hill.  By tilting your head upward at an angle of
  \(30^{\circ}\), you are looking straight at a hiker on the summit.  Assuming that your eye level is about
  \SI{5}{ft} from the ground, about how high is the hill (in feet, \SI{1}{mi}=\SI{5280}{ft}).

  \vspace{3in}

\item Rewrite the following expression in terms of \(x\) and \(y\):
  \[\cos(\sin^{-1}x+\cot^{-1}y)\]

  \newpage

\item Rewrite the following expression as a single \(sin\) expression:
  \[\sin(\pi x)+\sqrt{3}\cos(\pi x)\]

  \vspace{4in}

\item Find all solutions to the following equation and state the answer in the most efficient possible form:
  \[\tan^22x-1=0\]
\end{enumerate}

\end{document}
