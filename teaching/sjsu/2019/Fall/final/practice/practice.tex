\documentclass[letterpaper,12pt,fleqn]{article}
\usepackage{matharticle}
\usepackage{siunitx}
\pagestyle{plain}
\begin{document}

\begin{center}
  \large
  Math-71 Sections 02, 03, 60

  \Large
  Final Practice Exam
\end{center}

\vspace{0.5in}

\begin{enumerate}[left=0pt]
\item Consider the function:
  \[f(x)=\frac{x-2}{x^2+3x-10}\]
  \begin{enumerate}
  \item Determine \(\displaystyle\lim_{x\to2}f(x)\) if it exists.  Otherwise, state \emph{DNE}.
  \item Using interval notation, describe where \(f(x)\) is continuous.
  \end{enumerate}

\item Using the definition of the derivative (not the power rule), find \(f(x)=\sqrt{x}\).

\item Let \(\displaystyle f(x)=x^3+5x^2-2x+1\).  Determine \(f'(x)\).

\item Let \(\displaystyle f(x)=\ln(x)\sqrt{x^2+1}\).  Using the product and chain rules, determine \(f'(x)\).

\item Let \(\displaystyle f(x)=\frac{e^x}{x^3-2}\).  Using the quotient and chain rules, determine \(f'(x)\).

\item Let \(\displaystyle f(x)=\ln\left[\frac{x^2e^{x^2}}{\sqrt{x^2+1}}\right]\).  Determine \(f'(x)\).

\item You get your first paycheck from a new job.  On March 1, you go to the bank and open a savings account that
  pays interest of 3\% annually, compounded monthly, with an initial deposit of \$1000.  You set up an auto-deposit
  from your paycheck that will deposit \$500 each month on the first of the month to the account, starting in
  April.  In May, you need to withdraw \$750 from your account to pay some bills.  What is your account balance on
  June 2?

\item After watching a famous episode of the Nickelodean TV show iCarly, you decide to start your own penny tee t-shirt
  business.  You decide that a good monthly demand function (with respect to selling price, in dollars) is given by:
  \[n(p)=2000-100p\]
  You expect monthly fixed costs to be \$1000 and variable costs to be \$1 per shirt.
  \begin{enumerate}
  \item Determine the monthly revenue function \(R(p)\).
  \item Determine the monthly cost function \(C(p)\).
  \item Determine the monthly profit function \(P(p)\).
  \item At what per-shirt sales price is profit maximized?
  \item Prove (either by test points or concavity) that your answer is in fact a maximum.
  \end{enumerate}

\item You are working quality control at a precision ball-bearing factory.  The radius of each ball-bearing as it
  comes off of the production line follows a normal distribution as follows:
  \[p(x)=\frac{1}{0.01\sqrt{2\pi}}e^{\frac{-(x-2)^2}{0.0002}}\]
  where the radius \(x\) is measured in millimeters.
  \begin{enumerate}
  \item What is the mean radius of the ball-bearings?
  \item What is the standard deviation of the radius of the ball-bearings?
  \item At what \(x\) value does the corresponding bell curve have its absolute maximum?
  \item At what \(x\) values does the corresponding bell curve have its points of inflection?
  \item What is the probability that a ball-bearing radius is between \SI{1.97}{mm} and \SI{2.03}{mm}?
  \end{enumerate}

\item The half-life of carbon-14 is 5730 years.  How long does it take for the carbon-14 in an organic sample to
  decay to \(\frac{1}{5}\) of its original mass.

\item A new eBike company called Telfon, run by entrepreneur and international playboy Allen Tusk, is going to
  build its bikes at two factories: one in Singapore and one in South Korea.  Let \(x\) be the number of bikes made
  in the Singapore plant and \(y\) be the number of bikes made in the South Korea plant.  The cost function for
  making the bikes is given by:
  \[C(x,y)=\frac{1}{2}x^2+4x+\frac{1}{4}y^2+8y\]
  The initial combined order will be for 8,980 bikes.  Using the Lagrange multiplier method, determine how these
  8,980 bikes should be split between the two factories in order to minimize cost.

\item Evaluate the following indefinite integral:
  \[\int\left(3+\frac{2x}{\pi}+e^{2x}+\frac{x^2}{x^3+1}\right)dx\]

\item Evaluate the following indefinite integral:
  \[\int\frac{x^2-1}{\sqrt{x}}dx\]

\item You are an operations manager at the mall.  You estimate that the rate of change of the foot-traffic
  in the mall at hour \(t\) since opening to be:
  \[x'(t)=6t-t^2\]
  where \(x\) is measured in hundreds of people.  At \(t=3\) you estimate that there are 2800 people in the mall.
  How many people do you expect to be in the mall at \(t=6\)?

\item Assume that it is known that:
  \begin{gather*}
    \int_0^2 f(x)=5 \\
    \int_0^2 g(x)=3
  \end{gather*}
  Evaluate each of the following:\
  \begin{enumerate}
  \item\(\displaystyle \int_0^1 f(x)dx-\int_2^1 f(x)dx\) \\
  \item\(\displaystyle\int_0^2[4f(x)-3g(x)]dx\)
  \end{enumerate}
  
\item Evaluate the following definite integral:
  \[\int_{2}^{2}(x^3+x)e^x\ln(x^2)dx\]

\item Evaluate the following definite integral:
  \[\int_{-2}^{2}(x^4+x^{20}+x^{50})(x^{101}+x^{51})dx\]

\item Evaluate the following definite integral:
  \[\int_{-2}^{2}(x^2+1)dx\]

\item You work in the advertising department for a new fast-food restaurant that is launching in the bay area.  You
  decide to run a major``branding'' advertising campaign with a new radio spot (commercial).  The goal is for
  people to both hear the commercial and remember the company name and business when asked - thus establishing your
  brand in their mind.  This makes it more likely that people will try your restaurant when they are hungry.  Your
  advertising agency estimates that the rate of change of people successfully branded is given by:
  \[x'(t)=te^{-t^2}\]
  where \(x\) is millions of people branded and \(t\) is the number of months that the planned advertising campaign
  is running.  How many potential customers are branded after 2 months?
  
\item Use Simpson's Rule with a partition size of \(n=6\) to estimate the area under the curve \(f(x)=\ln(x)\)
  between \(x=1\) and \(x=7\).  Start by filling in the following table and then perform the calculation based on
  the values in the table.

  \begin{tabular}{p{1in}|p{1in}|p{1in}}
    \(x_i\) & \(f(x_i)\) & multiplier \\
    \hline
    & & \\
    & & \\
    \hline
    & & \\
    & & \\
    \hline
    & & \\
    & & \\
    \hline
    & & \\
    & & \\
    \hline
    & & \\
    & & \\
    \hline
    & & \\
    & & \\
    \hline
    & & \\
    & &
  \end{tabular}
\end{enumerate}

\end{document}
