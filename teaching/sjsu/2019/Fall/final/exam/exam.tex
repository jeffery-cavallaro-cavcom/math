\documentclass[letterpaper,12pt,fleqn]{article}
\usepackage{matharticle}
\usepackage{siunitx}
\pagestyle{plain}
\begin{document}

\begin{center}
  \large
  Math-71 Sections 02, 03, 60

  \Large
  Final Exam
\end{center}

\vspace{0.5in}

Name: \rule{4in}{1pt}

\vspace{0.5in}

This exam is closed book and notes. You may use a scientific calculator; however, no other electronics are allowed.
You may also use the instructor-provided cheatsheet.  Show all work; there is no credit for guessed answers.
Simplify your answers unless told otherwise.  All numerical answers should be in exact form unless you are
specifically asked for a decimal value.

\vspace{0.5in}

\begin{enumerate}[left=0pt]
\item Consider the function:
  \[f(x)=\frac{x^2-x-2}{x^3+x^2-6x}\]
  \begin{enumerate}
  \item Determine \(\displaystyle\lim_{x\to2}f(x)\) if it exists.  Otherwise, state \emph{DNE}.

    \vspace{3in}

  \item Using interval notation, describe where \(f(x)\) is continuous.
  \end{enumerate}

  \newpage

\item Using the definition of the derivative (not the power rule), find \(f(x)=\sqrt{2x}\).

  \vspace{5in}

\item Let \(\displaystyle f(x)=2x^4-x^3+5x-3\).  Determine \(f'(x)\).

  \newpage

\item Let \(\displaystyle f(x)=\ln(x^2+1)\sqrt{2x}\).  Using the product and chain rules, determine \(f'(x)\).

  \vspace{2in}

\item Let \(\displaystyle f(x)=\frac{x^3-2}{e^x}\).  Using the quotient and chain rules, determine \(f'(x)\).

  \vspace{2in}

\item Let \(\displaystyle f(x)=\ln\left[\frac{x^3\sqrt{x^2-1}}{e^{-x^2}}\right]\).  Determine \(f'(x)\).

  \newpage

\item You get your first paycheck from a new job.  On March 1, you go to the bank and open a savings account that
  pays interest of 2\% annually, compounded monthly, with an initial deposit of \$2500.  You set up an auto-deposit
  from your paycheck that will deposit \$250 each month on the first of the month to the account, starting in
  April.  In April you need to withdraw \$300 from your account to pay some bills, and in May you withdraw another
  \$250.  What is your account balance at the end of June?

  \newpage

\item After watching a famous episode of the Nickelodean TV show iCarly, you decide to start your own penny tee
  t-shirt business.  You decide that a good monthly demand function (with respect to selling price, in dollars) is
  given by:
  \[n(p)=1500-100p\]
  You expect monthly fixed costs to be \$750 and variable costs to be \$2 per shirt.
  \begin{enumerate}
  \item Determine the monthly revenue function \(R(p)\).

    \vspace{1in}

  \item Determine the monthly cost function \(C(p)\).

    \vspace{1in}

  \item Determine the monthly profit function \(P(p)\).

    \vspace{1in}

  \item At what per-shirt sales price is profit maximized?

    \vspace{1.5in}

  \item Prove (either by test points or concavity) that your answer is in fact a maximum.
  \end{enumerate}

  \newpage

\item You are working quality control at a precision ball-bearing factory.  The radius of each ball-bearing as it
  comes off of the production line follows a normal distribution as follows:
  \[p(x)=\frac{1}{0.005\sqrt{2\pi}}e^{\frac{-(x-1)^2}{0.00005}}\]
  where the radius \(x\) is measured in millimeters.
  \begin{enumerate}
  \item What is the mean radius of the ball-bearings?

    \vspace{1in}

  \item What is the standard deviation of the radius of the ball-bearings?

    \vspace{1in}

  \item At what \(x\) value does the corresponding bell curve have its absolute maximum?

    \vspace{1in}

  \item At what \(x\) values does the corresponding bell curve have its points of inflection?

    \vspace{1in}

  \item What is the probability that a ball-bearing radius is between \SI{0.995}{mm} and \SI{1.005}{mm}?

  \end{enumerate}

  \newpage

\item A new eBike company called Telfon, run by entrepreneur and international playboy Allen Tusk, is going to
  build its bikes at two factories: one in Singapore and one in South Korea.  Let \(x\) be the number of bikes made
  in the Singapore plant and \(y\) be the number of bikes made in the South Korea plant.  The cost function for
  making the bikes is given by:
  \[C(x,y)=\frac{1}{4}x^2+2x+\frac{1}{2}y^2+4y\]
  The initial combined order will be for 2,992 bikes.  Using the Lagrange multiplier method, determine how these
  2,992 bikes should be split between the two factories in order to minimize cost.

  \newpage

\item The half-life of carbon-14 is 5730 years.  How old is a mummy whose body has only \(\frac{1}{3}\) of its
  original carbon-14 mass?

  \vspace{3in}

\item Evaluate the following indefinite integral:
  \[\int\left(4x+\frac{e x}{\pi}+e^{3x+1}+\frac{x}{x^2+1}\right)dx\]

  \newpage

\item Evaluate the following indefinite integral but DO NOT use the quotient rule:
  \[\int\frac{x-\sqrt{x}}{x^2}dx\]

  \vspace{3in}

\item You are an operations manager at the mall.  You estimate that the rate of change of the foot-traffic
  in the mall at hour \(t\) since opening to be:
  \[x'(t)=4t-t^3\]
  where \(x\) is measured in hundreds of people.  At \(t=4\) you estimate that there are 2500 people in the mall.
  How many people do you expect to be in the mall at \(t=8\)?

  \newpage

\item Assume that it is known that:
  \begin{gather*}
    \int_0^4 f(x)=2 \\
    \int_0^4 g(x)=5
  \end{gather*}
  Evaluate each of the following:\
  \begin{enumerate}
  \item\(\displaystyle \int_0^2 f(x)dx-\int_4^2 f(x)dx\)

    \vspace{2in}


  \item\(\displaystyle\int_0^4[3f(x)+2g(x)]dx\)

    \vspace{2in}
  \end{enumerate}
  
\item Evaluate the following definite integral:
  \[\int_{4}^{4}\sqrt{e^x-\ln(x)}dx\]

  \newpage

\item Evaluate the following definite integral using the most efficient method possible.
  \[\int_{-2}^{2}x^{37}dx\]

  \vspace{3in}

\item Evaluate the following definite integral using the most efficient method possible.
  \[\int_{-1}^{1}(x^4+x^2)dx\]

  \newpage

\item You work in the advertising department for a new fast-food restaurant that is launching in the bay area.  You
  decide to run a major``branding'' advertising campaign with a new radio spot (commercial).  The goal is for
  people to both hear the commercial and remember the company name and business when asked - thus establishing your
  brand in their mind.  This makes it more likely that people will try your restaurant when they are hungry.  Your
  advertising agency estimates that the rate of change of people successfully branded is given by:
  \[x'(t)=te^{-t^2}\]
  where \(x\) is millions of people branded and \(t\) is the number of months that the planned advertising campaign
  is running.  How many potential customers are branded after 1 month?

  \newpage
  
\item Use Simpson's Rule with a partition size of \(n=4\) to estimate the area under the curve \(f(x)=\ln(x)\)
  between \(x=1\) and \(x=3\).  Start by filling in the following table and then perform the calculation based on
  the values in the table.

  \vspace{0.5in}

  \begin{tabular}{p{1in}|p{1in}|p{1in}}
    \(x_i\) & \(f(x_i)\) & multiplier \\
    \hline
    & & \\
    & & \\
    \hline
    & & \\
    & & \\
    \hline
    & & \\
    & & \\
    \hline
    & & \\
    & & \\
    \hline
    & & \\
    & &
  \end{tabular}
\end{enumerate}

\end{document}
