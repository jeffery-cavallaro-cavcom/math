\documentclass[letterpaper,12pt,fleqn]{article}
\usepackage{matharticle}
\usepackage{tikz}
\pagestyle{plain}
\renewcommand{\l}{\lambda}
\newcommand{\m}{\mu}
\renewcommand{\o}{\sigma}
\begin{document}

\begin{center}
  \large
  Math-71 Sections 02, 03, 60

  \Large
  Practice Exam \#2
\end{center}

\vspace{0.25in}

This exam will be closed book and notes. You may use a scientific calculator; however, no other electronics are
allowed.  You may also use the instructor-provided cheatsheet.  Show all work; there is no credit for guessed
answers.  Simplify your answers unless told otherwise.  In particular, all answers should contain no negative or
rational exponents.  All numerical answers should be in exact form unless you are specifically asked for a decimal
value.

\vspace{0.25in}

\begin{enumerate}[left=0pt]

\item You work for the corporate office of a fast food chain that is going to launch in California.  In the first
  week of operation (\(t=0\)), the new restaurants served 10,000 customers.  By the fifth week (\(t=4\)) that
  number has risen to 20,000.  Assuming that the growth is exponential and is expected to continue in the near
  future, predict how many customers will be served in the tenth week (\(t=9\)).

\item It's time for you to retire!  You have \$1,200,000 in your 401(k) that is earning about 6\% per year,
  compounded monthly on the first day of each the month.  Since you are also collecting social security, you decide
  that you only need to withdraw about \$5000 per month, which you will withdraw on the second day of each month,
  starting in the first month.  What is your account balance on the third day of the fifth month?

\item You are responsible for the environmental control systems at a silicon wafer manufacturing facility.  You
  sample the temperature in the main fabrication lab every hour.  The temperature measurements, in degrees Celsius,
  are expected to follow a normal distribution:
  \[p(T)=\frac{1}{\sqrt{2\pi}}e^{-\frac{(T-25)^2}{2}}\]
  \begin{enumerate}
  \item What is the mean temperature in the lab?
  \item What is the standard deviation of the temperature in the lab?
  \item At what \(T\) value does the corresponding bell curve have its absolute maximum?
  \item At what \(T\) values does the corresponding bell curve have its points of inflection?
  \item What is the probability that a temperature measurement will be between \(24^{\circ}C\) and \(26^{\circ}C\)?
  \end{enumerate}

\item Consider the follow function of two variables:

  \bigskip

  \begin{center}
    \scalebox{1.25}{\(\displaystyle f(x,y)=\ln\left[\frac{\sqrt{y^2+2}e^{(x^2+y^3+1)}}{4x^2(y^3-2)}\right]\)}
  \end{center}
  
  \bigskip

  Determine the following partials:
  \begin{enumerate}[label={\alph*)}]
  \item \(f_x\)
  \item \(f_y\)
  \item \(f_{xx}\)
  \item \(f_{yx}\)
  \end{enumerate}

\item{\label{pd}} Use the second partial derivative test to find the the \((x,y,z)\) coordinates \emph{and} type of the
  absolute extremum on the following surface:

  \bigskip

  \begin{center}
    \scalebox{1.25}{\(\displaystyle z=5-x^2-2x-y^2+4y\)}
  \end{center}

  \bigskip

\item Use the Lagrange multiplier technique to find the \((x,y,z)\) coordinates of the absolute minimum
  point on the surface in problem \ref{pd} given the following constraint:

  \bigskip

  \begin{center}
    \scalebox{1.25}{\(\displaystyle 2x-y+4=0\)}
  \end{center}
\end{enumerate}

\end{document}
