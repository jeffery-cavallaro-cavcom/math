\documentclass[letterpaper,12pt,fleqn]{article}
\usepackage{matharticle}
\usepackage{tikz}
\pagestyle{plain}
\begin{document}

\begin{center}
  \large
  Math-71 Sections 02, 03, 60

  \Large
  Exam \#2
\end{center}

\vspace{0.5in}

Name: \rule{4in}{1pt}

\vspace{0.25in}

This exam is closed book and notes. You may use a scientific calculator; however, no other electronics are allowed.
A cheat sheet is provided on the last page.  Show all work; there is no credit for guessed answers.  Simplify your
answers unless told otherwise.  In particular, all answers should contain no negative exponents.  If the problem
starts with a radical then it must end with a radical.  All numerical answers should be in exact form unless you
are specifically asked for a decimal value.

\vspace{0.25in}

\begin{enumerate}[left=0pt]

\item (10 points) You plan an experiment to test the power of word-of-mouth communication.  You start by giving
  \(100\) test subjects a secret message and an email address.  You instruct your test subjects to email the secret
  message to the email address, and then try to get as many of their friends as possible to do so as well.  After
  \(7\) days you have received a total of \(1000\) such emails.  Assuming that the growth is exponential and will
  continue, predict how many total messages you will have received after \(14\) days.

  \newpage

\item (20 points) You buy a new home for \$500,000 on the first day of the month.  You put down \$50,000 and
  finance the rest with a mortgage at 6\% annual interest compounded monthly on the last day of the month.  Your
  monthly payments, including principal and interest, are \$2500.  Your payments are due on the first of the month,
  starting next month.  What is your loan balance after your third monthly payment?

  \newpage

\item (10 points) You are testing the duration of certain fuses for a pyrotechnic company.  The manufacturer states
  that the duration (from ignition to explosion) follows a normal distribution as follows, where the mean and
  standard deviation are expressed in seconds:
  \[p(t)=\frac{1}{2\sqrt{2\pi}}e^{\frac{-(t-10)^2}{8}}\]
  \begin{enumerate}
  \item What is the mean of the fuse duration?

    \vspace{1in}
    
  \item What is the standard deviation of the fuse duration?

    \vspace{1in}
    
  \item At what \(t\) value does the corresponding bell curve have its absolute maximum?

    \vspace{1in}
    
  \item At what \(t\) values does the corresponding bell curve have its points of inflection?

    \vspace{1in}
    
  \item What is the probability that a fuse duration will be between 7 and 13 seconds?
  \end{enumerate}

  \newpage

\item (20 points) Consider the follow function of two variables:

  \bigskip

  \begin{center}
    \scalebox{1.25}{\(\displaystyle f(x,y)=\ln\left[\frac{5x^2y^3}{\sqrt{x^2+1}e^{(x^2+y^2)}}\right]\)}
  \end{center}
  
  \bigskip

  Determine the following partials:
  \begin{enumerate}[label={\alph*)}]
  \item \(f_x\)

    \vspace{3in}
    
  \item \(f_y\)

    \vspace{1in}

  \item \(f_{yy}\)

    \vspace{1in}

  \item \(f_{xy}\)
  \end{enumerate}

  \newpage

\item\label{pd} (20 points) Use the second partial derivative test to find the the \((x,y,z)\) coordinates
  \emph{and} type of the absolute extremum on the following surface:

  \bigskip

  \begin{center}
    \scalebox{1.25}{\(\displaystyle z=x^2+4x+y^2-2y+10\)}
  \end{center}

  \bigskip

  \newpage

\item (20 points) Use the Lagrange multiplier technique to find the \((x,y,z)\) coordinates of the absolute
  extremum on the surface in problem \ref{pd} given the following constraint:

  \bigskip

  \begin{center}
    \scalebox{1.25}{\(\displaystyle 2x-y=0\)}
  \end{center}
\end{enumerate}

\end{document}
