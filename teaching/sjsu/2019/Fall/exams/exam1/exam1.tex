\documentclass[letterpaper,12pt,fleqn]{article}
\usepackage{matharticle}
\usepackage{tikz}
\pagestyle{plain}
\begin{document}

\begin{center}
  \large
  Math-71 Sections 02, 03, 60

  \Large
  Exam \#1
\end{center}

\vspace{0.5in}

Name: \rule{4in}{1pt}

\vspace{0.5in}

This exam is closed book and notes. You may use a scientific calculator; however, no other electronics are allowed.
You may also use a cheatsheet of your own making on two sides of an \(8\frac{1}{2}\times11''\) paper.  Show all
work; there is no credit for guessed answers.  Simplify your answers unless told otherwise.  In particular, all
answers should contain no negative exponents.  If the problem starts with a radical then it must end with a
radical.  All numerical answers should be in exact form unless you are specifically asked for a decimal value.

\vspace{0.5in}

\begin{enumerate}[left=0pt]
\item Write down the three requirements for a function \(f(x)\) to be continuous at a point \(x=c\).
  \label{continuity}
  \begin{enumerate}[label={\arabic*)}]
    \setlength{\parskip}{0.5in}
  \item

  \item

  \item
  \end{enumerate}

  \vspace{0.5in}

\item Write down the definition of the derivative and the other two characterizations that we discussed in class:
  \label{derivative}
  \begin{description}
    \setlength{\parskip}{0.5in}
  \item{DEF:}

  \item{CHAR1:}

  \item{CHAR2:}
  \end{description}

  \newpage

\item Consider the rational function:
  \label{ratfunc}
  
  \scalebox{1.5}{
    \[f(x)=\begin{cases}
    \frac{(x^2-4x+3)(x+2)}{(x^2+6x+8)(x-1)}, & x\ne1 \\
    0, & x=1
    \end{cases}\]
  }
  Compute
  \[\lim_{x\to1}f(x)\]

  \vspace{4.5in}

\item For the rational function in question \ref{ratfunc}, identify all of the points of discontinuity and for each
  such point identify all of the continuity requirements that fail per your list of requirements in problem
  \ref{continuity}.

  \bigskip

  \begin{center}
    \begin{tabular}{|p{1in}|p{1in}|p{1in}|p{1in}|}
      \hline
      \(x\) value & rule 1 & rule 2 & rule 3\\
      \hline
      & & & \\[0.25in]
      \hline
      & & & \\[0.25in]
      \hline
      & & & \\[0.25in]
      \hline
    \end{tabular}
  \end{center}

  \newpage

\item Consider the following function:
  \label{derivdef}
  \[f(x)=\sqrt{x^2+1}\]
  Using the \emph{definition of the derivative} (from your answer in Problem \ref{derivative}), calculate \(f'(x)\).

  \vspace{5in}  

\item For the function in problem \ref{derivdef}, find \(f'(x)\) using the derivative formulas.

  \newpage
  
\item Differentiate the following function.  You do not need to simplify.
  \[f(x)=-\frac{3x^2}{4}+\frac{2}{x^3}-\frac{3}{\sqrt{x}}\]

  \vspace{2in}

\item Differentiate the following function using the product rule:
  \[f(x)=(x^2-3)(2x+1)\]

  \vspace{2in}

\item Differentiate the following function using the quotient rule:
  \[f(x)=\frac{x^2-3}{2x+1}\]

  \newpage

\item A company releases a particularly poor earnings report.  On the following trading day, the price \(p\) (in
  dollars) as a function of time \(t\) (in hours), is modeled by the following equation:
  \[p(t)=\frac{1}{3}t^3-\frac{5}{2}t^2+30\]
  for the entire \(6\frac{1}{2}\) hour trading day --- meaning that the domain of the function is
  \([0,\frac{13}{2}]\).  Using the first or second derivative test, determine the lowest stock price \(p\) (rounded
  to two decimal places) during the trading day and at what time \(t\) that price occurred.  You must \emph{prove}
  that your answer is indeed an absolute minimum.
\end{enumerate}

\end{document}
