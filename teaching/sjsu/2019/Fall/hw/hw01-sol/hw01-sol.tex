\documentclass[letterpaper,12pt,fleqn]{article}
\usepackage{matharticle}
\usepackage{cancel}
\pagestyle{plain}
\begin{document}

\begin{center}
  \large
  Math-71 Sections 02, 03, 60

  \Large
  Homework \#1 Solutions
\end{center}

\subsection*{Problem}

Consider the function: \(f(x)=x^2\).  Calculate the following limit:
\[\lim_{h\to0}\frac{f(x+h)-f(x)}{h}\]

First note that if we try substitution with this form we get the indeterminant form \(\frac{0}{0}\).  So, we need
to do some algebra first:

\begin{align*}
  \lim_{h\to0}\frac{f(x+h)-f(x)}{h} &= \lim_{h\to0}\frac{(x+h)^2-x^2}{h} \\
  &= \lim_{h\to0}\frac{\cancel{x^2}+2xh+h^2-\cancel{x^2}}{h} \\
  &= \lim_{h\to0}\frac{2xh+h^2}{h} \\
  &= \lim_{h\to0}\frac{\cancel{h}(2x+h)}{\cancel{h}} \\
  &= \lim_{h\to0}(2x+h) \\
  &= 2x+0 \\
  &= 2x
\end{align*}

This is how the solution should be presented, with no shortcuts.  Each line is a complete statement.  Some students
insist on syntactic shortcuts that don't make sense, like omitting the limit operator until the end.  When you do
this you are in danger of me not being able to follow what you are doing.  If I have to go back and reread your
solutions 3 times to try and figure out what is going on then I am just going to assume that it is wrong and I will
not entertain verbal explanations after the fact on why it is right --- i.e., no video replay in this game.

Also please note that some students insist on continuing to write the limit operator after they have done the
substitution:

\begin{gather*}
  \lim_{h\to0}(2x+0) \\
  \lim_{h\to0}2x
\end{gather*}

This is incorrect.  I didn't take off points this time, but I will in the future.

One final note, please submit your homework on a separate piece of college rule or graph paper --- not the original
problem sheet.  Remove any rippy edges.  Finally, do not add extra notes or scratched out work on your submission.
Pencil is best.

\end{document}
