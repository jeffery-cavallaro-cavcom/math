\documentclass[letterpaper,12pt,fleqn]{article}
\usepackage{matharticle}
\pagestyle{plain}
\begin{document}

\begin{center}
  \large
  Math-71 Sections 02, 03, 60

  \Large
  Homework \#6 Solutions
\end{center}

\subsection*{Problem}

You open a new savings account on April 1 with an initial \$1000 at a local bank that pays 2\% interest, compounded
monthly on the last day of the month.  You file the necessary direct deposit paperwork at you job, where you get
paid monthly on the first of the month, so that \$500 will be auto-deposited into your new account each month,
starting with your May paycheck.  In June, you withdraw \$250 to pay for a new mobile phone.  In July, you withdraw
\$750 to help pay for your summer vacation.  There are no other transactions.  What is your account balance on
August 2?

First, calculate the compounding factor:
\[x=1+\frac{r}{n}=1+\frac{0.02}{12}=1.001666667\]
There are lots of decimal places here, so store this result into a storage register on your calculator.

Now, construct the transaction table:

\begin{center}
  \begin{tabular}{|c|c|c|c|}
    \hline
    month & deposit & withdrawal & net \\
    \hline
    Apr & 1000 & & 1000 \\
    \hline
    May & 500 & & 500 \\
    \hline
    Jun & 500 & 250 & 250 \\
    \hline
    Jul & 500 & 750 & -250 \\
    \hline
    Aug & 500 & & 500 \\
    \hline
  \end{tabular}
\end{center}

Next, note that the first deposit compounds over 4 months, so we build the compounding polynomial as follows:
\[1000x^4+500x^3+250x^2-250x+500=\$2009.60\]
Note that the deposit occuring on Aug 1 is not compounded and thus becomes a constant term in the polynomial.
\end{document}
