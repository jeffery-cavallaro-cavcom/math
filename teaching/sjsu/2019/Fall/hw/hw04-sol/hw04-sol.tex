\documentclass[letterpaper,12pt,fleqn]{article}
\usepackage{matharticle}
\usepackage{siunitx}
\pagestyle{plain}
\begin{document}

\begin{center}
  \large
  Math-71 Sections 02, 03, 60

  \Large
  Homework \#4 Solutions
\end{center}

\subsection*{Problem}

You are a leader for your local Girl Scout troop and it has fallen upon you to plan next year's cookie sales.  It
has become common to adjust the price of a box of cookies to maximize profits based on the affluence of the
community.  From past years, you know that when the price was \$5.00 per box about 10,000 boxes were sold.  When
the price was raised to \$6.00 per box, only 7500 boxes were sold.  Assume that the demand function \(n(p)\) is
linear.  The factory that makes the cookies reports that the fixed costs are \$10,000 and the variable costs are
\$2.00 per box.
\begin{enumerate}
\item At what sales price will your troop maximize its profits?

  We start with the demand function.  We are told to assume that the demand function is linear, and we are given
  two points:
  \begin{align*}
    n(\$5) &= \SI{10000}{boxes} \\
    n(\$6) &= \SI{7500}{boxes}
  \end{align*}
  This is sufficient information to construct a line:
  \begin{gather*}
    m=\frac{10000-7500}{5-6}=\SI{-2500}{boxes/\$} \\
    n-10000=-2500(p-5) \\
    n(p)=-2500p+22500
  \end{gather*}
  We can now construct the cost function.  We are given the fixed and variable costs, so:
  \begin{align*}
    C(p) &= 10000+2n(p) \\
    &= 10000+2(-2500p+22500) \\
    &= 10000-5000p+45000 \\
    C(p) &= 55000-5000p
  \end{align*}
  Now, on to the revenue function:
  \begin{align*}
    R(p) &= p\cdot n(p) \\
    &= p(-2500p+22500) \\
    R(p) &= -2500p^2+22500p
  \end{align*}
  And finally, the profit function:
  \begin{align*}
    P(p) &= R(p)-C(p) \\
    &= (-2500p^2+22500p)-(55000-5000p) \\
    P(p) &= -2500p^2+27500p-55000
  \end{align*}
  
  In order to optimize \(P(p)\) we need to differentiate:
  \[P'(p)=-5000p+27500\]

  To find all critical points we set this to 0:
  \begin{gather*}
    0=-5000p+27500 \\
    5000p=27500 \\
    p=\$5.50
  \end{gather*}

  We need to show that this is a maximum.  We know that this is the case because \(P(p)\) is an inverted parabola
  with the max at the vertex, which will be the only critical point.  But, we can also use test points around
  \(p=5.5\).  For \(p<5.5\) we have \(P'(p)>0\) meaning that \(P(p)\) is increasing.  For \(p>5.5\) we have
  \(P'(p)<0\) meaning that \(P(p)\) is decreasing.  This, we have a maximum.

  We can also use the second derivative test:
  \[P''(p)=-5000<0\]
  and so \(P''(5.5)<0\) meaning \(P(p)\) is concave down, and thus there is a maximum at \(p=5.5\).

  Therefore, to maximize profits, boxes should be sold at \$5.50 per box.

\item At that price, how many boxes is your troop projected to sell?
  \[n(\$5.50)=-2500(5.50)+22500=\SI{8750}{boxes}\]

\item What is the expected profit?
  \[P(\$5.50)=-2500(5.50)^2+27500(5.50)-55000=\$20625\]

\end{enumerate}

\end{document}
