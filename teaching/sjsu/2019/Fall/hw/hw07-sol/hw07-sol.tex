\documentclass[letterpaper,12pt,fleqn]{article}
\usepackage{matharticle}
\usepackage{siunitx}
\pagestyle{plain}
\newcommand{\m}{\mu}
\renewcommand{\o}{\sigma}
\begin{document}

\begin{center}
  \large
  Math-71 Sections 02, 03, 60

  \Large
  Homework \#7 Solutions
\end{center}

\subsection*{Problem}

You are working quality control for a manufacturer of screws.  You are sampling a particular screw as it comes off
of the line.  You expect the length of the screw to follow a normal distribution as follows, where the mean and
standard deviation are expressed in centimeters (cm):
\[p(x)=\frac{1}{0.1\sqrt{2\pi}}e^{-50(x-2)^2}\]
\begin{enumerate}
\item What is the mean of the screw length?
  \[\m=\SI{2.0}{cm}\]
\item What is the standard deviation of the screw length?
  \[\o=\SI{0.1}{cm}\]
\item At what \(x\) value does the corresponding bell curve have its absolute maximum?
  \[x=\mu=\SI{2.0}{cm}\]
\item At what \(x\) values does the corresponding bell curve have its points of inflection?
  \[x=\mu\pm\o=\SI{1.9}{cm},\SI{2.1}{cm}\]
\item What is the probability that a screw length will be between 1.8 and 2.2 cm?

  Since this is in the \(2\o\) range, the probability is \(95\%\).
\end{enumerate}

\end{document}
