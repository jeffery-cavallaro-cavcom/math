\documentclass[letterpaper,12pt,fleqn]{article}
\usepackage{matharticle}
\pagestyle{plain}
\begin{document}

\begin{center}
  \large
  Math-71 Sections 02, 03, 60

  \Large
  Homework \#2

  \large
  \textbf{Due: 9/9/2019 1:15pm}
\end{center}

\subsection*{Reading}

Read sections 7.1 and 7.2

\subsection*{Problem}

You are a business analyst for the city of Lakeside, Calfornia.  Lakeside is famous for its huge lake, which
unfortunately has become polluted with carbon compounds over the years by a nearby factory.  The factory is now
closed, so the city council wants to clean up the lake.  They have made you the head of the project.  You contact
the EPA and find out three important things:
\begin{enumerate}
\item The lake can be cleaned up to what is considered to be a naturally-occurring level of the carbon pollutants
  by removing \(80\%\) of the pollutants.
\item The cost, in \emph{millions} of dollars, to reduce the pollutants in the lake to a given percentage \(p\) is
  estimated by the following formula (function):
  \[C(p)=\frac{10p}{100-p}\]
\item EPA standards say that cleaning up \(75\%\) of the pollutants is good enough, because the extra \(5\%\)
  doesn't do any significant harm.
\end{enumerate}
You decide that the best way to raise the money for this project is to hold a special election for a local ballot
initiative to assess a special charge on next year's property tax bill.  You know that the population of Lakeside
is 100,000 people, of which \(80\%\) are property owners.
\begin{enumerate}[label={(\alph*)}]
\item Assuming that the property tax burden is to be distributed equally, what will be the resulting assessment on
  each property tax bill under the federal requirements?
\item Governor Newsom visits the lake with a state EPA representative.  They tell you that California leads the
  nation in environmental policy, so he says that the state requirement is to bring the pollutant level all the way
  to the natural level.  What is the resulting property tax bill assessment now?
\item Some members from the group \emph{Save the Fish!} protest outside your office.  They claim that the lake is
  home to a rare species of fish that is being harmed by the pollutants.  They demand that you decrease the
  pollutant level by \(95\%\) and if you don't then they are going to sue you in federal court under the Endangered
  Species Act.  What would be the resulting property tax bill assessment for that level of cleanup?
\item What does the formula tell you about removing \emph{all} of the pollutants from the lake and why do you think
  this might be so?
\item Over what values of percentage \(p\), expressed in interval notation, is the function represented by the formula
  continuous?
\end{enumerate}

\end{document}
