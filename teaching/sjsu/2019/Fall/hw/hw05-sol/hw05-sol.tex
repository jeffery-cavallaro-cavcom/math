\documentclass[letterpaper,12pt,fleqn]{article}
\usepackage{matharticle}
\usepackage{tikz}
\pagestyle{plain}
\begin{document}

\begin{center}
  \large
  Math-71 Sections 02, 03, 60

  \Large
  Homework \#5 Solutions
\end{center}

\subsection*{Problem}

A triangle is inscribed inside a semi-circle of radius 2 as shown below:

\bigskip

\begin{center}
  \begin{tikzpicture}[see node/.style={draw,circle,fill,inner sep=0pt,minimum size=5pt}]
    \draw [<->] (-4,0) -- (4,0);
    \draw [<->] (0,-1) -- (0,4);
    \draw (3,0) arc (0:180:3);
    \pgfmathsetmacro{\r}{3}
    \pgfmathsetmacro{\x}{\r*cos(60)}
    \pgfmathsetmacro{\y}{\r*sin(60)}
    \node (X) [see node] at (\x,0) {};
    \node (Y) [see node] at (\x,\y) {};
    \draw (-3,0) -- (Y) -- (3,0);
    \node [below] at (-3,0) {-2};
    \node [below] at (3,0) {2};
    \node [above right] at (0,3) {2};
    \node [below right] at (0,0) {0};
    \draw [dashed] (Y) -- (X);
    \node [below] at (X) {\(x\)};
    \node [above right] at (Y) {\(y=\sqrt{4-x^2}\)};
  \end{tikzpicture}
\end{center}

\bigskip

Find the maximum possible area of the inscribed triangle.  Be sure to prove why it is a maximum by either a plausible
explanation or by using the first or second derivative test.

The equation for a circle with center at the origin and radius 2 is given by:
\[x^2+y^2=4\]
Solving for \(y\) and taking the upper semicircle only:
\[y=\sqrt{4-x^2}\]
Recall that the area of a triangle is given by:
\[A=\frac{1}{2}bh\]
In this case, the base \(b\) is a constant value of 4.  The height \(h\) for a given \(x\) value is
\(y(x)=\sqrt{4-x^2}\).  Thus, the equation for the area of the triangle as a function of \(x\) is given by:
\[A(x)=\frac{1}{2}(4)\sqrt{4-x^2}=2(4-x^2)^{\frac{1}{2}}\]
where the domain for \(x\) is \([-2,2]\).

We want to maximize the value of \(A(x)\) with respect to \(x\), so we determine the derivative:
\[A'(x)=2\left[\frac{1}{2}(4-x^2)^{-\frac{1}{2}}(-2x)\right]=-\frac{2x}{\sqrt{4-x^2}}\]
The first derivative has critical points at \(x=0\) (zero) and \(x=\pm2\) (poles).

Note that \(x=\pm2\) are both the endpoints and the zeros of the original function \(A(x)\).  Since \(A(x)\ge0\)
these two points must be absolute minimums, so we can discard them.

Let's use test points (the second derivative is too messy) to prove that there is a relative maximum at \(x=0\).
Since the denominator of \(A'(x)\) is always positive, the sign is completely determined by the numerator.  For a
test point a little less than \(x=0\) it is the case that \(A'(x)>0\).  For a test point a little greater than
\(x=0\) it is the case that \(A'(x)<0\).  Thus, there is a relative maximum at \(x=0\).  Due to the absense of any
more critical points and the fact that the endpoints are absolute minimums, we can conclude that we have an
absolute maximum at \(x=0\).

Therefore, the maximum area is:
\[A(0)=2\sqrt{4-0^2}=2\cdot2=4\]

\end{document}
