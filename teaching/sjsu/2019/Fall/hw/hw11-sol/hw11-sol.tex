\documentclass[letterpaper,12pt,fleqn]{article}
\usepackage{matharticle}
\pagestyle{plain}
\begin{document}

\begin{center}
  \large
  Math-71 Sections 02, 03, 60

  \Large
  Homework \#11 Solutions
\end{center}

\subsection*{Problem}

You work for a robotics company that is making a new line of hamburger-making robots to be sold to fast-food
chains.  This is a big-ticket item, so sales will be slow at first, but should pick up over time.  Your marketing
department estimates that the sales growth rate will increase linearly by 2 robots per month per month.  In the
first month (\(t=0\)), for which you have already booked sales for 10 units, the growth rate is expected to be 5
robots per month.  How many total robots do you expect to sell by the end of the tenth month (\(t=9\))?

First, build the rate-of-change function:
\[f'(t)=2t+5\]
Now integrate to get the total function:
\[f(t)=\int(2t+5)dt=t^2+5t+C\]
Now use the initial condition:
\[f(0)=0^2+5(0)+C=10\]
and so \(C=10\).  Thus, the total function is:
\[f(t)=t^2+5t+10\]
Now plug in \(t=9\):
\[f(9)=9^2+5(9)+10=81+45+10=136\]
Thus, you expect to sell 136 robots by the end of the tenth month.


\end{document}
