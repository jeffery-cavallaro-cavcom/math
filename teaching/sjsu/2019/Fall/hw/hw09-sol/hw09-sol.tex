\documentclass[letterpaper,12pt,fleqn]{article}
\usepackage{matharticle}
\pagestyle{plain}
\renewcommand{\l}{\lambda}
\begin{document}

\begin{center}
  \large
  Math-71 Sections 02, 03, 60

  \Large
  Homework \#9 Solutions
\end{center}

\subsection*{Problem}

You work for a large conglomerate of 100 associated companies.  In order to avoid antitrust issues with the DOJ,
you need to divide the 100 companies into 4 different groups such that:
\begin{enumerate}
\item Companies within the same group cannot do business with each other.
\item Companies in different groups can do business with each other.
\end{enumerate}
What organization of the 100 companies into the 4 groups maximimizes the number of business opportunities?

\begin{enumerate}[label={\alph*)}]
\item Start by labeling the groups \(X\), \(Y\), \(Z\), and \(W\).  Let \(x=\) the number of companies assigned to
  group \(X\) and so on for the other groups.  Construct an equation in \(x\) for the number of business
  opportunities for a company in group \(X\) --- i.e., how many companies can that company do business with?

  Each company in \(X\) can do business with the other \(100-x\) companies.
  
\item Now build an equation in \(x\) for the total number of business opportunities for all companies in group
  \(X\).

  There are \(x\) companies, each of with can do business with \(100-x\) companies, for a total of \(x(100-x)\)
  business opportunities for all of the companies in X.

\item Do likewise for the remaining groups and construct a function \(f(x,y,z,w)\) that gives the total number of
  business opportunities across all the groups.
  \[f(x,y,z,w)=x(100-x)+y(100-y)+z(100-z)+w(100-w)\]
  
\item What is the constraint on \(x\), \(y\), \(z\), and \(w\)?
  \[x+y+z+w=100\]

\item Introduce a Lagrange multiplier \(\l\) and determine \(f_x=\l g_x\), where \(g\) is the function constructed
  from the above constraint.
  \[100-2x=\l\]

\item Do likewise for \(f_y\), \(f_z\), and \(f_w\), and combine them with the constraint so that you have 5
  equations in 5 unknowns.
  \begin{gather*}
    100-2x=\l \\
    100-2y=\l \\
    100-2z=\l \\
    100-2w=\l \\
    x+y+z+w=100
  \end{gather*}

\item Use substitution to determine a value for \(\l\).
  \begin{gather*}
    2x=100-\l \\
    \\
    x=50-\frac{\l}{2} \\
    y=50-\frac{\l}{2} \\
    z=50-\frac{\l}{2} \\
    w=50-\frac{\l}{2} \\
    \\
    4\left(50-\frac{\l}{2}\right)=100 \\
    50-\frac{\l}{2}=25 \\
    \frac{\l}{2}=25 \\
    \l=50
  \end{gather*}

\item Substitute the value for \(\l\) into the other equations to determine the optimal distribution of companies
  into the 4 groups.
  \begin{gather*}
    x=50-\frac{50}{2}=50-25=25 \\
    y=25 \\
    z=25 \\
    w=25
  \end{gather*}
\end{enumerate}

\end{document}
