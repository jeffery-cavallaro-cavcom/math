\documentclass[letterpaper,12pt,fleqn]{article}
\usepackage[margin=1in]{geometry}
\usepackage{libertine}
\usepackage{parskip}
\usepackage{url}
\usepackage{fancyhdr}
\usepackage{lastpage}
\lhead{}
\chead{}
\rhead{}
\lfoot{Math-71; Sections 02,03,60; Fall 2019 --- Cavallaro}
\cfoot{}
\rfoot{Page \thepage\ of \pageref{LastPage}}
\setlength{\footskip}{0.5in}
\renewcommand{\headrulewidth}{0pt}
\renewcommand{\footrulewidth}{1pt}
\pagestyle{fancy}
\begin{document}

\begin{center}
  \emph{San Jos\'{e} State University}

  Department of Mathematics and Statistics

  {\large Fall 2019}
  
  \begin{Large}
    \bfseries
    Math-71: Calculus for Business and Aviation

    Sections 02, 03, 60
  \end{Large}
\end{center}

\vspace{0.25in}

\subsection*{Course and Contact Information}

\begin{tabular}{|p{2in}|p{4.5in}|}
  \hline
  Instructor: & Jeffery Cavallaro \\
  \hline
  Office Location: & Duncan Hall 209 (the TA room) \\
  \hline
  Telephone: & (510)-697-7231 (cell) \\
  \hline
  Email: & \url{jeffery.cavallaro@sjsu.edu} \\
  \hline
  Office Hours: & F 9am-noon \\
  \hline
  Class Days/Time: & \begin{minipage}{4.5in}
    \vspace{0.1cm}
    Section 02: MW 10:30am--11:45am
    
    Section 03: MW 12:00pm--1:15pm
    
    Section 60: MW 9:00am--10:15am
    \vspace{0.1cm}
  \end{minipage} \\
  \hline
  Classroom: & Duncan Hall 318 \\
  \hline
  Prerequisites: & \begin{minipage}{4.5in}
    \vspace{0.1cm}
    Any of the following:
    \begin{itemize}
    \item A grade of C- or higher in Math 8.
    \item A grade of C or higher in Math 19.
    \item A score of 550 or higher on the math portion of the SAT.
    \item A score of 23 or higer on the math ACT.
    \item A score of 55 or higher on the CPE.
    \end{itemize}
    \vspace{0.1cm}
  \end{minipage} \\
  \hline
  Corequisite: & Math-71W (the workshop) \\
  \hline
  GE/SJSU Studies Category: & Area B4 \\
  \hline
\end{tabular}

\subsection*{Course Description}

Functions and graphs, limits, continuity, differentiation, integration, partial differentiation. Emphasis on
business and economics applications.

\subsection*{Course Learning Outcomes}

Upon successful completion of this course, students will be able to:
\begin{itemize}
\item Use the properties of exponential functions and logarithms.
\item Compute simple, compound, and continuous interest.
\item Explain the meaning of the limit of a function.
\item Compute limits.
\item Find derivatives of functions.
\item Use the concept of the derivative in applications and be able to solve problems on maxima and minima.
\item Evaluate integrals using substitutions and compute areas.
\item Compute partial derivatives.
\item Use Lagrange multipliers to find the extreme values of functions of several variables.
\item Understand probability density functions, properties of various continuous distributions including mean and
  variance.
\end{itemize}

\subsection*{Required Texts/Readings}

\subsubsection*{Corequisite}

The workshop (Math-71W) is a corequisite for this course.  Workshops are designed to help students succeed in Math
courses.  In a typical workshop, students work together in small groups on problems and projects to help them
better understand concepts covered in the class.

\subsubsection*{Textbook}

\emph{College Algebra and Applied Calculus}, Larson/Hodgkins, \textbf{2th edition}, ISBN: 978-1133105060. Book
and/or ebook is fine (your preference) and we \emph{will} be using it in class.  If using a book, then the special
SJSU edition in the bookstore is suggested, since it contains chapter 16 (Probability), which we will cover.  If
using the ebook then make sure that you have a device on which you can access it during class.

\subsubsection*{Web}

We will use both canvas and webassign. All class communications, including written homework assignments and grades,
are via canvas (\url{sjsu.instructure.com}).  Webassign (\url{webassign.com}) will be used for the major portion of
the homework (see below).  Once you are registered for the course you should be able to see the course listed on
your canvas account.  Each student must purchase a webassign license (or continue to use your one year license from
last semester). The necessary webassign class code is posted in a canvas announcement and will be announced on the
first day of class. Once you register your license, you will need this class code to access the class.

\subsubsection*{Calculator}

You should have a scientific calculator for use on your exams.  You may use a TI-84 during class and while doing
your homework when graphing may help you check your answers.  \emph{No other graphing calculators, cell phones,
  tablets, or computers are allowed in lieu of a scientific calculator on exams}.  In particular, use of a TI-89 is
prohibited.

\subsection*{Course Requirements and Assignments}

\subsubsection*{Time}

You will need to spend a \emph{minimum} of 10 hours per week outside of class doing homework and studying. This
class is intensive and will require disciplined study habits.  Please, please, please do \emph{not} register for 16
units and commit to a 20+ hours per week job; if you do then your chances of passing this class drop dramatically.

\subsubsection*{Reading}

Reading from the textbook will be assigned each Wednesday (in class and on canvas) for the material to be covered in
the coming week.  Please read everything, not just the stuff in the boxes, prior to lecture.  Make sure that you
can work all of the example problems prior to attempting any of the homework problems.

\subsubsection*{Web Homework}

The web-based homework will be submitted via Webassign.  Due dates, which occur frequently, are listed with the
assignments.  Webassign requires that you format your answers with math symbols using their answer tool.  Don't get
frustrated!  It may take a couple times for you to get the hang of it; it will get easier the more you use it.  The
problems assigned on Webassign are problems from the book; however, the software may change some of the values
involved.  Since you will spend most of your time on this homework, it constitutes the largest percentage of your
grade.  So don't fall behind because there are \emph{NO} extensions.

\subsubsection*{Written Homework}

In addition to the web-based homework, there are twelve small written homework sets.  Whereas the web-based
problems are typically based on single concepts, the written homework problem will combine concepts and will need a
little more thought.  Homework will be assigned each non-exam week on Monday and is generally due on the following
Monday by 1:15pm (when I leave campus).  Late homework will not be accepted; however, I will drop your two lowest
score.  See \emph{Homework Rules} for more information.

\subsubsection*{Exams}

There will be two regular exams and a comprehensive final exam.  The regular exam schedule is as follows:

\bigskip

\begin{tabular}{ll}
  Exam 1 & Wednesday, 10/2 \\
  Exam 2 & Wednesday, 11/6
\end{tabular}
  
\bigskip

Prior to an exam, I will post an announcement on canvas telling you exactly what to expect on the exam.  All exams
are closed book and closed notes.  A calculator (as described above) is allowed; however, any answers without
supporting work receive zero credit.  Instead of a note card, I will provide a derivative/integral cheat sheet
where appropriate.

\subsubsection*{Final}

The final exam is comprehensive and is scheduled as follows:
\begin{description}
  \bfseries
\item{Section 02:} Thursday, 12/12, 9:45am to 12:00pm
\item{Section 03:} Monday, 12/16, 9:45am to 12:00pm
\item{Section 60:} Friday, 12/13, 7:15am to 9:30am
\end{description}
\emph{Do not make any travel plans that occur prior to your exam date --- attendance is mandatory.}

\subsection*{Determination of Grades}

Your semester grade is determined as follows:

\bigskip

\begin{minipage}{3in}
  \begin{tabular}{|c|c|}
    \hline
    Webassign Homework & 40\% \\
    \hline
    Written Homework & 10\% \\
    \hline
    Regular Exams & 30\% \\
    \hline
    Final Exam & 20\% \\
    \hline
  \end{tabular}
\end{minipage}
\begin{minipage}{3in}
  \begin{tabular}{|l|c|}
    \hline
    A+ & 100--97 \\
    A & 96--94 \\
    A- & 93--90 \\
    B+ & 89-87 \\
    B & 86-84 \\
    B- & 83-80 \\
    C+ & 79-77 \\
    C & 76-74 \\
    C- & 73-70 \\
    D+ & 69-67 \\
    D & 66-64 \\
    D- & 63-60 \\
    F & \(<\)60 \\
    \hline
  \end{tabular}
\end{minipage}

A grade of C- or higher fulfills the Area B4 GE requirement; however, most majors require a passing grade of C or
better for degree credit.

\subsection*{Course Content}

We will cover materials from chapters 7--11, 13, and 16 as follows:

\begin{tabular}{|l|l|c|}
  \hline
  \textbf{Sections} & \textbf{Weeks} \\
  \hline
  7.1--7.7 & 3 \\
  \hline
  8.4--8.6 & 1.5 \\
  \hline
  9.1--9.2 & 1 \\
  \hline
  10.1--10.5 & 2 \\
  \hline
  13.3--13.6 & 2.5 \\
  \hline
  11.1--11.4 & 2.5 \\
  \hline
  16.3--16.4 & 1 \\
  \hline
\end{tabular}

There will also be review material from chapters 0, 1, 4, and 5.  Students will be responsible for self-review of
this material; however, I will do some light review in class when needed.  The workshop and office hours are also
good places to obtain help with this review material.

\subsection*{Classroom Protocol}
  
\subsubsection*{Attendance}

I will not take attendance after the first week; however, it is important that you come (on time) to every
class. The book has more information than we could possibly cover, so I will highlight in class what is
important. Bring your book and calculator to every class meeting. If you miss a class, it is your responsibility to
talk to your peers and find out what you missed.

\subsubsection*{Holidays}

Class will not meet on the following days:

\begin{tabular}{ccl}
  Monday & 9/2 & Labor Day \\
  Monday & 11/11 & Veteran's Day \\
  Wednesday & 11/27 & Thanksgiving Break
\end{tabular}

\subsection*{University Policies}

Per University Policy S16-9 (\url{http://www.sjsu.edu/senate/docs/S16-9.pdf}), information relevant to all courses:
academic integrity, accommodations, dropping and adding, consent for recording of class, etc., is available on the
Office of Graduate and Undergraduate Programs’ Syllabus Information web page at
\url{http://www.sjsu.edu/gup/syllabusinfo}.  Please make sure to review these university policies and resources.

\end{document}
