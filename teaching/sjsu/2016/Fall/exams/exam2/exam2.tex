\documentclass[letterpaper,12pt,fleqn]{article}
\usepackage{matharticle}
\usepackage{tikz}
\pagestyle{plain}
\begin{document}

\begin{center}
\Large Math-8 Exam Chapter 2
\end{center}

\vspace{0.5in}

Name: \rule{4in}{1pt}

\vspace{0.5in}

This exam is closed book and notes. You may use a calculator; however, no cell
phones or tablets are allowed. Show all work; there is no credit for guessed
answers. All values should be exact unless you are specifically asked for an
approximate value answer.

\vspace{0.5in}

\newcommand{\fillin}{\rule{3in}{1pt}}

\begin{enumerate}
\item Identify each of the following formulas. Be very specific.
  Differentiate between general and standard forms and call out important
  points (like the center of a circle).

\vspace{0.25in}

\begin{tabular}{cc}
$x=\frac{-b\pm\sqrt{b^2-4ac}}{2a}$ & \fillin \\
\\
$d=[(x_1-x_2)^2+(y_1-y_2)^2]^{1/2}$ & \fillin \\
\\
$\left(\frac{x_1+x_2}{2},\frac{y_1+y_2}{2}\right)$ & \fillin \\
\\
$m=\frac{y_1-y_2}{x_1-x_2}$ & \fillin \\
\\
$y-y_1=m(x-x_1)$ & \fillin \\
\\
$y=mx+b$ & \fillin \\
\\
$m_1=m_2$ & \fillin \\
\\
$m_1m_2=-1$ & \fillin \\
\\
$(x-h)^2+(y-k)^2=r^2$ & \fillin \\
\\
$x^2+y^2+Dx+Ey+F=0$ & \fillin \\
\end{tabular}

\newpage

\item You have two dogs: Fido and Fluffy. Each dog is tied to its
own stake in your backyard by a leash. Fido's stake and leash allow him to roam
around an area defined by:
\[(x-2)^2+(y-1)^2=9\]
Fluffy's stake and leash allow her to roam around an area defined by:
\[x^2+y^2-10x-8y+37=0\]
\begin{enumerate}
\item What are the coordinates of Fido's stake and the length of his leash?

\vspace{1in}

\item What are the coordinates of Fluffy's stake and the length of her leash?

\vspace{3in}

\item What is the equation of the line between the two stakes, in
slope-intercept form?

\newpage

\item It is mating season. Fido and Fluffy and not fixed, but you do not want
them to mate. To be safe, you decide to erect a straight wall that is
perpendicular to the line joining the two stakes and going through the midpoint
of that line. What is the equation of the wall, in slope-intercept form?
\end{enumerate}

\newpage

\item Identify each of the following standard functions:

\vspace{0.25in}

\begin{tabular}{cc}
\begin{tikzpicture}
\draw [<->] (-2,0) -- (2,0);
\draw [<->] (0,-2) -- (0,2);
\draw [domain=0.5:1.9] plot (\x,{1/\x});
\draw [domain=-1.9:-0.5] plot (\x,{1/\x});
\node at (-1,-3) {$y=$};
\end{tikzpicture} \hspace{1in} &
\begin{tikzpicture}
\draw [<->] (-2,0) -- (2,0);
\draw [<->] (0,-2) -- (0,2);
\draw [domain=-2:2] plot (\x,{abs(\x)});
\node at (-1,-3) {$y=$};
\end{tikzpicture} \\
\end{tabular}

\begin{tabular}{cc}
\begin{tikzpicture}
\draw [<->] (-2,0) -- (2,0);
\draw [<->] (0,-2) -- (0,2);
\draw [domain=0:2] plot (\x,{sqrt(\x)});
\node at (-1,-3) {$y=$};
\end{tikzpicture} \hspace{1in} &
\begin{tikzpicture}
\draw [<->] (-2,0) -- (2,0);
\draw [<->] (0,-2) -- (0,2);
\draw [domain=0:2] plot (\x,{(\x)^(1/3)});
\draw [domain=-2:0] plot (\x,{-(-\x)^(1/3)});
\node at (-1,-3) {$y=$};
\end{tikzpicture} \\
\end{tabular}

\begin{tabular}{cc}
\begin{tikzpicture}
\draw [<->] (-2,0) -- (2,0);
\draw [<->] (0,-2) -- (0,2);
\draw [domain=-1.4:1.4] plot (\x,{(\x)^2});
\node at (-1,-3) {$y=$};
\end{tikzpicture} \hspace{1in} &
\begin{tikzpicture}
\draw [<->] (-2,0) -- (2,0);
\draw [<->] (0,-2) -- (0,2);
\draw [domain=-1.25:1.25] plot (\x,{(\x)^3});
\node at (-1,-3) {$y=$};
\end{tikzpicture} \\
\end{tabular}

\newpage

\item Consider the function:
\[f(x)=-(x-1)^2+4\]
\begin{enumerate}
\item Provide a list of transformations in the order that they should be
  applied. Remember to start with one of the standard functions.
\begin{enumerate}
\item
\item
\item
\item
\end{enumerate}

\item Determine any x-intercepts.

\vspace{2in}

\item Determine any y-intercepts.

\vspace{1in}

\item Sketch the function. Be sure to label all key points!
\newpage
\item On which interval(s) is the function increasing (if any)?

\vspace{1in}

\item On which interval(s) is the function decreasing (if any)?

\vspace{1in}

\item What are the maxima (if any)?

\vspace{1in}

\item What are the minima (if any)?

\vspace{1in}

\item What is the domain?

\vspace{1in}

\item What is the range?
\end{enumerate}
\newpage
\item Consider the function:
  \[h(x)=(x+1)^3+2(x+1)^2-3(x+1)+4\]
  \begin{enumerate}
    \item Determine two functions $f(x)$ and $g(x)$ such that:
      \[h(x)=(f\circ g)(x)\]
      You must specify something other than the trivial answer (i.e., neither of
      your functions can be just $x$).

      \vspace{2in}

    \item Using your answer for $g(x)$, calculate the difference quotient:
      \[\frac{g(x+h)-g(x)}{h}\]
  \end{enumerate}
\end{enumerate}
\end{document}
