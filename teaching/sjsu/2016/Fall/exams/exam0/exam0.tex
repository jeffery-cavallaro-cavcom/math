\documentclass[letterpaper,12pt,fleqn]{article}
\usepackage{matharticle}
\usepackage{tikz}
\pagestyle{plain}
\begin{document}

\begin{center}
\Large Math-8 Exam \#0
\end{center}

\vspace{0.5in}

Name: \rule{4in}{1pt}

\vspace{0.5in}

This exam is closed book and notes. You may use a calculator; however, no other
electronics are allowed. Show all work; there is no credit for guessed
answers. All answers should be in exact values, unless you are specifically
asked for an approximate value.

\vspace{1in}

\newcommand{\fillin}{\rule[-10pt]{2in}{1pt}}
\newcommand{\sfillin}{\rule[-10pt]{0.5in}{1pt}}

\begin{enumerate}
\item Identify each subset of the real numbers and give an example of an
  element from each set.

  \bigskip

\begin{tabular}{ccc}
\textbf{subset} & \textbf{name} & \textbf{example} \\
\\
$\N$ & \fillin & \sfillin \\
\\
$\Z$ & \fillin & \sfillin \\
\\
$\Q$ & \fillin & \sfillin \\
\\
$\R-\Q$ & \fillin & \sfillin \\
\\
$\R$ & \fillin & \sfillin \\
\end{tabular}

\newpage

\item Let:
  \[A=\{x\in\R\mid-2\le x<2\}\]
  \[B=\{x\in\R\mid x>1\}\]

  \bigskip
  
  \begin{enumerate}
  \item Graph $A$
    \vspace{1in}
    
  \item Graph $B$
    \vspace{1in}
    
  \item What is the interval notation for $A$?
    \vspace{0.5in}
    
  \item What is the interval notation for $B$?
    \vspace{0.5in}
    
  \item What is $A\cup B$ in interval notation?
    \vspace{1.5in}

  \item What is $A\cap B$ in interval notation?
    \vspace{1in}
  \end{enumerate}

\newpage
  
\item Identify each of the following real number axioms:

\vspace{0.25in}

\begin{tabular}{lc}
$\forall\,a,b\in\R,a+b\in\R$ & \fillin \\
\\
$\forall\,a,b\in\R,ab\in\R$ & \fillin \\
\\
$\forall\,a,b\in\R,a+b=b+a$ & \fillin \\
\\
$\forall\,a,b\in\R,ab=ba$ & \fillin \\
\\
$\forall\,a,b,c\in\R,(a+b)+c=a+(b+c)$ & \fillin \\
\\
$\forall\,a,b,c\in\R,(ab)c=a(bc)$ & \fillin \\
\\
$\exists\,0\in\R,\forall\,a\in\R,a+0=0+a=a$ & \fillin \\
\\
$\exists\,1\in\R,\forall\,a\in\R,a1=1a=a$ & \fillin \\
\\
$\forall\,a\in\R,\exists\,(-a)\in\R,a+(-a)=(-a)+a=0$ & \fillin \\
\\
$\forall\,a\in\R-\{0\},\exists\,a^{-1}\in\R,aa^{-1}=a^{-1}a=1$ & \fillin \\
\\
$\forall\,a,b,c\in\R,a(b+c)=ab+ac$ & \fillin \\
\\
$\forall\,a,b,c\in\R,(a+b)c=ac+bc$ & \fillin \\
\end{tabular}

\newpage

\item Complete each of the following exponent rules:

  \begin{large}
    \begin{tabular}{cc}
      $a^ma^n=$ & \fillin \\
      \\
      $(a^m)^n=$ & \fillin \\
      \\
      $\frac{a^m}{a^n}=$ & \fillin \\
      \\
      $a^{-1}=$ & \fillin \\
      \\
      $a^{-n}=$ & \fillin \\
      \\
      $\sqrt[n]{a}=$ & \fillin \\
      \\
      $\sqrt[q]{a^p}=$ & \fillin \\
    \end{tabular}
  \end{large}

  \vspace{0.5in}

\item Simplify the following. Your answer should contain \emph{no} radicals
  and \emph{no} negative exponents.
  \begin{large}
    \[\frac{z\sqrt{x^2yz^{\frac{1}{2}}}}{xy^{\frac{1}{2}}z^2}\]
  \end{large}

\newpage

\item Complete the following expansion and factoring rules:

  \bigskip

  \begin{large}
    \begin{tabular}{cc}
      $(a-b)^2=$ & \fillin \\
      \\
      $a^2-b^2=$ & \fillin \\
      \\
      $a^2+2ab+b^2=$ & \fillin \\
      \\
      $a^3+3a^2b+3ab^2+b^3=$ & \fillin \\
    \end{tabular}
  \end{large}

  \vspace{1in}
  
\item Expand the following:
  \begin{large}
    \[(x\sqrt{y}+2z)^2\]
  \end{large}

\newpage
  
\item Factor each of the following by inspection:
  \begin{enumerate}
  \item $x^2+2x+1$
    \vspace{0.5in}
    
  \item $x^2-2x+1$
    \vspace{0.5in}

  \item $x^2+4x-21$
    \vspace{0.5in}

  \item $x^2-4x-21$
    \vspace{0.5in}

  \item $4x^2-11x-3$
    \vspace{2in}
  \end{enumerate}

\item Complete each of the following fraction rules:

  \bigskip

  \begin{large}
    \begin{tabular}{cc}
      $\frac{a}{c}+\frac{b}{c}=$ & \fillin \\
      \\
      $\frac{a}{b}+\frac{c}{d}=$ & \fillin \\
      \\
      $\frac{a}{b}\cdot\frac{c}{d}=$ & \fillin \\
      \\
      $\frac{\frac{a}{b}}{\frac{c}{d}}=$ & \fillin \\
    \end{tabular}
  \end{large}

\newpage
  
\item Consider the following:
  \begin{large}
    \[\frac{x-1}{x+2}\left(\frac{1}{x-1}-\frac{1}{x}\right)\]
  \end{large}

  \begin{enumerate}
  \item Simplify the expression.
    \vspace{4in}
    
  \item State the domain in setbuilder notation.
    \vspace{1in}

  \item Graph the domain.
    \vspace{1in}
    
  \item State the domain in interval notation.
  \end{enumerate}
  
\end{enumerate}

\end{document}
