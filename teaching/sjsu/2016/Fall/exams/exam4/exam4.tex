\documentclass[letterpaper,12pt,fleqn]{article}
\usepackage{matharticle}
\pagestyle{plain}
\begin{document}

\begin{center}
  \Large Math-8 Exam Chapter 4
\end{center}

\vspace{0.5in}

Name: \rule{4in}{1pt}

\vspace{0.5in}

This is a take-home exam. You may use your books, notes, and calculator;
however, you are not allowed to work together or get outside help. Show all
work; there is no credit for guessed answers. All values should be exact
(no decimals) unless you are specifically asked for an approximate value
answer. All domains and ranges should be expressed in interval notation.

\begin{textbf}
  Do the exam on the front side only of $8\times11$ college rule or
  graph paper. Staple all pages and make sure that your name is on the first
  page. Treat this like you would a term paper!

  Your final version is due at 9:30am, Saturday, December 17, just prior to
  the final exam in room MH324. Late papers or papers that do not meet the
  submission guidelines will not be accepted.
\end{textbf}

\vspace{0.5in}

\begin{enumerate}
\item Consider the following exponential function:
  \[y=2e^{-x}-3\]
  \begin{enumerate}
  \item What are the x-intercepts (if any)?
  \item What are the y-intercepts (if any)?
  \item Where is the horizontal asymptote (if any)?
  \item Sketch the graph. Be sure to label all key points and asymptotes
    for full credit.
  \item Determine the inverse function.
  \end{enumerate}

\item Take the natural log of both sides and fully expand the result. There
  should be \emph{no} exponents left in your answer.
  \[y=\frac{\sqrt{(x+1)e^x}}{x(x-3)}\]

\item Given the following:
  \[\log_b{2}=0.4307\]
  \[\log_b{3}=0.6826\]
  \[\log_b{7}=1.2091\]
  calculate $\log_b{126}$ (without determining the base $b$).
  
\item Some archaeologists are digging at what appears to be a pre-Columbian
  human campsite in California.  They find some animal bones with human teeth
  marks on them.  Upon carbon-14 analysis, it is found that the bones have 75\%
  of their original $C^{14}$.  About how old are the bones, and hence the
  campsite?  The half-life of $C^{14}$ is 5730 years.
\end{enumerate}
\end{document}
