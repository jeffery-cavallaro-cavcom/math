\documentclass[letterpaper,12pt,fleqn]{article}
\usepackage{matharticle}
\usepackage{url}
\pagestyle{plain}
\begin{document}

\begin{center}
\emph{San Jos\'{e} State University}

\Large{Math-08 (College Algebra)}\normalsize

\large{Fall-2016}\normalsize

Section 01: MW 9:00am--10:15am \\
Section 04: MW 1:30pm--2:45pm \\

Duncan Hall 416
\end{center}

\vspace{0.5in}

\begin{description}

\item[Instructor:] Jeffery Cavallaro (\url{jeffery.cavallaro@sjsu.edu})

\item[Office:] Duncan Hall 209 (the TA room)

\item[Office Hours:] MW 10:30am-11:45am, F 9am-12 noon by appointment

\item[Texts:] \emph{College Algebra and Calculus: An Applied Approach},
  Larson and Hodgkins, \textbf{2nd edition}.  Make sure that you have the
  second edition and not the first. Since we will be using the book in class,
  and since this is the same book that is used in Math-71, I highly recommend
  that you purchase (not rent) an actual book and not rely on an eBook.

\item[Web:] We will use both canvas and webassign. All class communications,
  including reading assignments, homework assignments, helpful resource
  documents, and grades, are via canvas (\url{sjsu.instructure.com}). Webassign
  (\url{webassign.com}) will be used for a portion of the homework (see below).
  Once you are registered for the course you should be able to see the course
  listed on your canvas account. Each student must purchase a webassign
  license. Note that the ``enhanced'' license is not required; however, you
  may find the extra teaching materials that it provides helpful. The
  webassign class code will be distributed on the first day of class and via
  a canvas announcement. Once you register your license, you will need this
  class code to access the class.

\item[Calculator:] You are required to have a TI-83 or 84 graphing calculator.
  If you are buying a new one, I suggest the TI-84 Plus CE. You will also need
  a cable so that you can connect your calculator to a computer - you will turn
  in screenshots as part of your homework. The newer models use a USB cable
  (included in the package). The older models use a special datalink cable,
  purchased separately. You will also need to have the TIconnect software,
  available on the TI website, loaded on your computer. I have \emph{no}
  problem with you checking your homework and exam
  answers using your calculator --- in fact, I encourage it; however,
  answers with no supporting work will receive zero credit.
  \emph{No other scientific calculators, cell phones, tablets, or computers
    are allowed in lieu of a TI-83/4 calculator!}

\item[Math 8W.] You are strongly encouraged to register and participate in the
  workshop that corresponds to this class.  Math 8W is a 1 unit class (two
  sessions/week) where you will work in teams on problems, aided by a workshop
  facilitator.  Although you are not required to take Math 8W, experience has
  shown that students who take the workshop are more successful than those who
  do not.

\item[Learning Objectives.] This is a preparatory class for Math-71 Business
  Calculus. In this course you will master basic algebra skills, understand and
  apply fundamental ideas about functions, and study some specific types of
  functions (e.g., polynomials, exponentials, logarithms).  You will use
  mathematical methods to solve quantitative (word) problems and arrive at
  conclusions based on numerical and graphical data.  These learning
  objectives, as well as the minimum 500 word writing requirement (homework
  and exams), satisfy the Area B4 (Mathematical Concepts) GE requirement.

\item[Attendance:] I will not take attendance after the first week; however, it
  is vitally important that you come (on time) to every class. The book has
  more information than we could possibly cover, so I will highlight in class
  what is important. I will also enhance certain subjects that I feel are
  important for your calculus preparation. Bring your book and calculator to
  every class meeting. If you miss a class, it is your responsibility to talk
  to your peers and find out what you missed.

\item[Time:] You will probably need to spend a \emph{minimum} of 10 hours per
  week outside of class doing homework and studying. This class is \emph{very}
  intensive and will require disciplined study habits. Please, please, please
  do \emph{not} register for 16 units and/or commit to a 20+ hours per week
  job; if you do then your chances of passing this class drop exponentially!

\item[Holidays.] Class will not meet on 9/5 (Labor Day) or 11/23 (Day before
  Thanksgiving). Please note that the Monday of Thanksgiving week is
  \emph{not} a holiday and is a likely exam day candidate.

\item[Reading:] Reading from the textbook will be assigned each Friday for the
  material to be covered in the coming week. Please read everything, not
  just the stuff in the boxes, prior to lecture. Make sure that you can work
  all of the example problems prior to attempting any of the homework problems.

\item[WebAssing Homework:] The web-based homework will be submitted via
  webassign. Webassign requires that you format your answers with math symbols
  using their answer tool. Don't get frustrated! It may take a couple times for
  you to get the hang of it; it will get easier the more you use it. The
  problems assigned on webassign are even problems from the book. WebAssign
  homework for each chapter will be due at midnight prior to the corresponding
  chapter exam. There are no extensions, so please do not fall behind.

\item[Written Homework:] In addition to the web-based homework, you will be
  required to turn in a small set of written homework problems that I will
  assign approximately each week via canvas. Whereas the web-based problems are
  typically based on single concepts, the written homework will combine
  concepts and will need a little more thought. Homework will be assigned each
  week on Monday and is due on the following Monday at the start of class. Late
  homework will not be accepted; however, I will only count your top ten
  homework scores. See \emph{Homework Rules} for more information.

\item[Exams:] There will be five end-of-chapter exams for chapters 0-4. We will
  cover part of chapter 5 as well, but the chapter 5 material will be tested as
  part of the final exam (see below). Each chapter exam is scheduled for one
  week after we cover the chapter material in class. Prior to an exam, I will
  post an announcement on canvas telling you exactly what to expected on the
  exam. All exams are closed book and notes. A calculator (as described above)
  is allowed; however, as noted above, any answers without supporting work
  (i.e., guesses or copying an answer directly from your calculator) receive
  zero credit. Instead of a note card, I will provide all needed formulas,
  except for those that I explicitly ask that you memorize.

\item[Final.] The final exam is cumulative and is scheduled for
  \textbf{Saturday, 12/17, 9:45am--noon}.  All sections of Math-8 are taking the
  same final at the same time, and the exam \emph{must} be taken at that time.
  Any and all excuses must be discussed directly with the math office (MH-308).
  Once again, travel arrangements are not a valid excuse, so don't plan to leave
  town prior to noon on 12/17.  The final exam follows the same rules as the
  exams; however, a $3\times 5$ notecard is allowed.  Sample finals are
  available on canvas.

\item[Grading.]  Your semester grade is determined as follows:

  \bigskip

  \begin{minipage}{3in}
    \begin{tabular}{|c|c|}
      \hline
      WebAssign Homework & 25\% \\
      \hline
      Written Homework & 25\% \\
      \hline
      Chapter Exams & 25\% \\
      \hline
      Final Exam & 25\% \\
      \hline
    \end{tabular}
  \end{minipage}
  \begin{minipage}{3in}
    \begin{tabular}{|c|c|}
      \hline
      A+ & 97--100 \\
      \hline
      A & 93--96 \\
      \hline
      A- & 90--92 \\
      \hline
      B+ & 87--89 \\
      \hline
      B & 83--86 \\
      \hline
      B- & 80--82 \\
      \hline
      C+ & 77--79 \\
      \hline
      C & 73--76 \\
      \hline
      C- & 70--72 \\
      \hline
      D & 60--69 \\
      \hline
      F & 0-59 \\
      \hline
    \end{tabular}
  \end{minipage}
  
\item[Credit:] A grade of C or better meets the \emph{Area B4: Mathematical
    Concepts} GE requirement. A grade of C- or better is required for placement
    into Math 71.

\item[Tutoring:] Peer tutoring is available to all SJSU students, free of
    charge, from the PeerConnections center. See
    \url{http://peerconnections.sjsu.edu} for more information.

\item[Academic integrity:] Your commitment to learning (as shown by your
    enrollment at SJSU) and SJSU's Academic Integrity Policy require you to be
    honest in all of your academic course work.  Faculty are required to report
    all infractions to the Office of Student Conduct and Ethical Development.
    See \url{http://www.sjsu.edu/studentconduct} for more information.

\item[Disabilities:] If you need course adaptations or accommodations due to a
    disability, or if you need special arrangements in case the building must
    be evacuated, please make an appointment with me as soon as possible. All
    students with disabilities must register with the Accessible Education
    Center (AEC) to establish a record of their disability. See
    \url{http://www.sjsu.edu/aec} for more information.

\end{description}

\end{document}
