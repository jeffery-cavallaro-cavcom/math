\documentclass[letterpaper]{article}

\setlength{\textwidth}{5.5in}
\setlength{\oddsidemargin}{0.5in}
\setlength{\evensidemargin}{0.5in}

\pagestyle{empty}

\setlength{\topmargin}{0in}
\setlength{\topskip}{0in}
\setlength{\headsep}{0in}
\setlength{\footskip}{0in}
\setlength{\textheight}{9in}
\addtolength{\textheight}{-\headheight}
\addtolength{\textheight}{-\headsep}
\addtolength{\textheight}{-\footskip}

\begin{document}

\begin{center}\bf
College Algebra (Math 8), Fall 2016, San Jos\'{e} State University \\
Duncan Hall 416, TuTh 3:00pm - 4:15pm Section 12 Code: 43814
\end{center}

\begin{description}
\setlength{\itemsep}{0in}
\item[Instructor:] Alan Ghazarians
\item[Office and phone:] Duncan Hall 209
\item[Office hours:] TBD
\item[E-mail:] {\tt alan.ghazarians@sjsu.edu}
\item[Course web page:] I'll be posting everything on Canvas. 
\item[Texts:] {\it College Algebra and Calculus: An Applied Approach},
	Larson and Hodgkins, {\bf 2nd edition}.  Make sure you have the second
	edition and not the first.
\item[Calculator:] You should have a ``scientific calculator'' (e.g.,
  TI-30X) for use on exams.  You are {\it not\/} allowed to use either
  a graphing calculator (e.g., TI-83) or a calculator that can do symbolic
  algebra (e.g., TI-89 or TI-92) on exams.
\end{description}

{\bf Learning objectives.}  In this course, you will master the
algebra skills required for later math classes; understand and apply
fundamental ideas about functions; and study some
specific types of functions.
Also, as with any GE math course,
you will use mathematical methods to solve quantitative problems,
including real-life problems, and arrive at conclusions based on
numerical and graphical data.

You will achieve these learning objectives, as well as the minimum
writing requirement of 500 words for a GE class, in homework, quizzes, and exams.

{\bf Math 8W Workshops.\/} Math 8W is a 1 unit class (two sessions/week)
in which you will work in teams on problems, aided by
workshop facilitators.  You are not required to take Math 8W, but
our experience has been that students in workshops are more
successful than students who are not.

{\bf Class.}  Bring your textbook and calculator to class every day.
Class will consist of lecture, group activities, and discussion.
Please silence all cellphones before class.

{\bf Time expectations.\/} You should be aware that the standard
expectation for
a 3-unit class at SJSU is that you will spend at least 9 hours
per week working on this class (i.e., at least 6 hours outside
the classroom).

{\bf Reading.}  Do the assigned reading before anything else, i.e.,
before the topics come up in class or in the homework.  Read {\it
  all\/} of the text, and not just ``the stuff in the boxes.''  The
course web page will always have a complete list of all reading
assigned to date.

{\bf Homework.}  Written homework will be due every class day, except for exam
days; for more details, see the handout on homework.  Specific
assignments will be determined as the term progresses.  The 3 lowest homework 
grades will be dropped from your overall grade.

{\bf Webassign.}  I have set up a Webassign account that has lots of problems
from each chapter that you can do for practice on your own time. The assignments will
not count for anything, but they are there for your own benefit if you need extra problems. 
Your add code for Webassign is: {\bf sjsu 6767 7737}

{\bf Quizzes.}  Once a week, except exam weeks, we will have a
closed-book in-class quiz (no notes, calculators OK).  Our first quiz
will be on {\bf Tuesday Aug 30}. The 2 lowest quiz grades will be dropped
from your overall grade.

\newcommand{\finaldate}{Sat Dec 17}
\newcommand{\finaltime}{9:45am--noon}

{\bf Exams.}  The exam dates are described in the syllabus below.
{\it You must plan to take the exams at their scheduled times}.  In
particular, previous travel arrangements are not a valid excuse, so do
{\it not\/} leave campus before the common final on {\bf\finaldate},
{\bf\finaltime}.

{\bf Exam policies.}  All exams will be closed-book.  Calculators are
allowed (though not the TI-89, TI-92, and similar calculators), and
you are also allowed to bring one $3\times 5$ card of notes.  Exams
are primarily based on the reading and the homework, so the best way
to prepare for exams is to do all of the reading and the homework.

{\bf Grading.}  Semester grade is: Homework 15\%, Quizzes 5\%, Exam 1
10\%, Exams 2 and 3 20\% each, and Final exam 30\%.  Grades given:
A, A--, B+, B, B--, C+, C, C--, D+, D, D--, F.  Letter grades are
adjusted by difficulty of exams, and do not correspond to a fixed
percentage; details will be given as the semester proceeds.  Note
that you need to get a C or better to satisfy the Area B4 (Mathematical
Concepts) GE
requirement.

\newcommand{\adddate}{Tue Sep 13}
\newcommand{\dropdate}{Tue Sep 06}

{\bf How to add this course.\/} If you are not registered for this
course, and you would like to add it, you must first put a full effort
into completing all of the work in the course.  Second, if you are a
graduating senior, you need to produce documentation to verify that.

I'll make a waiting list, which you get on by filling out and turning
in the information form for the course.  I'll give out add codes
starting {\bf\dropdate\/} (or possibly earlier), mainly based on
completeness of homework, and as long as there is room, I will
continue to give out add codes until add date ({\bf\adddate\/}).
Note, however, that graduating seniors have the highest priority, and
that Open University students have the lowest priority.

{\bf How to drop this course.\/} Until {\bf\dropdate}, you can drop at
{\tt my.sjsu.edu}.  Nothing will appear on your transcript, but please
let me know if you drop.

To drop after \dropdate, you must go to the student services center
and submit a Course Drop form to the Director of Academic Services.
Dropping under these circumstances is only allowed for ``serious and
compelling reasons'' (University policy).  A low grade is not a serious
and compelling reason.

{\bf Academic integrity.\/} Your commitment to learning (as shown by
your enrollment at SJSU) and SJSU's Academic Integrity Policy require
you to be honest in all of your academic course work.  Faculty are
required to report all infractions to the Office of Student Conduct
and Ethical Development.  See: {\tt www.sjsu.edu/studentconduct\/}

{\bf Disabilities.\/} If you need course adaptations or accommodations
due to a disability, or if you need special arrangements in case the
building must be evacuated, please make an appointment with me as soon
as possible, or see me during office hours.  Presidential Directive
97-03 requires that students with disabilities register with the
Accessible Education Center
(formerly the Disability Resources Center) to establish a record of their
disability.

\bigskip


\centerline{\bf Tentative Syllabus}

\begin{center}
\begin{tabular}{|r|p{1.5in}||r|p{1.5in}|}\hline
Date & Reading & Date & Reading \\ \hline
 &  & 
  Tu Oct 25 & 3.2 \\
Th Aug 25 & Introduction and Ch.\ 0 &
	Th Oct 27 & 3.3 \\ \hline
%
Tu Aug 30 & 1.1 &
	Tu Nov 01 & 3.4 \\
Th Sep 1 & 1.2 &
	Th Nov 03 & Review \\ \hline
%
Tu Sep 06 & 1.3 &
	Tu Nov 08 & 4.1 \\
Th Sep 08 & 1.4 &
	Th Nov 10 & 4.2 \\ \hline
%
Tu Sep 13 & 1.5 &
	Tu Nov 15 & 4.3 \\
Th Sep 15 & 1.6 &
	Th Nov 17 & 4.4 \\ \hline
%
Tu Sep 20 & 1.7 &
	Tu Nov 22 & {\bf Exam 3\/} \\
Th Sep 22 & {\bf Exam 1\/} &
	Th Nov 24 & {\bf NO CLASSES\/} \\ \hline
%
Tu Sep 27 & 2.1 &
	Tu Nov 29 & 4.5 \\
Th Sep 29 & 2.2 &
	Th Dec 01 & 5.1 \\ \hline
%
Tu Oct 04 & 2.4 &
	Tu Dec 06 & 5.2 \\
Th Oct 06 & 2.5 &
	Th Dec 08 & 5.3 \\ \hline
%
Tu Oct 11 & 2.6 &
  Tu Dec 13 & {\bf NO CLASSES\/}  \\
Th Oct 13 & 2.7 &
   &  \\ \hline
%
Tu Oct 18 & 3.1 &
	{\bf \finaldate\/} & {\bf FINAL EXAM} \\
Th Oct 20 & {\bf Exam 2\/} &
	& {\bf \finaltime\/} \\ \hline
\end{tabular}
\end{center}

{\bf Tutoring.\/} Peer tutoring is available to all SJSU
students, free of charge, through:
\begin{itemize}
\item The College of Science Advising Center ({\tt www.sjsu.edu/cosac\/})
\item Peer Connections ({\tt peerconnections.sjsu.edu\/})
\end{itemize}
\end{document}
