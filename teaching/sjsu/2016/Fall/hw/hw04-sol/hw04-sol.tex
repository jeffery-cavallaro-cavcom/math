\documentclass[letterpaper,12pt,fleqn]{article}
\usepackage{matharticle}
\usepackage{mathtools}
\pagestyle{plain}
\begin{document}

\begin{center}
\Large Math-08 Homework \#4 Solutions
\end{center}

\vspace{0.5in}

\underline{Reading}

\begin{itemize}
\item Text book section 0.5, 0.6, and 0.7
\end{itemize}

\underline{Problems}

\begin{enumerate}
\item Expand the following. Your answer should contain \emph{no} radicals.
  \begin{eqnarray*}
    (4x^2-3y)(x^{\frac{1}{3}}+2\sqrt{y}) &=&
    (4x^2-3y)(x^{\frac{1}{3}}+2y^{\frac{1}{2}}) \\
    &=& (4x^2)(x^{\frac{1}{3}})+(4x^2)(2y^{\frac{1}{2}})+(-3y)(x^{\frac{1}{3}})+
    (-3y)(2y^{\frac{1}{2}}) \\
    &=& 4x^{\frac{7}{3}}+8x^2y^{\frac{1}{2}}-3x^{\frac{1}{3}}y-6y^{\frac{3}{2}} \\
  \end{eqnarray*}

\item Explain why the following is incorrect, and then state what it should
  be:
  \[(2x+3y)^2=4x^2+9y^2\]

  Never distribute an exponent across addition! This needs to be FOIL'ed:
  \[(2x+3y)^2=(2x)^2+2(2x)(3y)+(3y)^2=4x^2+12xy+9x^2\]

\item Factor the following. Your answer should contain \emph{no} negative
  exponents:
  \begin{eqnarray*}
    x^{\frac{5}{3}}-2x^{\frac{2}{3}}+x^{-\frac{1}{3}} &=&
    x^{-\frac{1}{3}}\left(x^{\frac{5}{3}+\frac{1}{3}}-2x^{\frac{2}{3}+\frac{1}{3}}+1\right) \\
    &=& x^{-\frac{1}{3}}(x^2-2x+1) \\
    &=& x^{-\frac{1}{3}}(x-1)^2 \\
    &=& \frac{(x-1)^2}{x^{\frac{1}{3}}} \\
    &=& \frac{(x-1)^2}{\sqrt[3]{x}} \\
  \end{eqnarray*}

\item Consider:
  \[\frac{x+1}{x+2}\left(\frac{1}{x}+\frac{2}{x+1}-\frac{x-5}{x-2}\right)\]
  \begin{enumerate}
  \item Combine into a single, simplified rational expression.

    Work on the inside first. We need a common denominator, which will be
    $x(x+1)(x-2)$. We adjust each numerator for the common denominator by
    multiplying by the missing factors:
    \[\frac{x+1}{x+2}\left[\frac{(x+1)(x-2)+2x(x-2)-x(x-5)(x+1)}{x(x+1)(x-2)}
      \right]\]

    Now, do the necessary FOIL'ing in the numerator. Be careful of the minus
    sign in the last term!
    \[\frac{x+1}{x+2}\left[\frac{(x^2-x-2)+(2x^2-4x)-(x^3-4x^2-5x)}{x(x+1)(x-2)}
      \right]\]

    Now, combine like terms in the numerator:
    \[\frac{x+1}{x+2}\left[\frac{-x^3+7x^2-2}{x(x+1)(x-2)}\right]\]

    We can now do the multiplication. Note that the $(x+1)$ factor cancels;
    however, we make a little note to ourselves that $x\ne-1$ for when we
    determine the domain at the end:
    \[\frac{-x^3+7x^2-2}{x(x+1)(x-2)(x+2)}\]

    We can leave it like this. Don't bother to multiply out the denominator -
    the factored form gives us more information. One optional step is to
    factor out the negative in the numerator:
    \[-\frac{x^3-7x^2+2}{x(x-2)(x+2)}\]
    
  \item State the domain.

    A zero denominator is not allowed:
    \[x(x-2)(x+2)\ne0\]
    We use the property of zero that says if you multiply a bunch of factors
    and the result in non-zero, then none of the factors is zero:

    $x\ne0$ \\
    $x-2\ne0$ \\
    $x+2\ne0$ \\

    Which results in $x\ne0,\pm2$. But don't forget, we eliminated $x=-1$
    earlier as well. So the final domain is:
    \[\{x\in\R\mid x\ne0,-1,\pm2\}\]
  \end{enumerate}
\end{enumerate}
  
\end{document}
