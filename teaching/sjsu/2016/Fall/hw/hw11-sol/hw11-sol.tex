\documentclass[letterpaper,12pt,fleqn]{article}
\usepackage{matharticle}
\usepackage{enumitem}
\pagestyle{plain}
\begin{document}

\begin{center}
\Large Math-08 Homework \#11 Solutions
\end{center}

\vspace{0.5in}

\underline{Reading}

\begin{itemize}
\item Text book section 2.4-2.7
\end{itemize}

\underline{Problems}

Note that all sketches of graphs must have all found intercepts and
discontinuities labeled. All domains and ranges must be expressed in
interval notation. Remember, sketches do not have to be to scale!

\begin{enumerate}
\item Consider the following piecewise function:
  \[f(x)=\begin{cases}
  x, & (-4,0) \\
  x^2, & (0,2) \\
  \end{cases}\]
  \begin{enumerate}
  \item Sketch the graph for $f(x)$.

    \begin{tikzpicture}[scale=0.5]
      \draw [<->] (-12,0) -- (12,0);
      \draw [<->] (0,-12) -- (0,12);
      \draw [fill=black] (-4,-4) circle [radius=0.1]
      node [below] {$(-4,-4)$};
      \draw [fill=black] (0,0) circle [radius=0.1]
      node [below right] {$(0,0)$};
      \draw [fill=black] (2,4) circle [radius=0.1]
      node [above] {$(2,4)$};
      \draw (-4,-4) -- (0,0) parabola (2,4);
    \end{tikzpicture}
    
  \item List the transformations for $g(x)=-2f(x-1)+3$ in the proper order.

    \begin{enumerate}[label={\arabic*)}]
    \item Start with $f(x)$
    \item Translate right by 1
    \item Scale by 2
    \item Reflex across x-axis
    \item Translate up 3
    \end{enumerate}
    
  \item Sketch the graph for $g(x)$.

    We start with $f(x)$ as above and then apply the first translation:
    right by 1:
    
    \begin{tikzpicture}[scale=0.5]
      \draw [<->] (-12,0) -- (12,0);
      \draw [<->] (0,-12) -- (0,12);
      \draw [fill=black] (-3,-4) circle [radius=0.1]
      node [below] {$(-3,-4)$};
      \draw [fill=black] (1,0) circle [radius=0.1]
      node [below right] {$(1,0)$};
      \draw [fill=black] (3,4) circle [radius=0.1]
      node [above] {$(3,4)$};
      \draw (-3,-4) -- (1,0) parabola (3,4);
    \end{tikzpicture}

    Next, we scale by 2. Note that this affects the y-coordinate only:
    
    \begin{tikzpicture}[scale=0.5]
      \draw [<->] (-12,0) -- (12,0);
      \draw [<->] (0,-12) -- (0,12);
      \draw [fill=black] (-3,-8) circle [radius=0.1]
      node [below] {$(-3,-8)$};
      \draw [fill=black] (1,0) circle [radius=0.1]
      node [below right] {$(1,0)$};
      \draw [fill=black] (3,8) circle [radius=0.1]
      node [above] {$(3,8)$};
      \draw (-3,-8) -- (1,0) parabola (3,8);
    \end{tikzpicture}

    Now, reflect across the x-axis: all $y$ values become $-y$:

    \begin{tikzpicture}[scale=0.5]
      \draw [<->] (-12,0) -- (12,0);
      \draw [<->] (0,-12) -- (0,12);
      \draw [fill=black] (-3,8) circle [radius=0.1]
      node [below] {$(-3,8)$};
      \draw [fill=black] (1,0) circle [radius=0.1]
      node [above right] {$(1,0)$};
      \draw [fill=black] (3,-8) circle [radius=0.1]
      node [above] {$(3,-8)$};
      \draw (-3,8) -- (1,0) parabola (3,-8);
    \end{tikzpicture}

    Finally, translate up by 3 - this also only affects the y-coordinates:

    \begin{tikzpicture}[scale=0.5]
      \draw [<->] (-12,0) -- (12,0);
      \draw [<->] (0,-12) -- (0,12);
      \draw [fill=black] (-3,11) circle [radius=0.1]
      node [below] {$(-3,11)$};
      \draw [fill=black] (1,3) circle [radius=0.1]
      node [above right] {$(1,3)$};
      \draw [fill=black] (3,-5) circle [radius=0.1]
      node [above] {$(3,-5)$};
      \draw (-3,11) -- (1,3) parabola (3,-5);
    \end{tikzpicture}

  \item What are the $x$ and $y$ intercepts for $g(x)$ (if any)?

    When determining the intercepts, we need to know on which part of the
    piecewise graph they occur. This is evident from our sketch.

    To find the y-intercept, which occurs on the linear part, set $x=0$:

    $g(0)=-2f(0-1)+3=-2f(-1)+3=-2(-1)+3=2+3=5$

    Thus, the y-intercept is (0,5).

    To find the x-intercept, which occurs on the parabola, set $y=0$:

    \begin{eqnarray*}
      -2f(x-1)+3 &=& 0 \\
      -2(x-1)^2+3 &=& 0 \\
      2(x-1)^2 &=& 3 \\
      (x-1)^2 &=& \frac{3}{2} \\
      x-1 &=& \pm\sqrt{\frac{3}{2}} \\
      x &=& 1\pm\sqrt{\frac{3}{2}} \\
    \end{eqnarray*}

    Note that we only need the positive one:
    $\left(1+\sqrt{\frac{3}{2}},0\right)$

    So the final sketch looks like this:
    
    \begin{tikzpicture}[scale=0.5]
      \draw [<->] (-12,0) -- (12,0);
      \draw [<->] (0,-12) -- (0,12);
      \draw [fill=black] (-3,11) circle [radius=0.1]
      node [below] {$(-3,11)$};
      \draw [fill=black] (1,3) circle [radius=0.1]
      node [above right] {$(1,3)$};
      \draw [fill=black] (3,-5) circle [radius=0.1]
      node [above] {$(3,-5)$};
      \draw [fill=black] (0,5) circle [radius=0.1]
      node [left] {$(0,5)$};
      \draw [fill=black] ({1+sqrt(3/2)},0) circle [radius=0.1]
      node [above right] {$(1+\sqrt{\frac{3}{2}},0)$};
      \draw (-3,11) -- (1,3) parabola (3,-5);
    \end{tikzpicture}

  \item What are the domain and range of $g(x)$?

    Domain: $[-3,3]$ \\
    Range: $[-5,11]$
  \end{enumerate}

\item Consider the function:
  \[h(x)=\sqrt{x+1}-3\]
  \begin{enumerate}
  \item Write $h(x)$ as a composition of two functions $f\circ g$, neither of
    which is just $x$.

    The two most straightforward answers would be:

    $f(x)=\sqrt{x}-3$ and $g(x)=x+1$ or
    
    $f(x)=x-3$ and $g(x)=\sqrt{x+1}$
    
  \item Determine the $x$ and $y$ intercepts for $h(x)$ (if any).

    For the y-intercept, set $x=0$:
    \[y=\sqrt{0+1}-3=\sqrt{1}-3=1-3=-2\]
    So the y-intercept is $(0,-2)$

    For the x-intercept, set $y=0$:
    \[0=\sqrt{x+1}-3\]
    \[\sqrt{x+1}=3\]
    \[x+1=9\]
    \[x=8\]
    So the x-intercept is $(8,0)$
    
  \item Sketch the graph for $h(x)$.

    The transformations for this graph are:
    \begin{enumerate}[label={\arabic*)}]
    \item Start with $y=\sqrt{x}$
    \item Translate left by 1
    \item Translate down by 3
    \end{enumerate}

    This graph is fairly simple, so let's just transform the key point at
    $(0,0)$ and rely on our found intercepts:
    
    \begin{tikzpicture}[scale=0.5]
      \draw [<->] (-10,0) -- (10,0);
      \draw [<->] (0,-10) -- (0,10);
      \draw [fill=black] (-1,-3) circle [radius=0.1];
      \draw [fill=black] (0,-2) circle [radius=0.1];
      \draw [fill=black] (8,0) circle [radius=0.1];
      \draw [domain=-1:10] plot({\x},{sqrt(\x+1)-3});
      \node [below left] at (-1,-3) {$(-1,-3)$};
      \node [above left] at (0,-2) {$(0,-2)$};
      \node [above left] at (8,0) {$(8,0)$};
    \end{tikzpicture}
    
  \item Determine the domain and range for $h(x)$.

    Domain: $[-1,\infty)$ \\
    Range: $[-3,\infty)$
  \end{enumerate}

\item Consider the function:
  \[f(x)=\frac{1}{x-2} + 1\]
  \begin{enumerate}
  \item List the transformations, starting with one of the standard
    functions.

    \begin{enumerate}[label={\arabic*)}]
    \item Start with $y=\frac{1}{x}$
    \item Translate right 2
    \item Translate up 1
    \end{enumerate}

  \item Determine the $x$ and $y$ intercepts for $f(x)$ (if any).
    \[0=\frac{1}{x-2}+1\]
    \[\frac{1}{x-2}=-1\]
    \[-(x-2)=1\]
    \[-x+2=1\]
    \[x=1\]
    So the x-intercept is at $(1,0)$
    \[y=\frac{1}{0-2}+1=-\frac{1}{2}+1=\frac{1}{2}\]
    So the y-intercept is at $\left(0,\frac{1}{2}\right)$
    
  \item Sketch the graph for $f(x)$.

    The trick here is to determine how the asymptotes move. The vertical
    asymptote moves with the horizontal translation from $x=0$ to $x=2$.
    The horizontal asymptote at moves with the vertical translation from
    $y=0$ to $y=1$:
    
    \begin{tikzpicture}
      \draw [<->] (-5,0) -- (5,0);
      \draw [<->] (0,-5) -- (0,5);
      \draw [dashed] (2,5) -- (2,-5);
      \draw [dashed] (-5,1) -- (5,1);
      \draw [fill=black] (1,0) circle [radius=0.1];
      \draw [fill=black] (0,0.5) circle [radius=0.1];
      \node [below] at (1,0) {$(1,0)$};
      \node [left] at (0,0.5) {$\left(0,\frac{1}{2}\right)$};
      \node [below right] at (2,0) {$2$};
      \node [above right] at (0,1) {$1$};
      \draw [domain=-5:1.83] plot({\x},{1/(\x-2)+1});
      \draw [domain=2.25:5] plot({\x},{1/(\x-2)+1});
    \end{tikzpicture}

  \item Determine the domain and range for $f(x)$.

    Note that the domain has a hole at $x=2$ and the range has a hole at
    $y=1$:
    
    Domain: $(-\infty,2)\cup(2,\infty)$ \\
    Range: $(-\infty,1)\cup(1,\infty)$
  \end{enumerate}

\item Consider the following two functions:
  \[f(x)=\sqrt{x}+1\]
  \[g(x)=x^2\]
  Determine the following and state the domain for each:
  \begin{enumerate}
  \item $f+g$

    $(f+g)(x)=\sqrt{x}+1+x^2$

    The domain cannot include negative values because of the square root:

    Domain: $[0,\infty)$
      
  \item $fg$

    $(fg)(x) = (\sqrt{x}+1)x^2$
    
    Once again, the domain cannot include negative values because of the
    square root:

    Domain: $[0,\infty)$

  \item $\frac{f}{g}$

    $\left(\frac{f}{g}\right)(x)=\frac{\sqrt{x}+1}{x^2}$

    This time, the value 0 is excluded as well because of the denominator:
    
    Domain: $(0,\infty)$
      
  \item $\frac{f}{f}$

    $\left(\frac{f}{f}\right)(x)=\frac{\sqrt{x}+1}{\sqrt{x}+1}=1$

    Although the simplified function is constant and thus can accept all real
    values, we still need to honor the domains of the original functions:

    Domain: $[0,\infty)$

  \item $f\circ g$
      
    $(f\circ g)(x)=f(g(x))=f(x^2)=\sqrt{x^2}+1=\abs{x}+1$

    Even though the simplified function can take all real numbers, there may
    be a problem with the original functions; however, in this case, the
    limitation (no negatives) is in the outer function - the inner function
    scrubs all negative values out of its range, which becomes the domain for
    the second function. Thus, the second function will never see a negative
    number:

    Domain: $\R$
  \end{enumerate}
  
\end{enumerate}
\end{document}
