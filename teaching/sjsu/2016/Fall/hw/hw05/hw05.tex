\documentclass[letterpaper,12pt,fleqn]{article}
\usepackage{matharticle}
\usepackage{mathtools}
\pagestyle{plain}
\begin{document}

\begin{center}
\Large Math-08 Homework \#5
\end{center}

\vspace{0.5in}

\underline{Reading}

\begin{itemize}
\item Text book section 1.1, review 0.2
\end{itemize}

\underline{Problems}

\begin{enumerate}
\item At the beginning of January you open up a savings account with \$1000
  that pays 3\% yearly interest compounded at the end of the month. You set up
  an autodeposit of \$500 per month out of your paycheck that is deposited on
  the first of each month, starting in February. In January, February, March,
  and April you pay the following expenses:
  
  \begin{tabular}{|c|c|}
    \hline
    Jan & \$250 \\
    \hline
    Feb & \$750 \\
    \hline
    Mar & \$600 \\
    \hline
    Apr & \$200 \\
    \hline
  \end{tabular}
  
  What is your account balance on May 1, after the monthly deposit?

  \bigskip

\item Solve the following linear equation in a step-by-step fashion, justifying
  each step with one of the 10 axioms, substitution, and/or left/right
  cancellation:
  \[3(x+2)=3\]
\end{enumerate}
  
\end{document}
