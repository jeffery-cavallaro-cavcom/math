\documentclass[letterpaper,12pt,fleqn]{article}
\usepackage{matharticle}
\usepackage{polynom}
\pagestyle{plain}
\begin{document}

\begin{center}
\Large Math-08 Homework \#13 Solutions
\end{center}

\vspace{0.5in}

\underline{Reading}

\begin{itemize}
\item Text book section 3.3, 3.4, 4.1
\end{itemize}

\underline{Problems}

Note that all sketches of graphs must have all found intercepts and
discontinuities labeled. All domains and ranges must be expressed in
interval notation. Remember, sketches do not have to be to scale!

\begin{enumerate}

\item Let $p(x)$ be a polynomial whose graph contains the points $(-2,1)$ and
  $(3,0)$.
  \begin{enumerate}
  \item What is the remainder with $p(x)$ is divided by $(x+2)$?

    By the remainder theorem: $p(-2)=1$
    
  \item What is the remainder with $p(x)$ is divided by $(x-3)$?

    By the remainder theorem: $p(3)=0$
  \end{enumerate}

\item Let:
  \[p(x)=2x^4+x^3-9x^2+8x-2\]
  \begin{enumerate}
  \item Fully factor $p(x)$. Note that you may end up with irrational zeros!
    You must show all work, including the possible candidates for zeros, and
    long (or synthetic) divisions that lead you to the final answer.  There is
    \emph{no} credit for simply stating an answer.

    First, we determine the leading and constant coefficients and then list
    all of their factors (positive and negative!):

    $a_n=2: \pm1,\pm2$ \\
    $a_0=-2:  \pm1,\pm2$

    Next, we determine all possible candidates: $\frac{a_0}{a_n}$:

    $\pm1,\pm2,\pm\frac{1}{2}$

    Now, look for actual zeros using the remainder/factor theorem:

    $p(1)=2+1-9+8-2=0$ it works! so divide out $(x-1)$

    $\polylongdiv{2x^4+x^3-9x^2+8x-2}{x-1}$

    At this point we have:

    $p(x)=(x-1)(2x^3+3x^2-6x+2)$

    We now repeat the process with the third degree polynomial:

    $a_n=2: \pm1,\pm2$ \\
    $a_0=2:  \pm1,\pm2$

    $\pm1,\pm2,\pm\frac{1}{2}$

    We need to try 1 again, since it may be a repeated zero:

    $p(1)=2+3-6+2\ne0$ \\
    $p(-1)=-2+3+6+2\ne0$ \\
    $p(2)=16+12-12+2\ne0$ \\
    $p(-2)=-16+12+12+2\ne0$ \\
    $p(\frac{1}{2})=\frac{1}{4}+\frac{3}{4}-3+2=0$ works! \\

    Instead of dividing by $(x-\frac{1}{2})$ we can divide out the
    $\frac{1}{2}$ and instead divide by $(2x-1)$. This is OK because the
    leading coefficient of the polynomial is $2$:

    $\polylongdiv{2x^3+3x^2-6x+2}{2x-1}$

    So, now we have:

    $p(x)=(x-1)(2x-1)(x^2+2x-2)$

    We are down to a quadratic that does not factor by inspection, so we
    need to use the quadratic formula:

    $x=\frac{-2\pm\sqrt{2^2-4(1)(-2)}}{2(1)}=\frac{-2\pm\sqrt{12}}{2}
    =\frac{-2\pm2\sqrt{3}}{2}=-1\pm\sqrt{3}$

    Thus, the final factorization is:

    $p(x)=(x-1)(2x-1)[x-(-1+\sqrt{3})][x-(-1-\sqrt{3})]$
    
  \item Determine all x-intercepts (if any).

    The x-intercepts are just the zeros:

    $(1,0),(\frac{1}{2},0),(-1\pm\sqrt{3},0)$
    
  \item Determine all y-intercepts (if any).

    $f(0)=-2$, so $(0,-2)$
    
  \item Sketch the graph. Be sure to label all intercepts.

    \begin{tikzpicture}
      \draw [<->] (-5,0) -- (5,0);
      \draw [<->] (0,-5) -- (0,5);
      \node [fill,circle,scale=0.5] (x0) at ({-1-sqrt(3)},0) {}
      node at (x0) [below] {$-1-\sqrt{3}$};
      \node [fill,circle,scale=0.5] (x1) at (0.5,0) {}
      node at (x1) [below] {$\frac{1}{2}$};
      \node [fill,circle,scale=0.5] (x2) at (2,0) {}
      node at (x2) [below] {$-1+\sqrt{3}$};
      \node [fill,circle,scale=0.5] (x3) at (3.5,0) {}
      node at (x3) [below] {$1$};
      \node [fill,circle,scale=0.5] (y0) at (0,-2) {}
      node at (y0) [below] {$-2$};
      \node [fill,circle,scale=0.5] (min1) at (-1,-4) {}
      node at (min1) [below] {$m_1$};
      \node [fill,circle,scale=0.5] (max1) at (1.25,0.75) {}
      node at (max1) [above] {$m_2$};
      \node [fill,circle,scale=0.5] (min2) at (3,-0.75) {}
      node at (min2) [below] {$m_3$};
      \draw plot [smooth] coordinates {
        (-3.5,5)
        (x0) (min1) (y0) (x1) (max1) (x2) (min2) (x3)
        (5,5)
      };
    \end{tikzpicture}
    
  \item Use a calculator to determine all extrema. Just state the points; you
    don't need to attach screenshots this time.

    minima: $m_1(-1.87,-30.51)$ and $m_3(0.89,-0.05)$ \\
    maxima: $m_2(0.60,0.04)$
  \end{enumerate}

\item Consider a circle: $x^2+y^2=r^2$.
  \begin{enumerate}
  \item Solve for $y$.

    $y=\pm\sqrt{r^2-x^2}$

    \begin{tikzpicture}
      \draw [<->] (-3,0) -- (3,0);
      \draw [<->] (0,-3) -- (0,3);
      \draw (0,0) circle [radius=2];
      \node [below right] at (2,0) {$r$};
      \draw [dashed] (1,-3) -- (1,3);
    \end{tikzpicture}
    
  \item Limit the range so that you get a function of $x$.

    Since a circle fails the vertical line test, we need to take either the
    top half or the bottom half. Normally, we take the top half:

    $y=\sqrt{r^2-x^2}$
    
    \begin{tikzpicture}
      \draw [<->] (-3,0) -- (3,0);
      \draw [<->] (0,-3) -- (0,3);
      \draw (-2,0) arc (180:0:2);
      \node [below right] at (2,0) {$r$};
      \draw [dashed] (-3,1) -- (3,1);
    \end{tikzpicture}

  \item Limit the domain so that you get a one-to-one function of $x$.

    Since the half-circle fails the horizontal line test, we need to eliminate
    half of it in order to make the function one-to-one. We normally take
    the quarter circle in QI:
    
    \begin{tikzpicture}
      \draw [<->] (-3,0) -- (3,0);
      \draw [<->] (0,-3) -- (0,3);
      \draw (0,2) arc (90:0:2);
      \node [below right] at (2,0) {$r$};
      \draw [dashed] (-3,-3) -- (3,3);
    \end{tikzpicture}

  \item Determine the inverse function based on your limited domain by solving
    for $x$.

    Upon observing the graph of the final function, we see that it is
    reflected across the line $y=x$. Thus, it is its own inverse:

    $x=\sqrt{r^2-y^2}$
  \end{enumerate}

\item Consider a function $f(x)$ whose graph contains the following points:

  \begin{tabular}{|c|c|c|c|c|c|}
    \hline
    x & -5 & -3 & 0 & 1 & -2 \\
    \hline
    y & 1 & -1 & -3 & -2 & 6 \\
    \hline
  \end{tabular}

  Evaluate the following:
  \[2f^{-1}(-2)-f(-3)+[f(1)]^{-1}\]

  From the table, we see that:

  $f^{-1}(-2)=1$ \\
  $f(-3)=-1$ \\
  $f(1)=-2$

  So: $2(1)-(-1)+(-2)^{-1}=2+1-\frac{1}{2}=\frac{5}{2}$
\end{enumerate}
\end{document}
