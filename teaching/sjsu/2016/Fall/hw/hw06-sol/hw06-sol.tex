\documentclass[letterpaper,12pt,fleqn]{article}
\usepackage{matharticle}
\usepackage{mathtools}
\pagestyle{plain}
\begin{document}

\begin{center}
\Large Math-08 Homework \#6 Solutions
\end{center}

\vspace{0.5in}

\underline{Reading}

\begin{itemize}
\item Text book section 1.1, 1.3, 1.4
\end{itemize}

\underline{Problems}

\begin{enumerate}
\item Determine if and explain why each of the following statements is either
  correct or incorrect (or misleading).
\begin{enumerate}
\item $\sqrt{9}=3$

  This is correct. This form asks for the \emph{principle} root, which is
  positive.
  
\item $\sqrt{9}=\pm3$

  This is incorrect, since we only want the principle root, not the negative
  one. If we want the negative one, we must explicitly write $-\sqrt{9}$.
  
\item $\left(x^{\frac{1}{2}}\right)^2=\abs{x}$

  This is misleading. By writing $x^{\frac{1}{2}}$, we are implicitly stating that
  $x\ge0$ - x is never negative, since we cannot take the square root of a
  negative number. So, since $x$ is always positive, the absolute value is not
  needed here.
  
\item $\left(x^{\frac{1}{2}}\right)^2=x$

  This is correct. As stated above, $x\ge0$ so no absolute value is needed.
  
\item $\left(x^2\right)^{\frac{1}{2}}=x$

  Incorrect! This is the case where we need absolute value, since $x$ can be
  negative. To see this, plug in a negative number:
  \[[(-2)^2]^{\frac{1}{2}}=4^{\frac{1}{2}}=2\]
  Note that we started with a negative value but got a positive value, so we
  need the absolute value here: $\abs{-2}=2$.
  
\item $\left(x^2\right)^{\frac{1}{2}}=\abs{x}$

  Correct. Since $x$ can be negative, which is lost after squaring, we need the
  absolute value.
  
\item $\left(x^3\right)^{\frac{1}{3}}=\abs{x}$
  
  Incorrect! Remember, odd powers/roots preserve negative values, so the
  absolute value is wrong:
  \[[(-2)^3]^{\frac{1}{3}}=(-8)^{\frac{1}{3}}=-2\]
  
\item $\left(x^3\right)^{\frac{1}{3}}=x$

  Correct. Odd powers/roots preserve negative values.
\end{enumerate}

\item Consider the quadratic equation: $2x^2+x-6=0$
  \begin{enumerate}
  \item Solve using factor by inspection.
    \[(2x-3)(x+2)=0\]
    \[x=-2,\frac{3}{2}\]
  \item Solve by manually completing the square.
    \begin{eqnarray*}
      2x^2+x &=& 6 \\
      x^2+\frac{1}{2}x &=& 3 \\
      x^2+\frac{1}{2}x+\frac{1}{16} &=& 3+\frac{1}{16} \\
      \left(x+\frac{1}{4}\right)^2 &=& \frac{49}{16} \\
      \abs{x+\frac{1}{4}} &=& \frac{7}{4} \\
      x+\frac{1}{4} &=& \pm\frac{7}{4} \\
      x &=& \frac{-1\pm7}{4} \\
      x &=& -2,\frac{3}{2} \\
    \end{eqnarray*}
  \item Solve using the quadratic formula.
    \[a=2, b=1, c=-6\]
    \begin{eqnarray*}
      x &=& \frac{-1\pm\sqrt{1^2-4(2)(-6)}}{2(2)} \\
      &=& \frac{-1\pm\sqrt{1+48}}{4} \\
      &=& \frac{-1\pm\sqrt{49}}{4} \\
      &=& \frac{-1\pm7}{4} \\
      &=& -2,\frac{3}{2} \\
    \end{eqnarray*}

    Note that the discriminant was a perfect square (not 0), so we get two
    rational roots.
    
  \item Show how the answers from the quadratic formula can be used to come up
    with a factoring of the LHS of the original equation.
    \[\left(x-\frac{3}{2}\right)(x-(-2))=0\]
    \[\left(x-\frac{3}{2}\right)(x+2)=0\]
    Note that this doesn't look quite the same since there is a missing factor
    of 2, so multiply both sides:
    \begin{eqnarray*}
      2\left(x-\frac{3}{2}\right)(x+2) &=& 2\cdot0 \\
      \left(2x-2\cdot\frac{3}{2}\right)(x+2) &=& 0 \\
      (2x-3)(x+2) &=& 0 \\
    \end{eqnarray*}
  \end{enumerate}
\end{enumerate}
  
\end{document}
