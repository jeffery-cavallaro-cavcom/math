\documentclass[letterpaper,12pt,fleqn]{article}
\usepackage{matharticle}
\usepackage{mathtools}
\pagestyle{plain}
\begin{document}

\begin{center}
\Large Math-08 Homework \#5 Solutions
\end{center}

\vspace{0.5in}

\underline{Reading}

\begin{itemize}
\item Text book section 1.1, review 0.2
\end{itemize}

\underline{Problems}

\begin{enumerate}
\item At the beginning of January you open up a savings account with \$1000
  that pays 3\% yearly interest compounded at the end of the month. You set up
  an autodeposit of \$500 per month out of your paycheck that is deposited on
  the first of each month, starting in February. In January, February, March,
  and April you pay the following expenses:
  
  \begin{tabular}{|c|c|}
    \hline
    Jan & \$250 \\
    \hline
    Feb & \$750 \\
    \hline
    Mar & \$600 \\
    \hline
    Apr & \$200 \\
    \hline
  \end{tabular}
  
  What is your account balance on May 1, after the monthly deposit?

  \bigskip

  Note that there are 4 compounding periods. May does not count, because we
  ask for the balance immediately after the autodeposit on the first of the
  month; in other words, we don't wait until the end of month 5.

  The resulting polynomial is as follows:
  \[A(x)=750x^4-250x^3-100x^2+300x+500\]
  Since the interest rate is $3\%$ or $0.03$ per year and the compounding
  period is 1 month, meaning 12 compounding periods per year, the interest rate
  per compounding period is give by:
  \[\frac{r}{n}=\frac{0.03}{12}=0.0025\]
  which is $0.25\%$. Thus, the compounding factor is given by:
  \[x=1+\frac{r}{n}=1+0.0025=1.0025\]
  Plugging this value into the polynomial results in the answer:
  \[A(1.0025)=1205.90\]
  Remember, this is money, so round to the nearest cent!
  
  So, your current balance on May 1 after the autodeposit is \$1205.90.

  \bigskip

\item Solve the following linear equation in a step-by-step fashion, justifying
  each step with one of the 10 axioms, substitution, and/or left/right
  cancellation:
  \[3(x+2)=3\]

  There are two possible ways to do this. The first way starts off with the
  distributive rule. The second way starts by multiplying both sides by
  $\frac{1}{3}$. Here is the first solution:

  \begin{tabular}{ll}
    $3x+6=3$ & Left Distributive (LD) \\
    $(3x+6)-6=3-6$ & Well-defined Addition (WD) \\
    $(3x+6)-6=-3$ & Substitution (SUB) \\
    $3x+(6-6)=-3$ & Additive Associativity (AA) \\
    $3x+0=-3$ & Additive Inverse (AI) \\
    $3x=-3$ & Additive Identity (A0) \\
    $\frac{1}{3}(3x)=\frac{1}{3}(-3)$ & Well-defined Multiplication (WD) \\
    $\frac{1}{3}(3x)=-1$ & Substitution (SUB) \\
    $\left(\frac{1}{3}\cdot3\right)x=-1$ & Multiplicative Associativity (MA) \\
    $1x=-1$ & Multiplicative Inverse (M1) \\
    $x=-1$ & Multiplicative Identity (MI) \\
  \end{tabular}

  Here is the second solution:
  
  \begin{tabular}{ll}
    $\frac{1}{3}[3(x+2)]=\frac{1}{3}\cdot3$ & Well-defined Multiplication \\
    $\frac{1}{3}[3(x+2)]=1$ & Multiplicative Inverse (MI) \\
    $\left(\frac{1}{3}\cdot3\right)(x+2)=1$ &
    Multiplicative Associativity (MA) \\
    $1(x+2)=1$ & Multiplicative Inverse (MI) \\
    $x+2=1$ & Multiplicative Identity (M1) \\
    $(x+2)-2=1-2$ & Well-defined Addition (WD) \\
    $(x+2)-2=-1$ & Substitution (SUB) \\
    $x+(2-2)=-1$ & Additive Associativity (AA) \\
    $x+0=-1$ & Additive Inverse (AI) \\
    $x=-1$ & Additive Identity (A0) \\
  \end{tabular}
\end{enumerate}
  
\end{document}
