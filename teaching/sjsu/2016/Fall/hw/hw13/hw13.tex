\documentclass[letterpaper,12pt,fleqn]{article}
\usepackage{matharticle}
\pagestyle{plain}
\begin{document}

\begin{center}
\Large Math-08 Homework \#13
\end{center}

\vspace{0.5in}

\underline{Reading}

\begin{itemize}
\item Text book section 3.3, 3.4, 4.1
\end{itemize}

\underline{Problems}

Note that all sketches of graphs must have all found intercepts and
discontinuities labeled. All domains and ranges must be expressed in
interval notation. Remember, sketches do not have to be to scale!

\begin{enumerate}

\item Let $p(x)$ be a polynomial whose graph contains the points $(-2,1)$ and
  $(3,0)$.
  \begin{enumerate}
  \item What is the remainder with $p(x)$ is divided by $(x+2)$?
  \item What is the remainder with $p(x)$ is divided by $(x-3)$?
  \end{enumerate}

\item Let:
  \[p(x)=2x^4+x^3-9x^2+8x-2\]
  \begin{enumerate}
  \item Fully factor $p(x)$. Note that you may end up with irrational zeros!
    You must show all work, including the possible candidates for zeros, and
    long (or synthetic) divisions that lead you to the final answer.  There is
    \emph{no} credit for simply stating an answer.
  \item Determine all x-intercepts (if any).
  \item Determine all y-intercepts (if any).
  \item Sketch the graph. Be sure to label all intercepts.
  \item Use a calculator to determine all extrema. Just state the points; you
    don't need to attach screenshots this time.
  \end{enumerate}

\item Consider a circle: $x^2+y^2=r^2$.
  \begin{enumerate}
  \item Solve for $y$.
  \item Limit the range so that you get a function of $x$.
  \item Limit the domain so that you get a one-to-one function of $x$.
  \item Determine the inverse function based on your limited domain by solving
    for $x$.
  \end{enumerate}

\item Consider a function $f(x)$ whose graph contains the following points:

  \begin{tabular}{|c|c|c|c|c|c|}
    \hline
    x & -5 & -3 & 0 & 1 & -2 \\
    \hline
    y & 1 & -1 & -3 & -2 & 6 \\
    \hline
  \end{tabular}

  Evaluate the following:
  \[2f^{-1}(-2)-f(-3)+[f(1)]^{-1}\]
  
\end{enumerate}
\end{document}
