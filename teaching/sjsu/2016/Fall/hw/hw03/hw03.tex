\documentclass[letterpaper,12pt,fleqn]{article}
\usepackage{matharticle}
\usepackage{mathtools}
\pagestyle{plain}
\begin{document}

\begin{center}
\Large Math-08 Homework \#3
\end{center}

\vspace{0.5in}

\underline{Reading}

\begin{itemize}
\item Text book section 0.3 and 0.4.
\end{itemize}

\underline{Problems}

\begin{enumerate}
\item This problem investigates the meaning of $a^b$ when both $a$ and $b$ are
  irrational.
  \begin{enumerate}
  \item Type $\pi^{\sqrt{2}}$ into your calculator and write down the answer
    to five decimal places.
  \item Build a table like we have done in class to show how finer and finer
    approximations of $\pi$ and $\sqrt{2}$ result in an answer that is
    arbitrarily close to $\pi^{\sqrt{2}}$. The first column should be
    approximations of $\pi$. The second column should be approximations of
    $\sqrt{2}$. The third column should be a calculation based on your current
    approximated values. Do this for up to five decimal places.
  \end{enumerate}

\item Simplify:
  \[\sqrt{75}-\sqrt{27}\]

\item Simplify:
  \[\frac{\sqrt{\sqrt[3]{x+1}xy^2}}{(x+1)x^{-\frac{3}{2}}y^3}\]
  Your answer should have no negative exponents each factor should appear
  only once. Do not rationalize the denominator. Beware of even roots of
  even powers!

\item On your calculator, store the value $1$ into the variable $x$ and the
  value $-1$ into the variable $y$. Then type the original expression (not
  your simplied one) from problem (3) into your calculator. Note that you will
  need to type $(x+1)^{\frac{1}{3}}$ instead of $\sqrt[3]{x+1}$. Make sure that
  you do this all in only 3 steps: 2 store operations and then the expression.
  Turn in a screenshot showing all 3 steps. (Hint: the answer should be
  $-0.56123\ldots$)
\end{enumerate}
\end{document}
