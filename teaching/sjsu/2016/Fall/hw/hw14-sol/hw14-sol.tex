\documentclass[letterpaper,12pt,fleqn]{article}
\usepackage{matharticle}
\pagestyle{plain}
\begin{document}

\begin{center}
\Large Math-08 Homework \#14 Solutions
\end{center}

\vspace{0.5in}

\underline{Reading}

\begin{itemize}
\item Text book chapter 4
\end{itemize}

\underline{Problems}

Make sure that all sketches have all important points and asymptotes
clearly marked.

\begin{enumerate}
\item List the transformations, find all intercepts, and sketch:
  \[y=-2e^{x+1}+5\]

  Transformations:
  \begin{enumerate}[label={\arabic*)}]
  \item Start with $y=e^x$
  \item Translate left 1
  \item Scale by 2
  \item Reflect across x-axis
  \item Translate up 5
  \end{enumerate}

  y-intercept:
  \[y=-2e^{0+1}+5=-2e+5\approx-0.44\]

  x-intercept:
  \begin{eqnarray*}
    0 &=& -2e^{x+1}+5 \\
    2e^{x+1} &=& 5 \\
    e^{x+1} &=& \frac{5}{2} \\
    x+1 &=& \ln{\frac{5}{2}} \\
    x &=& -1+\ln{\frac{5}{2}} \\
    x &\approx& -0.1
  \end{eqnarray*}

  \begin{tikzpicture}
    \draw [<->] (-5,0) -- (5,0);
    \draw [<->] (0,-5) -- (0,5);
    \draw [dashed] (-5,3) -- (5,3);
    \node [below right] at (0,3) {$5$};
    \node [circle,fill,scale=0.5] (x) at ({4*(-1+ln(5/2))},0) {};
    \node [circle,fill,scale=0.5] (y) at (0,{4*(-2*e+5)}) {};
    \draw plot [smooth] coordinates{
      (0.5,-5)
      (y)
      (x)
      (-1,2.0)
      (-2,2.9)
      (-5,2.95)
    };
    \node [left] at (y) {$-1+\ln\frac{5}{2}$};
    \node [above left] at (x) {$-2e+5$};
  \end{tikzpicture}

\item List the transformations, find all intercepts, and sketch:
  \[y=3\ln(x-2)+1\]

  Transformations:
  \begin{enumerate}[label={\arabic*)}]
  \item Start with $y=\ln{x}$
  \item Translate right 2
  \item Scale by 3
  \item Translate up 1
  \end{enumerate}

  y-intercept: none

  x-intercept:
  \begin{eqnarray*}
    0 &=& 3\ln(x-2)+1 \\
    3\ln(x-2) &=& -1 \\
    \ln(x-2) &=& -\frac{1}{3} \\
    x-2 &=& e^{-\frac{1}{3}} \\
    x &=& 2+e^{-\frac{1}{3}} \\
    x &\approx& 2.7
  \end{eqnarray*}

  \begin{tikzpicture}
    \draw [<->] (-5,0) -- (5,0);
    \draw [<->] (0,-5) -- (0,5);
    \draw [dashed] (1,-5) -- (1,5);
    \node [below right] at (1,0) {$2$};
    \node [circle,fill,scale=0.5] (x) at ({2+e^(-1/3)},0) {};
    \node [below right] at (x) {$2+e^{-\frac{1}{3}}$};
    \draw plot [smooth] coordinates{
      (1.05,-5)
      (1.5,-1.5)
      (x)
      (5,1)
    };
  \end{tikzpicture}

\item Given:
  \[\log_b{2}=0.6931\]
  \[\log_b{3}=1.0986\]
  \[\log_b{5}=1.6094\]
  find $\log_b{\left(\frac{75}{4}\right)}$. You must use each one of the given
  values, you are not allowed to determine the value of $b$, and you must show
  exactly how you obtained the answer.

  \begin{eqnarray*}
    \log_b{\left(\frac{75}{4}\right)} &=&
    \log_b{\left(\frac{3\cdot5^2}{2^2}\right)} \\
    &=& \log_b{3}+\log_b{5^2}-\log_b{2^2} \\
    &=& \log_b{3}+2\log_b{5}-2\log_b{2} \\
    &=& 1.0986+2(1.6094)-2(0.6931) \\
    &=& 2.9312
  \end{eqnarray*}

\item Consider the equation: $y=\log_a{x}$
  \begin{enumerate}
  \item Derive the change of base formula for some arbitrary base $b$.

    \begin{eqnarray*}
      y &=& \log_a{x} \\
      a^y &=& x \\
      \log_b{a^y} &=& \log_b{x} \\
      y\log_b{a} &=& \log_b{x} \\
      y &=& \frac{\log_b{x}}{\log_b{a}}
    \end{eqnarray*}
    
  \item Use your formula with $b=e$ and your calculator to compute
    $\log_7 100$.

    \[\log_7 100=\frac{\ln{100}}{\ln{7}}=2.3666\]
    You can check this by:
    \[7^{2.3666}\approx100\]
    
  \item Assume that you made a mistake and used the common log key instead of
    the natural log key in the above calculation. Would you get a different
    answer? Why or why not?

    You would get the same answer because the change-of-base formula is
    independent of the base selected.
  \end{enumerate}

\item Researchers tend to prefer exponential (base $e$) equations. For example,
  the normal equation for the radioactive decay of Carbon-14, which has a
  half-life of 5730 years, would be:
  \[A=A_0\cdot2^{-\frac{t}{5730}}\]
  But the preferred exponential equations is:
  \[A=A_oe^{-\frac{t}{a}}\]
  Solve for $a$, rounding to the nearest integer value.

  \begin{eqnarray*}
    2^{-\frac{t}{5730}} &=& e^{-\frac{t}{a}} \\
    \ln{2^{-\frac{t}{5730}}} &=& \ln{e^{-\frac{t}{a}}} \\
    -\frac{t}{a} &=& -\frac{t}{5730}\ln{2} \\
    \frac{1}{a} &=& \frac{1}{5730}\ln{2} \\
    a &=& \frac{5730}{\ln{2}} \\
    a &\approx& 8267
  \end{eqnarray*}
\end{enumerate}
\end{document}
