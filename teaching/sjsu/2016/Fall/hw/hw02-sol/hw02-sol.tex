\documentclass[letterpaper,12pt,fleqn]{article}
\usepackage{matharticle}
\usepackage{mathtools}
\pagestyle{plain}
\begin{document}

\begin{center}
\Large Math-08 Homework \#2 Solutions
\end{center}

\vspace{0.5in}

\underline{Reading}

\begin{itemize}
\item Text book section 0.2.
\end{itemize}

\underline{Problems}

\begin{enumerate}
\item Let $A=\{x\in\R\mid-1\le x\le2\}$
  \begin{enumerate}
  \item Is the statement $\forall\,a\in A,a<2$ true or false? Why? (Hint: if
    false, state a counterexample)

    In order for the statement to be true, \emph{all} of the values in $A$
    must be $<2$. But consider the value $2\in A$. $2\not<2$. so $2$ is a
    counterexample and the statement is false.

  \item Is the statement $\exists\,a\in A,a\le-1$ true or false? Why?

    In order for the statement to be true, there must be \emph{at least} one
    (there may be more) value in $A$ that is $\le-1$. Consider $-1\in A$.
    $-1\le -1$, so there is at least one value and the statement is true.
  \end{enumerate}

\item Explain why $\frac{\not{x}+y}{\not{x}}=1+y$ is incorrect? What should it
  equal?

  This is an incorrect application of the definition of division and the
  distributive law. The correct form is:
  \[\frac{x+y}{x}=\frac{1}{x}(x+y)=1+\frac{y}{x}\]

\item Let $x=y-1$. Explain why $y-x=y-y-1=-1$ is incorrect? What should it
  equal?

  This is an incorrect application of the substitution principle. The correct
  form is:
  \[y-x=y-(y-1)=y-y+1=1\]

\item Type the following equation into your calculator to obtain an answer.
  You must type it in all at once---not in pieces. Turn in a screenshot
  showing how you entered it and your answer:
  \[\frac{1-(2+\frac{3}{2})}{\frac{2}{3}-\frac{4}{5}}\]

  See file on canvas.
\end{enumerate}
\end{document}
