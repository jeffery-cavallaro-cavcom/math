\documentclass[letterpaper,12pt,fleqn]{article}
\usepackage{matharticle}
\usepackage{mathtools}
\pagestyle{plain}
\begin{document}

\begin{center}
\Large Math-08 Homework \#1
\end{center}

\vspace{0.5in}

\underline{Reading}

\begin{itemize}
\item If necessary, skim through the documents in the resources module on
  canvas dealing with mathematical logic, sets, rational numbers, and prime
  factorization. The documents have more information than you will need;
  however, if you have something in your notes that you do not understand then
  you can find more description in the documents.
\item Text book sections 0.1 and 0.2.
\end{itemize}

\underline{Problems}

\begin{enumerate}
\item Let:
\begin{eqnarray*}
P &\coloneqq& 0\ \mbox{is a positive number} \\
Q &\coloneqq& 0\ \mbox{is a rational number} \\
\end{eqnarray*}
Determine whether the following are true or false:
\begin{enumerate}
  \item P
  \item Q
  \item not P
  \item not Q
  \item P and Q
  \item P or Q
\end{enumerate}

\item Decimal to rational form conversion.
\begin{enumerate}
\item Convert $0.14\overline{23}$ to rational form.

\item Show that $0.\overline{1} = \frac{1}{9}$.

\item If this is so, then $\frac{2}{9}$ should equal $0.\overline{2}$,
  $\frac{3}{9}$ should equal $0.\overline{3}$, and so on until $\frac{8}{9}$
  should equal $0.\overline{8}$. So, what do you think that $0.\overline{9}$
  should equal?

\item Show that this is so by converting $0.\overline{9}$ to rational form.

\item Take a guess at what $25.3\overline{9}$ equals.
\end{enumerate}
\newpage
\item Rational numbers and closure.
  \begin{enumerate}
  \item Write down the definition of $\Q$ using setbuilder notation.
  \item Prove that $\Q$ is closed under addition (Hint: Assume that two numbers
    are in $\Q$, use the definition to express them as a ratio of integers,
    then add then and show why the result must be rational).
  \item Prove that $\Q$ is closed under multiplication (Hint: same as above,
    but multiply the two numbers).
  \item Give a counterexample showing that $\R-\Q$ is not closed under addition.
  \item Give a counterexample showing that $\R-\Q$ is not closed under
    multiplication.
  \end{enumerate}

\item Let:
\begin{eqnarray*}
A &=& \mbox{the set of all positive real numbers} \\
B &=& \mbox{the set of real numbers between -3 (exclusive) and 3 (inclusive)} \\
\end{eqnarray*}
\begin{enumerate}
\item Graph each set on the real number line.
\item Represent each set using set-builder notation.
\item Represent each set using interval notation.
\item Graph $A\cup B$ and represent it in interval notation.
\item Graph $A\cap B$ and represent it in interval notation.
\end{enumerate}

\end{enumerate}
\end{document}
