\documentclass[letterpaper,12pt,fleqn]{article}
\usepackage{matharticle}
\pagestyle{plain}
\begin{document}

\begin{center}
\Large Math-08 Homework \#12
\end{center}

\vspace{0.5in}

\underline{Reading}

\begin{itemize}
\item Text book section 3.1, 3.2
\end{itemize}

\underline{Problems}

Note that all sketches of graphs must have all found intercepts and
discontinuities labeled. All domains and ranges must be expressed in
interval notation. Remember, sketches do not have to be to scale!

\begin{enumerate}
\item Some kids are playing with a toy that launches balls straight up into
  the air. The balls leaving the launcher with a velocity of 96 ft/s. How high
  do the balls go?
  \begin{enumerate}
  \item Find the answer by completing the square.
  \item Find the answer using the $-\frac{b}{2a}$ shortcut.
  \end{enumerate}

\item Consider the polynomial:
  \[f(x)=(x-1)^3(x+2)^2(x-3)\]
  \begin{enumerate}
  \item What is the end behavior?
  \item What are the x-intercept(s)?
  \item What are the y-intercept(s)?
  \item Sketch the graph. Be sure to mark all intercepts and show the proper
    shape at each zero. A sign table or multiplicity info must be included for
    credit.
  \item Use a calculator to determine any extrema on the graph using the
    minimum and maximum functions. Attach a screenshot showing the
    determination of each value.
  \end{enumerate}
  
\end{enumerate}
\end{document}
