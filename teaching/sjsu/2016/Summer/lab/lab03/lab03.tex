\documentclass[letterpaper,12pt,fleqn]{article}
\usepackage{matharticle}
\usepackage{mathtools}
\begin{document}

\begin{center}
\Large Math-19 Lab \#3
\end{center}

\vspace{0.5in}

\begin{enumerate}
\item In class we did a proof to show that $\forall a\in\mathbb{R}$, the
inverse $-a$ is unique:
\begin{theproof}
Assume $a\in\mathbb{R}$ \\
Assume that $a$ has two inverses, call them $a'$ and $a''$

\begin{tabular}{ll}
$a+a'=0$ & A4 \\
$a+a''=0$ & A4 \\
$a+a'=a+a''$  & SUB \\
$\therefore a'=a''$ & LCAN \\
\end{tabular}
\end{theproof}

Use this proof as an example and produce a similar proof that shows that the
additive identity (i.e., 0) is also unique. Be sure to justify each step.

\item We know that addition and multiplication are closed operations on the
real numbers: if you add or multiply two real numbers the result is a real
number. We know that addition and multiplication are also closed for
integers, but how about for rational numbers?
\begin{enumerate}
\item Prove that if you add two rational numbers then the result is a rational
number. Start by assuming that you have two rational numbers in fractional form
and add them. Make sure that you apply all parts of the definition of a
rational number to show that the result is indeed a rational number.

\item Similarly, prove that if you multiple two rational numbers you also get a
rational number.

\item Find some easy counterexamples to show that the set of irrational numbers
is not closed under addition or multiplication. Come up with some irrational
numbers that when you add them you get something that is not irrational.
Repeat for multiplication.

\item Use part (a) to show that if you add a rational number and an irrational
number then the result is irrational. Start by assuming that you have a
rational and an irrational and you add then. Assume (incorrectly) that the
result is in fact rational and use part (a) to arrive at a contradiction. Thus,
the assumption that the result is rational is incorrect and the only
alternative is that it is irrational.
\end{enumerate}

\item Section 1.1 Problems 67-76 odd

\item Section 1.1 Problems 79-84 odd

\item Section 1.1 Problem 92
\end{enumerate}
\end{document}
