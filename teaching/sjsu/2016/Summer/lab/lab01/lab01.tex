\documentclass[letterpaper,12pt,fleqn]{article}
\usepackage{matharticle}
\usepackage{mathtools}
\begin{document}

\begin{center}
\Large Math-19 Lab \#1
\end{center}

\vspace{0.5in}

\begin{enumerate}
\item Classify each of the listed numbers by putting an `X' in the appropriate
columns (Hint: some numbers will be in more than one set).

\begin{tabular}{|c|c|c|c|c|c|c|}
\hline
 & $\mathbb{N}$ & $\mathbb{W}$ & $\mathbb{Z}$ & $\mathbb{Q}$ &
    $\mathbb{R}-\mathbb{Q}$ & $\mathbb{R}$ \\
\hline
0 & & & & & & \\
\hline
$\frac{4}{2}$ & & & & & & \\
\hline
-3 & & & & & & \\
\hline
1.036 & & & & & & \\
\hline
$10.14\overline{23}$ & & & & & & \\
\hline
$\sqrt{2}$ & & & & & & \\
\hline
$-\pi$ & & & & & & \\
\hline
\end{tabular}

\item Decimal to rational form conversion.
\begin{enumerate}
\item Convert $0.14\overline{23}$ to rational form.

\item Show that $0.\overline{1} = \frac{1}{9}$. If this is so, then
$\frac{2}{9}$ should equal $0.\overline{2}$,
$\frac{3}{9}$ should equal $0.\overline{3}$,
and so on until $\frac{8}{9}$ should equal $0.\overline{8}$. So, what does
$0.\overline{9}$ equal? Show that this is so by converting
$0.\overline{9}$ to rational form.

\item What is an equivalent decimal value for $24.1\overline{9}$?
\end{enumerate}

\item Section 1.1 Problems 41 and 43

\end{enumerate}
\end{document}
