\documentclass[letterpaper,12pt,fleqn]{article}
\usepackage{matharticle}
\usepackage{url}
\pagestyle{plain}
\begin{document}

\begin{center}
\emph{San Jos\'{e} State University}

\Large{Math-19 (Precalculus)}\normalsize

\large{Fall-2016}\normalsize

Section 01: MTWRF 9:00--10:20 \\
Lab (19W): MTWRF 11:00-12:20 \\
Industrial Studies 215
\end{center}

\vspace{0.5in}

\begin{description}

\item[Instructor:] Jeffery Cavallaro (\url{jeffery.cavallaro@sjsu.edu})

\item[Office:] Duncan Hall 209 (the TA room)

\item[Office Hours:] Since I am going to be with you 5 days a week during lab,
those hours will serve as office hours. If you need to meet with me outside of
that time then please make an appointment.

\item[Text:] \emph{Precalculus}, Stewart, Redlin, and Watson, $7^{th}$ ed. You
    must have the correct edition and you should try to have it by the end of
    the first day of class. The eBook version is OK, but make sure that you
    have a way to access it during class and lab.

\item[Web:] We will use both canvas and webassign. All class communications,
    including reading assignments, homework assignments, helpful resource
    documents, and grades, are via canvas (\url{sjsu.instructure.com}). Webassign
    (\url{webassign.com}) will be used for a portion of the homework (see below).
    Once you are registered for the course you should be able to see the course
    listed on your canvas account. The webassign class code will be distributed
    on the first day of class and via a canvas announcement. Each student must
    purchase a webassign license, either with the (e)book or separately.

\item[Calculator:] You are required to have a TI-83, 84, or 92 graphing
    calculator. I suggest that you invest in a TI-89 if you are going to
    continue taking STEM classes. You will also need a datalink so that you
    can connect your calculator to your PC and print screenshots. I have
    \emph{no} problem with you checking your homework and exam answers using
    your calculator --- in fact, I encourage it; however, answers with no
    supporting work will receive zero credit. \emph{No cell phones, tablets,
    or computers are allowed in lieu of a calculator!}

\item[Math 19W:] Co-registration in the workshop is not required; however, you
    will be at a severe disadvantage if you do not. It is a 1 unit lab that
    meets after lecture with a 40 minute break in-between. Grading is CR/NC,
    based on attendance and participation. You will work in teams on
    precalculus problems, aided by me when necessary. This is an ideal time for
    you to ask me questions regarding some of the tougher homework problems. I
    will also be able to work one-on-one with students that are having a
    particularly tough time with some particular concept.  Math is learned by
    doing problems, so the workshop is one of the best ways to help you pass
    this class.

\newpage

\item[Learning Objectives:] We will start by reviewing the real number system,
    algebraic expressions, and algebraic equations. We will then take a
    detailed look at functions and study various types of functions
    (polynomials, exponential, trigonometric, polar) using data, equations,
    and graphs. We will conclude with some analytical geometry, with an
    emphasis on trigonometry and conic sections. The overall goal is to
    provide you with all the tools needed for calculus.

\item[Attendance:] I will not take attendance after the first week; however, it
    is vitally important that you come (on time) to every class. The book has
    more information than we could possibly cover, so I will highlight in class
    what is important. I will also enhance certain subjects that I feel are
    important for your calculus preparation. Bring your book and calculator to
    every class meeting. If you miss a class, it is your responsibility to talk
    to your peers and find out what you missed.

\item[Time:] You will probably need to spend a \emph{minimum} of 20 hours per
    week outside of class doing homework and studying. This class is \emph{very}
    intensive and will require disciplined study habits. Please, please, please
    do \emph{not} register for a second summer class or commit to a (near)
    full-time summer job; if you do then your chances of passing this class
    drop exponentially!

\item[Reading:] Reading from the textbook will be assigned each Friday for the
    material to be covered in the coming week. Please read everything, not
    just the stuff in the boxes, prior to lecture. Make sure that you can work
    all of the example problems and answer all of the ``concept'' questions at
    the end of each section prior to attempting any of the homework problems.

\item[Homework:] There will be both web-based and written homework for this
    class. The web-based homework will be submitted via webassign. Webassign
    requires that you format your answers with math symbols from their answer
    tool. Don't get frustrated! It may take a couple times for you to get the
    hang of it; it will get easier the more you use it. The problems assigned
    on webassign are even problems from the book. In addition to the web-based
    homework, you will be required to turn in a small set of written homework
    problems that I will assign each week via canvas. Whereas the web-based
    problems are typically based on single concepts, the written homework will
    combine concepts and will need a little more thought. Homework will be
    assigned each week on Monday and is due on the following Monday at the
    start of class. Late homework will not be accepted.  See
    \emph{Homework Rules} for more information.

\item[Exams:] There will be three, non-cumulative exams, tentatively scheduled
    for: 7/1, 7/22, and 8/12 (all Fridays). \emph{You must plan to take the
    exams at their scheduled times}. In particular, summer vacations and/or
    previous travel arrangements are \emph{not} a valid excuse to miss an exam.
    All exams are closed book and notes. A calculator is allowed; however, as
    noted above, any answers without supporting work (i.e., guesses or copying
    an answer directly from your calculator) receive zero credit. Instead of a
    note card, I will provide all needed formulas, except for those that I
    explicitly ask that you memorize.

\item[Final:] The third exam will count as the final exam.

\newpage

\item[Grading:] Your final grade is weighted as follows:

\begin{tabular}{lc}
Web Homework & 20\% \\
Written Homework & 20\% \\
Exam 1 & 20\% \\
Exam 2 & 20\% \\
Exam 3 & 20\% \\
\end{tabular}

Grades given are: A, A--, B+, B, B--, C+, C, and NC. A final average of 70\% or
better is a passing grade.

\item[Credit:] A grade of C or better meets the \emph{Area B4: Mathematical
    Concepts} GE requirement. A grade of C or better is required for placement
    into Math 30P. A grade of B or better is required for placement into
    Math 30.

\item[Tutoring:] Peer tutoring is available to all SJSU students, free of
    charge, from the PeerConnections center. See
    \url{http://peerconnections.sjsu.edu} for more information.

\item[Academic integrity:] Your commitment to learning (as shown by your
    enrollment at SJSU) and SJSU's Academic Integrity Policy require you to be
    honest in all of your academic course work.  Faculty are required to report
    all infractions to the Office of Student Conduct and Ethical Development.
    See \url{http://www.sjsu.edu/studentconduct} for more information.

\item[Disabilities:] If you need course adaptations or accommodations due to a
    disability, or if you need special arrangements in case the building must
    be evacuated, please make an appointment with me as soon as possible. All
    students with disabilities must register with the Accessible Education
    Center (AEC) to establish a record of their disability. See
    \url{http://www.sjsu.edu/aec} for more information.

\end{description}

\end{document}
