\documentclass[letterpaper,12pt,fleqn]{article}
\usepackage{matharticle}
\usepackage{mathtools}
\pagestyle{plain}
\begin{document}

\begin{center}
\Large Math-19 Homework \#3
\end{center}

\vspace{0.5in}

\underline{Reading}

Please read sections 1.8 through 1.12 and do all concept problems in the posted
sections on web\-assign.

\underline{Problems}

\begin{enumerate}
\item Solve for $x$.  Remember, the answer should be a subset of the real
numbers expressed in interval notation --- not just single numbers.
\[\frac{x^{\frac{5}{2}}-3x^{\frac{3}{2}}-10x^{\frac{1}{2}}}{x^2-9x+20}\ge0\]

\item We want a circle whose diameter is the line segment between the points
$(5,4)$ and $(-3,-2)$. Using the distance and midpoint formulas:
\begin{enumerate}
\item{Determine the center of the circle.}
\item{Determine the radius of the circle.}
\item{What is the equation of the circle in standard form?}
\item{What is the equation of the circle in general form?}
\end{enumerate}

\item Find the equation of the line containing the diameter in question (2):
\begin{enumerate}
\item{In point/slope form.}
\item{In slope-intercept form.}
\item{In general form.}
\item{Find the equation of the line through the center of the circle and
perpendicular to the line containing the stated diameter.}
\end{enumerate}

\item The amount of heat energy ($Q$) needed to change the temperature of an
object (without going through a phase change like melting or boiling) is jointly
proportional to the mass of the object ($m$) and the \emph{change} in
temperature ($\Delta T$).
\begin{enumerate}
\item Write an equation that models this physical phenomenon. Use $c$ for the
constant of proportionality.
\item The MKS unit for heat energy is the Joule (J). The constant of
proportionality is specific to the substance being heated and is referred to as
the \emph{specific heat} of the substance. If $Q$ is measured in Joules ($J$),
$m$ is measured in grams ($g$), and temperature is measured in Kelvin (K), what
are the units of $c$?
\item In the lab, it is found that $41790J$ of heat energy raises the
temperature of $1L$ of water by $10K$. What is the specific heat of water?
(1L of water=1000g)
\end{enumerate}

\item Consider the equation:
\[y=x^2+2x-5\]
For each of the parts below, use the graphing functions under the \emph{math}
(TI-89) or \emph{calc} (TI-83/84) menus to find the answer and submit a
screen-shot from your calculator that shows the correct answer.
\begin{enumerate}
\item Find the $y$-value when $x=1.3$ using the \emph{value} function.
\item Find the $x$-intercepts using the \emph{zero} function.
\item Determine the minimum value using the \emph{minimum} function.
\item Determine the $x$-values for $y=5$ using the \emph{intersect} function.
Note that you will need to add something to your graph to do this. Also note
that there are multiple answers.
\item Now graph the function $y=x^2+11$. Huh!? Nothing seems to appear! Why,
and how can you fix this? Submit a screen shot that uses your fix.
\end{enumerate}

\end{enumerate}
\end{document}
