\documentclass[letterpaper,12pt,fleqn]{article}
\usepackage{matharticle}
\usepackage{mathtools}
\pagestyle{plain}
\begin{document}

\begin{center}
\Large Math-19 Homework \#2 Solutions
\end{center}

\vspace{0.5in}

\underline{Reading}

Please read sections 1.1 through 1.5 and 1.7 and then do all concept problems
in the posted sections on web\-assign.

\underline{Problems}

\begin{enumerate}
\item Simplify completely. Your answer should have no negative exponents and
please rationalize the denominator. Don't worry if the exponents get messy.
\[\frac{\sqrt[4]{\sqrt{75}+\sqrt{27}}}{\sqrt{4\sqrt{20}\sqrt[3]{54}}}\]

\bigskip

This is really messy, so you need to take it step by step.  Let's work with the
numerator first:
\begin{eqnarray*}
\sqrt[4]{\sqrt{75}+\sqrt{27}} &=& (\sqrt{75}+\sqrt{27})^{1/4} \\
    &=& (\sqrt{25\cdot3}+\sqrt{9\cdot3})^{1/4} \\
    &=& (5\sqrt{3}+3\sqrt{3})^{1/4} \\
    &=& (8\sqrt{3})^{1/4} \\
    &=& (2^33^{1/2})^{1/4} \\
    &=& 2^{3/4}3^{1/8} \\
\end{eqnarray*}
and now the denominator:
\begin{eqnarray*}
\sqrt{4\sqrt{20}\sqrt[3]{54}} &=& (4\sqrt{20}\sqrt[3]{54})^{1/2} \\
    &=& (2^2\sqrt{4\cdot5}\sqrt[3]{27\cdot2})^{1/2} \\
    &=& (2^2\cdot2\sqrt{5}\cdot3\sqrt[3]{2})^{1/2} \\
    &=& (2^3\sqrt{5}\cdot3\sqrt[3]{2})^{1/2} \\
    &=& (2^3\cdot5^{1/2}\cdot3\cdot2^{1/3})^{1/2} \\
    &=& (2^{10/3}3^15^{1/2})^{1/2} \\
    &=& 2^{5/3}3^{1/2}5^{1/4} \\
\end{eqnarray*}
Finally, putting it all together and rationalizing:
\begin{eqnarray*}
\frac{\sqrt[4]{\sqrt{75}+\sqrt{27}}}{\sqrt{4\sqrt{20}\sqrt[3]{54}}} &=&
    \frac{2^{3/4}3^{1/8}}{2^{5/3}3^{1/2}5^{1/4}} \\
    &=& 2^{(3/4-5/3)}3^{(1/8-1/2)}5^{(0-1/4)} \\
    &=& 2^{-11/12}3^{-3/8}5^{-1/4} \\
    &=& \frac{1}{2^{11/12}3^{3/8}5^{1/4}} \\
    &=& \left(\frac{1}{2^{11/12}3^{3/8}5^{1/4}}\right)
        \left(\frac{2^{1/12}3^{5/8}5^{3/4}}{2^{1/12}3^{5/8}5^{3/4}}\right) \\
    &=& \frac{2^{1/12}3^{5/8}5^{3/4}}{2\cdot3\cdot5} \\
    &=& \frac{2^{1/12}3^{5/8}5^{3/4}}{30} \\
\end{eqnarray*}

\bigskip

\item A student writes the following statements. Determine if each is either
correct or incorrect (or misleading). Explain why incorrect statements are
incorrect.
\begin{enumerate}
\item $\sqrt{9}=\pm3$

Incorrect. $\sqrt{9}$ is asking for the principle root: $+3$.

\item $\left(x^{\frac{1}{2}}\right)^2=\abs{x}$

Misleading. $x^{1/2}$ implies that $x>=0$, so the absolute value is extraneous.

\item $\left(x^2\right)^{\frac{1}{2}}=x$

Very incorrect! $\left(x^2\right)^{\frac{1}{2}}$ is always $\ge0$, so we need
$\abs{x}$ here, just in case $x<0$!

\item $\left(x^3\right)^{\frac{1}{3}}=\abs{x}$

Incorrect. We don't use the absolute value with odd roots. Plug in $x=-1$ and
see why this is wrong.
\end{enumerate}

\bigskip

\item Solve for $x$ by \emph{completing the square}.
\[2x^2+4x-3=0\]
\begin{eqnarray*}
2x^2+4x-3 &=& 0 \\
2x^2+4x &=& 3 \\
x^2+2x &=& \frac{3}{2} \\
x^2+2x+1 &=& \frac{3}{2}+1 \\
(x+1)^2 &=& \frac{5}{2} \\
x+1 &=& \pm\sqrt{\frac{5}{2}} \\
x &=& -1\pm\sqrt{\frac{5}{2}} \\
\end{eqnarray*}

\bigskip

\item A man stands atop a $256ft$ cliff with a  ball.
\begin{enumerate}
\item How long does it take for the ball to hit the ground if he simply
releases the ball?

\begin{eqnarray*}
256+0t-16t^2 &=& 0 \\
16(16-t^2) &=& 0 \\
16-t^2 &=& 0 \\
(4+t)(4-t) &=& 0 \\
\end{eqnarray*}
This yields two solutions: $t=\pm4$ seconds. We take the positive solution
here: $t=4$ seconds.

\bigskip

\item How long does it take for the ball to hit the ground if he throws the
ball up with a velocity of $16 ft/s$?

\begin{eqnarray*}
256+16t-16t^2 &=& 0 \\
-16(t^2-t-16) &=& 0 \\
t^2-t-16 &=& 0 \\
\end{eqnarray*}
\begin{eqnarray*}
t &=& \frac{1\pm\sqrt{(-1)^2-4(1)(-16)}}{2(1)} \\
    &=& \frac{1\pm\sqrt{1+64}}{2} \\
    &=& \frac{1\pm\sqrt{65}}{2} \\
\end{eqnarray*}

This yields two solutions: $t\approx-3.5$ seconds and $t\approx4.5$ seconds.
We take the positive solution here: $t\approx4.5$; however, we save the
negative solution for later.

\bigskip

\item How long does it take for the ball to hit the ground if he throws the
ball down with a velocity of $16 ft/s$? (Hint: no additional calculations are
needed).

\bigskip

Note that in the previous problem the man threw the ball up at $+16 ft/s$. The
ball is going to travel up, slow down due to gravity, eventually stop, and then
start falling. When the ball passes the man again it must be going at
$-16 ft/s$. Thus, the two problems are the same! Furthermore, the negative
solution from the previous problem is the answer here: $t\approx3.5$ seconds.

\bigskip

\item Assume that a lady is standing on the ground below the cliff and throws
a ball up so that it passed the man on the cliff at a velocity of $16 ft/s$.
How long would it be before the ball hits the ground? (Hint: you already have
all the information that you need).

Once again, this is the \emph{same} problem. The roundtrip time is the sum of
the previous two times: $t\approx3.5+4.5\approx8.0$ seconds.
\end{enumerate}

\bigskip

\item Muri is a shopkeeper that specializes in pickled vegetables. She has
determined over the years that the best brine (salt solution) for pickling
vegetables is 2 kg of salt per liter of water (2 kg/L).  One day, she has her
not-so-bright nephew helping her and he uses too much salt, resulting in
a 5 kg/L solution.  If her nephew made up 10 liters of the too-salty solution,
how much pure water must he add to it to get the ideal 2 kg/L solution? For
full credit, show the mixture equation and the appropriate values for each
concentration and volume value in the equation.
\[c_1v_1+c_2v_2=c_3(v_1+v_2)\]
$c_1=$ initial salt concentration $=5 kg/L$ \\
$v_1=$ original water volume $=10 L$ \\
$c_2=$ salt concentration of pure water $=0 kg/L$ \\
$v_2=$ amount of water to be added $=x$ (unknown) \\
$c_3=$ desired salt concentration $=2 kg/L$ \\
\begin{eqnarray*}
5(10)+0x &=& 2(10+x) \\
50+0 &=& 20+2x \\
2x &=& 30 \\
x &=& 15 \\
\end{eqnarray*}
So, 15 L of pure water must be added to achieve the desired concentration.
\end{enumerate}
\end{document}
