\documentclass[letterpaper,12pt,fleqn]{article}
\usepackage{matharticle}
\usepackage{mathtools}
\usepackage{tikz}
\renewcommand{\o}{\theta}
\newcommand{\w}{\omega}
\pagestyle{plain}
\begin{document}

\begin{center}
\Large Math-19 Homework \#7 Solutions
\end{center}

\vspace{0.5in}

\underline{Reading}

Please read sections 6.1 through 6.3 and 5.1 through 5.3 and do all concept
problems in the posted sections on web\-assign.

\underline{Problems}

\begin{enumerate}
\item A blu-ray disk has a diameter of $12cm$. The track of recorded
information on the disk spirals in from the outside toward the center. A blu-ray
player must make sure that the track passing under the optical reader stays at
a constant linear speed, so the motor will vary from 200 to 500 rpm, depending
on where the disk is being read.
\begin{enumerate}
\item Which angular speed is used for the  part of the track on the outer rim
  of the disk? Why?

  \bigskip

  For constant angular speed, a point on the outer rim is going to have a
  faster linear spped than a point closer to the center. The equation $v=r\w$
  tells us this. Thus, the player motor must start slow at the rim and speed up
  as the reader moves along the spiral toward the center. Therefore, the lowest
  speed of 200 rpm is used when reading from near the rim.
  
  \bigskip
  
\item What is the linear speed of the outer track in cm/s?
  \[v=r\w=(6 cm)(200 rev/min)(2\pi rev^{-1})(1 min/60 sec)=40\pi cm/s
  \approx 126 cm/s\]
  
\item What is the distance from the center of the disk for the innermost part
  of the track?

  \bigskip

  Since the linear speed is constant, we use the highest angular speed and
  solve for the radius. Note that we can keep the original units because they
  will cancel properly.
  \begin{eqnarray}
    (6 cm)(200 rpm) &=& r(500 rpm) \\
    r &=& (6 cm)\left(\frac{200 rpm}{500 rpm}\right) \\
    r &=& 2.4 cm \\
  \end{eqnarray}
\end{enumerate}

\item Determine a positive and a negative coterminal angle for the angle
  $\frac{2\pi}{3}$.

  Any coterminal angle for $\frac{2\pi}{3}$ can be determined by:
  \[\o=\frac{2\pi}{3}+2\pi k\]
  where $k\in\Z$. So, two examples would be:
  \[\o=\frac{2\pi}{3}+2\pi(1)=\frac{8\pi}{3}\]
  and
  \[\o=\frac{2\pi}{3}+2\pi(-1)=-\frac{4\pi}{3}\]

\item Use a sketch of the unit circle to show why $\sin^2\o+\cos^2\o=1$, and
  then use that formula to prove the other two forms of the pythagrean identity.

\begin{minipage}{3in}
  \begin{tikzpicture}
    \draw [<->] (-3,0) -- (3,0);
    \draw [<->] (0,-3) -- (0,3);
    \draw (0,0) circle [radius=2];
    \draw [fill=black] (1, 1.7231) circle [radius=0.05];
    \draw [dashed] (0,0) -- (1, 1.7231);
    \draw [dashed] (1,0) -- (1, 1.7231);
    \node at (0.3, 1) {$1$};
    \node [below] at (0.5,0) {$x$};
    \node [right] at (1,0.75) {$y$};
    \node [above right] at (0.1,0) {$\o$};
  \end{tikzpicture}
\end{minipage}
\begin{minipage}{3in}
$x^2+y^2=1$ \\
$x=\cos\o$ \\
$y=\sin\o$ \\
$\sin^2\o+\cos^2\o=1$
\end{minipage}

Divide both sides by $\cos^2\o$ to get:
\[\tan^2\o+1=\sec^2\o\]

Divide both sides by $sin^2\o$ to get:
\[1+\cot^2\o=\csc^2\o\]

\item Write $\cos{x}$ in terms of $\tan{x}$, assuming that $x$ is in QI.

  \begin{eqnarray*}
    \sec^2{x} &=& \tan^2x+1 \\
    \frac{1}{\cos^2x} &=& \tan^2x+1 \\
    \cos^2x &=& \frac{1}{\tan^2x+1} \\
    \cos{x} &=& \frac{1}{\sqrt{\tan^2x+1}} \\
  \end{eqnarray*}

\item Consider the following sinusoidal function:
\[f(x)=-3\sin\frac{\pi}{2}(x-1)\]
\begin{enumerate}
\item What is the amplitude?
  \[A=\abs{-3}=3\]
  
\item What is the period?
  \[P=\frac{2\pi}{\frac{\pi}{2}}=4\]
  
\item What is $b$ (the horizontal translation)?

  \bigskip

  1 unit to the right

  \bigskip
  
\item What is $\phi$ (the phase angle)?
  \[\phi=\frac{\pi}{2}(-1)=-\frac{\pi}{2}\]
  
\item Is the phase angle leading or lagging?

  \bigskip

  Lagging

  \bigskip
  
\item Sketch the graph from $[0, b+\mbox{period}]$, i.e., one full period
starting from the horizontal shift point, and then extended back to 0. You must
clearly show the amplitude and the x values for each zero/min/max.

\begin{tikzpicture}
  \draw (-1,0) -- (6,0);
  \draw (0,-4) -- (0,4);
  \draw [dashed] (0,3) -- (6,3);
  \draw [dashed] (0,-3) -- (6,-3);
  \draw [dashed] (2,0) -- (2,-3);
  \draw [dashed] (4,0) -- (4,3);
  \draw [domain=0:5] plot ({\x},{-3*sin(deg(pi*(\x-1)/2))});
  \node [left] at (0,3) {$3$};
  \node [left] at (0,-3) {$-3$};
  \node [below left] at (0,0) {$0$};
  \node [below right] at (1,0) {$1$};
  \node [below right] at (2,0) {$2$};
  \node [below right] at (3,0) {$3$};
  \node [below] at (4,0) {$4$};
  \node [below] at (5,0) {$5$};
\end{tikzpicture}

\item Looking at your sketch, what is an equivalent function in terms of
  $\cos$? (Hint: try to find where a $cos$ graph overlays your graph)

  \[y=3\cos\frac{\pi}{2}x\]
\end{enumerate}
\end{enumerate}
\end{document}
