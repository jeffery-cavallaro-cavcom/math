\documentclass[letterpaper,12pt,fleqn]{article}
\usepackage{matharticle}
\usepackage{mathtools}
\usepackage{tikz}
\pagestyle{plain}
\begin{document}

\begin{center}
\Large Math-19 Homework \#4 Solutions
\end{center}

\vspace{0.5in}

\underline{Reading}

Please read sections 2.1 through 2.7 and do all concept problems in the posted
sections on web\-assign.

\underline{Problems}

State all domains in interval notation!

\begin{enumerate}
\item Consider the function:
\[y=-\sqrt[3]{x-5}+1\]
\begin{enumerate}
\item List the transformations, starting from a basic function.

\bigskip

\begin{enumerate}
\item Start with the standard function $y=\sqrt[3]{x}$.
\item Translate right by 5.
\item Reflect across the x-axis.
\item Translate up by 1.
\end{enumerate}

\bigskip

\item Determine any x-intercepts.

\begin{eqnarray*}
-\sqrt[3]{x-5}+1 &=& 0 \\
\sqrt[3]{x-5} &=& 1 \\
(\sqrt[3]{x-5})^3 &=& 1^3 \\
x-5 &=& 1 \\
x &=& 6 \\
\end{eqnarray*}
So there is an x-intercept at $(6,0)$.

\bigskip

\item Determine any y-intercepts.
\[-\sqrt[3]{0-5}+1=-\sqrt[3]{-5}+1=\sqrt[3]{5}+1\]
So there is a y-intercept at $(0,\sqrt[3]{5}+1)$.

\bigskip

\item Sketch a graph of the function.

\begin{enumerate}
\item Start with the standard function $y=\sqrt[3]{x}$.

\begin{tikzpicture}
\draw [<->] (-3,0) -- (3,0);
\draw [<->] (0,-3) -- (0,3);
\draw [domain=0:3] plot (\x,{(\x)^(1/3)});
\draw [domain=-3:0] plot (\x,{-(-\x)^(1/3)});
\end{tikzpicture}

\item Translate right by 5.

\begin{tikzpicture}
\draw [<->] (-3,0) -- (3,0);
\draw [<->] (0,-3) -- (0,3);
\draw [domain=1:3] plot (\x,{(\x-1)^(1/3)});
\draw [domain=-3:1] plot (\x,{-(-(\x-1))^(1/3)});
\node [below right] at (1,0) {5};
\end{tikzpicture}

\item Reflect across the x-axis.

\begin{tikzpicture}
\draw [<->] (-3,0) -- (3,0);
\draw [<->] (0,-3) -- (0,3);
\draw [domain=1:3] plot (\x,{-(\x-1)^(1/3)});
\draw [domain=-3:1] plot (\x,{(-(\x-1))^(1/3)});
\node [below right] at (1,0) {5};
\end{tikzpicture}

\item Translate up by 1.

\begin{tikzpicture}
\draw [<->] (-3,0) -- (3,0);
\draw [<->] (0,-3) -- (0,3);
\draw [domain=1:3] plot (\x,{-(\x-1)^(1/3)+0.5});
\draw [domain=-3:1] plot (\x,{(-(\x-1))^(1/3)+0.5});
\draw [dashed] (1,0) -- (1,0.5);
\draw [dashed] (0,0.5) -- (1,0.5);
\draw [fill=black] (1.1,0) circle [radius=0.05];
\draw [fill=black] (0,1.5) circle [radius=0.05];
\draw [fill=black] (1,0.5) circle [radius=0.05];
\node [below] at (1,0) {5};
\node [left] at (0,0.5) {1};
\node [above left] at (0,1.5) {6};
\node [above right] at (1,0) {$1+\sqrt[3]{5}$};
\end{tikzpicture}
\end{enumerate}

\item Determine the domain of the function.

\bigskip

Domain: $\R$

\bigskip

\item Determine the range of the function.

\bigskip

Range: $\R$

\bigskip

\item On which intervals is the function increasing?

\bigskip

Increasing: none!

\bigskip

\item On which intervals is the function decreasing?

\bigskip

Decreasing: $\R$

\bigskip

\end{enumerate}

\bigskip

\item Let:
\[f(x)=\sqrt{x}(x+1)\]
\[g(x)=\sqrt{x}\]
\begin{enumerate}
\item Determine $f+g$ and state the domain.
\[(f+g)(x)=\sqrt{x}(x+1)+\sqrt{x}=\sqrt{x}(x+2)\]
Domain: $[0,\infty)$

\bigskip

\item Determine $fg$ and state the domain.
\[(fg)(x)=\sqrt{x}(x+1)\sqrt{x}=x(x+1)\]
Domain: $[0,\infty)$

\bigskip

\item Determine $\frac{f}{g}$ and state the domain.
\[\left(\frac{f}{g}\right)(x)=\frac{\sqrt{x}(x+1)}{\sqrt{x}}=x+1\]
Domain: $(0,\infty)$

\bigskip

\item Determine $\frac{f}{f}$ and state the domain.
\[\left(\frac{f}{f}\right)(x)=1\]
Domain: $(0,\infty)$

\end{enumerate}

\bigskip

\item Let:
\[h(x)=\sqrt[3]{\frac{x+1}{x-1}}-5\]
Find a suitable $f(x)$ and $g(x)$ such that $h=f\circ g$. Remember, neither is
allowed to be just $x$. Be careful to correctly determine which is the inner
function and which is the outer function.

\bigskip

One possible solution is:
\[f(x)=\sqrt[3]{x}-5\]
\[g(x)=\frac{x+1}{x-1}\]

This also works:
\[f(x)=x-5\]
\[g(x)=\sqrt[3]{\frac{x+1}{x-1}}\]

A bit more complicated is:
\[f(x)=\sqrt[3]{\frac{x}{x-1}}-5\]
\[g(x)=x+1\]

\bigskip

\item Let:
\[f(x)=\frac{1}{x}\]
Compute the difference quotient $\frac{f(x+h)-f(x)}{h}$
\begin{eqnarray*}
\frac{f(x+h)-f(x)}{h} &=& \frac{\frac{1}{x+h}-\frac{1}{x}}{h} \\
    &=& \frac{x-(x+h)}{x(x+h)}\cdot\frac{1}{h} \\
    &=& \frac{-h}{x(x+h)}\cdot\frac{1}{h} \\
    &=& -\frac{1}{x(x+h)} \\
\end{eqnarray*}

\item A certain chemical reaction proceeds at a linear pace with 4kg of product
being produced every 30 seconds. At the start of the reaction there was already
2kg of product existing.
\begin{enumerate}
\item Express the amount of product at time $t$ (starting at $t=0$) by a linear
equation: $p(t)=At+B$.
\[p(t)=\frac{4}{30}t+2=\frac{2}{15}t+2\]

\item What does $A$ represent?

\bigskip

The constant rate of the creation of product.

\bigskip

\item What does $B$ represent?

\bigskip

The initial amount of product (at $t=0$).

\bigskip

\item How much product is there after 15 seconds?
\[p(15)=\frac{2}{15}(15)+2=2+2=4\]
After 15 seconds there will be 4kg of product.
\end{enumerate}

\end{enumerate}
\end{document}
