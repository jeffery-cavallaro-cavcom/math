\documentclass[letterpaper,12pt,fleqn]{article}
\usepackage{matharticle}
\usepackage{mathtools}
\usepackage{tikz}
\pagestyle{plain}
\begin{document}

\begin{center}
\Large Math-19 Homework \#6 Solutions
\end{center}

\vspace{0.5in}

\underline{Reading}

Please read sections 2.8 and 4.1 through 4.7 and do all concept problems in the
posted sections on web\-assign.

\underline{Problems}

\begin{enumerate}
\item You use \$1000 to open a savings account at your local bank on the first
of February. The savings account has an interest rate of 1.5\% per year and
compounds monthly on the last day of the month. You set up an auto-deposit of
\$100 from your paycheck to occur on the first of each month, starting with the
second month (March).  During April, you withdraw \$250 to purchase a new
gameboy (gotta catch em all!).
\begin{enumerate}
\item Who is the lender and who is the borrower?

\bigskip

Lender: You \\
Borrower: The Bank

\bigskip

\item Calculate $x=1+\frac{r}{n}$

\[x=1+\frac{0.015}{12}=1.00125\]

\item Construct a polynomial in $x$ to determine the account value on July 2.

\begin{tabular}{|c|c|c|c|}
\hline
month & deposits & withdrawals & net \\
\hline
feb & 1000 & 0 & 1000 \\
mar & 100 & 0 & 100 \\
apr & 100 & 250 & -150 \\
may & 100 & 0 & 100 \\
jun & 100 & 0 & 100 \\
jul & 100 & 0 & 100 \\
\hline
\end{tabular}

\[A(x)=1000x^5+100x^4-150x^3+100x^2+100x+100\]

\item What is the account value on July 2?
\[\$1256.58\]
\end{enumerate}

\item Consider the circle $x^2+y^2=r^2$ and remember that we needed to restrict
the range in order to obtain the function $y=\sqrt{r^2-x^2}$.
\begin{enumerate}
\item Sketch the half-circle function and demonstrate why it is not one-to-one?

\begin{tikzpicture}
\draw [<->] (-3,0) -- (3,0);
\draw [<->] (0,-3) -- (0,3);
\draw (2,0) arc (0:180:2);
\draw [color=red, dashed] (-3,1) -- (3,1);
\node [below] at (-2,0) {$-r$};
\node [below] at (2,0) {$r$};
\end{tikzpicture}

It fails the horizontal line test: each value in the range is mapped by two
values: $+x$ and $-x$.

\bigskip

\item Suggest a how to limit the domain so that it is a one-to-one function.

\bigskip

Limit the domain for a quarter circle.

\bigskip

\item Sketch the new graph for the one-to-one function and state its domain and
range.

\begin{tikzpicture}
\draw [<->] (-3,0) -- (3,0);
\draw [<->] (0,-3) -- (0,3);
\draw (2,0) arc (0:90:2);
\node [below left] at (0,0) {$0$};
\node [below] at (2,0) {$r$};
\draw [dashed] (0,0) -- (3,3);
\end{tikzpicture}

Domain: $[0,r]$ \\
Range: $[0,r]$

\bigskip


\item By observing the graph (and the line $y=x$), predict something about the
inverse function.

\bigskip

Note that the graph is symmetry about the line $y=x$, and is therefore its
own inverse.

\bigskip

\item Derive the inverse to prove your prediction.

\begin{eqnarray*}
f(x) &=& \sqrt{r^2-x^2} \\
y &=& \sqrt{r^2-x^2} \\
y^2 &=& r^2-x^2 \\
x^2 &=& r^2-y^2 \\
x &=& \sqrt{r^2-y^2} \\
f^{-1}(x) &=& \sqrt{r^2-x^2}
\end{eqnarray*}

$\therefore f(x)=f^{-1}(x)$

\bigskip

\end{enumerate}

\item Consider the equation: $y=\log_a{x}$
\begin{enumerate}
\item Derive the change of base formula for some arbitrary base $b$.

\begin{eqnarray*}
y &=& \log_a{x} \\
a^y &=& x \\
\log_b{a^y} &=& \log_b{x} \\
y\log_b{a} &=& \log_b{x} \\
y &=& \frac{\log_b{x}}{\log_b{a}} \\
\end{eqnarray*}

\item Use your formula with $b=e$ and your calculator to compute $\log_7100$.

\[\log_7100=\frac{\ln{100}}{\ln{7}}=2.36659\]

An easy way to check this is to use your calculator to show that:

\[7^{2.36659}=100\]

\item Assume that you made a mistake and used the common log key instead of the
natural log key in the above calculation. Would you get a different answer?
Why or why not?

\bigskip

No, because the log rules do not depend on the base.

\bigskip
\end{enumerate} 

\item Researchers tend to prefer exponential (base $e$) equations. For example,
the normal equation for the radioactive decay of Carbon-14, which has a
half-life of 5730 years, would be:
\[A=A_0\cdot2^{-\frac{t}{5730}}\]
But the preferred exponential equations is:
\[A=A_oe^{-\frac{t}{8267}}\]
Prove that these two equations are equivalent.

\bigskip

In order for these two equations to be the same, the exponential factors must
match:

\begin{eqnarray*}
2^{-\frac{t}{5730}} &=& e^{-\frac{t}{x}} \\
\ln{2^{-\frac{t}{5730}}} &=& \ln{e^{-\frac{t}{x}}} \\
-\frac{t}{5730}\ln{2} &=& -\frac{t}{x} \\
x &=& \frac{5730}{\ln{2}} \\
x &=& 8267 \\
\end{eqnarray*}

\item The San Francisco earthquake of 1906 was estimated at 7.6 on the Richter
Scale. Current building codes have resulted in skyscrapers that can withstand
an 8.0 earthquake. How much strong is this that the 1906 quake?

\bigskip

First, solve the Richter equation to find the intensity $I$ of an earthquake
given its rating $M$, where the reference $S=10^{-4}cm$:

\begin{eqnarray*}
M &=& \log{\frac{I}{S}} \\
10^M &=& \frac{I}{S} \\
I &=& 10^M\cdot S \\
\end{eqnarray*}

Given two ratings: $M_1$ and $M_2$:

\[\frac{I_1}{I_2}=\frac{10^{M_1}\cdot S}{10^{M_2}\cdot S}=10^{M_1-M_2}\]

Note that this is not dependent on the reference value.  For our case:

\[\frac{I_1}{I_2}=10^{8.0-7.6}=10^{0.4}=2.5\]

So, the buildings in SF can withstand an earthquake two and a half times the
1906 quake.
\end{enumerate}
\end{document}
