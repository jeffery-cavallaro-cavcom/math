\documentclass[letterpaper,12pt,fleqn]{article}
\usepackage{matharticle}
\usepackage{amsfonts}

\begin{document}

\section*{Prime Factorization}

In mathematics, the term ``divides'' when applied to integers means something
very specific:
\begin{definition}[Divides]
To say that one integer $n$ \emph{divides} another integer $m$, denoted $n|m$,
mean that $\exists k\in\mathbb{Z},m=kn$. In this case, $n$ is said to be a
\emph{divisor} of $m$.
\end{definition}
In arithmetic terms, this means that $n$ divides $m$ evenly; there is no
remainder. We will study this in depth later when we encounter the so-called
\emph{division algorithm}.

This concept of division gives rise to an important subset of the integers
called the prime numbers:
\begin{definition}[Prime]
To say that an integer is a \emph{prime} number means that it has only two
divisors: 1 and itself. Otherwise, it is called \emph{composite}.
\end{definition}
The first few prime numbers are: $2, 3, 5, 7, 11, 13, 17, 19, \ldots$.

The prime numbers play a pivotal role in what is called the \emph{Fundamental
Theorem of Arithmetic}:
\begin{theorem}
Every integer can be represented uniquely as a product of powers of primes.
\end{theorem}
For example, $120=2^3\cdot3\cdot5$ and this representation is unique.

Prime factorization is useful when we want to add or subtract two
fractions with different denominators; we need to modify one or more
of the fractions so that we have a common denominator.  We will then
be able to use the rule:

\[\frac{a}{c}\pm\frac{b}{c}=\frac{a\pm b}{c} \]

We could just multiple the denominators by each other in order to achieve such
a common denominator, but that typically leads to large, error-prone numbers---
especially when there are more than two fractions in our expression. Instead,
we want to pick the lowest common multiple (LCM) of all the denominators.

Consider the problem: $\frac{3}{10}+\frac{7}{12}-\frac{3}{5}$

\begin{enumerate}
\item Perform a prime factorization of each denominator.

\begin{tabular}{l}
$10=2\cdot5$ \\
$12=2\cdot6=2\cdot2\cdot3=2^2\cdot3$ \\
$5=5$ (already prime!) \\
\end{tabular}

\item Take the highest power of each prime across all the factorizations.

From $2^1$, $2^2$ get $2^2$.

From $3^1$ we get just $3$.

From $5^1$ and $5^1$ we get $5$.

So the LCM of the denominators is $2^2\cdot3\cdot5=60$.

\item Multiply each fraction above and below the achieve the common
denominator and then perform the operation.

\begin{eqnarray*}\
\frac{3}{10}+\frac{7}{12}-\frac{3}{5} &=&
    \frac{3\cdot6}{10\cdot6}+\frac{7\cdot5}{12\cdot5}-
    \frac{3\cdot12}{5\cdot12} \\
&=& \frac{18}{60}+\frac{35}{60}-\frac{36}{60} \\
&=& \frac{18+35-26}{60} \\
&=& \frac{27}{60} \\
\end{eqnarray*}

\item Use prime factorizations on the numerator and denominator to simply the
result.

\begin{tabular}{l}
$27=3\cdot9=3\cdot3\cdot3=3^3$ \\
$60=2^2\cdot3\cdot5$.
\end{tabular}

\begin{eqnarray*}
\frac{27}{60} &=& \frac{3^3}{2^2\cdot3\cdot5} \\
              &=& \frac{3^2}{2^2\cdot5} \\
              &=& \frac{9}{4\cdot5} \\
              &=& \frac{9}{20} \\
\end{eqnarray*}
\end{enumerate}

\end{document}
