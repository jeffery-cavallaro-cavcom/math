\documentclass[letterpaper,12pt,fleqn]{article}
\usepackage{matharticle}
\usepackage{amsfonts}

\begin{document}

\section*{Converting from Decimal to Rational Form}

Rational numbers can be encountered in three possible forms:
\begin{itemize}
\item Fractional
\item Finite Decimal
\item Infinitely-repeating Decimal
\end{itemize}
When we see one of the decimal forms, we should be able to convert it to the
fractional form.

\subsection*{Fractional Form}

We define the set of rational numbers as follows:
\[\mathbb{Q}=\left\{\left.\frac{p}{q}\right|p,q\in\mathbb{Z}\ \mbox{and}\ q\ne0\right\}\]
In other words, a rational number is any real number that can be expressed as
an integer divided by an integer, with the denominator not equal to 0. Note
that this is not the same thing as a fraction: all rational numbers can be
expressed as fractions, but not all fractions are rational numbers. For
example, consider the fraction $\frac{\pi}{2}$; this is not rational because
the numerator is not an integer.

\subsection*{Finite Decimal Form}

Rational numbers can be expressed as decimal numbers where the number of digits
is finite. For example, the value:
\[12.345\]
has a finite number of digits and is therefore a rational number. We convert it
to fractional form by using the digits in the numerator and then dividing by
the appropriate power of 10:
\[\frac{12345}{100000}\]
Of course, this fraction can be reduced to $\frac{2469}{20000}$, if desired.

Note that a rational number with a finite number of decimal digits actually
contains a repeating pattern of 0's after the last digit:
\[12.3450000\ldots\]
which we represent by drawing a bar over the repeating pattern:
\[12.345\overline{0}\]
So the finite case is actually a simple form of the repeating case.

\subsection*{Repeating Decimal Form}

Rational numbers can be expressed as decimal numbers where the fractional part
contains an infinite, repeating digit sequence.  For example:
\[0.\overline{3}=0.33333\ldots\]
To convert such a number to fractional form, we procede as follows.
\begin{enumerate}
\item Let $x=0.\overline{3}$.
\item Multiply $x$ by a power of 10 so that one set of repeating digits moves
to the left of the decimal point. In this case: $10x=3.\overline{3}$.
\item Now subtract the two equations. Note that the repeating part on the right
of the decimal point cancels and we have: $9x=3$.
\item Solve for $x$ and reduce: $x=\frac{3}{9}=\frac{1}{3}$.
\end{enumerate}

When a set of non-repeating digits appears before the repeating sequence, we
adjust to procedure to capture the non-repeating digits to the left of the
decimal point. For example, to convert $3.298\overline{145}$:
\begin{enumerate}
\item Let $x=3.298\overline{145}$
\item Multiple by a power of 10 to capture the non-repeating digits only:
$1000x=3298.\overline{145}$
\item Multiply by a power of 10 to capture both the non-repeating digits and
one set of the repeating digits: $1000000x=3298145.\overline{145}$
\item Subtract the two equations to cancel the repeating part to the right of
the decimal point: $999000x=3294847$.
\item Finally, solve and reduce (if possible): $x=\frac{3294847}{999000}$.
\end{enumerate}

\end{document}
