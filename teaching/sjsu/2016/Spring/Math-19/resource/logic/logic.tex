\documentclass[letterpaper,12pt,fleqn]{article}
\usepackage{matharticle}
\usepackage{mathtools}

\begin{document}

\section*{Mathematical Logic}

When we speak in mathematical terms, our use of language needs to be as precise
as possible. In short, clarity is good and ambiguity is bad. To help ensure
clarity, we use logical constructs to help communicate the gist of what we are
trying to say. Thus, a short introduction to these logical constructs is in
order.

\subsection*{Statements}

The most basic logical construct is the \emph{statement}, which is defined as
follows:

\begin{definition}[Statement]
A statement is a fact that is \emph{unambiguously} either true or false.
\end{definition}

Thus, $2<3$ is a fact that is unambiguously \emph{true} and is therefore a
statement.  Likewise, $3<2$ is a fact that is unambiguously \emph{false} and is
therefore also a statement. But, $x<2$, assuming that we know nothing about the
value of $x$, could be true or false, so it is \emph{not} a statement. However,
if we claim that $x$ has the value 1 then $x<2$ becomes unambiguously true and
is now a statement.

We tend to use capital letters, starting with $P$, to represent statements and
the `$\coloneqq$' symbol to indicate definitions.  For example:
\[P\coloneqq2<3\]
would indicate that $P$ is defined to be the true statement $2<3$.

\subsection*{Logical Operators}

Single statements can be combined into \emph{compound} statements using logical
operators. There are three logical operators commonly in use:
\begin{enumerate}
\item NOT
\item AND
\item OR
\end{enumerate}

\subsubsection*{NOT}

The NOT operator flips the truth about a statement: $not\ P$ is true when $P$
is false and false when $P$ is true. For example, if $P\coloneqq 2<3$ then $P$
is true and $not\ P$ is false. Note that in this case $not\ 2<3$ would mean
$2\ge3$, which is clearly false.

We can capture the effect of a logical operator using something called a
\emph{truth table}. The truth table for the NOT operator applied to a statement
$P$ would be as follows (T=true and F=false):

\begin{tabular}{|c|c|}
\hline
$P$ & $not\ P$ \\
\hline
T & F \\
\hline
F & T \\
\hline
\end{tabular}

Note that the first column lists all of the possible truth states of $P$ and the
second column lists the resulting truth states of $not\ P$

\subsubsection*{AND}

The AND operator combines two statements $P$ and $Q$, such that $P\ and\ Q$ is
true when $P$ and $Q$ are both true, and otherwise false. The truth table is as
follows:

\begin{tabular}{|cc|c|}
\hline
$P$ & $Q$ & $P\ and\ Q$ \\
\hline
T & T & T \\
\hline
T & F & F \\
\hline
F & T & F \\
\hline
F & F & F \\
\hline
\end{tabular}

For example, let:
\begin{eqnarray*}
P &\coloneqq& 2<3 \\
Q &\coloneqq& 3<4 \\
R &\coloneqq& 5<4 \\
\end{eqnarray*}
Then $P\ and\ Q$ is true (since both $P$ and $Q$ are true), but $P\ and\ R$ is
false because even though $P$ is true, $R$ is false.

\subsubsection*{OR}

The OR operator combines two statements $P$ and $Q$, such that $P\ or\ Q$ is
true when either $P$, $Q$, or both are true. In fact, it is only false when
both $P$ and $Q$ are false. The truth table is as follows:

\begin{tabular}{|cc|c|}
\hline
$P$ & $Q$ & $P\ or\ Q$ \\
\hline
T & T & T \\
\hline
T & F & T \\
\hline
F & T & T \\
\hline
F & F & F \\
\hline
\end{tabular}

For example, assuming the above definitions of $P$, $Q$, and $R$, $P\ or\ Q$ and
$P\ or\ R$ are both true. However, $(not\ Q)\ or\ R$ is false, because both
$not\ Q$ and $R$ are false.

Note that this type of OR is often referred to as an \emph{inclusive} OR, since
the OR statement is true when both statements are true. There is another type
of OR called an \emph{exclusive} OR, which is false when both statements are
true, and thus has the following truth table:

\begin{tabular}{|cc|c|}
\hline
$P$ & $Q$ & $P\ xor\ Q$ \\
\hline
T & T & F \\
\hline
T & F & T \\
\hline
F & T & T \\
\hline
F & F & F \\
\hline
\end{tabular}

We assume that all of our OR statements are inclusive, unless explicitly stated
otherwise.

\subsection*{Operator Precedence}

When a compound statement contains multiple operators then we need to pay
attention to operator precedence, just like we do with the arithmetic operators
plus and multiply. For logical operators, the order of precedence is NOT, AND,
then OR. When consecutive operators are the same then NOT is evaluated from
right to left, and AND and OR are evaluated from left to right. We use
parentheses if we need to override this precedence, again, just like in
arithmetic.

For example, the compound statement:
\[P\ and\ Q\ or\ not\ not\ P\ and\ not\ Q\]
would be evaluated as follows:
\[(P\ and\ Q)\ or\ (not\ (not\ P)\ and\ (not\ Q))\]
Assuming the definitions of $P$ and $Q$ above, this would be a true statement:
$(P\ and\ Q)$ is true, and thus the OR statement is true. Make sure that you
can see why this is true.

\subsection*{Implication}

Mathematical systems start with a small collection of relatively simple facts
and use those facts to discover new, more complicated facts. This is done by
implication, thus making implication the most important logical construct in
mathematics.

Implication is an if-then statement of the form:

\hspace{0.5in}if \emph{(hypothesis)} then \emph{(consequence)}

The hypothesis and consequence are statements. We make the claim that if the
hypothesis is true (know facts), then the consequence must be true (new
facts). Note that such implication must have a proof that supports it.
Discovery of such proofs is the main task in higher mathematics.

As a simple example, consider the implication:

\hspace{0.5in}if $x=1$ then $x<2$

We claim that $x=1$ is a true statement, and can thus conclude that $x<2$; the
implication is a true statement. A more complicated example would be:

\hspace{0.5in}if $x=1$ and $y=2$ then $x+y=3$

Now, the hypothesis is a compound AND statement. We are claiming that the
hypothesis is true, so both parts of the AND must be true, so the consequence
does follow; indeed, $x+y=1+2=3$ and the implication is a true statement. But
consider this example:

\hspace{0.5in}if $x=1$ or $y=2$ then $x+y=3$

Now, the hypothesis is an OR statement, so only one of its parts need be true.
If they are both true then the consequence follows; however, if $x=1$ is true
but $y=2$ is false, then the consequence does not follow. We cannot conclude
that the consequence is true in all cases, so the implication is false.

Sometimes, an arrow is used to indication implication: $P\rightarrow Q$.

\subsection*{Equivalence}

We need to be careful that we properly observe the direction of an implication
statement.  For example:

\hspace{0.5in}if $x=1$ then $x<2$

is a true statement; however,

\hspace{0.5in}if $x<2$ then $x=1$

does not follow (for example, $x$ could be 0 and $0\ne1$). However, there are
some implications that are true in both directions. Consider the example:

\hspace{0.5in}if $x$ is even then $x$ is divisible by 2

This is a true statement. But the other direction:

\hspace{0.5in}if $x$ is divisible by 2 then $x$ is even

is also a true statement. When $P\rightarrow Q$ and $Q\rightarrow P$ are both
true then we say that $P$ and $Q$ are \emph{equivalent}. This means that either
both $P$ and $Q$ must be true or that both must be false. As a short cut, we
represent equivalences using \emph{if and only if} statements, usually
abbreviated $P$ iff $Q$ or also $P\leftrightarrow Q$.

\subsection*{Quantifiers}

Although the statement $1<2$ may be true, it is sort of obvious and not very
interesting. Instead, what we might want to do is compare all the values in
some, possibly infinite, collection of values to the number 2 --- something
like $x<2$ for all considered values of $x$. We do this using parameterized
statements and quantifiers.

\subsubsection*{Parameterized Statements}

We determined that something like $x<2$ is not a statement until we have a
definite value of $x$. We call something like this a \emph{parameterized}
statement, where $x$ is the parameter. When using letters to represent
statements, we add the parameter as follows:
\[P(x)\coloneqq x<2\]
Now, we need a way to provide various values of $x$. We will see how to
identify a collection of possible $x$ when we look at \emph{sets}. Until then,
we will assume that we have some collection of $x$ values in which we are
interested. We then apply those $x$ values to our parameterized statement using
a \emph{quantifier}.

\subsubsection*{Universal Quantifier}

We use the universal quantifier to test all of the values in our collection.
The universal quantifier uses the word ``all''. Continuing with the example
above, we would say something like:

\hspace{0.5in}For all $x$, $x<2$ is true

Using mathematical syntax, we would write: $\forall x, P(x)$. The upside-down
`A' stands for ``all''. Note that this is a statement that is true if $P(x)$ is
true for all of our $x$ values. It is false if there is at least one $x$ value
for which P(x) is false. So, you can think of the universal quantifier as a
shortcut for a long compound AND statement:
\[\forall x,P(x)\coloneqq P(x_1)\ and\ P(x_2)\ and\ P(x_3)\ and\ \ldots\]
If we can find at least one $x$ value such that $P(x)$ is false we call that
$x$ value a \emph{counterexample}.

For example, let our collection of $x$ values be the numbers 3, 4, and 5 and
$P(x):=x\ge3$. We can see that $\forall x,P(x)$ is a true statement. But if we
added the number 1 to our set of $x$ then the statement would be false, since
$1\not\ge3$.

\subsubsection*{Existential Quantifier}

We use the existential quantifier to see if there is at least one value in our
collection that makes our parameterized statement true. The existential
quantifier uses the words ``there exists''. Once again, continuing with the
example above, we would say something like:

\hspace{0.5in}There exists $x$ such that $x<2$ is true

Using mathematical syntax, we would write: $\exists x,P(x)$. The backwards `E'
stands for ``there exists''. Note that this is a statement that is true if
$P(x)$ is true for at least one of our $x$ values --- there can be more, but
there is at least one. It is false if $P(x)$ is false for all of our $x$
values. So, you can think of the existential quantifier as a shortcut for a
long compound OR statement:
\[\exists x,P(x)\coloneqq P(x_1)\ or\ P(x_2)\ or\ P(x_3)\ or\ \ldots\]

For example, let our collection of $x$ values be the numbers 3, 4, and 5 and
$P(x):=x<4$. We can see that $\exists x,P(x)$ is a true statement. But if we
remove the number 3 from our set of $x$ then the statement would be false, since
neither 4 nor 5 are less than 4.

Note that there is a similar quantifier denoted $\exists!$ that stands for
``there exists one and only one'' that we use when we are interested in the
existence of a unique value. For example, let our collection of $x$ values be
the numbers 3, 4, and 5 and $P(x):=x\le4$. We can see that $\exists! x,P(x)$ is
a false statement because both 3 and 4 are less than or equal to 4.

\subsubsection*{Nested Quantifiers}

Quantifiers can be nested, mixed, and matched. For example, the statement:

\hspace{0.5in}For all $x$ there exists a $y$ such that $x+y=0$

would be written as:
\[\forall x,\exists y,x+y=0\]

\end{document}
