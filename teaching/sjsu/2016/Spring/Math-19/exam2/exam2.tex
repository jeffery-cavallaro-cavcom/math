\documentclass[letterpaper,12pt,fleqn]{article}
\usepackage{matharticle}
\usepackage{mathtools}
\usepackage{amsfonts}
\usepackage{tikz}
\pagestyle{empty}
\begin{document}

\begin{center}
\Large Math-19 Exam \#2
\end{center}

\vspace{0.25in}

Name: \rule{4in}{1pt}

\vspace{0.25in}

This exam is closed book and notes. You may use a calculator; however, no cell
phones or tablets are allowed. You are also allowed notes on both sides of a
3x5'' note card. Show all work; there is no credit for guessed answers. All
values should be exact unless you are specifically asked for an approximate
value answer.  In particular, you may leave answers to trig questions in terms
of $\pi$.

\vspace{0.25in}

\begin{enumerate}

\item You have two dogs: Fido and Fluffy. Each dog is tied to its own stake in
your backyard by a leash. Fido's stake and leash allow him to roam around an
area defined by:
\[(x-2)^2+(y-1)^2=9\]
Fluffy's stake and leash allow her to roam around an area defined by:
\[x^2+y^2-10x-8y+37=0\]
\begin{enumerate}
\item What are the coordinates of Fido's stake and the length of his leash?

\vspace{0.5in}

\item What are the coordinates of Fluffy's stake and the length of her leash?
\end{enumerate}

\vspace{2.0in}

\item What is the equation of the line between the two stakes, in
slope-intercept form?

\newpage

\item It is mating season. Fido and Fluffy and not fixed, but you do not want
them to mate. Is this a problem based on the positions of the two stakes and
the lengths of the leashes?  (Hint: determine the distance between the two
stakes and compare to the lengths of the leashes).

\vspace{3.0in}

\item To be safe, you decide to erect a straight wall that is perpendicular to
the line joining the two stakes and going through the midpoint of that line.
What is the equation of the wall, in slope-intercept form?

\newpage

\item The area $A$ of a sector of a circle defined by a central angle $\theta$
and its subtended arc is jointly proportional to $\theta$ and the square of the
radius $r$ of the circle.  Let $k$ be the constant of proportionality.
\begin{enumerate}
\item Write an expression for $A$ in terms of $k$, $r$, and $\theta$.

\vspace{1.0in}

\item If $\theta=90^{\circ}$ then the defined sector has an area that is
one-fourth of the area of the circle.  Use this information to determine $k$.
\end{enumerate}

\newpage

\item Consider the function:
\[f(x)=-(x+1)^3-8\]
\begin{enumerate}
\item Provide a list of transformations, starting from a basic graph, for the
function in the order that they are applied.
\begin{enumerate}
\item
\item
\item
\item
\end{enumerate}

\vspace{0.5in}

\item Sketch the graph. Be sure to calculate and label all important points
(including intercepts).  Remember, this is a sketch - do not waste time and
space trying to draw the graph to exact scale.
\end{enumerate}

\newpage

\item Consider the function:
\[f(x)=x^2+1\]
Determine the average rate of change of the function from $x=a$ to $x=a+h$ and
simplify the resulting expression.

\newpage

\item Consider the following two functions:
\begin{eqnarray*}
f(x) &=& \sqrt{x+1} \\
g(x) &=& x^2 \\
\end{eqnarray*}
\begin{enumerate}
\item Determine $\left(\frac{f}{g}\right)(x)$ and state its domain in interval
notation.

\vspace{3.0in}

\item Determine $(g\circ f)(x)$ and state its domain in interval notation.
\end{enumerate}

\newpage

\item Consider the following triangle:

\begin{tikzpicture}[scale=0.5]
\draw (0,0) -- (12,0) -- (12,5) -- (0,0);
\draw (11.5,0) -- (11.5,0.5) -- (12,0.5);
\node at (2.5,0.5) {$\theta$};
\node [below] at (6,0) {12};
\node [right] at (12,2.5) {5};
\end{tikzpicture}

Determine all six basic trigonometric ratios.

\newpage

\item You want to measure the height of a tree; however, you can't reach the
top to measure it with a tape measurer. So you stand 10 feet from the tree,
look up to the top of the tree at an angle of $45^{\circ}$ and look down at the
bottom of the tree at an angle of $\frac{\pi}{6}$ radians.  How tall is the
tree (to the nearest tenth of a foot)?

\end{enumerate}

\end{document}
