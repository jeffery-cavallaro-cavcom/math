\documentclass[letterpaper,12pt,fleqn]{article}
\usepackage[margin=1in]{geometry}
\usepackage{parskip}
\usepackage{libertine}
\usepackage{url}
\renewcommand{\familydefault}{\sfdefault}
\pagestyle{plain}
\begin{document}

\center{\Large Math-19 Homework Rules}

\vspace{0.5in}

\begin{enumerate}

\item Reading assignments will be given in class and posted in canvas each
    Friday, and will include the material to be covered in the following week.
    You must read the textbook and make sure that you are able to do all of the
    example problems or you will not do well in the class.

\item As a general rule, all odd problems in the textbook are assigned as
    practice problems (not turned in). Try as many as you can and check your
    answers in the back of the book. I will tend to do even problems as
    examples in class, but feel free to try those as well. Math is learned by
    practice, so you will not do well if you do not spend time doing these
    problems.

\item Homework sets will be posted in canvas on Monday and will be due the
    following Monday. These homework sets will be graded and returned to you
    prior to the next exam. There will be 12 such homework sets, but I will
    only count the top 10 scores for each student.

\item Homework must be done in pencil (no pen!).

\item Homework must be submitted on $8\times11''$ college rule, pad, or graph
    paper. Do not print out the homework assignment sheet and attempt to cram
    all of your work onto it.

\item Paper ripped out of a spiral notebook will \emph{not} be accepted.

\item Be sure that your name is on the first page, and that all pages are
    stapled, in order. Creatively corner-folded sheets will \emph{not} be
    accepted.

\item Just like you would an English paper, start with a rough draft and turn
    in a neat, legible final draft.

\item Problems must be in order.

\item All work must be shown for full credit. Answers with no supporting work
    receive zero credit.

\item It is OK to work in teams; however, make sure that the work that you turn
    in reflects your ability to work the problems. Remember, your team will not
    be able to help you during exams. Math is a very lonely subject!

\item When factoring a polynomial via inspection, just write down the factoring:
\begin{eqnarray*}
x^2+3x+2 &=& 0 \\
(x+2)(x+1) &=& 0 \\
\end{eqnarray*}
Don't draw the little cross and arrange the numbers in the slots like you
learned in high school --- that is OK for your rough draft, but just show me
the factoring in the final draft.

\item When doing algebraic manipulation, each line in your answer should
    embody a single step. Don't combine steps like this:

\begin{tabular}{rcl}
3x+6 & = & 0 \\
-6 & = & -6 \\
\hline
 & 3 & \\
x & = & -2 \\
\end{tabular}

Instead, do the following:
\begin{eqnarray*}
3x+6 &=& 0 \\
3x &=& -6 \\
x &=& -2 \\
\end{eqnarray*}

\item If I see anything resembling this:

\[\frac{x+y}{x} = 1+y\]

anywhere in one of your answers then you get an automatic 0 for that problem.

\item When you come to see me during office hours (and certainly do so if you
    need help), please do not tell me that you don't understand anything and
    expect me to repeat the previous lectures. What you need to do is do the
    reading, try the example and practice problems, then come to me and show me
    a particular problem that is giving you trouble. Tell me how you have tried
    to approach the problem.

\end{enumerate}

\end{document}
