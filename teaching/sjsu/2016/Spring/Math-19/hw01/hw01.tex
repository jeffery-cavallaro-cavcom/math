\documentclass[letterpaper,12pt,fleqn]{article}
\usepackage{matharticle}
\usepackage{mathtools}
\usepackage{amsfonts}
\pagestyle{plain}
\begin{document}

\begin{center}
\Large Math-19 Homework \#1
\end{center}

\vspace{0.5in}

\begin{enumerate}
\item Let:
\begin{eqnarray*}
P &\coloneqq& 0\ \mbox{is a positive number} \\
Q &\coloneqq& 2\ge2 \\
R &\coloneqq& \forall n,m\in\mathbb{N}, n+m\in\mathbb{N} \\
\end{eqnarray*}
Determine whether the following compound statement is true or false:

\hspace{0.5in}P and Q and R or P and not Q and R or not P and Q and R

Start by rewriting the statement with parentheses to show operation order,
then substitute the truth value for each individual statement, and then
show the stepwise evaluation to the final result.

\item There is a theorem called DeMorgan's Theorem that helps us negate
complex logical statements.
\begin{enumerate}
\item Consider the statement ``not (A and B)''. We know (A and B) is false
whenever either A, B, or both are false, so not (A and B) = (not A) or (not B).
Find a similar result for not (A or B).

\item Remember that $\forall x,P(x)$ can be viewed as a big compound AND
statement. Since it will be false whenever there exists an $x$ value for which
$P(x)$ is false, we can conclude that: not $(\forall x,P(x)) = \exists x,
(not\ P(x))$.  Find a similar result for not $(\exists x,P(x))$.
\end{enumerate}

\item Classify each of the listed numbers by putting an `X' in the appropriate
columns (Hint: some numbers will be in more than one set).

\begin{tabular}{|c|c|c|c|c|c|c|}
\hline
 & $\mathbb{N}$ & $\mathbb{W}$ & $\mathbb{Z}$ & $\mathbb{Q}$ &
    $\mathbb{R}-\mathbb{Q}$ & $\mathbb{R}$ \\
\hline
0 & & & & & & \\
\hline
$\frac{4}{2}$ & & & & & & \\
\hline
-3 & & & & & & \\
\hline
1.036 & & & & & & \\
\hline
$10.14\overline{23}$ & & & & & & \\
\hline
$\sqrt{2}$ & & & & & & \\
\hline
$-\pi$ & & & & & & \\
\hline
\end{tabular}

\item Decimal to rational form conversion.
\begin{enumerate}
\item Convert $0.14\overline{23}$ to rational form.

\item Show that $0.\overline{1} = \frac{1}{9}$. If this is so, then
$\frac{2}{9}$ should equal $0.\overline{2}$,
$\frac{3}{9}$ should equal $0.\overline{3}$,
and so on until $\frac{8}{9}$ should equal $0.\overline{8}$. So, what does
$0.\overline{9}$ equal? Show that this is so by converting
$0.\overline{9}$ to rational form.
\end{enumerate}

\item Let:
\begin{eqnarray*}
A &=& \mbox{the set of all positive numbers} \\
B &=& \{x\in\mathbb{R}|-3<x\le3\} \\
\end{eqnarray*}
Represent each set in interval notation and graph each set on separate real
number lines.  Determine $A\cup B$, $A\cap B$, and $A-B$, showing the results
in both interval notation and graph form.

\item Solve by finding the LCM:
\[\frac{3}{8}+\frac{2}{9}-\frac{1}{12}\]
Show the prime factorization for each denominator and how you used the prime
factorizations to determine the LCM.

\item Simplify completely:
\[\frac{\frac{5}{6}-\left(\frac{1}{2}+\frac{2}{3}\right)}
       {\frac{1}{10}+\frac{3}{15}}\]

\item Prove: $\forall a\in\mathbb{R}-\{0\}$, the multiplicative inverse
$a^{-1}$ is unique.

\item Prove: $\forall a,b\in\mathbb{R}, a(-b)=-(ab)$. You are only allowed to
use the properties in the box on page 3 and properties 1 and 2 in the box on
page 4. Make sure that your proof is syntactically complete, with a
justification for each step.

\item Prove: $\forall a,b\in\mathbb{R}, |a-b|=|b-a|$ using the properties up to
and including those in the box at the top of page 9. Make sure that your proof
is syntactically complete, with a justification for each step. Do \emph{not}
just justify this based on the definition of distance.

\end{enumerate}

\end{document}
