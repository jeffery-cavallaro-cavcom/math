\documentclass[letterpaper,12pt,fleqn]{article}
\usepackage{matharticle}
\usepackage{mathtools}
\usepackage{amsfonts}
\pagestyle{empty}
\begin{document}

\begin{center}
\Large Math-19 Exam \#1
\end{center}

\vspace{0.25in}

Name: \rule{4in}{1pt}

\vspace{0.25in}

This exam is closed book and notes. You may use a calculator; however, no cell
phones or tablets are allowed. You are also allowed notes on both sides of a
3x5'' note card. Show all work; there is no credit for guessed answers. All
values should be exact unless you are specifically asked for an approximate
value answer.

\vspace{0.25in}

\newcommand{\fillin}{\rule{2in}{1pt}}
\newcommand{\sfillin}{\rule{0.5in}{1pt}}

\begin{enumerate}
\item The following questions are related to the classifications of the real
numbers:
\begin{enumerate}
\item Give an example of an integer that is not a natural number. \sfillin

\vspace{0.25in}

\item State the definition of the set of rational numbers.

\vspace{1in}

\item State how we represent the set of irrational numbers using set difference
notation.

\vspace{1in}

\item True or false: 0 is a rational number. Explain your answer (hint: refer
to the definition).

\vspace{1in}

\item True or false: all fractions are rational numbers. If not then show a
counterexample and explain why it makes the statement false.
\end{enumerate}

\newpage

\item Convert the number $2.1\overline{35}$ to rational form.

\vspace{3in}

\item Shown below is a careful proof of the fact that
$\forall a\in\mathbb{R},a0=0$. Fill in the reason for each step. You may use
either the codes or complete names for each rule. Note that substitution is
assumed.

\begin{tabular}{ll}
Assume $a\in\mathbb{R}$ & \\
$a0=a0$ & \\
$a(0+0)=a0$ & \fillin \\
$a0+a0=a0$ & \fillin \\
$(a0+a0)+(-a0)=a0+(-a0)$ & \fillin\\
$(a0+a0)+(-a0)=0$ & \fillin \\
$a0+(a0+(-a0))=0$ & \fillin \\
$a0+0=0$ & \fillin \\
$a0=0$ & \fillin \\
\end{tabular}

\vspace{0.25in}

\item Logic problem.
\begin{enumerate}
\item Determine if each statement is either true or false:

\begin{tabular}{ll}
$P:=\forall r,s\in\mathbb{Q},r+s\in\mathbb{Q}$ & \sfillin \\
$Q:=0\in(0,5]$ & \sfillin \\
$R:=$ The number $\pi$ can be expressed with a repeating decimal & \sfillin \\
\end{tabular}

\vspace{0.25in}

\item Show whether the following statement is true or false:

P and Q or P and not R
\end{enumerate}

\newpage

\item Simplify. Your answer should have no negative exponents or compound
fractions. (Hint: convert all of the radicals to fractional exponents and see
if anything factors out).
\[\frac{x^{\frac{3}{2}}-3\sqrt{x}-4\sqrt{\frac{1}{x}}}{x^2-16}\]

\vspace{4in}

\item Simplify. Your answer should have no negative exponents or compound
fractions and should be left in factored form.
\[\frac{\frac{1}{x-1}-\frac{1}{x-3}}{\frac{1}{x+1}}\]

\newpage

\item Solve for $x$ by \emph{completing the square}.
\[2x^2+4x-3=0\]

\vspace{4in}

\item Solve for $x$:
\[\frac{1}{x^3}+\frac{4}{x^2}+\frac{4}{x}=0\]

\newpage

\item Solve for $x$:
\[4|3x-2|-1=3(x+1)\]

\vspace{4in}

\item Muri is a shopkeeper that specializes in pickled vegetables. She has
determined over the years that the best brine (salt solution) for pickling
vegetables is 2 kg of salt per liter of water (2 kg/L).  One day, she has her
not-so-bright nephew helping her and he uses too much salt, resulting in
a 5 kg/L solution.  If her nephew made up 10 liters of the too-salty solution,
how much pure water must he add to it to get the ideal 2 kg/L solution?

\end{enumerate}

\end{document}
