\documentclass[letterpaper,12pt,fleqn]{article}
\usepackage{matharticle}
\usepackage{mathtools}
\usepackage{amsfonts}
\pagestyle{empty}
\begin{document}

\begin{center}
\Large Math-19 Final Exam
\end{center}

\vspace{0.25in}

Name: \rule{4in}{1pt}

\vspace{0.25in}

This exam is closed book and notes. You may use a calculator; however, no cell
phones or tablets are allowed. You are also allowed notes on both sides of a
3x5'' note card. Show \emph{all} work; there is \textbf{no} credit for guessed
answers. All values should be exact unless you are specifically asked for an
approximate value answer.  In particular, you may leave answers to trig
questions in terms of $\pi$.

\vspace{0.25in}

\begin{enumerate}

\item You open up a savings account at your local bank on January $1^{st}$ with
\$1000 from your current paycheck.  The account pays 1.5\% annual interest
compounded monthly.  You set up an auto-deposit from your paycheck that will
deposit an additional \$1000 to the account on the first of every month,
starting with your next (February) paycheck.  On March $1^{st}$ you withdraw
\$1500 from the account to help with the down payment on a new car.

\begin{enumerate}
\item Who is the lender?

\vspace{0.25in}

\item Who is the borrower?

\vspace{0.25in}

\item Construct a polynomial in $(1+\frac{r}{n})$ that describes the account
activity up through May $31^{st}$ (i.e., after 4 months).

\vspace{1.5in}

\item What is the account balance on May $31^{st}$?

\vspace{1.0in}

\item What would be the account balance if the only transaction was the
initial \$1000 deposit and the account compounds continually?
\end{enumerate}

\newpage

\item Consider the function $f(x)=5-e^{-x}$.

\begin{enumerate}
\item What are the x-intercepts (if any)?

\vspace{2.0in}

\item What are the y-intercepts (if any)?

\vspace{2.0in}

\item What is the end behavior as $x\to\infty$?  If it is asymptotic, be sure
to state what the asymptote is, whether it is horizontal or vertical, and if
the function approaches it from above or below.

\vspace{1.0in}

\item Repeat for $x\to-\infty$.

\newpage

\item Sketch the graph of the function.  Be sure to label all intercepts and
asymptotes and show the proper end behavior.  Any attempt to simply plot points
results in zero credit.

\vspace{4in}

\item What is the domain?

\vspace{1in}

\item What is the range?
\end{enumerate}

\newpage

\item Consider $y=\log_ax$.
\begin{enumerate}
\item What is the corresponding exponential equation?

\vspace{1in}

\item You need to caculate $y=\log_5100$, but your calculator can only do
common and natural logarithms.  Derive a change-of-base formula to compute
$\log_5$ values using your natural logarithm key.

\vspace{3in}

\item Use your formula to calculate $\log_5100$.  Round your answer to 4
decimal places.

\vspace{2in}

\item If you accidently used your common log key instead of your natural log
key in the calculation, would you get a different answer?  Why or why not?
\end{enumerate}

\newpage

\item Let $log_a10=2.0859$, $log_a5=1.4650$, and $log_a30=3.0959$.  Calculate
$log_a60$ without knowing the base $a$.  Determining $a$ and then calculating
directly receives zero credit - you must use the log rules!

\vspace{2in}

\item Solve for $x$, leaving your answer as an exact value.  Points will be
deducted for approximate (decimal) answers!
\[\frac{10}{1-e^{-x}}=2\]

\vspace{2in}

\item Solve for $x$, leaving your answer as an exact value.  Points will be
deducted for approximate (decimal) answers!
\[\log_5(x+1)-log_5(x-1)=2\]

\newpage

\item Some archaeologists are digging at what appears to be a pre-Columbian
human campsite in California.  They find some animal bones with human teeth
marks on them.  Upon carbon-14 analysis, it is found that the bones have 75\%
of their original $C_{14}$.  About how old are the bones, and hence the
campsite?  The half-life of $C_{14}$ is 5730 years.

\vspace{3in}

\item Let $\tan x=-\frac{8}{15}$ with $x$ in Quadrant II.
\begin{enumerate}
\item Find $\sin2x$.

\vspace {3in}

\item Find $\cos2x$.
\end{enumerate}

\vspace {2in}

\item Rewrite in terms of $x$ and $y$ without trig functions:
$\cos(\sin^{-1}x+\tan^{-1}y)$.

\vspace{3in}

\item Find all possible solutions for $\theta$:
\[2\sin^2\theta+(\sqrt{3}-4)\sin\theta-2\sqrt{3}=0\]

\end{enumerate}

\end{document}
