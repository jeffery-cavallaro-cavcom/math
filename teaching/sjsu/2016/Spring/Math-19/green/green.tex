\documentclass[letterpaper,12pt,fleqn]{article}
\usepackage[margin=1in]{geometry}
\usepackage{parskip}
\usepackage{libertine}
\usepackage{url}
\renewcommand{\familydefault}{\sfdefault}
\pagestyle{plain}
\begin{document}

\begin{center}
\emph{San Jos\'{e} State University}

\Large{Math-19 (Precalculus)}\normalsize

Section 02 \\
MWF 9:00--10:20am \\
Sweeney Hall 347
\end{center}

\vspace{0.5in}

\begin{description}

\item[Instructor:] Jeffery Cavallaro (\url{jeffery.cavallaro@sjsu.edu})

\item[Office:] Duncan Hall 209 (the TA room)

\item[Office Hours:] TR 1:00--3:00pm

\item[Text:] \emph{Precalculus}, Stewart, Redlin, and Watson, $7^{th}$ ed. You
    must have the correct edition and you should try to have it by Monday, 2/1.
    Bring your book to class. The eBook version is OK, but make sure that you
    have a way to access it during class.

\item[Web:] All class communications, including reading assignments, homework
    assignments, helpful resource documents, and grades, are via canvas
    (\url{sjsu.instructure.com}). We will not use webassign; however, if you
    would like to access the corresponding online material then go ahead and
    purchase the license and I will create a course for our section.

\item[Calculator:] You are allowed to use any reasonable (to be determined by
    me) \emph{scientific} calculator. I suggest that you invest in a TI-89. I
    have no problem with you checking your answers on homework and tests using
    your calculator; however, answers with no supporting work will receive zero
    credit. \emph{No cell phones, tablets, or computers are allowed in lieu
    of a calculator!}

\item[Learning Objectives:] We will start by reviewing the real number system,
    algebraic expressions, and algebraic equations. We will then take a
    detailed look at functions and study various types of functions
    (polynomials, exponential, trigonometric, polar) using data, equations,
    and graphs. We will conclude with some analytical geometry, with an
    emphasis on trigonometry and conic sections. The overall goal is to
    provide you with all the tools needed for calculus.

\item[Math 19W:] Co-registration in the workshop is required. It is a 1 unit
    class that meets for 1 hour, twice a week, and is graded CR/NC based on
    attendance. You will work in teams on precalculus problems, aided by
    workshop facilitators. Math is learned by doing problems, so the workshop
    is one of the best ways to help you pass this class.

\item[Attendance:] I will not take attendance after the first week; however, it
    is vitally important that you come (on time) to every class. The book has
    more information than we could possibly cover, so I will highlight in class
    what is important. I will also enhance certain subjects that I feel are
    important for your calculus preparation. Bring your book and calculator to
    every class meeting. If you miss a class, it is your responsibility to talk
    to your peers and find out what you missed.

\item[Time:] You will need to spend a \emph{minimum} of 10 hours per week
    outside of class doing homework and studying. This class is \emph{very}
    intensive and will require disciplined study habits.

\item[Reading:] Reading from the textbook will be assigned each Friday for the
    material to be covered in the coming week. Please read everything, not
    just the stuff in the boxes, prior to lecture. Make sure that you can work
    all of the example problems prior to attempting any homework problems.

\item[Homework:] Homework will be assigned each week on Monday and is due on
    the following Monday at the start of class. Late homework will not be
    accepted. There will be 12 homework assignments, but I will drop the lowest
    2 scores. See \emph{Homework Rules} for more information.

\item[Exams:] There will be three, non-cumulative exams: 2/26, 3/25, and 4/29.
    \emph{You must plan to take the exams at their scheduled times}. In
    particular, previous travel arrangements are \emph{not} a valid excuse to
    miss an exam. All exams are closed book and notes. A calculator and a
    double-sided $3\times5''$ note card are allowed.

\item[Final:] The last day of class is Monday, 5/16. The final exam is
    cumulative and occurs on Wednesday, 5/18, 7:15--9:30am. All above exam
    policies apply to the final as well.

\item[Grading:] The semester grade is: Homework 20\%, Exams 1, 2, and 3 20\%
    each, and Final 20\%. Grades given are: A, A--, B+, B, B--, C+, C, and NC.

\item[Credit:] A grade of C or better meets the \emph{Area B4: Mathematical
    Concepts} GE requirement. A grade of C or better is required for placement
    into Math 30P. A grade of B or better is required for placement into
    Math 30.

\item[Tutoring:] Peer tutoring is available to all SJSU students, free of
    charge, from the PeerConnections center. See
    \url{http://peerconnections.sjsu.edu} for more information.

\item[Academic integrity:] Your commitment to learning (as shown by your
    enrollment at SJSU) and SJSU's Academic Integrity Policy require you to be
    honest in all of your academic course work.  Faculty are required to report
    all infractions to the Office of Student Conduct and Ethical Development.
    See \url{http://www.sjsu.edu/studentconduct} for more information.

\item[Disabilities:] If you need course adaptations or accommodations due to a
    disability, or if you need special arrangements in case the building must
    be evacuated, please make an appointment with me as soon as possible, or
    see me during office hours. All students with disabilities must register
    with the Accessible Education Center (AEC) to establish a record of their
    disability. See \url{http://www.sjsu.edu/aec} for more information.

\end{description}

\end{document}
