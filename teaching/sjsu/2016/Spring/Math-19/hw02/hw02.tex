\documentclass[letterpaper,12pt,fleqn]{article}
\usepackage{matharticle}
\usepackage{mathtools}
\usepackage{amsfonts}
\pagestyle{plain}
\begin{document}

\begin{center}
\Large Math-19 Homework \#2
\end{center}

\vspace{0.5in}

\begin{enumerate}
\item We have fairly straightforward definitions for $a^b$ when $a$ and $b$ are
rational numbers. When $b$ is an integer we have
\[a^b=a\cdot a\cdot \ldots \cdot a\]
where $a$ is multiplied $b$ times. When $b$ is rational we have:
\[a^{\frac{p}{q}}=\sqrt[q]{a^p}\]
where $p,q\in\mathbb{Z}$. But what about when $a$ or $b$ are irrational? For
example, what does $\pi^{\sqrt{2}}$ even mean? Once again, we turn to the notion
of \emph{approximate} values getting arbitrarily close to the \emph{exact}
value.

Start by using your calculator to get an approximate value for $\pi^{\sqrt{2}}$.
Then create a table as follows:

\begin{tabular}{|c|c|c|}
\hline
$\pi$ & $\sqrt{2}$ & $\pi^{\sqrt{2}}$ \\
\hline
3 & 1 & 3 \\
\hline
3.1 & 1.4 & 4.87423 \\
\hline
 & & \\
\hline
 & & \\
\hline
 & & \\
\hline
 & & \\
\hline
\end{tabular}

In other words, make the approximations of $\pi$ and $\sqrt{2}$ finer and
finer and see if the value approaches your value for $\pi^{\sqrt{2}}$ that you
got directly from your calculator. Do up to 5 decimal places for each
approximation.

\item A careful proof for that fact that a rational number plus a rational
number is always a rational number may look something like this:

\begin{tabular}{ll}
Let $q_1,q_2\in\mathbb{Q}$ & \\
$q_1=\frac{a}{b}$ where $a,b\in\mathbb{Z}$ and $b\ne0$ &
    Definition of rationals \\
$q_2=\frac{c}{d}$ where $c,d\in\mathbb{Z}$ and $d\ne0$ &
    Definition of rationals \\
$q_1+q_2=\frac{a}{b}+\frac{c}{d}$ & Substitution \\
$q_1+q_2=\frac{ad+bc}{bd}$ & Rule of fraction addition, Substitution \\
$ad+bc\in\mathbb{Z}$ & Integers closed under addition and multiplication \\
$bd\in\mathbb{Z}$ & Integers closed under multiplication \\
$b\ne0$ and $d\ne0$, so $bd\ne0$ & Property of 0 \\
$\frac{ad+bc}{bd}$ is a rational number & Definition of rationals \\
Therefore, $q_1+q_2\in\mathbb{Q}$ & \\
\end{tabular}

Note that each step in the proof has a stated reason.

In a similar manner, prove that a rational number plus an irrational
number is always an irrational number. Start by chosing a rational number
$q=\frac{a}{b}$ and an irrational number $i$. Assume that their sum is another
rational number --- form an equation that represents this. Then see if that
equation leads to some sort of contradiction. If is does, then the assumption
that the sum is rational is incorrect, and thus the sum must be irrational.

\item Simplify completely. Your answer should have no negative exponents and
please rationalize the denominator. Don't worry if the exponents get messy.
\[\frac{\sqrt[4]{\sqrt{75}+\sqrt{27}}}{\sqrt{4\sqrt{20}\sqrt[3]{54}}}\]

\item Simplify completely:
\[\frac{(9st)^{\frac{3}{2}}}{(27s^3t^{-4})^{\frac{2}{3}}}
    \left(\frac{3s^{-2}}{4t^{\frac{1}{3}}}\right)^{-\frac{3}{2}}\]

\item Simplify completely. Your answer should have no negative exponents but
do not rationalize the denominator.
\[\frac{(x^2y^{\frac{1}{3}})^{\frac{1}{2}}}{(xy^3)^{\frac{1}{3}}}\]
\end{enumerate}

\end{document}
