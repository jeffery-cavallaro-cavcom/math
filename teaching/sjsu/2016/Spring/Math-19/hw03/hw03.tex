\documentclass[letterpaper,12pt,fleqn]{article}
\usepackage{matharticle}
\usepackage{mathtools}
\usepackage{amsfonts}
\pagestyle{plain}
\begin{document}

\begin{center}
\Large Math-19 Homework \#3
\end{center}

\vspace{0.5in}

\begin{enumerate}
\item Simplify completely.  Your answer should contain no negative exponents.
\[\frac{x^2+5x-24}{x^2-64}-
    \frac{\sqrt{x}+2x^{-\frac{1}{2}}+x^{-\frac{3}{2}}}{x^2-7x-8}+
    \frac{1}{x-2}\]

\item What is the domain of expression in problem 1? Express in set-difference
(i.e., $\mathbb{R}-\{\ldots\}$) notation, interval notation, and graph.

\item Section 1.3 Problem 135.

\item Simplify.
\[\frac{(x+h)^2+(x+h)+1-(x+2)^2}{h}\]

\item Expand and then simplify completely.
\[\left(xy^2+\sqrt{x}\sqrt[3]{y}\right)^2\]

\item A careful solution of $4(x+2)=11$ is given below. Give the rationale for
each step from the ten real number rules (A1--A4, M1--M4, LD, RD) and two
additional rules (SUB, CAN).  Note that some steps have two things to identify.

\newcommand{\fillin}{\rule{1in}{1pt}}

\begin{tabular}{ll}
$4(x+2)=11$ & \\
$4x+8=11$ & \fillin, \fillin \\
$(4x+8)-8=11-8$ & \fillin \\
$(4x+8)-8=3$ & \fillin \\
$4x+(8-8)=3$ & \fillin \\
$4x+0=3$ & \fillin, \fillin \\
$4x=3$ & \fillin, \fillin \\
$\frac{1}{4}(4x)=\frac{1}{4}(3)$ & \fillin \\
$\frac{1}{4}(4x)=\frac{3}{4}$ & \fillin \\
$(\frac{1}{4}4)x=\frac{3}{4}$ & \fillin \\
$1x=\frac{3}{4}$ & \fillin, \fillin \\
$x=\frac{3}{4}$ & \fillin, \fillin \\
\end{tabular}

\newpage

\item Section 1.4 Problem 70.

\item Solve for $x$.
\[5(x+3)-6(x-1)=-2(x+1)\]

\item Section 1.5 Problem 34.

\item Solve for $x$. Make sure that you find all solutions. Then, plug each
solution back into the equation and evaluate both sides to show that each is
in fact a proper solution.
\[|5(x+3)-2(x-1)|=5x+6\]

\end{enumerate}

\end{document}
