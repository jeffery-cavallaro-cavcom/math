\documentclass[letterpaper,12pt,fleqn]{article}
\usepackage{matharticle}
\usepackage{mathtools}
\usepackage{amsfonts}
\usepackage{tikz}
\pagestyle{empty}
\begin{document}

\begin{center}
\Large Math-19 Exam \#3
\end{center}

\vspace{0.25in}

Name: \rule{4in}{1pt}

\vspace{0.25in}

This exam is closed book and notes. You may use a calculator; however, no cell
phones or tablets are allowed. You are also allowed notes on both sides of a
3x5'' note card. Show \emph{all} work; there is \textbf{no} credit for guessed
answers. All values should be exact unless you are specifically asked for an
approximate value answer.  In particular, you may leave answers to trig
questions in terms of $\pi$.

\vspace{0.25in}

\begin{enumerate}

\item Consider the following quadratic (parabola) function:
\[f(x)=-2x^2-12x-13\]
\begin{enumerate}
\item Complete the square to put the function in standard form.
\vspace{2in}
\item What is the vertex?
\vspace{1in}
\item What are the x-intercepts (if any)?
\vspace{2in}
\item What are the y-intercepts (if any)?
\vspace{1in}
\item What is the axis of symmetry?
\vspace{1in}
\item What is the domain?
\vspace{1in}
\item What is the range?
\vspace{1in}
\item Sketch the graph of the function. You must show and label the vertex,
all intercepts, and the axis of symmetry for full credit; however, your sketch
need not be to scale. Remember to state your answers in exact values (i.e., no
decimal values allowed, but you may want to calculate them for relative
positioning on your graph).
\end{enumerate}
\newpage
\item Consider the following polynomial function:
\[f(x)=x^6-8x^5+17x^4+6x^3-44x^2+8x+32\]
\begin{enumerate}
\item Completely factor it into linear factors. It must be \emph{clear} how
you obtained candidate factors, determined that they are indeed factors, and
how you reduced the problem for each found factor (i.e., long or synthetic
division).
\newpage
\underline{Problem 2a - Continued}
\newpage
\item What is the end behavior?
\vspace{1in}
\item What are the x-intercepts (if any)?
\vspace{1in}
\item What are the y-intercepts (if any)?
\vspace{1in}
\item Construct a sign table or a list of behaviors due to multiplicity to
explain what happens at each zero.
\vspace{2in}
\item Sketch a graph of the function. You must show and label all zeros and
intercepts and show the proper behavior and shape at each zero.  Your sketch
need not be to scale.
\end{enumerate}
\newpage
\item Consider the following rational function:
\[f(x)=\frac{x^2+3x-4}{x^3-x^2-6x}\]
\begin{enumerate}
\item What are the zeros (if any)?
\vspace{1in}
\item What are the vertical asymptotes (if any)?
\vspace{1in}
\item What are the horizontal asymptotes (if any)?
\vspace{1in}
\item What is the end behavior for $x\to+\infty$? If it is asymptotic, be
sure to indicate whether the asymptote is approached from below or above and
why.
\vspace{1in}
\item What is the end behavior for $x\to-\infty$? If it is asymptotic, be
sure to indicate whether the asymptote is approached from below or above and
why.
\vspace{1in}
\item What are the y-intercepts (if any)?
\vspace{1in}
\item Sketch a graph of the function. You must show and label all zeros,
asymptotes, and intercepts, with the correct behavior at each.  Your sketch
need not be to scale.
\end{enumerate}
\vspace{4in}
\item Without using your calculator, determine the following \emph{exact}
values:
\begin{enumerate}
\item $\sin210^{\circ}$
\vspace{1in}
\item $\cos\frac{7\pi}{6}$
\vspace{1in}
\item $\tan\left(-\frac{5\pi}{6}\right)$
\end{enumerate}
\vspace{1in}
\item Consider the following sinusoidal function:
\[f(x)=-3\sin\frac{\pi}{2}(x-1)\]
\begin{enumerate}
\item What is the amplitude?
\vspace{1in}
\item What is the period?
\vspace{1in}
\item What is $b$ (the horizontal translation)?
\vspace{1in}
\item What is $\phi$ (the phase angle)?
\vspace{1in}
\item Is the phase angle leading or lagging?
\newpage
\item Sketch the graph from $[0, b+\mbox{period}]$, i.e., one full period
starting from the horizontal shift point, and then extended back to 0. You must
clearly show the amplitude and the x values for each zero/min/max.
\vspace{4in}
\item Looking at your sketch, what is an equivalent function in terms of
$\cos$? (Hint: try to find where a $cos$ graph overlays your graph)
\end{enumerate}

\end{enumerate}

\end{document}
