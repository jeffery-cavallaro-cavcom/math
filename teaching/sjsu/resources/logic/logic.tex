\documentclass[letterpaper,12pt,fleqn]{article}
\usepackage{matharticle}
\usepackage{mathtools}

\begin{document}

\section*{Mathematical Logic}

When we speak in mathematical terms, our use of language needs to be as precise
as possible. In short, clarity is good and ambiguity is bad. We use logical
constructs to describe \emph{exactly} what we mean, so a short introduction to
these constructs is in order.

\subsection*{Statements}

The most basic logical construct is the \emph{statement}, which is defined as
follows:

\begin{definition}[Statement]
A \emph{statement} is a fact that is \emph{unambiguously} either \emph{true} or
\emph{false}.
\end{definition}

Thus, $2<3$ is a fact that is unambiguously \emph{true} and is therefore a
statement.  Likewise, $3<2$ is a fact that is unambiguously \emph{false} and is
therefore also a statement. But, $x<2$, assuming that we know nothing about the
value of $x$, could be true or false, so it is \emph{not} a statement. However,
if we claim that $x$ has the value 1 then $x<2$ becomes unambiguously true and
is now a statement.

We tend to use capital letters, starting with $P$, to represent statements and
the `$\coloneqq$' symbol to indicate definitions.  For example:
\[P\coloneqq2<3\]
would indicate that $P$ is defined to be the true statement $2<3$.

\subsection*{Logical Operators}

Single statements can be combined into \emph{compound} statements using logical
operators. There are three logical operators commonly in use:
\begin{enumerate}
\item NOT
\item AND
\item OR
\end{enumerate}

\subsubsection*{NOT}

The NOT operator flips the truth about a statement: $not\ P$ is true when $P$
is false and false when $P$ is true. For example, if $P\coloneqq 2<3$ then $P$
is true and $not\ P$, which would be $2\ge3$, is false.

\newpage

We can capture the effect of a logical operator using something called a
\emph{truth table}. The truth table for the NOT operator applied to a statement
$P$ would be as follows (T=true and F=false):

\begin{tabular}{|c|c|}
\hline
$P$ & $not\ P$ \\
\hline
T & F \\
\hline
F & T \\
\hline
\end{tabular}

Note that the left side lists all of the possible truth states of $P$ and the
right side lists the resulting truth states of $not\ P$

\subsubsection*{AND}

The AND operator combines two statements $P$ and $Q$, such that $P\ and\ Q$ is
true only when $P$ and $Q$ are both true; it is false whenever $P$, $Q$, or
both are false. The truth table for AND is as follows:

\begin{tabular}{|cc|c|}
\hline
$P$ & $Q$ & $P\ and\ Q$ \\
\hline
T & T & T \\
\hline
T & F & F \\
\hline
F & T & F \\
\hline
F & F & F \\
\hline
\end{tabular}

For example, let:
\begin{eqnarray*}
P &\coloneqq& 2<3 \\
Q &\coloneqq& 3<4 \\
R &\coloneqq& 5<4 \\
\end{eqnarray*}
Then $P\ and\ Q$ is true (since both $P$ and $Q$ are true), but $P\ and\ R$ is
false because even though $P$ is true, $R$ is false.

\subsubsection*{OR}

The OR operator combines two statements $P$ and $Q$, such that $P\ or\ Q$ is
true whenever $P$, $Q$, or both are true. In fact, it is only false when both
$P$ and $Q$ are false. The truth table is as follows:

\begin{tabular}{|cc|c|}
\hline
$P$ & $Q$ & $P\ or\ Q$ \\
\hline
T & T & T \\
\hline
T & F & T \\
\hline
F & T & T \\
\hline
F & F & F \\
\hline
\end{tabular}

For example, assuming the above definitions of $P$, $Q$, and $R$, $P\ or\ Q$ and
$P\ or\ R$ are both true. However, $(not\ Q)\ or\ R$ is false, because both
$not\ Q$ and $R$ are false.

Note that this type of OR is often referred to as an \emph{inclusive} OR, since
the OR statement is true when both statements are true. There is another type
of OR called an \emph{exclusive} OR, which is false when both statements are
true, and thus has the following truth table:

\begin{tabular}{|cc|c|}
\hline
$P$ & $Q$ & $P\ xor\ Q$ \\
\hline
T & T & F \\
\hline
T & F & T \\
\hline
F & T & T \\
\hline
F & F & F \\
\hline
\end{tabular}

Unless explicitly stated otherwise, we assume that all of our OR statements are
inclusive,

\subsection*{Operator Precedence}

When a compound statement contains multiple operators then we need to pay
attention to operator precedence, just like we do with the arithmetic operators
plus and multiply. For logical operators, the order of precedence is:
\begin{enumerate}
\item{NOT}
\item{AND}
\item{OR}
\end{enumerate}
When consecutive operators are the same then NOT is evaluated from right to
left, and AND and OR are evaluated from left to right. We use parentheses if we
need to override this precedence, again, just like in arithmetic.

For example, the compound statement:
\[P\ and\ Q\ or\ not\ not\ P\ and\ not\ Q\]
would be evaluated as follows:
\[(P\ and\ Q)\ or\ (not\ (not\ P)\ and\ (not\ Q))\]
Assuming the definitions of $P$ and $Q$ above, this would be a true statement:
$(P\ and\ Q)$ is true, and thus the OR statement is true. Make sure that you
can see why this is true.

\subsection*{Implication}

Mathematical systems start with a small collection of relatively simple facts
and use those facts to discover new, more complicated facts. This is done by
implication, thus making implication the most important logical construct in
mathematics.

\newpage

Implication is an if-then statement of the form:

\hspace{0.5in}if \emph{(hypothesis)} then \emph{(consequence)}

The hypothesis and consequence are statements. We make the claim that if the
hypothesis is true (know facts), then the consequence must be true (new
facts). Note that such implication must have a proof that supports it.
Discovery of such proofs is the main task in higher mathematics.

As a simple example, consider the implication:

\hspace{0.5in}if $x=1$ then $x<2$

We claim that $x=1$ is a true statement, and can thus conclude that $x<2$; the
implication is a true statement. A more complicated example would be:

\hspace{0.5in}if $x=1$ and $y=2$ then $x+y=3$

Now, the hypothesis is a compound AND statement. We are claiming that the
hypothesis is true, so both parts of the AND must be true, so the consequence
does follow; indeed, $x+y=1+2=3$ and the implication is a true statement. But
consider this example:

\hspace{0.5in}if $x=1$ or $y=2$ then $x+y=3$

Now, the hypothesis is an OR statement, so only one of its parts need be true.
If they are both true then the consequence follows; however, if $x=1$ is true
but $y=2$ is false, then the consequence does not follow. We cannot conclude
that the consequence is true in all cases, so the implication is false.

Often, an arrow is used to indication implication: $P\rightarrow Q$.

Implication actually has a truth table associated with it:

\begin{tabular}{|cc|c|}
\hline
$P$ & $Q$ & $P\to Q$ \\
\hline
T & T & T \\
\hline
T & F & F \\
\hline
F & T & T \\
\hline
F & F & T \\
\hline
\end{tabular}

Note that an implication is only false when the hypothesis is true and the
consequence is false. Thus, the equivalent logic statement for implication is:
$(not P) or Q$. The reason why the implication is always true when the
hypothesis is false is important when dealing with quantifiers, which will be
discussed later.

The direction of the implication is important. For example, consider the
statement:
\[\mbox{If}\ x=2\ \mbox{then}\ x^2=4\]
This is certainly a true statement; however, the other direction:
\[\mbox{If}\ x^2=4\ \mbox{then}\ x=2\]
is not true because $x$ could actually be equal to $-2$.

\subsection*{Equivalence}

There are some implications that are true in both directions. For example, by
slightly modifying the previous example:
\[\mbox{If}\ x=2\ \mbox{or}\ x=-2\ \mbox{then}\ x^2=4\]
This statement is now true in both directions.

When $P\rightarrow Q$ and $Q\rightarrow P$ are both
true then we say that $P$ and $Q$ are \emph{equivalent}. This means that either
both $P$ and $Q$ must be true or that both must be false. As a short cut, we
represent equivalences using \emph{if and only if} statements, usually
abbreviated $P$ iff $Q$ or also $P\leftrightarrow Q$.

The corresponding truth table is as follows:

\begin{tabular}{|cc|c|}
\hline
$P$ & $Q$ & $P\leftrightarrow Q$ \\
\hline
T & T & T \\
\hline
T & F & F \\
\hline
F & T & F \\
\hline
F & F & T \\
\hline
\end{tabular}

A variation on equivalence is to join a list of statements together and say
that \emph{the following are equivalent} --- they are all true together or
all false together. This is sometimes abbreviated as $TFAE$. For example,
let $n$ be an integer. The following are equivalent:
\begin{enumerate}
\item{$n=5$}
\item{$4<n<6$}
\item{$n-3=2$}
\item{$n+1=6$}
\end{enumerate}
Notice that if any one of these statements is true then they must \emph{all}
be true. If any one of them is false then they must \emph{all} be false. In
otherwords, each implies the others.

\subsection*{Quantifiers}

Although the statement $1<2$ may be true, it is sort of obvious and not very
interesting. Instead, what we might want to do is compare all the values in
some, possibly infinite, collection of values to the number 2 --- something
like $x<2$ for all considered values of $x$. We do this using parameterized
statements and quantifiers.

\subsubsection*{Parameterized Statements}

We determined that something like $x<2$ is not a statement until we have a
definite value of $x$. We call something like this a \emph{parameterized}
statement, where $x$ is the parameter. When using letters to represent
statements, we add the parameter as follows:
\[P(x)\coloneqq x<2\]
Now, we need a way to provide various values of $x$. We will see how to
identify a collection of possible $x$ when we look at \emph{sets}. Until then,
we will assume that we have some collection of $x$ values in which we are
interested. We then apply those $x$ values to our parameterized statement using
a \emph{quantifier}.

\subsubsection*{Universal Quantifier}

We use the universal quantifier to test all of the values in our collection.
The universal quantifier uses the words ``for all'', so we would say something
like:

\hspace{0.5in}For all $x$, $x<2$

Using mathematical syntax, we would write: $\forall x, P(x)$. The upside-down
`A' stands for ``for all''. This is a statement that is true if $P(x)$ is true
for all of our $x$ values. It is false if there is at least one $x$ value for
which P(x) is false. So, you can think of the universal quantifier as a
shortcut for a long compound AND statement:
\[\forall x,P(x)\coloneqq P(x_1)\ and\ P(x_2)\ and\ P(x_3)\ and\ \ldots\]
and indeed we must use a universal quantifier when there are a large or even
infinite number of possible $x$ values. If we can find at least one $x$ value
such that $P(x)$ is false then the whole statement is false. We call such a
failing $x$ value a \emph{counterexample}.

For example, let our collection of $x$ values be the following collection of
numbers: 2,4,6,8, and 11. Define $P(x)$ to be the parameterized statement:
$x<12$. We can see that $\forall x,P(x)$ is a true statement because all of the
candidate values for $x$ are indeed less than 12. But if we let
$Q(x)\coloneqq x$ is an even number, then $\forall x,Q(x)$ is a false
statement because the value 11 is not even and is thus a counterexample.

We mentioned before that an if-then statement is always true when the
hypothesis is false. We can see why this is convenient when implication is used
with the universal quantifier. Continuing with the previous example, define
$R(x)$ to be the implication: if $x<10$ then $x$ is an even number. Now
consider $\forall x,R(x)$. This time, when we get to the value 11, the
hypothesis fails and the implication is true, thus preventing it from being a
counterexample. In other words, the implication acts as a sort of short-circuit
that filters out those values of $x$ that do not concern us.

\newpage

\subsubsection*{Existential Quantifier}

We use the existential quantifier to see if there is at least one value in our
collection that makes our parameterized statement true. The existential
quantifier uses the words ``there exists'', so we would say something line:

\hspace{0.5in}There exists $x$ such that $x<2$.

Using mathematical syntax, we would write: $\exists x,P(x)$. The backwards `E'
stands for ``there exists''. This is a statement that is true if $P(x)$ is true
for at least one of our $x$ values --- there can be more, but there has to be
at least one. It is false if $P(x)$ is false for all of our $x$ values. So, you
can think of the existential quantifier as a shortcut for a long compound OR
statement:
\[\exists x,P(x)\coloneqq P(x_1)\ or\ P(x_2)\ or\ P(x_3)\ or\ \ldots\]

For example, let our collection of $x$ values be the numbers 3, 4, and 5 and
define $P(x):=x<4$. We can see that $\exists x,P(x)$ is a true statement since
there is at least one $x$ value (3) that is less than 4. If we define
$Q(x):=x<5$ then $\exists x,Q(x)$ is also true since 3 and 4 are less than 5
--- we only need one! But if we define $R(x):=x<3$ then $\exists x,R(x)$ is a
false statement because all of the candidate $x$ values are greater than or
equal to 3.

There is a similar quantifier denoted $\exists!$ that stands for ``there exists
one and only one'' that we use when we are interested in the existence of a
unique value. For example, let our collection of $x$ values be the numbers 3,
4, and 5 and $P(x):=x\le4$. We can see that $\exists! x,P(x)$ is a false
statement because both 3 and 4 are less than or equal to 4.

\subsubsection*{Nested Quantifiers}

Quantifiers can be nested, mixed, and matched. For example, the statement:

\hspace{0.5in}For all $x$ there exists a $y$ such that $x+y=0$

would be written as:
\[\forall x,\exists y,x+y=0\]
If we let our candidate $x$ values be the numbers 1, 2, and 3 and the
candidate y values be -1, -2, and -3, then the above nested quantifier
statement is true: for each candidate value of $x$, there is a $y$ value such
that $x+y=0$. If we were to add the value 4 to our possible values of $x$ then
the nested quantifier statement is false, because for the candidate $x$ value
of 4, none of the possible $y$ values satisfies $4+y=0$.

\end{document}
