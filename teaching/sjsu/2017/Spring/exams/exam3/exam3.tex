\documentclass[letterpaper,12pt,fleqn]{article}
\usepackage{matharticle}
\pagestyle{plain}
\begin{document}

\begin{center}
\Large Math-1003b Exam \#3
\end{center}

\vspace{0.5in}

Name: \rule{4in}{1pt}

\vspace{0.5in}

This exam is closed book and notes. You may use a scientific calculator; however, no
other electronics are allowed. Show all work; there is no credit for guessed answers. All
answers must be in simplified form with no negative exponents.

\vspace{0.25in}

\begin{enumerate}
\item Simplify the following expressions:
  \begin{enumerate}
  \item $\sqrt[4]{16r^8t^{32}}$

    \vspace{1in}
    
  \item $\sqrt[5]{-32x^{30}y^{45}}$

    \vspace{1in}
    
  \end{enumerate}

\item You are standing in front of a building with a second floor window. You don't
  have access to the building, but you need to know how high up the window is on the
  wall of the building. You have a 20 foot ladder that you lay against the side of the
  building so that the top of the ladder touches the bottom of the window. The bottom
  of the ladder is 3 feet away from the building. How high is the window?

  \newpage
  
\item Convert each of the following expressions to radical form and then simplify
  them. If the expression does not represent a real number then say
  ``not a real number''. When simplifying, you may start with either the rational
  exponent or radical form:
  \begin{enumerate}
  \item $16^{\frac{3}{4}}$

    \vspace{1in}
    
  \item $81^{-\frac{1}{2}}$

    \vspace{1in}
    
  \item $(-49)^{\frac{3}{2}}$

    \vspace{1in}
    
  \item $-100^{\frac{1}{2}}$

    \vspace{1in}
    
  \item $(-8)^{-\frac{4}{3}}$

    \vspace{1in}
    
  \end{enumerate}

  \newpage

\item Simplify the following expressions. You may assume that the domain for all the
  variables is $[0,\infty)$.
  \begin{enumerate}
  \item \[\left(x^{\frac{1}{3}}x^{\frac{1}{6}}\right)^{12}\]

    \vspace{3in}
      
  \item \[\left(\frac{4t^{-\frac{2}{3}}}{t^{\frac{4}{3}}}\right)^2\]

    \vspace{3in}
      
  \end{enumerate}

  \newpage
  
\item Simplify the following expressions. You may assume that the domain for all the
  variables is $[0,\infty)$:
  \begin{enumerate}
  \item $\sqrt[3]{54}-\sqrt[3]{128}$

    \vspace{3in}
      
  \item $5p\sqrt{18p^2q^3}+p^2q\sqrt{32q}$

    \vspace{3in}
      
  \end{enumerate}

  \newpage
  
\item Perform the following operations and simplify the results:
  \begin{enumerate}
  \item $(2\sqrt{3}-3\sqrt{5})(\sqrt{3}+2\sqrt{5})$

    \vspace{3in}
      
  \item $(\sqrt{x}-7)(\sqrt{x}+7)$

    \vspace{3in}
      
  \end{enumerate}

  \newpage

\item Rationalize the denomimators for the following expressions and simplify. You may
  assume that the domain for all the variables is $[0,\infty)$:
  \begin{enumerate}
  \item \[\frac{4}{\sqrt[3]{2a}}\]

    \vspace{3in}
      
  \item \[\frac{3}{\sqrt{x}-3}\]

    \vspace{3in}
      
  \end{enumerate}

  \newpage
  
\item Solve for $x$:
  \[3+\sqrt[3]{x-16}=1\]

    \vspace{2in}
      
\item Solve for $x$:
  \[3+\sqrt{x-16}=1\]

    \vspace{2in}
      

\item Solve for $x$:
  \[\sqrt{2x+6}+1=x\]
\end{enumerate}

\end{document}
