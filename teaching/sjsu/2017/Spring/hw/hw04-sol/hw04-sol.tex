\documentclass[letterpaper,12pt,fleqn]{article}
\usepackage{matharticle}
\usepackage{siunitx}
\pagestyle{plain}
\begin{document}

\begin{center}
\Large Math-1003b Homework \#4 Solutions
\end{center}

\vspace{0.5in}

\underline{Reading}

\bigskip

\begin{itemize}
\item Review pages 168-9 on set-builder and interval notations
\item Sections 7.7, 9.1 (on sets), 8.1, 8.2
\end{itemize}

\bigskip

\underline{Problems}

\bigskip

\begin{enumerate}
\item It costs a farmer \$0.27 to farm each square foot of land. The farmer has
  one field that is 2 hectares in size, and another that is 3 acres in size.
  How much does it cost to farm all of the land? Use the following conversions:

  1 hectare = 2.47 acres \\
  640 acres = 1 square mile \\
  1 mile = 5280 feet

  We need the combined area of the two fields in square feet, so we use the
  proper conversion factors. Note that the units cancel properly to leave us
  with $ft^2$. Also note that we need to square the linear miles/feet conversion
  factor in order to get area units:
  \[\SI{2}{hectares}\left(\frac{\SI{2.47}{acres}}{\SI{1}{hectare}}\right)
  \left(\frac{\SI{1}{mi^2}}{\SI{640}{acres}}\right)
  \left(\frac{\SI{5280}{feet}}{\SI{1}{mile}}\right)^2=\SI{215186}{ft^2}\]
  \[\SI{3}{acres}\left(\frac{\SI{1}{mi^2}}{\SI{640}{acres}}\right)
  \left(\frac{\SI{5280}{feet}}{\SI{1}{mile}}\right)^2=\SI{130680}{ft^2}\]
  Notice that any fractional square feet is meaningless because of the size of
  the values, so we round to the nearest square foot. Now, calculate the cost
  for the two fields:
  \[\left(\SI{215186}{ft^2}+\SI{130680}{ft^2}\right)
  \left(\frac{\$0.27}{\SI{1}{ft^2}}\right)=\$93,384\]
  Again, we drop the cents as meaningless compared to the large dollar value.

\item A biker starts out from home and rides for 4 miles before getting a
  flat tire. He walks back home carrying his bike, which takes one hour
  longer than the ride. The biker can ride nine miles per hour faster than he
  can walk. How fast can he walk and how fast can he ride?

  This problem is like Example 6 in the book that we worked on in class. If we
  let $t_b$ be the riding time and $t_w$ be the walking time, then the basic
  structure is:
  \[t_w-t_b=\SI{1}{hour}\]
  From $d=rt$, we get the necessary equation: $t=\frac{d}{r}$, so the equation
  becomes:
  \[\frac{d_w}{r_w}-\frac{d_b}{r_b}=\SI{1}{hour}\]
  The distances are the same ($\SI{4}{miles}$), but we will need to express the
  rates with a variable.  So let $r_w=x$ the walking rate. Since biking is
  $\SI{9}{mph}$ faster, $r_b=r_w+9=x+9$. Everything is in miles and hours, so
  the units are consistent and we can ignore them for now:
  \[\frac{4}{x}-\frac{4}{x+9}=1\]
  Multiplying through by the LCM of $x(x+9)$ we get:
  \begin{eqnarray*}
    4(x+9)-4x &=& x(x+9) \\
    4x+36-4x &=& x^2+9x \\
    x^2+9x-36 &=& 0 \\
    (x+12)(x-3) &=& 0 \\
    x &=& -12,3
  \end{eqnarray*}
  We take the positive value and state the final answer:

  Walking rate $=\SI{3}{mph}$

  Biking rate $=3+9=\SI{12}{mph}$

  Notice how the negative solution gives us the other answer - as we mentioned
  in class, this is common in these types of problems. In fact, if we had let
  $x=r_b$ we would have gotten $x=-3,12$ as solutions!

\item A parent tells his/her kids to wash the family car. Working alone, the
  first kid can wash the car 10 minutes faster than the other kid, but working
  together they can wash the car in 12 minutes. How long does it take for each
  kid to wash the car working alone?

  This is a shared work problem, so the basic structure is:
  \[r_1+r_2=r_{1+2}\]
  where each rate is in the form: job/min. Let $t$ minutes be the time for the
  slower kid to wash the car. The faster kid's time is then $t-10$ minutes. The
  rate equation is thus:
  \[\frac{1}{t}+\frac{1}{t-10}=\frac{1}{12}\]
  Multiplying by the LCM of $12t(t-10)$ we get:
  \begin{eqnarray*}
    12(t-10)+12x &=& x(x-10) \\
    12t-120+12t &=& t^2-10t \\
    t^2-34t+120 &=& 0 \\
    (t-30)(t-4) &=& 0 \\
    t &=& 4,30
  \end{eqnarray*}
  Note that the $\SI{4}{min}$ answer doesn't make any sense (can you see why?),
  so we state the final answer:

  Kid 1 time $=\SI{30}{min}$

  Kid 2 time $=30-10=\SI{20}{min}$

  Note that this time we didn't get a positive and negative answer --- you
  cannot count on that always happening, but it is nice when it does!

\item Let:
  \[A=\{1,2,3,4\}\]
  \[B=\{2,4,6\}\]
  Determine the following sets:
  \begin{enumerate}
  \item $A\cup B$

    For an element to be in the union, it can be in set $A$ \emph{or} set $B$
    (or both). Remember that elements are distinct, so we select each element
    at most once:
    \[A\cup B=\{1,2,3,4,6\}\]
      
  \item $A\cap B$

    For an element to be in the intersection, it must be in both $A$ \emph{and}
    $B$:
    \[A\cap B=\{2,4\}\]
    
  \item $A-B$

    We start with the elements in $A$ and discard anything that is also in $B$.
    Note that we ignore any additional elements in $B$ not in $A$:
    \[A-B=\{1,3\}\]
    
  \item $A\times B$

    There are $4$ elements in $A$ and $3$ in $B$, so the resulting cross
    product should have $4\cdot3=12$ elements. Note that each element of the
    cross-product set is an ordered pair:
    \[A\times B=\{(1,2),(1,4),(1,6),(2,2),(2,4),(2,6),(3,2),(3,4),(3,6),
    (4,2),(4.4),(4,6)\}\]
  \end{enumerate}
\end{enumerate}
\end{document}
