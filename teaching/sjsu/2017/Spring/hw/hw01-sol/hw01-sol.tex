\documentclass[letterpaper,12pt,fleqn]{article}
\usepackage{matharticle}
\pagestyle{plain}
\begin{document}

\begin{center}
\Large Math-1003b Homework \#1 Solutions
\end{center}

\vspace{0.5in}

\underline{Reading}

\bigskip

\begin{itemize}
\item Really focus on your lecture notes for the information on rational
  numbers.
\item Review textbook chapters 1, 2, and 5.
\item Text book sections 7.1 and 7.2.
\end{itemize}

\bigskip

\underline{Problems}

\bigskip

\begin{enumerate}
\item Questions about rational numbers. Be sure to explain why a statement is
  either true or false. Provide and explain counterexamples for all false
  statements.
  \begin{enumerate}
  \item Describe the three forms of a rational number covered in class and give
    an example of each.

    \begin{enumerate}[label={\arabic*.}]
    \item Fractional: $\frac{p}{q}$ where $p$ and $q$ are integers and
      $q\ne0$. Example: $\frac{2}{3}$

    \item Finite Decimal. Example: $1.23$

    \item Infinite Repeating Decimal. Example: $1.\overline{23}$
    \end{enumerate}

  \item Is zero a rational number?

    Yes, because it can be written as $\frac{0}{1}$ (different syntax, same
    semantic), which fulfills the definition of a rational number.

  \item Is $\frac{\sqrt{9}}{2}$ a rational number?

    Yes, because it is the same as $\frac{3}{2}$ (once again, different syntax,
    same semantic), which is a rational number.

  \item Is every fraction a rational number?

    No. Counterexample: $\frac{\pi}{2}$. Since $\pi$ is not an integer, the
    fraction does not meet the requirements for a rational number.

  \item Based on the class discussion of rational numbers (i.e., the meaning
    of $\frac{p}{q}$ on the number line), explain why zero can never appear in
    the denominator.

    By the definition of division, $\frac{p}{q}=p\left(\frac{1}{q}\right)$.
    Thus, we divide the space on the number line between $0$ and $1$ into
    $q$ finite partitions, each of length $\frac{1}{q}$, and then take $p$ of
    them. You cannot divide the space between $0$ and $1$ into $0$ finite
    partitions.
  \end{enumerate}

\vspace{0.5in}

\item Consider the numbers $120$ and $252$.
  \begin{enumerate}
  \item State the prime factorization for each.

    $120=2^3\cdot3\cdot5$

    $252=2^2\cdot3^2\cdot7$

    When we take the GCD and LCM of these two numbers, we assume that any
    missing primes in one of the numbers actually has a power of $0$. Thus, it
    may be helpful to view the prime factorizations as:

    $120=2^3\cdot3^1\cdot5^1\cdot7^0$
    
    $252=2^2\cdot3^2\cdot5^0\cdot7^1$

  \item Show how to determine and then state their least common multiple (LCM).

    For the LCM we take the highest power of each prime:

    $[120,252]=2^3\cdot3^2\cdot5^1\cdot7^1=2520$

  \item Show how to determine and then state their greatest common denominator
    (GCD).

    For the GCD we take the lowest power of each prime:

    $(120,252)=2^2\cdot3^1\cdot5^0\cdot7^0=12$
    
  \item Show how to simplify $\frac{120}{252}$ using the above information.

    To reduce to simplest form, divide the numerator and the denominator by
    the GCD:

    $\frac{120}{252}=\frac{\frac{120}{12}}{\frac{252}{12}}=\frac{10}{21}$

    Alternatively, use the prime factorizations and the exponent rule:

    $\frac{120}{252}=\frac{2^3\cdot3\cdot5}{2^2\cdot3^2\cdot7}=
    \frac{2\cdot5}{3\cdot7}=\frac{10}{21}$
  \end{enumerate}

\vspace{0.5in}

\item Rewrite $4-9x^2$ by factoring out $(-3x)$.

  Remember that in the reals, we can factor any non-zero value out of any other
  value. To start, multiply by $\frac{-3x}{-3x}=1$:
  \[4-9x^2=1(4-9x^2)=\frac{-3x}{-3x}(4-9x^2)\]
  Note that since we multiplied by $1$ (the multiplicative identity), we do not
  change the value of the expression. Now apply the definition of division and
  the associative rule:
  \[4-9x^2=(-3x)\left[\frac{1}{-3x}(4-9x^2)\right]\]
  We now use the distributive rule:
  \[4-9x^2=(-3x)\left(\frac{4}{-3x}+\frac{-9x^2}{-3x}\right)=
  (-3x)\left(-\frac{4}{3x}+3x\right)\]
  When done, always check by working backwards and seeing if you get the
  original expression.

\newcommand{\rea}{\frac{2x^2+5x+2}{x^2-3x}}
\newcommand{\reb}{\frac{2x^3+4x^2}{x^2-9}}

\item Operations on rational expressions:
  \begin{enumerate}
  \item Fully factor:
    \[\rea=\frac{(2x+1)(x+2)}{x(x-3)}\]
    
  \item Fully factor:
    \[\reb=\frac{2x^2(x+2)}{(x+3)(x-3)}\]
    
  \item Determine and fully simplify:
    \begin{eqnarray*}
      \rea\cdot\reb &=&
      \frac{(2x+1)(x+2)}{x(x-3)}\cdot\frac{2x^2(x+2)}{(x+3)(x-3)} \\
      &=& \frac{(2x+1)(x+2)(2x^2)(x+2)}{x(x-3)(x+3)(x-3)} \\
      &=& \frac{2x^2(x+2)^2(2x+1)}{x(x+3)(x-3)^2} \\
      &=& \frac{2x(x+2)^2(2x+1)}{(x+3)(x-3)^2} \\
    \end{eqnarray*}
    
  \item Determine and fully simplify:
    \begin{eqnarray*}
      \rea\div\reb &=& \rea\cdot\frac{x^2-9}{2x^3+4x^2} \\
      &=& \frac{(2x+1)(x+2)}{x(x-3)}\cdot\frac{(x+3)(x-3)}{2x^2(x+2)} \\
      &=& \frac{(2x+1)(x+2)(x+3)(x-3)}{x(x-3)(2x^2)(x+2)} \\
      &=& \frac{(2x+1)(x+2)(x+3)(x-3)}{2x^3(x-3)(x+2)} \\
      &=& \frac{(2x+1)(x+3)}{2x^3} \\
    \end{eqnarray*}
  \end{enumerate}
\end{enumerate}
\end{document}
