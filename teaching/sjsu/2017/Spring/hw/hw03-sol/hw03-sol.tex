\documentclass[letterpaper,12pt,fleqn]{article}
\usepackage{matharticle}
\pagestyle{plain}
\begin{document}

\begin{center}
\Large Math-1003b Homework \#3 Solutions
\end{center}

\vspace{0.5in}

\underline{Reading}

\bigskip

\begin{itemize}
\item Review sections 7.1 and 7.4.
\item Sections 7.5, 7.6, and 7.7.
\end{itemize}

\bigskip

\underline{Problems}

\bigskip

\begin{enumerate}
\item Simplify:
  \[\frac{x^2-\frac{2}{3}x-\frac{5}{4}}{9x^4-4x^2}\cdot(2x+5)^2\div
  \frac{(3x-1)^3}{3x-2}\]

  We want to start by getting rid of the rational coefficients in the first numerator.
  We can do this by using the LCM (12):
  \begin{eqnarray*}
    x^2-\frac{2}{3}x-\frac{5}{4} &=&
    \frac{12}{12}\left(x^2-\frac{2}{3}x-\frac{5}{4}\right) \\
    &=& \frac{1}{12}\left[12\left(x^2-\frac{2}{3}x-\frac{5}{4}\right)\right] \\
    &=& \frac{1}{12}(12x^2-8x-15) \\
    &=& \frac{1}{12}(6x+5)(2x-3)
  \end{eqnarray*}
  Note that the last factoring step is tough because there are so many possibilities.
  In this case, I think that the AC method (p441) gets you to an answer quicker.

  Look at the first denomimator, we can factor out an $x^2$ and then factor the resulting
  difference of squares:
  \[9x^4-4x^2=x^2(9x^2-4)=x^2(3x+2)(3x-2)\]

  Now, rewriting the problem with the factored first factor and resolving the division
  sign (i.e., flip the last factor):
  \[\frac{\frac{1}{12}(6x+5)(2x-3)}{x^2(3x+2)(3x-2)}\cdot\frac{(2x+5)^2}{1}\cdot
  \frac{3x-2}{(3x-1)^3}\]

  This is just a multiplication of fractions problem now, so multiply across:
  \[\frac{\frac{1}{12}(6x+5)(2x-3)(2x+5)^2(3x-2)}{x^2(3x+2)(3x-2)(3x-1)^3}\]

  And now cancel the $(3x-2)$:
  \[\frac{\frac{1}{12}(6x+5)(2x-3)(2x+5)^2}{x^2(3x+2)(3x-1)^3}\]

  We aren't quite done yet. We don't want to leave a fraction in the numerator. Note
  that we can bring the factor of $\frac{1}{12}$ out front:
  \[\frac{1}{12}\cdot\frac{(6x+5)(2x-3)(2x+5)^2}{x^2(3x+2)(3x-1)^3}\]

  And then the 12 just moves to the denominator:
  \[\frac{(6x+5)(2x-3)(2x+5)^2}{12x^2(3x+2)(3x-1)^3}\]

\item Simplify:
  \[\frac{\frac{5}{w^2-25}-\frac{3}{w+5}}{\frac{4}{w-5}}\]

  First, factor the $w^2-25$ as a difference of squares:
  \[\frac{\frac{5}{(w+5)(w-5)}-\frac{3}{w+5}}{\frac{4}{w-5}}\]

  The goal is to get rid of the complexity.  Once again, we use the LCM to do this -
  multiplying above and below gets rid of all the denominators:
  \begin{eqnarray*}
    \frac{(w+5)(w-5)}{(w+5)(w-5)}\cdot
    \frac{\frac{5}{(w+5)(w-5)}-\frac{3}{w+5}}{\frac{4}{w-5}} &=&
    \frac{(w+5)(w-5)\left[\frac{5}{(w+5)(w-5)}-\frac{3}{w+5}\right]}
         {(w+5)(w-5)\left[\frac{4}{w-5}\right]} \\
    &=& \frac{5-3(w-5)}{4(w+5)} \\
    &=& \frac{5-3w+15}{4(w+5)} \\
    &=& \frac{20-3w}{4(w+5)}
  \end{eqnarray*}

  Finally, we like out highest order terms to have a positive coefficient, so lets go
  ahead and factor a (-1) out of the numerator:
  \[\frac{-(3w-20)}{4(w+5)}\]
  Now we can bring the minus sign out front for the final answer:
  \[-\frac{3w-20}{4(w+5)}\]

\item Solve for $a$:
  \[\frac{6}{5a+10}-\frac{1}{a-5}=\frac{4}{a^2-3a-10}\]

  Start by factoring where necessary:
  \[\frac{6}{5(a+2)}-\frac{1}{a-5}=\frac{4}{(a-5)(a+2)}\]

  Now multiply both sides by the LCM:
  \begin{eqnarray*}
    5(a-5)(a+2)\left[\frac{6}{5(a+2)}-\frac{1}{a-5}\right] &=&
    5(a-5)(a+2)\left[\frac{4}{(a-5)(a+2)}\right] \\
    6(a-5)-5(a+2) &=& 5(4) \\
    6a-30-5a-10 &=& 20 \\
    a-40 &=& 20 \\
    a &=& 60
  \end{eqnarray*}

  Before you conclude that this is the final answer, make sure that plugging it into the
  original equation doesn't result in an zero denominators.

\item A building has a window that is higher than you can reach and you have a
  short measuring tape that doesn't reach all of the way (i.e., you can't just
  measure from the window to the ground). You do have a 20ft ladder. So, you
  mark 5ft up the ladder and then lean it against the building so that the top
  of the ladder is at the bottom of the window. You then measure from your 5ft
  mark to the ground - it measures 4ft. How high up is the window? (Hint: draw
  a picture, mark the distances, and look for triangles!)

  First, note that all of the units are consistent - everything is in feet - so we can
  ignore the units for now.

  \begin{minipage}{3in}
    \begin{tikzpicture}
      \draw (0,0) -- (5,0) -- node [right] {$x$} (5,7) -- (0,0);
      \node [circle,fill,scale=0.2] (p) at ({3/2},{21/10}) {};
      \draw ({3/2},0) -- node [right] {$4$} (p);
      \node [above left] at ({3/4},{21/20}) {$5$};
      \node [above left] at ({5/2},{7/2}) {$20$};
    \end{tikzpicture}
  \end{minipage}
  \begin{minipage}{3in}
    Consider the inner and outer triangles. They share a common angle on the left and
    both have a right angle (with the ground). Thus, since the sum of the interior angles
    is fixed ($180^{\circ}$), their third angles must also be equal. You can also note
    that the third angles are equal by seeing that the string and the building wall are
    parallel and the ladder acts as a traversal - the two corresponding angles are thus
    equal.
  \end{minipage}

  Since all the corresponding angles are equal, the triangles are similar. So, we can
  build a proportionality equation with the sides. This can be dones several different
  ways, but here is one:
  \[\frac{4}{5}=\frac{x}{20}\]
  We then solve for $x$:
  \[5x=80\]
  \[x=16\]
  Finally, state the final answer with units:

  The window height is $16$ feet.
\end{enumerate}
\end{document}
