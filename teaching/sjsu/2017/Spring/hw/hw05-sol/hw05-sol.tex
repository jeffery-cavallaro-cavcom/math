\documentclass[letterpaper,12pt,fleqn]{article}
\usepackage{matharticle}
\pagestyle{plain}
\begin{document}

\begin{center}
\Large Math-1003b Homework \#5 Solutions
\end{center}

\vspace{0.5in}

\underline{Reading}

\bigskip

\begin{itemize}
\item Sections 8.1-8.4
\end{itemize}

\bigskip

\underline{Problems}

\bigskip

\begin{enumerate}
\item Let $f(x)=x^2+2x-5$. Evaluate:
  \[\frac{f(x+h)-f(x)}{h}\]

  Remember, to evaluate a function at a particular value, even if that value is
  specified in terms of the same or other variables, simply replace each
  occurrence of the variable in the original equation with the target value:
  \[f(x+h)=(x+h)^2+2(x+h)-5=x^2+2xh+h^2+2x+2h-5\]
  Now, putting it altogether (remember to use parentheses during substitution!):
  \begin{eqnarray*}
    \frac{f(x+h)-f(x)}{h} &=& \frac{(x^2+2xh+h^2+2x+2h-5)-(x^2+2x-5)}{h} \\
    &=& \frac{2xh+h^2+2h}{h} \\
    &=& 2x+h+2
  \end{eqnarray*}
  Notice that only terms with an $h$ were left in the numerator, to be canceled
  by the denominator! This is a very important form, called the
  \emph{difference quotient} form, in calculus. In fact, those of you who go on
  to take calculus (and I sincerely hope that some of you do), will encounter
  this form in the first week.

\item Let $f(x)=2x+5$ and $g(x)=x^2$. For each of the following,
  evaluate and state the domain.
  \begin{enumerate}
  \item $f+g$
    
    $(f+g)(x)=f(x)+g(x)=(2x+5)+(x^2)=x^2+2x+5$

    Domain: $\R$
    
  \item $fg$

    $(fg)(x)=f(x)g(x)=(2x+5)(x^2)=2x^3+5x^2$
    
    Domain: $\R$
    
  \item $\frac{f}{g}$

    $\left(\frac{f}{g}\right)(x)=\frac{f(x)}{g(x)}=\frac{2x+5}{x^2}$

    Since there is a possibility of a zero denominator, $x\ne0$. So, stating
    domain in setbuilder and interval form:

    Domain: $\{x\in\R\mid x\ne0\}=(-\infty,0)\cup(0,\infty)$
    
  \item $\frac{f}{f}$

    $\left(\frac{f}{f}\right)(x)=\frac{f(x)}{f(x)}=\frac{2x+5}{2x+5}=1$

    Since the result is a constant function, it looks like there are no
    limitations on $x$; however, we need to honor the original form, where
    $x\ne-\frac{5}{2}$:
    
    Domain: $\{x\in\R\mid x\ne-\frac{5}{2}\}=
    (-\infty,-\frac{5}{2})\cup(-\frac{5}{2},\infty)$

  \item $f\circ g$

    Remember that the \emph{inner} function goes first:

    $(f\circ g)(x)=f(g(x))=f(x^2)=2x^2+5$

    Domain: $\R$

    Make sure that you also understand the following:
    
    $(g\circ f)(x)=g(f(x))=g(2x+5)=(2x+5)^2$
  \end{enumerate}
\end{enumerate}
\end{document}
