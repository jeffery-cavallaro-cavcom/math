\documentclass[letterpaper,12pt,fleqn]{article}
\usepackage{matharticle}
\usepackage[makeroom]{cancel}
\pagestyle{plain}
\begin{document}

\begin{center}
\Large Math-1003b Homework \#2 Solutions
\end{center}

\vspace{0.5in}

\underline{Reading}

\bigskip

\begin{itemize}
\item Text book sections 7.3 and 7.4.
\end{itemize}

\bigskip

\underline{Problems}

\bigskip

\begin{enumerate}
\item Simplify:
  \[\frac{3x}{x^2+6x+9}-\frac{x}{x^2+5x+6}\]

  Start by factoring everything:
  \[\frac{3x}{(x+3)^2}-\frac{x}{(x+2)(x+3)}\]
  Next, determine the LCM. List all of the unique factors in all denominators
  and then take the highest power of each factor:
  \[LCM=(x+2)(x+3)^2\]
  Now, combine the two rational expressions using the LCM as the common
  denominator, and multiplying each numerator by what is ``missing'' in the
  corresponding denominator:
  \[\frac{3x(x+2)-x(x+3)}{(x+2)(x+3)^2}\]
  Next, use polynomial multiplication and addition to simplify the numerator.
  Note that this is the only time that we expand things. Be careful of the
  minus sign!:
  \[\frac{(3x^2+6x)-(x^2+3x)}{(x+2)(x+3)^2}\]
  \[\frac{(3x^2-x^2)+(6x-3x)}{(x+2)(x+3)^2}\]
  \[\frac{2x^2+3x}{(x+2)(x+3)^2}\]
  Finally, factor the numerator to see if anything cancels with the
  denominator; nothing does in this case:
  \[\frac{x(2x+3)}{(x+2)(x+3)^2}\]

  \bigskip
  
\item Simplify:
  \[\frac{2}{a+b}-\frac{2}{a-b}+\frac{4a}{a^2-b^2}\]

  \bigskip

  \[\frac{2}{a+b}-\frac{2}{a-b}+\frac{4a}{(a+b)(a-b)}\]
  \[\frac{2(a-b)-2(a+b)+4a}{(a+b)(a-b)}\]
  \[\frac{2a-2b-2a-2b+4a}{(a+b)(a-b)}\]
  \[\frac{(\cancel{2a}-\cancel{2a}+4a)+(-2b-2b)}{(a+b)(a-b)}\]
  \[\frac{4a-4b}{(a+b)(a-b)}\]
  \[\frac{4\cancel{(a-b)}}{(a+b)\cancel{(a-b)}}\]
  \[\frac{4}{a+b}\]
\end{enumerate}
\end{document}
