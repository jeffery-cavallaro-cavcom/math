\documentclass[letterpaper,12pt,fleqn]{article}
\usepackage{matharticle}
\usepackage{siunitx}
\pagestyle{plain}
\begin{document}

\begin{center}
\Large Math-1003b Homework \#6
\end{center}

\vspace{0.5in}

\underline{Reading}

\bigskip

\begin{itemize}
\item Section 8.5
\end{itemize}

\bigskip

\underline{Problems}

\bigskip

The amount of heat energy ($Q$) needed to change the temperature of an
object (without going through a phase change like melting or boiling) is jointly
proportional to the mass of the object ($m$) and the \emph{change} in
temperature ($\Delta T$).
\begin{enumerate}
\item Write an equation that models this physical phenomenon. Use $c$ for the
  constant of proportionality.
\item The MKS unit for heat energy is the Joule (J). The constant of
  proportionality is specific to the substance being heated and is referred to
  as the \emph{specific heat} of the substance. If $Q$ is measured in Joules
  ($J$), $m$ is measured in grams ($g$), and temperature is measured in Kelvin
  (K), what are the units of $c$?
\item In the lab, it is found that $\SI{41790}{J}$ of heat energy raises the
  temperature of $\SI{1}{L}$ of water by $\SI{10}{K}$. What is the specific heat
  of water? ($\SI{1}{L}$ of water$=\SI{1000}{g}$)
\item How much energy (in Joules) is required to raise the temperature of
  $\SI{5}{L}$ of water by $\SI{5}{K}$?
\end{enumerate}

\end{document}
