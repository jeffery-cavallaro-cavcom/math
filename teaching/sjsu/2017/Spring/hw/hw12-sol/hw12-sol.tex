\documentclass[letterpaper,12pt,fleqn]{article}
\usepackage{matharticle}
\usepackage{siunitx}
\pagestyle{plain}
\begin{document}

\begin{center}
\Large Math-1003b Homework \#12 Solutions
\end{center}

\vspace{0.5in}

\underline{Reading}

\bigskip

\begin{itemize}
\item Sections 10.2 and 10.3
\end{itemize}

\bigskip

\underline{Problems}

\bigskip

\begin{enumerate}
\item A student writes the following statements. Determine if each is either
correct or incorrect (or misleading). Explain why incorrect statements are
incorrect.
\begin{enumerate}
\item $\sqrt{9}=\pm3$

  This is incorrect. When we write $\sqrt{9}$ we want the principal (positive) root
  only, not the negative root. The correct form is:
  \[\sqrt{9}=3\]
  
\item $\left(x^{\frac{1}{2}}\right)^2=\abs{x}$

  This is misleading. Because there is an even fractional root in the inside, the implied
  domain for $x$ is $[0,\infty)$. Thus, the result is always positive and the absolute
  value is not needed.
    
\item $\left(x^2\right)^{\frac{1}{2}}=x$

  This is incorrect. Notice that $x$ went from an even power to an odd power. Consider
  what would happen if $x<0$ - the LHS would be positive but the RHS would be negative.
  We need absolute value here:
  \[\left(x^2\right)^{\frac{1}{2}}=\abs{x}\]
  
\item $\left(x^3\right)^{\frac{1}{3}}=\abs{x}$

  This is incorrect. We don't use absolute value with odd powers/roots. Consider what
  would happen if $x<0$ - the LHS would be negative, so we definitely don't want the
  absolute value on the RHS:
  \[\left(x^3\right)^{\frac{1}{3}}=x\]
  
\item $\left(x^{\frac{1}{3}}\right)^3=x$

  This is correct. No absolute value is needed with odd powers/roots.
\end{enumerate}

\item Simplify completely. Your answer should contain no negative exponents and
  should include absolute values where appropriate:
  \[\left(\frac{x^4}{y^{-6}}\right)^{-\frac{1}{2}}\left(x^3y^{-2}\right)\]
  \begin{eqnarray*}
    \left(\frac{x^4}{y^{-6}}\right)^{-\frac{1}{2}}\left(x^3y^{-2}\right) &=&
    \left[\frac{(x^4)^{-\frac{1}{2}}}{(y^{-6})^{-\frac{1}{2}}}\right]\left(x^3y^{-2}\right) \\
    &=& \left(\frac{x^{-2}}{y^3}\right)\left(x^3y^{-2}\right) \\
    &=& xy^{-5} \\
    &=& \frac{x}{y^5}
  \end{eqnarray*}
  Now we need to check for the need for absolute value. Note that $x$ has an odd power
  in the original problem so no absolute value needed. But $y$ goes from an even power in
  the original problem to an odd power in the result, so absolute value is needed:
  \[\frac{x}{\abs{y}^5}\]
  
\end{enumerate}

\end{document}
