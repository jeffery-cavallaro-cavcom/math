\documentclass[letterpaper,12pt,fleqn]{article}
\usepackage{matharticle}
\usepackage{siunitx}
\pagestyle{plain}
\newcommand{\D}{\Delta}
\begin{document}

\begin{center}
\Large Math-1003b Homework \#6 Solutions
\end{center}

\vspace{0.5in}

\underline{Reading}

\bigskip

\begin{itemize}
\item Section 8.5
\end{itemize}

\bigskip

\underline{Problems}

\bigskip

The amount of heat energy ($Q$) needed to change the temperature of an
object (without going through a phase change like melting or boiling) is jointly
proportional to the mass of the object ($m$) and the \emph{change} in
temperature ($\D T$).
\begin{enumerate}
\item Write an equation that models this physical phenomenon. Use $c$ for the
  constant of proportionality.
  \[Q=cm(\D T)\]
  
\item The MKS unit for heat energy is the Joule (J). The constant of
  proportionality is specific to the substance being heated and is referred to
  as the \emph{specific heat} of the substance. If $Q$ is measured in Joules
  ($J$), $m$ is measured in grams ($g$), and temperature is measured in Kelvin
  (K), what are the units of $c$?

  In terms of units, we want:
  \[J=\left(\frac{?}{?}\right)gK\]
  So we want the grams and kelvin to cancel out, and joules to be introduced, so the
  units for $c$ should be $\frac{J}{gK}$.
  
\item In the lab, it is found that $\SI{41790}{J}$ of heat energy raises the
  temperature of $\SI{1}{L}$ of water by $\SI{10}{K}$. What is the specific heat
  of water? ($\SI{1}{L}$ of water$=\SI{1000}{g}$)

  Note that $\SI{1}{L}$ of water has a mass of $\SI{1000}{g}$, so:
  \begin{eqnarray*}
    \SI{41790}{J} &=& c(1000g)(10K) \\
    c &=& \frac{\SI{41790}{J}}{(1000g)(10K)} \\
    c &=& \SI{4.1790}{\frac{J}{gK}}
  \end{eqnarray*}
  
\item How much energy (in Joules) is required to raise the temperature of
  $\SI{5}{L}$ of water by $\SI{5}{K}$?

  Since we now know $c$, we have an equation that we can use to calculate the needed
  heat energy for different masses and temperature changes:
  \[Q=4.1790m(\D T)\]
  For $\SI{5}{L}=\SI{5000}{g}$ of water and a temperature change of $\SI{5}{K}$:
  \[Q=4.1790(5000)(5)=\SI{104475}{J}\]
\end{enumerate}

\end{document}
