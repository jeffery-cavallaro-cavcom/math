\documentclass[letterpaper,12pt,fleqn]{article}
\usepackage{matharticle}
\usepackage{url}
\pagestyle{plain}
\begin{document}

\begin{center}
\emph{San Jos\'{e} State University}

\Large{Math-1003b

  Intensive Learning Mathematics II}\normalsize

\large{Spring-2017}\normalsize

Section 01: MTWR 8:00am--8:50am

Section 02: MTWR 9:00am--9:50am

Duncan Hall 303
\end{center}

\vspace{0.5in}

\begin{description}

\item[Instructor:] Jeffery Cavallaro
  (\url{jeffery.cavallaro@sjsu.edu})

\item[Office:] Duncan Hall 209 (the TA room)

\item[Office Hours:] MW 2:00-3:30pm (or by appointment)

\item[Text:] \emph{Beginning and Intermediate Algebra},
  Miller, O'Neill, and Hyde, \textbf{4th edition}. Hardcopy and/or ebook is fine
  (your preference); however, we will be using the book in class, so if you
  opt for the ebook only, make sure that you have a device on which you can read
  it in class.

\item[Web:] We will use both canvas and ConnectMath. All class communications,
  including written homework assignments and grades, are via canvas
  (\url{sjsu.instructure.com}). ConnectMath (\url{connectmath.com}) will be used
  for a portion of the homework (see below). Once you are registered for the
  course you should be able to see the course listed on your canvas account.
  Each student must purchase a ConnectMath license. The necessary ConnectMath
  class code is posted in a canvas announcement and will be announced on the
  first day of class. Once you register your license, you will need this class
  code to access the class.

\item[Calculator:] You should have a scientific calculator, although calculator use in
  this course is minimal (and generally discouraged). \emph{No other graphing
    calculators, cell phones, tablets, or computers are allowed in lieu of a scientific
    calculator}.

\item[Learning Objectives.] This is the second course in a two semester sequence
  of courses designed to review topics from elementary and intermediate algebra.
  This semester, we will focus on rational expressions and equations, relations
  and functions, inequalities, radicals and rational exponents, and quadratic
  equations. This corresponds to chapters 7--11 in the text.

\item[Prerequisites.] A grade of CR in Math-1003a.
\newpage
\item[Attendance:] I will not take attendance after the first week; however, it
  is important that you come (on time) to every class. The book has more
  information than we could possibly cover, so I will highlight in class
  what is important. Bring your book and calculator to every class meeting. If
  you miss a class, it is your responsibility to talk to your peers and find out
  what you missed.

\item[Time:] You will need to spend a \emph{minimum} of 10 hours per week outside
  of class doing homework and studying. This class is \emph{very} intensive and
  will require disciplined study habits. Please, please, please do \emph{not}
  register for 16 units and/or commit to a 20+ hours per week job; if you do then
  your chances of passing this class drop exponentially!

\item[Holidays.] Class will not meet during Spring break (3/27-3/30).

\item[Reading:] Reading from the textbook will be assigned each Thursday for the
  material to be covered in the coming week. Please read everything, not
  just the stuff in the boxes, prior to lecture. Make sure that you can work
  all of the example problems prior to attempting any of the homework problems.

\item[Web Homework:] The web-based homework will be submitted via ConnectMath.
  ConnectMath requires that you format your answers with math symbols
  using their answer tool. Don't get frustrated! It may take a couple times for
  you to get the hang of it; it will get easier the more you use it. The
  problems assigned on ConnectMath are problems from the book; however, the
  software may change some of the values involved. ConnectMath homework for each
  chapter will be due at midnight prior to the corresponding chapter exam.
  There are no extensions, so please do not fall behind.

\item[Written Homework:] In addition to the web-based homework, you will be
  required to turn in a small set of written homework problems that I will
  assign approximately each week via canvas. Whereas the web-based problems are
  typically based on single concepts, the written homework will combine
  concepts and will need a little more thought. Homework will be assigned each
  week on Monday and is due on the following Monday at the start of class. Late
  homework will not be accepted; however, I will only count your top ten
  homework scores. See \emph{Homework Rules} for more information.

\item[Exams:] There will be four exams in addition to the final exam. The
  \emph{tentative} exam schedule is as follows:

  \bigskip

  \begin{tabular}{lll}
    Exam 1 & Chapters 7 and 8 & Thursday, 3/2 \\
    Exam 2 & Chapter 9 & Thursday, 3/23 \\
    Exam 3 & Chapter 10 & Thursday, 4/20 \\
    Exam 4 & Chapter 11 & Thursday, 5/11 \\
  \end{tabular}
  
  \bigskip

  Prior to an exam, I will post an announcement on canvas telling you exactly what to
  expect on the exam. All exams are closed book and notes. A calculator (as described
  above) is allowed; however, any answers without supporting work or containing decimal
  values not explicitly asked for receive zero credit. Instead of a note card, I will
  provide helpful formulas on the test either by listing them or asking you to identify
  them.

\item[Final.] The final exam is cumulative and is scheduled for
  \textbf{Saturday, 5/20, 9:45am--noon}.  All sections of Math-1003b are taking
  the same final at the same time, and the exam \emph{must} be taken at that time.
  Any and all excuses must be discussed directly with the math office (MH-308).
  Travel arrangements are not a valid excuse, so don't plan to leave town prior
  to noon on 5/20.  The final exam follows the same rules as the other exams.

\item[Grading.]  Your semester grade is determined as follows:

  \bigskip

  \begin{minipage}{3in}
    \begin{tabular}{|c|c|}
      \hline
      ConnectMath Homework & 25\% \\
      \hline
      Written Homework & 25\% \\
      \hline
      Chapter Exams & 25\% \\
      \hline
      Final Exam & 25\% \\
      \hline
    \end{tabular}
  \end{minipage}
  \begin{minipage}{3in}
    \begin{tabular}{|c|c|}
      \hline
      CR (pass) & 70--100 \\
      \hline
      NC (fail) & 0--69 \\
      \hline
    \end{tabular}
  \end{minipage}

  \bigskip
  
\item[Credit:] This is a pass/fail class. It does not count toward your degree;
  however, you \emph{must} pass this class in order to move on and take a math
  course that does fulfill the Area B4 GE requirement (Math 8, 10, or 19).

\item[Tutoring:] Peer tutoring is available to all SJSU students, free of
    charge, from the PeerConnections center. See
    \url{http://peerconnections.sjsu.edu} for more information.

\item[Academic integrity:] Your commitment to learning (as shown by your
    enrollment at SJSU) and SJSU's Academic Integrity Policy require you to be
    honest in all of your academic course work.  Faculty are required to report
    all infractions to the Office of Student Conduct and Ethical Development.
    See \url{http://www.sjsu.edu/studentconduct} for more information.

\item[Disabilities:] If you need course adaptations or accommodations due to a
    disability, or if you need special arrangements in case the building must
    be evacuated, please make an appointment with me as soon as possible. All
    students with disabilities must register with the Accessible Education
    Center (AEC) to establish a record of their disability. See
    \url{http://www.sjsu.edu/aec} for more information.

\end{description}

\end{document}
