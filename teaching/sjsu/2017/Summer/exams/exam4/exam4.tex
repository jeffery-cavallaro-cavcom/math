\documentclass[letterpaper,12pt,fleqn]{article}
\usepackage{matharticle}
\usepackage{siunitx}
\pagestyle{plain}
\begin{document}

\begin{center}
\Large Math-8 Exam \#4
\end{center}

\vspace{0.5in}

Name: \rule{4in}{1pt}

\vspace{0.5in}

This is a take-home exam. It is due on Wednesday morning, 05 July, at 8:00am
(start of class). It is open book and you may use your class notes, lab sheets,
and a graphing calculator. No other printed material, electronics, or
web-searching is allowed. You must do this exam by yourself, without any help
from or discussion with anyone else. Show all work; there is no credit for
guessed answers. All answers should be in exact values, unless you are
specifically asked for an approximate value.

\vspace{0.5in}

\begin{enumerate}
\item Let $f(x)=x^2$ and $g(x)=x^2(x-1)$. Evaluate each of the following,
  simplify where appropriate, and state the domain of each simplified
  expression:
  \begin{enumerate}
  \item $(f+g)(x)$

    \vspace{2in}
    
  \item $(fg)(x)$

    \newpage
    
  \item $\left(\frac{f}{g}\right)(x)$

    \vspace{2in}
    
  \item $(f\circ g)(x)$

    \vspace{2in}
    
  \item $(g\circ f)(x)$

    \vspace{2in}
    
  \end{enumerate}

\item Consider the following function:
  \[h(x)=\sqrt{x-1}+(x-1)^2+2\]
  Find an $f(x)$ and a $g(x)$ such that $h(x)=(f\circ g)(x)$. Neither of your
  functions is allowed to be just $x$.

  \newpage

\item The freezing point of water is $0^{\circ}$C, which is $32^{\circ}$F. The
  boiling point of water is $100^{\circ}$C, which is $212^{\circ}$F. Let $y=$
  degrees Celsius and $x=$ degrees Fahrenheit:
  \begin{enumerate}
  \item Derive the linear function $y=f(x)$ to convert from Fahrenheit to
    Celsius.

    \vspace{2.5in}
    
  \item Derive the linear function $x=g(y)$ to convert from Celsius to
    Fahrenheit.

    \vspace{2.5in}
    
  \item Prove that $f$ and $g$ are inverse functions.
  \end{enumerate}

  \newpage

\item Evaluate the difference quotient for $f(x)=x^3$. Your answer should be
  simplified appropriately.

  \newpage

\item Consider the parabolic function:
  \[y=-x^2+3x-5\]
  \begin{enumerate}
  \item Convert the general form to standard form by completing the square.

    \vspace{3in}
    
  \item What are the coordinates of the vertex?

    \vspace{1in}
    
  \item What are the $x$-intercepts (if any)?

    \vspace{2in}

  \item What are the $y$-intercepts (if any)?

    \newpage

  \item Sketch the graph. Make sure that you label all important points.
  \end{enumerate}

  \newpage

\item Consider the following graph of some polynomial function whose equation
  is unknown but is known to pass through the labeled points:

  \begin{tikzpicture}[scale=5]
    \draw [<->] (-1,0) -- (1,0);
    \draw [<->] (0,-1) -- (0,1);
    \draw [smooth,domain=-1:1] plot({\x},{(\x+1/2)*(\x+1/4)*(\x-1/2)});
    \node [draw,circle,fill=black,scale=0.5] at (-1/2,0) {};
    \node [below right] at (-1/2,0) {$(-1,0)$};
    \node [draw,circle,fill=black,scale=0.5] at (3/4,5/16) {};
    \node [below right] at (3/4,5/16) {$(2,3)$};
  \end{tikzpicture}

  \begin{enumerate}
  \item What is the remainder when $f(x)$ is divided by $x-2$?

    \vspace{1in}
    
  \item Explain why you know that $f(x)$ is divisible by $x+1$.

    \newpage

  \end{enumerate}

\item Divide $x^4-2x^2+x-3$ by $x^2+1$ using long division and state the
  answer in division-algorithm form.

  \newpage

\item Consider the polynomial function:
  \[y=x^7-9x^5-4x^4+12x^3\]
  Factor completely. For full credit you must show how you obtained candidate
  roots, how you selected the actual roots, and then how you decomposed the
  function into linear factors using long (or synthetic) division.

  \newpage

\item Sketch the graph for the function in (8). You must label all $x$ and $y$
  intercepts and indicate the behavior at each zero using either a sign table
  or a list of multiplicity decisions. Be sure to show the correct shape at
  each zero.

  \vspace{5in}

\item Using a calculator, identify all minima and maxima on the graph in
  (9). There are four of them.
  
\end{enumerate}

\end{document}
