\documentclass[letterpaper,12pt,fleqn]{article}
\usepackage{matharticle}
\usepackage{siunitx}
\pagestyle{plain}
\begin{document}

\begin{center}
\Large Math-8 Practice Exam \#3
\end{center}

\vspace{0.5in}

\newcommand{\fillin}{\rule[-10pt]{3in}{1pt}}
\newcommand{\sfillin}{\rule[-10pt]{0.5in}{1pt}}

\begin{enumerate}
\item Identify each of the following formulas. Be very specific.
  Differentiate between general and standard forms and call out important
  points (like the center of a circle).

\vspace{0.25in}

\begin{tabular}{cc}
$x^2+y^2+Dx+Ey+F=0$ & \fillin \\
\\
$(x-h)^2+(y-k)^2=r^2$ & \fillin \\
\\
$\left(\frac{x_1+x_2}{2},\frac{y_1+y_2}{2}\right)$ & \fillin \\
\\
$d=[(x_1-x_2)^2+(y_1-y_2)^2]^{1/2}$ & \fillin \\
\\
$m=\frac{y_1-y_2}{x_1-x_2}$ & \fillin \\
\\
$y-y_1=m(x-x_1)$ & \fillin \\
\\
$y=mx+b$ & \fillin \\
\\
$m_1=m_2$ & \fillin \\
\\
$m_1m_2=-1$ & \fillin \\
\end{tabular}

\newpage

\item Consider the following equation of a circle:
\[(x+1)^2+(y-5)^2=16\]
What are the coordinates of the center and the length of the radius?

\vspace{1in}

\item Consider the following equation of a circle:
  \[x^2+y^2-6x=0\]
  Convert the equation to standard form and determine the center and radius.

\vspace{2in}

\item What is the equation of the line between the two points, in
  slope-intercept form?

\vspace{2in}

\newpage

\item What is the equation, in slope-intercept form, of the line that passes
  threw the midpoint of the two centers and is perpendicular to the line
  joining the two centers?

\newpage

\item Identify each of the following standard functions:

\vspace{0.25in}

\begin{tabular}{cc}
\begin{tikzpicture}
\draw [<->] (-2,0) -- (2,0);
\draw [<->] (0,-2) -- (0,2);
\draw [domain=-1.4:1.4] plot (\x,{(\x)^2});
\node at (-1,-3) {$y=$};
\end{tikzpicture} \hspace{1in} &
\begin{tikzpicture}
\draw [<->] (-2,0) -- (2,0);
\draw [<->] (0,-2) -- (0,2);
\draw [domain=-1.25:1.25] plot (\x,{(\x)^3});
\node at (-1,-3) {$y=$};
\end{tikzpicture} \\
\end{tabular}

\begin{tabular}{cc}
\begin{tikzpicture}
\draw [<->] (-2,0) -- (2,0);
\draw [<->] (0,-2) -- (0,2);
\draw [domain=0:2] plot (\x,{sqrt(\x)});
\node at (-1,-3) {$y=$};
\end{tikzpicture} \hspace{1in} &
\begin{tikzpicture}
\draw [<->] (-2,0) -- (2,0);
\draw [<->] (0,-2) -- (0,2);
\draw [domain=0:2] plot (\x,{(\x)^(1/3)});
\draw [domain=-2:0] plot (\x,{-(-\x)^(1/3)});
\node at (-1,-3) {$y=$};
\end{tikzpicture} \\
\end{tabular}

\begin{tabular}{cc}
\begin{tikzpicture}
\draw [<->] (-2,0) -- (2,0);
\draw [<->] (0,-2) -- (0,2);
\draw [domain=0.5:1.9] plot (\x,{1/\x});
\draw [domain=-1.9:-0.5] plot (\x,{1/\x});
\node at (-1,-3) {$y=$};
\end{tikzpicture} \hspace{1in} &
\begin{tikzpicture}
\draw [<->] (-2,0) -- (2,0);
\draw [<->] (0,-2) -- (0,2);
\draw [domain=-2:2] plot (\x,{abs(\x)});
\node at (-1,-3) {$y=$};
\end{tikzpicture} \\
\end{tabular}

\newpage

\item Consider the function:
\[f(x)=-\abs{x-2}+1\]
Identify the initial standard function and then the three transformations in the
  order that they should be applied.
\begin{enumerate}
\item
\item
\item
\item
\end{enumerate}

\item For the function in (7):
  \begin{enumerate}
  \item Determine the x-intercept(s) (if any).

    \vspace{2in}

  \item Determine the y-intercept(s) (if any).

    \vspace{1in}

  \item Sketch the graph. Be sure to label all key points for full credit!
  \end{enumerate}

\newpage
\item Based on your graph in (8):
  \begin{enumerate}
  \item Where is the function increasing, if anywhere (in interval notation)?

    \vspace{1in}
    
  \item Where is the function decreasing, if anywhere (in interval notation)?

    \vspace{1in}

  \item Identify any minima.

    \vspace{1in}
    
  \item Identify any maxima.

    \vspace{1in}

  \end{enumerate}

\item Based on your graph in (8):
  \begin{enumerate}
  \item State the domain, in interval notation.

    \vspace{1in}
    
  \item State the range, in interval notation.
  \end{enumerate}
\end{enumerate}

\end{document}
