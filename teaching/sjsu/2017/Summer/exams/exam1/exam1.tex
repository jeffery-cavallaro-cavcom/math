\documentclass[letterpaper,12pt,fleqn]{article}
\usepackage{matharticle}
\pagestyle{plain}
\begin{document}

\begin{center}
\Large Math-8 Exam \#1
\end{center}

\vspace{0.5in}

Name: \rule{4in}{1pt}

\vspace{0.5in}

This exam is closed book and notes. You may use a calculator; however, no other
electronics are allowed. Show all work; there is no credit for guessed
answers. All answers should be in exact values, unless you are specifically
asked for an approximate value.

\vspace{0.5in}

\newcommand{\fillin}{\rule[-10pt]{2in}{1pt}}
\newcommand{\sfillin}{\rule[-10pt]{0.5in}{1pt}}

\begin{enumerate}
\item Identify each subset of the real numbers and give an example of an
  element from each set.

  \bigskip

\begin{tabular}{ccc}
\textbf{subset} & \textbf{name} & \textbf{example} \\
\\
$\N$ & \fillin & \sfillin \\
\\
$\Z$ & \fillin & \sfillin \\
\\
$\Q$ & \fillin & \sfillin \\
\\
$\R$ & \fillin & \sfillin \\
\end{tabular}

\bigskip

\item Mark each of the following statement as either (T)rue or (F)alse. If
  false, provide a counterexample to show why the statement is false.
  \begin{enumerate}
  \item $\N\subseteq\Q$
    \vspace{0.25in}
  \item Every integer is also a rational number.
    \vspace{0.25in}
  \item Every rational number is an integer.
    \vspace{0.25in}
  \item Every rational number is a fraction.
    \vspace{0.25in}
  \item Every fraction is a rational number.
    \vspace{0.25in}
  \end{enumerate}

\item Give an example of an integer that is not a natural number.

  \vspace{0.25in}

\item Give an example of a real number that is not a rational number.

  \vspace{0.25in}
  
\item List the three possible forms of a rational number:
  \begin{enumerate}
  \item \fillin
  \item \fillin
  \item \fillin
  \end{enumerate}

  \bigskip

\item Convert to fractional form. You do not need to reduce:
  \begin{enumerate}
  \item $12.345$

    \vspace{1in}
    
  \item $12.3\overline{45}$

  \end{enumerate}

  \newpage

\item Graph the following two sets on a number line. Don't bother with scale;
  relative positioning of the endpoints is OK:
  \begin{enumerate}
  \item $\{x\in\R\mid -1\le x\le 3\}$

    \vspace{2in}
    
  \item $\{x\in\Z\mid -1\le x\le 3\}$

  \end{enumerate}

  \newpage

\item Perform the following calculations:
  \begin{enumerate}
  \item Determine the prime factorization for $60$.

    \vspace{1in}
    
  \item Determine the prime factorization for $126$.

    \vspace{1in}
    
  \item Calculate using the LCM of $60$ and $126$:
    \[\frac{3}{126}-\frac{5}{60}\]

    \vspace{2in}
    
  \item Reduce using the GCD of $60$ and $126$:
    \[\frac{126}{60}\]
  \end{enumerate}

  \newpage

\item Solve for $x$:
  \[x-3(2x+3)=8-5x\]

  \vspace{3in}

\item Solve for $x$:
  \[\frac{1}{x-3}+\frac{3}{x+3}=\frac{10}{x^2-9}\]

\end{enumerate}

\end{document}
