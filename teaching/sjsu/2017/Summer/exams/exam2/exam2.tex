\documentclass[letterpaper,12pt,fleqn]{article}
\usepackage{matharticle}
\usepackage{siunitx}
\pagestyle{plain}
\begin{document}

\begin{center}
\Large Math-8 Exam \#2
\end{center}

\vspace{0.5in}

Name: \rule{4in}{1pt}

\vspace{0.5in}

This exam is closed book and notes. You may use a calculator; however, no other
electronics are allowed. Show all work; there is no credit for guessed
answers. All answers should be in exact values, unless you are specifically
asked for an approximate value.

\vspace{0.5in}

\newcommand{\fillin}{\rule[-10pt]{3in}{1pt}}
\newcommand{\sfillin}{\rule[-10pt]{0.5in}{1pt}}

\begin{enumerate}
\item Identify the following:

  \begin{tabular}{ll}
    \\
    $ax^2+bx+c = 0$ & \fillin \\
    \\
    $x=\frac{-b\pm\sqrt{b^2-4ac}}{2a}$ & \fillin \\
    \\
    $D=b^2-4ac$ & \fillin
  \end{tabular}

  \bigskip

\item You own a candy and nut store. Peanuts sell for \$4.00 per lb and cashews sell
  for \$7.50 per lb. Unfortunately, the more expensive cashews are not selling very well,
  so you decide to make a peanut/cashew mix. You mix $\SI{10}{lb}$ of peanuts with
  $\SI{10}{lb}$ of cashews. How much should you charge per pound for the mix?

  \newpage

\item Solve for $x$ by completing the square:
  \[2x^2-5x-3=0\]

  \vspace{3in}

\item Solve the same equation using the quadratic formula.

  \vspace{3in}

\item Using the results from the previous two problems, state a factorization for the
  original quadratic equation. You must use the previous results for full credit.

  \newpage

\item Solve the following:
  \begin{enumerate}
  \item Solve for $x$. State you answer both graphically and using setbuilder notation:
    \[\abs{2x-5}-2=4\]

    \vspace{2in}
    
  \item Solve for $x$. State you answer graphically, using interval notation, and
    using interval notation.
    \[\abs{2x-5}-2<4\]
    
    \vspace{2in}
    
  \item Solve for $x$. State you answer graphically, using interval notation, and
    using interval notation.
    \[\abs{2x-5}-2\ge4\]
  \end{enumerate}

  \newpage
  
\item We discussed in class the four possibilities of a discriminant and how they
  predict the number and type of solutions to the corresponding quadratic equation.
  List the four cases, including discriminant value, number of solutions, and type of
  solutions.

  \bigskip

  \begin{tabular}{ccc}
    D & COUNT & TYPE \\
    \\
    \sfillin & \sfillin & \fillin \\
    \\
    \sfillin & \sfillin & \fillin \\
    \\
    \sfillin & \sfillin & \fillin \\
    \\
    \sfillin & \sfillin & \fillin
  \end{tabular}

  \vspace{0.5in}

\item Solve for $x$;
  \[x^4+7x^2-18=0\]

  \newpage

\item Solve for $x$:
  \[(x-5)^{\frac{2}{3}}-16=0\]

  \newpage

\item Solve for $x$, stating your answer in interval notation. A full credit answer
  includes the work that shows a graphical representation and a sign table.
  \[\frac{x^2(x+2)(1-x)}{(3-x)^2(4+x)}\le0\]
\end{enumerate}

\end{document}
