\documentclass[letterpaper,12pt,fleqn]{article}
\usepackage{matharticle}
\usepackage{url}
\pagestyle{plain}
\begin{document}

\begin{center}
\emph{San Jos\'{e} State University}

\Large{Math-08 (College Algebra)}\normalsize

\large{Summer-2017}\normalsize

MTWRF 8:00am--9:50am \\

Macquarrie Hall 324
\end{center}

\vspace{0.5in}

\begin{description}

\item[Instructor:] Jeffery Cavallaro (\url{jeffery.cavallaro@sjsu.edu})

\item[Office:] Duncan Hall 209 (the TA room)

\item[Office Hours:] MTRF immediately after class (as needed).

\item[Texts:] \emph{College Algebra and Calculus: An Applied Approach},
  Larson and Hodgkins, \textbf{2nd edition}.  Make sure that you have the second edition
  and not the first. This is the same book that is used for Math-71. Hardcopy or ebook is
  OK. We will cover Chapters 1--5 in class. You are responsible for the material in
  Chapter 0.

\item[Web:] We will use both canvas and webassign. All class communications,
  including reading assignments, homework assignments, helpful resource
  documents, and grades, are via canvas (\url{sjsu.instructure.com}). Webassign
  (\url{webassign.com}) will be used for a portion of the homework (see below).
  Once you are registered for the course you should be able to see the course
  listed on your canvas account. Each student must purchase a webassign
  license. Note that the ``enhanced'' license is not required; however, you
  may find the extra teaching materials that it provides helpful. The
  webassign class code will be distributed on the first day of class and via
  a canvas announcement. Once you register your license, you will need this
  class code to access the class.

\item[Calculator:] You are required to have a TI-83 or 84 graphing calculator.
  If you are buying a new one, I suggest the TI-84 Plus CE. I have \emph{no}
  problem with you checking your homework and exam answers using your calculator --- in
  fact, I encourage it; however, answers with no supporting work will receive zero credit.
  \emph{No other scientific calculators, cell phones, tablets, or computers
    are allowed in lieu of a TI-83/4 calculator!}

\item[Learning Objectives.] This is a preparatory class for Math-71 Business Calculus. In
  this course you will master basic algebra skills, understand and apply fundamental
  ideas about functions, and study some specific types of functions (e.g., polynomials,
  exponentials, logarithms). You will use mathematical methods to solve quantitative
  (word) problems and arrive at conclusions based on numerical and graphical data.  These
  learning objectives, as well as the minimum 500 word writing requirement (written
  homework and exams), satisfy the Area B4 (Mathematical Concepts) GE requirement.
\newpage
\item[Attendance:] I will not take attendance after the first week; however, it
  is vitally important that you come (on time) to every class. The book has
  more information than we could possibly cover, so I will highlight in class
  what is important. I will also enhance certain subjects that I feel are
  important for your calculus preparation. Bring your book (or ebook reader) and
  calculator to every class meeting. If you miss a class, it is your responsibility to
  talk to your peers and find out what you missed.

\item[Time:] You will probably need to spend a \emph{minimum} of 20 hours per
  week outside of class doing homework and studying. This class is \emph{very}
  intensive and will require disciplined study habits. Please, please, please
  do \emph{not} register for another first session class and/or commit to a 20+ hours
  per week job; if you do then your chances of passing this class drop exponentially!

\item[Holidays:] Class will not meet on Tue, 7/4 (Independence Day). Note that we
  \emph{will} meet on Mon, 7/3.

\item[Reading:] Reading from the textbook will be assigned the day before the material
  will be discussed in class. Please read everything, not just the stuff in the boxes,
  prior to lecture. Make sure that you can work all of the example problems prior to
  attempting any of the homework problems.

\item[WebAssign Homework:] The web-based homework will be submitted via WebAssign.
  Web\-Assign requires that you format your answers with math symbols using an answer
  tool. Don't get frustrated! It may take a couple times for you to get the hang of it;
  it will get easier the more you use it. The problems assigned on WebAssign are even
  problems from the book. WebAssign homework will be assigned on Monday and will be due
  on the following Sunday at 11:59pm (midnight). There are \emph{no} extensions, so
  please do not fall behind.

\item[Written Homework:] Each week I will pass out sheets that will help you review some
  basic concepts of the real numbers, the number line, algebraic expressions, and
  algebraic equations. This will be a good review and enhancement of the material in
  Chapter 0. You will be responsible for collecting the sheets in order in a notebook.
  I will review your notebook for completeness while you are taking your exams
  (see below).

\item[Exams:] There will be an exam every Monday (except the first day) during the second
  half of the class that will cover the material from the previous week (4 total). All
  exams are closed book and notes. A calculator (as described above) is allowed; however,
  as noted above, any answers without supporting work (i.e., guesses or copying an answer
  directly from your calculator) receive zero credit. Instead of a note card, I will
  provide all needed formulas in the form of ``fill in the blank'' questions.

\item[Final:] The final exam is cumulative and is scheduled for the last day of class:
  Fri, July 7. The final \emph{must} be taken at that time. The final exam follows the
  same rules as the exams.
\newpage
\item[Grading:]  Your semester grade is determined as follows:

  \bigskip

  \begin{minipage}{3in}
    \begin{tabular}{|c|c|}
      \hline
      WebAssign Homework & 25\% \\
      \hline
      Written Homework & 25\% \\
      \hline
      Weekly Exams & 25\% \\
      \hline
      Final Exam & 25\% \\
      \hline
    \end{tabular}
  \end{minipage}
  \begin{minipage}{3in}
    \begin{tabular}{|c|c|}
      \hline
      A+ & 97--100 \\
      \hline
      A & 93--96 \\
      \hline
      A- & 90--92 \\
      \hline
      B+ & 87--89 \\
      \hline
      B & 83--86 \\
      \hline
      B- & 80--82 \\
      \hline
      C+ & 77--79 \\
      \hline
      C & 73--76 \\
      \hline
      C- & 70--72 \\
      \hline
      D+ & 67--69 \\
      \hline
      D & 63--67 \\
      \hline
      D- & 60--62 \\
      \hline
      F & 0-59 \\
      \hline
    \end{tabular}
  \end{minipage}
  
\item[Credit:] A grade of C or better meets the \emph{Area B4: Mathematical
    Concepts} GE requirement. A grade of C- or better is required for placement
    into Math 71.

\item[Tutoring:] Peer tutoring is available to all SJSU students, free of
    charge, from the PeerConnections center. See
    \url{http://peerconnections.sjsu.edu} for more information.

\item[Academic integrity:] Your commitment to learning (as shown by your
    enrollment at SJSU) and SJSU's Academic Integrity Policy require you to be
    honest in all of your academic course work.  Faculty are required to report
    all infractions to the Office of Student Conduct and Ethical Development.
    See \url{http://www.sjsu.edu/studentconduct} for more information.

\item[Disabilities:] If you need course adaptations or accommodations due to a
    disability, or if you need special arrangements in case the building must
    be evacuated, please make an appointment with me as soon as possible. All
    students with disabilities must register with the Accessible Education
    Center (AEC) to establish a record of their disability. See
    \url{http://www.sjsu.edu/aec} for more information.

\end{description}

\end{document}
