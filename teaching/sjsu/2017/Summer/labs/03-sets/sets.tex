\documentclass[letterpaper,12pt,fleqn]{article}
\usepackage{matharticle}
\pagestyle{empty}
\begin{document}
\section*{Lab 3: Sets}

In mathematics, we use the concept of a \emph{set} to describe groups of
related things (like the natural numbers) and solutions to certain types of
equations. We will encounter this throughout the course, so it is good to get
a basic understanding of sets now at the beginning. We will use the natural
numbers for our examples, so review the notes on natural numbers if you have
forgotten what they are.

An important part of any discipline like mathematics are the definitions that
provide the vocabulary and understanding of the subject matter. These
definitions tend to be rather formal and hard to understand at first. The
definition for a set that we will use is ugly. Read it through, taking note of
the emphasized words, and then we will parse it bit by bit with examples:

\begin{definition}[Set]
  A set is an \emph{unambiguous}, \emph{distinct}, and \emph{unordered}
  collection of members called elements, where the elements are selected from
  an all-inclusive set called a universe.
\end{definition}

First of all, we need to admit that this is not a very good definition. We
haven't explained what we mean by a ``collection'' or to be a ``member'' of a
collection. Even worse, we used the word ``set'' in the definition when
describing the universe. This is all due to the limitations of language. You can
only break down a definition of something so far. At some point, you need to
rely on the reader's intuition or experience relating to certain concepts.
Hence, we are relying on your intuition on what it means to be a ``collection
of members.''

So let's consider our first example of a set: the natural numbers. The easiest
way of representing such a set is by \emph{description}:
\[\N=\mbox{the set of natural numbers}\]
We typically use a capital letter to represent a set. Some of the well-known
sets like the natural numbers use a more blocky form of letter --- note the
extra stroke in the $\N$. But you can just write $N$ if you like.

Now, back to the definition and the three emphasized adjectives:
\begin{enumerate}
\item Unambiguous

  This means that given a thing, you can objectively determine whether or not
  it is in the set; there is no room for subjectivity. For example, we know
  that the number '$1$' is definitely an element of the set of natural numbers.
  The syntax for this is as follows:
  \[1\in\N\]
  Note that the little curvy 'e' stands for ``is in.'' On the other hand, we
  know that the number '$0$' is definitely not an element of the set of
  natural numbers. The syntax for this puts a slash through the curvy 'e':
  \[0\notin\N\]
  The curvy 'e' with a slash stands for ``is not in.''

  Here is an example of something that is ambiguous:
  \[A=\mbox{the set of large numbers}\]
  Can you see why this is not a well-formed set? Is $100\in A$? Is
  $1,000,000\in A$? Note that the choice is subjective and hence ambiguous.

\item Distinct

  This means that each element in the set appears (i.e., is selected from the
  universe) once and only once --- there are no duplicates or copies.

\item Unordered

  This means that the order of the elements in the set is not significant
  (doesn't matter). This may seem odd, since you know that the natural numbers
  do have a counting order (e.g., 2 comes after 1); however, this is not
  inherent in the definition of a set.
\end{enumerate}

In the previous example, the set of natural numbers is its own universe. Now,
lets make some smaller sets using the set of natural numbers as the universe.
When the number of elements selected from the universe is small, we can use the
\emph{roster} notation to list all of the elements. For example:
\[A=\{1,2,4,7\}\]
The set $A$ contains four elements. The elements are enclosed in curly-braces
and are separated by commas. Note that this set is unambiguous:
\[1\in A\hspace{4ex}2\in A\hspace{4ex}4\in A\hspace{4ex}7\in A\]
But, for example:
\[3\notin A\hspace{4ex}10\notin A\]
Also note that each element only occurs once. So the set:
\[B=\{1,2,4,7,1\}\]
would be the same set as $A$, since the second, extra $1$ in $B$ is considered
to be identical to the first $1$ in $B$. In fact, we would never represent a
set this way, with duplicate elements.

And finally note that the actual order of the elements of $A$ is unimportant.
Thus:
\[\{1,2,4,7\}\hspace{4ex}\{1,4,2,7\}\hspace{4ex}\{7,4,2,1\}\]
are all valid syntactical representations of $A$. But, since we know that
the natural numbers do have an ordering, we feel more comfortable using the
first form.

Now, let's look at a couple additional techniques that can be employed when
using roster notation. As long as the number of elements in a set is small, we
can use the explicit roster format:
\[A=\{1,2,3,4,5,6,7,8,9,10\}=\mbox{the set of natural numbers from 1 to 10}\]
But how would be represent the first 100 natural numbers? We can do it like
this:
\[B=\{1,2,3,4,5,\ldots,99,100\}\]
Once again, we run into our friend the ellipsis. Once we establish a pattern,
we use the ellipsis to represent the stuff in the middle. We end the set with
one or more of the last elements after the ellipsis.

When a set has a definite number of elements we say that it is \emph{finite}.
So, for example, the set:
\[A=\{2,4,6,8\}\]
is finite because it has 4 elements. Likewise, the set:
\[B=\{1,2,3,4,5,\ldots,50\}\]
is finite because it has 50 elements. But sets need not be finite, they can
be \emph{infinite} - i.e., have infinitely many elements. An example is $\N$,
the set of natural numbers. Here is how $\N$ might be represented using roster
notation:
\[\N=\{1,2,3,4,5,\ldots\}\]
Note that we establish the pattern, use the ellipsis, and leave off an ending
value. This indicates that we are supposed to continue on to infinity. Remember
that we do not write the following:
\[\N=\{1,2,3,4,5,\ldots,\infty\}\hspace{4ex}\mbox{WRONG!}\]
There is an addition set syntax that we will eventually need to know: the
so-called ``set-builder'' notation. We won't need it until later, so we will
defer learning about it until then.

\subsection*{Questions}

\begin{enumerate}
\item Let $A=$ the set of natural numbers between $20$ and $25$, inclusive
  (meaning include the endpoints).
  \begin{enumerate}
  \item Write the roster notation for the set $A$.

    \vspace{0.5in}
    
  \item Fill in either $\in$ or $\notin$ as is appropriate:
    
    $22\hspace{2ex}A\hspace{8ex}-1\hspace{2ex}A\hspace{8ex}26\hspace{2ex}A
    \hspace{8ex}20\hspace{2ex}A\hspace{8ex}10\hspace{2ex}A$
  \end{enumerate}

  \newpage

\item Let $B=$ the set of even natural numbers between $20$ and $30$, exclusive
  (meaning exclude the endpoints).
  \begin{enumerate}
  \item Write the roster notation for the set $B$.

    \vspace{0.5in}
    
  \item Fill in either $\in$ or $\notin$ as is appropriate:
    
    $24\hspace{2ex}B\hspace{8ex}28\hspace{2ex}B\hspace{8ex}30\hspace{2ex}B
    \hspace{8ex}18\hspace{2ex}B\hspace{8ex}0\hspace{2ex}B$
  \end{enumerate}

  \bigskip

\item Let $C=$ the set natural numbers between $100$ and $500$, inclusive.
  \begin{enumerate}
  \item Write the roster notation for the set $A$ (hint: do not write all
    $401$ numbers!).

    \vspace{1in}
    
  \item Fill in either $\in$ or $\notin$ as is appropriate:
    
    $121\hspace{2ex}C\hspace{8ex}500\hspace{2ex}C\hspace{8ex}499\hspace{2ex}C
    \hspace{8ex}99\hspace{2ex}C\hspace{8ex}400\hspace{2ex}C$
  \end{enumerate}

  \bigskip
  
\item Here are some other examples of infinite sets. Can you guess what they
  are:
  \begin{enumerate}
  \item $E=\{2,4,6,8,\ldots\}$

    \vspace{0.5in}
    
  \item $O=\{1,3,5,7,\ldots\}$

    \vspace{0.5in}
    
  \item $P=\{2,3,5,7,11,13,17,\ldots\}$

    \vspace{0.5in}
  \end{enumerate}

\item Let $D=$ the set of odd natural numbers greater than or equal to $100$.
  \begin{enumerate}
  \item Write the roster notation for the set $D$ (Hint: Use an ellipsis).

    \vspace{0.5in}
    
  \item Fill in either $\in$ or $\notin$ as is appropriate:
    
    $109\hspace{2ex}D\hspace{8ex}100\hspace{2ex}D\hspace{8ex}200\hspace{2ex}D
    \hspace{8ex}1,000,001\hspace{2ex}D\hspace{8ex}-101\hspace{2ex}D$
  \end{enumerate}

\end{enumerate}

\end{document}
