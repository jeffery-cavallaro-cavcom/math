\documentclass[letterpaper,12pt,fleqn]{article}
\usepackage{matharticle}
\pagestyle{plain}
\begin{document}
\section*{Lab 8: Factoring and Sketching Polynomials}

In class we learned how to factor a polynomial:
\begin{enumerate}
\item Use the rational root test to identify a set of candidate roots/zeros.
\item Use the remainder (factor) theorem to identify actual roots/zeros from
  the set of candidates.
\item Use long division to decompose the polynomial into linear factors.
\end{enumerate}

The examples we did in class all had the leading coefficient $a_n=1$ and the
constant coefficient $a_0\ne0$. Lets examine a case where neither of these two
things are true. Consider the polynomial:
\[y=-2x^5+3x^4+2x^3-3x^2\]
Note that $a_n=-2$ and $a_0=0$.

\textbf{Step one} is to simply divide out a $-1$. Do so now:
\[y=\]

Now you have something of the form $y=-\left(p(x)\right)$, where the leading coefficient
of $p(x)$ is now positive.

\textbf{Step two} is to factor out some power of $x$ so that the resulting polynomial has
a non-zero constant term. Do so now:
\[y=\]

Your answer should now look somthing like: $y=-x^k\left(q(x)\right)$ for some integer
$k$ and some cubic polynomial $q(x)$ with an $a_0\ne0$.

\textbf{Step three} is to apply the rational root test to the resulting $q(x)$ in order
to find candidate roots/zeros.
Let $p$ represent all of the factors of $a_0$:
\[p:\]
Let $q$ represent all of the factors of $a_n$:
\[q:\]
Now determine the set of candidate roots by making all possible combinations of
$\frac{p}{q}$:
\[\frac{p}{q}:\]

\newpage

If you did this correctly then you have some fractional roots, since $a_n\ne1$. One of
these should be $\frac{3}{2}$. Apply the remainder/factor test to verify that this is in
fact a root/zero of the polynomial $q(x)$:
\[q\left(\frac{3}{2}\right)=\]

\vspace{1in}

\textbf{Step four} is to use long division to factor the $\left(x-\frac{3}{2}\right)$ out
of $q(x)$. But this would be a mess due to the fraction, so let's think about an easier
way. If it is true that $\left(x-\frac{3}{2}\right)$ divides $q(x)$, then we would have
the following:
\[y=-x^k\left(x-\frac{3}{2}\right)q_2(x)\]
for some quadratic polynomial $q_2(x)$. Now, if we factor a $\frac{1}{2}$ out of the
linear factor, we have:
\[y=-\frac{1}{2}x^k(2x-3)q_2(x)\]
Make sure that you understand how to factor out the fractional value! Thus, if
$\left(x-\frac{3}{2}\right)$ divides the original $q(x)$ then it must be the case that
$(2x-3)$ also divides the original $q(x)$. Make sure that you understand this! So
instead of doing long division with $\left(x-\frac{3}{2}\right)$, we can do so with the
much easier $(2x-3)$. Do that now:

\newpage

Using the result to rewrite the original polynomial in the form
$y=-\frac{1}{2}x^k(2x-3)q_2(x)$, where $q_2(x)$ is some quadratic polynomial:
\[y=\]
The $q_2(x)$ should be factorable by inspection. Write the final factored polynomial:
\[y=\]
Now we are ready to sketch. Plot the zeros and y-intercept, determine the end behavior,
and use multiplicity to sketch the graph:

\newpage

Your graph should have two local minima and two local maxima. Use your calculator to
determine where these four points occur:
\[min_1=\]
\[min_2=\]
\[max_1=\]
\[max_2=\]
Remember that these are points, so they should be stated in $(x,y)$ form.
\end{document}
