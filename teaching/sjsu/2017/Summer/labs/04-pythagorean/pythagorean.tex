\documentclass[letterpaper,12pt,fleqn]{article}
\usepackage{matharticle}
\pagestyle{plain}
\begin{document}
\section*{Lab 4: Pythagorean Theorem}

The Pythagorean theorem is one of the oldest and best known theorems known to
mankind. It is amazing how many times it pops up in mathematics. We start with
a right triangle (a triangle containing a $90^{\circ}$ angle) and label the
length of the sides as $a$, $b$, and $c$:

\begin{minipage}{\textwidth}
  \centering
  \begin{tikzpicture}
    \node [scale=0.01] (A) at (0,0) {};
    \node [scale=0.01] (B) at (4,0) {};
    \node [scale=0.01] (C) at (4,3) {};
    \draw (A) to node [below] {$b$} (B) to node [right] {$a$} (C)
    to node [above left] {$c$} (A);
  \end{tikzpicture}
\end{minipage}

Note that $c$ is the length of the hypotenuse.

The theorem states the following:
\[a^2+b^2=c^2\]

Let work through a simple, but elegant, proof of the theorem. First, write
down the equation for the area of a square with side $s$:
\[A=\]
Now, write down the equation for a triangle with base $b$ and height $h$:
\[A=\]
Now consider the following diagram of a square within a square:

\begin{minipage}{\textwidth}
  \centering
  \begin{tikzpicture}
    \draw (0,0) to node [below] {$b$} (4,0);
    \draw (4,0) to node [below] {$a$} (7,0);
    \draw (7,0) to node [right] {$b$} (7,4);
    \draw (7,4) to node [right] {$a$} (7,7);
    \draw (7,7) to node [above] {$b$} (3,7);
    \draw (3,7) to node [above] {$a$} (0,7);
    \draw (0,7) to node [left] {$b$} (0,3);
    \draw (0,3) to node [left] {$a$} (0,0);
    \draw (4,0) to node [below right] {$c$} (7,4);
    \draw (7,4) to node [above right] {$c$} (3,7);
    \draw (3,7) to node [above left] {$c$} (0,3);
    \draw (0,3) to node [below left] {$c$} (4,0);
  \end{tikzpicture}
\end{minipage}

Note that there are four congruent right triangles positioned around the inner
square. What is the equation for the area of the inner square?
\[A=\]
What is the equation for the area of each of the outer triangles?
\[A=\]
What is the equation for the area of the outer square?
\[A=\]
Finally, note that the area of the outer square is the sum of the areas of the
inner square and the four outer triangles. Use this fact to derive the
Pythagorean Theorem:

\vspace{2in}

When the lengths of the three sides are integers, we call the triplet a
\emph{Pythagorean Triple}. The most common is $3-4-5$:
\[3^2+4^2=9+16=25=5^2\]
Fill in the missing values in the following table of Pythagorean triples:
\renewcommand{\arraystretch}{2}
\[\begin{array}{|c|c|c|}
\hline
\mathbf{a} & \mathbf{b} & \mathbf{c} \\
\hline
3 & 4 & \\
\hline
5 & & 13 \\
\hline
& 15 & 17 \\
\hline
7 & 24 & \\
\hline
\end{array}\]

\end{document}
