\documentclass[letterpaper,12pt,fleqn]{article}
\usepackage{matharticle}
\pagestyle{plain}
\begin{document}
\section*{Lab 9: Real Exponents}

Consider the exponential expression $a^b$ for $a,b\in\R$. There are three cases
for $b$:
\begin{enumerate}
\item $b\in\Z$ (integer)
\item $b\in\Q$ (rational)
\item $b\in\R-\Q$ (irrational)
\end{enumerate}

We have already studied what the first two cases mean: when $b$ is an integer
then we have $a\cdot a\cdots a$ a total of $b$ times (or the reciprocal when
$b<0$) and when $b$ is a rational number $\frac{p}{q}$ then
$a^b=\sqrt[q]{a^p}$. But what does it mean when $b$ is irrational? For example,
what the heck does something like $2^\pi$ possibly mean?

If you enter $2^\pi$ into your calculator you will get an answer like:
$2^\pi=8.824977827$. Of course, $2^\pi$ is irrational, so this is just an
approximation. But how did the calculator come by this answer? For this, we
turn to our old friend \emph{arbitrarily close}. Remember that $\pi$ is the
result of a sequence of fixed decimal (and therefore rational) approximations
that get arbitrarily close to the exact value of $\pi$:
\[\begin{array}{l}
3 \\
3.1 \\
3.14 \\
3.141 \\
3.1415 \\
3.14159 \\
3.141592 \\
3.1415926 \\
\vdots
\end{array}\]
As such, we can calculate $2^p$ for each approximation of $\pi=p$:
\[\begin{array}{l|l}
\pi & 2^\pi \\
\hline
3 & 8 \\
3.1 & 8.574187700 \\
3.14 & 8.815240927 \\
3.141 & 8.821353305 \\
3.1415 & 8.824411082 \\
3.14159 & 8.824961595 \\
3.141592 & 8.824973829 \\
3.1415926 & 8.824977499
\end{array}\]

Note that as the approximation for $\pi$ gets better and better, the
approximation for $2^\pi$ also gets better. In fact, if you give me any
$\epsilon>0$, no matter how small, I can eventually find an approximation $p$
of $\pi$ such that $2^p$ is within $\epsilon$ of the exact value of $2^\pi$.

Now you do it. Consider the irrational value $\pi^{\sqrt{2}}$. This time, both
the base and exponent are irrational, so you will need to approximate both at
each step. First get a value from your calculator. Be sure to list all of the
decimal digits that your calculator provides:
\[\pi^{\sqrt{2}}=\]
Now get values for $\pi$ and $\sqrt{2}$:
\[\pi=\]
\[\sqrt{2}=\]
Now complete the following table (I have done the first two rows for you):
\[\begin{array}{|l|l|l|}
  \hline
  \pi & \sqrt{2} & \pi^{\sqrt{2}} \\
  \hline
  3 & 1 & 3.000000000 \\
  \hline
  3.1 & 1.4 & 4.874233962 \\
  \hline
  \hspace{16ex} & \hspace{16ex} & \hspace{32ex} \\
  \hline
  & & \\
  \hline
  & & \\
  \hline
  & & \\
  \hline
  & & \\
  \hline
  & & \\
  \hline
  & & \\
  \hline
  & & \\
  \hline
\end{array}\]
Notice how your approximations for $\pi^{\sqrt{2}}$ converge to your
calculator answer.

\end{document}
