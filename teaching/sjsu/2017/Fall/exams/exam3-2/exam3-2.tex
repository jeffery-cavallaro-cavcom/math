\documentclass[letterpaper,12pt,fleqn]{article}
\usepackage{matharticle}
\usepackage{siunitx}
\pagestyle{plain}
\begin{document}

\begin{center}
\Large Math-19 Exam \#3 (2)
\end{center}

\vspace{0.5in}

Name: \rule{4in}{1pt}

\vspace{0.5in}

This exam is closed book and notes. You may use a calculator and both sides of
a $3\times 5$ note card; however, no cell phones or tablets are allowed. Show
all work; there is no credit for guessed answers. All values should be exact
with no decimals unless you are specifically asked for an approximate or
decimal answer.

\vspace{0.5in}

\begin{enumerate}
\item A certain chemical reaction produces product at a rate of $\SI{7}{g/s}$.
  After $\SI{1}{minute}$, $\SI{450}{g}$ of product is produced. Let $p(t)$ be
  the amount of product after $t$ seconds and let $p(0)$ be the amount of
  product at $t=0$.
  \begin{enumerate}
  \item Construct a linear model for this reaction.

    \vspace{1in}
    
  \item Find $p(0)$.
  \end{enumerate}

  \vspace{2in}

\item Consider $y^2=4x$. Answer true or false:
  \begin{enumerate}
  \item This is an example of a relation.
    
    \bigskip

  \item This is an example of a function.
  \end{enumerate}

  \bigskip

  \newpage

\item Calculate the difference quotient for $f(x)=\frac{1}{\sqrt{2}x}$.

  \vspace{3in}

\item Let $h(x)=(x-2)^3-x+2$. Find an $f(x)$ and a $g(x)$ such that
  $h(x)=(f\circ g)(x)$ with the constraint that neither $f(x)=x$ nor
  $g(x)=x$.

  \newpage
  
\item Consider the following graph of a polynomial $p(x)$:

  \begin{tikzpicture}
    \draw (-4,0) -- (4,0);
    \draw (0,-4) -- (0,4);
    \draw [smooth,domain=-2.3:2.3]
    plot ({\x},{-0.5*(\x-2)*(\x-1)*(\x+1)*(\x+2)});
    \node [draw,circle,fill,scale=0.5] at (2,0) {};
    \node [below left] at (2,0) {$(2,0)$};
    \node [draw,circle,fill,scale=0.5] at (-1/2,-45/32) {};
    \node [below left] at (-1/2,-45/32) {$(-\frac{3}{4},-\frac{6}{5})$};
  \end{tikzpicture}

  \begin{enumerate}
  \item Solve for $x$: $p(x)=-\frac{6}{5}$

    \vspace{1in}
    
  \item What is the remainder when the polynomial is divided by
    $\left(x+\frac{3}{4}\right)$?

    \vspace{1in}
    
  \item What is the remainder when the polynomial is divided by $(x-2)$?

    \vspace{1in}
    
  \item List two equivalent statements to the fact that $(2,0)$ is a point
    in the graph of $p(x)$.
    \begin{enumerate}
    \item
    \item
    \end{enumerate}
  \end{enumerate}

  \newpage

\item Let $f(x)=3-2\abs{x+1}$.
  \begin{enumerate}
  \item List the starting function and the three transformations in the order
    that they should be applied:
    \begin{enumerate}
      \item
      \item
      \item
      \item
    \end{enumerate}
  \item What are the $x$ intercepts (if any)?

    \vspace{2in}
    
  \item What are the $y$ intercepts (if any)?

    \vspace{1in}
    
  \item Sketch the graph. Be sure to label \emph{all} key points.

    \bigskip

    \begin{figure}[h]
      \setlength{\leftskip}{1in}
      \begin{tikzpicture}
        \draw (-3.5,0) -- (3.5,0);
        \draw (0,-3.5) -- (0,3.5);
      \end{tikzpicture}
    \end{figure}
  \end{enumerate}

  \newpage

\item Consider the parabola: $y=1-2x+\frac{1}{2}x^2$.
  \begin{enumerate}
  \item Convert the general form to standard form by completing the square.

    \vspace{3in}
    
  \item What are the coordinates of the vertex?

    \vspace{1in}

  \item What are the $x$ intercepts (if any)?

    \vspace{2in}
    
  \item What are the $y$ intercepts (if any)?

    \newpage
    
  \item Where is the function increasing (in interval notation)?

    \vspace{0.5in}
    
  \item Where is the function decreasing (in interval notation)?

    \vspace{0.5in}
    
  \item What is the domain (in interval notation)?

    \vspace{0.5in}

  \item What is the range (in interval notation)?

    \vspace{0.5in}
    
  \item Where are the local maxima (if any)?

    \vspace{0.5in}
    
  \item Where are the local minima (if any)?

    \vspace{0.5in}
    
  \item Where is the axis of symmetry?

    \vspace{0.5in}
    
  \item Sketch the graph. Be sure to label \emph{all} key points.

    \bigskip

    \begin{figure}[h]
      \setlength{\leftskip}{1in}
      \begin{tikzpicture}
        \draw (-3.5,0) -- (3.5,0);
        \draw (0,-3.5) -- (0,3.5);
      \end{tikzpicture}
    \end{figure}
  \end{enumerate}

  \newpage

\item A child throws a ball up with an initial velocity of $\SI{64}{ft/s}$.
  When the ball leaves the child's hand it is already $\SI{4}{ft}$ in the air.
  How high does the ball go before it stops and comes back down?

  \newpage

\item Consider the polynomial $p(x)=2x^3-7x^2+4x+4$
  \begin{enumerate}
  \item List all of the possible candidates for real zeros of the polynomial.

    \vspace{2in}
    
  \item Completely factor $p(x)$. How you select candidates and pull out
    linear factors should be clear for full credit.

    \newpage

  \item What are the $x$ intercepts (if any)?

    \vspace{1in}
    
  \item What are the $y$ intercepts (if any)?

    \vspace{1in}

  \item Sketch the graph.

    \bigskip

    \begin{figure}[h]
      \setlength{\leftskip}{1in}
      \begin{tikzpicture}
        \draw (-5,0) -- (5,0);
        \draw (0,-5) -- (0,5);
      \end{tikzpicture}
    \end{figure}
  \end{enumerate}

  \newpage

\item Consider the rational function $r(x)=\frac{x^2-5}{(x^2-16)^2}$.
  \begin{enumerate}
  \item What are the zeros?

    \vspace{0.5in}
    
  \item What are the poles?

    \vspace{0.5in}

  \item What are the $y$-intercepts (if any)?

    \vspace{1in}

  \item Where is the horizontal asymptote?

    \vspace{0.5in}

  \item Where are the vertical asymptotes?

    \vspace{0.5in}

  \item Sketch the graph. Be sure to label all intercepts and show all
    asymptotes.

    \bigskip

    \begin{figure}[h]
      \setlength{\leftskip}{1in}
      \begin{tikzpicture}
        \draw (-5,0) -- (5,0);
        \draw (0,-3) -- (0,3);
      \end{tikzpicture}
    \end{figure}
    
  \item The graph has one local extrema. Use your calculator to locate it and
    indicate whether it is a minimum or a maximum.
  \end{enumerate}
\end{enumerate}

\end{document}
