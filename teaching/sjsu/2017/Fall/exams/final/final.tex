\documentclass[letterpaper,12pt,fleqn]{article}
\usepackage{matharticle}
\usepackage{siunitx}
\pagestyle{plain}
\newcommand{\w}{\omega}
\newcommand{\p}{\phi}
\begin{document}

\begin{center}
\Large Math-19 Final Exam
\end{center}

\vspace{0.5in}

Name: \rule{4in}{1pt}

\vspace{0.5in}

This exam is closed book and notes except for the provided trig cheat sheet and
your $3\times 5$ note card (both sides). You may use a calculator; however, no
cell phones or tablets are allowed. Show all work; there is no credit for
guessed answers. All values should be exact with no decimals unless you are
specifically asked for an approximate or decimal answer. Be careful to check
for extraneous solutions.

\vspace{0.5in}

\begin{center}
  \scalebox{1.25}{
    \begin{tabular}{|c|c|}
      \hline
      problem & score \\
      \hline
      1 & \hspace{0.5in} \\
      \hline
      2 & \\
      \hline
      3 & \\
      \hline
      4 & \\
      \hline
      5 & \\
      \hline
      6 & \\
      \hline
      7 & \\
      \hline
      8 & \\
      \hline
      9 & \\
      \hline
      10 & \\
      \hline
      11 & \\
      \hline
      12 & \\
      \hline
      13 & \\
      \hline
      14 & \\
      \hline
      15 & \\
      \hline
      16 & \\
      \hline
      17 & \\
      \hline
      18 & \\
      \hline
      19 & \\
      \hline
      20 & \\
      \hline
      TOTAL & \\
      \hline
    \end{tabular}}
\end{center}

\newpage

\begin{enumerate}

\item Simplify completely. You may assume that all variable values are
  positive.
  \[\frac{8x^{\frac{1}{2}}y^{-3}}{2x^{-2}s^4}\]

  \vspace{3in}

\item Solve for $x$:
  \[\frac{1}{x}+\frac{2}{x-1}=3\]

  \newpage

\item Consider the following conic section in general form:
  \[x^2+y^2-6x-10y+34=0\]
  \begin{enumerate}
  \item Convert to standard form.

    \vspace{3in}

  \item Sketch its graph. Be sure to label all important parts of the sketch.

    \begin{tikzpicture}
      \draw (-5,0) -- (5,0);
      \draw (0,-5) -- (0,5);
    \end{tikzpicture}
  \end{enumerate}

  \newpage

\item The maximum range of a projectile is directly proportional to the
  square of its velocity. A baseball pitcher throws a ball at $\SI{60}{mph}$,
  with a maximum range of $\SI{242}{ft}$. What is his maximum range if he
  throws the ball at $\SI{70}{mph}$?

  \vspace{3in}

\item Calculate the difference quotient for $f(x)=x^2+3x$

  \vspace{3in}

\item Consider the function $h(x)=\sqrt{x-1}+(x-1)^2-2$. Determine two
  functions $f(x)$ and $g(x)$ such that $h=f\circ g$.

  \newpage

\item Consider the following function:
  \[f(x)=-2\abs{x-1}+3\]
  \begin{enumerate}
  \item List the starting function and all transformations in the order
    that they should be applied:
    \begin{enumerate}
    \item
    \item
    \item
    \item
    \item
    \end{enumerate}

  \item Determine the $x$-intercepts (if any).

    \vspace{1.5in}

  \item Determine the $y$-intercepts (if any).

    \vspace{1in}

  \item Sketch the graph. Be sure to label all important parts.
    
    \begin{tikzpicture}
      \draw (-3.5,0) -- (3.5,0);
      \draw (0,-3.5) -- (0,3.5);
    \end{tikzpicture}
  \end{enumerate}

\item Consider the following parabola in general form:
  \[y=2x^2-8x+4\]
  \begin{enumerate}
  \item Convert to standard form.

    \vspace{2in}
    
  \item Determine the $x$-intercepts (if any).

    \vspace{1.5in}

  \item Determine the $y$-intercepts (if any).

    \vspace{0.5in}

  \item Sketch the graph. Be sure to label all important parts.

    \begin{tikzpicture}
      \draw (-3.5,0) -- (3.5,0);
      \draw (0,-3.5) -- (0,3.5);
    \end{tikzpicture}
  \end{enumerate}

\item Consider the following polynomial function:
  \[y=x^4-3x^3+2x\]
  \begin{enumerate}
  \item Factor completely. For full credit you must show how you construct
    candidate zeros, how you determine which candidates are actual zeros, and
    then how you factor out the identified zeros.

    \vspace{4in}

  \item Sketch the polynomial. Be sure to show the proper end behavior,
    label all zeros, and determine all extrema using your calculator.
  
    \begin{tikzpicture}
      \draw (-4,0) -- (4,0);
      \draw (0,-4) -- (0,4);
    \end{tikzpicture}
  \end{enumerate}

\item Sketch the following rational function. Be sure to show the proper
  end behavior and label all zeros, $y$-intercepts, and asymptotes.
  \[y=\frac{x-2}{x^2-2x-8}\]

  \begin{tikzpicture}
    \draw (-5,0) -- (5,0);
    \draw (0,-5) -- (0,5);
  \end{tikzpicture}

\item Completely expand the following:
  \[\log\left[\frac{4x^3}{y^2(x-1)^5}\right]\]

  \newpage

\item Solve for $x$:
  \[\log_3(x-8)+\log_3x=2\]

  \vspace{3.5in}

\item A sample of bismuth-210 decayed to $33\%$ of its original mass after
  $\SI{8}{days}$. Find the half-life of this element.

  \newpage

\item Determine the standard form equation for an ellipse with foci at
  $(-2,2)$ and $(4,2)$ and a vertex at $(1,6)$.

  \vspace{4in}

\item You are standing about $\SI{1}{mile}$ away from a hill. You notice
  someone hiking at the highest point of the hill. Your head is tilted up
  at an angle of $30^{\circ}$. How high is the hill?

  \newpage

\item Sketch the graph for one period of the following sinusoidal function. Be
  sure to show how you calculate the period and the five key points on the
  graph. Be sure to show the amplitude and the $t$ values for the five key
  points.
  \[y=-2\sin\left(\frac{\pi}{2}t-\frac{\pi}{6}\right)\]
  
  \begin{tikzpicture}
    \draw (-5,0) -- (5,0);
    \draw (0,-5) -- (0,5);
  \end{tikzpicture}

  \newpage
  
\item Find all solutions to the following equation and state the answer in the
  most efficient possible form.
  \[2\sin^2x-1=0\]

  \vspace{3in}

\item Find all solutions to the following equation:
  \[sin^2t=4-2\cos^2t\]

  \newpage

\item Simplify the following:
  \[\cos(\sin^{-1}x+\cot^{-1}y)\]

  \vspace{3.5in}

\item Rewrite the following in $A\sin(\w t+\p)$ form:
  \[\sin(2x)+\sqrt{3}\cos(2x)\]
\end{enumerate}

\end{document}
