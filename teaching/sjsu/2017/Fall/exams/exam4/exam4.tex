\documentclass[letterpaper,12pt,fleqn]{article}
\usepackage{matharticle}
\usepackage{siunitx}
\pagestyle{plain}
\begin{document}

\begin{center}
\Large Math-19 Exam \#4
\end{center}

\vspace{0.5in}

Name: \rule{4in}{1pt}

\vspace{1in}

This exam is closed book and notes. You may use a calculator and both sides of
a $3\times 5$ note card; however, no cell phones or tablets are allowed. Show
all work; there is no credit for guessed answers. All values should be exact
with no decimals unless you are specifically asked for an approximate or
decimal answer.

\vspace{0.5in}

\begin{center}
  \scalebox{2}{
    \begin{tabular}{|c|c|}
      \hline
      problem & score \\
      \hline
      1 & \hspace{0.5in} \\
      \hline
      2 & \\
      \hline
      3 & \\
      \hline
      4 & \\
      \hline
      5 & \\
      \hline
      TOTAL & \\
      \hline
    \end{tabular}}
\end{center}

\newpage

\begin{enumerate}
\item Shown below is the graph for the parabola $y=2(x+1)^2-2$:

  \begin{tikzpicture}
    \draw (-4,0) -- (4,0);
    \draw (0,-4) -- (0,4);
    \draw [smooth,domain=-2.75:0.75] plot ({\x},{2*(\x+1)^2-2});
    \node [draw,circle,fill,scale=0.5] (v) at (-1,-2) {};
    \node [below] at (v) {$(-1,-2)$};
    \node [draw,circle,fill,scale=0.5] (a) at (-2,0) {};
    \node [above left] at (a) {$(-2,0)$};
    \node [draw,circle,fill,scale=0.5] (b) at (0,0) {};
    \node [above right] at (b) {$(0,0)$};
  \end{tikzpicture}

  \begin{enumerate}
  \item State a limited explicit domain (in interval notation) such that
    the function with your limited domain has an inverse.

    \vspace{0.5in}
    
  \item How do you know that the function with your limited domain has an
    inverse?

    \vspace{0.5in}
    
  \item Determine the inverse function.

    \vspace{2.5in}
    
  \item Assuming that it is in your selected domain, what is $f^{-1}(-2)$?
  \end{enumerate}

  \newpage

\item A parabola has its focus at $(-1,2)$ and its directrix at $x=1$.
  \begin{enumerate}
  \item What are the coordinates of the vertex?

    \vspace{1in}
    
  \item What is the value for $p$?

    \vspace{1in}
    
  \item What are the y-intercepts (if any)?

    \vspace{1in}
    
  \item What are the x-intercepts (if any)?

    \vspace{1in}
    
  \item Sketch the graph. For full credit you must show the vertex, focus,
    directrix, and any intercepts.

    \begin{tikzpicture}
      \draw (-5,0) -- (5,0);
      \draw (0,-3.5) -- (0,3.5);
    \end{tikzpicture}
  \end{enumerate}

  \newpage

\item Consider the conic section:
  \[16x^2+y^2+32x-2y+13=0\]
  \begin{enumerate}
  \item What are the values for $a$, $b$, and $c$?

    \vspace{4in}
      
  \item What are the coordinates of the center?

    \vspace{1in}
      
  \item What are the coordinates of the foci?

    \vspace{1in}
      
  \item What are the coordinates of the vertices?

    \newpage
      
  \item Sketch the graph. For full credit you must show the center, foci, and
    all vertices.

    \begin{tikzpicture}
      \draw (-5,0) -- (5,0);
      \draw (0,-5) -- (0,5);
    \end{tikzpicture}
  \end{enumerate}

  \vspace{0.5in}

\item Solve for $x$:
  \[\log_4(x)+\log_4(x-1)=\frac{1}{2}\]

  \newpage
  
\item Some archaeologists are digging at what appears to be a pre-Columbian
  human campsite in California.  They find some animal bones with human teeth
  marks on them.  Upon carbon-14 analysis, it is found that the bones have 75\%
  of their original $C_{14}$.  About how old are the bones, and hence the
  campsite?  The half-life of $C_{14}$ is 5730 years.

\end{enumerate}

\end{document}
