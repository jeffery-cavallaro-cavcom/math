\documentclass[letterpaper,12pt,fleqn]{article}
\usepackage{matharticle}
\usepackage{siunitx}
\usepackage{pgfplots}
\pgfplotsset{compat=1.14}
\pagestyle{plain}
\begin{document}

\begin{center}
\Large Math-19 Exam \#2
\end{center}

\vspace{0.5in}

Name: \rule{4in}{1pt}

\vspace{0.5in}

This exam is closed book and notes. You may use a calculator; however, no cell
phones or tablets are allowed. Show all work; there is no credit for guessed
answers. All values should be exact with no decimals unless you are
specifically asked for an approximate or decimal answer.

\vspace{0.5in}

\newcommand{\fillin}{\rule[-6pt]{3in}{1pt}}
\newcommand{\sfillin}{\rule[-6pt]{1in}{1pt}}

\begin{enumerate}
\item (10 points) Identify each of the following formulas. Be very specific.
  Differentiate between general and standard forms and call out important
  points (like the center of a circle).

\vspace{0.25in}

\begin{tabular}{cc}
$d=[(x_1-x_2)^2+(y_1-y_2)^2]^{1/2}$ & \fillin \\
\\
$\left(\frac{x_1+x_2}{2},\frac{y_1+y_2}{2}\right)$ & \fillin \\
\\
$m=\frac{y_1-y_2}{x_1-x_2}$ & \fillin \\
\\
$y-y_1=m(x-x_1)$ & \fillin \\
\\
$y=mx+b$ & \fillin \\
\\
$Ax+By+C=0$ & \fillin \\
\\
$m_1=m_2$ & \fillin \\
\\
$m_1m_2=-1$ & \fillin \\
\\
$(x-h)^2+(y-k)^2=r^2$ & \fillin \\
\\
$x^2+y^2+Dx+Ey+F=0$ & \fillin \\
\end{tabular}

\newpage

\item (10 points) Identify each of the following standard functions:

  \vspace{0.25in}

  \begin{tabular}{cc}
    \begin{tikzpicture}
      \draw [<->] (-2,0) -- (2,0);
      \draw [<->] (0,-2) -- (0,2);
      \draw [domain=-2:2] plot (\x,{\x});
      \node at (-1,-3) {$y=$};
    \end{tikzpicture} \hspace{1in} &
    \begin{tikzpicture}
      \draw [<->] (-2,0) -- (2,0);
      \draw [<->] (0,-2) -- (0,2);
      \draw [domain=-2:2] plot (\x,{abs(\x)});
      \node at (-1,-3) {$y=$};
    \end{tikzpicture} \\
  \end{tabular}

  \begin{tabular}{cc}
    \begin{tikzpicture}
      \draw [<->] (-2,0) -- (2,0);
      \draw [<->] (0,-2) -- (0,2);
      \draw [domain=0:2] plot (\x,{sqrt(\x)});
      \node at (-1,-3) {$y=$};
    \end{tikzpicture} \hspace{1in} &
    \begin{tikzpicture}
      \draw [<->] (-2,0) -- (2,0);
      \draw [<->] (0,-2) -- (0,2);
      \draw [domain=0:2] plot (\x,{(\x)^(1/3)});
      \draw [domain=-2:0] plot (\x,{-(-\x)^(1/3)});
      \node at (-1,-3) {$y=$};
    \end{tikzpicture} \\
  \end{tabular}

  \begin{tabular}{cc}
    \begin{tikzpicture}
      \draw [<->] (-2,0) -- (2,0);
      \draw [<->] (0,-2) -- (0,2);
      \draw [domain=-1.4:1.4] plot (\x,{(\x)^2});
      \node at (-1,-3) {$y=$};
    \end{tikzpicture} \hspace{1in} &
    \begin{tikzpicture}
      \draw [<->] (-2,0) -- (2,0);
      \draw [<->] (0,-2) -- (0,2);
      \draw [domain=-1.25:1.25] plot (\x,{(\x)^3});
      \node at (-1,-3) {$y=$};
    \end{tikzpicture} \\
  \end{tabular}

  \newpage

\item (10 points) You own a candy and nut store. Peanuts sell for \$4.00 per
  lb and cashews sell for \$7.50 per lb. Unfortunately, the more expensive
  cashews are not selling very well, so you decide to make a peanut/cashew mix.
  You mix $\SI{10}{lb}$ of peanuts with $\SI{10}{lb}$ of cashews. How much
  should you charge per pound for the mix?

  \vspace{3in}

\item (10 points) Consider the inequality: $x^2-7x-18<0$.
  \begin{enumerate}
  \item Solve for $x$. Your answer should be in interval notation and should
    have an accompanying graph.

    \vspace{2.5in}

  \item Construct the corresponding absolute value inequality.
    
  \end{enumerate}

  \newpage
  
\item (20 points) A new rock band is performing at the SJSU event center. The
  band wants to spread some posters around town advertising the performance.
  The printer says that the cost to print each poster varies directly with the
  area of the poster and inversely with the number of posters ordered.
  \begin{enumerate}
  \item Let $P=$ the cost to print each poster, $A=$ area of each poster and
    $n=$ number of posters printed. Write an equation that expresses the cost
    of each poster in terms of $A$ and $n$.

    \vspace{2in}

  \item What is the value of the constant of proportionality if the price per
    poster is \$4 when printing 100 posters of size 80 square inches each?

    \vspace{2in}

  \item The band decides that they would like the posters to be a bit bigger
    and that they don't need 100 of them. How much is the price per poster when
    printing 80 posters of size 160 square inches each?
  \end{enumerate}

  \newpage

\item (20 points) You have two dogs: Fido and Fluffy. Each dog is tied to its
own stake in your backyard by a leash. Fido's stake and leash allow him to roam
around an area defined by:
\[(x-2)^2+(y-1)^2=9\]
Fluffy's stake and leash allow her to roam around an area defined by:
\[x^2+y^2-10x-8y+37=0\]
\begin{enumerate}
\item What are the coordinates of Fido's stake and the length of his leash?

\vspace{0.5in}

\item What are the coordinates of Fluffy's stake and the length of her leash?

\vspace{1.5in}

\item What is the equation of the line between the two stakes, in
slope-intercept form?

\vspace{1.5in}

\item It is mating season. Fido and Fluffy and not fixed, but you do not want
them to mate. To be safe, you decide to erect a straight wall that is
perpendicular to the line joining the two stakes and going through the midpoint
of that line. What is the equation of the wall, in slope-intercept form?
\end{enumerate}

\newpage

\item (20 points) Below is the graph for the polynomial function
  $f(x)=x(x-1)(x+2)(x-3)$.

  \begin{tikzpicture}[scale=2]
    \begin{axis}[
        xmin=-10,xmax=10,
        ymin=-10,ymax=10,
        grid style={line width=.1pt, draw=gray!10},
        major grid style={line width=.2pt,draw=gray!50},
        axis lines=middle,
        axis line style={latex-latex},
        xtick={-5,-4,-3,-2,-1,0,1,2,3,4,5},
        ytick={-5,0,5},
        ticklabel style={font=\tiny},
      ]
      \draw [domain=-2.5:3.5,smooth] plot ({\x},{\x*(\x-1)*(\x+2)*(\x-3)});
    \end{axis}
  \end{tikzpicture}

  \begin{enumerate}
  \item Is this a function? If so, why?

    \vspace{1in}
    
  \item What is $f(-1)$?

    \vspace{1in}

  \item What are the $x$-intercepts (if any)?

    \vspace{1in}

  \item What are the $y$-intercepts (if any)?

    \vspace{1in}

  \item Use your calculator to determine all of the local maxima (if any).
    Round all answers to one decimal point.

    \vspace{1in}

  \item Use your calculator to determine all of the local minima (if any).
    Round all answers to one decimal point.

    \vspace{1in}
    
  \item What is the domain?

    \vspace{1in}
    
  \item What is the range?

    \vspace{1in}
    
  \item Where is the function increasing (in interval notation)?

    \vspace{1in}
    
  \item Where is the function decreasing (in interval notation)?
  \end{enumerate}
  
\end{enumerate}

\end{document}
