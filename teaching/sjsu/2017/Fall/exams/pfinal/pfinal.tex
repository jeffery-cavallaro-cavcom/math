\documentclass[letterpaper,12pt,fleqn]{article}
\usepackage{matharticle}
\usepackage{siunitx}
\pagestyle{plain}
\newcommand{\w}{\omega}
\newcommand{\p}{\phi}
\begin{document}

\begin{center}
\Large Math-19 Practice Final Exam
\end{center}

\vspace{0.5in}

\begin{enumerate}

\item Simplify completely. You may assume that all variable values are
  positive.
  \[\frac{27ab^2c^{-3}}{3a^3b^{-4}}\]

  \vspace{3in}

\item Solve for $x$:
  \[\frac{1}{x-2}+\frac{1}{x+2}=\frac{8}{x^2-4}\]

  \newpage

\item Consider the following conic section in general form:
  \[x^2+y^2+2x-6y+9=0\]
  \begin{enumerate}
  \item Convert to standard form.

    \vspace{3in}

  \item Sketch its graph. Be sure to label all important parts of the sketch.

    \begin{tikzpicture}
      \draw (-5,0) -- (5,0);
      \draw (0,-5) -- (0,5);
    \end{tikzpicture}
  \end{enumerate}

  \newpage

\item The terminal velocity of a parachutist is directly proportional to the
  square root of his weight. A $\SI{160}{lb}$ parachutist attains a terminal
  velocity of $\SI{9}{mph}$. What is the terminal velocity for a parachutist
  weighing $\SI{240}{lb}$?
  
  \vspace{3in}

\item Calculate the difference quotient for $f(x)=2x^2-x+3$

  \vspace{3in}

\item Consider the function $h(x)=\abs{x}+\sqrt{x}-10$. Determine two
  functions $f(x)$ and $g(x)$ such that $h=g\circ f$.

  \newpage

\item Consider the following function:
  \[f(x)=-2\sqrt{x-1}+3\]
  \begin{enumerate}
  \item List the starting function and all transformations in the order
    that they should be applied:
    \begin{enumerate}
    \item
    \item
    \item
    \item
    \item
    \end{enumerate}

  \item Determine the $x$-intercepts (if any).

    \vspace{1.5in}

  \item Determine the $y$-intercepts (if any).

    \vspace{1in}

  \item Sketch the graph. Be sure to label all important parts.
    
    \begin{tikzpicture}
      \draw (-3.5,0) -- (3.5,0);
      \draw (0,-3.5) -- (0,3.5);
    \end{tikzpicture}
  \end{enumerate}

\item Consider the following parabola in general form:
  \[y=-2x^2+12x\]
  \begin{enumerate}
  \item Convert to standard form.

    \vspace{2in}

  \item Determine the $x$-intercepts (if any).

    \vspace{1.5in}

  \item Determine the $y$-intercepts (if any).

    \vspace{0.5in}

  \item Sketch the graph. Be sure to label all important parts.

    \begin{tikzpicture}
      \draw (-3.5,0) -- (3.5,0);
      \draw (0,-3.5) -- (0,3.5);
    \end{tikzpicture}
  \end{enumerate}

\item Consider the following polynomial function:
  \[y=x^4-3x^3+4x\]
  \begin{enumerate}
  \item Factor completely. For full credit you must show how you construct
    candidate zeros, how you determine which candidates are actual zeros, and
    then how you factor out the identified zeros.

    \vspace{4in}

  \item Sketch the polynomial. Be sure to show the proper end behavior,
    label all zeros, and determine all extrema using your calculator.
  
    \begin{tikzpicture}
      \draw (-4,0) -- (4,0);
      \draw (0,-4) -- (0,4);
    \end{tikzpicture}
  \end{enumerate}

\item Sketch the following rational function. Be sure to show the proper
  end behavior and label all zeros, $y$-intercepts, and asymptotes.
  \[y=\frac{x-3}{x^2+x-6}\]

  \begin{tikzpicture}
    \draw (-5,0) -- (5,0);
    \draw (0,-5) -- (0,5);
  \end{tikzpicture}

\item Completely expand the following:
  \[\log\left[\frac{x\sqrt{x^2+1}}{(x-1)y^2}\right]\]

  \newpage

\item Solve for $x$:
  \[\log_8(x-5)-\log_8(x-2)=1\]

  \vspace{3.5in}

\item A sample of palladium-100 decayed to $29.73\%$ of its original mass after
  $\SI{7}{days}$. Find the half-life of this element.

  \newpage

\item Determine the standard form equation for an ellipse with foci at
  $(-1,3)$ and $(-1,-1)$ and a vertex at $(-1,4)$.

  \vspace{4in}

\item You are standing about $\SI{5}{ft}$ from your car when you notice a bird
  on a power line above your car. To your horror, the bird poops on your just
  washed and waxed car. You look up at the bird with extreme anger at an angle
  of $76^{\circ}$. How high up is the bird?

  \newpage

\item Sketch the graph for one period of the following sinusoidal function. Be
  sure to show how you calculate the period and the five key points on the
  graph. Be sure to show the amplitude and the $t$ values for the five key
  points.
  \[y=3\sin\left(\frac{\pi}{4}t-\frac{\pi}{2}\right)\]
  
  \begin{tikzpicture}
    \draw (-5,0) -- (5,0);
    \draw (0,-5) -- (0,5);
  \end{tikzpicture}
  
  \newpage

\item Find all solutions to the following equation and state the answer in the
  most efficient possible form.
  \[4\cos^2x-3=0\]

  \vspace{3in}

\item Find all solutions to the following equation:
  \[2\cos^2t+\sin t=1\]

  \newpage
  
\item Simplify the following:
  \[\sin(\cos^{-1}x+\tan^{-1}y)\]

  \vspace{3.5in}

\item Rewrite the following in $A\sin(\w t+\p)$ form:
  \[\sin(x)-\cos(x)\]
\end{enumerate}
  
\end{document}
