\documentclass[letterpaper,12pt,fleqn]{article}
\usepackage{matharticle}
\pagestyle{plain}
\begin{document}

\begin{center}
\Large Math-19 Homework \#12
\end{center}

\vspace{0.5in}

\underline{Problems}

\begin{enumerate}

\item Consider the circle $x^2+y^2=r^2$ and remember that we needed to restrict
  the range in order to obtain the function $y=\sqrt{r^2-x^2}$.
  \begin{enumerate}
  \item Sketch the half-circle function and demonstrate why it is not
    one-to-one?
  \item Suggest a how to limit the domain so that it is a one-to-one function.
  \item Sketch the new graph for the one-to-one function and state its domain
    and range.
  \item By observing the graph (and the line $y=x$), predict something about the
    inverse function.
  \item Derive the inverse to prove your prediction.
  \end{enumerate}
  
\item You use \$1000 to open a savings account at your local bank on the first
  of February. The savings account has an interest rate of 1.5\% per year and
  compounds monthly on the last day of the month. You set up an auto-deposit
  of \$100 from your paycheck to occur on the first of each month, starting
  with the second month (March).  During April, you withdraw \$250 to
  purchase a new gameboy (gotta catch em all!).
  \begin{enumerate}
  \item Who is the lender and who is the borrower?
  \item Calculate $x=1+\frac{r}{n}$
  \item Construct a polynomial in $x$ to determine the account value on
    July 2.
  \item What is the account value on July 2?
  \end{enumerate}

\item Consider the exponential function $y=-2e^{-(x+1)}-3$
  \begin{enumerate}
  \item List the transformations in the order that they should be applied.
  \item What is the y-intercept (if any)?
  \item What is the domain (in interval notation)?
  \item What is the range (in interval notation)?
  \end{enumerate}

\item Consider the logarithmic function $y=\log(-2(x+1))+3$
  \begin{enumerate}
  \item List the transformations in the order that they should be applied.
  \item What is the y-intercept (if any)?
  \item What is the domain (in interval notation)?
  \item What is the range (in interval notation)?
  \end{enumerate}
\end{enumerate}
 
\end{document}
