\documentclass[letterpaper,12pt,fleqn]{article}
\usepackage{matharticle}
\pagestyle{plain}
\begin{document}

\begin{center}
\Large Math-19 Homework \#8 Solutions
\end{center}

\underline{Problems}

\begin{enumerate}
\item You are a cost accountant at a firm that manufactures pencils. You
  prepare next month's cost study for your boss as follows:

  \begin{tabular}{|l|l|}
    \hline
    item & cost (\$) \\
    \hline
    wood (per pencil) & 0.05 \\
    graphite (per pencil) & 0.02 \\
    paint (per pencil) & 0.01 \\
    rubber (per pencil) & 0.04 \\
    metal (per pencil) & 0.03 \\
    lease on plant & 10,000 \\
    labor & 20,000 \\
    depreciation & 10,000 \\
    utilities & 5,000 \\
    other plant costs & 7,500 \\
    \hline
  \end{tabular}

  Your marketing department's research indicates that you can sell your pencils
  to retailers at $25$ cents per pencil.
  \begin{enumerate}
  \item Construct a linear model for this situation.

    Variable Costs$=0.05+0.02+0.01+0.04+0.03=0.15$

    Fixed Costs$=10000+20000+10000+5000+7500=52500$

    $P(n)=R(n)-C(n)=0.25n-(0.15n+52500)=0.10n-52500$
    
  \item How many pencils do you need to sell to break even?

    $P(n)=0=0.10n-52500$ \\
    $0.10n=52500$ \\
    $n=525000$
  \end{enumerate}

\item Consider the function: $y=-\abs{x-2}+3$
  \begin{enumerate}
  \item List the starting standard function and the three transformation
    steps in the order that they should be applied.
    \begin{enumerate}
    \item Start with $y=\abs{x}$
    \item Translate right $2$
    \item Reflect x-axis
    \item Translate up 3
    \end{enumerate}

  \item What are the $x$-intercepts (if any)?

    $0=-\abs{x-2}+3$ \\
    $\abs{x-2}=3$ \\
    $x-2=\pm3$ \\
    $x=-1,5$

    $(-1,0)$ and $(5,0)$
    
  \item What are the $y$-intercepts (if any)?

    $y=-\abs{0-2}+3=-\abs{-2}+3=-2+3=1$

    $(0,1)$

  \item What are the local maxima (if any)?

    There is one local maximum at the translated vertex: $(2,3)$

  \item What are the local minima (if any)?

    None.

  \item What is the domain?

    $\R$

  \item What is the range?

    $(-\infty,3]$

  \item What is the axis of symmetry?

    $x=2$

  \item Sketch the graph of the function. Be sure to label all important
    points.

    \begin{tikzpicture}
      \draw (-3,0) -- (7,0);
      \draw (0,-4) -- (0,4);
      \node [draw,circle,fill,scale=0.5] (x1) at (-1,0) {};
      \node [above left] at (x1) {$(-1,0)$};
      \node [draw,circle,fill,scale=0.5] (x2) at (5,0) {};
      \node [above right] at (x2) {$(5,0)$};
      \node [draw,circle,fill,scale=0.5] (y) at (0,1) {};
      \node [right] at (y) {$(0,1)$};
      \node [draw,circle,fill,scale=0.5] (v) at (2,3) {};
      \node [above] at (v) {$(2,3)$};
      \draw (-3,-2) to (v) to (7,-2);
      \draw [dashed] (2,-4) -- (2,4);
      \node [below right] at (2,0) {$2$};
    \end{tikzpicture}
    
  \end{enumerate}

\item Consider the function: $y=-\frac{1}{x+2}+1$
  \begin{enumerate}
  \item List the starting standard function and the three transformation
    steps in the order that they should be applied.
    \begin{enumerate}
    \item Start with $y=\frac{1}{x}$
    \item Translate right 2
    \item Reflect x-axis
    \item Translate up 1
    \end{enumerate}

  \item What is the equation of the horizontal asymptote?

    $y=1$

  \item What is the equation of the vertical asymptote?

    $x=-2$

  \item What is the end behavior as $x\to\infty$ - be very specific.

    $f(x)\to1^-$

  \item What is the end behavior as $x\to-\infty$ - be very specific.

    $f(x)\to1^+$

  \item What are the $x$-intercepts (if any)?

    $0=-\frac{1}{x+2}+1$ \\
    $\frac{1}{x+2}=1$ \\
    $x+2=1$ \\
    $x=-1$

    $(-1,0)$
    
  \item What are the $y$-intercepts (if any)?

    $y=-\frac{1}{0+2}+1=-\frac{1}{2}+1=-\frac{1}{2}$

    $(0,-\frac{1}{2})$

  \item What are the local maxima (if any)?

    none.

  \item What are the local minima (if any)?

    none.

  \item What is the domain?

    $(-\infty,-2)\cup(-2,\infty)$
    
  \item What is the range?

    $(-\infty,1)\cup(1,\infty)$

  \item Sketch the graph of the function. Be sure to label all important
    points.

    \begin{tikzpicture}
      \draw (-5,0) -- (5,0);
      \draw (0,-5) -- (0,5);
      \draw [dashed] (-5,1) -- (5,1);
      \draw [dashed] (-2,-5) -- (-2,5);
      \node [below right] at (-2,0) {$-2$};
      \node [above right] at (0,1) {$1$};
      \node [draw,circle,fill,scale=0.5] (x) at (-1,0) {};
      \node [below right] at (x) {$(-1,0)$};
      \node [draw,circle,fill,scale=0.5] (y) at (0,0.5) {};
      \node [above left] at (y) {\scalebox{0.75}{$\left(0,\frac{1}{2}\right)$}};
      \draw [smooth,domain=-5:-2.25] plot ({\x},{-1/(\x+2)+1});
      \draw [smooth,domain=-1.835:5] plot ({\x},{-1/(\x+2)+1});
    \end{tikzpicture}
  \end{enumerate}

\item Evaluate the difference quotient for the function: $f(x)=x^2+3x-1$.
  \begin{eqnarray*}
    \frac{f(x+h)-f(x)}{h} &=& \frac{[(x+h)^2+3(x+h)-1]-[x^2+3x-1]}{h} \\
    &=& \frac{[x^2+2xh+h^2+3x+3h-1]-[x^2+3x-1]}{h} \\
    &=& \frac{2xh+h^2+3h}{h} \\
    &=& 2x+3+h
  \end{eqnarray*}
  
\item Consider the following functions:

  $f(x)=\sqrt{x}+1$

  $g(x)=x^2$

  $h(x)=2x-1$

  Compute AND find the domain for the following functions:

  \begin{enumerate}
  \item $(g-h)(x)$

    $(g-h)(x)=g(x)-h(x)=x^2-(2x-1)=x^2-2x+1=(x-1)^2$

    Domain: $\R$

  \item $(fg)(x)$

    $(fg)(x)=f(x)g(x)=(\sqrt{x}+1)x^2$

    Domain: $[0,\infty)$

  \item $\left(\frac{g}{h}\right)(x)$
      
    $\left(\frac{g}{h}\right)(x)=\frac{g(x)}{h(x)}=\frac{x^2}{2x-1}$

    Domain: $\left\{x\in\R\mid x\ne\frac{1}{2}\right\}=
    \left(-\infty,\frac{1}{2}\right)\cup\left(\frac{1}{2},\infty\right)$
    
  \item $(f\circ g)(x)$

    $(f\circ g)(x)= f(g(x))=f(x^2)=\sqrt{x^2}+1=\abs{x}+1$

    Note that $g(x)$ guarantees that only nonnegative values are fed to $f(x)$.

    Domain: $\R$
    
  \item $\left(\frac{f}{f}\right)(x)$

    $\left(\frac{f}{f}\right)(x)=\frac{f(x)}{f(x)}=
    \frac{\sqrt{x}+1}{\sqrt{x}+1}=1$

    Note that even though the final result looks like it can take $\R$, we
    must honor the domain of $f(x)$.

    Domain: $[0,\infty)$
  \end{enumerate}
\end{enumerate}

\end{document}
