\documentclass[letterpaper,12pt,fleqn]{article}
\usepackage{matharticle}
\pagestyle{plain}
\begin{document}

\begin{center}
\Large Math-19 Homework \#13
\end{center}

\vspace{0.5in}

\underline{Problems}

\begin{enumerate}

\item Given:
  \[\log_b{2}=0.6931\]
  \[\log_b{3}=1.0986\]
  \[\log_b{5}=1.6094\]
  find $\log_b{\left(\frac{75}{4}\right)}$. You must use each one of the given
  values, you are not allowed to determine the value of $b$, and you must show
  exactly how you obtained the answer.

\item Consider the equation: $y=\log_a{x}$
\begin{enumerate}
\item Derive the change of base formula for some arbitrary base $b$.
\item Use your formula with $b=e$ and your calculator to compute $\log_7100$.
\item Assume that you made a mistake and used the common log key instead of the
natural log key in the above calculation. Would you get a different answer?
Why or why not?
\end{enumerate} 

\item Researchers tend to prefer exponential (base $e$) equations. For example,
the normal equation for the radioactive decay of Carbon-14, which has a
half-life of 5730 years, would be:
\[A=A_0\cdot2^{-\frac{t}{5730}}\]
Find a value for $x$ such that:
\[A=A_oe^{xt}\]
is an equivalent equation.

\item The San Francisco earthquake of 1906 was estimated at 7.8 on the Richter
Scale. Current building codes have resulted in skyscrapers that can withstand
an 8.0 earthquake. How much strong is this than the 1906 quake?
\end{enumerate}
 
\end{document}
