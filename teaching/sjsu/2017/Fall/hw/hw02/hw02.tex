\documentclass[letterpaper,12pt,fleqn]{article}
\usepackage{matharticle}
\usepackage{mathtools}
\pagestyle{plain}
\begin{document}

\begin{center}
\Large Math-19 Homework \#2
\end{center}

\vspace{0.5in}

\underline{Problems}

\begin{enumerate}
\item Let:
\begin{eqnarray*}
P &\coloneqq& 0\ \mbox{is a positive number} \\
Q &\coloneqq& 2\ge2 \\
R &\coloneqq& \forall n,m\in\mathbb{N}, n+m\in\mathbb{N} \\
\end{eqnarray*}
Determine whether the following (compound) statements are true or false:

\begin{tabular}{|c|c|}
  \hline
  Statement & T/F \\
  \hline
  P & \\
  \hline
  Q & \\
  \hline
  R & \\
  \hline
  not P & \\
  \hline
  not Q & \\
  \hline
  not R & \\
  \hline
  P and Q & \\
  \hline
  P and R & \\
  \hline
  Q and R & \\
  \hline
  P or Q & \\
  \hline
  P or R & \\
  \hline
  Q or R & \\
  \hline
\end{tabular}

\bigskip

\item Convert $10.2\overline{45}$ to rational form.

\bigskip

\item Let:
\begin{eqnarray*}
A &=& \mbox{the set of all positive real numbers} \\
B &=& \mbox{the set of real numbers between -3 (exclusive) and 3 (inclusive)} \\
\end{eqnarray*}
\begin{enumerate}
\item Graph each set on the real number line.
\item Represent each set using set-builder notation.
\item Represent each set using interval notation.
\item Graph $A\cup B$ and represent it in interval notation.
\item Graph $A\cap B$ and represent it in interval notation.
\item Graph $A-B$ and represent it in interval notation.
\end{enumerate}

\item A careful solution of $4(x+2)=11$ is given below. Give the rationale for
each step from the ten real number rules (AC,AA,A0,AI,MC,MA,M1,MI,LD,RD) and
the additional rules (SUB,WD).

\newcommand{\fillin}{\rule{1in}{1pt}}

\begin{tabular}{ll}
$4(x+2)=11$ & \\
$4x+8=11$ & \fillin \\
$(4x+8)-8=11-8$ & \fillin \\
$(4x+8)-8=3$ & \fillin \\
$4x+(8-8)=3$ & \fillin \\
$4x+0=3$ & \fillin \\
$4x=3$ & \fillin \\
$\frac{1}{4}(4x)=\frac{1}{4}(3)$ & \fillin \\
$\frac{1}{4}(4x)=\frac{3}{4}$ & \fillin \\
$(\frac{1}{4}4)x=\frac{3}{4}$ & \fillin \\
$1x=\frac{3}{4}$ & \fillin \\
$x=\frac{3}{4}$ & \fillin \\
\end{tabular}

\bigskip

\item Consider the statement: $\forall\,a,b\in\R,|a-b|=|b-a|$
  \begin{enumerate}
  \item Give a careful proof of this statement. You will need to use one of
    the distributive rules (hint: factor out a -1), one of the properties in
    the box at the top of page 9 of your textbook, and the definition of
    absolute value.

  \item What does this statement mean (what are the semantics)? (Hint: think
    distance)
  \end{enumerate}
\end{enumerate}
\end{document}
