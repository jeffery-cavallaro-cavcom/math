\documentclass[letterpaper,12pt,fleqn]{article}
\usepackage{matharticle}
\pagestyle{plain}
\begin{document}

\begin{center}
\Large Math-19 Homework \#11 Solutions
\end{center}

\vspace{0.5in}

\underline{Problems}

\begin{enumerate}
\item You are designing a new spotlight with a parabolic reflector. The
  diameter of the reflector is 2 feet and the height is 3 feet. How high above
  the bottom of the dish should the lightbulb be placed.

  Let's place a sketch of the spotlight on a coordinate system with the vertex
  at the origin:

  \begin{tikzpicture}
    \draw (-3,0) -- (3,0);
    \draw (0,-1) -- (0,4);
    \draw [smooth,domain=-1:1] plot ({\x},{3*(\x)^2});
    \draw (-1,3) -- (1,3);
    \node [left] at (-1,3) {$(-1,3)$};
    \node [right] at (1,3) {$(1,3)$};
  \end{tikzpicture}

  Based on the reflective nature of the parabola, the lightbulb should be
  placed at the focus: $(0,p)$, so we need to find $p$:

  $x^2=4py$ \\
  $(1)^2=4p(3)$ \\
  $1=12p$ \\
  $p=\frac{1}{12}$

  And so the lightbulb should be placed at $\frac{1}{12}$ of a foot, or at
  the $1$ inch mark.

\item The earth's orbit around the sun is elliptical with the sun at one of
  the foci. The closest the earth gets to the sun is about 91 million miles.
  The eccentricity of the earth's orbit is about 0.0167. What is the farthest
  distance between the earth and the sun?

  let's place the orbit of the earth on a coordinate system with the center at
  the origin and the sun at one focus:

  \begin{tikzpicture}
    \draw (-4,0) -- (4,0);
    \draw (0,-4) -- (0,4);
    \draw [dashed] (0,0) ellipse (3 and 2);
    \node [draw,circle] at (1,0) {$S$};
    \node [draw,circle,scale=0.5] at (3,0) {};
    \node [below right] at (3,0) {$e$};
    \node [above] at (2,0) {$a-c$};
    \node [above] at (-1,0) {$a+c$};
  \end{tikzpicture}

  Place the sun at the focus $(c,0)$. When the earth is at its perihelion
  (closest point to sun) it is at the vertex $(a,0)$. Thus, the distance from
  the sun to the earth is $a-c=91$ million miles. We want the distance at
  aphelion (farthest point from sun), which is $a+c$. Since we know that the
  eccentricity $e=\frac{c}{a}=0.0167$, we have two equations in two unknowns:
  \begin{eqnarray*}
    a-c &=& 91 \\
    c &=& 0.0167a
  \end{eqnarray*}
  Using substitution to solve for $a$:
  \begin{eqnarray*}
    a-0.0167a &=& 91 \\
    0.9833a &=& 91 \\
    a &=& \frac{91}{0.9833}
  \end{eqnarray*}
  Don't jump right to the calculator yet; that introduces needless error.
  Instead, keep things rational until the end. Now, solve for $c$:
  \[c=0.0167a=\frac{0.0167}{0.9833}\cdot91\]
  And finally, calculate $a+c$:
  \[a+c=\frac{91}{0.9833}+\frac{0.0167}{0.9833}\cdot91=
  \frac{91}{0.9833}(1+0.0167)=\frac{1.0167}{0.9833}\cdot91\approx94\]
  So, the earth is at about $94$ million miles away from the sun at its
  aphelion position.

\item Consider the following parabola:
  \[y^2-6y-12x+33=0\]
  \begin{enumerate}
  \item What is the vertex?
    \begin{eqnarray*}
      y^2-6y &=& 12x-33 \\
      y^2-6y+9 &=& 12x-33+9 \\
      (y-3)^2 &=& 12x-24 \\
      (y-3)^2 &=& 12(x-2)
    \end{eqnarray*}

    $V(2,3)$
    
  \item What is the axis of symmetry?

    $y=3$
    
  \item What is the focus?

    $4p=12$ \\
    $p=3$

    This is a horizontal/open right parabola, so the focus is at:

    $(2+p,3)=(2+3,3)=(5,3)$

    $F(5,3)$
    
  \item What is the directrix?

    $x=2-p=2-3=-1$

    $x=-1$
    
  \item What is the focal diameter?

    $\abs{4p}=12$
    
  \item Sketch the parabola, labeling all of the above items.

    \begin{tikzpicture}
      \draw (-2,0) -- (10,0);
      \draw (0,-5) -- (0,10);
      \draw [dashed] (-2,3) -- (10,3);
      \draw [dashed] (-1,-5) -- (-1,10);
      \node [below left] at (-1,0) {$-2$};
      \node [below left] at (0,3) {$3$};
      \node [draw,circle,fill,scale=0.5] (v) at (2,3) {};
      \node [above left] at (v) {$(2,3)$};
      \node [draw,circle,fill,scale=0.5] (f) at (5,3) {};
      \node [above] at (f) {$(5,3)$};
      \draw [smooth,domain=2:6] plot ({\x},{sqrt(12*(\x-2))+3});
      \draw [smooth,domain=2:6] plot ({\x},{-sqrt(12*(\x-2))+3});
   \end{tikzpicture}
    
  \end{enumerate}

\item Consider the following ellipse:
  \[4x^2+25y^2-50y-75=0\]
  \begin{enumerate}
  \item What is the center?
    \begin{eqnarray*}
      4x^2+25(y^2-2y) &=& 75 \\
      4x^2+25(y^2-2y+1) &=& 75+25 \\
      4x^2+25(y-1)^2 &=& 100 \\
      \frac{x^2}{25}+\frac{(y-1)^2}{4} &=& 1
    \end{eqnarray*}
    $C(0,1)$
    
  \item What is the length of the major axis?

    $a=5$

    Length of major axis $=2a=2(5)=10$
    
  \item What is the length of the minor axis?

    $b=2$
    
    Length of minor axis $=2b=2(2)=4$

  \item What are the four vertices?

    $(0+a,1)=(0+5,1)=(5,1)$ \\
    $(0-a,1)=(0-5,1)=(-5,1)$ \\
    $(0,1+b)=(0,1+2)=(0,3)$ \\
    $(0,1-b)=(0,1-2)=(0,-1)$
    
  \item What are the two foci?

    $c=\sqrt{a^2-b^2}=\sqrt{25-4}=\sqrt{21}$

    The major axis is parallel to the $x$-axis, so the foci are also on the
    $x$-axis:

    $(0+c,1)=(0+\sqrt{21},1)=(\sqrt{21},1)$ \\
    $(0-c,1)=(0-\sqrt{21},1)=(-\sqrt{21},1)$

  \item What is the eccentricity?

    $e=\frac{c}{a}=\frac{\sqrt{21}}{5}\approx0.92$
    
  \item Sketch the ellipse, labeling all of the above items.

    \begin{tikzpicture}
      \draw (-6,0) -- (6,0);
      \draw (0,-5) -- (0,5);
      \node [draw,circle,fill,scale=0.5] (c) at (0,1) {};
      \node [above right] at (c) {$(0,1)$};
      \draw (0,1) ellipse (5 and 2);
      \draw [dashed] (-5,1) -- (5,1);
      \node [draw,circle,fill,scale=0.5] (a1) at (5,1) {};
      \node [right] at (a1) {$(5,1)$};
      \node [draw,circle,fill,scale=0.5] (a2) at (-5,1) {};
      \node [left] at (a2) {$(-5,1)$};
      \node [draw,circle,fill,scale=0.5] (b1) at (0,3) {};
      \node [above right] at (b1) {$(0,3)$};
      \node [draw,circle,fill,scale=0.5] (b2) at (0,-1) {};
      \node [below right] at (b2) {$(0,-1)$};
      \node [draw,circle,fill,scale=0.5] (f1) at (4,1) {};
      \node [above] at (f1) {$(\sqrt{21},1)$};
      \node [draw,circle,fill,scale=0.5] (f2) at (-4,1) {};
      \node [above] at (f2) {$(-\sqrt{21},1)$};
    \end{tikzpicture}
  \end{enumerate}
\newpage
\item Consider the following hyperbola:
  \[36x^2+72x-4y^2+32y+116=0\]
  \begin{enumerate}
  \item What is the center?
    \begin{eqnarray*}
      36(x^2+2x)-4(y^2-8y) &=& -116 \\
      36(x^2+2x+1)-4(y^2-8y+16) &=& -116+36-64 \\
      36(x+1)^2-4(y-4)^2 &=& -144 \\
      \frac{(y-4)^2}{36}-\frac{(x+1)^2}{4} &=& 1
    \end{eqnarray*}
    $C(-1,4)$
    
  \item What is the length of the horizontal axis?

    $b=2$

    Length of horizontal axis $=2b=2(2)=4$
    
  \item What is the length of the vertical axis?

    $a=6$
    
    Length of vertical axis $=2a=2(6)=12$

  \item What are the two vertices?

    This is a vertically-oriented hyperbola, so the vertices (and foci) are
    on the major axis parallel to the $y$-axis:

    $(-1,4+a)=(-1,4+6)=(-1,10)$ \\
    $(-1,4-a)=(-1,4-6)=(-1,-2)$

  \item What are the two foci?

    $c=\sqrt{a^2+b^2}=\sqrt{36+4}=\sqrt{40}=2\sqrt{10}\approx6.3$

    $(-1,4+c)=(-1,4+2\sqrt{10})\approx(-1,10.3)$ \\
    $(-1,4-c)=(-1,4-2\sqrt{10})\approx(-1,-2.3)$
    
  \item What are the two asymptotes?

    $y-4=\pm\frac{a}{b}(x+1)=\pm\frac{6}{2}(x+1)=\pm3(x+1)$
    
  \item Sketch the hyperbola, labeling all of the above items.

    \begin{tikzpicture}[scale=0.5]
      \draw (-10,0) -- (10,0);
      \draw (0,-10) -- (0,20);
      \node [draw,circle,fill,scale=0.5] (c) at (-1,4) {};
      \node [left] at (c) {$(-1,4)$};
      \draw [dashed] (-3,-2) rectangle (1,10);
      \draw [dashed,domain=-5.5:4.5] plot ({\x},{3*(\x+1)+4});
      \draw [dashed,domain=-6.25:3.5] plot ({\x},{-3*(\x+1)+4});
      \node [left] at (-3,-2) {$(-3,-2)$};
      \node [left] at (-3,10) {$(-3,10)$};
      \node [right] at (1,10) {$(1,10)$};
      \node [right] at (1,-2) {$(1,-2)$};
      \node [draw,circle,fill,scale=0.5] (v1) at (-1,10) {};
      \node [above] at (v1) {$(-1,10)$};
      \node [draw,circle,fill,scale=0.5] (v2) at (-1,-2) {};
      \node [below] at (v2) {$(-1,-2)$};
      \node [draw,circle,fill,scale=0.5] (f1) at (-1,12) {};
      \node [above] at (f1) {$(-1,4+2\sqrt{10})$};
      \node [draw,circle,fill,scale=0.5] (f2) at (-1,-4) {};
      \node [below] at (f2) {$(-1,4-2\sqrt{10})$};
      \draw [smooth,domain=-6.25:4.5] plot ({\x},{4+6*sqrt((\x+1)^2/4+1)});
      \draw [smooth,domain=-5.5:3.5] plot ({\x},{4-6*sqrt((\x+1)^2/4+1)});
    \end{tikzpicture}
    
  \end{enumerate}
\end{enumerate}

\end{document}
