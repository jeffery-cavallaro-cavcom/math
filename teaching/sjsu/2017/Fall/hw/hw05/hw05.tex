\documentclass[letterpaper,12pt,fleqn]{article}
\usepackage{matharticle}
\usepackage{mathtools}
\pagestyle{plain}
\begin{document}

\begin{center}
\Large Math-19 Homework \#5
\end{center}

\vspace{0.5in}

\underline{Problems}

\begin{enumerate}

\item Solve each of the following for $x$. Express your answers both
  graphically and in interval notation:
  \begin{enumerate}
  \item $x^2+9x-36=0$
  \item $x^2+9x-36<0$
  \item $x^2+9x-36\ge0$
  \end{enumerate}
  
\item Solve each of the following for $x$. Express your answers both
  graphically and in interval notation:
  \begin{enumerate}
  \item $\frac{x+3}{x-1}=0$
  \item $\frac{x+3}{x-1}>0$
  \item $\frac{x+3}{x-1}\le0$
  \end{enumerate}
  
\item Solve each of the following for $x$. Express your answers both
  graphically and in interval notation:
  \begin{enumerate}
  \item $3\abs{x-1}+1=7$
  \item $3\abs{x-1}+1\le7$
  \item $3\abs{x-1}+1\ge7$
  \end{enumerate}

\item Find the domain of the following expressions. Express your answers both
  graphically and in interval notation:
  \begin{enumerate}
    \item $\sqrt{\frac{x^2+2x-3}{x^2+5x+6}}$
    \item $\sqrt[3]{\frac{x^2+2x-3}{x^2+5x+6}}$
  \end{enumerate}

\item Muri is a shopkeeper that specializes in pickled vegetables. She has
  determined over the years that the best brine (salt solution) for pickling
  vegetables is 2 kg of salt per liter of water (2 kg/L).  One day, she has her
  not-so-bright nephew helping her and he uses too much salt, resulting in
  a 5 kg/L solution.  If her nephew made up 10 liters of the too-salty solution,
  how much pure water must he add to it to get the ideal 2 kg/L solution? For
  full credit, show the mixture equation and the appropriate values for each
  concentration and volume value in the equation.
\end{enumerate}

\end{document}
