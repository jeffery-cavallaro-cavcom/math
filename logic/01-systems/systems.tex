\documentclass[letterpaper,12pt,fleqn]{article}
\usepackage{matharticle}
\usepackage{graphtheory}
\pagestyle{empty}
\begin{document}
\section*{Mathematical Systems}

Question: Does mathematics represent some absolute truth about the universe?

It seems to.  We use mathematics to solve problems in science, technology, business, medicine, and practically every
area of life.  But to answer this question properly we need to define what is meant by a
\emph{mathematical system}.

\begin{definition}[Mathematical System]
  A \emph{mathematical system} is composed of a set of objects, a set of \emph{definitions} that describe the
  nature of the objects, a set of inherent operations (\emph{axioms}) that can be performed on the objects, and new
  true propositions about the objects that are proved with logical \emph{arguments} using the definitions, axioms,
  and previously proved propositions (\emph{theorems}).
\end{definition}

\bigskip

\begin{center}
  \begin{tikzpicture}[
      block node/.style={draw,rectangle,minimum width=1in,minimum height=0.5in},
      node distance=0.75in
    ]
    \node (o) [draw,circle,minimum size=0.5in] at (0,0) {objects};
    \node (a) [block node,right=of o] {axioms};
    \node (d) [block node,above=of a] {definitions};
    \node (t) [block node,below=of a] {theorems};
    \node (A) [block node,right=of a] {argument};
    \node (p) [block node,above=of A] {proposition};
    \draw (o) -- (d.south west);
    \draw (o) -- (a.west);
    \draw (o) -- (t.north west);
    \draw [->] (d.south east) -- (A.north west);
    \draw [->] (a.east) -- (A.west);
    \draw [->] (t.north east) -- (A.south west);
    \draw [->] (p) -- (A);
    \draw [->] (A) |- (t);
  \end{tikzpicture}
\end{center}

\bigskip

So a mathematical system expresses relative truth based on the selected definitions and axioms.  Thus, we attempt
to construct the definitions and axioms to reflect perceived reality.

\begin{example}[The Real Number System]
  The real number system is constructed as follows:
  \begin{enumerate}
    \item Specify the objects (real numbers) by definitions for natural numbers, zero, integers,
      rational numbers, and irrational numbers.
    \item Add a definition for the notion of equality.
    \item Add definitions for the binary operations of addition and multiplication.
    \item Add axioms for precedence rules (multiplication before addition).
    \item Add the so-called \emph{field} axioms:

      For all real numbers \(a\), \(b\), and \(c\):
      \begin{description}
      \item[Additive Commutativity:] \(a+b=b+a\)
      \item[Multiplicative Commutativity:] \(ab=ba\)
      \item[Additive Associativity:] \((a+b)+c=a+(b+c)\)
      \item[Multiplicative Associativity:] \((ab)c=a(bc)\)
      \item[Additive Identity:] There exists a real number \(0\) such that for every real number \(a\):
        \[0+a=a\]
      \item[Multiplicative Identity:] There exists a real number \(1\) such that for every real number \(a\):
        \[1a=a\]
      \item[Additive Inverse:] For every real number \(a\) there exists a real number \(-a\) such that
        \[a+(-a)=0\]
      \item[Multiplicative Inverse:] For every non-zero real number \(a\) there exists a real number \(\frac{1}{a}\)
        such that:
        \[a\left(\frac{1}{a}\right)=1\]
      \item[Distributivity:] \(a(b+c)=ab+ac\)
      \end{description}
    \item Extend the system with theorems.  For example, consider the proposition: \((b+c)a=ba+ca\):

      \begin{tabular}{ll}
        \((b+c)a=a(b+c)\) & Multiplicative Commutativity \\
        \((b+c)a=ab+ac\) & Distributivity \\
        \((b+c)a=ba+ca\) & Multiplicative Commutativity \\
      \end{tabular}
  \end{enumerate}
\end{example}

Arguments are made using \(Boolean\) logic, which is itself a mathematical system.  An argument starts with the
statement of a \emph{theorem} and is accepted as either true or false based on a \emph{proof}.

\end{document}
