\documentclass[letterpaper,12pt,fleqn]{article}
\usepackage{matharticle}
\pagestyle{empty}
\renewcommand{\implies}{\rightarrow}
\renewcommand{\iff}{\leftrightarrow}
\newcommand{\lxor}{\oplus}
\begin{document}
\section*{Implication}

A mathematical system is extended by collecting a set of facts called \emph{preconditions} and showing that if the
preconditions hold (are true) then some other fact must also be true.  This is called \emph{implication}.  This is
expressed using an if-then construct.

\begin{examples}
  If \(x=2\) then \(x^2=4\).

  If \(x\) is an even number then \(x^2\) is an even number.
\end{examples}

\begin{definition}[Implication]
  Let \(p\) and \(q\) be propositions.  The \emph{conditional} statement: ``if \(p\) then \(q\),'' also called
  \emph{implication} and denoted by \(p\implies q\), is the proposition that is false when \(p\) is true and
  \(q\) is false and true otherwise.  \(p\) is called the \emph{hypothesis} or \emph{antecedent} and \(q\) is
  called the \emph{conclusion} or \emph{consequence}.

  \begin{center}
    \begin{tabular}{|cc|c|}
      \hline
      \(p\) & \(q\) & \(p\implies q\) \\
      \hline
      \(F\) & \(F\) & \(T\) \\
      \hline
      \(F\) & \(T\) & \(T\) \\
      \hline
      \(T\) & \(F\) & \(F\) \\
      \hline
      \(T\) & \(T\) & \(T\) \\
      \hline
    \end{tabular}
  \end{center}

  Note that \(p\implies q=\lnot p\lor q=\bar{p}+q\).
\end{definition}

Implications can be stated in multiple ways:
\begin{itemize}
\item If \(p\) then \(q\).
\item \(p\) implies \(q\).
\item \(p\) is sufficient for \(q\).
\item \(p\) only if \(q\).
\item \(q\) is necessary for \(p\).
\item \(q\) when \(p\).
\item \(q\) unless \(\lnot p\).
\end{itemize}

\begin{example}
  If \(x=2\) then \(x^2=4\).

  If \(x^2=4\) then we cannot conclude that \(x=2\) because \(x=\pm2\).
\end{example}

If the hypothesis is false, then the truth value of the consequence doesn't matter.

\begin{example}
  If you are in this class then you are under 7 feet tall.

  If the hypothesis is false, meaning a person is not in this class, then that person may or may not be 7 feet
  tall.
\end{example}

Note that conditionals in logic are different from if statements in computer languages, which act as guards to
blocks of statements.

\begin{definition}[Implication Forms]
  Let \(p\) and \(q\) be propositions:
  \begin{itemize}
  \item Implication: \(p\implies q\)
  \item Inverse: \(\lnot p\implies \lnot q\)
  \item Converse: \(q\implies p\)
  \item Contrapositive \(\lnot q\implies \lnot p\)
  \end{itemize}
\end{definition}

From the truth tables: the implication is equal to the contrapositive and the inverse is equal to the converse.

\begin{minipage}{2in}
  \centering
  \begin{tabular}{cc|c}
    \(p\) & \(q\) & \(\lnot p\implies \lnot q\) \\
    \hline
    \(F\) & \(F\) & \(T\) \\
    \(F\) & \(T\) & \(F\) \\
    \(T\) & \(F\) & \(T\) \\
    \(T\) & \(T\) & \(T\) \\
  \end{tabular}
\end{minipage}
\begin{minipage}{2in}
  \centering
  \begin{tabular}{cc|c}
    \(p\) & \(q\) & \(q\implies p\) \\
    \hline
    \(F\) & \(F\) & \(T\) \\
    \(F\) & \(T\) & \(F\) \\
    \(T\) & \(F\) & \(T\) \\
    \(T\) & \(T\) & \(T\) \\
  \end{tabular}
\end{minipage}
\begin{minipage}{2in}
  \centering
  \begin{tabular}{cc|c}
    \(p\) & \(q\) & \(\lnot q\implies \lnot p\) \\
    \hline
    \(F\) & \(F\) & \(T\) \\
    \(F\) & \(T\) & \(T\) \\
    \(T\) & \(F\) & \(F\) \\
    \(T\) & \(T\) & \(T\) \\
  \end{tabular}
\end{minipage}

Conditionals are unidirectional.  A bidirectional implication is an equivalence.

\begin{definition}[Equivalence]
  Let \(p\) and \(q\) be propositions.  The \emph{biconditional} statement: ``\(p\) if and only if \(q\),'' also
  called \emph{equivalence} and denoted by \(p\iff q\) or \(p\ \text{iff}\ q\), is the proposition that is true
  when \(p\) and \(q\) have the same truth value and false otherwise.

  \begin{center}
    \begin{tabular}{|cc|c|}
      \hline
      \(p\) & \(q\) & \(p\iff q\) \\
      \hline
      \(F\) & \(F\) & \(T\) \\
      \hline
      \(F\) & \(T\) & \(F\) \\
      \hline
      \(T\) & \(F\) & \(F\) \\
      \hline
      \(T\) & \(T\) & \(T\) \\
      \hline
    \end{tabular}
  \end{center}

  Note that \(p\iff q=(p\implies q)\land(q\implies p)=\lnot(p\lxor q)\).
\end{definition}

The complete preference table is as follows:

\begin{center}
  \begin{tabular}{|c|}
    \hline
    \(\lnot\) \\
    \hline
    \(\land\) \\
    \hline
    \(\lor\) \\
    \hline
    \(\implies\) \\
    \hline
    \(\iff\),\(\lxor\) \\
    \hline
  \end{tabular}
\end{center}

\end{document}
