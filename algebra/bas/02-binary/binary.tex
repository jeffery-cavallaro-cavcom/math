\documentclass[letterpaper,12pt,fleqn]{article}
\usepackage{matharticle}
\pagestyle{empty}
\begin{document}
\section*{Binary Operators}

\begin{definition}
  A \emph{binary operator} `$*$' on a non-empty set $S$ is a function
  $*:S\times S\to S$, where $*(a,b)$ is typically denoted by $a*b$, or even
  $ab$ (juxtaposition) when there is no ambiguity.
\end{definition}

Thus, a binary operator `$*$' on a set $S$ must be:
\begin{enumerate}
\item Closed: $\forall\,a,b\in S,a*b\in S$.
\item Well-defined: $\forall\,a,b,c,d\in S,a*b=c$ and $a*b=d\implies c=d$.
\end{enumerate}

\begin{definition}
  Let `$*$' be a binary operator on a set $S$ and let $H\subset S$. To say that
  `$*$' is an \emph{induced} operation on $H$ means that $H$ is closed under
  `$*$': $\forall\,a,b\in H,a*b\in H$.
\end{definition}

To count the number of possible operators for a set $S$, consider the
following:
\[S=\{a,b\}\]

\begin{tabular}{c|cc}
  $*$ & a & b \\
  \hline
  a & aa & ab \\
  b & ba & bb \\
\end{tabular}\hspace{0.25in}
\begin{tabular}{c|cc}
  $*$ & a & b \\
  \hline
  a & 2 & 2 \\
  b & 2 & 2 \\
\end{tabular}\hspace{0.25in}
$2^4=2^{2^2}=16$ possibilities

\[S=\{a,b,c\}\]
\begin{tabular}{c|ccc}
  $*$ & a & b & c \\
  \hline
  a & aa & ab & ac \\
  b & ba & bb & bc \\
  c & ca & cb & cc \\
\end{tabular}\hspace{0.25in}
\begin{tabular}{c|ccc}
  $*$ & a & b & c\\
  \hline
  a & 3 & 3 & 3 \\
  b & 3 & 3 & 3 \\
  c & 3 & 3 & 3 \\
\end{tabular}\hspace{0.25in}
$3^9=3^{3^2}$ possibilities

\[S=\{a,b,c,d\}\]
\begin{tabular}{c|cccc}
  $*$ & a & b & c & d \\
  \hline
  a & aa & ab & ac & ad \\
  b & ba & bb & bc & bd \\
  c & ca & cb & cc & cd \\
  d & da & db & dc & dd \\
\end{tabular}\hspace{0.25in}
\begin{tabular}{c|cccc}
  $*$ & a & b & c & d \\
  \hline
  a & 4 & 4 & 4 & 4 \\
  b & 4 & 4 & 4 & 4 \\
  c & 4 & 4 & 4 & 4 \\
  d & 4 & 4 & 4 & 4 \\
\end{tabular}\hspace{0.25in}
$4^{16}=4^{4^2}$ possibilities

\bigskip
  
In general, for $\abs{S}=n$, there are $n^{n^2}$ possible operations.
\newpage
\begin{definition}
  To say that a binary operator `$*$' on a set $S$ is \emph{commutative} means:
  \[\forall\,a,b\in S,a*b=b*a\]
\end{definition}

The table for a commutative binary operator must be symmetric:

\begin{tabular}{c|cc}
  $*$ & a & b \\
  \hline
  a & 2 & 2 \\
  b & 1 & 2 \\
\end{tabular}\hspace{0.25in}
$2^3=2^{\left(\frac{2\cdot3}{2}\right)}=8$ possibilities

\begin{tabular}{c|ccc}
  $*$ & a & b & c\\
  \hline
  a & 3 & 3 & 3 \\
  b & 1 & 3 & 3 \\
  c & 1 & 1 & 3 \\
\end{tabular}\hspace{0.25in}
$3^6=3^{\left(\frac{3\cdot4}{2}\right)}$ possibilities

\begin{tabular}{c|cccc}
  $*$ & a & b & c & d \\
  \hline
  a & 4 & 4 & 4 & 4 \\
  b & 1 & 4 & 4 & 4 \\
  c & 1 & 1 & 4 & 4 \\
  d & 1 & 1 & 1 & 4 \\
\end{tabular}\hspace{0.25in}
$4^{10}=4^{\left(\frac{4\cdot5}{2}\right)}$ possibilities

\bigskip
  
In general, for $\abs{S}=n$, there are $n^{\left[\frac{n(n+1)}{2}\right]}$ possible
commutative operations.

\begin{definition}
  To say that a binary operator `$*$' on a set $S$ has an \emph{identity}
  element $e$ means:
  \[\exists\,e\in S,\forall\,a\in S,e*a=a*e=a\]
\end{definition}

\begin{tabular}{c|cc}
  $*$ & e & a \\
  \hline
  e & e & a \\
  a & a & 2 \\
\end{tabular}\hspace{0.25in}
$2=2^{(2-1)^2}$ possibilities

\begin{tabular}{c|ccc}
  $*$ & e & a & b \\
  \hline
  e & e & a & b \\
  a & a & 3 & 3 \\
  b & b & 3 & 3 \\
\end{tabular}\hspace{0.25in}
$3^4=3^{(3-1)^2}=81$ possibilities

\begin{tabular}{c|cccc}
  $*$ & e & a & b & c \\
  \hline
  e & e & a & b & c \\
  a & a & 4 & 4 & 4 \\
  b & b & 4 & 4 & 4 \\
  c & c & 4 & 4 & 4 \\
\end{tabular}\hspace{0.25in}
$4^9=4^{(4-1)^2}$ possibilities

\bigskip
  
In general, for $\abs{S}=n$, there are $n^{(n-1)^2}$ possible operations when
there is an identity element.

Combining cummutativity and identity:

\begin{tabular}{c|cc}
  $*$ & e & a \\
  \hline
  e & e & a \\
  a & a & 2 \\
\end{tabular}\hspace{0.25in}
$2=2^{\left(\frac{2\cdot1}{2}\right)}$ possibilities

\begin{tabular}{c|ccc}
  $*$ & e & a & b \\
  \hline
  e & e & a & b \\
  a & a & 3 & 3 \\
  b & b & 1 & 3 \\
\end{tabular}\hspace{0.25in}
$3^3=3^{\left(\frac{3\cdot2}{2}\right)}=27$ possibilities

\begin{tabular}{c|cccc}
  $*$ & e & a & b & c \\
  \hline
  e & e & a & b & c \\
  a & a & 4 & 4 & 4 \\
  b & b & 1 & 4 & 4 \\
  c & c & 1 & 1 & 4 \\
\end{tabular}\hspace{0.25in}
$4^6=4^{\left(\frac{4\cdot3}{2}\right)}$ possibilities

\bigskip
  
In general, for $\abs{S}=n$, there are $n^{\left[\frac{n(n-1)}{2}\right]}$ possible
communtative operations when there is an identity element.

\begin{definition}
  To say that a binary operator `$*$' on a set $S$ is \emph{associative} means:
  \[\forall\,a,b,c\in S,(a*b)*c=a*(b*c)\]
\end{definition}

Determining associativity is a bit more tedius:

\begin{example}
  Let $S=\{e,a\}$.  The two possible operations are:

\begin{tabular}{c|cc}
  $*$ & e & a \\
  \hline
  e & e & a \\
  a & a & e \\
\end{tabular}\hspace{0.25in}
\begin{tabular}{c|cc}
  $\cdot$ & e & a \\
  \hline
  e & e & a \\
  a & a & a \\
\end{tabular}

\begin{tabular}{ccc|cc|cc}
  a & b & c & $(a*b)*c$ & $a*(b*c)$ &
      $(a\cdot b)\cdot c$ & $a\cdot(b\cdot c)$ \\
  \hline
  e & e & e & e & e & e & e \\
  e & e & a & a & a & a & a \\
  e & a & e & a & a & a & a \\
  e & a & a & e & e & a & a \\
  a & e & e & a & a & a & a \\
  a & e & a & e & e & a & a \\
  a & a & e & e & e & a & a \\
  a & a & a & a & a & a & a \\
\end{tabular}
\end{example}

So an operator on a set with identity is always associative.

\begin{theorem}
  Composition is associative.
\end{theorem}

\begin{theproof}
  Assume that $f,g,h$ are binary operators on a set $S$. \\
  Assume $x\in S$.
  \[[(f\circ g)\circ h](x)=(f\circ g)(h(x))=f(g(h(x)))\]
  \[[f\circ(g\circ h)](x)=f((g\circ h)(x))=f(g(h(x)))\]
\end{theproof}

However, composition is not necessarily commutative.

\begin{example}
  Let $S=\{a,b\}$ and define the following functions:

  \begin{tabular}{c|c|c|c|c}
      & E & A & B & C \\
    \hline
    a & a & a & b & b \\
    b & b & a & b & a \\
  \end{tabular}

  \bigskip

  \begin{tabular}{c|cccc}
    $\circ$ & E & A & B & C \\
    \hline
    E & E & A & B & C \\
    A & A & A & A & A \\
    B & B & B & B & B \\
    C & C & B & A & E \\
  \end{tabular}

  \bigskip

  The table is not symmetric, and thus the composition is not commutative.
\end{example}
\end{document}
