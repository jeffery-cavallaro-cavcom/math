\documentclass[letterpaper,12pt,fleqn]{article}
\usepackage{matharticle}
\usepackage{mathrsfs}
\pagestyle{empty}
\newcommand{\bas}[2]{\left<#1,#2\right>}
\newcommand{\g}{\gamma}
\renewcommand{\o}{\theta}
\newcommand{\p}{\phi}
\newcommand{\z}{\zeta}
\renewcommand{\S}{\mathscr{S}}
\begin{document}
\section*{Binary Algebraic Structures}

\begin{definition}
  A \emph{binary algebraic structure} is a non-empty set $S$ equipped with a
  binary operator `$*$' and is denoted $\bas{S}{*}$.
\end{definition}

When the binary operation is understood, the structure is simply referred to
as $S$ and operations are written using the more convenient $ab$ form
(juxtaposition) instead of $a*b$.

\begin{definition}
Let $\bas{S}{*}$ and $\bas{T}{*'}$ be two binary algebraic structures and
$\p:S\to T$. To say that $S$ is \emph{homomorphic} to $T$ means there exists
$\p:S\to T$ such that:
\[\forall\,a,b\in S,\p(a*b)=\p(a)*'\p(b)\]

If such a $\p$ exists then it is referred to as a \emph{homomorphism}.

When the binary operations are understood, the statement of homomorphism is
written using the shorted $\p(ab)=\p(a)\p(b)$ form. Note that the $ab$
operation takes place is structure $S$ with its equipped binary operator, and
the $\p(a)\p(b)$ operation takes place in structure $T$ with its equipped
binary operator.
\end{definition}

\begin{definition}
  Let $\bas{S}{*}$ and $\bas{T}{*'}$ be two binary algebraic structures. To say
  that $S$ is \emph{isomorphic} to $T$, denoted $S\simeq T$, means there exists
  $\p:S\to T$ such that:
  \begin{enumerate}
  \item $\p$ is a bijection
  \item $\p$ is a homomorphism
  \end{enumerate}
  If such a $\p$ exists then it is referred to as an \emph{isomorphism}.
\end{definition}

\begin{example}
  It was previously shown that $\bas{U}{\cdot}\simeq\bas{\R_{2\pi}}{+_{2\pi}}$. In
  particular:
  \[\p\left(e^{i\o}\right)=\o\]
  \begin{eqnarray*}
    \p(u_1u_2) &=& \p\left(e^{i\o_1}e^{1\o_2}\right) \\
    &=& \p\left(e^{i(\o_1+_{2\pi}\,\o_2)}\right) \\
    &=& \o_1+_{2\pi}\,\o_2 \\
    &=& \p\left(e^{i\o_1}\right)+_{2\pi}\,\p\left(e^{i\o_2}\right) \\
    &=& \p(u_1)+_{2\pi}\,\p(u_2) \\
  \end{eqnarray*}

  Similarly, $\bas{U_n}{\cdot}\simeq\bas{\Z_n}{+_n}$:
  \[\p\left(e^{i\left[\frac{2\pi k}{n}\right]}\right)=k\]
  \begin{eqnarray*}
    \p(\z^h\z^k) &=&
    \p\left(e^{i\left[\frac{2\pi h}{n}\right]}e^{i\left[\frac{2\pi k}{n}\right]}\right) \\
    &=& \p\left(e^{i\left[\frac{2\pi(h+_n\,k}{n}\right]}\right) \\
    &=& h+_n\,k \\
    &=& \p\left(e^{i\left[\frac{2\pi h}{n}\right]}\right)+_n\,
        \p\left(e^{i\left[\frac{2\pi k}{n}\right]}\right) \\
    &=& \p(\z^h)+_n\,\p(\z^k) \\
  \end{eqnarray*}

  Let $n=4$:

  \bigskip

\begin{minipage}{2.5in}
  \begin{tabular}{c|cccc}
    $U_4$ & 1 & i & -1 & -i \\
    \hline
    1 & 1 & i & -1 & -i \\
    i & i & -1 & -i & i \\
    -1 & -1 & -i & 1 & i \\
    -i & -i & 1 & i & -1 \\
  \end{tabular}
  \end{minipage}
  \begin{minipage}{2.5in}
  \begin{tabular}{c|cccc}
    $\Z_4$ & 0 & 1 & 2 & 3 \\
    \hline
    0 & 0 & 1 & 2 & 3 \\
    1 & 1 & 2 & 3 & 0 \\
    2 & 2 & 3 & 0 & 1 \\
    3 & 3 & 0 & 1 & 2 \\
  \end{tabular}
\end{minipage}

\bigskip

\begin{tabular}{c|c}
$\z$ & $\p(\z)$ \\
\hline
1 & 0 \\
i & 1 \\
-1 & 2 \\
-i & 3 \\
\end{tabular}

\[\p((-1)(-i))=\p(i)=1\]
\[\p((-1)(-i))=\p(-1)+_4\,\p(-i))=2+_4\,3=1\]
\end{example}

\begin{example}
Prove: $\bas{\R}{+}\simeq\bas{\R^+}{\cdot}$

Let $\p:\R\to\R^+$ be defined by $\p(x)=e^x$

\bigskip

\begin{minipage}{2.5in}
\underline{one-to-one}

Assume $\p(x)=\p(y)$ \\
$e^x=e^y$ \\
$x=y$ \\
$\therefore \p$ is one-to-one
\end{minipage}
\begin{minipage}{2.5in}
\underline{onto}

Assume $y\in\R^+$ \\
Let $x=\ln{y}\in\R$ \\
$e^x=y$ \\
$\therefore \p$ is onto
\end{minipage}

\bigskip

\underline{homo}

Assume $x,y\in\R^+$ \\
$\p(x+y)=e^{x+y}=e^xe^y=\p(x)\p(y)$ \\
$\therefore \p$ is a homomorphism

\bigskip

Thus, $\p$ is an isomorphism and therefore
$\bas{\R}{+}\simeq\bas{\R^+}{\cdot}$
\end{example}

\begin{theorem}
Let $S$ and $T$ be binary algebraic structures.
\[\p:S\to T\ \mbox{is an isomorphism}\iff
    \p^{-1}:T\to S\ \mbox{is an isomorphism}\]
\end{theorem}

\begin{theproof}
\listbreak
\begin{description}
\item{$\implies$}: Assume $\p:S\to T$ is an isomorphism

$\p$ is a bijection \\
$\p^{-1}$ also exists and is a bijection \\
Assume $t_1,t_2\in T$ \\
$\p$ is onto \\
$\exists\,s_1,s_2\in S,\p(s_1)=t_1$ and $\p(s_2)=t_2$ \\
$\p$ is a homomorphism \\
$\p^{-1}(t_1t_2)=\p^{-1}(\p(s_1)\p(s_2))=\p^{-1}(\p(s_1s_2))=
    (\p^{-1}\p)(s_1s_2)=s_1s_2=\p^{-1}(t_1)\p^{-1}(t_2)$ \\
$\p^{-1}$ is a homomorphism

$\therefore \p^{-1}:T\to S$ is an isomorphism \\

\item{$\impliedby$}: Assume $\p^{-1}:T\to S$ is an isomorphism

$\p^{-1}$ is a bijection \\
$\p$ also exists and is a bijection \\
Assume $s_1,s_2\in S$ \\
$\p^{-1}$ is onto \\
$\exists\,t_1,t_2\in T,\p^{-1}(t_1)=s_1$ and $\p^{-1}(t_2)=s_2$ \\
$\p^{-1}$ is a homomorphism \\
$\p(s_1s_2)=\p(\p^{-1}(t_1)\p^{-1}(t_2))=\p(\p^{-1}(t_1t_2))=
    (\p\p^{-1})(t_1t_2)=t_1t_2=\p(s_1)\p(s_2)$ \\
$\p$ is a homomorphism

$\therefore \p:S\to T$ is an isomorphism \\
\end{description}
\end{theproof}

\begin{theorem}
Let $S,T,U$ be binary algebraic structures such that $\p:S\to T$ is an
isomorphism and $\g:T\to U$ is an isomorphism.
\[\g\p:S\to U\ \mbox{is an isomorphism}\]
\end{theorem}

\begin{theproof}
$\p$ and $\g$ are bijections and homomorphisms \\
$\g\p$ is a bijection \\
Assume $s_1,s_2\in S$ \\
$(\g\p)(s_1s_2)=\g(\p(s_1s_2))=\g(\p(s_1)\p(s_2))=\g(\p(s_1))\g(\p(s_2))=
(\g\p)(s_1)(\g\p)(s_2)$ \\
$\g\p$ is a homomorphism \\
$\therefore \g\p$ is an isomorphism
\end{theproof}

\begin{theorem}
  Let $\S$ be the set of all binary algebraic structures. \\
  Isomorphism is an equivalence relation on $\S$.
\end{theorem}

\begin{theproof}
  \listbreak
  \begin{itemize}
  \item{Reflexive}

    Assume $S\in\S$ \\
    The identity function $i_S$ is clearly bijective and homomorphic \\
    $i_S:S\to S$ is an isomorphism \\
    $\therefore S\simeq S$

  \item{Symmetric}

    Assume $S\simeq T$ \\
    There exists isomorphism $\p:S\to T$ \\
    So there exists isomorphism $\p^{-1}:T\to S$ \\
    $\therefore T\simeq S$

  \item{Transitive}

    Assume $S\simeq T$ and $T\simeq U$ \\
    There exists isomorphisms $\p\to T$ and $\g:T\to U$ \\
    So $\g\p:S\to U$ is an isomorphism \\
    $\therefore S\simeq U$
  \end{itemize}
\end{theproof}

\end{document}
