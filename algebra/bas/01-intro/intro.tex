\documentclass[letterpaper,12pt,fleqn]{article}
\usepackage{matharticle}
\pagestyle{empty}
\renewcommand{\o}{\theta}
\newcommand{\z}{\zeta}
\begin{document}
\section*{Abstract Algebra}
Abstract algebra is concerned with the structure imposed upon sets by one or
more binary operators, and which such sets have the same structure.

\begin{definition}
  Let $c\in\R, c>0$:
  \[\R_c=[0,c)=\{x\in\R\mid0\le x<c\}\]
\end{definition}

\begin{definition}
  Let $a,b\in\R_c$. Addition of $a$ and $b$ modulo $c$, denoted $a+_c b$, is
  given by:
  \[a+_c b=\begin{cases}
  a+b, & a+b\in\R_c \\
  a+b-c, & a+b\notin\R_c \\
  \end{cases}\]
\end{definition}

\begin{example}
  $\R_1=[0,1)$:
  \[0.5+_10.25=0.75\]
  \[0.5+_10.5=1-1=0\]
  \[0.5+_10.75=1.25-1=0.25\]
\end{example}

The angular arithmetic in polar and complex exponential forms usually takes
place in $\R_{2\pi}$, although other intervals such as $(-\pi,\pi]$ are
sometimes used.

Let $U$ be the locus of points on the unit circle:
\[U=\{z\in\C\mid\abs{z}=1\]

$U$ has the following properties:
\begin{enumerate}
\item $U$ is \emph{closed} under multiplication:
  \[\forall\,z_1,z_2\in U, z_1z_2\in U\]
  Let $z_1=e^{i\o_1}$ and $z_2=e^{i\o_2}$:
  \[z_1z_2=e^{i\o_1}e^{i\o_2}=e^{i(\o_1+_{2\pi}\,\o_2)}\in U\]

\item There exists a unique element $e^{i0}=1\in U$ such that
  \[\forall\,z\in U,1z=z1=z\]
  Such an element is called the identity element.

\item There exists a bijection $\phi:U\to\R_{2\pi}$ defined by:
  \[\phi(z)=\phi(e^{i\o})=\o\in\R_{2\pi}\]

\item The bijection $\phi$ is also a \emph{homomorphism}:
  \[\phi(z_1z_2)=\phi\left(e^{i(\o_1+_{2\pi}\,\o_2)}\right)=\o_1+_{2\pi}\,\o_2=
  \phi(z_1)+_{2\pi}\,\phi(z_2)\]

\item The equation $z\cdot z\cdot z\cdot z=1$ in $U$ has four solutions:
  $1,-1,i,-i$. Thus, the equation: $x+_{2\pi}\,x+_{2\pi}\,x+_{2\pi}\,x=0$ in
  $\R_{2\pi}$ has four corresponding solutions:
  $0,\frac{\pi}{2},\pi,\frac{3\pi}{2}$.
\end{enumerate}

A bijection that is also a homomorphism is called an \emph{isomorphism}.
Isomorphisms indicate that two sets have the same structure, although the names
of the elements may be different.

\begin{definition}
  \listbreak
  \[\Z_n=\{m\in\Z\mid0\le m<n\}=\{0,1,2,\ldots,n-1\}\]
\end{definition}

\begin{definition}
  The $n^{th}$ roots of unity, denoted $U_n$, are given by:
  \[U_n=\{z\in\C\mid z^n=1\}\]

  The $k^{th}$ root, denoted $\z^k$, is given by:
  \[\z^k=e^{i\frac{2\pi k}{n}}\]
\end{definition}

Note that $U_n\subset U$, where the members of $U_n$ start at $(1,0)$ and are
equally spaced by $\frac{2\pi}{n}$. This results in a total of $n-1$ unique
roots, corresponding to $0\le k<n$.

Similarly, $U_n$ has the following properties:
\begin{enumerate}
\item $U_n$ is closed under multiplication:
  \[\forall\,\z^h,\z^k\in U_n, \z^h\z^k\in U_n\]
  Let $\z^h=e^{i\frac{2\pi h}{n}}$ and $\z^k=e^{i\frac{2\pi k}{n}}$:
  \[\z^h\z^k=e^{i\frac{2\pi h}{n}}e^{i\frac{2\pi k}{n}}=
      e^{i\left[\frac{2\pi(h+_n\,k)}{n}\right]}\in U_n\]

\item There exists a unique element $\z^0=e^{i0}=1\in U_n$ such that
  \[\forall\,\z^k\in U_n,\z^0\z^k=\z^k\z^0=\z^k\]
  Such an element is called the identity element.

\item There exists a bijection $\phi:U_n\to\Z_n$ defined by:
  \[\phi(\z^k)=\phi(e^{i\frac{2\pi k}{n}})=k\]

\item The bijection $\phi$ is also a \emph{homomorphism}:
  \[\phi(\z^h\z^k)=\phi\left(e^{i\left[\frac{2\pi(h+_n\,k)}{n}\right]}\right)=
  h+_n\,k=\phi(h)+_n\,\phi(k)\]
\end{enumerate}

\begin{example}
  Find all solutions to the equation:
  \[x+_8\,x+_8\,x+_8\,x+_8\,x+_8\,x+_8\,x+_8\,x=0\]

  Since $\Z_8$ is isomorphic to $U_8$, consider the equation:
  \[z\cdot z\cdot z\cdot z\cdot z\cdot z\cdot z\cdot z=z^8=1\]
  The solutions to this equation are the $8^{th}$ roots of unity, which
  correspond to the eight solutions: $0,1,2,3,4,5,6,7$ in $\Z_8$.
\end{example}
\end{document}
