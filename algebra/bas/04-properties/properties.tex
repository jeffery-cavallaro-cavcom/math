\documentclass[letterpaper,12pt,fleqn]{article}
\usepackage{matharticle}
\pagestyle{empty}
\newcommand{\p}{\phi}
\newcommand{\bas}[2]{\left<#1,#2\right>}
\begin{document}
\section*{Structural Properties}

\begin{definition}
  A \emph{structural property} is a property that must be shared by any two
  isomorphic binary algebraic structures.

  In fact, to show that two structures are not isomorphic, show that there is a
  property held by one but not the other.
\end{definition}

\begin{properties}
  \listbreak
  \begin{enumerate}
  \item Cardinality
  \item Commutativity
  \item Associativity
  \item Identity Element
  \item Inverse Elements
  \item Equation Solutions
  \item Idempotent Elements
  \end{enumerate}
\end{properties}

\begin{theorem}
  Let $S,T$ be binary algebraic structures.
  \[S\simeq T\implies\abs{S}=\abs{T}\]
\end{theorem}

\begin{theproof}
  Assume $S\simeq T$ \\
  There exists isomorphism $\p:S\to T$ \\
  $\p$ is a bijection \\
  $\therefore \abs{S}=\abs{T}$
\end{theproof}

\begin{theorem}
  Let $S,T$ be binary algebraic structures.
  \[\p:S\to T\ \mbox{is an isomorphism}\implies
      \left(\forall\,x,y\in S, xy=yx\implies\p(x)\p(y)=\p(y)\p(x)\right)\]
\end{theorem}

\begin{theproof}
  Assume $\p:S\to T$ is an isomorphism \\
  Assume $x,y\in S$ \\
  Assume $xy=yx$ \\
  $\p(xy)=\p(x)\p(y)$ \\
  $\p(xy)=\p(yx)=\p(y)\p(x)$ \\
  $\therefore \p(x)\p(y)=\p(y)\p(x)$
\end{theproof}

\begin{example}
  $\bas{\R}{\cdot}\not\simeq\bas{M(\R)}{\cdot}$ because $\bas{\R}{\cdot}$ is
  commutative; however, $\bas{M(\R)}{\cdot}$ is not.
\end{example}

\begin{theorem}
  Let $S,T$ be binary algebraic structures.
  
  $\p:S\to T\ \mbox{is an isomorphism}\implies$
  \[\left(\forall\,x,y,z\in S, (xy)z=x(yz)\implies
  [\p(x)\p(y)]\p(z)=\p(x)[\p(y)\p(z)]\right)\]
\end{theorem}

\begin{theproof}
  Assume $\p:S\to T$ is an isomorphism \\
  Assume $x,y,z\in S$ \\
  Assume $x(yz)=x(yz)$ \\
  $\p((xy)z)=\p(xy)\p(z)=[\p(x)\p(y)]\p(z)$ \\
  $\p((xy)z)=\p(x(yz))\p(x)\p(yz)=\p(x)[\p(y)\p(z)]$ \\
  $\therefore [\p(x)\p(y)]p(z)=\p(x)[\p(y)\p(z)]$
\end{theproof}

\begin{definition}
  Let $S$ be a binary algebraic structure. To say that $e\in S$ is an
  \emph{identity} element for $S$ means:
  \[\forall\,a\in S,ea=ae=a\]
\end{definition}

\begin{theorem}
  A binary algebraic structure has at most one identity element.
\end{theorem}

\begin{theproof}
  Assume $S$ is a binary algebraic structure \\
  Assume $e_1,e_2\in S$ are identity elements for $S$ \\
  $e_1e_2=e_1$ \\
  $e_1e_2=e_2$ \\
  $\therefore e_1=e_2$
\end{theproof}

\begin{theorem}
  Let $S$ and $T$ be binary algebraic structures
  
  $\p:S\to T$ is an isomorphism $\implies$
  \[\left(e\ \mbox{is an identity element for}\ S\implies
  \p(e)\ \mbox{is an identity element for}\ T\right)\]
\end{theorem}
\newpage
\begin{theproof}
  Assume $\p:S\to T$ is an isomorphism \\
  Assume $e$ is an identity element for $S$ \\
  $\forall\,s\in S,es=se=s$ \\
  $\p$ is well-defined \\
  $\exists\,\p(e)\in T$ \\
  Assume $t\in T$ \\
  $\p$ is onto \\
  $\exists\,s\in S,\p(s)=t$ \\
  $\p$ is a homomorphism \\
  $\p(es)=\p(e)\p(s)=\p(e)t$ \\
  $\p(es)=\p(se)=\p(s)\p(e)=t\p(e)$ \\
  $\p(es)=\p(s)=t$ \\
  $\p(e)t=t\p(e)=t$ \\
  $\therefore \p(e)$ is an identity for $T$
\end{theproof}

\begin{example}
  \[S={a,b,c,d}\]

  \begin{tabular}{c|cccc}
    $*$ & a & b & c & d \\
    \hline
    a & a & c & b & d \\
    b & c & d & a & c \\
    c & b & a & c & a \\
    d & d & c & a & b \\
  \end{tabular}

  \bigskip

  $S\not\simeq \Z_4$ because $Z_4$ has an identity (0); however, $S$ does not
  have an identity element.
\end{example}

\begin{definition}
  Let $S$ be a binary algebraic structure with identity $e$ and let $a\in S$.
  To say that $b\in S$ is an inverse for $a$ means:
  \[ab=ba=e\]
\end{definition}

\begin{notation}
  \listbreak
  \begin{description}
  \item{Additive:} $b=-a$
  \item{Multiplicative:} $b=a^{-1}$
  \end{description}
\end{notation}

\begin{theorem}
  Let $S$ and $T$ be binary algebraic structures such that $e$ is an identity
  element for $S$.
  
  $\p:S\to T$ is an isomorphism $\implies$
  \[\left(\forall\,a\in S,b\in S\ \mbox{is an inverse for}\ a\implies
  \p(b)\ \mbox{is an inverse for}\ \p(a)\right)\]
\end{theorem}

\begin{theproof}
  Assume $\p:S\to T$ is an isomorphism \\
  Assume $a\in S$ \\
  Assume $b\in S$ is an inverse for $a$ \\
  $ab=ba=e$ \\
  $\p$ is well-defined \\
  $\exists\,\p(a)\in T$ and $\exists\,\p(b)\in T$ \\
  $\p(e)$ is an identity for $T$ \\
  $\p$ is a homomorphism \\
  $\p(ab)=\p(a)\p(b)$ \\
  $\p(ab)=\p(ba)=\p(b)\p(a)$ \\
  $\p(ab)=\p(e)$ \\
  $\p(a)\p(b)=\p(b)\p(a)=\p(e)$ \\
  $\therefore \p(b)$ is an inverse for $\p(a)$
\end{theproof}

\begin{theorem}
  Let $S$ and $T$ be binary algebraic structures.
  \[\p:S\to T\ \mbox{is an isomorphism}\implies
    \left(\forall\,x,a\in S,xx=a\implies\p(x)\p(x)=\p(a)\right)\]
\end{theorem}

\begin{theproof}
  Assume $\p:S\to T$ is an isomorphism \\
  Assume $x,a\in S$ \\
  Assume $xx=a$ \\
  $\p$ is well-defined \\
  $\exists\,\p(x),\p(a)\in T$ \\
  $\p$ is a homomorphism \\
  $\p(xx)=\p(x)\p(x)$ \\
  $\p(xx)=\p(a)$ \\
  $\therefore \p(x)\p(x)=\p(a)$ \\
\end{theproof}

\begin{example}
  $\bas{\C}{\cdot}\not\simeq\bas{\R}{\cdot}$ because $\forall\,c\in\C$ the
  equation $zz=c$ has a solution in $\C$; however, $xx=-1$ has no solutions in
  $\R$.

  ABC: $\exists\,\p:\bas{\C}{\cdot}\to\bas{\R}{\cdot}$
  
  $-1\in\R$ \\
  $\p$ is onto \\
  $\exists\,c\in\C,\p(c)=-1$ \\
  $zz=c$ has a solution in $\C$ \\
  $\p$ is well-defined \\
  $\exists\,x\in\R,\p(z)=x$ \\
  $\p$ is a homomorphism \\
  $\p(zz)=\p(z)\p(z)=xx=x^2$ \\
  $\p(zz)=\p(c)=-1$ \\
  $x^2=-1$ \\
  Contradiction! \\
  $z$ has no image under $\p$ \\
  $\p$ is not well-defined \\
  no such $\p$ exists \\
  $\therefore \bas{\C}{\cdot}\not\simeq\bas{\R}{\cdot}$
\end{example}

\begin{example}
  $\bas{\Z}{\cdot}\not\simeq\bas{\Z}{+}$ because $nn=n$ has two solutions in
  $\bas{\Z}{\cdot}$; however, $m+m=m$ has only one solution in $\bas{\Z}{+}$.

  ABC: $\exists\,\p:\bas{\Z}{\cdot}\to\bas{\Z}{+}$

  Assume $n\in\bas{\Z}{\cdot}$ \\
  Assume $nn=n$ \\
  $n^2-n=0$ \\
  $n(n-1)=0$ \\
  $n=0,1$

  $\p$ is well-defined \\
  $\exists\,m\in\bas{\Z}{+},\p(n)=m$ \\
  $\p$ is a homomorphism \\
  $\p(nn)=\p(n)+\p(n)=m+m$ \\
  $\p(nn)=\p(n)=m$ \\
  $m+m=m$ \\
  $2m=m$ \\
  $m=0$

  $\p(0)=0$ and $\p(1)=0$ \\
  $\p$ is not one-to-one \\
  Contradiction! \\
  No such $\p$ exists \\
  $\therefore \bas{\Z}{\cdot}\not\simeq\bas{\Z}{+}$
\end{example}

\begin{definition}
  Let $S$ be a binary algebraic structure. To say that $a\in S$ is
  \emph{idempotent} in $S$ means $aa=a$.
\end{definition}

\begin{theorem}
  Let $S$ and $T$ be binary algebraic structures.
  
  $\p:S\to T\ \mbox{is an isomorphism}\implies$
  \[\left(\forall\,a\in S,a\ \mbox{is idempotent in}\ S\implies
    \p(a)\ \mbox{is idempotent in}\ T\right)\]
\end{theorem}

\begin{theproof}
  Assume $\p:S\to T$ is an isomorphism \\
  Assume $a\in S$ \\
  Assume $a$ is idempotent in $S$ \\
  $aa=a$ \\
  By previous theorem, $\p(a)\p(a)=\p(a)$ \\
  $\therefore \p(a)$ is idempotent in $T$
\end{theproof}

\end{document}
