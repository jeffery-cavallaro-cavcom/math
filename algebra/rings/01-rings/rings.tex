\documentclass[letterpaper,12pt,fleqn]{article}
\usepackage{matharticle}
\pagestyle{empty}
\newcommand{\ring}[1]{\left<#1,+,\cdot\right>}
\newcommand{\group}[2]{\left<#1,#2\right>}
\begin{document}
\section*{Rings}

\begin{definition}
  Let $R$ be a non-empty set equipped with two binary operators: addition and
  multiplication. To say that $\ring{R}$ is a \emph{ring} means:
  \begin{enumerate}
  \item $\group{R}{+}$ is an abelian group
  \item Multiplication is associative
  \item The left and right distributive rules hold: $\forall\,a,b,c\in R$:
    \[a(b+c)=ab+ac\]
    \[(a+b)c=ac+bc\]
  \end{enumerate}
\end{definition}

\begin{example}
  The following are all rings:
  \begin{enumerate}
  \item The trivial ring: $\{0\}$
  \item $\Z, \Q, \R, \C$
  \item $n\Z$
  \item $\Z_n$
  \item $M_n(R), R$ is a ring
  \item A direct product of rings: $\prod_{i\in I}R_i$, with component-wise
    operators
  \item The set $F$ of real-valued functions such that:
    \begin{enumerate}
    \item $(f+g)(x)=f(x)+g(x)$
    \item $(fg)(x)=f(x)g(x)$
    \end{enumerate}
  \end{enumerate}
\end{example}

\begin{theorem}
  Let $R$ be a ring. $\forall\,a,b\in R$:
  \begin{enumerate}
  \item $a0=0a=0$
  \item $(-a)b=a(-b)=-(ab)$
  \item $(-a)(-b)=ab$
  \end{enumerate}
\end{theorem}
\newpage
\begin{theproof}
  \listbreak
  \begin{enumerate}
  \item Assume $a\in R$

    $a0=a(0+0)=a0+a0$ \\
    $\therefore a0=0$ (cancellation)

    $0a=(0+0)a=0a+0a$ \\
    $\therefore 0a=0$ (cancellation)

  \item Assume $a,b\in R$

    $(-a)b+ab=(-a+a)b=0b=0$ \\
    So $(-a)b$ is an inverse of $ab$ \\
    But inverses are unique \\
    $\therefore (-a)b=-(ab)$

    $a(-b)+ab=a(-b+b)=a0=0$ \\
    So $a(-b)$ is an inverse of $ab$ \\
    But inverses are unique \\
    $\therefore a(-b)=-(ab)$

  \item Assume $a,b\in R$

    $(-a)(-b)=a[-(-b)]=ab$
  \end{enumerate}
\end{theproof}

\begin{notation}
  Let $R$ be a ring, $a\in R$, and $n\in\Z$:
  \[n\cdot a=\begin{cases}
  a+a+\cdots+a & n>0 \\
  (-a)+(-a)+\cdots+(-a) & n<0 \\
  0 & n=0
  \end{cases}\]

  This helps distinguish these cases from multiplication between two elements of $R$.
\end{notation}

\end{document}
