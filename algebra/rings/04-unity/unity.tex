\documentclass[letterpaper,12pt,fleqn]{article}
\usepackage{matharticle}
\pagestyle{empty}
\begin{document}
\section*{Unity}

\begin{definition}
  To say that $R$ is a \emph{ring with unity} means:
  \begin{enumerate}
  \item $R$ is a ring
  \item $R$ has a multiplicative identity element, called \emph{unity}, and
    denoted by $1$.
  \end{enumerate}
\end{definition}

Note that since $\left<R,\cdot\right>$ is a binary algebraic structure, $1\in R$ is
unique.

\begin{theorem}
  Let $R$ be a ring with unity. $0=1$ iff $R$ is the trivial ring.
\end{theorem}

\begin{theproof}
  \listbreak
  \begin{description}
  \item $\implies$ Assume $0=1$
    
    Assume $a\in R$ \\
    $a0=0$ \\
    $a1=a0=a$ \\
    $\therefore a=0$

  \item $\impliedby$ Assume $R$ is the trivial ring

    $00=0$ \\
    So, $0$ is unity for $R$ \\
    $R$ is a ring with unity, \\
    But $\abs{R}=1$ \\
    $\therefore 0=1$
  \end{description}
\end{theproof}

\begin{theorem}
  Let $r,s\in\N$ such that $(r,s)=1$:
  \[Z_{rs}\simeq\Z_r\times\Z_s\]
\end{theorem}

\begin{theproof}
  $\Z_r$ and $\Z_s$ are cyclic with generator $1$ \\
  $\Z_{rs}$ is cyclic with generator $(1,1)$ \\
  Let $\phi:\Z_{rs}\to\Z_r\times\Z_s$ be defined by $\phi(n)=n\cdot(1,1)$

  Assume $\phi(n)=\phi(m)$ \\
  $n\cdot(1,1)=m\cdot(1,1)$ \\
  But additon in $Z_{rs}$ is well-defined \\
  $n=m$ \\
  $\therefore\phi$ is one-to-one.
\newpage
  Assume $x\in\Z_{rs}$ \\
  $\exists n\in\N,n\cdot(1,1)=x$ \\
  $\phi(n)=n\cdot(1,1)=x$ \\
  $\therefore\phi$ is onto, and thus a bijection.

  $\phi(n+m)=(n+m)\cdot(1,1)=n\cdot(1,1)+m\cdot(1,1)=\phi(n)+\phi(m)$

  $\phi(nm)=(nm)\cdot(1,1)=[n\cdot(1,1)][m\cdot(1,1)]=\phi(n)\phi(m)$

  $\therefore\phi$ is a ring homomorphism, and thus a ring isomorphism.
\end{theproof}

\end{document}
