\documentclass[letterpaper,12pt,fleqn]{article}
\usepackage{matharticle}
\pagestyle{empty}
\begin{document}
\section*{Fields}

\begin{definition}
  To say that $R$ is a \emph{commutative ring} means:
  \begin{enumerate}
  \item $R$ is a ring
  \item Multiplication in $R$ is commutative
  \end{enumerate}
\end{definition}

Note that the binomial theorem holds for any commutative ring $R$:

$\forall\,a,b\in R,\forall\,n\in\N$:
\[(a+b)^n=\sum_{k=0}^n\binom{n}{k}\cdot a^{n-k}b^k\]

\begin{definition}
  To say that $R$ is a \emph{division ring (skew field)} means:
  \begin{enumerate}
  \item $R$ is a ring with unity $1\ne0$
  \item $\left<R-\{0\},\cdot\right>$ is a group
  \end{enumerate}
  In other words, every non-zero element of $R$ is a unit.
\end{definition}

\begin{definition}
  To say that $R$ is a \emph{field} means:
  \begin{enumerate}
  \item $R$ is a ring with unity $1\ne0$
  \item $\left<R-\{0\},\cdot\right>$ is an abelian group
  \end{enumerate}
  In other words, $R$ is a commutative division ring (skew field).

  If $R$ is a non-commutative division ring then it is called a
  \emph{strictly skew field}.
\end{definition}

\begin{example}
  $\Z$ is not a field, because $2\in\Z$ but $2$ has no multiplicative inverse.

  $\Q$, $\R$, and $\C$ are fields.
\end{example}

\end{document}
