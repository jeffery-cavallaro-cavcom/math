\documentclass[letterpaper,12pt,fleqn]{article}
\usepackage{matharticle}
\pagestyle{empty}
\begin{document}
\section*{Zero Divisors}

\begin{definition}
  Let $R$ be a ring and $a\in R$ such that $a\ne0$:
  \begin{itemize}
  \item To say that $a$ is a \emph{left zero divisor} of $R$ means:
    \[\exists\,b\in R,b\ne0\ \mbox{and}\ ab=0\]

  \item To say that $a$ is a \emph{right zero divisor} of $R$ means:
    \[\exists\,b\in R,b\ne0\ \mbox{and}\ ba=0\]

  \item To say that $a$ is a \emph{zero divider} of $R$ means that $a$ is a
    left or a right zero divisor of $R$.

  \item To say that $a$ is a \emph{two-sided zero divider} of $R$ means that
    $a$ is a left and a right zero divisor of $R$.
  \end{itemize}
\end{definition}

Note that for a two-sided zero divisor: $ax=ya=0$, where $x$ need not equal
$y$, unless $R$ is commutative.

\begin{example}
  $\Z_{12}$

  \begin{tabular}{cc}
    \textbf{0-divisors} & \textbf{units} \\
    $2\cdot6=0$ & $1\cdot1=1$ \\
    $3\cdot4=0$ & $5\cdot5=1$ \\
    $8\cdot3=0$ & $7\cdot7=1$ \\
    $9\cdot4=0$ & $11\cdot11=1$ \\
    $10\cdot6=0$ & \\
  \end{tabular}
\end{example}

\begin{theorem}
  $z\in\Z_n$ is a zero divisor $\iff (z,n)\ne1$
\end{theorem}

\begin{theproof}
  \listbreak
  \begin{description}
    \begin{minipage}[t]{3in}
    \item $\implies$ Assume $(z,n)=1$

      Assume $\exists\,s\in\Z_n,zs=0$ \\
      $n\mid zs$ \\
      But $n\nmid z$, so $n\mid s$ \\
      Thus, $s=0$ \\
      $\therefore z$ is not a zero divisor.
    \end{minipage}
    \begin{minipage}[t]{3in}
    \item $\impliedby$ Assume $(u,n)\ne1$

      Let $(z,n)=d>1$ \\
      $d\mid z$ and $d\mid n$ \\
      $z\frac{n}{d}=n\frac{z}{d}=0$ \\
      But $z,\frac{n}{d}\ne0$ \\
      $\therefore z$ is a zero divisor.
    \end{minipage}
  \end{description}
\end{theproof}
\newpage
\begin{corollary}
  $\forall\,a\in\Z_n$, exactly one of the following is true:
  \begin{enumerate}
  \item $a=0$
  \item $a$ is a unit
  \item $a$ is a zero divisor
  \end{enumerate}
\end{corollary}

\begin{corollary}
  $p$ prime $\implies \Z_p$ has no zero divisors.
\end{corollary}

\begin{theorem}
  Let $R$ be a ring. The cancellation laws hold in $R$ iff $R$ has no zero
  divisors.
\end{theorem}

\begin{theproof}
  \listbreak
  \begin{description}
  \item $\implies$ Assume the cancellation laws hold in $R$

    Assume $a,b\in R,ab=0$
    \begin{description}
      \begin{minipage}[t]{3in}
      \item Case 1: $a\ne0$

        a0=0 \\
        ab=a0 \\
        b=0

        $\therefore a$ is not a zero divisor.
      \end{minipage}
      \begin{minipage}[t]{3in}
      \item Case 2: $b\ne0$

        0b=0 \\
        ab=0b \\
        a=0

        $\therefore b$ is not a zero divisor.
      \end{minipage}
    \end{description}

    $\therefore R$ has no zero divisors.
    
  \item $\impliedby$ Assume $R$ has no zero divisors

    Assume $a,b,c\in R$ such that $a\ne0$ and $ab=ac$ \\
    $ab-ac=0$ \\
    $a(b-c)=0$ \\
    Since $a\ne0$ and $R$ has no zero divisors, $b-c=0$ \\
    $b=c$

    $\therefore$ the left cancellation law holds in $R$.

    Assume $a,b,c\in R$ such that $a\ne0$ and $ba=ca$ \\
    $ba-ca=0$ \\
    $(b-c)a=0$ \\
    Since $a\ne0$ and $R$ has no zero divisors, $b-c=0$ \\
    $b=c$

    $\therefore$ the right cancellation law holds in $R$.

    $\therefore$ the cancellation laws hold in $R$.
  \end{description}
\end{theproof}
\newpage
\begin{theorem}
  Let $R$ be ring with no zero divisors. $\forall\,a,b\in R$, the equations $ax=b$ and
  $xa=b$ each have at most one solution.
\end{theorem}

\begin{theproof}
  Assume $a,b\in R$

  Assume $ax=b$ has two solutions: $x_1$ and $x_2$ \\
  $ax_1=ax_2$ \\
  But the left cancellation law holds in $R$ \\
  $\therefore x_1=x_2$.

  Assume $xa=b$ has two solutions: $x_1$ and $x_2$ \\
  $x_1a=x_2a$ \\
  But the right cancellation law holds in $R$ \\
  $\therefore x_1=x_2$.
\end{theproof}

Note that when $R$ is a ring with unity $1\ne0$ and $a$ is a unit in $R$ then
the unique solution to $ax=b$ is given by $x=a^{-1}b$. Likewise, the unique
solution to $xa=b$ is $x=ba^{1}$. If $R$ is commutative then these two
solutions are the same.

\begin{notation}
  Let $F$ be a field. $\forall\,a,b\in F$, since $a^{-1}b=ba^{-1}$, this
  element in $F$ is denoted by $\frac{b}{a}$.
\end{notation}

\end{document}
