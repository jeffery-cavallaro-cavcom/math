\documentclass[letterpaper,12pt,fleqn]{article}
\usepackage{matharticle}
\pagestyle{empty}
\newcommand{\ring}[1]{\left<#1,+,\cdot\right>}
\newcommand{\group}[2]{\left<#1,#2\right>}
\begin{document}
\section*{Ring Homomorphisms}

\begin{definition}
  Let $R$ and $R'$ be rings. To say that $\phi:R\to R'$ is a
  \emph{ring homomorphism} means $\forall\,a,b\in R$:
  \begin{enumerate}
  \item $\phi(a+b)=\phi(a)+\phi(b)$
  \item $\phi(ab)=\phi(a)\phi(b)$
  \end{enumerate}
\end{definition}

\begin{theorem}
  Let $F$ be the set of real-values functions and let $\phi:F\to\R$ be defined
  by $\phi_a(f)=f(a)$. $\phi_a$ is a ring homomorphism, referred to as the
  \emph{evaluation homomorphism}.
\end{theorem}

\begin{theproof}
  Assume $f,g\in F$ \\
  Assume $a\in\R$

  $\phi_a(f+g)=(f+g)(a)=f(a)+g(a)=\phi_a(f)+\phi_a(g)$

  $\phi_a(fg)=(fg)(a)=f(a)g(a)=\phi_a(f)\phi_a(g)$
\end{theproof}

\begin{theorem}
  Let $\phi:\Z\to\Z_n$ be defined by $\phi(a)=a\bmod n$. $\phi$ is a ring
  homomorphism.
\end{theorem}

\begin{theproof}
  Assume $a,b\in\Z$ \\
  $a=nq_1+r_1$ \\
  $b=nq_2+r_2$

  $\phi(a)=r_1$ \\
  $\phi(b)=r_2$

  $\phi(a+b)=\phi(n(q_1+q_2)+(r_1+r_2))=0+(r_1+r_2)\bmod n=\phi(a)+\phi(b)$

  $\phi(ab)=\phi(n^2q_1q_2+nq_1r_2+nq_2r_1+r_1r_2)=0+0+0+r_1r_2\bmod n=
  \phi(a)\phi(b)$
\end{theproof}

\end{document}
