\documentclass[letterpaper,12pt,fleqn]{article}
\usepackage{matharticle}
\pagestyle{empty}
\newcommand{\ide}{\trianglelefteq}
\begin{document}
\section*{Ideals}

\begin{definition}
  Let $R$ be a ring and $I$ an additive subgroup of $R$:
  \begin{itemize}
  \item To say that $I$ is a \emph{left ideal} in $R$ means:
    \[\forall\,r\in R,\forall\,i\in I,ri\in I\]
  \item To say that $I$ is a \emph{right ideal} in $R$ means:
    \[\forall\,r\in R,\forall\,i\in I,ir\in I\]
  \item To say that $I$ is a (two-sided) \emph{ideal} in $R$, denoted $I\ide R$, means
    that $I$ is both a left ideal and a right ideal in $R$.
  \end{itemize}
\end{definition}

\begin{definition}
  $I=\{0\}$ is called the \emph{zero ideal}.
\end{definition}    

\begin{theorem}
  Let $\phi:R\to S$ be a homomorphism of rings:
  \[\ker(\phi)\ide R\]
\end{theorem}

\begin{theproof}
  $\ker(\phi)$ is an additive subgroup of R \\
  Assume $r\in R$ \\
  Assume $k\in \ker(\phi)$ \\
  $\phi(rk)=\phi(r)\phi(k)=\phi(r)\cdot0=0$ \\
  $rk\in\ker(\phi)$, so $\ker(\phi)$ is a left ideal in $R$ \\
  $\phi(kr)=\phi(k)\phi(r)=0\cdot\phi(r)=0$ \\
  $kr\in\ker(\phi)$, so $\ker(\phi)$ is a right ideal in $R$

  $\therefore\ker(\phi)\ide R$.
\end{theproof}

\begin{theorem}
  Let $R$ be a ring and $I$ be an ideal in $R$:
  \[I\le R\]
\end{theorem}

\begin{theproof}
  By definition, $I$ is an additive subgroup of $R$ \\
  Assume $r,s\in I$ \\
  By definition, $rs\in I$

  Therefore, by the subring test, $I\le R$.
\end{theproof}

\begin{theorem}[Ideal Test]
  Let $R$ be a ring and $I$ a non-empty subset of $R$. $I\ide R$ iff
  \begin{enumerate}
  \item $\forall\,x,y\in I,x-y\in I$
  \item $\forall\,x\in I,\forall\,z\in R, zx\in I$ and $xz\in I$
  \end{enumerate}
\end{theorem}

\begin{theproof}
  Assume $x,y\in I$

  \begin{description}
  \item $\implies$ Assume $I\ide R$

    $I$ is an additive subgroup of $R$, so $(-y)\in I$ \\
    By closure, $x-y\in I$

    Assume $z\in R$ \\
    $I$ is a left ideal, so $zx\in I$ \\
    $I$ is a right ideal, so $xz\in I$

    Therefore, the two conditions hold.

  \item $\impliedby$ Assume the two conditions hold

    $x,y\in R$ \\
    Thus $xy\in I$ \\
    So by the subring test, $I\le R$ \\
    But $I$ is both a right and left ideal

    $\therefore I\ide R$
  \end{description}
\end{theproof}

\begin{theorem}
  Let $R$ be a ring and $\{I_a\mid a\in A\}$ be a family of ideals in $R$:
  \[I=\bigcap_{a\in A}I_a\ide R\]
\end{theorem}

\begin{theproof}
  $I\le R$ \\
  Assume $x\in I$ and $z\in R$ \\
  Assume $a\in A$ \\
  $x\in I_a$ \\
  But $I_a\ide R$, so $zx\in I_a$ and $xz\in I_a$ \\
  $zx\in I$ and $xz\in I$

  Therefore, by the ideal test, $I\ide R$.
\end{theproof}
\newpage
\begin{theorem}
  $\forall\,n\in\Z,n\Z\ide Z$
\end{theorem}

\begin{theproof}
  Assume $n\in\Z$
  
  \begin{description}
  \item Case 1: $n=0$
    
    $0\Z=\{0\}\ide\Z$

  \item Case 2: $n>0$

    Assume $m\in n\Z$ \\
    $\exists\,k\in\Z,m=kn$ \\
    Assume $h\in=Z$ \\
    $hm=h(kn)=(hk)n\in n\Z$, so $nZ$ is a left ideal in $\Z$ \\
    $mh=(kn)h=(kh)n\in n\Z$, so $nZ$ is a right ideal in $\Z$

    $\therefore n\Z\ide\Z$

  \item Case 3: $n<0$

    $(-n)>0$

    Assume $m\in n\Z$ \\
    $\exists\,k\in\Z,m=kn$ \\
    $m=kn=(-k)(-n)$ \\
    $m\in (-n)\Z$

    Assume $m\in(-n)\Z$ \\
    $\exists\,k\in\Z,m=k(-n)$ \\
    $m=k(-n)=(-k)n$ \\
    $m\in n\Z$

    Thus $n\Z=(-n)\Z$

    $\therefore n\Z\ide Z$
  \end{description}
\end{theproof}

\begin{theorem}
  $I\ide\Z\implies\exists\,n\in\Z,I=n\Z$
\end{theorem}

\begin{theproof}
  \listbreak
  \begin{description}
  \item Case 1: $I=\{0\}$

    $I=0Z$

  \item Case 2: $\exists i\in I,i\ne0$

    $I$ is an additive group, so $(-i)\in I$ \\
    Thus $I$ contains a least positive element \\
    Let the least positive element be $n$ \\
    Assume $m\in I$ \\
    By the DA, $m=qn+r$, where $0\le r<n$ \\
    $r=m-qn$ \\
    But $I$ is an ideal, so $qn\in I$ \\
    So by closure, $r\in I$ \\
    But by the minimality of $n$, $r=0$ \\
    $m=qn$

    $\therefore I=n\Z$

  \end{description}
\end{theproof}

\begin{definition}
  Let $R$ be a ring, $r\in R$, and $(r)=\{rs\mid s\in R\}\ide R$. $(r)$ is called a
  \emph{principal ideal} in $R$.
\end{definition}

\begin{definition}
  Let $R$ be a ring and $S$ be a non-empty subset of $R$. The ideal:
  \[\bigcup_{\tiny{\begin{array}{c}I\ide R \\ S\subseteq I\end{array}}}I\]
  is called the ideal in $R$ generated by $S$.
\end{definition}

\end{document}
