\documentclass[letterpaper,12pt,fleqn]{article}
\usepackage{matharticle}
\pagestyle{empty}
\begin{document}
\section*{Nilpotence}

\begin{definition}
  Let $R$ be a ring and $a\in R$. To say that $a$ is \emph{nilpotent} in $R$
  means: $\exists\,n\in\Z^+,a^n=0$.
\end{definition}

\begin{theorem}
  Let $R$ be a commutative ring and $N=\{a\in R\mid a$ is nilpotent in $R\}$. \\
  $N$ is closed under addition.
\end{theorem}

\begin{theproof}
  Assume $a,b\in N$ \\
  $\exists\,n\in\Z^+,a^n=0$ \\
  $\exists\,m\in\Z^+,b^m=0$ \\
  $(a+b)^{n+m}=\sum_{k=0}^{n+m}\binom{n+m}{k}\cdot a^{n+m-k}b^k$ \\
  For $0\le k\le m, a^{n+m-k}=a^na^r=0a^r=0$ \\
  For $m\le k\le n+m, b^k=b^mb^s=0b^s=0$ \\
  $(a+b)^{n+m}=0$ \\
  $a+b\in N$

  $\therefore N$ is closed under addition.
\end{theproof}

\begin{theorem}
  Let $\phi:R\to R'$ be a homomorphism of rings:

  $a$ nilpotent in $R\implies\phi(a)$ nilpotent in $R'$
\end{theorem}

\begin{theproof}
  Assume $a$ is nilpotent in $R$ \\
  $\exists\,n\in\Z^+,a^n=0$ \\
  $\phi(a^n)=\phi(0)=0'$ \\
  $\phi(a^n)=\phi(a)^n$ \\
  $\phi(a)^n=0'$

  $\therefore\phi(a)$ is nilpotent in $R'$.
\end{theproof}

\begin{theorem}
  Let $R$ be a ring:

  $R$ has no non-zero nilpotent elements $\iff \left(x^2=0\iff x=0\right)$.
\end{theorem}
\newpage
\begin{theproof}
  \listbreak
  \begin{description}
  \item $\implies$ Assume $R$ has no non-zero nilpotent elements
    \begin{description}
    \item $\implies$ Assume $x\ne0$

      $x^2\ne0$, otherwise $x$ would be nilpotent (contradiction)

    \item $\impliedby$ Assume $x=0$

      $x^2=0^2=(0)(0)=0$
    \end{description}
    
    $\therefore x^2=0\iff x=0$

  \item $\impliedby$ Assume $x^2=0\iff x=0$

    ABC: $x\ne0$ is nilpotent in $R$ \\
    Let $n\in\Z^+$ be the smallest $n$ such that $x^n=0$
    \begin{description}
    \item Case 1: $n$ even

      $\left(x^{\frac{n}{2}}\right)^2=0$ \\
      But by minimality of $n$, $x^{\frac{n}{2}}\ne0$ \\
      Contradiction!
      
    \item Case 2: $n$ odd
      \begin{description}
      \item Case A: $n=1$

        $x=0$ \\
        Contradiction!

      \item Case B: $n>1$
        
        $\left(x^{\frac{n+1}{2}}\right)^2=0$ \\
        But by minimality of $n$, $x^{\frac{n+1}{2}}\ne0$ \\
        Contradiction!
      \end{description}
    \end{description}

    So $x$ is not nilpotent in $R$ \\
    $\therefore R$ contains no non-zero nilpotent elements.
  \end{description}
\end{theproof}

\end{document}
