\documentclass[letterpaper,12pt,fleqn]{article}
\usepackage{matharticle}
\usepackage{tikz}
\pagestyle{empty}
\newcommand{\p}{\phi}
\begin{document}
\section*{Subgroups}

\begin{definition}
  To say that a set $H$ is a \emph{subgroup} of a group $G$, denoted $H\le G$,
  means:
  \begin{enumerate}
  \item{$H\subseteq G$}
  \item{$H$ is a group using the induced operation of $G$}
  \end{enumerate}
  When $H=G$, $H$ is called the \emph{improper} subgroup of $G$.
  
  When $H\subset G$, $H$ is called a \emph{proper} subgroup of $G$, denoted
  $H<G$.
\end{definition}

\begin{example}
  \begin{minipage}[t]{3in}
    \listbreak 
    \[Z_4=\{0,1,2,3\}\]
  
    \begin{tabular}{c|cccc}
      $+$ & 0 & 1 & 2 & 3 \\
      \hline
      0 & 0 & 1 & 2 & 3 \\
      1 & 1 & 2 & 3 & 0 \\
      2 & 2 & 3 & 0 & 1 \\
      3 & 3 & 0 & 1 & 2 \\
    \end{tabular}

    \[\begin{array}{l}
    \{0\} \\
    \{0,2\} \\
    \{0,1,2,3\} \\
    \end{array}\]

    One proper, non-trivial subgroup
  \end{minipage}
  \begin{minipage}[t]{3in}
    \listbreak
    \[V=D_4=\{e,a,b,c\}\]
  
    \begin{tabular}{c|cccc}
      $*$ & e & a & b & c \\
      \hline
      e & e & a & b & c \\
      a & a & e & c & b \\
      b & b & c & e & a \\
      c & c & b & a & e \\
    \end{tabular}

    \[\begin{array}{l}
    \{e\} \\
    \{e,a\} \\
    \{e,b\} \\
    \{e,c\} \\
    \{e,a,b,c\} \\
    \end{array}\]

    Three proper, non-trivial subgroup
  \end{minipage}

  \bigskip

  \begin{minipage}{3in}
    \setlength{\leftskip}{0.75in}
    \begin{tikzpicture}
      \node at (0,0) {$\{0\}$};
      \draw (0,0.5) -- (0,1.5);
      \node at (0,2) {$\{0,2\}$};
      \draw (0,2.5) -- (0,3.5);
      \node at (0,4) {$\Z_4$};
    \end{tikzpicture}
  \end{minipage}
  \begin{minipage}{3in}
    \begin{tikzpicture}
      \node at (0,0) {$\{e\}$};
      \draw (0,0.5) -- (-2,1.5);
      \draw (0,0.5) -- (0,1.5);
      \draw (0,0.5) -- (2,1.5);
      \node at (-2,2) {$\{e,a\}$};
      \node at (0,2) {$\{e,b\}$};
      \node at (2,2) {$\{e,c\}$};
      \node at (0,4) {$D_4$};
      \draw (-2,2.5) -- (0,3.5);
      \draw (0,2.5) -- (0,3.5);
      \draw (2,2.5) -- (0,3.5);
    \end{tikzpicture}
  \end{minipage}
\end{example}

\begin{theorem}
  Let $G$ be a group:
  \[\{e\}\le G\]
\end{theorem}

\begin{theproof}
  $\{e\}\subseteq G$
  
  $(ee)e=ee=e$ \\
  $e(ee)=ee=e$ \\
  $\therefore \{e\}$ is associative.

  $e\in\{e\}$ \\
  $\therefore \{e\}$ has identity.

  $ee=ee=e$ \\
  $\therefore \{e\}$ has inverses.

  $\therefore \{e\}\le G$
\end{theproof}

\begin{definition}
  $\{e\}\subseteq G$ is called the \emph{trivial} subgroup of $G$. All other
  subgroups are referred to as \emph{non-trivial}.
\end{definition}

\begin{theorem}
  Let $G$ be a group and $H\subseteq G$. $H\le G$ iff the following three
  properties hold:
  \begin{enumerate}
  \item{$H$ is closed under the induced operation of $G$}
  \item{$e\in H$}
  \item{$\forall\,a\in H,a^{-1}\in H$}
  \end{enumerate}
\end{theorem}

\begin{theproof}
  \listbreak
  \begin{description}
  \item{$\implies$} Assume $H\le G$.

    $H$ is a group, so it is closed under the induced operation and
    $\forall\,a\in H,a^{-1}\in H$. Also, by closure, $aa^{-1}=e\in H$.

    $\therefore$ the three properties hold.
    
  \item{$\impliedby$} Assume the three properties hold.

    Assume $a,b,c\in H$. \\
    By closure, $(ab)c\in H$ and $a(bc)\in H$. \\
    $a,b,c\in G$ \\
    $(ab)c=a(bc)$ in $G$, so this must also hold in $H$. \\
    $\therefore H$ is associative.

    $e\in H$, and since $e$ is the identity for $G$, it must also be the
    identity for $H$.

    $\forall\,a\in H,a^{-1}\in H$.

    $\therefore H\le G$.
  \end{description}
\end{theproof}
\newpage
\begin{example}
  $G=\Z_4=\{0,1,2,3\}$
  
  $H=\{0,2\}$

  \bigskip

  \begin{tabular}{c|cc}
    $+_4$ & 0 & 2 \\
    \hline
    0 & 0 & 2 \\
    2 & 2 & 0 \\
  \end{tabular}

  \bigskip

  $H\ \mbox{is closed}$
  
  $0=e\in H$
  
  $0^{-1}=0\in H$
  
  $2^{-1}=2\in H$
  
  $\therefore H\le G$
\end{example}

\begin{theorem}[Subgroup Test]
  Let $G$ be a group and $H\ne\emptyset,H\subseteq G$
  \[H\le G\iff \forall\,a,b\in H,ab^{-1}\in H\ \ (b^{-1}\in G)\]
\end{theorem}

\begin{theproof}
  \listbreak
  \begin{description}
  \item{$\implies$ Assume $H\le G$}

    Assume $a,b\in H$ \\
    Since $H$ is a group, $b^{-1}\in H$ \\
    But $H\subseteq G$, so $b^{-1}\in G$ \\
    By closure, $ab^{-1}\in H$
    
  \item{$\impliedby$ Assume
    $\forall\,a,b\in H,ab^{-1}\in H\ \ (b^{-1}\in G)$}

    $H\ne\emptyset$ \\
    Assume $a,b\in H$

    $b=(b^{-1})^{-1}\in H$ \\
    But $H\subseteq G$, so $b=(b^{-1})^{-1}\in G$ \\
    By assumption, $a(b^{-1})^{-1}\in H$ \\
    $ab\in H$ \\
    $\therefore H$ is closed under the induced operation of G.

    Since $H\subseteq G,a\in G$ \\
    But since $G$ is a group, $a^{-1}\in G$ \\
    By assumption, $aa^{-1}\in H$ \\
    $\therefore e\in H$

    $e\in H$ and $a^{-1}\in G$ \\
    So by assumption, $ea^{-1}\in H$ \\
    $\therefore a^{-1}\in H$
  \end{description}
\end{theproof}

\begin{example}
  Let $G=GL(n,\R)$ and $H=\{A\in G\mid \det(A)=1\}$ \\
  Prove: $H<G$

  Assume $A,B\in H$ \\
  Clearly, $H\subset G$ \\
  So, $B\in G$ \\
  $\det(B)=1\ne0$, so $B$ is invertible \\
  $B^{-1}\in G$ \\
  $\det(AB^{-1})=\frac{\det(A)}{\det(B)}=\frac{1}{1}=1$ \\
  $AB^{-1}\in H$ \\
  $\therefore H<G$
\end{example}

\begin{theorem}
  Let $G$ and $G'$ be groups and $\p:G\to G'$ be an isomorphism:
  \[H\le G\implies \p[H]\le G'\]
  Isomorphisms map subgroups to subgroups.
\end{theorem}

\begin{theproof}
  Assume $H\le G$ \\
  Assume $x,y\in\p[H]$ \\
  $\exists\,a,b\in H,\p(a)=x$ and $\p(b)=y$ \\
  $\p[H]\subseteq G'$ \\
  So, $y\in G'$ \\
  But $G'$ is a group, so $y^{-1}\in G'$ \\
  $\p$ is a homomorphism \\
  $xy^{-1}=\p(a)\p(b)^{-1}=\p(a)\p(b^{-1})=\p(ab^{-1})$ \\
  $H$ is a group, so $b^{-1}\in H$ \\
  By closure, $ab^{-1}\in H$ \\
  $\p$ is well-defined, so $\p(ab^{-1})\in \p[H]$ \\
  $xy^{-1}\in\p[H]$ \\
  $\therefore$ by the subgroup test, $\p[H]\le G'$
\end{theproof}

\begin{corollary}
  Let $G\simeq G'$:
  \[\forall\,H\le G,\exists\,H'\le G',H\simeq H'\]
\end{corollary}
\newpage
\begin{theorem}
  Let $G$ be a group:
  \[H,K\le G\implies H\cap K\le G\]
\end{theorem}

\begin{theproof}
  Assume $H,K\le G$ \\
  Assume $a,b\in H\cap K$ \\
  $a,b\in H$ and $a,b\in K$ \\
  But $H$ and $K$ are groups, so $b^{-1}\in H$ and $b^{-1}\in K$ \\
  $H\cap K\subseteq H,K\subseteq G$ \\
  So $b^{-1}\in G$ \\
  $ab^{-1}\in H$ and $ab^{-1}\in K$ \\
  $ab^{-1}\in H\cap K$ \\
  $\therefore$ by the subgroup test, $H\cap K\le G$
\end{theproof}

\end{document}
