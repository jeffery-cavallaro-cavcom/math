\documentclass[letterpaper,12pt,fleqn]{article}
\usepackage{matharticle}
\pagestyle{empty}
\begin{document}
\section*{Notation}

\begin{description}
\item{Additive Groups:}
  \begin{eqnarray*}
    0 &=& \mbox{identity} \\
    -a &=& \mbox{inverse of}\ a \\
    0a &=& 0 \\
    na &=& a+a+\ldots+a\hspace{0.25in}\mbox{(n times)} \\
    -na &=& (-a)+(-a)+\ldots+(-a)=n(-a) \\
  \end{eqnarray*}
    
\item{Multiplicative Groups:}
  \begin{eqnarray*}
    1 &=& \mbox{identity} \\
    a^{-1} &=& \mbox{inverse of}\ a \\
    a^0 &=& 1 \\
    a^n &=& aa\ldots a\hspace{0.25in}\mbox{(n times)} \\
    a^{-n} &=& (a^{-1})(a^{-1})\ldots(a^{-1})=(a^{-1})^n \\
  \end{eqnarray*}
\end{description}
  
Unless the form of the group is explicitly stated, the multiplicative
(juxtaposition) notation is generally used by default. An exception is for
abelian groups, where the additive notation is preferred.

\begin{theorem}
  Let $G$ be a group and $n\in\Z$:
  \[\forall\,a\in G,a^{-n}=(a^n)^{-1}\]
\end{theorem}

\begin{theproof}
  Assume $a\in G$

  \begin{description}
  \item{Case 1: $n=0$}

    $a^{-0}=a^0=e=e^{-1}=(a^0)^{-1}$ \\

  \item {Case 2: $n>0$}

    Proof by induction on $n$:

    \begin{description}
    \item{Base: $n=1$:}

      $a^{-1}=(a^1)^{-1}$

    \item{Assume $a^{-n}=(a^n)^{-1}$}

    \item{Consider $a^{-(n+1)}$:}
      
      $a^{-(n+1)}=(a^{-1})^{n+1}=(a^{-1})^na^{-1}=a^{-n}a^{-1}=(a^n)^{-1}a^{-1}=
      (aa^n)^{-1}=(a^{n+1})^{-1}$
    \end{description}

  \item{Case 3: $n<0$}

    Let $m=-n$ \\
    $m>0$ \\
    $a^{-n}=a^m=[(a^m)^{-1}]^{-1}=(a^{-m})^{-1}=(a^{-(-n)})^{-1}=(a^n)^{-1}$
  \end{description}
\end{theproof}

\begin{corollary}
  Let $G$ be a group and $n\in\Z$:
  \[\forall\,a\in G,a^na^{-n}=e\]
\end{corollary}

\begin{theproof}
  Assume $a\in G$ \\
  $a^na^{-n}=a^n(a^n)^{-1}=e$
\end{theproof}

\begin{theorem}
  Let $G$ be a group and $n,m\in\Z$:
  \[\forall\,a\in G,a^ma^n=a^{m+n}\]
\end{theorem}

\begin{theproof}
  Assume $a\in G$ \\

  \begin{description}
  \item{Case 1: $m=0$}

    $a^0a^n=ea^n=a^n=a^{0+n}$

  \item{Case 2: $n=0$}

    $a^ma^0=a^me=a^m=a^{m+0}$

  \item{Case 3: $m,n>0$}

    Proof by induction on $n$ for a fixed value of $m$:
    
    \begin{description}
    \item Base: $n=1$:
    
      $a^ma^1=a^ma=a^{m+1}$
    
    \item Assume $a^ma^n=a^{m+n}$

    \item Consider $a^ma^{n+1}$:
    
      $a^ma^{n+1}=a^m(a^na)=(a^ma^n)a=a^{m+n}a=a^{(m+n)+1}=a^{m+(n+1)}$
    \end{description}

  \item {Case 4: $m,n<0$}

    Let $h=-m$ and $k=-n$ \\
    $h,k>0$ \\
    $a^ma^n=a^{-h}a^{-k}=(a^h)^{-1}(a^k)^{-1}=(a^ka^h)^{-1}=(a^{k+h})^{-1}=
    (a^{h+k})^{-1}=a^{-(h+k)}$ \\
    $a^ma^n=a^{-(-m-n)}=a^{m+n}$

  \item{Case 5: $m>0$ and $n<0$}

    \begin{description}
    \item{Case 1: $m=\abs{n}$}

      $m+n=0$ \\
      $n=-m$ \\
      $a^ma^n=a^ma^{-m}=e=a^0=a^{m-m}=a^{m-(-n)}=a^{m+n}$

    \item{Case 2: $m>\abs{n}$}

      $(a^ma^n)(a^{m+n})^{-1}=[a^ma^{-(-n)}](a^{m+n})^{-1}=
      a^m[(a^{-n})^{-1}(a^{m+n})^{-1}]=a^m(a^{m+n}a^{-n})^{-1}$ \\
      But $m+n>0$ and $-m>0$, so \\
      $(a^ma^n)(a^{m+n})^{-1}=a^m[a^{(m+n)-n}]^{-1}=a^m(a^m)^{-1}=e$ \\
      By uniqueness of the inverse, $a^ma^n=[(a^{m+n})^{-1}]^{-1}$ \\
      $\therefore a^ma^n=a^{m+n}$

    \item{Case 3: $m<\abs{n}$}

      $(a^{m+n})^{-1}(a^ma^n)=(a^{m+n})^{-1}[a^{-(-m)}a^n]=
      [(a^{m+n})^{-1}(a^{-m})^{-1}]a^n=(a^{-m}a^{m+n})^{-1}a^n$ \\
      But $-m<0$ and $m+n<0$, so \\
      $(a^{m+n})^{-1}(a^ma^n)=[a^{-m+(m+n)}]^{-1}a^n=(a^n)^{-1}a^n=e$ \\
      By uniqueness of the inverse, $a^ma^n=[(a^{m+n})^{-1}]^{-1}$ \\
      $\therefore a^ma^n=a^{m+n}$
    \end{description}

  \item{Case 6: $m<0$ and $n>0$}

    \begin{description}
    \item{Case 1: $\abs{m}=n$}

      $m+n=0$ \\
      $n=-m$ \\
      $a^ma^n=a^ma^{-m}=e=a^0=a^{m-m}=a^{m-(-n)}=a^{m+n}$

    \item{Case 2: $\abs{m}<n$}

      $(a^{m+n})^{-1}(a^ma^n)=(a^{m+n})^{-1}[a^{-(-m)}a^n]=
      [(a^{m+n})^{-1}(a^{-m})^{-1}]a^n=(a^{-m}a^{m+n})^{-1}a^n$ \\
      But $-m>0$ and $m+n>0$, so \\
      $(a^{m+n})^{-1}(a^ma^n)=[a^{-m+(m+n)}]^{-1}a^n=(a^n)^{-1}a^n=e$ \\
      By uniqueness of the inverse, $a^ma^n=[(a^{m+n})^{-1}]^{-1}$ \\
      $\therefore a^ma^n=a^{m+n}$

    \item{Case 3: $\abs{m}>n$}

      $(a^ma^n)(a^{m+n})^{-1}=[a^ma^{-(-n)}](a^{m+n})^{-1}=
      a^m[(a^{-n})^{-1}(a^{m+n})^{-1}]=a^m(a^{m+n}a^{-n})^{-1}$ \\
      But $m+n<0$ and $-n<0$, so \\
      $(a^ma^n)(a^{m+n})^{-1}=a^m[a^{(m+n)-n}]^{-1}=a^m(a^m)^{-1}=e$ \\
      By uniqueness of the inverse, $a^ma^n=[(a^{m+n})^{-1}]^{-1}$ \\
      $\therefore a^ma^n=a^{m+n}$
    \end{description}
  \end{description}
\end{theproof}

\begin{corollary}
  Let $G$ be a group and $n,m\in\Z$:
  \[\forall\,a\in G,a^ma^n=a^na^m\]
\end{corollary}

\begin{theproof}
  Assume $a\in G$ \\
  $a^ma^n=a^{m+n}+a^{n+m}=a^na^m$
\end{theproof}

\begin{theorem}
  Let $G$ be a group and $n,m\in\Z$:
  \[\forall\,a\in G,(a^m)^n=a^{mn}\]
\end{theorem}

\begin{theproof}
  Assume $a\in G$
  
  \begin{description}
  \item{Case 1: $n=0$}

    $(a^m)^0=e=a^0=a^{m0}$

  \item{Case 2: $n>0$}

    Proof by induction on $n$:

    \begin{description}
    \item{Base: $n=1$}

      $(a^m)^1=a^m=a^{m1}$

    \item{Assume $(a^m)^n=a^{mn}$}

    \item{Consider $(a^m)^{n+1}$}

      $(a^m)^{n+1}=(a^m)^n(a^m)=a^{mn}a^m=a^{mn+m}=a^{m(n+1)}$
    \end{description}

  \item{Case 3: $n<0$}

    Let $k=-n$ \\
    $k>0$ \\
    $(a^m)^n=(a^m)^{-k}=[(a^m)^k]^{-1}=(a^{mk})^{-1}=a^{-mk}=a^{-m(-n)}=a^{mn}$
  \end{description}
\end{theproof}

Thus, all of the exponent rules work as expected.

\begin{example}
  \[a^{-2}a^5=(a^{-1})(a^{-1})aaaaa=(a^{-1})eaaaa=(a^{-1})aaaa=eaaa=aaa=a^3\]
\end{example}

\begin{theorem}
  Let $G$ be an abelian group and $n\in\Z$:
  \[\forall\,a,b\in G,(ab)^n=a^nb^n\]
\end{theorem}

\begin{theproof}
  Assume $a,b\in G$.
  \begin{description}
  \item{Case 1: $n=0$}
    
    $(ab)^0=e=ee=a^0b^0$

  \item{Case 2: $n>0$}

    Proof by induction on $n$:

    \begin{description}
    \item{Base: $n=1$}

      $(ab)^1=ab=a^1b^1$

    \item Assume $(ab)^n=a^nb^n$

    \item Consider $(ab)^{n+1}$
      \begin{eqnarray*}
        (ab)^{n+1} &=& (ab)(ab)^n \\
        &=& (ab)(a^nb^n) \\
        &=& (ab)(b^na^n) \\
        &=& a(bb^n)a^n \\
        &=& ab^{n+1}a^n \\
        &=& aa^nb^{n+1} \\
        &=& a^{n+1}b^{n+1} \\
      \end{eqnarray*}
    \end{description}

  \item{Case 3: $n<0$}

    Let $m=-n$ \\
    $(ab)^n=(ab)^{-m}=[(ab)^m]^{-1}$ \\
    But $m>0$, so: \\
    $(ab)^n=(a^mb^m)^{-1}=(b^m)^{-1}(a^m)^{-1}=b^{-m}a^{-m}=b^{-(-n)}a^{-(-n)}=
    b^na^n=a^nb^n$
  \end{description}
\end{theproof}

\begin{theorem}
  Let $G$ be a finite group with order $m$:
  \[\forall\,a\in G,\exists\,n\in\Z^+,a^n=e\]
\end{theorem}

\begin{theproof}
  Assume $a\in G$ \\
  Consider the set $S=\{a^0=e,a,a^2,a^3,\ldots,a^m\}$ \\
  By closure, $S\subseteq G$ \\
  $\abs{S}=m+1$, so at least one of the elements in $S$ is repeated \\
  If the repeated element is $e$ then done \\
  Otherwise, assume $a^i=a^j$, where $i<j$ \\
  $a^ia^{-i}=a^ja^-i$ \\
  $a^0=a^{j-i}$ \\
  $e=a^{j-i}$ \\
  Let $n=j-i$ \\
  $0<n<m$ so $a^n\in S$ \\
  $\therefore\exists n\in\Z^+,a^n=e$
\end{theproof}

\end{document}
