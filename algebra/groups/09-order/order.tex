\documentclass[letterpaper,12pt,fleqn]{article}
\usepackage{matharticle}
\pagestyle{empty}
\begin{document}
\section*{Order}

\begin{definition}
  The \emph{order} of a group $G$, denoted $\abs{G}$ is the cardinality of the
  set of $G$.
\end{definition}

\begin{minipage}{2in}
\underline{$\abs{G}=1$}

\bigskip

$G=\{e\}$

\bigskip

\begin{tabular}{c|c}
  $*$ & $e$ \\
  \hline
  $e$ & $e$ \\
\end{tabular}

\bigskip

$G\simeq\Z_1$ \\
(abelian)
\end{minipage}
\begin{minipage}{2in}
\underline{$\abs{G}=2$}

\bigskip

$G=\{e,a\}$

\bigskip

\begin{tabular}{c|cc}
  $*$ & $e$ & $a$ \\
  \hline
  $e$ & $e$ & $a$ \\
  $a$ & $a$ & $e$ \\
\end{tabular}

\bigskip

$G\simeq\Z_2$ \\
(abelian)
\end{minipage}
\begin{minipage}{2in}
\underline{$\abs{G}=3$}

\bigskip

$G=\{e,a,b\}$

\bigskip

\begin{tabular}{c|ccc}
  $*$ & $e$ & $a$ & $b$ \\
  \hline
  $e$ & $e$ & $a$ & $b$ \\
  $a$ & $a$ & $b$ & $e$ \\
  $b$ & $b$ & $e$ & $a$ \\
\end{tabular}

\bigskip

$G\simeq\Z_3$ \\
(abelian)
\end{minipage}

\bigskip

Note that each element may appear only once in each row; otherwise, the linear
equation $ax=b$ would have more than one solution. Likewise, each element may
appear only once in each column; otherwise, the linear equation $xa=b$ would
have more than one solution.

\underline{$\abs{G}=4$}

\bigskip

$G=\{e,a,b,c\}$

\bigskip

\begin{tabular}{c|cccc}
  $*$ & $e$ & $a$ & $b$ & $c$ \\
  \hline
  $e$ & $e$ & $a$ & $b$ & $c$ \\
  $a$ & $a$ & $b$ & $c$ & $e$ \\
  $b$ & $b$ & $c$ & $e$ & $a$ \\
  $c$ & $c$ & $e$ & $a$ & $b$ \\
\end{tabular}

\bigskip

$G\simeq\Z_4$ \\
(abelian)

\bigskip

\begin{minipage}{1.5in}
\begin{tabular}{c|cccc}
  $*$ & $e$ & $a$ & $b$ & $c$ \\
  \hline
  $e$ & $e$ & $a$ & $b$ & $c$ \\
  $a$ & $a$ & $c$ & $e$ & $b$ \\
  $b$ & $b$ & $e$ & $c$ & $a$ \\
  $c$ & $c$ & $b$ & $a$ & $e$ \\
\end{tabular}
\end{minipage}
\begin{minipage}{0.5in}
  $b\leftrightarrow c$
\end{minipage}
\begin{minipage}{1.5in}
\begin{tabular}{c|cccc}
  $*$ & $e$ & $a$ & $c$ & $b$ \\
  \hline
  $e$ & $e$ & $a$ & $c$ & $b$ \\
  $a$ & $a$ & $b$ & $e$ & $c$ \\
  $c$ & $c$ & $e$ & $b$ & $a$ \\
  $b$ & $b$ & $c$ & $a$ & $e$ \\
\end{tabular}
\end{minipage}
\begin{minipage}{0.5in}
  $\to$
\end{minipage}
\begin{minipage}{1.5in}
\begin{tabular}{c|cccc}
  $*$ & $e$ & $a$ & $b$ & $c$ \\
  \hline
  $e$ & $e$ & $a$ & $b$ & $c$ \\
  $a$ & $a$ & $b$ & $c$ & $e$ \\
  $b$ & $b$ & $c$ & $e$ & $a$ \\
  $c$ & $c$ & $e$ & $a$ & $b$ \\
\end{tabular}
\end{minipage}

\bigskip

$G\simeq\Z_4$ \\
(abelian)

\bigskip

\begin{minipage}{1.5in}
\begin{tabular}{c|cccc}
  $*$ & $e$ & $a$ & $b$ & $c$ \\
  \hline
  $e$ & $e$ & $a$ & $b$ & $c$ \\
  $a$ & $a$ & $e$ & $c$ & $b$ \\
  $b$ & $b$ & $c$ & $a$ & $e$ \\
  $c$ & $c$ & $b$ & $e$ & $a$ \\
\end{tabular}
\end{minipage}
\begin{minipage}{0.5in}
  $a\leftrightarrow b$
\end{minipage}
\begin{minipage}{1.5in}
\begin{tabular}{c|cccc}
  $*$ & $e$ & $b$ & $a$ & $c$ \\
  \hline
  $e$ & $e$ & $b$ & $a$ & $c$ \\
  $b$ & $b$ & $e$ & $c$ & $a$ \\
  $a$ & $a$ & $c$ & $b$ & $e$ \\
  $c$ & $c$ & $a$ & $e$ & $b$ \\
\end{tabular}
\end{minipage}
\begin{minipage}{0.5in}
  $\to$
\end{minipage}
\begin{minipage}{1.5in}
\begin{tabular}{c|cccc}
  $*$ & $e$ & $a$ & $b$ & $c$ \\
  \hline
  $e$ & $e$ & $a$ & $b$ & $c$ \\
  $a$ & $a$ & $b$ & $c$ & $e$ \\
  $b$ & $b$ & $c$ & $e$ & $a$ \\
  $c$ & $c$ & $e$ & $a$ & $b$ \\
\end{tabular}
\end{minipage}

\bigskip

$G\simeq\Z_4$ \\
(abelian)

\bigskip

\begin{minipage}{2in}
\begin{tabular}{c|cccc}
  $*$ & $e$ & $a$ & $b$ & $c$ \\
  \hline
  $e$ & $e$ & $a$ & $b$ & $c$ \\
  $a$ & $a$ & $e$ & $c$ & $b$ \\
  $b$ & $b$ & $c$ & $e$ & $a$ \\
  $c$ & $c$ & $b$ & $a$ & $e$ \\
\end{tabular}
\end{minipage}
\begin{minipage}{3in}
  This group is also abelian; however, $G\not\simeq\Z_4$ because:
  \[\forall\,x\in G,xx=e\]
  Thus, it is structurally different from $\Z_4$. It is referred to as the
  Klein-4 group, denoted by $V$ or $K_4$.
\end{minipage}

\bigskip

So, every group of 4 elements is isometric to either $\Z_4$ or $K_4$. Thus,
there are only 2 distinct groups \emph{up to isomorphism}.
\end{document}
