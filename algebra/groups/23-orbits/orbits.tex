\documentclass[letterpaper,12pt,fleqn]{article}
\usepackage{matharticle}
\pagestyle{empty}
\renewcommand{\o}{\sigma}
\renewcommand{\O}{\mathcal{O}}
\newcommand{\orb}[2]{\O_{#1,#2}}
\newcommand{\oorb}[1]{\orb{#1}{\o}}
\begin{document}
\section*{Orbits}

\begin{definition}
  Let $\o$ be a permutation on a set $A$ and let $a\in A$. The \emph{orbit} of
  $a$ under $\o$ is given by:
  \[\oorb{a}=\{\o^n(a)\mid n\in\Z\}\]
\end{definition}

\begin{theorem}
  Let $\o$ be a permutation on a set $A$ and define the relation $a\sim b$ iff
  $b\in\oorb{a}$:
  \[\sim\ \mbox{is an equivalence relation}\]
\end{theorem}

\begin{theproof}
  \listbreak
  \begin{description}
  \item{R}: Assume $a\in A$ \\
    $\o^0(a)=a$ \\
    $\therefore a\sim a$
  \item{S}: Assume $a\sim b$ \\
    $\exists\,n\in\Z,\o^n(a)=b$ \\
    $\o^{-n}(b)=a$ \\
    $\therefore b\sim a$
  \item{T}: Assume $a\sim b$ and $b\sim c$ \\
    $\exists\,n,m\in\Z,\o^n(a)=b$ and $\o^m(b)=c$ \\
    $\o^m(\o^n(a))=c$ \\
    $\o^{n+m}(a)=c$ \\
    $\therefore a\sim c$
  \end{description}
  $\therefore ~$ is an equivalence relation.
\end{theproof}

Thus, the orbits are the equivalence classes of the above equivalence relation.

\begin{corollary}
  Let $\o$ be a permutation on a set $A$ and $a,b\in A$:
  \[\left(\exists\,c\in A,c\in\oorb{a}\ \mbox{and}\ c\in\oorb{b}\right)\implies
  \oorb{a}=\oorb{b}\]
\end{corollary}

\begin{theproof}
  Assume $\exists\,c\in A,c\in\oorb{a}\ \mbox{and}\ c\in\oorb{b}$ \\
  $a\sim c$ and $b\sim c$ \\
  $c\sim b$ \\
  $a\sim b$ \\
  $\therefore \oorb{a}=\oorb{b}$
\end{theproof}

\end{document}
