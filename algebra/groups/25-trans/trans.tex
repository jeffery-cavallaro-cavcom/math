\documentclass[letterpaper,12pt,fleqn]{article}
\usepackage{matharticle}
\pagestyle{empty}
\renewcommand{\o}{\sigma}
\begin{document}
\section*{Transpositions}

\begin{definition}
  A \emph{transposition} is a cycle of length $2$.
\end{definition}

\begin{theorem}
  Every permutation $\o$ of a finite set $A$ with at least $2$ elements can be
  written as a product of transpositions.
\end{theorem}

\begin{theproof}
  Since each $\o$ of $A$ can be expressed as a product of disjoint cycles,
  AWLOG that $\o$ contains one cycle \\
  The case $n=2$ is trivial, so assume $n>2$

  Proof by induction on the length of the cycle $n$

  \begin{description}
  \item Base Case: $n=3$ \\
    $(x_1x_2x_3)=(x_1x_2)(x_2x_3)$
  \item Assume $(x_1x_2x_3\ldots x_n)=(x_1x_2)(x_2x_3)\ldots(x_{n-1}x_n)$
  \item $(x_1x_2x_3\ldots x_nx_{n+1})=(x_1x_2x_3\ldots x_{n-1}x_n)(x_nx_{n+1})=
    (x_1x_2)(x_2x_3)\ldots(x_{n-1}x_n)(x_nx_{n+1})$
  \end{description}
\end{theproof}

\begin{example}
  $(12345678)=(12)(23)(34)(45)(56)(67)(78)$ \\
  $(12345678)=(18)(17)(16)(15)(14)(13)(12)$
\end{example}

\begin{corollary}
  An $n$-cycle can be represented using $n-1$ transpositions.
\end{corollary}

\begin{definition}
  A permutation $\o$ on a set $A$ that can be expressed as an even number of
  transpositions is called \emph{even}. Otherwise, it is called \emph{odd}.
\end{definition}

\begin{theorem}
  The evenness or oddness of a permutation is well-defined.
\end{theorem}

\begin{theproof}
  Assumed $\o\in S_n$ is expressed as a composition of transpositions \\
  Associate $\o$ with its corresponding permutation matrix \\
  The determinant of the matrix is either $1$ or $-1$, depending on the either
  an odd ($-1$) or even (1) number of transpositions \\
  Therefore the evenness or oddness is well-defined.
\end{theproof}

\end{document}
