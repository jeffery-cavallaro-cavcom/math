\documentclass[letterpaper,12pt,fleqn]{article}
\usepackage{matharticle}
\pagestyle{empty}
\newcommand{\cycle}[1]{\left<#1\right>}
\newcommand{\p}{\phi}
\begin{document}
\section*{Cyclic Groups}

\begin{theorem}
$\cycle{a}=\{a^n\mid n\in Z\}$ is a group.
\end{theorem}

\begin{theproof}
  \listbreak
  \begin{description}
  \item{Associativity}

    Assume $x,y,z\in\cycle{a}$ \\
    $\exists\,r,s,t\in\Z,x=a^r,y=a^2,z=a^t$ \\
    $(xy)z=(a^ra^s)a^t=a^{r+s}a^t=a^{r+s+t}$ \\
    $x(yz)=a^r(a^sa^t)=a^ra^{s+t}=a^{r+s+t}$ \\
    $\therefore\cycle{a}$ is associative.

  \item{Identity}

    $a^0\in\cycle{a}$ \\
    Assume $a^n\in\cycle{a}$ \\
    $a^0a^n=a^{0+n}=a^n$ \\
    $a^na^0=a^{n+0}=a^n$ \\
    $\therefore a^0=e\in\cycle{a}$

  \item{Inverses}
    
    Assume $a^n\in\cycle{a}$ \\
    $a^{-n}\in\cycle{a}$ \\
    $a^{-n}a^n=a^{-n+n}=a^0=e$ \\
    $a^na^{-n}=a^{n+(-n)}=a^0=e$
  \end{description}
  $\therefore\cycle{a}$ is a group.
\end{theproof}

\begin{definition}
  To say that a group $G$ is \emph{cyclic} means $\exists\,a\in G,\cycle{a}=G$.
  The element $a$ is said to \emph{generate} $G$ and $a$ is called a
  \emph{generator} for $G$. $G$ is said to be \emph{generated} by $a$.
\end{definition}

\begin{theorem}
  \listbreak
  \[\cycle{a^{-1}}=\cycle{a}\]
\end{theorem}

\begin{theproof}
  \listbreak
  \begin{description}
  \item{$\implies$ Assume $(a^{-1})^n\in\cycle{a^{-1}}$}

    $(a^{-1})^n=a^{-n}\in\cycle{a}$ \\
    $\therefore\cycle{a^{-1}}\subseteq\cycle{a^n}$
\newpage
  \item{$\impliedby$ Assume $a^n\in\cycle{a}$}

    $a^n=(a^{-1})^{-n}\in\cycle{a^{-1}}$ \\
    $\therefore\cycle{a^n}\subseteq\cycle{a^{-1}}$

  \end{description}

  $\therefore\cycle{a^{-1}}=\cycle{a}$
\end{theproof}

\begin{example}
  $Z_4=\{0,1,2,3\}$
  
  $1+3=3+1=0$ \\
  $-1=3$ and $-3=1$

  $\cycle{1}=\{1,2,3,0\}=\Z_4$ \\
  $\cycle{3}=\{3,2,1,0\}=\Z_4$

  $\cycle{1}=\cycle{3}=\Z_4$
\end{example}

\begin{theorem}
  Let $G$ be a group:
  \[G\ \mbox{cyclic}\implies G\ \mbox{abelian}\]
\end{theorem}

\begin{theproof}
  Assume $G$ is cyclic \\
  $\exists\,a\in G,\cycle{a}=G$ \\
  Assume $x,y\in G$ \\
  $\exists\,n,m\in G,x=a^n$ and $y=a^m$ \\
  $xy=a^na^m=a^{n+m}=a^{m+n}=a^ma^n=yx$ \\
  $\therefore G$ is abelian.
\end{theproof}

Note that the inverse is not true: consider $K_4$:

\begin{minipage}{2in}
  \begin{tabular}{c|cccc}
    $*$ & $e$ & $a$ & $b$ & $c$ \\
    \hline
    $e$ & $e$ & $a$ & $b$ & $c$ \\
    $a$ & $a$ & $e$ & $c$ & $b$ \\
    $b$ & $b$ & $c$ & $e$ & $a$ \\
    $c$ & $c$ & $b$ & $a$ & $e$ \\
  \end{tabular}
\end{minipage}
\begin{minipage}{2in}
  \begin{eqnarray*}
    \cycle{e} &=& \{e\} \\
    \cycle{a} &=& \{e,a\} \\
    \cycle{b} &=& \{e,b\} \\
    \cycle{c} &=& \{e,c\} \\
  \end{eqnarray*}
\end{minipage}

It is abelian; however, it has no generator and is thus not cyclic.

\begin{theorem}
  Let $G$ and $G'$ be groups and $\p:G\to G'$ be an isomorphism:
  \[\forall\,a\in G,\forall\,n\in\Z,\p(a^n)=\p(a)^n\]
\end{theorem}

\begin{theproof}
  Assume $a\in G$
  Assume $n\in\Z$
  \begin{description}
  \item Case 1: $n>0$

    Proof by induction on $n$

    \begin{description}
    \item Base: $n=1$

      $\p(a^1)=\p(a)=\p(a)^1$

    \item Assume $\p(a^n)=\p(a)^n$

    \item Consider $\p(a^{n+1})$

      $\p(a^{n+1})=\p(a^na)=\p(a^n)\p(a)=\p(a)^n\p(a)=\p(a)^{n+1}$
    \end{description}

  \item Case 2: $n=0$

    $\p(a^0)=\p(e)=e'=\p(a)^0$
    
  \item Case 3: $n<0$

    Let $m=-n$ \\
    $n>0$ \\
    $\p(a^n)=\p(a^{-m})=\p[(a^m)^{-1}]=\p(a^m)^{-1}=[\p(a)^m]^{-1}=\p(a)^{-m}=
    \p(a)^{-(-n)}=\p(a)^n$
  \end{description}
\end{theproof}

\begin{theorem}
  Let $G$ and $G'$ be groups and $\p:G\to G'$ be an isomorphism:
  \begin{enumerate}
  \item{G cyclic $\implies$ $G'$ cyclic}
  \item{$\p$ maps generators in $G$ to generators in $G'$}
  \end{enumerate}
\end{theorem}

\begin{theproof}
  Assume $G$ is cyclic \\
  $\exists\,a\in G,\cycle{a}=G$ \\
  Assume $b'\in G'$ \\
  $\p$ is onto \\
  $\exists\,b\in G,\p(b)=b'$ \\
  $\exists\,n\in\Z,b=a^n$ \\
  $b'=\p(b)=\p(a^n)=\p(a)^n$ \\
  $\therefore \p(a)$ is a generator for $G'$ and $G'$ is cyclic.
\end{theproof}

\begin{definition}
  Let $G$ be a group. An \emph{automorphism} of $G$ is an isomorphism between
  $G$ and itself: $\p:G\to G$.
\end{definition}

To determine the number of possible automorphisms for a cyclic group,
determine the number of generators.

\begin{theorem}
  Let $\cycle{a}=G$ and $\abs{G}=n$:
  \begin{itemize}
  \item $G$ finite$\implies n\in\Z^+$ is the smallest positive number such
    that $a^n=e$.
  \item $G$ infinite$\implies n=\aleph_0$
  \end{itemize}
\end{theorem}

\begin{theproof}
  \listbreak
  \begin{description}
  \item Assume $G$ is finite

    $a\in G$, so $n>0$ \\
    $n\in\Z^+$ \\
    $\exists\,m\in\Z^+,a^m=e$ \\
    Let $m\in\Z^+$ be the smallest positive number such that $a^m=e$ \\
    Let $G'=\{a^k\mid 0\le k<m\}$ \\
    ABC: $G'$ contains duplicates \\
    $\exists\,h,k\in\Z^+,0\le h<k<m$ and $a^h=a^k$ \\
    $a^{k-h}=e$ \\
    But $1\le h-k<m$ \\
    CONTRADICTION! (on the minimality of $m$) \\
    So $G'$ contains $m$ distinct elements, and all $a^k,k\ge m$ are
    duplicates \\
    $\therefore \abs{G}=m=n$

  \item Assume $G$ is infinite

    $\forall\,n\in\Z^+,a^n\ne e$ \\
    ABC: $\exists\,h,k\in\Z^+,1\le h<k$ and $a^k=a^k$ \\
    $a^{k-h}=e$ \\
    CONTRADITION! \\
    So $\forall\,n\in\Z^+,a^n$ is distinct \\
    So $G$ is countably infinite \\
    $\therefore \abs{G}=\aleph_0$
  \end{description}
\end{theproof}

\begin{theorem}
  Let $G$ be a cyclic group and $\abs{G}=n$:
  \begin{itemize}
  \item $G$ finite$\implies G\simeq\Z_n$
  \item $G$ infinite$\implies G\simeq\Z$
  \end{itemize}
\end{theorem}

\begin{theproof}
  \listbreak
  \begin{description}
  \item Assume $G$ is finite

    $n\in\Z+$ \\
    Let $a$ be a generator for $G$ \\
    $G=\{a^k\mid k\in\Z^+\cup\{0\}$ and $0\le k<n\}$ \\
    All of the $a^k\in G$ are distinct \\
    Let $\p:G\to\Z_n$ such that $\p(a^k)=k$ \\
    Clearly, $\p$ is bijective \\
    Assume $a^k\in G$ \\
    Per the division algorithm, $k=qn+r$ where $q,r\in\Z$ and $0\le r<n$ \\
    $a^k=a^{qn+r}=(a^n)^qa^r=e^qa^r=a^r$ \\
    $k\equiv r\pmod{n}$ \\
    Assume $a^h,a^k\in G$ \\
    $\p(a^ha^k)=\p(a^{h+_n k})=h+_n k=\p(a^h)+_n\p(a^k)$ \\
    So $\p$ is bijective and a homomorphism, and thus an isomorphism \\
    $\therefore G\simeq\Z_n$

  \item Assume $G$ is infinite

    $n\in\aleph_0$ \\
    Let $G=\{a^k\mid k\in\Z^+\cup\{0\}\}$ \\
    All of the $a^k\in G$ are distinct \\
    Let $\p:G\to\Z$ such that $\p(a^k)=k$ \\
    Clearly, $\p$ is bijective \\
    Assume $a^h,a^k\in G$ \\
    $\p(a^ha^k)=\p(a^{h+k})=h+k=\p(a^h)+\p(a^k)$ \\
    So $\p$ is bijective and a homomorphism, and thus an isomorphism \\
    $\therefore G\simeq\Z$
  \end{description}
\end{theproof}

Thus, all finite cyclic groups are isomorphic to each other (via $\Z_n$) and
all infinite cyclic groups are isomorphic to each other (via $\Z$).

Furthermore, all structural proofs on cyclic groups can be performed more
easily in terms of $Z_n$ or $Z$.
\end{document}
