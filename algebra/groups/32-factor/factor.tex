\documentclass[letterpaper,12pt,fleqn]{article}
\usepackage{matharticle}
\pagestyle{empty}
\newcommand{\cycle}[1]{\left<#1\right>}
\newcommand{\n}{\mathrel{\triangleleft}}
\newcommand{\p}{\phi}
\newcommand{\m}{\mu}
\begin{document}
\section*{Factor Groups}

\begin{definition}
  Let $H\n G$. The set of all cosets of $H$, denoted $G/H$ and often referred
  to as $G$ modulo $H$, is given by:
  \[G/H=\{gH\mid g\in G\}\]
\end{definition}

\begin{theorem}
  Let $H\n G$:
  \[(aH)(bH)=(ab)H\]
  is a binary operation on $G/H$.
\end{theorem}

\begin{theproof}
  Assume $a_1,a_2,b\in G$ \\
  Assume $a_1H=a_2H$

  $a_1H,a_2H,bH\in G/H$ \\
  $(a_1H)(bH)=(a_1b)H$ \\
  $(a_2H)(bH)=(a_2b)H$ \\
  $a_1b,a_2b\in G$ \\
  So $(a_1b)H,(a_2b)H\in G/H$ \\
  Therefore the operation is closed.

  $a_1^{-1}a_2\in H$ \\
  $\exists\,h\in H,a_1^{-1}a_2=h$ \\
  $(a_1b)^{-1}(a_2b)=b^{-1}(a_1^{-1}a_2)b=b^{-1}hb$ \\
  But $H\n G$ \\
  So $b^{-1}hb\in H$ \\
  $(a_1b)H=(a_2b)H$ \\
  Therefore the operation is well-defined.

  Therefore the operation is a binary operation.
\end{theproof}

\begin{theorem}
  Let $H\n G$:
  \[G/H\ \mbox{is a group}\]
\end{theorem}

$G/H$ is called a factor or quotient group.
\newpage
\begin{theproof}
  Assume $a,b,c\in G$

  $(aH)(bH)=(ab)H$ is a well-defined and closed operation.

  $[(aH)(bH)](cH)=[(ab)H](cH)=[(ab)c]H=[a(bc)]H=(aH)[(bc)H]=(aH)[(bH)(cH)]$ \\
  $\therefore G/H$ is associative under the operation.

  $H(aH)=(eH)(aH)=(ea)H=aH$ \\
  $(aH)H=(aH)(eH)=(ae)H=aH$ \\
  $\therefore G/H$ has identity $H$

  $a^{-1}\in G$ \\
  $(a^{-1}H)(aH)=(a^{-1}a)H=eH=H$ \\
  $(aH)(a^{-1}H)=(aa^{-1})H=eH=H$ \\
  $\therefore G/H$ is closed under inverses.

  $\therefore G/H$ is a group under the operation.
\end{theproof}

\begin{theorem}
  Let $\p:G\to G'$ be a homomorphism of groups and $K=\ker(\p)$:
  \[G/K\simeq\p[G]\]
\end{theorem}

\begin{theproof}
  Let $\m:G/K\to\p[G]$ be defined by $\m(aK)=\p(a)$ \\
  By previous theorem, $\m$ is well-defined

  \begin{minipage}[t]{3in}
    Assume $\m(aK)=\m(bK)$ \\
    $\p(a)=\p(b)$ \\
    $\p(a)^{-1}\p(b)=e'$ \\
    $\p(a^{-1})\p(b)=e'$ \\
    $\p(a^{-1}b)=e'$ \\
    $a^{-1}b\in K$ \\
    $aK=bK$ \\
    $\therefore\m$ is one-to-one.
  \end{minipage}
  \begin{minipage}[t]{3in}
    Assume $g'\in\p[G]$ \\
    $\exists\,g\in G,\p(g)=g'$
    $gK\in G/K$ \\
    $\m(gK)=\p(g)=g'$ \\
    $\therefore\m$ is onto and is thus a bijection.
  \end{minipage}

  $\m((aK)(bK))=\m((ab)K)=\p(ab)=\p(a)\p(b)=\m(aK)\m(bK)$ \\
  $\therefore\m$ is a homomorphism and thus an isomorphism

  $\therefore G/K\simeq\p[G]$
\end{theproof}

\begin{example}
  $G=\Z_2\times\Z_4$ \\
  $H=\cycle{(1,2)}=\{(0,0),(1,2)\}$

  Since $G$ is abelian and $H\le G$ we have $H\n G$

  $\abs{G}=2\cdot4=8$ \\
  $\abs{H}=2$ \\
  $\abs{G/H}=(G:H)=\frac{8}{2}=4$

  \begin{minipage}[t]{2in}
    $(0,0)H=H$ \\
    $(1,0)H$ \\
    $(0,1)H$ \\
    $(1,1)H$
  \end{minipage}
  \begin{minipage}[t]{3in}
    $-(1,0)+(0,1)=(-1,1)=(1,1)\notin H$ \\
    $-(1,0)+(1,1)=(0,1)\notin H$ \\
    $-(0,1)+(1,1)=(1,0)\notin H$
  \end{minipage}
\end{example}

\end{document}
