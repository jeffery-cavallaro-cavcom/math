\documentclass[letterpaper,12pt,fleqn]{article}
\usepackage{matharticle}
\pagestyle{empty}
\renewcommand{\o}{\sigma}
\renewcommand{\t}{\tau}
\newcommand{\p}{\phi}
\begin{document}
\section*{Alternating Groups}

\begin{definition}
  The subgroup of $S_n$ consisting of the even permutations of $n$ letters,
  denoted $A_n$, is called the \emph{alternating group} on $n$ letters:
  \[A_n=\{\o\in S_n\mid \o\ \mbox{is an even}\}\]
\end{definition}

\begin{theorem}
  $A_n\le S_n$
\end{theorem}

\begin{theproof}
  Assume $\o,\t\in A_n$
  
  $\o$ and $\t$ can each be expressed as an even number of transpositions \\
  $\o\t$ can be expressed as an even number of transpositions \\
  $\therefore A_n$ is closed under the operation.

  $()$ has length $0$ and so $()\in A_n$ \\
  $\therefore A_n$ has an identity.

  Assume $\o^{-1}$ can be expressed by listing the same transpositions in
  reverse order \\
  So $\o^{-1}$ has an even number of transpositions \\
  $o^{-1}\in A_n$ \\
  $\therefore A_n$ is closed under inverses

  $\therefore A_n\le S_n$
\end{theproof}

\begin{example}
  \begin{tabular}{ll}
    $S_2=\{(),(12)\}$ & $\abs{S_2}=2!=2$ \\
    $A_2=\{()\}$ & $\abs{A_2}=\frac{\abs{S_2}}{2}=\frac{2}{2}=1$ \\
    \\
    $S_3=\{(),(12),(13),(23),(123),(132)\}$ & $\abs{S_3}=3!=6$ \\
    $A_3=\{(),(123),(132)\}$ & $\abs{A_3}=\frac{\abs{S_3}}{2}=\frac{6}{2}=3$ \\
    \\
    $S_4=\{(),6\cdot(ab),8\cdot(abc),6\cdot(abcd),3\cdot(ab)(cd)\}$ &
    $\abs{S_4}=4!=1+6+8+6+3=24$ \\
    $A_4=\{(),8\cdot(abc),3\cdot(ab)(cd)\}$ &
    $\abs{A_4}=\frac{\abs{S_4}}{2}=\frac{24}{2}=12$ \\
  \end{tabular}
\end{example}
\newpage
\begin{theorem}
  $\abs{A_n}=\frac{\abs{S_n}}{2}=\frac{n!}{2}$
\end{theorem}

\begin{theproof}
  Let $B_n=\{\t\in S_n\mid\t\ \mbox{is odd}\}$ \\
  $(1,2)\in B_n$ \\
  Let $\p:A_n\to B_n$ be defined by $\p(\o)=(1,2)\o$

  Assume $\p(\o)=\p(\t)$ \\
  $(1,2)\o=(1,2)\t$ \\
  $\o=\t$ \\
  $\therefore \p$ is one-to-one.

  Assume $\t\in B_n$ \\
  Let $\o=(1,2)\t$ \\
  $\p(\o)=(1,2)(1,2)\t=()\t=\t$ \\
  $\therefore \p$ is one-to-one.

  So $\p$ is a bijection and $\abs{A_n}=\abs{B_n}$ \\
  But $\abs{S_n}=\abs{A_n}+\abs{B_n}=\abs{A_n}+\abs{A_n}=2\abs{A_n}$ \\
  $\therefore \abs{A_n}=\frac{\abs{S_n}}{2}=\frac{n!}{2}$
\end{theproof}

\begin{theorem}
  $A_n$ can be generated by 3-cycles.
\end{theorem}

\begin{theproof}
  Assume $\o\in A_n$ \\
  $\o$ is composed of an even number of transpositions \\
  \begin{description}
  \item Case 1: $(ab)(cd)$

    $(ab)(cd)=(acb)(acd)$

  \item Case 2: $(ab)(ac)$

    $(ab)(ac)=(acb)$
  \end{description}

  Therefore, each pair of transpositions can be condensed into a single
  3-cycle.
\end{theproof}

\end{document}
