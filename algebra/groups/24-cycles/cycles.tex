\documentclass[letterpaper,12pt,fleqn]{article}
\usepackage{matharticle}
\pagestyle{empty}
\renewcommand{\o}{\sigma}
\newcommand{\m}{\mu}
\begin{document}
\section*{Cycles}

\begin{definition}
  To say that $\o\in S_n$ is a \emph{cycle} means that it has at most one
  orbit that contains more than one element.

  The \emph{length} of a cycle is the number of elements in the cycle.

  To say that two cycles are \emph{disjoint} means that they contain no common
  elements.
\end{definition}

\begin{theorem}
  Every permutation $\o$ of a finite set is a product of disjoint cycles.
\end{theorem}

\begin{theproof}
  Let $B_1,B_2,\ldots,B_r$ be the orbits of $\o$ \\
  The $B_k$ are equivalence classes and are thus disjoint \\
  Let $\mu_k(x)=\begin{cases}
  \o(x), & x\in B_k \\
  x, & x\notin B_k \\
  \end{cases}$ \\
  $\o=\prod_{k=1}^r\mu_k$ \\
  But the $\mu_k$ are disjoint cycles \\
  $\therefore \o$ is a product of disjoint cycles.
\end{theproof}

\begin{corollary}
  Composition of disjoint cycles is commutative.
\end{corollary}

\begin{theorem}
  The order of a cycle of length $n$ is $n$.
\end{theorem}

\begin{corollary}
  The order of a permutation is the least common multiple of the orders of its
  disjoint cycles.
\end{corollary}

\begin{theorem}
  Let $\o$ be a cycle represented by $(i_1 i_2 \ldots i_{n-1} i_n)$. The inverse
  of $\o$ is given by:
  \[\o^{-1}=(i_n i_{n-1} \ldots i_2 i_1)\]
\end{theorem}

\end{document}
