\documentclass[letterpaper,12pt,fleqn]{article}
\usepackage{matharticle}
\pagestyle{empty}
\renewcommand{\o}{\sigma}
\renewcommand{\t}{\tau}
\newcommand{\g}{\gamma}
\renewcommand{\i}{\iota}
\begin{document}
\section*{Permutations}

\begin{definition}
  A permutation of a set $A$ is a bijection on $A$.

  $S_A=\{\o:\o\ \mbox{is a permutation of}\ A\}$
\end{definition}

\begin{lemma}
  Let $A$ be a set. Composition of elements in $S_A$ is associative.
\end{lemma}

\begin{theproof}
  Assume $\o,\t,\g\in S_A$ \\
  Assume $x\in A$ \\
  $ ((\o\t)\g)(x)=(\o\t)(\g(x))=\o(\t(\g(x)))=\o((\t\g)(x))=(\o(\t\g))(x)$
\end{theproof}

\begin{lemma}
  Let $A$ be a set. Composition of elements in $S_A$ is closed.
\end{lemma}

\begin{theproof}
  Assume $\o,\t\in S_A$

  Assume $(\o\t)(x)=(\o\t)(y)$ \\
  Assume $\o(\t(x))=\o(\t(y))$ \\
  But $\o$ is a bijection and thus one-to-one \\
  So $\t(x)=\t(y)$ \\
  But $\t$ is a bijection and thus one-to-one \\
  $x=y$ \\
  $\therefore \o\t$ is one-to-one.

  Assume $y\in A$ \\
  $\o$ is onto \\
  So $\exists\,a\in A,\o(a)=y$ \\
  But $\t$ is also onto so $\exists\,x\in A,\t(x)=a$ \\
  $\o(\t(x))=y$ \\
  $(\o\t)(x)=y$ \\
  $\therefore \o\t$ is onto.

  $\therefore \o\t$ is a bijection and thus a permutation on $S_A$

  $\therefore S_A$ is closed under the operation of composition.
\end{theproof}
\newpage
\begin{theorem}
  Let $A\ne\emptyset$. $S_A$ is a group under the operation of composition.
\end{theorem}

\begin{theproof}
  Function composition is closed and associative (lemmas) \\
  $\i_A(x)=x$ is an identity permutation \\
  $\o\in S_A\implies\o^{-1}\in S_A$, since $\o$ is a bijection \\
  $\therefore S_A$ is a group.
\end{theproof}

\begin{definition}
  $[n]=\{1,2,3,\ldots,n\}$

  $S_n=\{\o:\o\ \mbox{is a permutation of}\ [n]\}$
\end{definition}

Note that $\abs{S_n}=n!$.

Permutations can be represented by $2\times n$ matrices, where the top row
contains $1,\ldots,n$ and the bottom row represents how the top row is
permuted.

\begin{example}
  $S_2=\left\{\begin{pmatrix}1 & 2 \\ 1 & 2\end{pmatrix},
  \begin{pmatrix}1 & 2 \\ 2 & 1\end{pmatrix}\right\}$
    
  $\abs{S_2}=2!=2$
\end{example}

Permutations can also be represented by a decomposition of cycles:
\[(abcd\cdots z)=\begin{pmatrix}a & b & c & \ldots & z \\
b & c & d & \ldots & a\end{pmatrix}\]
Elements that do not change are omitted.

The identity permutation is represented by $()$

\begin{example}
  $S_3=\left\{(123),(132),(213),(231),(312),(321)\right\}$

  $\abs{S_3}=3!=6$
\end{example}
\newpage
\begin{example}
  $\begin{array}{lll}
    S_4: & () & 1 \\
    \\
    & (ab) & \frac{4\cdot3}{2}=6 \\
    \\
    & (abc) & \frac{4\cdot3\cdot2}{3}=8 \\
    \\
    & (abcd) & \frac{4\cdot3\cdot2\cdot1}{4}=6 \\
    \\
    & (ab)(cd) & \frac{4\cdot3}{2\cdot2}=3 \\
  \end{array}$

  $\abs{S_4}=4!=1+6+8+6+3=24$
\end{example}

\end{document}
