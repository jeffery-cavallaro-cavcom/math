\documentclass[letterpaper,12pt,fleqn]{article}
\usepackage{matharticle}
\pagestyle{empty}

\begin{document}

\section*{Limits}

\begin{example}

  Consider the quadratic function \(f(x)=x^2-5x+6\):

  \bigskip

  \begin{center}
    \begin{tikzpicture}
      \begin{axis}[
          axis lines=middle,
          xmin=-1,
          xmax=5,
          ymin=-1/2,
          ymax=2,
          ticks=none,
          xlabel={\(x\)},
          ylabel={\(y\)},
          x label style={at={(axis cs:5,0)},anchor=west},
          y label style={at={(axis cs:0,2)},anchor=south},
          clip=false
        ]
        \addplot [domain=1:4,blue] {x^2-5*x+6};
        \node [below left] at (2,0) {\(2\)};
        \node [below right] at (3,0) {\(3\)};
      \end{axis}
    \end{tikzpicture}
  \end{center}

  What happens to \(f(x)\) as \(x\to2\), but \(x\ne 2\)?

  \bigskip

  \begin{center}
    \(\begin{array}{|l|l|}
    \hline
    x & f(x) \\
    \hline
    \hline
    2.1 & -0.09 \\
    \hline
    2.01 & -0.0099 \\
    \hline
    2.001 & -0.000999 \\
    \hline
    2 & \\
    \hline
    1.999 & 0.001001 \\
    \hline
    1.99 & 0.0101 \\
    \hline
    1.9 & 0.11 \\
    \hline
    \end{array}\)
  \end{center}

  \bigskip

  It appears that from either direction, \(f(x)\to 0\).
\end{example}

\end{document}
