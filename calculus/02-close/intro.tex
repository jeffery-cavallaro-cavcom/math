\documentclass[letterpaper,12pt,fleqn]{article}
\usepackage{matharticle}

\renewcommand{\d}{\delta}
\newcommand{\e}{\epsilon}

\pagestyle{empty}

\begin{document}

\section*{Introduction}

There are some problems that algebra alone cannot solve.  A new principle is needed in order to solve these
harder problems.

\subsection*{Arbitrarily Large}

Infinity (\(\infty\)) is not an actual number, but instead is indicative of a process:

\begin{enumerate}
\item Select a positive number.
\item\label{step:large} Now select a next number that is larger than the previous number.
\item Go to \ref{step:large}.
\end{enumerate}

This is possible because the real numbers are unbounded: for every \(y\in\R\) there exists some \(x\in\R\) such
that \(x>y\).

\bigskip

\begin{center}
  \begin{tikzpicture}
    \draw [help lines,<->] (-3,0) -- (3,0) node [right] {\(\infty\)};
    \node (y) [closed point] at (-1,0) {};
    \node [below] at (y) {\(y\)};
    \node (x) [closed point] at (1,0) {};
    \node [below] at (x) {\(x\)};
  \end{tikzpicture}
\end{center}

\begin{definition}[Arbitrarily Large]
  To say that a value \(x\in\R\) is \emph{arbitrarily large}, denoted by \(x\to\infty\), means that for every
  \(y\in\R\), \(x>y\).
\end{definition}

This also works in the negative direction.  For \(x\to-\infty\), select a negative number and then continually
select numbers that are less than the previous number.  In other words, for every \(y\in\R,x<y\).

\bigskip

\begin{center}
  \begin{tikzpicture}
    \draw [help lines,<->] (3,0) -- (-3,0) node [left] {\(-\infty\)};
    \node (y) [closed point] at (1,0) {};
    \node [below] at (y) {\(y\)};
    \node (x) [closed point] at (-1,0) {};
    \node [below] at (x) {\(x\)};
  \end{tikzpicture}
\end{center}

\subsection*{Arbitrarily Small}

A number can also be said to be arbitrarily small.  Like infinity, this is not an actual number, but is indicative
of a process:

\begin{enumerate}
\item Select a positive number.
\item\label{step:small} Now select a next positive number that is smaller than the previous number.
\item Go to \ref{step:small}.
\end{enumerate}

This is possible because between any two real numbers there are an infinite number of real numbers.  Thus, for any
value \(y>0\) there exists some \(x\) such that \(0<x<y\).

\bigskip

\begin{center}
  \begin{tikzpicture}
    \draw [help lines,<->] (-1,0) -- (5,0);
    \node (z) [closed point] at (0,0) {};
    \node [below] at (z) {\(0\)};
    \node (y) [closed point] at (3,0) {};
    \node [below] at (y) {\(y\)};
    \node (x) [closed point] at (1.5,0) {};
    \node [below] at (x) {\(x\)};
  \end{tikzpicture}
\end{center}

\begin{definition}[Arbitrarily Small]
  To say that a value \(x\in\R^+\) is \emph{arbitrarily small}, denoted by \(x\to0^+\), means that for every
  \(y\in\R^+,0<x<y\).
\end{definition}

The Greek letters epsilon (\(\e\)) and delta (\(\d\)) are typically used to represent arbitrarily small values.

\subsection*{Distance}

The first time students are introduced to the concept of \emph{absolute value}, they are given a formula:
\[\abs{x}=\begin{cases}
x, & x\ge0 \\
-x, & x<0
\end{cases}\]
This is a perfectly good definition; however, it lacks meaning.  Instead, consider two points on the real
number line:

\bigskip

\begin{center}
  \begin{tikzpicture}
    \draw [help lines,<->] (-1,0) -- (5,0);
    \node (x) [closed point] at (1,0) {};
    \node [below] at (x) {\(1\)};
    \node (y) [closed point] at (4,0) {};
    \node [below] at (y) {\(4\)};
  \end{tikzpicture}
\end{center}

How does one calculate the \emph{distance} from \(1\) to \(4\):
\[4-1=3\]
How about from \(4\) to \(1\):
\[1-4=-3\]
But distance is an unsigned quantity (different from displacement).  Furthermore, the distance from \(1\) to \(4\)
should be the same as the distance from \(4\) to \(1\).  Thus, we use absolute value:
\[\abs{4-1}=\abs{1-4}=3\]

\begin{definition}[Distance]
  Let \(x,y\in\R\).  The \emph{distance} from \(x\) to \(y\) (and from \(y\) to \(x\)) is given by:
  \[d(x,y)=\abs{x-y}=\abs{y-x}\]
\end{definition}

Thus, \(\abs{x}=\abs{x-0}\), which is the distance from \(x\) to \(0\).

\subsection*{Arbitrarily Close}

\begin{definition}[Arbitrarily Close]
  To say that a value \(x\in\R\) is \emph{arbitrarily close} to another value \(c\in\R\), denoted by \(x\to c\),
  means that for all \(\e>0,\abs{x-c}<\e\).  In other words: \(c-\e<x<c+\e\).
\end{definition}

\bigskip

\begin{center}
  \begin{tikzpicture}
    \draw [help lines,<->] (-3,0) -- (3,0);
    \node (c) [closed point] at (0,0) {};
    \node [below] at (c) {\(c\)};
    \node (cpe) [closed point] at (2,0) {};
    \node [below] at (cpe) {\(c+\e\)};
    \node (cme) [closed point] at (-2,0) {};
    \node [below] at (cme) {\(c-\e\)};
    \node (x) [closed point,red] at (1,0) {};
    \node [below] at (x) {\(x\)};

    \draw [help lines,<->] (-3,-2) -- (3,-2);
    \node (c) [closed point] at (0,-2) {};
    \node [below] at (c) {\(c\)};
    \node (cpe) [closed point] at (2,-2) {};
    \node [below] at (cpe) {\(c+\e\)};
    \node (cme) [closed point] at (-2,-2) {};
    \node [below] at (cme) {\(c-\e\)};
    \node (x) [closed point,red] at (-1,-2) {};
    \node [below] at (x) {\(x\)};
  \end{tikzpicture}
\end{center}

Thus, as \(\e\) gets arbitrarily small, \(x\) gets arbitrarily close to \(c\).

Also important is the negation: there exists an \(\e>0\) such that \(\abs{x-c}\ge\e\).

\begin{theorem}
  Arbitrarily close is equivalent to equality.
\end{theorem}

\begin{proof}
  \begin{description}
    \item[]
    \item[\(\implies\)] Assume that \(x\to c\).

      ABC that \(x\ne c\).  Thus, there exist some \(d>0\) such that \(\abs{x-c}\ge d\).  So let \(\e=d\):
      \[\abs{x-c}\ge d=\e\]
      This means that there exists an \(\e>0\) such that \(\abs{x-c}\ge\e\) and hence \(x\not\to c\), contradicting
      the assumption.

      Therefore \(x=c\).

    \item[\(\impliedby\)] Assume that \(x=c\)

      Assume that \(\e>0\):
      \[\abs{x-c}=0<\e\]
      Therefore \(x\to c\).
  \end{description}
\end{proof}

\subsection*{Problems}

\subsubsection*{Slope}

\subsubsection*{Area}

\subsubsection*{Sequences and Series}

\end{document}
