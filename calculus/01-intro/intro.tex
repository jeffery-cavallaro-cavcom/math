\documentclass[letterpaper,12pt,fleqn]{article}
\usepackage{matharticle}
\usepackage{siunitx}
\pagestyle{empty}

\begin{document}

\section*{Introduction}

There are some problems that algebra alone cannot solve.

\begin{definition}[Rate-of-change Problem]
  A \emph{rate-of-change} problem seeks to determine how much one quantity changes with respect to a change in
  another quantity.
\end{definition}

\begin{examples}[Rate-of-change Problems with respect to Time]
  \begin{itemize}[left=0in]
  \item[]
  \item The velocity (speed) of a moving object: change in position (miles per hour, feet per second).
  \item The rate at which a product is produced during a chemical reaction (grams per second).
  \item The rate of radioactive decay (grams per year).
  \item The rate of population growth (members per year).
  \item The rate of change in the price of a stock during a particular trading day (dollars per hour).
  \end{itemize}
\end{examples}

\begin{examples}[Rate-of-change Problems with respect to Other Quantities]
  \begin{itemize}[left=0in]
  \item[]
  \item The change in gravitational force applied to the earth with respect to changing distance from the sun
    (newtons per kilometer).
  \item The change in magnetic force applied to an iron nail with respect to changing distance from a magnet
    (newtons per centimeter).
  \item Elasticity of demand: the change in the quantity of a commodity sold with respect to a change in price
    (units per dollar).
  \end{itemize}
\end{examples}

Why do we care?
\begin{itemize}[left=0in]
\item Was a car exceeding the speed limit when it passed a checkpoint?
\item Is a rocket moving fast enough to escape the earth's gravity (\(V_e\approx\SI{11}{km/s}\))?
\item Which chemical reaction produces product the fastest?
\item Is a radioactive substance safe for humans?
\item Has a population's reproduction rate fallen beneath the replacement level?
\item Will an increase in the price of a product produce more revenue?
\end{itemize}

\bigskip

Algebraically, these situations are modeled by a function \(y=f(x)\).  The quantity measured is represented by the
dependent variable \(y\) and the quantity causing the change is represented by the independent variable \(x\).
Note that the variable names can change to represent the actual quantities in the problem.  Furthermore, the
function name usually matches the independent variable name.

\begin{example}

  The position \(s\) of a moving body with respect to time \(t\) is modeled by \(s=s(t)\):

  \bigskip

  \begin{center}
    \begin{tikzpicture}
      \draw [help lines,->] (0,0) -- (6,0) node [right] {\(t\)};
      \draw [help lines,->] (0,0) -- (0,4) node [above] {\(s\)};
      \draw [smooth,rounded corners=5mm]
      (0,2) .. controls (1,3) .. (2,3) .. controls (3,3) .. (4,1) .. controls (5,1) .. (6,3.5)
      node [above] {\(s(t)\)};
      \node [closed point] at (2,3) {};
      \draw [dashed] (2,3) -- (2,0);
      \node [below] at (2,0) {\(t_1\)};
      \node [closed point] at (3.5,2) {};
      \draw [dashed] (3.5,2) -- (3.5,0);
      \node [below] at (3.5,0) {\(t_2\)};
      \node [closed point] at (5.5,2.2) {};
      \draw [dashed] (5.5,2.2) -- (5.5,0);
      \node [below] at (5.5,0) {\(t_3\)};
      \draw [dashed] (2,3) -- (0,3);
      \node [left] at (0,3) {\(s(t_1)\)};
    \end{tikzpicture}
  \end{center}

  Determining the body's position \(s\) at some time \(t\) is easy: \(s(t)\).  However, how the position is
  changing at a particular time \(t\) is a completely different question:

  \bigskip

  \begin{center}
    \begin{tabular}{|c|c|}
      \hline
      time & position \\
      \hline
      \hline
      \(t_1\) & constant \\
      \hline
      \(t_2\) & decreasing \\
      \hline
      \(t_3\) & increasing \\
      \hline
    \end{tabular}
  \end{center}
\end{example}

The goal of rate-of-change problems is to quantify the magnitude of such changes.

This is easy when the model is linear: \(y=f(x)\) is a line.

\begin{example}
  Consider a body moving in a straight line at constant velocity \SI{0.75}{ft/sec} with initial position
  \(s(0)=\SI{2}{ft}\).  The equation of motion is given by:
  \[s(t)=\frac{3}{4}t+2\]
  \begin{center}
    \begin{tikzpicture}[scale=0.9]
      \begin{axis}[
          axis lines=middle,
          xmin=0,
          xmax=5,
          ymin=0,
          ymax=8,
          ticks=none,
          xlabel={\(t\)},
          ylabel={\(s\)},
          x label style={at={(axis cs:5,0)},anchor=west},
          y label style={at={(axis cs:0,8)},anchor=south},
          clip=false
        ]
        \addplot [domain=0:5,blue] {(3/4)*x+2} node [right] {\(s(t)\)};
        \node [left] at (0,2) {\(s_0=2\)};
      \end{axis}
    \end{tikzpicture}
  \end{center}

  \bigskip

  The equation of motion is linear and the velocity is the slope of the line.

  From a unit analysis standpoint:
  \[\left(\frac{\si{ft}}{\si{sec}}\right)\si{sec}+\si{ft}=\si{ft}\]
  Thus, the change in position with respect to a change in time at any time \(t\) is simply the slope of the line.
  In fact, this is exactly how the slope of a line is calculated:
  \[v=\frac{\Delta s}{\Delta t}=\frac{s_2-s_1}{t_2-t_1}\]
\end{example}

But what happens when the function is not linear?

\begin{definition}[Rate-of-change of a Function at a Point]
  Let \(f(x)\) be a function.  The \emph{rate-of-change} of the function at a point \((c,f(c))\) is the slope of
  the tangent line to the function at that point.
\end{definition}

\begin{center}
  \begin{tikzpicture}[scale=0.9]
    \begin{axis}[
        axis lines=middle,
        xmin=0,
        xmax=5,
        ymin=0,
        ymax=25,
        ticks=none,
        xlabel={\(x\)},
        ylabel={\(y\)},
        x label style={at={(axis cs:5,0)},anchor=west},
        y label style={at={(axis cs:0,25)},anchor=south},
        clip=false
        ]
      \addplot [domain=0:4.5,blue] {x^2} node [above] {\(f(x)\)};
      \addplot [domain=1.5:4.5,red] {6*x-9};
      \node [closed point] at (3,9) {};
      \node [right=1em] at (3,9) {\((c,f(c))\)};
      \draw [dashed] (3,9) -- (3,0) node [below] {\(c\)};
      \draw [dashed] (3,9) -- (0,9) node [left] {\(f(c)\)};
    \end{axis}
  \end{tikzpicture}
\end{center}

We use this definition because it works.  In the previous example, the tangent line at \((t_1,f(t_1))\) appears to
be horizontal (slope=0), indicating that the function is constant.  At \((t_2,f(t_2))\) the tangent has a negative
slope, indicating that the function is decreasing.  At \((t_3,f(t_3))\) the tangent has a positive slope,
indicating that the function is increasing.  Furthermore, the steeper the function, the steeper the tangent line.

Unfortunately, there is no way using just algebra to determine the slope of a tangent line to a general curve at a
point.

\end{document}
