\documentclass[letterpaper,12pt,fleqn]{article}
\usepackage{matharticle}

\renewcommand{\d}{\delta}
\newcommand{\e}{\epsilon}

\pagestyle{empty}

\begin{document}

\section*{Introduction}

There are some problems that algebra alone cannot solve.  A new principle is needed in order to solve these
harder problems.

\subsection*{Arbitrarily Large}

Infinity (\(\infty\)) is not an actual number, but instead is indicative of a process:

\begin{enumerate}
\item Select a positive number.
\item\label{step:large} Now select a next number that is larger than the previous number.
\item Go to \ref{step:large}.
\end{enumerate}

This is possible because the real numbers are unbounded: for every \(y\in\R\) there exists some \(x\in\R\) such
that \(x>y\).

\bigskip

\begin{center}
  \begin{tikzpicture}
    \draw [help lines,<->] (-3,0) -- (3,0) node [right] {\(\infty\)};
    \node (y) [closed point] at (-1,0) {};
    \node [below] at (y) {\(y\)};
    \node (x) [closed point] at (1,0) {};
    \node [below] at (x) {\(x\)};
  \end{tikzpicture}
\end{center}

\begin{definition}[Arbitrarily Large]
  To say that a value \(x\in\R\) is \emph{arbitrarily large}, denoted by \(x\to\infty\), means that for every
  \(y\in\R\), \(x>y\).
\end{definition}

This also works in the negative direction.  For \(x\to-\infty\), select a negative number and then continually
select numbers that are less than the previous number.  In other words, for every \(y\in\R,x<y\).

\bigskip

\begin{center}
  \begin{tikzpicture}
    \draw [help lines,<->] (3,0) -- (-3,0) node [left] {\(-\infty\)};
    \node (y) [closed point] at (1,0) {};
    \node [below] at (y) {\(y\)};
    \node (x) [closed point] at (-1,0) {};
    \node [below] at (x) {\(x\)};
  \end{tikzpicture}
\end{center}

\subsection*{Arbitrarily Small}

A number can also be said to be arbitrarily small.  Like infinity, this is not an actual number, but is indicative
of a process:

\begin{enumerate}
\item Select a positive number.
\item\label{step:small} Now select a next positive number that is smaller than the previous number.
\item Go to \ref{step:small}.
\end{enumerate}

This is possible because between any two real numbers there are an infinite number of real numbers.  Thus, for any
value \(y>0\) there exists some \(x\) such that \(0<x<y\).

\bigskip

\begin{center}
  \begin{tikzpicture}
    \draw [help lines,<->] (-1,0) -- (5,0);
    \node (z) [closed point] at (0,0) {};
    \node [below] at (z) {\(0\)};
    \node (y) [closed point] at (3,0) {};
    \node [below] at (y) {\(y\)};
    \node (x) [closed point] at (1.5,0) {};
    \node [below] at (x) {\(x\)};
  \end{tikzpicture}
\end{center}

\begin{definition}[Arbitrarily Small]
  To say that a value \(x\in\R^+\) is \emph{arbitrarily small}, denoted by \(x\to0^+\), means that for every
  \(y\in\R^+,0<x<y\).
\end{definition}

The Greek letters epsilon (\(\e\)) and delta (\(\d\)) are typically used to represent arbitrarily small values.

\subsection*{Arbitrarily Close}

\begin{definition}[Arbitrarily Close]
  To say that a value \(x\in\R\) is \emph{arbitrarily close} to another value \(c\in\R\), denoted by \(x\to c\),
  means that for all \(\e>0,\abs{x-c}<\e\).  In other words: \(c-\e<x<c+\e\).
\end{definition}

\bigskip

\begin{center}
  \begin{tikzpicture}
    \draw [help lines,<->] (-3,0) -- (3,0);
    \node (c) [closed point] at (0,0) {};
    \node [below] at (c) {\(c\)};
    \node (cpe) [closed point] at (2,0) {};
    \node [below] at (cpe) {\(c+\e\)};
    \node (cme) [closed point] at (-2,0) {};
    \node [below] at (cme) {\(c-\e\)};
    \node (x) [closed point,red] at (1,0) {};
    \node [below] at (x) {\(x\)};

    \draw [help lines,<->] (-3,-2) -- (3,-2);
    \node (c) [closed point] at (0,-2) {};
    \node [below] at (c) {\(c\)};
    \node (cpe) [closed point] at (2,-2) {};
    \node [below] at (cpe) {\(c+\e\)};
    \node (cme) [closed point] at (-2,-2) {};
    \node [below] at (cme) {\(c-\e\)};
    \node (x) [closed point,red] at (-1,-2) {};
    \node [below] at (x) {\(x\)};
  \end{tikzpicture}
\end{center}

\subsection*{Problems}

\subsubsection*{Slope}

\subsubsection*{Area}

\subsubsection*{Sequences and Series}

\end{document}
