\documentclass[letterpaper,12pt,fleqn]{article}
\usepackage{matharticle}
\pagestyle{empty}
\renewcommand{\u}{\underline}
\begin{document}
\section*{Cardinality}
\begin{definition}
The \emph{cardinality} of a set $A$, denoted $\abs{A}$, is the number of
elements in $A$.
\end{definition}
This is fairly intuitive for finite sets, but what about infinite sets like
$\Z$ and $\R$?
\begin{definition}
To say that two sets $A$ and $B$ are \emph{equivalent}, denoted $A\approx B$,
means there exists a bijection $\phi:A\to B$. When two sets are equivalent they
are said to have the same cardinality, denoted $\abs{A}=\abs{B}$.
\end{definition}
\begin{notation}
\[[0]=\emptyset\]
\[[n]=\{1,2,3,...,n\}\]
\end{notation}
\begin{definition}
To say that a set $A$ is finite means $\exists\,n\in\Z^+,A\approx [n]$.
Otherwise, $A$ is said to be infinite.

To say that $A$ is countable means $A$ is finite or $A\approx Z^+$.
Otherwise, $A$ is said to be uncountable.
\end{definition}
Note that by definition, $\Z^+$ is countably infinite. We denote the cardinality
of $\Z^+$ by:
\[\abs{\Z^+}=\aleph_0\]
\begin{theorem}
$\Z$ is countable, and in fact:
\[\abs{\Z}=\aleph_0\]
\end{theorem}
\begin{theproof}
Consider the following bijection mapping $\Z$ to $\Z^+$:

\begin{tabular}{c|cccccccc}
$\Z$ & 0 & 1 & -1 & 2 & -2 & 3 & -3 & \ldots \\
\hline
$\Z^+$ & 1 & 2 & 3 & 4 & 5 & 6 & 7 & \ldots \\
\end{tabular}

Thus, $\Z\approx\Z^+$ and is therefore countable. Furtherfore:
\[\abs{\Z}=\abs{\Z^+}=\aleph_0\]
\end{theproof}
So, even though $\Z^+\subset\Z$, they have the same cardinality. In fact:
\begin{theorem}
Every subset of a countable set is countable. Furthermore, if the set and the
subset are infinite then they have the same cardinality.
\end{theorem}
\begin{theorem}
$\Q$ is countable, and in fact:
\[\abs{\Q}=\aleph_0\]
\end{theorem}
\begin{theorem}
Arrange and traverse the elements of $\Q$ as follows:

\begin{tabular}{cccccccccc}
0 & & 1 & $\rightarrow$ & -1 & & 2 & $\rightarrow$ & -2 & \ldots \\
  $\downarrow$ & & $\uparrow$ & & $\downarrow$ & & $\uparrow$ & &
    $\downarrow$ & \\
$\frac{1}{2}$ & $\rightarrow$ & $-\frac{1}{2}$ & &
    $\frac{3}{2}$ & & $-\frac{3}{2}$ & & $\frac{5}{2}$ & \ldots \\
  & & & & $\downarrow$ & & $\uparrow$ & & $\downarrow$ & \\
$\frac{1}{3}$ & $\leftarrow$ & $-\frac{1}{3}$ & $\leftarrow$ &
    $\frac{2}{3}$ & & $-\frac{2}{3}$ & & $\frac{4}{3}$ & \ldots \\
  $\downarrow$ & & & & & & $\uparrow$ & & $\downarrow$ & \\
$\frac{1}{4}$ & $\rightarrow$ & $-\frac{1}{4}$ & $\rightarrow$ &
    $\frac{3}{4}$ & $\rightarrow$ & $-\frac{3}{4}$ & & $\frac{5}{4}$ & \ldots \\
  & & & & & & & & $\downarrow$ & \\
$\frac{1}{5}$ & $\leftarrow$ & $-\frac{1}{5}$ & $\leftarrow$ &
    $\frac{2}{5}$ & $\leftarrow$ & $-\frac{2}{5}$ & $\leftarrow$ &
    $\frac{3}{5}$ & \ldots \\
  $\downarrow$ & & & & & & & & & \\
$\vdots$ & & $\vdots$ & & $\vdots$ & & $\vdots$ & & $\vdots$ & \\
\end{tabular}

Note that every rational number is counted once; none are missed. \\
Therefore, $\Q\approx\Z^+$ and $\abs{\Q}=\abs{\Z^+}=\aleph_0$.
\end{theorem}
\begin{theorem}[Cantor Diagonalization]
$[0,1]$ is uncountable, and thus $\R$ is uncountable.
\end{theorem}
\begin{theproof}
ABC that $\R$ is countable. \\
Consider $[0,1]\subset\R$; it must also be countable. \\
Let the following represent an exhaustive list of the elements in $[0,1]$: \\
\[0.\u{a_{1,1}}a_{1,2}a_{1,3}a_{1,4}a_{1,5}\ldots\]
\[0.a_{2,1}\u{a_{2,2}}a_{2,3}a_{2,4}a_{2,5}\ldots\]
\[0.a_{3,1}a_{3,2}\u{a_{3,3}}a_{3,4}a_{3,5}\ldots\]
\[0.a_{4,1}a_{4,2}a_{4,3}\u{a_{4,4}}a_{4,5}\ldots\]
\[0.a_{5,1}a_{5,2}a_{5,3}a_{5,4}\u{a_{5,5}}\ldots\]
\[\hspace{5in}\vdots\]
Now, consider a real number $x\in[0,1]$ whose $n^{th}$ digit differs from the
$n^{th}$ digit in the $n^{th}$ number in the list. Such a number is not in the
original list, and thus the list can never be exhaustive. Thus, $[0,1$] is not
countable, and therefore neither is $\R$.
\end{theproof}
The uncountable cardinality of $\R$ is simply denoted by $\abs{\R}$, and the
cardinality of $\R^n$ is denoted by $\abs{\R}^n$, where:
\[\aleph_0<\abs{\R}<\abs{\R}^2<\ldots\]
\end{document}
