\documentclass[letterpaper,12pt,fleqn]{article}
\usepackage{matharticle}
\usepackage{multicol}
\pagestyle{empty}
\begin{document}
\section*{Equivalence Relations and Partitions}
\begin{definition}
An \emph{equivalence relation} on a set $A$ is a relation on $A$ that satisfies
the following three properties:
\begin{enumerate}
\item \textbf{Reflexive}
\[\forall\,a\in A,a\sim a\]
\item \textbf{Symmetric}
\[\forall\,a,b\in A,a\sim b\implies b\sim a\]
\item \textbf{Transitive}
\[\forall\,a,b,c\in A,a\sim b\ \mbox{and}\ b\sim c\implies a\sim c\]
\end{enumerate}
\end{definition}
\begin{definition}
To say that two sets $A$ and $B$ are \emph{disjoint} means:
\[A\cap B=\emptyset\]

To say that a family of indexed sets $\{A_i\mid i\in I\}$ is
\emph{mutually disjoint} means:
\[\forall\,i,j\in I,i\ne j\implies A_i\cap A_j=\emptyset\]
\end{definition}
\begin{definition}
Let $A$ be a non-empty set. To say that an indexed family of sets
$\{A_i\mid i\in I\}$ \emph{partitions} $A$ means:
\begin{enumerate}
\item All of the $A_i$ are non-empty:
\[\forall\,i\in I,A_i\ne\emptyset\]
\item The $A_i$ are mutually disjoint:
\[\forall\,i,j\in I,i\ne j\implies A_i\cap A_j=\emptyset\]
\item The union of the $A_i$ equals $A$:
\[A=\bigcup_{i\in I}A_i\]
\end{enumerate}
Each $A_i$ is called a \emph{cell} of the partition.
\end{definition}
\begin{definition}
An \emph{equivalence class} of an equivalence relation on a set $A$ is given by:
\[\bar{a}=\{b\in A\mid a\sim b\}\]
\end{definition}
\begin{theorem}
An equivalence relation on a set $A$ defines a partition of $A$, where the
equivalence classes of the equivalence relation are the cells of the partition.
\end{theorem}
\begin{theproof}
\listbreak
\begin{enumerate}
\item Assume $\bar{a}$ is an equivalence class. \\
$a\in\bar{a}$ \\
$\bar{a}\ne\emptyset$ \\
Therefore, the equivalence classes are non-empty.

\item Assume $\bar{a}\cap\bar{b}\ne\emptyset$ \\
$\exists\,y\in A,y\in\bar{a}\cap\bar{b}$ \\
$y\in\bar{a}$ and $y\in\bar{b}$ \\
$y\sim a$ and $y\sim b$ \\

\begin{multicols}{2}
Assume $x\in\bar{a}$ \\
$x\sim a$ \\
$a\sim y$ \\
$x\sim y$ \\
$x\sim b$ \\
$x\in\bar{b}$ \\

\columnbreak
Assume $x\in\bar{b}$ \\
$x\sim b$ \\
$b\sim y$ \\
$x\sim y$ \\
$x\sim a$ \\
$x\in\bar{a}$ \\
\end{multicols}

$\bar{a}\cap\bar{b}\ne\emptyset\implies\bar{a}=\bar{b}$ \\
$\bar{a}\ne\bar{b}\implies\bar{a}\cap\bar{b}=\emptyset$ \\
Therefore, the equivalence classes are mutually disjoint.

\item
\begin{multicols}{2}
Assume $x\in\bigcap_{a\in A}\bar{a}$ \\
$\exists\,a\in A,x\in\bar{a}$ \\
$x\in A$

\columnbreak
Assume $x\in A$ \\
$x\sim x$ \\
$x\in\bar{x}$ \\
Let $a=x$ \\
$\exists\,a\in A,x\in\bar{a}$ \\
$x\in\bigcap_{a\in A}\bar{a}$
\end{multicols}
Therefore, $\bigcap_{a\in A}\bar{a}=A$
\end{enumerate}
\end{theproof}
\newpage
\begin{theorem}
A partition on a set $A$ defines an equivalence relation on $A$, where the
cells of the partition are the equivalence classes for the equivalence
relation.
\end{theorem}
\begin{theproof}
Assume $\{A_i\mid i\in I\}$ is a partition of $A$.
\begin{enumerate}
\item Assume $a\in A$ \\
$a\in\bigcap_{i\in I}A_i$ \\
$\exists\,i\in I,a\in A_i$ \\
$a\in A_i$ and $a\in A_i$ \\
$a\sim a$ \\
Therefore, the relation is reflexive.

\item Assume $a\sim b$ \\
$\exists\,i\in I,a\in A_i$ and $b\in A_i$ \\
$b\in A_i$ and $a\in A_i$ \\
$b\sim a$ \\
Therefore, the relation is symmetric.

\item Assume $a\sim b$ and $b\sim c$ \\
$\exists\,i\in I,a\in A_i$ and $b\in A_i$ \\
$\exists\,j\in I,b\in A_j$ and $c\in A_j$ \\
$b\in A_i$ and $b\in A_j$ \\
$A_i\cap A_j\ne\emptyset$ \\
$A_i=A_j$ \\
$a\in A_i$ and $c\in A_i$ \\
$a\sim c$ \\
Therefore, the relation is transitive.
\end{enumerate}
\end{theproof}
\end{document}
