\documentclass[letterpaper,12pt,fleqn]{article}
\usepackage{matharticle}
\pagestyle{empty}
\begin{document}
\section*{Relations and Functions}
\begin{definition}
A relation $\Re$ between two sets $A$ and $B$ is a subset of $A\times B$:
\[\Re\subseteq A\times B\]
To say that $a\in A$ is related to $b\in B$, often denoted $a\sim b$, means
$(a,b)\in\Re$.

A relation between a set $A$ and itself ($\Re\subseteq A\times A$) is referred
to as a relation \emph{on} $A$.
\end{definition}
\begin{definition}
A function $\phi$ is a relation between sets $A$ and $B$ such that:
\begin{enumerate}
\item $\forall\,a\in A,\exists\,b\in B,(a,b)\in\phi$
\item $\forall\,a\in A,\forall\,b,c\in B,(a,b)\in\phi\ \mbox{and}\ (a,c)\in\phi
    \implies b=c$
\end{enumerate}
The function is said to map $A$ into $B$ and is denoted $\phi:A\to B$, where $A$
is called the \emph{domain} of $\phi$ and $B$ is called the \emph{codomain} of
$\phi$. Also, $(a,b)\in\phi$ is denoted using the more conventional functional
notation: $\phi(a)=b$.

A function between a set $A$ and itself is referred to as a function \emph{on}
$A$.
\end{definition}
\begin{definition}
Let $\phi:A\to B$ with $C\subseteq A$ and $D\subseteq B$:
\begin{itemize}
\item The \emph{image} of $C$ under $\phi$, denoted $\phi[C]$, is give by:
\[\phi[C]=\{\phi(c)\mid c\in C\}\subseteq B\]
$\phi[A]$ is called the \emph{range} of $\phi$.

\item The \emph{pre-image} of $D$ under $\phi$, denoted $\phi^{-1}[D]$, is given
by:
\[\phi^{-1}[D]=\{a\in A\mid\phi(a)\in D\}\subseteq A\]
Note that $\phi^{-1}[D]$ is a (possibly empty) set and should not be confused
with any inverse function $\phi^{-1}:B\to A$.
\end{itemize}
\end{definition}
\begin{definition}
Let $\phi:A\to B$ and $\mu:C\to D$. To say that $\phi$ and $\mu$ are equal,
denoted $\phi=\mu$, means:
\begin{enumerate}
\item $A=C$ (same domain)
\item $B=D$ (same codomain)
\item $\forall\,a\in A,\phi(a)=\mu(a)$
\end{enumerate}
\end{definition}
\begin{definition}
Let $\phi:A\to B$:
\begin{itemize}
\item To say that $\phi$ is \emph{one-to-one} (\emph{injective}) means:
\[\forall\,a,b\in A,\phi(a)=\phi(b)\implies a=b\]
\item To say that $\phi$ is \emph{onto} (\emph{surjective}) means $\phi[A]=B$,
or:
\[\forall\,b\in B,\exists\,a\in A,\phi(a)=b\]
\item To say that $\phi$ is a \emph{one-to-one correspondence}
(\emph{bijective}) means $\phi$ is both one-to-one and onto.
\end{itemize}
\end{definition}
\begin{definition}
The \emph{identity} function on a set $A$, denoted $i_A$, is given by:
\[\forall\,a\in A,\phi(a)=a\]
\end{definition}
\begin{definition}
Let $\phi:A\to B$ and $\mu:B\to C$. The composition of $\phi$ and $\mu$,
denoted $\mu\circ\phi$ or simply $\mu\phi$, is a new function $\mu\phi:A\to C$
given by:
\[\forall\,a\in A,(\mu\phi)(a)=\mu[\phi(a)]\]
\end{definition}
\begin{theorem}
Function composition is associative.
\end{theorem}
\begin{theproof}
Let $\phi:A\to B$, $\mu:B\to C$, and $\gamma:C\to D$. \\
Assume $a\in A$ \\
\[[(\gamma\mu)\phi)](a)=(\gamma\mu)[\phi(a)]=\gamma\{\mu[\phi(a)]\}=
    \gamma[\mu\phi(a)]=[\gamma(\mu\phi)](a)\]
\end{theproof}
\begin{definition}
Let $\phi:A\to B$. To say that $\phi^{-1}:B\to A$ is an inverse function for
$\phi$ means:
\begin{enumerate}
\item $\phi^{-1}\phi=i_A$
\item $\phi\phi^{-1}=i_B$
\end{enumerate}
Note that an inverse must be both a left and a right inverse; there are cases
where only one side exists.
\end{definition}
\begin{theorem}
Let $\phi:A\to B$:
\[\phi\ \mbox{is invertible iff}\ \phi(a)=b\iff\phi^{-1}(b)=a\]
\end{theorem}
\begin{theproof}
\listbreak
\begin{description}
\item $\implies$ Assume $\phi$ is invertible.
\begin{description}
\item $\implies$ Assume $\phi(a)=b$
\begin{eqnarray*}
(\phi^{-1}\phi)(a) &=& \phi^{-1}(b) \\
\therefore a &=& \phi^{-1}(b) \\
\end{eqnarray*}
\listbreak
\item $\impliedby$ Assume $\phi^{-1}(b)=a$
\begin{eqnarray*}
(\phi\phi^{-1})(b) &=& \phi(a) \\
\therefore b &=& \phi(a) \\
\end{eqnarray*}
\end{description}
\listbreak
\item $\impliedby$ Assume $\phi(a)=b\iff\phi^{-1}(b)=a$
\[(\phi^{-1}\phi)(a)=\phi^{-1}[\phi(a)]=\phi^{-1}(b)=a=i_A(a)\]
\[(\phi\phi^{-1})(b)=\phi[\phi^{-1}(b)]=\phi(a)=b=i_B(b)\]
\end{description}
\end{theproof}
\begin{theorem}
Let $\phi:A\to B$:
\[\phi\ \mbox{is invertible}\iff\phi\ \mbox{is bijective}\]
\end{theorem}
\begin{theproof}
\listbreak
\begin{description}
\item $\implies$ Assume $\phi$ is invertible

Assume $\phi(a)=\phi(b)$ \\
\[(\phi^{-1}\phi)(a)=(\phi^{-1}\phi)(b)\]
\[a=b\]
$\therefore \phi$ is one-to-one
\newpage
$\phi^{-1}$ is a function with domain $B$ \\
Assume $b\in B$ \\
$\exists\,a\in A,\phi^{-1}(b)=a$ \\
\[(\phi\phi^{-1})(b)=\phi(a)\]
\[b=\phi(a)\]
$\therefore \phi$ is onto

$\therefore \phi$ is bijective
\item $\impliedby$ Assume $\phi$ is bijective

Assume $a\in A$ \\
Since $\phi$ is a function, $\exists\,b\in B,\phi(a)=b$ \\
But, since $\phi$ is one-to-one, $\exists\,\phi^{-1},\phi^{-1}(b)=a$. \\
($\phi\phi^{-1})(b)=\phi(a)=b$

Assume $b\in B$ \\
But, since $\phi$ is onto, $\exists\,a\in A,\phi(a)=b$ \\
($\phi^{-1}\phi)(a)=\phi^{-1}(b)=a$

$\therefore \phi^{-1}$ exists and $\phi$ is invertible.
\end{description}
\end{theproof}
\begin{theorem}
Inverse functions are unique.
\end{theorem}
\begin{theproof}
Assume $\phi:A\to B$ is invertible \\
Assume $\mu:B\to A$ and $\gamma:B\to A$ are both inverses of $\phi$ \\
Assume $b\in B$
\begin{eqnarray*}
(\phi\mu)(b) &=& b \\
\left[\gamma(\phi\mu)\right](b) &=& \gamma(b) \\
\left[(\gamma\phi)\mu)\right](b) &=& \gamma(b) \\
(\gamma\phi)[\mu(b)] &=& \gamma(b) \\
\mu(b) &=& \gamma(b) \\
\therefore \mu &=& \gamma \\
\end{eqnarray*}
\end{theproof}
\end{document}
