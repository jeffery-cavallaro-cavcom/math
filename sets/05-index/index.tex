\documentclass[letterpaper,12pt,fleqn]{article}
\usepackage{matharticle}
\usepackage{tikz}
\pagestyle{empty}
\theoremstyle{mathitem}
\newtheorem*{operations}{Operations}
\begin{document}
\section*{Index Sets}
\begin{definition}
Let $I$ be a set. The collection of sets given by:
\[\{A_i\mid i\in I\}\]
is called an indexed family of sets and $I$ is called an index set.

Note that $I$ may be infinite.
\end{definition}
\begin{operations}
\listbreak
\begin{enumerate}
\item Union
\[\bigcup_{i\in I}A_i=\{a\mid\exists i\in I,a\in A_i\}\]
\item Intersection
\[\bigcap_{i\in I}A_i=\{a\mid\forall i\in I,a\in A_i\}\]
\end{enumerate}
\end{operations}
\begin{example}
\[\mbox{Let}\ I=Z^{+}\ \mbox{and}\ S_i=[0,i]=\{x\in\R\mid0\le x\le i\}\]
\begin{figure}[h]
\setlength{\leftskip}{0.5in}
\begin{tikzpicture}
\draw (-1,0) -- (3,0);
\draw [fill=black] (0,0) circle [radius=0.1];
\draw [fill=black] (1.5,0) circle [radius=0.1];
\node [below] at (0,0) {0};
\node [below] at (1.5,0) {i};
\draw [ultra thick] (0,0) -- (1.5,0);
\end{tikzpicture}
\end{figure}
\[S=\bigcup_{i\in I}S_i=[0,\infty)=\{x\in\R\mid x\ge0\}\]
\begin{figure}[h]
\setlength{\leftskip}{0.5in}
\begin{tikzpicture}
\draw (-1,0) -- (3,0);
\draw [fill=black] (0,0) circle [radius=0.1];
\node [below] at (0,0) {0};
\draw [ultra thick,->] (0,0) -- (3,0);
\end{tikzpicture}
\end{figure}
\[T=\bigcap_{i\in I}S_i=[0,1]=\{x\in\R\mid0\le x\le1\}\]
\begin{figure}[h]
\setlength{\leftskip}{0.5in}
\begin{tikzpicture}
\draw (-1,0) -- (3,0);
\draw [fill=black] (0,0) circle [radius=0.1];
\draw [fill=black] (1,0) circle [radius=0.1];
\node [below] at (0,0) {0};
\node [below] at (1,0) {1};
\draw [ultra thick] (0,0) -- (1,0);
\end{tikzpicture}
\end{figure}
\newpage
\[S-T=(1,\infty)=\{x\in\R\mid x>1\}\]
\begin{figure}[h]
\setlength{\leftskip}{0.5in}
\begin{tikzpicture}
\draw (-1,0) -- (3,0);
\draw (1,0) circle [radius=0.1];
\node [below] at (1,0) {1};
\draw [ultra thick,->] (1,0) -- (3,0);
\end{tikzpicture}
\end{figure}
\end{example}
\end{document}
