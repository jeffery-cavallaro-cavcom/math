\documentclass[letterpaper,12pt,fleqn]{article}
\usepackage{matharticle}
\pagestyle{empty}
\newcommand{\U}{\mathcal{U}}
\newcommand{\E}{\emptyset}
\begin{document}
\section*{Sets}
\begin{definition}
A set is a \emph{well-defined}, \emph{distinct}, and \emph{unordered}
collection of members called elements, where the elements are selected from
an all-inclusive set called a universe.
\end{definition}
This definition is poor because it is circular, since it uses the word ``set''
in the definition, and incomplete because it appeals to the reader's intuition
of what it means to be a collection and a member of a collection. This is due to
limitations in the recursive nature of the definition of things via language.
Thus, this approach to sets is often referred to as na\"{i}ve set theory.
\begin{properties}
\listbreak
\begin{enumerate}
\item To say that a set $A$ is \emph{well-defined} means that every element $a$
from the universe $\U$ is unambiguously either in set $A$ ($a\in A$) or not in
set $A$ ($a\notin A$):
\[\forall a\in\U,a\in A\ \mbox{or}\ a\notin A\]

\item The elements of a set are distinct; if two elements are equal then they
are considered to be the \emph{same} element---not two separate elements.

\item The elements of a set are unordered; when listing the elements of a set,
any convenient ordering is acceptable and all orderings represent the same set.
\end{enumerate}
\end{properties}
\begin{notation}
Sets can be specified in any of the following ways:
\begin{enumerate}
\item \textbf{Description}. The use of words to describe the elements of a set:
\[A=\mbox{the set of natural numbers from 2 to 5, inclusive}\]
\item \textbf{Roster}. A comma-separated list of the elements enclosed in
curly-braces. Ellipses ($\ldots$) can be used to represent omitted elements
when the pattern of the elements is obvious, or in the case of infinite sets:
\[A=\{1,2,3\}\]
\[B=\{1,2,3,\ldots,10\}\]
\[\N=\{1,2,3,\ldots\}\]
\[\Z=\{\ldots,-3,-2,-1,0,1,2,3,\ldots\}\]
\newpage
\item \textbf{Setbuilder Notation}.
\begin{enumerate}
\item Universe/Condition
\[E=\{e\in\Z\mid e=2n,n\in\Z\}\]
Although sometimes the universe is omitted when obvious or understood:
\[E=\{e\mid e=2n,n\in\Z\}\]
\item Pattern/Qualifier
\[E=\{2n\mid n\in\Z\}\]
\end{enumerate}
\end{enumerate}
\end{notation}
\begin{definition}
The set with no elements is called the \emph{empty set} and is denoted by $\E$:
\[\E=\{\}\]
\end{definition}
Sometimes we will say something like, ``let S be a set of$\ldots$,'' where any
possible such set is acceptable; however, any such set selected will always be
well-defined.
\end{document}
