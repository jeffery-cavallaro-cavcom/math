\documentclass[letterpaper,12pt,fleqn]{article}
\usepackage{matharticle}
\pagestyle{empty}
\newcommand{\U}{\mathcal{U}}
\newcommand{\E}{\emptyset}
\begin{document}
\section*{Subsets and Equality}
\begin{definition}
To say that a set $A$ is a \emph{subset} of a set $B$, denoted $A\subseteq B$,
means:
\[\forall a\in A,a\in B\]
or more conveniently for proofs:
\[x\in A\implies x\in B\]
\end{definition}
\begin{definition}
To say that a set $A$ equals a set $B$, denoted $A=B$, means:
\[A\subseteq B\ \mbox{and}\ B\subseteq A\]
or more conveniently for proofs:
\[x\in A\iff x\in B\]
\end{definition}
\begin{definition}
To say that $A$ is a \emph{proper} subset of $B$, denoted $A\subset B$, means
that $A\subseteq B$ but $A\ne B$. If $A=B$ then $A$ is called an \emph{improper}
subset of $B$.
\end{definition}
\begin{theorem}
For all sets $A$:
\begin{enumerate}
\item $\E\subseteq A$
\item $A\subseteq A$
\item $A\subseteq\U$
\end{enumerate}
The proofs follow trivially from the definitions.
\end{theorem}
\end{document}
