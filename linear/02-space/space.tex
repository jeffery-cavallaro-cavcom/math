\documentclass[letterpaper,12pt,fleqn]{article}
\usepackage{matharticle}
\pagestyle{empty}
\newcommand{\vx}{\vec{x}}
\newcommand{\vxp}{\vec{x'}}
\newcommand{\vy}{\vec{y}}
\newcommand{\vz}{\vec{z}}
\newcommand{\vi}{\vec{0}}
\newcommand{\F}{\mathbb{F}}
\begin{document}
\section*{Vector Space}

\begin{definition}[Vector Space]
  A \emph{vector (linear) space} is an algebraic structure that consists of:
  \begin{enumerate}
  \item A set of objects $V$ called vectors.
  \item A field $\F$ called scalars.
  \item An operation of vector addition ($\vx+\vy$).
  \item An operation of scalar multiplication ($c\vx$).
  \end{enumerate}
  such that $\forall\,\vx,\vy,\vz\in V$ and $\forall\,a,b\in F$ the following ten
  properties hold:
  \begin{enumerate}
  \item Additive Closure: $\vx+\vy\in V$
  \item Additive Commutativity: $\vx+\vy=\vy+\vx$
  \item Additive Associativity: $(\vx+\vy)+\vz=\vx+(\vy+\vz)$
  \item Additive Identity: $\exists\,\vi\in V,\vx+\vi=\vi+\vx=\vx$
  \item Additive Inverse: $\exists\,(-\vx)\in V,\vx+(-\vx)=(-\vx)+\vx=\vi$
  \item Multiplicative Closure: $a\vx\in V$
  \item Multiplicative Inverse: $1\vx=\vx$
  \item Multiplicative Associativity $(ab)\vx=a(b\vx)$
  \item Scalar (Left) Distributivity: $a(\vx+\vy)=a\vx+a\vy$
  \item Vector (Right) Distributivity: $(a+b)\vx=a\vx+b\vx$
  \end{enumerate}
\end{definition}

Typical choices for scalar fields are the infinite fields: $\Q, \R,$ and $\C$; or a finite
field like $\Z_2$, but not $\Z$, which is only a ring.

\begin{example}
  \listbreak
  \begin{enumerate}
  \item $\F^n$, where the scalars are from $\F$ and the vectors are n-tuples of
    elements from $\F$, usually represented by column vectors. Addition and
    multiplication are component-wise:
    \[\vx+\vy=\begin{bmatrix}x_1 \\ x_2 \\ \vdots \\ x_n\end{bmatrix}+
    \begin{bmatrix}y_1 \\ y_2 \\ \vdots \\ y_n\end{bmatrix}=
      \begin{bmatrix}x_1+y_1 \\ x_2+y_2 \\ \vdots \\ x_n+y_n\end{bmatrix}\hspace{0.5in}
        c\vx=c\begin{bmatrix}x_1 \\ x_2 \\ \vdots \\ x_n\end{bmatrix}=
        \begin{bmatrix}cx_1 \\ cx_2 \\ \vdots \\ cx_n\end{bmatrix}\]
\newpage
  \item $M_{m\times n}(\F)$, where the scalars are from $\F$ and the vectors are
    $m\times n$ matrices whose components are also from $\F$. Addition and multiplication
    are the standard matrix operations:
    \[(A+B)_{ij}=A_{ij}+B_{ij}\]
    \[(cA)_{ij}=cA_{ij}\]

  \item $\mathcal{F}(S,\F)$, where the scalars are from $\F$ and the vectors are
    functions with domain $S$ and codomain $\F$. Addition and multiplication are the
    standard function operations:
    \[(f+g)(s)=f(s)+g(s)\]
    \[(cf)(s)=cf(s)\]

  \item $\F[x]$, where the scalars are from $\F$ and the vectors are polynomials with
    coefficients from $\F$. Note that this is an example of $\mathcal{F}(S,\F)$, so
    addition and multiplication are the standard function operations as well.
  \end{enumerate}
\end{example}

\newpage

\subsection*{Properties}

\begin{theorem}[Cancellation Rules]
  Let $V$ be a vector space over a field $F$. The cancellation rules hold in $V$:

  $\forall\,\vx,\vy,\vz\in V$ and $\forall\,a\in\F-\{0\}$:
  \begin{enumerate}
  \item Right: $\vx+\vz=\vy+\vz\implies\vx=\vy$
  \item Left: $\vz+\vx=\vz+\vy\implies\vx=\vy$
  \item Scalar: $a\vx=a\vy\implies\vx=\vy$
  \end{enumerate}
\end{theorem}

\begin{theproof}
  Assume $\vx,\vy,\vz\in V$ and $a\in\F-\{0\}$:
  \begin{enumerate}
  \item Assume $\vx+\vz=\vy+\vz$

    $\exists\,(-z)\in V$ \\
    $(\vx+\vz)+(-\vz)=(\vy+\vz)+(-\vz)$ \\
    $\vx+[\vz+(-\vz)]=\vy+[\vz+(-\vz)]$ \\
    $\vx+\vi=\vy+\vi$
    
    $\therefore\vx=\vy$

  \item Assume $\vz+\vx=\vz+\vy$

    $\vx+\vz=\vy+\vz$

    $\therefore \vx=\vy$

  \item Assume $a\vx=a\vy$

    Since $a\ne0$, $\exists\,a^{-1}\in\F$ \\
    $a^{-1}(a\vx)=a^{-1}(a\vy)$ \\
    $(a^{-1}a)\vx=(a^{-1}a)\vy$ \\
    $1\vx=1\vy$

    $\therefore\vx=\vy$
  \end{enumerate}
\end{theproof}

\newpage

\begin{theorem}[Zero]
  Let $V$ be a vector space over a field $\F$:
  
  $\forall\,\vx\in V$ and $\forall\,c\in\F$:
  \begin{enumerate}
  \item $\vi\in V$ is unique
  \item $(-\vi)=\vi$
  \item $0\vx=\vi$
  \item $c\vi=\vi$
  \end{enumerate}
\end{theorem}

\begin{theproof}
  Assume $\vx\in V$ and $c\in\F$
  \begin{enumerate}
  \item Assume $\vi,\vi'\in V$ are both additive identities in $V$
    
    $\vx+\vi=\vx$ \\
    $\vx+\vi'=\vx$ \\
    $\vx+\vi=\vx+\vi'$
  
    $\therefore\vi=\vi'$

  \item

    $\vi+\vi=\vi$ \\
    $\vi+(-\vi)=\vi$ \\
    $\vi+(-\vi)=\vi+\vi$

    $\therefore (-\vi)=\vi$

  \item

    $0\vx=(0+0)\vx=0\vx+0\vx$ \\
    $0\vx=0\vx+\vi$ \\
    $0\vx+0\vx=0\vx+\vi$

    $\therefore 0\vx=\vi$

  \item

    $c\vi=c(\vi+\vi)=c\vi+c\vi$ \\
    $c\vi=c\vi+\vi$ \\
    $c\vi+c\vi=c\vi+\vi$

    $\therefore c\vi=\vi$
  \end{enumerate}
\end{theproof}

\newpage

\begin{theorem}[Inverses]
  Let $V$ be a vector space over a field $\F$:
  
  $\forall\,\vx\in V$ and $\forall\,c\in\F$:
  \begin{enumerate}
  \item $(-\vx)$ is unique
  \item $(-1)\vx=(-\vx)$
  \item $(-c)\vx=-(c\vx)=c(-\vx)$
  \end{enumerate}
\end{theorem}

\begin{theproof}
  Assume $\vx\in V$ and $c\in\F$
  \begin{enumerate}
    \item Assume $(-\vx),(-\vxp)\in V$ are both additive inverses for $\vx$

      $\vx+(-\vx)=\vi$ \\
      $\vx+(-\vxp)=\vi$ \\
      $\vx+(-\vx)=\vx+(-\vxp)$

      $\therefore(-\vx)=(-\vxp)$

    \item

      $(-1)\vx+\vx=(-1)\vx+1\vx=[(-1)+1]\vx=0\vx=\vi$ \\
      But inverses are unique

      $\therefore (-1)\vx=(-\vx)$

    \item

      $(-c)\vx+c\vx=[(-c)+c]\vx=0\vx=\vi$ \\
      But additive inverses are unique

      $\therefore (-c)x=-(cx)$

      In particular, let $c=1$ \\
      $c(-\vx)=c[(-1)\vx]=[c\cdot(-1)]\vx=(-c)\vx$

      $\therefore (-c)\vx=-(c\vx)=c(-\vx)$
  \end{enumerate}
\end{theproof}

\end{document}
