\documentclass[letterpaper,12pt,fleqn]{article}
\usepackage{matharticle}
\usepackage{arydshln}
\pagestyle{empty}
\begin{document}
\section*{Elementary Row Operations}

\begin{definition}
  A \emph{matrix} is a rectangular collection of objects organized into
  \emph{rows} and \emph{columns}. Matrices are determined by the type of their
  objects and their size: an $m\times n$ matrix has $m$ rows and $n$ columns.
  When the objects are from $\R$ or $\C$ then matrices are useful for
  representing and solving SOLEs.
\end{definition}

\begin{example}
  \begin{minipage}{1in}
    $2x-y=4$ \\
    $x+2y=8$
  \end{minipage}
  \begin{minipage}{0.5in}
    \begin{center}
      $\implies$
    \end{center}
  \end{minipage}
  \begin{minipage}[t]{0.75in}
    \begin{center}
      $\begin{bmatrix}
        2 & -1 \\
        1 & 2
      \end{bmatrix}$

      \bigskip

      Coefficient \\
      Matrix
    \end{center}
  \end{minipage}
  \begin{minipage}{0.5in}
    \begin{center}
      $\implies$
    \end{center}
  \end{minipage}
  \begin{minipage}[t]{1in}
    \begin{center}
      $\begin{bmatrix}[cc:c]
        2 & -1 & 4 \\
        1 & 2 & 8
      \end{bmatrix}$

      \bigskip

      Augmented \\
      Matrix
    \end{center}
  \end{minipage}
\end{example}

An $m\times n$ augmented matrix represents a SOLE with $m$ equations and
$n-1$ unknowns.

\begin{definition}
The following are called the \emph{elementary row operations} (EROs):
\begin{enumerate}
\item Interchange: $R_i\leftrightarrow R_j$
\item Scaling: $cR_i\rightarrow R_i$ $(c\ne0)$
\item Replacement: $cR_i+R_j\rightarrow R_j$
\end{enumerate}
\end{definition}

EROs are reversible and do not change the solution set of an SOLE:
\begin{enumerate}
\item $R_j\leftrightarrow R_i$
\item $\frac{1}{c}(cR_i)\rightarrow R_i$
\item $(-cR_i)+(cR_i+R_j)\rightarrow R_j$
\end{enumerate}

\begin{definition}
  To say that two matrices (SOLEs) are row equivalent means that there exists
  a sequence of EROs that transform one matrix into the other.

  Thus, row equivalent matrices (SOLEs) have the same solution set.
\end{definition}

\begin{example}
  \begin{minipage}[t]{2in}
    \vspace{0pt}
    $2x+y=4$ \\
    $-x+2y=3$
  \end{minipage}
  \begin{minipage}[t]{2in}
    \vspace{0pt}
    $\begin{bmatrix}[cc:c]
      2 & 1 & 4 \\
      -1 & 2 & 3
    \end{bmatrix}$
  \end{minipage}
\newpage
  $\frac{1}{2}R_1\rightarrow R_1$

  \begin{minipage}[t]{2in}
    \vspace{0pt}
    $x+\frac{1}{2}y=2$ \\
    $-x+2y=3$
  \end{minipage}
  \begin{minipage}[t]{2in}
    \vspace{0pt}
    $\begin{bmatrix}[cc:c]
      1 & \frac{1}{2} & 2 \\
      -1 & 2 & 3
    \end{bmatrix}$
  \end{minipage}

  \bigskip

  $(1)R_1+R_2\rightarrow R_2$

  \begin{minipage}[t]{2in}
    \vspace{0pt}
    $x+\frac{1}{2}y=2$ \\
    $\frac{5}{2}y=5$
  \end{minipage}
  \begin{minipage}[t]{2in}
    \vspace{0pt}
    $\begin{bmatrix}[cc:c]
      1 & \frac{1}{2} & 2 \\
      0 & \frac{5}{2} & 5
    \end{bmatrix}$
  \end{minipage}

  \bigskip

  $\frac{2}{5}R_2\rightarrow R_2$

  \begin{minipage}[t]{2in}
    \vspace{0pt}
    $x+\frac{1}{2}y=2$ \\
    $y=2$
  \end{minipage}
  \begin{minipage}[t]{2in}
    \vspace{0pt}
    $\begin{bmatrix}[cc:c]
      1 & \frac{1}{2} & 2 \\
      0 & 1 & 2
    \end{bmatrix}$
  \end{minipage}

  \bigskip
  
  $-\frac{1}{2}R_2+R_1\rightarrow R_1$

  \begin{minipage}[t]{2in}
    \vspace{0pt}
    $x=1$ \\
    $y=2$
  \end{minipage}
  \begin{minipage}[t]{2in}
    \vspace{0pt}
    $\begin{bmatrix}[cc:c]
      1 & 0 & 1 \\
      0 & 1 & 2
    \end{bmatrix}$
  \end{minipage}

  Consistent with one (unique) solution: $(1,2)$

  Check:
  \[2(1)+2=2-2=4\checkmark\]
  \[-1+2(2)=-1+4=3\checkmark\]
\end{example}

\begin{example}
  \begin{minipage}[t]{2in}
    \vspace{0pt}
    $x_2-4x_3=8$ \\
    $2x_1-3x_2+2x_3=1$ \\
    $5x_1-8x_2+7x_3=1$
  \end{minipage}
  \begin{minipage}[t]{2in}
    \vspace{0pt}
    $\begin{bmatrix}[ccc:c]
      0 & 1 &-4 & 8 \\
      2 & -3 & 2 & 1 \\
      5 & -8 & 7 & 1
    \end{bmatrix}$
  \end{minipage}

  \bigskip

  $R_1\leftrightarrow R_3$

  \begin{minipage}[t]{2in}
    \vspace{0pt}
    $5x_1-8x_2+7x_3=1$ \\
    $2x_1-3x_2+2x_3=1$ \\
    $x_2-4x_3=8$
  \end{minipage}
  \begin{minipage}[t]{2in}
    \vspace{0pt}
    $\begin{bmatrix}[ccc:c]
      5 & -8 & 7 & 1 \\
      2 & -3 & 2 & 1 \\
      0 & 1 &-4 & 8
    \end{bmatrix}$
  \end{minipage}

  \bigskip

  $-2R_2+R_1\rightarrow R_1$

  \begin{minipage}[t]{2in}
    \vspace{0pt}
    $x_1-2x_2+3x_3=-1$ \\
    $2x_1-3x_2+2x_3=1$ \\
    $x_2-4x_3=8$
  \end{minipage}
  \begin{minipage}[t]{2in}
    \vspace{0pt}
    $\begin{bmatrix}[ccc:c]
      1 & -2 & 3 & -1 \\
      2 & -3 & 2 & 1 \\
      0 & 1 &-4 & 8
    \end{bmatrix}$
  \end{minipage}

  \bigskip

  $-2R_1+R_2\rightarrow R_2$

  \begin{minipage}[t]{2in}
    \vspace{0pt}
    $x_1-2x_2+3x_3=-1$ \\
    $x_2-4x_3=3$ \\
    $x_2-4x_3=8$
  \end{minipage}
  \begin{minipage}[t]{2in}
    \vspace{0pt}
    $\begin{bmatrix}[ccc:c]
      1 & -2 & 3 & -1 \\
      0 & 1 & -4 & 3 \\
      0 & 1 & -4 & 8
    \end{bmatrix}$
  \end{minipage}

  \bigskip

  $-R_2+R_3\rightarrow R_3$

  \begin{minipage}[t]{2in}
    \vspace{0pt}
    $x_1-2x_2+3x_3=-1$ \\
    $x_2-4x_3=3$ \\
    $0=5$
  \end{minipage}
  \begin{minipage}[t]{2in}
    \vspace{0pt}
    $\begin{bmatrix}[ccc:c]
      1 & -2 & 3 & -1 \\
      0 & 1 & -4 & 3 \\
      0 & 0 & 0 & 5
    \end{bmatrix}$
  \end{minipage}

  Inconsistent, because $0\ne5$
\end{example}

\end{document}
