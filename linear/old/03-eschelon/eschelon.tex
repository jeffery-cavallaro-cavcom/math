\documentclass[letterpaper,12pt,fleqn]{article}
\usepackage{matharticle}
\pagestyle{empty}
\begin{document}
\section*{Eschelon Forms}

\begin{definition}
  A \emph{non-zero row (column)} in a matrix is a row (column) that contains at least one
  non-zero entry.

  The \emph{leading entry} of a non-zero row is the leftmost non-zero entry.
\end{definition}

\begin{definition}
  A matrix is said to be in \emph{row eschelon form} (REF) when it has the following
  three properties:
  \begin{enumerate}
  \item All non-zero rows precede all zero rows.
  \item Each leading entry of a non-zero row occurs in a column to the right of the
    leading entries for all preceding rows.
  \item All entries in a column below a leading entry are zero.
  \end{enumerate}

  A matrix is said to be in \emph{reduced row eschelon form} (RREF) when it is in row
  eschelon form and has the following two additional properties:
  \begin{enumerate}
  \item The leading entry in each non-zero row equals $1$.
  \item Each leading entry is the only non-zero entry in its column.
  \end{enumerate}
\end{definition}

\begin{example}
  \begin{minipage}[t]{2in}
    \begin{center}
      $\begin{bmatrix}
        2 & 0 & 4 & -2 \\
        0 & 1 & 4 & 5 \\
        0 & 0 & 0 & -3 \\
      \end{bmatrix}$

      \bigskip

      row eschelon \\
      form
    \end{center}
  \end{minipage}
  \begin{minipage}[t]{2in}
    \begin{center}
      $\begin{bmatrix}
        1 & 0 & 0 & 4 \\
        0 & 1 & 1 & 3 \\
        0 & 0 & 1 & -1 \\
        0 & 0 & 0 & 1
      \end{bmatrix}$

      \bigskip

      row eschelon \\
      form
    \end{center}
  \end{minipage}

  \begin{minipage}[t]{2in}
    \begin{center}
      $\begin{bmatrix}
        -1 & 4 & 8 & 2 \\
        0 & 0 & 0 & 1 \\
        0 & 0 & 1 & 0
      \end{bmatrix}$

      \bigskip

      not in \\
      row eschelon \\
      form
    \end{center}
  \end{minipage}
  \begin{minipage}[t]{2in}
    \begin{center}
      $\begin{bmatrix}
        1 & 0 & 0 & 1 \\
        0 & 1 & 0 & 2 \\
        0 & 0 & 1 & 3
      \end{bmatrix}$

      \bigskip

      reduced \\
      row eschelon \\
      form
    \end{center}
  \end{minipage}
\end{example}

\begin{theorem}
  Each matrix is row equivalent to exactly one reduced row eschelon form.
\end{theorem}

\end{document}
