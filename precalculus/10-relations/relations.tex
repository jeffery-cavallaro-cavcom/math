\documentclass[letterpaper,12pt,fleqn]{article}
\usepackage{matharticle}
\pagestyle{empty}

\begin{document}

\section*{Relations}

Physical phenomena are defined by certain quantities and the how those quantities are \emph{related} to each other.

\begin{example}[A Chemical Reaction]
  Quantities:

  \begin{itemize}
  \item Mass of reactants
  \item Mass of products
  \item Heat energy absorbed (endothermic) or emitted (exothermic)
  \item Time
  \end{itemize}

  Relations:

  \begin{itemize}
  \item How much product has been produced by a given time?
  \item How much time has passed when a certain amount of product has been produced?
  \item How much energy has been released when a certain amount of a reactant has been consumed?
  \end{itemize}
\end{example}

\begin{example}[The Flight of an Aircraft]
  Quantities:

  \begin{itemize}
  \item Distance traveled
  \item Altitude
  \item Airspeed
  \item Time
  \end{itemize}

  Relations:

  \begin{itemize}
  \item What is the aircraft's altitude at a given time?
  \item At what times (multiple answers) is the aircraft at a particular altitude?
  \item What is the aircraft's speeds (multiple answers) at a given altitude?
  \end{itemize}
\end{example}

When time is involved a phenomenon is called \emph{dynamic}.  Otherwise, it is called \emph{static}.  Note that
static problems may initially be dynamic, but the focus is on the \emph{steady state}.

\begin{example}[Ideal Gas in a Container]
  \[PV=nRT\]

  Quantities:

  \begin{itemize}
  \item Number of molecules
  \item Volume of container
  \item Pressure inside container
  \item Temperature of the gas
  \end{itemize}

  Relations:

  \begin{itemize}
  \item What is the pressure for a given temperature?
  \item What temperature is required for a desired pressure?
  \end{itemize}
\end{example}

\end{document}
