\documentclass[letterpaper,12pt,fleqn]{article}
\usepackage{matharticle}
\pagestyle{empty}
\begin{document}
\section*{Rational Numbers}

\begin{definition}
The set of \emph{Rational Numbers} is given by:
\[\mathbb{Q}=\left\{\left.\frac{p}{q}\right|
    p,q\in\mathbb{Z}\ \mbox{and}\ q\ne0\right\}\]
If a number is not rational then it is called \emph{irrational}.
\end{definition}

\begin{properties}
A rational number can be written as:
\begin{enumerate}
\item{A ratio of integers}
\item{A finite decimal}
\item{A repeating decimal}
\end{enumerate}
\end{properties}

\begin{example}
Convert $0.1\overline{23}$ into a ratio of integers. \\
\\
Let $x=0.1\overline{23}$. \\
$1000x=123.\overline{23}$ \\
$10x=1.\overline{23}$ \\
$990x=122$ \\
$x=\frac{122}{990}=\frac{61}{495}$
\end{example}

\begin{theorem}
$\forall r,s\in\mathbb{Q},r+s\in\mathbb{Q}$
\end{theorem}

\begin{theproof}
Assume $r,s\in\mathbb{Q}$. \\
$\exists a,b\in\mathbb{Z},r=\frac{a}{b}, b\ne0$ \\
$\exists c,d\in\mathbb{Z},s=\frac{c}{d}, d\ne0$ \\
$r+s=\frac{a}{b}+\frac{c}{d}=\frac{ad+bc}{bd}$ \\
$ab+bc\in\mathbb{Z}$ \\
$bd\in\mathbb{Z}$ and $bd\ne0$ \\
$\therefore r+s\in\mathbb{Q}$ \\
\end{theproof}

\begin{lemma}
$r\in\mathbb{Q}\iff -r\in\mathbb{Q}$
\end{lemma}

\begin{theproof}
\begin{eqnarray*}
r\in\mathbb{Q} &\iff& \exists p,q\in\mathbb{Z},r=\frac{p}{q}, q\ne0 \\
    &\iff& \frac{-p}{q}\in\mathbb{Q} \\
    &\iff& -r\in\mathbb{Q} \\
\end{eqnarray*}
\end{theproof}

\begin{theorem}
$\forall r\in\mathbb{Q}$ and $s\notin\mathbb{Q},r+s\notin\mathbb{Q}$
\end{theorem}

\begin{theproof}
Assume $r\in\mathbb{Q}$ and $s\notin\mathbb{Q}$. \\
ABC: $r+s\in\mathbb{Q}$ \\
Let $t=r+s$ \\
$-r\in\mathbb{Q}$ \\
$s=t-r$ \\
But $t-r\in\mathbb{Q}$. \\
Thus $s\in\mathbb{Q}$. \\
Contradiction. \\
$\therefore r+s\notin\mathbb{Q}$
\end{theproof}

\begin{theorem}
$\forall r,s\in\mathbb{Q},rs\in\mathbb{Q}$
\end{theorem}

\begin{theproof}
Assume $r,s\in\mathbb{Q}$. \\
$\exists a,b\in\mathbb{Z},r=\frac{a}{b}, b\ne0$ \\
$\exists c,d\in\mathbb{Z},s=\frac{c}{d}, d\ne0$ \\
$rs=\frac{a}{b}\cdot\frac{c}{d}=\frac{ac}{bd}$ \\
$ac\in\mathbb{Z}$ \\
$bd\in\mathbb{Z}$ and $bd\ne0$ \\
$\therefore rs\in\mathbb{Q}$ \\
\end{theproof}

\begin{lemma}
$r\in\mathbb{Q}-\{0\}\iff \frac{1}{r}\in\mathbb{Q}$
\end{lemma}

\begin{theproof}
\begin{eqnarray*}
r\in\mathbb{Q}-\{0\} &\iff& \exists p,q\in\mathbb{Z}-\{0\},r=\frac{p}{q} \\
    &\iff& \frac{q}{p}\in\mathbb{Q} \\
    &\iff& \frac{1}{r}\in\mathbb{Q} \\
\end{eqnarray*}
\end{theproof}

\begin{theorem}
$\forall r\in\mathbb{Q}$ and $s\notin\mathbb{Q},rs\notin\mathbb{Q}$
\end{theorem}

\begin{theproof}
Assume $r\in\mathbb{Q}$ and $s\notin\mathbb{Q}$. \\
ABC: $rs\in\mathbb{Q}$ \\
Let $t=rs$ \\
$\frac{1}{r}\in\mathbb{Q}$ \\
$s=t\cdot\frac{1}{r}$ \\
But $t\cdot\frac{1}{r}\in\mathbb{Q}$. \\
Thus $s\in\mathbb{Q}$. \\
Contradiction. \\
$\therefore rs\notin\mathbb{Q}$
\end{theproof}

\end{document}
