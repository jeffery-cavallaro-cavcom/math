\documentclass[letterpaper,12pt,fleqn]{article}
\usepackage{matharticle}
\pagestyle{empty}
\begin{document}
\section*{Algebraic Numbers}

\begin{definition}
To say that a number is \emph{algebraic} means that there exists a polynomial
equation of the form $\sum_{k=0}^nc_kx^k$ such that $n\ge1$, $c_k\in\mathbb{Z}$,
and $c_n\ne0$ for which the number is a solution. The set of all algebraic
numbers is denoted by $\mathbb{A}$.
\end{definition}

\begin{theorem}
$\mathbb{Q}\subset\mathbb{A}$
\end{theorem}

\begin{theproof}
Assume $r\in\mathbb{Q}$. \\
$\exists p,q\in\mathbb{Z},r=\frac{p}{q}, q\ne0$ \\
Consider the polynomial equation $qx-p=0$. \\
$r$ is a solution. \\
$\therefore r\in\mathbb{A}$
\end{theproof}

\begin{example}
Show that $\sqrt{2}\in\mathbb{A}$. \\
\\
Let $x=\sqrt{2}$. \\
$x^2=2$ \\
$x^2-2=0$ \\
$\sqrt{2}$ is a solution to this polynomial equation. \\
$\therefore\sqrt{2}\in\mathbb{A}$
\end{example}

\begin{example}
Show that $\sqrt[3]{2+\sqrt{5}}\in\mathbb{A}$. \\
\\
Let $x=\sqrt[3]{2+\sqrt{5}}$. \\
$x^3=2+\sqrt{5}$ \\
$x^3-2=\sqrt{5}$ \\
$(x^3-2)^2=5$ \\
$x^6-4x^3+4=5$ \\
$x^6-4x^3-1=0$ \\
$\sqrt[3]{2+\sqrt{5}}$ is a solution to this polynomial equation. \\
$\therefore\sqrt[3]{2+\sqrt{5}}\in\mathbb{A}$
\end{example}
\end{document}
