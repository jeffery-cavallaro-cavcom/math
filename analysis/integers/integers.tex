\documentclass[letterpaper,12pt,fleqn]{article}
\usepackage{matharticle}
\pagestyle{empty}
\begin{document}
\section*{Integers}

\begin{definition}
The set of \emph{Integers}, denoted $\mathbb{Z}$, are the positive and negative
whole numbers and 0:
\[\mathbb{Z}=\{n|n\in\mathbb{N}\}\cup\{-n|n\in\mathbb{N}\}\cup\{0\}=
    \{\ldots, -3, -2, -1, 0, 1, 2, 3, \ldots\}\]
\end{definition}

\theoremstyle{mathitem}
\newtheorem*{closure}{Closure Property}
\begin{closure}
$\forall n,m\in\mathbb{Z}$:
\begin{enumerate}
\item{$n+m\in\mathbb{Z}$}
\item{$nm\in\mathbb{Z}$}
\end{enumerate}
\end{closure}

\begin{definition}
To say that an integer $n$ divides an integer $m$, denoted $n|m$, means
$\exists k\in\mathbb{Z},m=kn$.
\end{definition}

\begin{definition}
To say that an integer $n$ is \emph{even} mean that $2|n$. Otherwise $n$ is
said to be \emph{odd}. Thus, $\forall n\in\mathbb{Z}$:
\begin{enumerate}
\item{n even $\iff\exists k\in\mathbb{Z},n=2k$}
\item{n odd $\iff\exists k\in\mathbb{Z},n=2k+1$}
\end{enumerate}
\end{definition}

\begin{theorem}
$\forall n,m\in\mathbb{Z}$:
\begin{enumerate}
\item{$n$ even and $m$ even $\implies n+m$ even.}
\item{$n$ odd and $m$ odd $\implies n+m$ even}
\item{$n$ even and $m$ odd $\implies n+m$ odd}
\end{enumerate}
\end{theorem}

\begin{theproof}
Assume $n,m\in\mathbb{Z}$.
\begin{enumerate}
\item{Assume $n$ even and $m$ even \\}
$\exists k\in\mathbb{Z},n=2k$ \\
$\exists j\in\mathbb{Z},m=2j$ \\
$n+m=2k+2j=2(k+j)$ \\
But by closure, $k+j\in\mathbb{Z}$. \\
$\therefore n+m$ is even.

\item {Assume $n$ odd and $m$ odd \\}
$\exists k\in\mathbb{Z},n=2k+1$ \\
$\exists j\in\mathbb{Z},m=2j+1$ \\
$n+m=(2k+1)+(2j+1)=2k+2j+2=2(k+j+1)$ \\
But by closure, $k+j+1\in\mathbb{Z}$. \\
$\therefore n+m$ is even.

\item {Assume $n$ even and $m$ odd \\}
$\exists k\in\mathbb{Z},n=2k$ \\
$\exists j\in\mathbb{Z},m=2j+1$ \\
$n+m=2k+2j+1=2(k+j)+1$ \\
But by closure, $k+j\in\mathbb{Z}$. \\
$\therefore n+m$ is odd.
\end{enumerate}
\end{theproof}

\begin{theorem}
$\forall n,m\in\mathbb{Z}$:
\begin{enumerate}
\item{$n$ even and $m$ even $\implies nm$ even.}
\item{$n$ odd and $m$ odd $\implies nm$ odd}
\item{$n$ even and $m$ odd $\implies nm$ even}
\end{enumerate}
\end{theorem}

\begin{theproof}
Assume $n,m\in\mathbb{Z}$.
\begin{enumerate}
\item{Assume $n$ even and $m$ even \\}
$\exists k\in\mathbb{Z},n=2k$ \\
$\exists j\in\mathbb{Z},m=2j$ \\
$nm=(2k)(2j)=2(2kj)$ \\
But by closure, $2kj\in\mathbb{Z}$. \\
$\therefore nm$ is even.

\item {Assume $n$ odd and $m$ odd \\}
$\exists k\in\mathbb{Z},n=2k+1$ \\
$\exists j\in\mathbb{Z},m=2j+1$ \\
$nm=(2k+1)(2j+1)=4kj+2k+2j+1=2(2kj+k+j)+1$ \\
But by closure, $2kj+k+j\in\mathbb{Z}$. \\
$\therefore nm$ is odd.

\item {Assume $n$ even and $m$ odd \\}
$\exists k\in\mathbb{Z},n=2k$ \\
$\exists j\in\mathbb{Z},m=2j+1$ \\
$nm=(2k)(2j+1)=2(2kj+k)$ \\
But by closure, $2kj+k\in\mathbb{Z}$. \\
$\therefore nm$ is even.
\end{enumerate}
\end{theproof}

\begin{theorem}
$\forall n\in\mathbb{Z}$:
\begin{enumerate}
\item{$n$ even $\iff$ $n^2$ even}
\item{$n$ odd $\iff$ $n^2$ odd}
\end{enumerate}
\end{theorem}

\begin{theproof}
\listbreak
\begin{enumerate}
\item{Assume $n\in\mathbb{Z}$.}
\begin{description}
\item{$\implies$ Assume $n$ is even. \\}
$\exists k\in\mathbb{Z},n=2k$ \\
$n^2=(2k)^2=4k^2=2(2kk)$ \\
But by closure, $2kk\in\mathbb{Z}$. \\
$\therefore n^2$ is even.

\item{$\impliedby$ Assume $n^2$ is even. \\}
Contrapositive of (2).
\end{description}

\item{Assume $n\in\mathbb{Z}$.}
\begin{description}
\item{$\implies$ Assume $n$ is odd. \\}
$\exists k\in\mathbb{Z},n=2k+1$ \\
$n^2=(2k+1)^2=4k^2+4k+1=2(2kk+2k)+1$ \\
But by closure, $2kk+2k\in\mathbb{Z}$. \\
$\therefore n^2$ is odd.

\item{$\impliedby$ Assume $n^2$ is odd. \\}
Contrapositive of (1).
\end{description}
\end{enumerate}
\end{theproof}
\end{document}
