\documentclass[letterpaper,12pt,fleqn]{article}
\usepackage{matharticle}
\usetikzlibrary{arrows.meta}
\usetikzlibrary{decorations.markings}
\pagestyle{empty}
\newcommand{\g}{\gamma}
\renewcommand{\l}{\lambda}
\allowdisplaybreaks
\begin{document}
\section*{Rouche's Theorem}

\begin{theorem}[Cauchy]
  Let $f(z)=\sum_{k=0}^na_kz^k$ where $a_k\in\R$. All of the zeros of $f(z)$
  are enclosed in the circle:
  \[\abs{z}=1+\max\{\abs{a_k}\mid0\le k\le n\}\]
\end{theorem}

\begin{example}
  Let $f(z)=z^4-z^2-2z+2=(z-1)^2(z+1\pm i)$

  $\abs{z}=1+\max\{1,2\}=1+2=3$

  $|1|=1<3$ \\
  $\abs{-1\pm i}=\sqrt{2}<3$
\end{example}

But, we can do better with Rouche's Theorem:

\begin{theorem}
  Let $f(z)$ and $g(z)$ be analytic on $\overline{D}$ with boundary $\g$ such
  that $\abs{g(z)}<\abs{f(z)}$ on $\g$. The number of zeros of $(f+g)(z)$ in
  $\g$ equals the number of zeros of $f(z)$ in $\g$.
\end{theorem}

\begin{theproof}
  Let $F(z)=\frac{g(z)}{f(z)}$ \\
  On $\g$: $\abs{F(z)}=\frac{\abs{g(z)}}{\abs{f(z)}}<1$ \\
  $g(z)=f(z)F(z)$

  Let $N_1=$ the number of zeros of (f+g) inside $\g$ \\
  Let $N_2=$ the number of zeros of f inside $\g$
  \begin{eqnarray*}
    N_1-N_2 &=& \frac{1}{2\pi i}\int_{\g}\frac{f'(z)+g'(z)}{f(z)+g(z)}dz-
    \frac{1}{2\pi i}\int_{\g}\frac{f'(z)}{f(z)}dz \\
    &=& \frac{1}{2\pi i}\int_{\g}\left[\frac{f'+g'}{f+g}-\frac{f'}{f}
      \right]dz \\
    &=& \frac{1}{2\pi i}\int_{\g}\left[\frac{f'+(fF)'}{f+fF}-\frac{f'}{f}
      \right]dz \\
    &=& \frac{1}{2\pi i}\int_{\g}\left[\frac{f'+f'F+fF'}{f+fF}-\frac{f'}{f}
      \right]dz \\
    &=& \frac{1}{2\pi i}\int_{\g}\left[\frac{f'(1+F)+fF'}{f(1+F)}-\frac{f'}{f}
      \right]dz \\
    &=& \frac{1}{2\pi i}\int_{\g}\left[\frac{f'}{f}+\frac{F'}{1+F}-\frac{f'}{f}
      \right]dz \\
    &=& \frac{1}{2\pi i}\int_{\g}\left[\frac{F'}{1+F}\right]dz \\
    &=& \frac{1}{2\pi i}\int_{\g}\left[F(z)'\sum_{n=0}^{\infty}(-1)^n[F(z)]^n
      \right]dz \\
    &=& \frac{1}{2\pi i}\sum_{n=0}^{\infty}(-1)^n\int_{\g}[F(z)]^nF'(z)dz \\
    &=& \frac{1}{2\pi i}\sum_{n=0}^{\infty}(-1)^n\frac{[F(z)]^{n+1}}{n+1}
    \mid_{\g} \\
  \end{eqnarray*}
  But $\g$ is closed, so $N_1-N_2=0$ \\
  $\therefore N_1=N_2$
\end{theproof}

\begin{example}
  Let $a>e$ and $h(z)=e^z-az^n$ \\
  Show that $h(z)$ has $n$ zeros inside $\abs{z}=1$

  Let $f(z)=-az^n$ and $g(z)=e^z$ \\
  On $\abs{z}=1$: \\
  $\abs{f(z)}=\abs{-az^n}=a\abs{z}^n=a(1^n)=a$ \\
  $\abs{g(z)}=\abs{e^z}=e^x\abs{e^{iy}}=e^x(1)=e^x\le e^1=e$ \\
  So $\abs{g(z)}<\abs{f(z)}$ on $\abs{z}=1$ \\

  $f(z)=-az^n$ has $n$ repeated zeros at $z=0$, which is inside $\abs{z}=1$ \\
  $\therefore h(z)=(f+g)(z)$ has $n$ zeros inside $\abs{z}=1$
\end{example}

\begin{example}
  Let $h(z)=z^7-5z^3+12$ \\
  Show that $h(z)$ has 7 zeros between $\abs{z}=1$ and $\abs{z}=2$

  First, let $f(z)=z^7$ and $g(z)=12-5z^3$

  On $\abs{z}=2$:
  \[\frac{\abs{g(z)}}{\abs{f(z)}}=
  \frac{\abs{12-5z^3}}{\abs{z^7}}\le
  \frac{12+5\abs{z}^3}{\abs{z}^7}=
  \frac{12+5(8)}{128}=\frac{52}{128}<1\]
  So $\abs{g(x)}<\abs{f(x)}$ on $\abs{z}=2$ \\
  But $f(z)=x^7$ has 7 repeated zeros at $z=0$, which is inside $\abs{z}=2$ \\
  $\therefore h(x)=(f+g)(x)$ has 7 zeros inside $\abs{z}=2$

  Now, let $f(z)=12$ and $g(z)=z^7-5z^3$

  On $\abs{z}=1$:
  \[\frac{\abs{g(z)}}{\abs{f(z)}}=
  \frac{\abs{z^7-5z^3}}{\abs{12}}\le
  \frac{\abs{z}^7+5\abs{z}^3}{12}=
  \frac{1+5}{12}=\frac{1}{2}<1\]
  So $\abs{g(x)}<\abs{f(x)}$ on $\abs{z}=1$ \\
  But $f(z)=12$ has no zeros inside $\abs{z}=1$ \\
  $\therefore h(x)=(f+g)(x)$ has no zeros inside $\abs{z}=1$

  $\therefore h(z)$ has 7 zeros between $\abs{z}=1$ and $\abs{z}=2$
\end{example}

\begin{theorem}[Enestome]
  Let $p(z)=\sum_{k=0}^na_kz^k$ such that $a_k\in\R$ and $0<a_{k-1}<a_k<a_n$. \\
  All of the zeros of $p(z)$ are inside $\abs{z}=1$.
\end{theorem}

\begin{theproof}
  Let $0<\l_k<1$ such that $\l_ka_k>a_{k-1}$ \\
  Let $\l=\max\{\l_k\mid0\le k\le n\}$ \\
  $(\l-z)p(z)=(\l-z)\sum_{k=0}^na_kz^k=\sum_{k=0}^n(\l a_kz^k-a_kz^{n+1})=
  \l a_0+\sum_{k=1}^n(\l a_k-a_{k-1})z^k-a_nz^{n+1}$ \\
  $(\l-z)p(z)+a_kz^{n+1}=\l a_0+\sum_{k=1}^n(\l a_k-a_{k-1})z^k$

  Let $f(z)=-a_nz^{n+1}$ and $g(z)=(\l-z)p(z)+a_nz^{n+1}$

  On $\abs{z}=1$:
  \[\abs{f(z)}=\abs{-a_nz^{n+1}}=a_n\abs{z}^{n+1}=a_n\]
  \begin{eqnarray*}
    \abs{(\l-z)p(z)+a_nz^{n+1}} &=&
    \abs{\l a_0+\sum_{k=1}^n(\l a_k-a_{k-1})z^k}\\
    &\le& \abs{\l a_0}+\abs{\sum_{k=1}^n(\l a_k-a_{k-1})z^k} \\
    &\le& \l a_0+\sum_{k=1}^n\abs{(\l a_k-a_{k-1})z^k} \\
    &=& \l a_0+\sum_{k=1}^n(\l a_k-a_{k-1})\abs{z}^k \\
    &\le& \l a_0+\sum_{k=1}^n(\l a_k-a_{k-1}) \\
    &=& \sum_{k=0}^{n-1}(1-\l)a_k+\l a_n
  \end{eqnarray*}
  But $(\l-1)<0$ and $\l a_n>0$, so:
  \[\abs{(\l-z)p(z)+a_kz^{n+1}}\le\l a_n<a_n\]

  So, $\abs{g{z}}<\abs{f(z)}$ on $\abs{z}=1$ \\
  But $f(z)$ has $(n+1)$ repeated zeros at $z=0$, which is inside
  $\abs{z}=1$ \\
  So, $(f+g)(z)=(\l-z)p(z)$ has $(n+1)$ zeros inside $\abs{z}=1$

  Therefore $p(z)$ has $n$ zeros inside $\abs{z}=1$.
\end{theproof}

\begin{theorem}[Rouche Alternate Form]
  Let $f(z)$ and $g(z)$ be analytic on $\overline{D}$ with boundary $\g$ such
  that $\abs{f(z)-g(z)}<\abs{f(z)}$ on $\g$. The number of zeros of $f(z)$ in
  $\g$ equals the number of zeros of $g(z)$ in $\g$.
\end{theorem}

\begin{theproof}
  Let $h(z)=(g-f)(z)$ \\
  $\abs{h(z)}=\abs{g(z)-f(z)}=\abs{f(z)-g(z)}<\abs{f(z)}$ \\
  $N_f=N_{f+h}=N_{f+(g-f)}=N_g$
\end{theproof}

\begin{example}
  Show that $g(z)=z+3+2e^z$ \\
  Show that $g(z)$ has exactly one zero in the left-hand plane.

  \begin{minipage}{3in}
    \begin{tikzpicture}
      \draw [<->] (-3,0) -- (3,0);
      \draw [<->] (0,-3) -- (0,3);
      \draw [decoration={markings,mark=at position 0.75 with
          {\arrow{Stealth[scale=2]}}}] (0,2) [postaction={decorate}]
          arc (90:270:2);
      \draw [decoration={markings,mark=at position 0.5 with
        {\arrow{Stealth[scale=2]}}}] (0,-2) [postaction={decorate}] -- (0,0);
      \draw [dashed] (0,0) -- node [above right] {$r$} ({-sqrt(2)},{sqrt(2)})
          [fill=black] circle [radius=0.1] node [above left] {$z$};
      \draw [fill=black] (0,1) circle [radius=0.1] node [right] {$iy$};
      \node [below left] at ({-sqrt(2)},{-sqrt(2)}) {$C_r$};
      \node [right] at (0,-1) {$C_y$};
    \end{tikzpicture}
  \end{minipage}
  \begin{minipage}{3in}
    Let $f(z)=z+3$
    
    \bigskip
    
    On $C_y$:
    \[\abs{f(z)}=\abs{z+3}=\abs{3+iy}\ge3\]

    Assume $r>5$
    
    On $C_r$:
    \[\abs{f(z)}=\abs{z+3}\ge\abs{z}-\abs{3}>5-3=2\]

    Therefore, on $C_y\cup C_r$, $\abs{f(z)}> 2$
  \end{minipage}

  $\abs{f-g}=\abs{-2e^z}=\abs{-2}\abs{e^z}=2e^x$

  On $C_y$, $x=0$, so $\abs{f-g}=2$

  On $C_r$, $x\le0$, so $\abs{f-g}<2$

  Therefore, on $C_y\cup C_r$, $\abs{f-g}\le2$

  So, on $C_y\cup C_r$, $\abs{f(z)-g(z)}<\abs{f(z)}$ \\
  But $f(z)$ has only one zero at $x=-3$ inside $C_y\cup C_r$ \\
  Therefore, $g(z)$ has only one zero inside $C_y\cup C_r$ \\
  Now let $r\to\infty$ \\
  $g(z)$ has only one zero in the left-hand plane.
\end{example}

\end{document}
