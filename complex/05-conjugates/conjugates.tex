\documentclass[letterpaper,12pt,fleqn]{article}
\usepackage{matharticle}
\usepackage{tikz}
\pagestyle{empty}
\allowdisplaybreaks
\newcommand{\conj}[1]{\bar{#1}}
\newcommand{\Conj}[1]{\overline{#1}}
\begin{document}
\section*{Conjugates}
\begin{definition}
Let $z=x+iy\in\C$. The conjugate of $z$, denoted $\conj{z}$, is given by:
\[\conj{z}=x-iy\]
\end{definition}

\begin{figure}[h]
\setlength{\leftskip}{0.5in}
\begin{tikzpicture}
\draw [<->] (-1,0) -- (3,0);
\draw [<->] (0,-2) -- (0,2);
\draw [dashed] (0,0) -- (2,1);
\draw [dashed] (0,0) -- (2,-1);
\draw [dashed] (2,1) -- (2,-1);
\draw [fill=black] (2,1) circle [radius=0.1];
\draw [fill=black] (2,-1) circle [radius=0.1];
\node [above right] at (2,1) {$z=(x,y)$};
\node [below right] at (2,-1) {$\conj{z}=(x,-y)$};
\end{tikzpicture}
\end{figure}

\begin{properties}
\listbreak
\begin{enumerate}
\item{$\conj{\conj{z}}=z$}
\item{$\Conj{z_1\pm z_2}=\conj{z}_1\pm\conj{z}_2$}
\item{$\Conj{z_1z_2}=\conj{z}_1\conj{z}_2$}
\item{$\Conj{\left(\frac{1}{z}\right)}=\frac{1}{\conj{z}}$}
\item{$\Conj{\left(\frac{z_1}{z_2}\right)}=\frac{\conj{z}_1}{\conj{z}_2}$}
\end{enumerate}
\end{properties}

\begin{theproof}
\listbreak
\begin{enumerate}
\item
\begin{eqnarray*}
\conj{\conj{z}} &=& \Conj{\Conj{x+iy}} \\
    &=& \Conj{x-iy} \\
    &=& x+iy \\
    &=& z \\
\end{eqnarray*}

\item
\begin{eqnarray*}
\Conj{z_1\pm z_2} &=& \Conj{(x_1+iy_1)\pm(x_2+iy_2)} \\
    &=& \Conj{(x_1\pm x_2)+i(y_1\pm y_2)} \\
    &=& (x_1\pm x_2)-i(y_1\pm y_2) \\
    &=& (x_1-iy_1)\pm(x_2-iy_2) \\
    &=& \conj{z}_1\pm\conj{z}_2 \\
\end{eqnarray*}

\item
\begin{eqnarray*}
\Conj{z_1z_2} &=& \Conj{(x_1+iy_2)(x_2+iy_2)} \\
    &=& \Conj{(x_1x_2-y_1y_2)+i(x_1y_2+x_2y_1)} \\
    &=& (x_1x_2-y_1y_2)-i(x_1y_2+x_2y_1) \\
    &=& x_1(x_2-iy_2)-y_1(y_2+ix_2) \\
    &=& x_1(x_2-iy_2)-iy_1(x_2-iy_2) \\
    &=& (x_1-iy_1)(x_2-iy_2) \\
    &=& \conj{z}_1\conj{z}_2 \\
\end{eqnarray*}

\item
\begin{eqnarray*}
\Conj{\left(\frac{1}{z}\right)} &=& \Conj{\left(\frac{x-iy}{x^2+y^2}\right)} \\
    &=& \frac{x+iy}{x^2+y^2} \\
    &=& \frac{x-i(-y)}{x^2+(-y)^2} \\
    &=& \frac{1}{x-iy} \\
    &=& \frac{1}{\conj{z}} \\
\end{eqnarray*}

\item
\begin{eqnarray*}
\Conj{\left(\frac{z_1}{z_2}\right)} &=& \Conj{z_1\cdot\frac{1}{z_2}} \\
    &=& \conj{z}_1\Conj{\left(\frac{1}{z_2}\right)} \\
    &=& \conj{z}_1\left(\frac{1}{\conj{z}_2}\right) \\
    &=& \frac{\conj{z}_1}{\conj{z}_2} \\
\end{eqnarray*}
\end{enumerate}
\end{theproof}

\newpage

\begin{theorem}
Let $z=x+iy\in\C$:
\begin{itemize}
\item{$Re(z)=\frac{z+\conj{z}}{2}$}
\item{$Im(z)=\frac{z-\conj{z}}{2}$}
\end{itemize}
\end{theorem}

\begin{theproof}
$z+\conj{z}=(x+iy)+(x-iy)=2x=2Re(z)$ \\
$\therefore Re(z)=\frac{z+\conj{z}}{2}$

$z-\conj{z}=(x+iy)-(x-iy)=2iy=2iIm(z)$ \\
$\therefore Im(z)=\frac{z-\conj{z}}{2i}$
\end{theproof}

\begin{example}
Let $Z=\frac{1}{z+i}$.
\begin{eqnarray*}
Im(Z) &=& \frac{z-\conj{z}}{2i} \\
    &=& \frac{1}{2i}\left[\frac{1}{z+i}-
        \Conj{\left(\frac{1}{z+i}\right)}\right] \\
    &=& \frac{1}{2i}\left(\frac{1}{z+i}-\frac{1}{\Conj{z+i}}\right) \\
    &=& \frac{1}{2i}\left(\frac{1}{z+i}-\frac{1}{\conj{z}-i}\right) \\
    &=& \frac{1}{2i}\left[\frac{(\conj{z}-i)-(z+i)}
        {(z+i)\Conj{(z+i)})}\right] \\
    &=& \frac{1}{2i}\left(\frac{\conj{z}-z-2i}{\abs{z+i}^2}\right) \\
    &=& -\frac{\frac{z-\conj{z}}{2i}+1}{\abs{z+i}^2} \\
    &=& -\frac{y+1}{\abs{(x+iy)+i}} \\
    &=& -\frac{y+1}{\abs{x+i(y+1)}} \\
    &=& -\frac{y+1}{x^2+(y+1)^2} \\
\end{eqnarray*}
\end{example}
\end{document}
