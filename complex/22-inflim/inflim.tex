\documentclass[letterpaper,12pt,fleqn]{article}
\usepackage{matharticle}
\pagestyle{empty}
\newcommand{\ecp}{\C_{\infty}}
\renewcommand{\d}{\delta}
\newcommand{\e}{\epsilon}
\newcommand{\limz}{\lim_{z\to z_0}}
\newcommand{\limi}{\lim_{z\to\infty}}
\newcommand{\limo}{\lim_{z\to0}}
\begin{document}
\section*{Limits at Infinity}

\begin{definition}
  The \emph{extended complex plane}, denoted $\ecp$, is given by:
  \[\ecp=\C\cup\{\infty\}\]
\end{definition}

Consider the unit sphere with the complex plane passing through the equator,
thus intersecting along the unit circle. For each point $z\in\C$, a line through
$z$ and the north pole $N$ defines a single point of intersection on the
surface of the upper hemisphere of the sphere. Note that all points in the
interior of the unit circle ($0\le\abs{z}<1$) correspond to $N$. All boundary
points ($\abs{z}=1$) correspond to themselves. All points in the exterior
($\abs{z}>1$) and close to the boundary correspond to points near the equator.
As $\abs{z}$ increases, $P$ moves arbitrarily close to $N$. Thus, there is a
correspondence between $N$ and $\infty$.

\bigskip

\begin{tikzpicture}
  \draw (0,0) circle [radius=2];
  \draw [dashed] (0,0) ellipse (2 and 1);
  \draw [fill=black] (0,0) circle [radius=0.05];
  \draw [fill=black] (0,2) circle [radius=0.05];
  \node [below left] at (0,0) {$0$};
  \node [above] at (0,2) {$N$};
  \draw [dashed] (-5,-2.5) -- (3,-2.5) -- (6,3) -- (-2,3) -- cycle;
  \draw [fill=black] (2,-2) circle [radius=0.05];
  \node [right] at (2,-2) {$z$};
  \draw [dashed] (2,-2) -- (0,2);
  \draw [fill=black] (1,0) circle [radius=0.05];
  \node [right] at (1,0) {$P$};
  \node at (4,1.5) {$\C$};
\end{tikzpicture}

\bigskip

Such a sphere is referred to as a \emph{Riemann sphere}.

\begin{definition}
  For all small $\e>0$, $\abs{z}>\frac{1}{\e}$ is referred to as a
  neighborhood of infinity.
\end{definition}

In other words, $\abs{z}$ small is closer to $0$ and $\abs{z}$ large is
closed to $\infty$.

\begin{definition}
  To say that:
  \[\limz{f(z)}=\infty\]
  means:
  \[\forall\,\e>0,\exists\,\d>0,0<\abs{z-z_0}<\d\implies
  \abs{f(z)}>\frac{1}{\e}\]
\end{definition}

\begin{theorem}
  \listbreak
  \[\limz{f(z)}=0\iff\limz{\frac{1}{f(z)}}=\infty\]
\end{theorem}

\begin{theproof}
  Assume $\e>0$
  \begin{eqnarray*}
    \limz{f(z)}=0 &\iff& \exists\,\d>0,0<\abs{z-z_0}<\d\implies\abs{f(z)}<\e \\
    &\iff& \exists\,\d>0,0<\abs{z-z_0}<\d\implies
    \abs{\frac{1}{f(z)}}>\frac{1}{\e} \\
    &\iff& \limz{\frac{1}{f(z)}}=\infty \\
  \end{eqnarray*}
\end{theproof}

\begin{corollary}
  \listbreak
  \[\limz{f(z)}=\infty\iff\limz{\frac{1}{f(z)}}=0\]
\end{corollary}

\begin{theproof}
  \[\limz{\frac{1}{f(z)}}=0\iff\limz{\frac{1}{\frac{1}{f(z)}}}=\infty\iff
  \limz{f(z)}=\infty\]
\end{theproof}

\begin{definition}
  To say that:
  \[\limi{f(z)}=w_0\]
  means:
  \[\forall\,\e>0,\exists\,\d>0,\abs{z}>\d\implies\abs{f(z)-w_0}<\e\]
\end{definition}

\begin{theorem}
  \listbreak
  \[\limi{f(z)}=w_0\iff\limo{f\left(\frac{1}{z}\right)}=w_0\]
\end{theorem}
\newpage
\begin{theproof}
  Assume $\e>0$
  \begin{eqnarray*}
    \limi{f(z)}=w_0 &\iff&
    \exists\,\d>0,\abs{z}>\d\implies\abs{f(z)-w_0}<\e \\
    &\iff& \exists\,\d>0,0<\abs{\frac{1}{z}}<\d\implies\abs{f(z)-w_0}<\e \\
    &\iff& \exists\,\d>0,0<\abs{z}<\d\implies
    \abs{f\left(\frac{1}{z}\right)-w_0}<\e \\
    &\iff& \limo{f\left(\frac{1}{z}\right)}=w_0 \\
  \end{eqnarray*}
\end{theproof}

\begin{definition}
  To say that:
  \[\limi{f(z)}=\infty\]
  means:
  \[\forall\,\e>0,\exists\,\d>0,\abs{z}>\d\implies\abs{f(z)}>\frac{1}{\e}\]
\end{definition}

\begin{theorem}
  \listbreak
  \[\limi{f(z)}=\infty\iff\limo{\frac{1}{f\left(\frac{1}{z}\right)}}=0\]
\end{theorem}

\begin{theproof}
  Assume $\e>0$
  \begin{eqnarray*}
    \limi{f(z)}=\infty &\iff&
    \exists\,\d>0,\abs{z}>\d\implies\abs{f(z)}>\frac{1}{\e} \\
    &\iff& \exists\,\d>0,0<\abs{\frac{1}{z}}<\d\implies
    \abs{f(z)}>\frac{1}{\e} \\
    &\iff& \exists\,\d>0,0<\abs{z}<\d\implies
    \abs{f\left(\frac{1}{z}\right)}>\frac{1}{\e} \\
    &\iff& \exists\,\d>0,0<\abs{z}<\d\implies
    \abs{\frac{1}{f\left(\frac{1}{z}\right)}}<\e \\
    &\iff& \limo{\frac{1}{f\left(\frac{1}{z}\right)}}=0 \\
  \end{eqnarray*}
\end{theproof}

\end{document}
