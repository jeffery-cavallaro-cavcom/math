\documentclass[letterpaper,12pt,fleqn]{article}
\usepackage{matharticle}
\pagestyle{empty}
\allowdisplaybreaks
\begin{document}
\section*{Other Operations}
\begin{definition}
Subtraction in $\C$ is defined as follows:
\begin{eqnarray*}
z_1-z_2 &=& z_1+(-z_2) \\
    &=& (x_1,y_1)+(-x_2,-y_2) \\
    &=& (x_1-x_2,y_1-y_2) \\
    &=& (x_1-x_2)+i(y_1-y_2) \\
\end{eqnarray*}
\end{definition}

\begin{theorem}
\[z^{-1}=\frac{1}{z}\]
\end{theorem}

\begin{theproof}
\begin{eqnarray*}
\frac{1}{z} &=& \frac{1}{x+iy} \\
    &=& \left(\frac{1}{x+iy}\right)\left(\frac{x-iy}{x-iy}\right) \\
    &=& \frac{x-iy}{x^2+y^2} \\
\end{eqnarray*}
\end{theproof}

\begin{theorem}
\[(z_1z_2)^{-1}=z_1^{-1}z_2^{-1}\]
\end{theorem}

\begin{theproof}
$(z_1z_2)(z_1^{-1}z_2^{-1})=(z_1z_1^{-1})(z_2z_2^{-1})=1\cdot1=1$ \\
So, $z_1z_2$ and $z_1^{-1}z_2^{-2}$ are multiplicative inverses. \\
But multiplicative inverses are unique. \\
$\therefore (z_1z_2)^{-1}=z_1^{-1}z_2^{-1}$ \\
\end{theproof}

\begin{definition}
Division in $\C$ is defined as follows:
\begin{eqnarray*}
\frac{z_1}{z_2} &=& z_1z_2^{-1} \\
    &=& z_1\left(\frac{1}{z_2}\right) \\
    &=& (x_1,y_1)\left(\frac{x_2}{x_2^2+y_2^2},\frac{-y_2}{x_2^2+y_2^2}\right) \\
    &=& \left(\frac{x_1x_2+y_1y_2}{x^2+y^2},
        \frac{y_1x_2-x_1y_2}{x_2^2+y_2^2}\right) \\
\end{eqnarray*}
\end{definition}

We can also get the same result as follows:
\begin{eqnarray*}
\frac{z_1}{z_2} &=& \frac{x_1+iy_1}{x_2+iy_2} \\
    &=& \frac{(x_1+iy_1)(x_2-iy_2)}{(x_2+iy_2)(x_2-iy_2)} \\
    &=& \frac{(x_1x_2+y_1y_2)+i(y_1x_2-y_2x_1)}{x_2^2+y_2^2} \\
    &=& \frac{x_1x_2+y_1y_2}{x_2^2+y_2^2}+i\frac{y_1x_2-y_2x_1}{x_2^2+y_2^2} \\
\end{eqnarray*}

\begin{theorem}
\[\left(\frac{z_1}{z_3}\right)\left(\frac{z_2}{z_4}\right)=
    \frac{z_1z_2}{z_3z_4}\]
\end{theorem}

\begin{theproof}
\begin{eqnarray*}
\left(\frac{z_1}{z_3}\right)\left(\frac{z_2}{z_4}\right) &=&
    (z_1z_3^{-1})(z_2z_4^{-1}) \\
    &=& (z_1z_2)(z_3^{-1}z_4^{-1}) \\
    &=& (z_1z_2)(z_3z_4)^{-1} \\
    &=& \frac{z_1z_2}{z_3z_4} \\
\end{eqnarray*}
\end{theproof}

\begin{example}
\[\left(\frac{2i}{1+i}\right)^4=
    \left[\left(\frac{2i}{1+i}\right)\left(\frac{1-i}{1-i}\right)\right]^4=
    \left[\frac{2i(1-i)}{2}\right]^4=(1+i)^4=[(1+i)^2]^2=(2i)^2=-4\]
\end{example}

\newpage

\begin{example}
\[\frac{5}{(1-i)(2-i)(3-i)}=\frac{5(1+i)(2+i)(3+i)}{2\cdot5\cdot10}=
    \frac{(1+i)(5+5i)}{20}=\frac{(1+i)^2}{4}=\frac{2i}{4}=\frac{1}{2}i\]
\end{example}

\begin{theorem}
\listbreak
\[\sqrt{z}=\left(\frac{\sqrt{x^2+y^2}+x}{2}\right)^{\frac{1}{2}}+
    i\left(\frac{\sqrt{x^2+y^2}-x}{2}\right)^{\frac{1}{2}}\]
\end{theorem}

\begin{theproof}
$z=x+iy$ \\
$z^{\frac{1}{2}}=(x+iy)^{\frac{1}{2}}=u+iv$ \\
$x+iy=(u+iv)^2=u^2-v^2+i2uv$ \\
$x=u^2-v^2$ \\
$y=2uv$ \\
$x^2=u^4+v^4-2u^2v^2$ \\
$y^2=4u^2v^2$ \\
$x^2+y^2=u^4+v^4+2u^2v^2=(u^2+v^2)^2$ \\
$u^2+v^2=\sqrt{x^2+y^2}$ \\
$u^2-v^2=x$ \\
$2u^2=\sqrt{x^2+y^2}+x$ \\
$u=\left(\frac{\sqrt{x^2+y^2}+x}{2}\right)^{\frac{1}{2}}$ \\
$2v^2=\sqrt{x^2+y^2}-x$ \\
$v=\left(\frac{\sqrt{x^2+y^2}-x}{2}\right)^{\frac{1}{2}}$ \\
$\therefore \sqrt{z}=\left(\frac{\sqrt{x^2+y^2}+x}{2}\right)^{\frac{1}{2}}+
    i\left(\frac{\sqrt{x^2+y^2}-x}{2}\right)^{\frac{1}{2}}$
\end{theproof}

The binomial theorem also holds:
\begin{theorem}
$\forall z_1,z_2\in\C$ and $n\in\N$:
\[(z_1+z_2)^n=\sum_{k=0}^{n}\binom{n}{k}z_1^kz_2^{n-k}\]
\end{theorem}
The proof is the same as it is for $\R$.
\end{document}
