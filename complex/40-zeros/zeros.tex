\documentclass[letterpaper,12pt,fleqn]{article}
\usepackage{matharticle}
\pagestyle{empty}
\newcommand{\e}{\epsilon}
\begin{document}
\section*{Zeros of Analytic Functions}

\begin{theorem}
  Let $f(z)$ be analytic in a domain $D$:
  \[\exists\,a\in D,\forall\,n\in\Z\cup\{0\},f^{(n)}(a)=0\implies
  \forall\,z\in D,f(z)=0\]
\end{theorem}

\begin{theorem}
  Let $f(z)$ be analytic such that $\abs{f(z)}\le M$ in and on a circle
  $\overline{C}$ with center $a$ and radius $r$:
  \[\forall\,n\in\Z\cup\{0\},\abs{f^{(n)}(a)}\le\frac{Mn!}{r^n}\]
\end{theorem}

\begin{theproof}
  By the CITFD:
  \begin{eqnarray*}
    \abs{f^{(n)}(a)} &=& \abs{\frac{n!}{2\pi i}\int_C\frac{f(z)}
      {(z-a)^{n+1}}dz} \\
    &=& \abs{\frac{n!}{2\pi i}}\abs{\int_C\frac{f(z)}{(z-a)^{n+1}}dz} \\
    &\le& \frac{n!}{2\pi}\int_c\frac{\abs{f(z)}}{\abs{z-a}^{n+1}}\abs{dz} \\
    &\le& \frac{n!}{2\pi}\int_c\frac{M}{r^{n+1}}\abs{dz} \\
    &\le& \frac{Mn!}{2\pi r^{n+1}}\int_c\abs{dz} \\
    &=& \frac{Mn!}{2\pi r^{n+1}}(2\pi r) \\
    &=& \frac{Mn!}{r^n}
  \end{eqnarray*}
\end{theproof}

\begin{theorem}[Liouville]
  $f(z)$ entire and bounded$\implies f(z)$ constant.
\end{theorem}

\begin{theproof}
  Assume $f(z)\le M$ \\
  By the previous theorem with $n=1$ and $a$ in some circle $C$ with radius
  $r$:
  \[\abs{f'(z)}\le\frac{M}{r}\]
  As $r\to\infty$, $\abs{f'(z)}\to0$ \\
  Thus, $\forall\,z\in\C,f'(z)=0$
  
  $\therefore f(z)$ is constant.
\end{theproof}

\begin{definition}
  To say that a point $x$ is an \emph{accumulation point} of a set $X$ means
  that $\forall\,\e>0,N_{\e}(x)$ contains infinitely many points in $X$.
\end{definition}

\begin{theorem}[Uniqueness]
  Let $f(z)$ and $g(z)$ be analytic in a domain $D$. If there exists a set
  $E$ in which $f(z)=g(z)$ and which contains an accumulation point $z_0\in D$
  for $D$ then $f(z)=g(z)$ in $D$.
\end{theorem}

\begin{theproof}
  Assume that such a set $E$ exists \\
  There exists a sequence $\{z_n\}\subset E$ such that $\lim{z_n}=z_0$ \\
  So $f(z_n)=g(z_n)$ in $E$ \\
  Let $F(z)=(f-g)(z)$, which is also analytic in $D$ \\
  $F(z_n)=(f-g)(z_n)=0$ in $E$, and in particular: \\
  $F(z_0)=0$ in both $E$ and $D$

  Rewrite $F(z)$ as a Taylor series about $z_0$:
  \[F(z)=\sum_{n=0}^{\infty}\frac{F^{(k)}(z_0)}{k!}(z-z_0)^k=
  F(z_0)+\sum_{n=1}^{\infty}\frac{F^{(k)}(z_0)}{k!}(z-z_0)^k\]
  We already know that $F(z_0)=0$, so:
  \[F(z)=\sum_{n=1}^{\infty}\frac{F^{(k)}(z_0)}{k!}(z-z_0)^k\]
  Let $z=z_n$:
  \[F(z_n)=\sum_{n=1}^{\infty}\frac{F^{(k)}(z_0)}{k!}(z_n-z_0)^k=0\]
  \[F'(z_0)(z_n-z_0)+\sum_{n=2}^{\infty}\frac{F^{(k)}(z_0)}{k!}(z_n-z_0)^k=0\]
  But $(z_n-z_0)\ne0$, so:
  \[F'(z_0)+\sum_{n=2}^{\infty}\frac{F^{(k)}(z_0)}{k!}(z_n-z_0)^{k-1}=0\]
  As $n\to\infty$, $z_n\to z_0$, so:
  \[F'(z_0)=0\]
  By repeating the process, we find that $F^{(n)}(z_0)=0$ \\
  So, by previous theorem, $F(z)=0$ in $D$
  
  $\therefore f(z)=g(z)$ in $D$.
\end{theproof}

\begin{corollary}
  If $D$ is compact then $f(z)$ has a finite number of zeros in $D$.
\end{corollary}

Note that zeros of an analytic function must be isolated; otherwise, the
function is the zero function over the domain.

\end{document}
