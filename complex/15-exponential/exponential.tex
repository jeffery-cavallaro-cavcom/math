\documentclass[letterpaper,12pt,fleqn]{article}
\usepackage{matharticle}
\usepackage{tikz}
\usetikzlibrary{decorations.markings}
\usetikzlibrary{arrows.meta}
\usepackage{array}
\pagestyle{empty}
\renewcommand{\o}{\theta}
\renewcommand{\O}{\Theta}
\newcommand{\Arg}[1]{\mathrm{Arg}\,#1}
\newcommand{\conj}[1]{\overline{#1}}
\begin{document}
\section*{Exponential Form}
\begin{figure}[h]
\setlength{\leftskip}{0.5in}
\begin{tikzpicture}
\draw [<->] (-3,0) -- (3,0);
\draw [<->] (0,-3) -- (0,3);
\draw [fill=black] (1.5,2) circle [radius=0.05];
\draw [->] (0,0) -- (1.5,2);
\draw [dashed] (1.5,0) -- (1.5,2);
\draw [dashed] (0,2) -- (1.5,2);
\node [right] at (1.5,2) {$z=x+iy$};
\node [below] at (0.75,0) {$x$};
\node [right] at (1.5,1) {$y$};
\node at (0.4,0.25) {$\o$};
\node at (0.6,1.1) {$r$};
\node [below right] at (5,2) {$\begin{array}{l}
    r=\sqrt{x^2+y^2}=\abs{z}\ne0 \\
\\
    x=r\cos{\o}=\abs{z}\cos{\o} \\
\\
    y=r\sin{\o}=\abs{z}\sin{\o} \\
\\
    \tan\o = \frac{y}{x} \\
\end{array}$};
\end{tikzpicture}
\end{figure}
Note that the calculation for $\o$ depends on quadrant.

\begin{definition}
Let $z\in\C$. The polar and exponential forms for $z$ are given by:
\[z=x+iy=r\cos\o+ir\sin\o=r(\cos\o+isin\o)=\abs{z}(\cos\o+isin\o)=
    \abs{z}e^{i\o}\]
\end{definition}

\begin{definition}
Let $z=\abs{z}e^{i\o}$. $\o$ is called the \emph{argument} of $z$. The set of
all coterminal angles of $\o$, denoted $\arg{z}$, is given by:
\[\arg{z}=\{\o+2\pi n\mid n\in\Z\}\]
For convenience, this can be shortened to:
\[\arg{z}=\o+2\pi n\]
with the understanding that $\arg{z}$ is actually a set and $n\in\Z$.

The \emph{principle value} of $\arg{z}$, denoted $\Arg{z}$, is the value
$\O\in\arg{z}$ such that:
\[\O\in(-\pi,\pi]\]
Thus:
\[\arg{z}=\Arg{z}+2\pi n\]
\end{definition}

\begin{example}
Let $z=-1-i\sqrt{3}$.
\[\abs{z}=\sqrt{1^2+(\sqrt{3})^2}=\sqrt{1+3}=\sqrt{4}=2\]
\[\o=\arctan{\left(\frac{-\sqrt{3}}{-1}\right)}=\arctan{\sqrt{3}}=
    \frac{4\pi}{3}\]
\[\O=\frac{4\pi}{3}-2\pi=-\frac{2\pi}{3}\]
\[\arg{z}=\left\{-\frac{2\pi}{3}+2\pi n\mid n\in\Z\right\}\]
\[z=e^{-\frac{2\pi}{3}+2\pi n}=e^{\frac{2\pi}{3}(3n-1)}\]
\end{example}

\begin{theorem}
\[\arg{\conj{z}}=-\arg{z}=\arg{\frac{1}{z}}\]
\end{theorem}

This first part is obvious from the following diagram:

\begin{figure}[h]
\setlength{\leftskip}{0.5in}
\begin{tikzpicture}
\draw [<->] (-3,0) -- (3,0);
\draw [<->] (0,-3) -- (0,3);
\draw [fill=black] (1.5,2) circle [radius=0.05];
\draw [fill=black] (1.5,-2) circle [radius=0.05];
\draw [dashed] (0,0) -- (1.5,2);
\draw [dashed] (0,0) -- (1.5,-2);
\draw [dashed] (1.5,2) -- (1.5,-2);
\node [right] at (1.5,2) {$z$};
\node [right] at (1.5,-2) {$\conj{z}$};
\node at (0.5,0.3) {$\o$};
\node at (0.6,-0.3) {$-\o$};
\end{tikzpicture}
\end{figure}

\begin{theproof}
Let $z=\abs{z}e^{i\o}$ \\
$\conj{z}=\abs{\conj{z}}e^{-i\o}=\abs{z}e^{-i\o}$ \\
$\Arg{z}=-\Arg{\conj{z}}$ \\
$\Arg{\conj{z}}+2\pi n=-\Arg{z}+2\pi n$ \\
$\therefore \arg{\conj{z}}=-\arg{z}$ \\

$\frac{1}{z}=\frac{1}{\abs{z}e^{i\o}}=\frac{1}{\abs{z}}e^{-i\o}$ \\
$\Arg{\frac{1}{z}}=-\Arg{z}=\Arg{\conj{z}}$ \\
$\Arg{\frac{1}{z}}+2\pi n=\Arg{\conj{z}}+2\pi n$ \\
$\therefore \arg{\frac{1}{z}}=\arg{\conj{z}}$ \\
\end{theproof}

\begin{theorem}
Let $z_1=\abs{z_1}e^{i\o_1}$ and $z_2=\abs{z_2}e^{i\o_2}$:
\begin{enumerate}
\item $\arg{(z_1z_2)}=\o_1+\o_2+2\pi n$
\item $\arg{\left(\frac{z_1}{z_2}\right)}=\o_1-\o_2+2\pi n$
\end{enumerate}
\end{theorem}

\begin{theproof}
\listbreak
\begin{enumerate}
\item $\arg{(z_1z_2)}=\arg{\left(\abs{z_1}e^{i\o_1}\abs{z_2}e^{i\o_2}\right)}=
    \arg{\left(\abs{z_1}\abs{z_2}e^{i(\o_1+\o_2)}\right)}=
    \o_1+\o_2+2\pi n$

\item $\arg{\left(\frac{z_1}{z_2}\right)}=
    \arg{\left(\frac{\abs{z_1}e^{i\o_1}}{\abs{z_2}e^{i\o_2}}\right)}=
    \arg{\left(\frac{\abs{z_1}}{\abs{z_2}}e^{i(\o_1-\o_2)}\right)}=
    \o_1-\o_2+2\pi n$
\end{enumerate}
\end{theproof}

\begin{example}
Let $z_1=i$ and $z_2=-1+i$

$z_1=e^{i\frac{\pi}{2}}$ and $z_2=\sqrt{2}e^{i\frac{3\pi}{4}}$

$\o_1+\o_2=\frac{\pi}{2}+\frac{3\pi}{4}=\frac{5\pi}{4}$

$\Arg{(z_1z_2)}=\frac{5\pi}{4}-2\pi=-\frac{3\pi}{4}$

$\arg{(z_1z_2)}=-\frac{3\pi}{4}+2\pi n$
\end{example}

\begin{theorem}
  Let $z=e^{i\o}$:
  \[z^k+\frac{1}{z^k}=2\cos{k\o}\]
\end{theorem}

\begin{theproof}
  \listbreak
  \[z^k+\frac{1}{z^k}=e^{ik\o}+e^{-ik\o}=2\cos{k\o}\]
\end{theproof}

Note that $z=z_0+Re^{i\o}$ is the circle with center $z_0$ and radius $R$:

\begin{tikzpicture}
\draw [<->] (-3,0) -- (3,0);
\draw [<->] (0,-3) -- (0,3);
\draw [fill=black] (1,0.5) circle [radius=0.05];
\draw [fill=black] (1.75,1.8) circle [radius=0.05];
\draw (1,0.5) circle [radius=1.5];
\draw (1,0.5) circle [radius=1.5];
\draw [->] (0,0) -- (1,0.5);
\draw [->] (1,0.5) -- (1.75,1.8);
\draw [->] (0,0) -- (1.75,1.8);
\draw [dashed] (1,0.5) -- (3,0.5);
\node [below] at (1,0.5) {$z_0$};
\node [above right] at (1.75,1.8) {$z$};
\node at (1.3,0.7) {$\o$};
\node at (1.6,1.0) {$R$};
\node at (5,1) {$z=z_o+Re^{i\o}$};
\end{tikzpicture}
\end{document}
