\documentclass[letterpaper,12pt,fleqn]{article}
\usepackage{matharticle}
\usepackage{tikz}
\pagestyle{empty}
\newcommand{\conj}[1]{\bar{#1}}
\newcommand{\Conj}[1]{\overline{#1}}
\allowdisplaybreaks
\begin{document}
\section*{Modulus}
\begin{definition}
Let $z=x+iy\in\C$. The \emph{modulus} of $z$ is given by:
\[\abs{z}=\sqrt{x^2+y^2}\]
\end{definition}

\begin{figure*}[h]
\setlength{\leftskip}{0.5in}
\begin{tikzpicture}
\draw [<->] (-1,0) -- (4,0);
\draw [<->] (0,-1) -- (0,4);
\node [right] at (4,0) {Re};
\node [above] at (0,4) {Im};
\draw [fill=black] (3,2) circle [radius=0.1];
\draw [dashed] (0,2) -- (3,2);
\draw [dashed] (3,0) -- (3,2);
\node [below] at (3,0) {$x$};
\node [left] at (0,2) {$y$};
\node [above right] at (3,2) {$z=(x,y)$};
\draw (0,0) -- (3,2);
\node [above left,rotate=30] at (1.75,1.1) {$\abs{z}$};
\node at (8,2) {$\abs{z}=\sqrt{x^2+y^2}\ge0$};
\end{tikzpicture}
\end{figure*}

Note that $\abs{z}$ measures the distance from $z$ to the origin. Thus, if
$\abs{z_1}<\abs{z_2}$ then $z_1$ is closer to the origin than $z_2$.

\begin{theorem}
$\forall z\in\C$:
\begin{enumerate}
\item{$\abs{Re(z)}\le\abs{z}$}
\item{$\abs{Im(z)}\le\abs{z}$}
\item{$\abs{z}\le\abs{Re(z)}+\abs{Im(z)}$}
\end{enumerate}
\end{theorem}

\begin{theproof}
  Assume $z=x+iy\in\C$ \\
  $x=Re(z)$ \\
  $y=Im(z)$
  
  $\abs{z}^2=x^2+y^2=\abs{x}^2+\abs{y}^2=[Re(z)]^2+[Im(z)]^2$

  $\abs{z}^2\le\abs{Re(z)}^2$ \\
  $\therefore\abs{Re(z)}\le\abs{z}$

  $\abs{z}^2\le\abs{Im(z)}^2$ \\
  $\therefore\abs{Im(z)}\le\abs{z}$

  $\abs{z}=\abs{x+iy}\le\abs{x}+\abs{iy}=\abs{x}+\abs{i}\abs{y}=
  \abs{x}+1\cdot\abs{y}=\abs{x}+\abs{y}$ \\
  $\therefore\abs{z}\le\abs{Re(z)}+\abs{Im(z)}$
\end{theproof}

\begin{properties}
\listbreak
\begin{enumerate}
\item{$\abs{-z}=\abs{z}$}
\item{$\abs{\conj{z}}=\abs{z}$}
\item{$z\conj{z}=\abs{z}^2$}
\item{$\abs{z_1z_2}=\abs{z_1}\abs{z_2}$}
\item{$\abs{\frac{1}{z}}=\frac{1}{\abs{z}}$}
\item{$\abs{\frac{z_1}{z_2}}=\frac{\abs{z_1}}{\abs{z_2}}$}
\end{enumerate}
\end{properties}

\begin{theproof}
\listbreak
\begin{enumerate}
\item $\abs{-z}=\abs{-x-iy}=\sqrt{(-x)^2+(-y)^2}=\sqrt{x^2+y^2}=\abs{z}$

\item $\abs{\conj{z}}=\abs{x-iy}=\sqrt{x^2+(-y)^2}=\sqrt{x^2+y^2}=\abs{z}$

\item $z\conj{z}=(x+iy)(x-iy)=x^2-i^2y^2=x^2-(-1)y^2=x^2+y^2=\abs{z}^2$

\item{$\abs{z_1z_2}^2=(z_1z_2)\Conj{(z_1z_2)}=z_1z_2\conj{z_1}\conj{z_2}=
    (z_1\conj{z_1})(z_2\conj{z_2})=\abs{z_1}^2\abs{z_2}^2$} \\
$\therefore \abs{z_1z_2}=\abs{z_1}\abs{z_2}$

\item{$\abs{\frac{1}{z}}^2=
    \left(\frac{1}{z}\right)\Conj{\left(\frac{1}{z}\right)}=
    \left(\frac{1}{z}\right)\left(\frac{1}{\conj{z}}\right)=
    \frac{1}{z\conj{z}}=\frac{1}{\abs{z}^2}$} \\
$\therefore \abs{\frac{1}{z}}=\frac{1}{\abs{z}}$

\item{$\abs{\frac{z_1}{z_2}}=\abs{z_1\frac{1}{z_2}}=\abs{z_1}\abs{\frac{1}{z_2}}=
    \abs{z_1}\frac{1}{\abs{z_2}}=\frac{\abs{z_1}}{\abs{z_2}}$}
\end{enumerate}
\end{theproof}

\begin{example}
\begin{eqnarray*}
\abs{\frac{i(1-i)^3}{(\sqrt{2}+2i)^4}} &=&
    \abs{\frac{i(1-3i-3+i)}{4+16\sqrt{2}i-48-32\sqrt{2}i+16}} \\
    &=& \abs{\frac{i(-2-2i)}{-28-16\sqrt{2}i}} \\
    &=& \abs{\frac{i(1+i)}{14+8\sqrt{2}i}} \\
    &=& \abs{\frac{-1+i}{14+8\sqrt{2}i}} \\
    &=& \abs{\left(\frac{-1+i}{14+8\sqrt{2}i}\right)
        \left(\frac{14-8\sqrt{2}i}{14-8\sqrt{2}i}\right)} \\
    &=& \abs{\frac{(-1+i)(14-8\sqrt{2}i}{196+128}} \\
    &=& \abs{\frac{-14+8\sqrt{2}+i(14+8\sqrt{2})}{324}} \\
    &=& \abs{\frac{-7+4\sqrt{2}+i(7+4\sqrt{2})}{162}} \\
    &=& \frac{1}{162}\sqrt{(-7+4\sqrt{2})^2+(7+4\sqrt{2})^2} \\
    &=& \frac{1}{162}\sqrt{(49+32-56\sqrt{2})+(49+32+56\sqrt{2})} \\
    &=& \frac{1}{162}\sqrt{162} \\
    &=& \frac{9\sqrt{2}}{162} \\
    &=& \frac{\sqrt{2}}{18} \\
\end{eqnarray*}
\end{example}

\begin{theorem}
$\forall n\in\N$:
\[\abs{z^n}=\abs{z}^n\]
\end{theorem}

\begin{theproof}
(by induction)
\begin{description}
\item{Base Case: $n=1$} \\
\[\abs{z^1}=\abs{z}=\abs{z}^1\]

\item Assume $\abs{z^n}=\abs{z}^n$

\item Consider $\abs{z^{n+1}}$:
\[\abs{z^{n+1}}=\abs{z^nz}=\abs{z^n}\abs{z}=\abs{z}^n\abs{z}=\abs{z}^{n+1}\]
\end{description}
\end{theproof}

\begin{theorem}
$\forall z,a\in C$:
\[\abs{z+a}^2=\abs{z}^2+\abs{a}^2+2Re(\conj{a}z)\]
\end{theorem}

\begin{theproof}
\listbreak
\begin{eqnarray*}
\abs{z+a}^2 &=& (z+a)\Conj{(z+a)} \\
    &=& (z+a)(\conj{z}+\conj{a}) \\
    &=& z\conj{z}+a\conj{a}+z\conj{a}+\conj{z}a \\
    &=& \abs{z}^2+\abs{a}^2+z\conj{a}+\Conj{z\conj{a}} \\
    &=& \abs{z}^2+\abs{a}^2+2Re(\conj{a}z) \\
\end{eqnarray*}
\end{theproof}

\begin{corollary}
$\forall z\in\C, a\in\R$:
\[\abs{z+a}^2=\abs{z}^2+a^2+2aRe(z)\]
\end{corollary}
\end{document}
