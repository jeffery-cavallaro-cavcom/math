\documentclass[letterpaper,12pt,fleqn]{article}
\usepackage{matharticle}
\pagestyle{empty}
\newcommand{\conj}[1]{\overline{#1}}
\renewcommand{\S}{\mathcal{S}}
\newcommand{\G}{\Gamma}
\DeclareMathOperator{\im}{Im}
\begin{document}
\section*{Symmetry}

\begin{definition}
  To say $z$ and $z^*$ are symmetric with respect to a circle $C$ means there
  exists a LFT $=s(z)$ that maps $C$ onto the real number line and:
  \[\conj{s(z)}=s(z^*)\]
\end{definition}

\begin{minipage}[t]{3.5in}
  \begin{tikzpicture}
    \draw [<->] (-3,0) -- (3,0);
    \draw [<->] (0,-3) -- (0,3);
    \node [circle,fill,scale=0.5] (a) at (1,1) {};
    \node [below] at (a) {$a$};
    \draw (a) circle [radius=1.5];
    \draw [dashed] (a) -- node [below] {$R$} (2.5,1);
    \node [circle,fill,scale=0.5] (z) at (1.5,1.5) {};
    \node [circle,fill,scale=0.5] (zs) at (3,3) {};
    \draw (a) -- (z) node [above left] {$z$} --
    (zs) node [above right] {$z^*$};
    \draw [->] (3.5,1.5) -- node [above] {$s(z)$} (5.5,1.5);
  \end{tikzpicture}
\end{minipage}
\begin{minipage}[t]{3in}
  \begin{tikzpicture}
    \draw [<->] (-3,0) -- (3,0);
    \draw [<->] (0,-3) -- (0,3);
    \node [circle,fill,scale=0.5] (sz) at (2,1) {};
    \node [circle,fill,scale=0.5] (szs) at (2,-1) {};
    \draw [dashed] (0,0) -- (sz) node [above right] {$s(z)$};
    \draw [dashed] (0,0) -- (szs) node [below right] {$\conj{s(z)}=s(z^*)$};
    \draw [dashed] (sz) -- (szs);
  \end{tikzpicture}
\end{minipage}

\begin{theorem}
  Let $z$ and $z^*$ be symmetric with respect to circle $\abs{z-a}=R$:
  \[z^*=\frac{R^2}{\conj{z-a}}+a\]
  \[(z^*-a)(\conj{z}-\conj{a})=R^2\]
\end{theorem}

Note that when $a=0$, we get the familiar $z^*=\frac{R^2}{\conj{z}}$.

\begin{lemma}
  $\conj{(z_1,z_2,z_3,z_4)}=(\conj{z_1},\conj{z_2},\conj{z_3},\conj{z_4})$
\end{lemma}
\newpage
\begin{theproof}
  \listbreak
  \begin{eqnarray*}
    \conj{(z_1,z_2,z_3,z_4)} &=&
    \conj{\left[\frac{(z_1-z_3)(z_2-z_4)}{(z_1-z_4)(z_2-z_3)}\right]} \\
    &=& \frac{\conj{(z_1-z_3)(z_2-z_4)}}{\conj{(z_1-z_4)(z_2-z_3)}} \\
    &=& \frac{\conj{(z_1-z_3)}\,\conj{(z_2-z_4)}}
         {\conj{(z_1-z_4)}\,\conj{(z_2-z_3)}} \\
    &=& \frac{(\conj{z_1}-\conj{z_3})(\conj{z_2}-\conj{z_4})}
         {(\conj{z_1}-\conj{z_4})(\conj{z_2}-\conj{z_3})} \\
    &=& (\conj{z_1},\conj{z_2},\conj{z_3},\conj{z_4})
  \end{eqnarray*}
\end{theproof}

\begin{theorem}
  Let $z_1,z_2,z_3$ be on a circle $C$: \\
  $z$ and $z^*$ are symmetric wrt $C$ iff
  \[\conj{(z,z_1,z_2,z_3)}=(z^*,z_1,z_2,z_3)\]
\end{theorem}

\begin{theproof}
  \listbreak
  \begin{description}
    \item $\implies$ Assume $z$ and $z^*$ are symmetric wrt $C$

      Let $s(z)$ be a LFT from $C$ onto the real number line
      \begin{eqnarray*}
        \conj{(z,z_1,z_2,z_3)} &=& \conj{(s(z),s(z_1),s(z_2),s(z_3))} \\
        &=& (\conj{s(z)},\conj{s(z_1)},\conj{s(z_2)},\conj{s(z_3)}) \\
        &=& (s(z^*),s(z_1),s(z_2),s(z_3)) \\
      \end{eqnarray*}
      Now, apply the inverse relation $s^{-1}(z)$:
      \[\conj{(z,z_1,z_2,z_3)}=(z^*,z_1,z_2,z_3)\]

    \item $\impliedby$ Assume $\conj{(z,z_1,z_2,z_3)}=(z^*,z_1,z_2,z_3)$
  \end{description}
\end{theproof}

\begin{theorem}
  Let $z$ and $z^*$ be symmetric wrt a circle $C$ and let $s\in\S$: \\
  $s(z)$ and $s(z^*)$ are symmetric wrt some circle $\G$:
  \[\conj{(s(z),s(z_1),s(z_2),s(z_3))}=(s(z^*),s(z_1),s(z_2),s(z_3))\]
  Thus, a LFT preserves symmetry.
\end{theorem}
\newpage
\begin{theproof}
  Assume $z$ and $z^*$ are symmetric wrt circle $C$
  \begin{eqnarray*}
    \conj{(s(z),s(z_1),s(z_2),s(z_3))} &=& \conj{(z,z_1,z_2,z_3)} \\
    &=& (z^*,z_1,z_2,z_3) \\
    &=& (s(z^*),s(z_1),s(z_2),s(z_3))
  \end{eqnarray*}
\end{theproof}

\begin{example}
  Use symmetry to construct a conformal mapping from $\abs{z}<1$ to
  $\im(w)>0$.

  Find a suitable LFT:
  \begin{eqnarray*}
    0 &\to& i \\
    \infty &\to& -i \\
    1 &\to& 2 \\
    z &\to& w
  \end{eqnarray*}
  \begin{eqnarray*}
    (0,\infty,1,z) &=& (i,-i,2,w) \\
    \frac{(0-1)(\infty-z)}{(0-z)(\infty-1)} &=&
    \frac{(i-2)(-i-w)}{(i-w)(-i-2)} \\
    \frac{1}{z} &=& \frac{(i-2)(i+w)}{(i+2)(i-w)} \\
    z(i-2)(i+w) &=& (i+2)(i-w) \\
    -z+izw-i2z-2zw &=& -1-iw+i2-2w \\
    w(iz-2z+i+2) &=& z+i2z-1+i2 \\
    w((-2+i)z+(i+2)) &=& (1+2i)z+(-1+2i) \\
    w &=& \frac{(1+2i)z+(-1+2i)}{(-2+i)z+(i+2)}
  \end{eqnarray*}
\end{example}

\end{document}
