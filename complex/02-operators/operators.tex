\documentclass[letterpaper,12pt,fleqn]{article}
\usepackage{matharticle}
\pagestyle{empty}
\begin{document}
\section*{Equality and Operators}
\begin{definition}
To say that two complex numbers $z_1=(x_1,y_1)$ and $z_2=(x_2,y_2)$ are equal
$(z_1=z_2)$ means that $x_1=x_2$ and $y_1=y_2$.
\end{definition}
\begin{definition}
The following two binary operators are defined on $\C$:
\[z_1+z_2=(x_1,y_1)+(x_2,y_2)=(x_1+x_2,y_1+y_2)\]
\[z_1z_2=(x_1,y_1)(x_2,y_2)=(x_1x_2-y_1y_2,x_1y_2+x_2y_1)\]
\end{definition}
Note that these operators are closed and well-defined because each component in
the reals is closed and well-defined.

Every complex number $z$ can be expressed as follows:
\[z=(x,y)=(x,0)+(0,1)(0,y)=x+iy\]
where $i^2=(0,1)(0,1)=(-1,0)=-1$, or $i=\sqrt{-1}$.

Note that the powers of $i$ cycle every four:
\begin{eqnarray*}
i^0 &=& 1 \\
i^1 &=& i \\
i^2 &=& -1 \\
i^3 &=& -i \\
i^4 &=& 1 \\
\end{eqnarray*}
We can now redefine addition and multiplication as follows:
\begin{eqnarray*}
z_1+z_2 &=& (x_1+iy_1)+(x_2+iy_2) \\
    &=& (x_1+x_2) + i(y_1+y_2) \\
\end{eqnarray*}
\listbreak
\begin{eqnarray*}
z_1z_2 &=& (x_1+iy_1)(x_2+iy_2) \\
    &=& x_1x_2+ix_1y_2+ix_2y_1+i^2y_1y_2 \\
    &=& (x_1x_2-y_1y_2)+i(x_1y_2+x_2y_1) \\
\end{eqnarray*}

\newpage

\begin{theorem}
Let $z=x+iy$.
\begin{enumerate}
\item $Re(iz)=-Im(z)$
\item $Im(iz)=Re(z)$
\end{enumerate}
\end{theorem}

\begin{theproof}
$iz=i(x+iy)=ix+i^2y=-y+ix$ \\
$Re(iz)=-y=-Im(z)$ \\
$Im(iz)=x=Re(z)$
\end{theproof}
\end{document}
