\documentclass[letterpaper,12pt,fleqn]{article}
\usepackage{matharticle}
\pagestyle{empty}
\renewcommand{\o}{\theta}
\newcommand{\Arg}[1]{\mathrm{Arg}\,#1}
\newcommand{\conj}[1]{\overline{#1}}
\allowdisplaybreaks
\begin{document}
\section*{Functions}

\begin{definition}
  Let $S\subseteq\C$. A \emph{function} $f$ is a rule that assigns to each
  $z\in S$ a value $w\in\C$, denoted:
  \[w=f(z)\]
  where $w$ is called the value of the function $f$ at $z$.

  The set $S$ is called the \emph{domain} of $f$. When no domain is explicitly
  stated then the domain is assumed to be as large as possible.
\end{definition}

\begin{minipage}[t]{3in}
  $w=f(x+iy)=u(x,y)+iv(i,y)$

  $u(x,y)=Re(w)$ \\
  $v(x,y)=Im(w)$
\end{minipage}
\begin{minipage}[t]{3in}
  $w=f(re^{i\o})=u(r,\o)+iv(r,\o)$

  $u(r,\o)=Re(w)$ \\
  $v(r,\o)=Im(w)$
\end{minipage}

\begin{example}
  $f(z)=z^2$
  
  Domain: $\C$ \\
  \begin{minipage}[t]{3in}
    \begin{eqnarray*}
      f(x+iy) &=& (x+iy)^2 \\
      &=& x^2-y^2+i2xy \\
      \\
      u &=& x^2-y^2 \\
      v &=& 2xy \\
    \end{eqnarray*}
  \end{minipage}
  \begin{minipage}[t]{3in}
    \begin{eqnarray*}
      f(re^{i\o}) &=& (re^{i\o})^2 \\
      &=& r^2e^{i2\o} \\
      &=& r^2\cos{2\o}+ir^2\sin{2\o} \\
      \\
      u &=& r^2\cos{2\o} \\
      v &=& r^2\sin{2\o} \\
    \end{eqnarray*}
  \end{minipage}
\end{example}

When $Im(w)=0$ then $f$ is called a real-valued function of a complex variable.

\begin{example}
  $f(x)=\abs{z}=\sqrt{x^2+y^2}$

  $u=\sqrt{x^2+y^2}$ \\
  $v=0$
\end{example}
\newpage
\begin{example}
  Let $Z=f(z)=\frac{z-1}{z+1}$

  Find: $\Arg{Z}$

  \begin{eqnarray*}
    Re(Z) &=& \frac{Z+\conj{Z}}{2} \\
    &=& \frac{1}{2}\left[\frac{z-1}{z+1}+
      \conj{\left(\frac{z-1}{z+1}\right)}\right] \\
    &=& \frac{1}{2}\left[\frac{z-1}{z+1}+
      \frac{\conj{z}-1}{\conj{z}+1}\right] \\
    &=& \frac{1}{2}\left[\frac{(z-1)(\conj{z}+1)+(z+1)(\conj{z}-1)}
      {(z+1)(\conj{z}+1)}\right] \\
    &=& \frac{1}{2}\left[\frac{z\conj{z}+z-\conj{z}-1+z\conj{z}-z+\conj{z}-1}
      {z\conj{z}+z+\conj{z}+1}\right] \\
    &=& \frac{1}{2}\left[\frac{2\abs{z}^2-2}{\abs{z}^2+z+\conj{z}+1}\right] \\
    &=& \frac{\abs{z}^2-1}{\abs{z}^2+(z+\conj{z})+1} \\
    &=& \frac{\abs{z}^2-1}{\abs{z}^2+2Re(z)+1} \\
    &=& \frac{x^2+y^2-1}{x^2+y^2+2x+1} \\
  \end{eqnarray*}

  \begin{eqnarray*}
    Im(Z) &=& \frac{Z-\conj{Z}}{2i} \\
    &=& \frac{1}{2i}\left[\frac{z-1}{z+1}-
      \conj{\left(\frac{z-1}{z+1}\right)}\right] \\
    &=& \frac{1}{2i}\left[\frac{z-1}{z+1}-
      \frac{\conj{z}-1}{\conj{z}+1}\right] \\
    &=& \frac{1}{2i}\left[\frac{(z-1)(\conj{z}+1)-(z+1)(\conj{z}-1)}
      {(z+1)(\conj{z}+1)}\right] \\
    &=& \frac{1}{2i}\left[\frac{z\conj{z}+z-\conj{z}-1-z\conj{z}+z-\conj{z}+1}
      {z\conj{z}+z+\conj{z}+1}\right] \\
    &=& \frac{1}{2i}\left[\frac{2z-2\conj{z}}{\abs{z}^2+z+\conj{z}+1}\right] \\
    &=& 2\left[\frac{\frac{z-\conj{z}}{2i}}{\abs{z}^2+(z+\conj{z})+1}\right] \\
    &=& \frac{2Im(z)}{\abs{z}^2+2Re(z)+1} \\
    &=& \frac{2y}{x^2+y^2+2x+1} \\
  \end{eqnarray*}

  \[Z=\frac{x^2+y^2-1}{x^2+y^2+2x+1}+i\frac{2y}{x^2+y^2+2x+1}\]
  \[\Arg{Z}=\tan^{-1}\frac{Im(Z)}{Re(Z)}=\tan^{-1}\frac{2y}{x^2+y^2-1}\]
\end{example}

\end{document}
