\documentclass[letterpaper,12pt,fleqn]{article}
\usepackage{matharticle}
\usepackage{tikz}
\pagestyle{empty}
\begin{document}
\section*{Complex Numbers}
Complex numbers arise when attempting to solve polynomials that have more than
just real roots. For example:
\begin{eqnarray*}
x^2+1 &=& 0 \\
x^2 &=& -1 \\
x &=& \pm\sqrt{-1} \\
\end{eqnarray*}
\begin{definition}
The set of complex numbers $\C$ is defined by:
\[\C=\{z=(x,y)\mid x,y\in\R\}\]
$x=Re(z)$ is called the \emph{real} part of $z$ \\
$y=Im(z)$ is called the \emph{imaginary} part of $z$ \\
\end{definition}
Complex numbers are interpreted as points on the complex plane:
\begin{figure*}[h]
\setlength{\leftskip}{0.5in}
\begin{tikzpicture}
\draw [<->] (-3,0) -- (3,0);
\draw [<->] (0,-3) -- (0,3);
\node [right] at (3,0) {Re};
\node [above] at (0,3) {Im};
\draw [fill=black] (1,2) circle [radius=0.1];
\draw [dashed] (0,2) -- (1,2);
\draw [dashed] (1,0) -- (1,2);
\node [below] at (1,0) {$x$};
\node [left] at (0,2) {$y$};
\node [above right] at (1,2) {$z=(x,y)$};
\end{tikzpicture}
\end{figure*}

Note that $\R=\{(x,0)\mid x\in\R\}$ so $\R\subset\C$.
\end{document}
