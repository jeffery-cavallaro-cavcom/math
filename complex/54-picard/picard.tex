\documentclass[letterpaper,12pt,fleqn]{article}
\usepackage{matharticle}
\pagestyle{empty}
\begin{document}
\section*{Picard's Theorem}

\begin{theorem}
  In every neighborhood of an essential singularity, an analytic function
  takes on every value (with one possible exception) an infinite number of
  times.
\end{theorem}

\begin{example}
  $w=f(z)=e^{\frac{1}{z}}$

  $f(z)=\sum_{k=0}^{\infty}\frac{1}{k!z^k}$ \\
  Thus, $z=0$ is an essential singularity for $f(z)$ \\
  Note that $f(z)\ne0$; this is the one exception \\
  Let $w_0=e^{\frac{1}{z}}$ \\
  $\log{w_0}+2k\pi i=\frac{1}{z}$ \\
  $z=\frac{1}{\log{w_0}+2k\pi i}$

  Thus, each $w_0$ occurs at an infinite number of $z$.
\end{example}

\end{document}
