\documentclass[letterpaper,12pt,fleqn]{article}
\usepackage{matharticle}
\pagestyle{empty}
\begin{document}
\section*{Integers}

\begin{definition}
  The set of \emph{integers} includes the positive and negative whole numbers
  and zero and is given by:
  \[\Z=\{\ldots,-3,-2,-1,0,1,2,3,\ldots\}\]
\end{definition}

Thus, by the trichotomy principle, $\forall\,n\in\Z$, exactly one of the
following is true:
\begin{enumerate}
\item $n$ is positive
\item $n$ is zero
\item $n$ is negative
\end{enumerate}

\begin{definition}
  Let $a$ and $b$ be two integer values, regardless of representation. To say
  that $a$ equals $b$, denoted $a=b$, means that $a$ and $b$ represent the same
  element in $\Z$.
\end{definition}

\begin{axiom}[Substitution Principle]
  Let $a$ and $b$ be two integer values, regardless of representation. If $a=b$
  then $a$ and $b$ can syntactically replace each other in a given context
  without altering the context.
\end{axiom}

\begin{properties}[Equality]
  Let $a$ and $b$ be two integer values:
  \begin{enumerate}
  \item Reflexivity
    \[a=a\]
    
  \item Symmetry
    \[a=b\implies b=a\]

  \item Transitivity
    \[a=b\ \mbox{and}\ b=c\implies a=c\]
  \end{enumerate}
\end{properties}

\begin{properties}
  The set of integers is a commutative ring with unity under the binary
  operations of addition and multiplication, and as such, the following axioms
  hold:
\newpage
  $\forall\,a,b,c,d\in\Z:$

  \begin{enumerate}
  \item{Well-defined}
    \begin{itemize}
    \item $a+b=c$ and $a+b=d\implies c=d$
    \item $ab=c$ and $ab=d\implies c=d$
    \end{itemize}
    
  \item{Closure}
    \begin{itemize}
    \item $a+b\in\Z$
    \item $ab\in\Z$
    \end{itemize}
    
  \item{Cummutativity}
    \begin{itemize}
    \item $a+b=b+a$
    \item $ab=ba$
    \end{itemize}

  \item{Associativity}
    \begin{itemize}
    \item $(a+b)+c=a+(b+c)$
    \item $(ab)c=a(bc)$
    \end{itemize}
    
  \item{Distributivity}
    \begin{itemize}
    \item $a(b+c)=ab+ac$
    \end{itemize}

  \item{Identity}
    \begin{itemize}
    \item $a+0=a$
    \item $a1=a$
    \end{itemize}

  \item{Additive Inverse}
    \begin{itemize}
    \item $\exists\,(-a)\in\Z,a+(-a)=0$
    \end{itemize}
  \end{enumerate}
\end{properties}

Since the set of integers is a commutative ring with unity, the following
properties also hold:

\begin{properties}[Zero]
  $\forall\,a,b\in\Z$:
  \begin{enumerate}
  \item $-0=0$
  \item $a0=0$
  \item $ab=0\implies a=0$ or $b=0$
  \end{enumerate}
\end{properties}
\newpage
\begin{properties}[Negatives]
  $\forall\,a,b\in\Z$:
  \begin{enumerate}
  \item $(-1)a=-a$
  \item $-(-a)=a$
  \item $(-a)b=-(ab)=a(-b)$
  \item $(-a)(-b)=ab$
  \item $-(a+b)=(-a)+(-b)$
  \end{enumerate}
\end{properties}

\begin{notation}
  Integer subtraction is nothing more than a syntactic convenience:
  \[\forall\,a,b\in\Z,a-b=a+(-b)\]
\end{notation}

\begin{properties}[Cancellation]
  $\forall\,a,b,c\in\Z$:
  \begin{enumerate}
  \item $a=b\iff a+c=b+c$
  \item $a=b\implies ac=bc$
  \item $ac=bc$ and $c\ne0\implies a=b$
  \end{enumerate}
\end{properties}

The last cancellation law cannot be proven in the common manner due to the
lack of multiplication inverses.  Instead:

\begin{theproof}
  Assume $a,b,c\in\Z$ \\
  Assume $ac=bc$ and $c\ne0$ \\
  $ac-bc=0$ \\
  $(a-b)c=0$ \\
  Since $c\ne0$, $a-b=0$ \\
  $\therefore a=b$
\end{theproof}

\end{document}
