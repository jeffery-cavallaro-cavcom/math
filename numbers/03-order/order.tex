\documentclass[letterpaper,12pt,fleqn]{article}
\usepackage{matharticle}
\pagestyle{empty}
\begin{document}
\section*{Integer Ordering}

\begin{definition}
  The set of \emph{positive integers}, a subset of $\Z$, is given by:
  \[\Z^+=\{1,2,3,\ldots\}\]
\end{definition}

Note that $\Z^+$ inherits from $\Z$ all those properties that do not involve
inverses. In particular, $\Z^+$ is closed under addition and multiplication.

\begin{definition}
  $\forall a,b\in\Z$:
  \begin{itemize}
  \item To say that $a$ is less than $b$, denoted $a<b$, means $b-a$ is a
    positive value. This can also be stated as $b$ is greater than $a$,
    denoted $b>a$.

  \item To say that $a$ is less than or equal to $b$, denoted $a\le b$, means
    $a<b$ or $a=b$: $b-a$ is either positive or zero. This can also be stated
    as $b$ is greater than or equal to $a$, denoted $b\ge a$
  \end{itemize}
\end{definition}

\begin{theorem}
  $\forall\,a\in\Z$:
  \begin{itemize}
    \item $a$ is positive $\iff a>0$
    \item $a$ is negative $\iff a<0$
  \end{itemize}
\end{theorem}

\begin{theproof}
  Assume $a\in\Z$
  
  \begin{minipage}{3in}
    \begin{eqnarray*}
      a\ \mbox{is positive} &\iff& a\in\Z^+ \\
      &\iff& a-0\in\Z^+ \\
      &\iff& 0<a \\
      &\iff& a>0 \\
    \end{eqnarray*}
  \end{minipage}
  \begin{minipage}{3in}
    \begin{eqnarray*}
      a\ \mbox{is negative} &\iff& -a\in\Z^+ \\
      &\iff& -a+0\in\Z^+ \\
      &\iff& 0-a\in\Z^+ \\
      &\iff& a<0 \\
    \end{eqnarray*}
  \end{minipage}
\end{theproof}

Thus, the trichotomy principle can be rewritten as follows:

$\forall\,n\in\Z$, exactly one of the following is true:
\begin{enumerate}
\item $n>0$
\item $n=0$
\item $n<0$
\end{enumerate}
\newpage
\begin{properties}
  $\forall a,b,c\in\Z$:
  \begin{enumerate}
  \item $a<b$ and $b<c\implies a<c$
  \item $a<b$ and $c<d\implies a+c<b+d$
  \item $a<b\iff a+c<b+c$
  \item $c>0\implies\left(a<b\iff ac<bc\right)$
  \item $c<0\implies\left(a<b\iff ac>bc\right)$
  \end{enumerate}
\end{properties}

Note that all of the above properties hold if `$<$' is replaced with '$\le$'.

\begin{theproof}
  Assume $a,b,c\in\Z$
  \begin{enumerate}
  \item Assume $a<b$ and $b<c$

    $b-a\in\Z^+$ and $c-b\in\Z^+$ \\
    By closure $(b-a)+(c-b)\in\Z^+$ \\
    $(b-a)+(c-b)>0$ \\
    $c-a>0$ \\
    $\therefore a<c$

  \item Assume $a<b$ and $c<d$

    $b-a\in\Z^+$ and $d-c\in\Z^+$ \\
    By closure $(b-a)+(d-c)\in\Z^+$ \\
    $(b-a)+(d-c)>0$ \\
    $(b+d)-(a+c)>0$ \\
    $\therefore a+c<b+d$

  \item \begin{eqnarray*}
    a<b &\iff& b-a>0 \\
    &\iff& b-a+0>0 \\
    &\iff& b-a+c-c>0 \\
    &\iff& (b+c)-(a+c)>0 \\
    &\iff& a+c<b+c \\
  \end{eqnarray*}
\newpage
  \item Assume $c>0$
    
    $c\in\Z^+$

    \begin{eqnarray*}
      a<b &\iff& b-a\in\Z^+ \\
      &\iff& c(b-a)\in\Z^+ \\
      &\iff& c(b-a)>0 \\
      &\iff& bc-ac>0 \\
      &\iff& ac<bc \\
    \end{eqnarray*}

  \item Assume $c<0$
    
    $0-c=-c\in\Z^+$

    \begin{eqnarray*}
      a<b &\iff& b-a\in\Z^+ \\
      &\iff& (-c)(b-a)\in\Z^+ \\
      &\iff& (-c)(b-a)>0 \\
      &\iff& ac-bc>0 \\
      &\iff& bc<ac \\
      &\iff& ac>bc \\
    \end{eqnarray*}
  \end{enumerate}
\end{theproof}

\begin{theorem}
  \listbreak
  \[\forall\,a,k\in\Z^+,ka\ge a\]
\end{theorem}

\begin{theproof}
  Assume $k,a\in\Z^+$ \\
  ABC: $ka<a$ \\
  $ka-a=a(k-1)<0$
  \begin{description}
  \item{Case 1: $k=1$}

    $a(1-1)=a0=0$
    CONTRADICTION!
\newpage
  \item {Case 2: $k>1$}

    $k-1>0$ \\
    $k-1\in\Z^+$ \\
    $a(k-1)\in\Z^+$ \\
    $a(k-1)>0$ \\
    CONTRADICTION!
  \end{description}
  $\therefore ka\ge a$
\end{theproof}

\begin{definition}
  Let $S\subseteq\Z$:
  \begin{itemize}
  \item To say that $S$ has a \emph{minimum} element means:
    \[\exists\,m\in S,\forall\,n\in S,m\le n\]

  \item To say that $S$ has a \emph{maximum} element means:
    \[\exists\,m\in S,\forall\,n\in S,n\le m\]
  \end{itemize}
\end{definition}

\begin{axiom}[Well-ordering Principle]
  Every non-empty subset of $Z^+$ has a minimum value.
\end{axiom}

In fact, any ordered set that has this property is said to be
\emph{well-ordered}.

\end{document}
