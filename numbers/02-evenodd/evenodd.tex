\documentclass[letterpaper,12pt,fleqn]{article}
\usepackage{matharticle}
\pagestyle{empty}
\begin{document}
\section*{Even/Odd Integers}

\begin{definition}
  To say that $n\in\Z$ is \emph{even} means that it can be expressed as:
  \[n=2k,k\in\Z\]

  To say that $n\in\Z$ is \emph{odd} means that it can be expressed as:
  \[n=2k+1,k\in\Z\]
\end{definition}

\begin{theorem}
  An odd integer $n$ can be expressed as:
  \[n=2k-1,k\in\Z\]
\end{theorem}

\begin{theproof}
  Assume $n$ is odd \\
  $\exists\,h\in\Z,n=2h+1$ \\
  Let $h=k-1\in\Z$ \\
  $n=2h+1=2(k-1)+1=2k-1$ \\
\end{theproof}

\begin{theorem}
  An integer is either even or odd.
\end{theorem}

\begin{theproof}
  Assume $n\in\Z$
  \begin{description}
  \item{Case 1: $n$ is positive}

    Proof by strong induction on $n$

    \begin{description}
    \item {Base: n=1,2}

      $1=2\cdot0+1$ \\
      $2=2\cdot1$ \\
      $\therefore 1$ is odd and $2$ is even.

    \item Assume all integers from $1$ to $n$ are either even or odd.

    \item Consider $n+1$

      \begin{description}
      \item {Case 1: $n$ is even}

        $\exists\,k\in\Z,n=2k$ \\
        $n+1=2k+1$ \\
        $\therefore n+1$ is odd.
        
      \item {Case 2: $n$ is odd}

        $\exists\,k\in\Z,n=2k-1$ \\
        $n+1=2k-1+1=2k$ \\
        $\therefore n+1$ is even.
      \end{description}
    \end{description}

  \item {Case 2: $n=0$}

    $0=2\cdot0$ \\
    $\therefore 0$ is even.

  \item {Case 3: $n$ is negative}

    Let $m=-n$ \\
    $m$ is positive
    
    \begin{description}
    \item {Case 1: $m$ is even}

      $\exists\,k\in\Z,m=2k$ \\
      $n=-m=-(2k)=2(-k)$ \\
      But $(-k)\in\Z$ \\
      $\therefore n$ is even.
      
    \item {Case 2: $m$ is odd}

      $\exists\,k\in\Z,m=2k-1$ \\
      $n=-m=-(2k-1)=2(-k)+1$ \\
      But $(-k)\in\Z$ \\
      $\therefore n$ is odd.
    \end{description}
  \end{description}
\end{theproof}

\begin{theorem}
  $\forall\,n\in\Z,n$ is even $\iff n^2$ is even
\end{theorem}

\begin{theproof}
  \listbreak
  \begin{description}
  \item $\implies$ Assume $n$ is even

    $\exists\,k\in\Z,n=2k$ \\
    $n^2=(2k)^2=4k^2=2(2k^2)$ \\
    But, by closure, $2k^2\in\Z$ \\
    $\therefore n^2$ is even.

  \item $\impliedby$ Assume $n$ is odd

    $\exists\,k\in\Z,n=2k+1$ \\
    $n^2=(2k+1)^2=4k^2+4k+1=2(2k^2+2k)+1$ \\
    But, by closure, $2k^2+2k\in\Z$ \\
    $\therefore n^2$ is odd.
  \end{description}
\end{theproof}

\end{document}
