\documentclass[letterpaper,12pt,fleqn]{article}
\usepackage{matharticle}
\pagestyle{empty}
\renewcommand{\a}{\alpha}
\begin{document}
\section*{Real Numbers}

\begin{definition}
  \listbreak
  \begin{itemize}
    \item The set of \emph{rational} numbers is given by:
      \[\Q=\left\{\frac{p}{q}\mid p,q\in\Z\ \mbox{and}\ q\ne0\right\}\]

    \item A number that is not rational is called \emph{irrational}.

    \item The set of \emph{real} numbers, denoted $\R$, is the union of the
      sets of rational and irrational numbers.
  \end{itemize}
\end{definition}

The set of irrational numbers is often denoted: $\R-\Q$.

Note that $\Z\subset\Q$ because $\forall\,n\in\Z,n=\frac{n}{1}$.

\begin{definition}
  \listbreak
  \begin{itemize}
  \item To say that $\a\in\R$ is \emph{algebraic} means that it is a zero of
    some polynomial with integer coefficients:
    \[\exists\,a_k,\sum_{k=0}^na_k\a^k=0\]

    \item A number that is not algebraic is called \emph{transcendental}.
  \end{itemize}
\end{definition}

\begin{theorem}
  $\Q$ is closed under addition and multiplication.
\end{theorem}

\begin{theproof}
  Assume $a,b\in\Q$ \\
  Let $a=\frac{p}{q}$ and $b=\frac{r}{s}$ where $p,q,r,s\in\Z$ and $p,q\ne0$

  \bigskip

  $a+b=\frac{p}{q}+\frac{r}{s}=\frac{ps+qr}{qs}$ \\
  $ps+qr\in\Z$ \\
  $qs\in\Z$ and $qs\ne0$ \\
  $\therefore a+b\in\Q$

  \bigskip

  $ab=\frac{p}{q}\cdot\frac{r}{s}=\frac{pr}{qs}$ \\
  $pr\in\Z$ \\
  $qs\in\Z$ and $qs\ne0$ \\
  $\therefore ab\in\Q$
\end{theproof}

\begin{example}
  Prove that $\sqrt{2}$ is irrational.

  \underline{Proof 1}

  ABC: $\sqrt{2}$ is rational

  $\exists\,a,b\in\Z^+,\frac{a}{b}=\sqrt{2}$ and either $a$ or $b$ not even \\
  $\frac{a^2}{b^2}=2$ \\
  $a^2=2b^2$ \\
  $a^2$ is even, so $a$ is even, so $b$ must be odd \\
  Let $a=2k,k\in\Z$ \\
  $(2k)^2=4k^2=2(2k^2)=b^2$ \\
  $b^2$ is even, so $b$ is even \\
  CONTRADICTION! \\
  $\therefore\sqrt{2}$ is irrational.

  \bigskip

  \underline {Proof 2}

  ABC: $\sqrt{2}$ is rational

  $\exists\,a,b\in\Z^+,\frac{a}{b}=\sqrt{2}$ \\
  Let $S=\{k\sqrt{2}\mid k,k\sqrt{2}\in\Z^+\}$ \\
  Note that $S\ne\emptyset$ because $a=b\sqrt{2}$ \\
  By the well-ordering principle, $S$ has a minimum \\
  Let $s=\min S\in\Z^+$ \\
  Let $s=t\sqrt{2},t\in\Z^+$ \\
  $s\sqrt{2}-s=s\sqrt{2}-t\sqrt{2}=(s-t)\sqrt{2}$ \\
  $s\sqrt{2}=(t\sqrt{2})\sqrt{2}=2t\in\Z^+$ \\
  Since $\sqrt{2}>1$, $s\sqrt{2}>s$ and $s\sqrt{2}-s>0$ \\
  $(s-t)\sqrt{2}>0$ \\
  $s-t>0$ \\
  So $(s-t)\sqrt{2}\in S$ \\
  But $s\sqrt{2}-s=s(\sqrt{2}-1)<s$ \\
  CONTRADICTION (of the minimality of $s$)! \\
  $\therefore\sqrt{2}$ is irrational.

  \bigskip

  \underline {Proof 3}

  Let $x=\sqrt{2}$ \\
  $x^2=2$ \\
  $x^2-2=0$ \\
  $\sqrt{2}$ is algebraic \\
  The only possible rational zeros are $\pm2$ \\
  $(\pm2)^2-2=4-2=2\ne0$ \\
  So there are no rational zeros \\
  $\therefore\sqrt{2}$ is irrational.
\end{example}

\end{document}
