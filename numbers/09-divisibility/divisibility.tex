\documentclass[letterpaper,12pt,fleqn]{article}
\usepackage{matharticle}
\pagestyle{empty}
\begin{document}
\section*{Divisibility}

Division of integers is problematic because it involves an apparent jump to
rational numbers. In order to avoid this, division of integers is defined in
terms of multiplication:

\begin{definition}
  Let $n,m\in\Z,n\ne0$. To say that $n$ \emph{divides} $m$, denoted $n\mid m$,
  means:
  \[\exists\,k\in\Z,m=kn\]
  The integer $n$ is called a \emph{divisor} or \emph{factor} of $m$ and $m$ is
  called a \emph{multiple} of $n$.
\end{definition}

\begin{theorem}
  \listbreak
  \[\forall\,a,b\in\Z^+,a\mid b\implies a\le b\]
\end{theorem}

\begin{theproof}
  Assume $a,b\in\Z^+$ \\
  Assume $a\mid b$ \\
  $\exists\,k\in\Z^+,b=ka$ \\
  ABC: $a>b$ \\
  $a>ka$ \\
  CONTRADICTION! \\
  $\therefore a\le b$
\end{theproof}

\begin{theorem}
  Divisibility is transitive:
  \[\forall,a,b,c\in\Z,a\mid b\ \mbox{and}\ b\mid c\implies a\mid c\]
\end{theorem}

\begin{theproof}
  Assume $a,b,c\in\Z$ \\
  Assume $a\mid b$ and $b\mid c$ \\
  $\exists\,h,k,b=ha$ and $c=kb$ \\
  $c=k(ha)=(kh)a$ \\
  But, by closure, $kh\in\Z$ \\
  $\therefore a\mid c$
\end{theproof}
\newpage
\begin{theorem}
$\forall\,a,b,c\in\Z,c\mid a$ and $c\mid b\implies\forall\,m,n\in\Z,
    c\mid(ma+nb)$
\end{theorem}

\begin{theproof}
  Assume $a,b,c\in\Z$ \\
  Assume $c\mid a$ and $c\mid b$ \\
  $\exists\,h,k\in\Z,a=hc$ and $b=kc$ \\
  Assume $m,n\in\Z$ \\
  $ma=m(hc)=(mh)c$ \\
  $nb=n(kc)=(nk)c$ \\
  $ma+nb=(mh)c+(nk)c=(mh+nk)c$ \\
  But, by closure, $mh+nk\in\Z$ \\
  $\therefore c\mid ma+nb$
\end{theproof}

\end{document}
