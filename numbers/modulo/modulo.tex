\documentclass[letterpaper,12pt,fleqn]{article}
\usepackage{matharticle}
\usepackage{changepage}
\newcommand{\emod}[3]{#1\equiv#2\ (\mbox{mod}\ #3)}
\newcommand{\ec}[1]{\overline{#1}}
\pagestyle{empty}
\begin{document}
\section*{Modulo Congruence}
\begin{definition}
Let $n\in Z^+$. To say that $a$ is equivalent to $b$ modulo $n$, denoted
$a\equiv_nb$ or $\emod{a}{b}{n}$, means:
\[n\mid(b-a)\]
\end{definition}
\begin{theorem}
Let $n\in Z^+$:
\[\emod{a}{b}{n}\iff\exists\,k\in\Z,b=a+kn\]
\end{theorem}
\begin{theproof}
\listbreak
\begin{eqnarray*}
\emod{a}{b}{c} &\iff& n\mid(b-a) \\
    &\iff& \exists\,k\in\Z,b-a=kn \\
    &\iff& b=a+kn \\
\end{eqnarray*}
\end{theproof}
\begin{theorem}
Modulo congruence is an equivalence relation on $\Z$.
\end{theorem}
\begin{theproof}
Assume $n\in\Z^+$.
\begin{enumerate}
\item Assume $a\in\Z$. \\
$a-a=0$ \\
$n\mid0$ \\
$n\mid(a-a)$ \\
$\emod{a}{a}{n}$ \\
$a\sim a$ \\
Therefore, modulo congruence is reflexive.

\item Assume $a\sim b$. \\
$\emod{a}{b}{n}$ \\
$n\mid(b-a)$ \\
$\exists\,k\in\Z,b-a=kn$ \\
$a-b=(-k)n$ \\
$-k\in\Z$ \\
$n\mid(a-b)$ \\
$\emod{b}{a}{n}$ \\
$b\sim a$ \\
Therefore, modulo congruence is symmetric.
\newpage
\item Assume $a\sim b$ and $b\sim c$. \\
$\emod{a}{b}{n}$ and $\emod{b}{c}{n}$ \\
$n\mid(b-a)$ and $n\mid(c-b)$ \\
$\exists\,h\in\Z,(b-a)=hn$ \\
$\exists\,k\in\Z,(c-b)=kn$ \\
$(b-a)+(c-b)=hn+kn$ \\
$c-a=(h+k)n$ \\
$h+k\in\Z$ \\
$n\mid(c-a)$ \\
$\emod{a}{c}{n}$ \\
$a\sim c$ \\
Therefore, modulo congruence is transitive.
\end{enumerate}
\end{theproof}
The $n$ equivalence classes: $\ec{0},\ec{1},\ldots,\ec{n-1}$, are called the
\emph{residue} classes modulo $n$.
\begin{example}
Let $n=15$

$\ec{a}=\{a+kn\mid k\in\Z\}$

$\ec{0}=\{0,15,-15,30,-30,\ldots\}$ \\
$\ec{1}=\{1,16,-14,31,-29,\ldots\}$ \\
$\hspace{0.5in}\vdots$ \\
$\ec{14}=\{14,29,-1,44,-16,\ldots\}$ \\
\end{example}
Per the division algorithm, the residue class modulo $n$ for $m\in\Z$ is the
remainder $r$:
\begin{adjustwidth}{0.5in}{\rightskip}
$m=nq+r,\hspace{0.25in} 0\le r<n$ \\
$m-r=nq$ \\
$r-m=(-q)n$ \\
$n\mid(r-m)$ \\
$\emod{m}{r}{n}$ \\
\end{adjustwidth}
To find the residue class $\ec{r}$ modulo $n$ for a given $m\in\Z$:
\[r=m-\left\lfloor\frac{m}{n}\right\rfloor\cdot n\]
\newpage
\begin{example}
Let $n=15$

\bigskip

$\frac{1796}{15}\approx119.73$ \\
$r=1796-119\cdot 15=11$ \\
$\emod{1796}{11}{15}$ \\
$1796\in\ec{11}$

\bigskip

$\frac{-1796}{15}\approx-119.73$ \\
$r=-1796+120\cdot 15=4$ \\
$\emod{-1796}{4}{15}$ \\
$-1796\in\ec{4}$
\end{example}
\end{document}
