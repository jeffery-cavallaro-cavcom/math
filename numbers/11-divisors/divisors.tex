\documentclass[letterpaper,12pt,fleqn]{article}
\usepackage{matharticle}
\pagestyle{empty}
\begin{document}
\section*{Divisors}

\begin{definition}
  Let $n\in\Z$. The set of \emph{divisors} of $n$, denoted $D_n$, is given by:
  \[D_n=\{d\in\Z\mid d\mid n\}\]
\end{definition}

\begin{theorem}
  \listbreak
  \[D_0=\Z\]
\end{theorem}

\begin{theproof}
  \listbreak
  \begin{description}
  \item{$\implies$} By definition, $D_0\subseteq\Z$.
  \item{$\impliedby$} Assume $d\in\Z$

    $0\in\Z$ \\
    $0d=0$ \\
    $d\mid0$ \\
    $d\in D_0$ \\
    $\Z\subseteq D_0$
  \end{description}
  $\therefore D_0=\Z$
\end{theproof}

\begin{theorem}
  Let $n\in\Z,n\ne0$:
  \[0\notin D_n\]
\end{theorem}

\begin{theproof}
  ABC: $0\in D_n$ \\
  $\exists\,k\in\Z,k0=0=n$ \\
  CONTRADICTION! \\
  $\therefore 0\notin D_n$
\end{theproof}

\begin{theorem}
  \listbreak
  \[\forall\,n\in\Z,D_n=D_{-n}\]
\end{theorem}
\newpage
\begin{theproof}
  Assume $n\in\Z$
  \begin{eqnarray*}
    d\in D_n &\iff& d\mid n \\
    &\iff& \exists\,k\in\Z,kd=n \\
    &\iff& \exists\,-k\in\Z,(-k)d=-n \\
    &\iff& d\mid -n \\
    &\iff& d\in D_{-n} \\
  \end{eqnarray*}
\end{theproof}

\begin{theorem}
  $\forall\,n\in\Z,D_n\ne\emptyset$. In fact:
  \[\forall\,n\in\Z,\{\pm1,\pm n\}\subseteq D_n\]
\end{theorem}

\begin{theproof}
  Assume $n\in\Z$

  \begin{minipage}[t]{2in}
    $1\in\Z$ \\
    $n1=n$ \\
    $\therefore 1,n\in D_n$
  \end{minipage}
  \begin{minipage}[t]{2in}
    $-n\in\Z$ \\
    $-1\in\Z$ \\
    $(-n)(-1)=n$ \\
    $\therefore -1,-n\in D_n$
  \end{minipage}
\end{theproof}

\begin{theorem}
  Let $n\in\Z,n\ne0$. $D_n$ is finite. In fact:
  \[\forall\,d\in D_n,1\le\abs{d}\le\abs{n}\]
\end{theorem}

\begin{theproof}
  Assume $d\in D_n$

  $\abs{d}\ge0$ \\
  $0\ne D_n$ \\
  $1\in D_n$ \\
  $\therefore 1\le\abs{d}$

  $\exists\,k\in\Z^+,k\abs{d}=\abs{n}$ \\
  But $k\abs{d}\ge \abs{d}$, since $k\ge1$ \\
  $\therefore \abs{d}\le\abs{n}$
\end{theproof}

\end{document}
