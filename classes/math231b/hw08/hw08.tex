\documentclass[letterpaper,12pt,fleqn]{article}
\usepackage{matharticle}
\usepackage{enumitem}
\pagestyle{plain}
\newcommand{\p}{\phi}
\renewcommand{\t}{\psi}
\newcommand{\w}{\omega}
\newcommand{\ihb}{i\hbar}
\newcommand{\Hh}{\hat{H}}
\newcommand{\inner}[1]{\left<#1\right>}
\newcommand{\conj}[1]{\overline{#1}}
\newcommand{\et}[1]{\inner{#1}_{\t}}
\newcommand{\txs}{\abs{\t(x)}^2}
\DeclareMathOperator{\tr}{tr}
\begin{document}
Cavallaro, Jeffery \\
Math 231b \\
Homework \#8

\subsection*{3.2}

Suppose $A$ and $B$ are operators on a finite-dimensional Hilbert space and
suppose that $AB-BA=cI$ for some constant $c$. Show that $c=0$.

Let $\dim H=n$. \\
So $\tr(AB-BA)=\tr(cI)=nc$. \\

$(AB)_{ij}=\sum_{k=1}^na_{ik}b_{kj}$

$(BA)_{ij}=\sum_{k=1}^nb_{ik}a_{kj}$

$(AB-BA)_{ij}=\sum_{k=1}^n(a_{ik}b_{kj}-b_{ik}a_{kj})$

\begin{eqnarray*}
  \tr(AB-BA) &=& \sum_{i=1}^n(AB-BA)_{ii} \\
  &=& \sum_{i=1}^n\sum_{k=1}^n(a_{ik}b_{ki}-b_{ik}a_{ki}) \\
  &=& \sum_{i=1}^n\sum_{k=1}^na_{ik}b_{ki}-\sum_{i=1}^n\sum_{k=1}^nb_{ik}a_{ki} \\
  &=& \sum_{i=1}^n\sum_{k=1}^na_{ki}b_{ik}-\sum_{i=1}^n\sum_{k=1}^nb_{ik}a_{ki} \\
  &=& \sum_{i=1}^n\sum_{k=1}^na_{ki}b_{ik}-\sum_{i=1}^n\sum_{k=1}^na_{ki}b_{ik} \\
  &=& 0
\end{eqnarray*}
But $I\ne0$.

$\therefore c=0$

\subsection*{3.3}

If $A$ is a bounded operator on a Hilbert space $H$, then there exists a unique
bounded operator $A^*$ on $H$ satisfying $\inner{A\p,\t}=\inner{\p,A^*\t}$ for
all $\p$ and $\t$ in $H$. The operator $A^*$ is called the \emph{adjoint} of
$A$, and $A$ is called \emph{self-adjoint} if $A^*=A$.
\begin{enumerate}[label=(\alph*)]
\item Show that for any bounded operator $A$ and constant $c\in\C$, we have
  $(cA)*=\conj{c}A^*$, where $\conj{c}$ is the complex conjugate of $c$.

  Note: $\inner{\p,A\t}=\inner{\p,Bt}\implies A=B$
  
  Assume $\p,\t\in H$.
  \begin{eqnarray*}
    \inner{\p,(cA)^*\t} &=& \inner{(cA)\p,\t} \\
    &=& \inner{A(c\p),\t} \\
    &=& \inner{c\p,A^*\t} \\
    &=& c\inner{\p,A^*\t} \\
    &=& \inner{\p,\conj{c}(A^*\t)} \\
    &=& \inner{\p,(\conj{c}A^*)\t)}
  \end{eqnarray*}

  $\therefore(cA)^*=\conj{c}A^*$

\item Show that if $A$ and $B$ are self-adjoint, then the operator
  $\frac{1}{\ihb}[A,B]$ is also self-adjoint.

  Note: $(A+B)^*=A^*+B^*$ and $(AB)^*=B^*A^*$
  \begin{eqnarray*}
    (\ihb[A,B])^* &=& [\ihb(AB-BA)]^* \\
    &=& \conj{\ihb}(AB-BA)^* \\
    &=& -\ihb[(AB)^*-(BA)^*] \\
    &=& -\ihb(B^*A^*-A^*B^*) \\
    &=& -\ihb(BA-AB) \\
    &=& \ihb(AB-BA) \\
    &=& \ihb[A,B]
  \end{eqnarray*}

  Therefore $\ihb[A,B]$ is self-adjoint.
\end{enumerate}

\subsection*{3.5}

Suppose that $\t$ is a unit vector in $L^2(\R)$ such that the functions
$x\t(x)$ and $x^2\t(x)$ also belong to $L^2(\R)$. Show that:
\[\inner{X^2}_{\t}>\left(\inner{X}_{\t}\right)^2\]

Let $a=\et{X}$ and consider the integral:
\[\int_{-\infty}^{\infty}(x-a)^2\txs dx\]
Note that the integrand is positive (for non-constant $x$) and thus so is the
integral.
\begin{eqnarray*}
  \int_{-\infty}^{\infty}(x-a)^2\txs dx &=&
  \int_{-\infty}^{\infty}(x^2-2ax+a^2)\txs dx \\
  &=& \int_{-\infty}^{\infty}x^2\txs dx-2a\int_{-\infty}^{\infty}x\txs dx
  +a^2\int_{-\infty}^{\infty}\txs dx \\
  &=& \et{X^2}-2\et{X}\et{X}+\left(\et{X}\right)^2\cdot1 \\
  &=& \et{X^2}-2\left(\et{X}\right)^2+\left(\et{X}\right)^2 \\
  &=& \et{X^2}-\left(\et{X}\right)^2 \\
  &>& 0
\end{eqnarray*}
$\therefore\et{X^2}>\left(\et{X}\right)^2$

\subsection*{3.6}

Consider the Hamiltonian $\Hh$ for a quantum harmonic oscillator, given by:
\[\Hh=-\frac{\hbar^2}{2m}\frac{d^2}{dx^2}+\frac{k}{2}x^2\]
where $j$ is the spring constant of the oscillator. Show that the function
\[\t_0(x)=e^{-\frac{\sqrt{km}}{2h}x^2}\]
is an eigenvector for $\Hh$ with eigenvalue $\frac{\hbar\w}{2}$ where
$\w=\sqrt{\frac{k}{m}}$ is the classical frequency of the oscillator.

\[\frac{d}{dx}\t_0(x)=\frac{d}{dx}e^{-\frac{\sqrt{km}}{2h}x^2}=
-\frac{\sqrt{km}}{h}xe^{-\frac{\sqrt{km}}{2h}x^2}\]

\[\frac{d^2}{dx^2}\t_0(x)=-\frac{\sqrt{km}}{h}\left(
e^{-\frac{\sqrt{km}}{2h}x^2}-\frac{\sqrt{km}}{h}x^2e^{-\frac{\sqrt{km}}{2h}x^2}
\right)=-\frac{\sqrt{km}}{h}\t_0(x)+\frac{km}{\hbar^2}x^2\t_0(x)\]

\begin{eqnarray*}
  \Hh\t_0(x) &=& \left(-\frac{\hbar^2}{2m}\frac{d^2}{dx^2}+\frac{k}{2}x^2
  \right)\t_0(x) \\
  &=& -\frac{\hbar^2}{2m}\frac{d^2}{dx^2}\t_0(x)+\frac{k}{2}x^2\t_0(x) \\
  &=& -\frac{\hbar^2}{2m}
  \left(-\frac{\sqrt{km}}{h}\t_0(x)+\frac{km}{\hbar^2}x^2\t_0(x)\right)+
  \frac{k}{2}x^2\t_0(x) \\
  &=& \frac{\hbar}{2}\sqrt{\frac{k}{m}}\t_0(x)-\frac{k}{2}x^2\t_0(x)+
  \frac{k}{2}x^2\t_0(x) \\
  &=& \frac{\hbar\w}{2}\t_0(x)
\end{eqnarray*}

\end{document}
