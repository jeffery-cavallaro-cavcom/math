\documentclass[letterpaper,12pt,fleqn]{article}
\usepackage{matharticle}
\usepackage{enumitem}
\pagestyle{plain}
\newcommand{\ve}{\vec{e}}
\newcommand{\vx}{\vec{x}}
\newcommand{\vy}{\vec{y}}
\newcommand{\vo}{\vec{0}}
\newcommand{\norm}[1]{\left\|#1\right\|}
\newcommand{\inner}[1]{\left<#1\right>}
\newcommand{\p}{\varphi}
\renewcommand{\P}{\Phi}
\renewcommand{\l}{\lambda}
\newcommand{\mb}{\mathcal{B}}
\begin{document}
Cavallaro, Jeffery \\
Math 231b \\
Homework \#5

\subsection*{4.12.2}

If $A$ is an operator on a complex Hilbert space $H$ such that $A\vx\perp\vx$
for every $\vx\in H$, show $A\equiv0$.

By assumption: $\forall\,\vx\in H,\inner{A\vx,\vx}=0$.

Let $\p(\vx,\vy)=\inner{A\vx,\vy}$ be a bilinear functional on $H$ with
quadratic form:
\[\P(\vx)=\p(\vx,\vx)=\inner{A\vx,\vx}=0\]
Applying the polarization identity $\forall\,\vx,\vy\in H$:
\[4\p(\vx,\vy)=\P(\vx+\vy)-\P(\vx-\vy)+i\P(\vx+i\vy)-i\P(\vx-i\vy)\]
But by closure: $\vx+\vy,\vx-\vy,\vx+i\vy,\vx-i\vy\in H$. \\
And so $\P(\vx+\vy)=\P(\vx-\vy)=\P(\vx+i\vy)=\P(\vx-i\vy)=0$

Thus $\forall\,\vx,\vy\in H$ it must be the case that:
\[4\p(\vx,\vy)=4\inner{A\vx,\vy}=0\]
or $\inner{A\vx,\vy}=0$.

But this only holds $\forall\,\vy\in H$ if $A\vx=0$. \\
But this only holds $\forall\,\vx\in H$ if $A\equiv0$.

$\therefore A\equiv0$.

\subsection*{4.12.3}

Give an example of a bounded operator $A$ such that $\norm{A^2}\ne\norm{A}^2$.

Let $E=\R^2$ and let $A(u)=A(x,y)=(y,0)$.
Note that this corresponds to the matrix:
\[[A]_e=\begin{bmatrix} 0 & 1 \\ 0 & 0 \end{bmatrix}\]
$A$ is clearly bounded (triangle inequality) and linear (matrix).

$\norm{A}=\sup_{\norm{u}=1}\norm{Au}=\sup_{\norm{u}=1}\abs{y}=1$ and so
$\norm{A}^2=1$

But $A^2=0$ and so $\norm{A^2}=0$

$\therefore\norm{A^2}\ne\norm{A}^2$

Let $E=\R$ and let $A(x)=x+1$.

$\norm{A}=\sup_{\abs{x}=1}\abs{A(x)}=\sup_{\abs{x}=1}\abs{x+1}=1$

$\norm{A}^2=1^2=1$

\subsection*{4.12.6}

Let $(\ve_n)$ be a complete orthonormal sequence in a Hilbert space $H$ and let
$(\l_n)$ be a sequence of scalars.
\begin{enumerate}[label=(\alph*)]
\item Show that there exists a unique (linear) operator $T$ on $H$ such that
  $T\ve_n=\l_n\ve_n$.

  Note that $H$ is either finite dimensional or separable infinite dimensional,
  and so all (linear) operators on $H$ can be represented by (infinite) matrix
  multiplication.

  Assume $S\ve_n=T\ve_n=\l_n\ve_n$.

  $S\ve_n-T\ve_n=(S-T)\ve_n=
  \sum_{i=1}^{\infty}\sum_{j=1}^{\infty}(s_{ij}-t_{ij})e_{n,j}\ve_i=\vo$

  But $\norm{\ve_n}=1$ and thus $\ve_n\ne\vo$. \\
  And so $s_{ij}-t_{ij}=0$, and thus $s_{ij}=t_{ij}$.

  $\therefore S=T$.

\item Show that $T$ is bounded iff $(\l_n)$ is bounded.

  Since $\norm{\ve_n}=1$:
  \[\norm{T\ve_n}=\norm{\l_n\ve_n}=\abs{\l_n}\norm{\ve_n}=\abs{\l_n}\]
  \begin{description}
  \item $\implies$ Assume $T$ is bounded.

    $\exists\,M>0$ such that $\norm{T\ve_n}=\abs{\l_n}\le M\norm{\ve_n}=M$

    Therefore $(\l_n)$ is bounded.

  \item $\impliedby$ Assume $(\l_n)$ is bounded.

    $\exists\,M>0$ such that $\abs{\l_n}\le M$. \\
    $\norm{T\ve_n}=\abs{\l_n}\le M=M\norm{\ve_n}$.

    Therefore $T$ is bounded.
  \end{description}

\item For a bounded sequence $(\l_n)$, find the norm of $T$.

  Since $(\l_n)$ is bounded, $\abs{\l_n}$ has a supremum. \\
  Let $\l=\sup\abs{\l_n}$
  
  Claim: $\norm{T}=\l$

  Since $T\in\mb(H)$:

  $\norm{T\ve_n}\le\norm{T}\norm{\ve_n}=\norm{T}\cdot1=\norm{T}$ \\
  $\norm{T}\ge\norm{T\ve_n}=\norm{\l_n\ve_n}=\abs{\l_n}\norm{\ve_n}=
  \abs{\l_n}\cdot1=\abs{\l_n}$

  $\therefore\norm{T}\ge\l$

  Furthermore:
  \begin{eqnarray*}
    \norm{T} &=& \sup_{\norm{\vx}=1}\norm{T\vx} \\
    &=& \sup_{\norm{\vx}=1}\norm{T\sum_{k=1}^{\infty}x_k\ve_k} \\
    &=& \sup_{\norm{\vx}=1}\norm{\sum_{k=1}^{\infty}x_kT\ve_k} \\
    &=& \sup_{\norm{\vx}=1}\norm{\sum_{k=1}^{\infty}x_k\l_k\ve_k} \\
    &\le& \sup_{\norm{\vx}=1}\norm{\sum_{k=1}^{\infty}\abs{\l_kx_k}\ve_k} \\
    &=& \sup_{\norm{\vx}=1}\norm{\sum_{k=1}^{\infty}\abs{\l_k}\abs{x_k}\ve_k} \\
    &\le& \sup_{\norm{\vx}=1}\norm{\sum_{k=1}^{\infty}\l\abs{x_k}\ve_k} \\
    &=& \l\sup_{\norm{\vx}=1}\norm{\sum_{k=1}^{\infty}\abs{x_k}\ve_k}
  \end{eqnarray*}
  But note that:
  \begin{eqnarray*}
    \norm{\sum_{k=1}^{\infty}x_k\ve_k}^2 &=&
    \inner{\sum_{k=1}^{\infty}x_k\ve_k,\sum_{k=1}^{\infty}x_k\ve_k} \\
    &=& \sum_{k=1}^{\infty}\abs{x_k}^2 \\
    &=& \inner{\sum_{k=1}^{\infty}\abs{x_k}\ve_k,\sum_{k=1}^{\infty}\abs{x_k}\ve_k} \\
    &=& \norm{\sum_{k=1}^{\infty}\abs{x_k}\ve_k}^2
  \end{eqnarray*}
  So taking the absolute value of the components does not change the norm.

  Hence:
  \[\norm{T}\le\l\sup_{\norm{\vx}=1}\norm{\sum_{k=1}^{\infty}\abs{x_k}\ve_k}=
  \l\sup_{\norm{\vx}=1}\norm{\vx}=\l\cdot1=\l\]

  $\therefore\norm{T}=\l$
\end{enumerate}

\subsection*{4.12.8}

Let $T:\R^2\to\R^2$ be defined by $T(x,y)=(x+3y,2x+y)$. Show that $T^*\ne T$.

From matrix theory, we know that:
\[[T]_e=\begin{bmatrix} 1 & 3 \\ 2 & 1 \end{bmatrix}\]
And since $T^*$ is just the conjugate transpose, and in this case, just the
transpose of $T$:
\[[T^*]_e=\begin{bmatrix} 1 & 2 \\ 3 & 1 \end{bmatrix}\]
And thus $T\ne T^*$.

Using the definition of the transpose, we have:
\[\inner{Tu,v}=\inner{u,T^*v}\]
So let $u=(x_1,y_1)$ and $v=(x_2,y_2)$:
\begin{eqnarray*}
  \inner{Tu,v} &=& \inner{(x_1+3y_1,2x_1+y_1),(x_2,y_2)} \\
  &=& x_2(x_1+3y_1)+y_2(2x_1+y_1) \\
  &=& x_1x_2+3y_1x_2+2x_1y_2+y_1y_2 \\
  &=& x_1(x_2+2y_2)+y_1(3x_2+y_2) \\
  &=& \inner{(x_1,y_1),(x_2+2y_2,3x_2+y_2)} \\
  &=& \inner{u,T^*v}
\end{eqnarray*}
And so $T^*(x,y)=(x+2y,3x+y)$ (as expected) and therefore $T\ne T^*$.

\end{document}
