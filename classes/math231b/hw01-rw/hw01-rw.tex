\documentclass[letterpaper,12pt,fleqn]{article}
\usepackage{matharticle}
\pagestyle{plain}
\newcommand{\mc}{\mathcal{C}}
\newcommand{\norm}[1]{\left\|#1\right\|}
\newcommand{\inner}[2]{\left<#1,#2\right>}
\newcommand{\conj}[1]{\overline{#1}}
\newcommand{\F}{\mathbb{F}}
\renewcommand{\a}{\alpha}
\renewcommand{\b}{\beta}
\begin{document}
Cavallaro, Jeffery \\
Math 231b \\
Homework \#1 Rewrite
\subsection*{1.7.45}

Show that $L(f)(x)=\int_0^xf(t)dt$ defines a continuous linear mapping from
$\mc[0,1]$ into itself.

Claim: $L$ is linear.

Assume $f,g\in\mc[0,1]$ and $\a,\b\in\F$:
\[L(\a f+\b g)=\int_0^x(\a f+\b g)=\a\int_0^xf+\b\int_0^xg=\a Lf+\b Lg\]
Therefore, $L$ is linear.

Furthermore, by the FTC, since $f\in\mc[0,1]$, it must be the case that
$Lf\in\mc[0,1]$.

Claim: $L$ is continuous.

Let the norm be the sup norm: $\norm{\cdot}_{\infty}$.

Assume $(f_n)$ is a sequence in $\mc[0,1]$ such that $f_n\to f$ in the norm.

Thus $\norm{f_n(x)-f(x)}\to0$.

Check for pointwise converge and take the $\sup$ later:
\begin{eqnarray*}
  \abs{(Lf_n)(x)-(Lf)(x)} &=& \abs{L(f_n(x)-f(x))} \\
  &=& \abs{\int_0^x(f_n(t)-f(t))dt} \\
  &\le& \int_0^x\abs{f_n(t)-f(t)}dt \\
  &\le& \int_0^x\max_{t\in[0,x]}\abs{f_n(t)-f(t)}dt \\
  &=& \int_0^x\norm{f_n(t)-f(t)}dt \\
  &=& \norm{f_n(x)-f(x)} \\
  &\to& 0
\end{eqnarray*}

So we have $\abs{(Lf_n)(x)-(Lf)(x)}\to0$ and thus:
\[\max_{x\in[0,1]}\abs{(Lf_n)(x)-(Lf)(x)}=\norm{(Lf_n)(x)-(Lf)(x)}\to0\]

Therefore, $L$ is continuous.

\end{document}
