\documentclass[letterpaper,12pt,fleqn]{article}
\usepackage{matharticle}
\pagestyle{plain}
\newcommand{\mc}{\mathcal{C}}
\renewcommand{\mp}{\mathcal{P}}
\renewcommand{\a}{\alpha}
\renewcommand{\b}{\beta}
\renewcommand{\l}{\lambda}
\renewcommand{\O}{\Omega}
\newcommand{\F}{\mathbb{F}}
\newcommand{\norm}[1]{\left\|#1\right\|}
\DeclareMathOperator{\spn}{Span}
\begin{document}
Cavallaro, Jeffery \\
Math 231b \\
Homework \#1

\subsection*{1.7.9}

Prove: $1\le p<q\implies\ell^p$ is a proper subspace of $\ell^q$.

Assume $1\le p<q$. \\
It was previously shown that $\ell^p$ and $\ell^q$ are both vector spaces,
so it sufficies to show that $\ell^p\subset\ell^q$.

Assume $(a_n)\in\ell^p$. \\
By definition: $\sum_{n=1}^{\infty}\abs{a_n}^p<\infty$. \\
By the convergence theorem, it must be the case that $\abs{a_n}^p\to0$, and
thus $\abs{a_n}\to0$. \\
So $\exists\,N>0$ such that $n>N\implies\abs{a_n}<1$. \\
Assume $n>N$. \\
Since $\abs{a_n}<1$ and $1\le p<1$, we have $\abs{a_n}^q<\abs{a_n}^p$. \\
And so by the comparison theorem: $\sum_{n>N}\abs{a_n}^q<\infty$. \\
And then by the tail convergence theorem:
$\sum_{k=1}^{\infty}\abs{a_n}^q<\infty$.

$\therefore (a_n)\in\ell^q$ and $\ell^p\subseteq\ell^q$.

Now, consider $a_n=\left(\frac{1}{n}\right)^{\frac{1}{p}}$. \\
$\sum_{n=1}^{\infty}\left[\left(\frac{1}{n}\right)^{\frac{1}{p}}\right]^p=
\sum_{n=1}^{\infty}\frac{1}{n}$, which diverges (harmonic series). \\
$\sum_{n=1}^{\infty}\left[\left(\frac{1}{n}\right)^{\frac{1}{p}}\right]^q=
\sum_{n=1}^{\infty}\left(\frac{1}{n}\right)^{\frac{q}{p}}$, which converges since
$\frac{q}{p}>1$. \\
Thus, $(a_n)\in\ell^q$ but $(a_n)\notin\ell^p$.

Therefore, $\ell^p$ is a proper subset of $\ell^q$.

\subsection*{1.7.14}

Prove: $\mc(\O)$, $\mc^k(\R^N)$, and $\mc^{\infty}(\R^N)$ are infinite
dimensional, where $\O$ is a open subset of $\R^N$.

Consider $\mp(\O)$, the vector space of all polynomials in $N$ variables over
domain $\O$. \\
Since polynomials are infinitely continuously differentiable, $\mp(\O)$ is a
subspace of $\mc(\O)$, and $\mp(\O)$ with $\O=\R$ is a subspace of both
$\mc^k(\R^N)$ and $\mc^{\infty}(\R^N)$. \\
Claim: $\mp(\O)$ is infinite dimensional, and thus so are the rest.

ABC: $\mp(\O)$ is finite dimensional and let $\dim\mp(\O)=n$. \\
Then every linearly independent set of $n$ elements is a basis for $\mp(\O)$. \\
Let $\{1,x,x^2,\ldots,x^n\}$ be such a basis (i.e., the powers of the other
$n-1$ variables are $0$). \\
Consider $x^{n+1}\in\mp(\O)$. \\
But $x^{n+1}\notin\spn\{1,x,x^2,\ldots,x^n\}\implies$ CONTRADICTION!

Therefore, $\mp(\O)$ is infinite dimensional.

\subsection*{1.7.15}

Denote by $\ell_0$ the space of all infinite sequences of complex numbers
$(z_n)$ such that $z_n=0$ for all but a finite number of indices $n$. Find a
basis for $\ell_0$.

Let $e_n$ be the sequence in $\ell_0$ such that the $n^{th}$ element is $1$ and
all other elements are $0$.

Claim: $E=\{e_n\mid n\in\N\}$ is a basis for $\ell_0$.

Assume $S=\{e_{n_1},e_{n_2},\ldots,e_{n_r}\}$ is a finite subset of $E$. \\
Assume $\sum_{k=1}^r\a_ke_{n_k}=(0)$. \\
Consider the $j^{th}$ element in the sequence:
\[z_j=a_j\cdot1+\sum_{j\ne k}(a_k\cdot0)=0\]
And so $a_j=0$.
This means that all of the $a_k=0$ and so $S$ is a linearly independent set. \\
Thus, every finite subset of $E$ is a linearly independent set.

Therefore, $E$ is a linearly independent set.

Now, assume $(z_n)\in\ell_0$ such that the last non-zero element occurs in the
$r^{th}$ position. \\
Consider $S=\{e_n\mid1\le n\le r\}\subset E$.
\begin{eqnarray*}
  (z_n) &=& (z_1,z_2,\ldots,z_r,0,\ldots) \\
  &=& (z_1,0,\ldots)+(0,z_2,0,\ldots)+\cdots+(0,\ldots,0,z_r,0,\ldots) \\
  &=& \sum_{k=1}^rz_ke_k
\end{eqnarray*}
And so $(z_n)\in\spn S$. \\
Thus, every element in $\ell_0$ is in the span of some finite subset of
$\ell_0$.

Therefore, $E$ spans $\ell_0$.

Therefore, $E$ is a basis for $\ell_0$.

\subsection*{1.7.44}

Consider the space $\mc[a,b]$ with the norm defined as:
\[\norm{f}=\int_a^b\abs{f(t)}dt\]
Is this a Banach space?

Consider the following counterexample for $\mc[0,1]$. Define:

\begin{minipage}{3.5in}
\[f_n(t)=\begin{cases}
1-nt, & 0\le t\le\frac{1}{n} \\
0, & \frac{1}{n}\le t\le1
\end{cases}\]
\end{minipage}
\begin{minipage}{3in}
  \begin{tikzpicture}
    \draw (0,0) -- (4,0);
    \draw (0,0) -- (0,4);
    \node [left] at (0,3) {$1$};
    \node [below left] at (0,0) {$0$};
    \node [below] at (3,0) {$1$};
    \node [below] at (1,0) {$\frac{1}{n}$};
    \draw [line width=1] (0,3) -- (1,0) -- (3,0);
  \end{tikzpicture}
\end{minipage}

Clearly $(f_n)$ is a sequence in $\mc[0,1]$.

Claim: $(f_n)$ is Cauchy.

AWLOG: $n<m$. Since $\norm{f_n}$ is simply the area under the triangle:
\[\norm{f_n-f_m}=\frac{1}{2n}-\frac{1}{2m}\to0\]
Therefore, $(f_n)$ is Cauchy.

Now, define:

\begin{minipage}{3.5in}
\[f(t)=\begin{cases}
1, & t=0 \\
0, & 0<t\le1
\end{cases}\]
\end{minipage}
\begin{minipage}{3in}
  \begin{tikzpicture}
    \draw (0,0) -- (4,0);
    \draw (0,0) -- (0,4);
    \node [left] at (0,3) {$1$};
    \node [below left] at (0,0) {$0$};
    \node [below] at (3,0) {$1$};
    \node [draw,circle,fill,scale=0.5] (a) at (0,3) {};
    \node [draw,circle,scale=0.5] (b) at (0,0) {};
    \node [draw,circle,fill,scale=0.5] (c) at (3,0) {};
    \draw [line width=1] (b) to (c);
  \end{tikzpicture}
\end{minipage}

Claim: $f_n\to f$.

At $t=0$: $f_n=f=1$.

For $t\in(0,1]$: $\norm{f_n-f}=\norm{f_n}=\frac{1}{2n}\to0$

Therefore, $f_n\to f$.

But clearly, $f\notin\mc[0,1]$ and therefore $C[0,1]$ is not complete and thus
is not Banach.

\subsection*{1.7.45}

Show that $L(f)(x)=\int_0^xf(t)dt$ defines a continuous linear mapping from
$\mc[0,1]$ into itself.

Claim: $L$ is linear.

Assume $f,g\in\mc[0,1]$ and $\a,\b\in\F$:
\[L(\a f+\b g)=\int_0^x(\a f+\b g)=\a\int_0^xf+\b\int_0^xg=\a Lf+\b Lg\]
Therefore, $L$ is linear.

Furthermore, by the FTC, since $f\in\mc[a,b]$, it must be the case that
$Lf\in\mc[a,b]$.

Claim: $L$ is continuous.

Assume $(f_n)$ is a sequence in $\mc[a,b]$ such that $f_n\to f$. Note that
$\mc[a,b]$ need not be Banach. \\
Thus $\norm{f_n-f}\to0$. \\
$\norm{Lf_n-Lf}=\norm{L(f_n-f)}=\norm{\int_0^x(f_n-f)}\le\int_0^x\norm{f_n-f}
\to0$ (since $\norm{f_n-f}\to0$).

Therefore, $L$ is continuous.

\subsection*{1.7.46}

Give an example of a linear mapping from a normed space into a normed space
that is not continuous.

Let $E=\mc^{\infty}[0,\pi]$ and let $L=D$ (the differentiation operator) and let
the norm be the uniform convergence norm:
\[\norm{f(x)}=\norm{f(x)}_{\infty}=\sup_{x\in[0,\pi]}\abs{f(x)}\]
$D$ is clearly linear:
\[D(\a f+\b g)=\a Df+\b Dg\]
Furthermore, by definition, $Df\in\mc^{\infty}[0,\pi]$.

Claim: $D$ is not continuous.

Consider the counterexample:
\[f_n(x)=\frac{\sin(nx)}{n}\]
Note that $\norm{f_n(x)}=\frac{1}{n}\to0$, which occurs at
$x=\frac{\pi}{2n}\in[0,\pi]$. \\
And thus $f_n$ converges to the zero function. \\
$\norm{Df_n-Df}=\norm{Df_n}=\norm{\frac{n\cos(nx)}{n}}=\norm{\cos(nx)}=1$,
which occurs at $x=0\in[0,\pi]$.

Therefore, $D$ is not continuous.

\subsection*{1.7.52}

Let $E=\mc^{\infty}[a,b]$ be the space of infinitly differentiable functions on
the interval $[a,b]$ with $\norm{f}=\max_{x\in[a,b]}\abs{f(x)}$. Is the
differential operator $D=\frac{d}{dx}$ a contraction mapping?

Claim: $D$ is not a contraction mapping.

Let $E=\mc^{\infty}[0,\pi]$, $f(x)=\sin{x}$, and $g(x)=\cos{x}$.

$\norm{\sin{x}-\cos{x}}=\norm{\sqrt{2}\sin(x-\frac{\pi}{4})}=\sqrt{2}$,
which occurs at $x=\frac{3\pi}{4}\in[0,\pi]$.

$\norm{D\sin{x}-D\cos{x}}=\norm{\cos{x}+\sin{x}}=
\norm{\sqrt{2}\sin(x+\frac{\pi}{4})}=\sqrt{2}$,
which occurs at $x=\frac{\pi}{4}\in[0,\pi]$.

And so there is no $\l\in(0,1)$ such that
$\norm{D\sin{x}-D\cos{x}}\le\l\norm{\sin{x}-\cos{x}}$.

Therefore, $D$ is not a contraction mapping.

\end{document}
