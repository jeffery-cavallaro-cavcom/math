\documentclass[letterpaper,12pt,fleqn]{article}
\usepackage{matharticle}
\pagestyle{plain}
\newcommand{\mc}{\mathcal{C}}
\newcommand{\inner}[1]{\left<#1\right>}
\newcommand{\conj}[1]{\overline{#1}}
\newcommand{\inlt}{\overset{L_2}{\longrightarrow}}
\newcommand{\norm}[1]{\left\|#1\right\|}
\DeclareMathOperator{\sgn}{sgn}
\begin{document}
Cavallaro, Jeffery \\
Math 231b \\
Homework \#2 Rewrite

\subsection*{3.8.4}

b. Counterexample

Let $F[a,b]=\{f\in\mc^1[a,b]\mid f(a)=0\}$ with inner product:
\[\inner{f,g}=\int_a^bf'(x)\conj{g'(x)}dx\]
It was proven in the original submission that $F$ is an inner product space.
The following new (proper) counterexample shows that $F$ is not a Hilbert
space.

Consider the interval $[-1,1]$ and let:

\begin{minipage}{3in}
  $g_n(t)=\begin{cases}
  -1, & -1\le t\le-\frac{1}{n} \\
  nt, & -\frac{1}{n}\le t\le\frac{1}{n} \\
  1, & \frac{1}{n}\le t\le1
  \end{cases}$
\end{minipage}
\begin{minipage}{3in}
  \begin{tikzpicture}
    \draw (-3,0) -- (3,0);
    \draw (0,-3) -- (0,3);
    \node [right] at (0,-2) {$-1$};
    \node [left] at (0,2) {$1$};
    \node [above] at (-1,0) {$-\frac{1}{n}$};
    \node [below] at (1,0) {$\frac{1}{n}$};
    \node [above] at (-2.5,0) {$-1$};
    \node [below] at (2.5,0) {$1$};
    \draw [line width=1mm] (-2.5,-2) -- (-1,-2) -- (1,2) -- (2.5,2);
    \draw [dashed] (-1,0) -- (-1,-2);
    \draw [dashed] (1,0) -- (1,2);
    \draw [dashed] (-2.5,0) -- (-2.5,-2);
    \draw [dashed] (2.5,0) -- (2.5,2);
    \draw [dashed] (-1,-2) -- (0,-2);
    \draw [dashed] (1,2) -- (0,2);
  \end{tikzpicture}
\end{minipage}

Let $f_n(x)=\int_0^xg_n(t)dt$.

Note that $g_n(t)\in\mc[-1,1]$, and so, by the FTC:
\begin{enumerate}
  \item $f_n'(x)=g_n(x)$
  \item $f_n(x)\in\mc'[-1,1]$
\end{enumerate}

Also, $f_n(0)=0$, so we can conclude that $f_n(x)\in F[-1,1]$.

Now, consider the standard signum function:
\[\sgn(x)=\begin{cases}
-1, & x<0 \\
0, & x=0 \\
1, & x>0
\end{cases}\]

Claim: $g_n\inlt\sgn$

Note that $g_n(0)=\sgn(0)=0$ and we can ignore this single point (since we are
integrating):
\begin{eqnarray*}
  \norm{g_n-\sgn}_{L_2} &=& \int_{-1}^1\abs{g_n(t)-\sgn(t)}^2dt \\
  &=& \int_{-\frac{1}{n}}^{-\frac{1}{n}}[nt-\sgn(t)]^2dt \\
  &=& \int_{-\frac{1}{n}}^0(1+nt)^2dt+\int_0^{\frac{1}{n}}(1-nt)^2dt \\
  &=& \int_{-\frac{1}{n}}^0(1+2nt+n^2t^2)dt+
  \int_0^{\frac{1}{n}}(1-2nt+n^2t^2)dt \\
  &=& \left[t+nt^2+\frac{n^2}{3}t^3\right]_{-\frac{1}{n}}^0+
  \left[t-nt^2+\frac{n^2}{3}t^3\right]_0^{\frac{1}{n}} \\
  &=& \left[0-\left(-\frac{1}{n}+\frac{1}{n}-\frac{1}{3n}\right)\right]+
  \left[\left(\frac{1}{n}-\frac{1}{n}+\frac{1}{3n}\right)-0\right] \\
  &=& \frac{2}{3n} \\
  &\to& 0
\end{eqnarray*}

Claim: $f_n$ is Cauchy in $\norm{\cdot}_F$.
\begin{eqnarray*}
  \norm{f_n-f_m}_F &=& \norm{f_n'-f_m'}_{L_2} \\
  &=& \norm{g_n-g_m}_{L_2} \\
  &=& \norm{(g_n-\sgn)+(\sgn-g_m)}_{L_2} \\
  &\le& \norm{g_n-\sgn}_{L_2}+\norm{\sgn-g_m}_{L_2} \\
  &\to& 0+0 \\
  &=& 0
\end{eqnarray*}

By geometry, it is clear that:
\[f_n(x)=\begin{cases}
-\left(x-\frac{1}{2n}\right), & x\le0 \\
x-\frac{1}{2n}, & x\ge0
\end{cases}\]

And so $f_n\to f$, where
$f(x)=\begin{cases} -x, & x\le0 \\ x, & x\ge0\end{cases}=\abs{x}$ \\
But $\abs{x}\ne F[-1,1]$ because $\abs{x}$ is not differentiable at $0$.

Therefore $F$ is not complete, and thus not Hilbert.
\end{document}
