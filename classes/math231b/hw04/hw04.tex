\documentclass[letterpaper,12pt,fleqn]{article}
\usepackage{matharticle}
\usepackage{enumitem}
\pagestyle{plain}
\newcommand{\ve}{\vec{e}}
\newcommand{\vx}{\vec{x}}
\newcommand{\vy}{\vec{y}}
\newcommand{\vz}{\vec{z}}
\newcommand{\Sp}{S^{\perp}}
\DeclareMathOperator{\spn}{span}
\newcommand{\Snp}{(\spn{S})^{\perp}}
\newcommand{\inner}[1]{\left<#1\right>}
\newcommand{\norm}[1]{\left\|#1\right\|}
\renewcommand{\a}{\alpha}
\begin{document}
Cavallaro, Jeffery \\
Math 231b \\
Homework \#4

\subsection*{3.8.47}

What is the orthogonal complement in $L^2(\R)$ of the set of all even
functions?

Every function $f\in L^2(\R)$ can be written as a combination of an even
function and an odd function: $f=f_e+f_o$. Furthermore, the integral of an even
function over all of $\R$ is not necessarily $0$, but the integral of an odd
function over all of $\R$ is $0$.

So assume $f\in L^2(\R)$ is an even function and assume $g\in L^2(\R)$:
\[\int_{\R}fg=\int_{\R}f(g_e+g_o)=\int_{\R}fg_e+\int_{\R}fg_o=\int_{\R}fg_e\]
because an even function times an odd function is odd. But this will only be
$0$ for any even $f$ if $g_e=0$ and thus $g$ must be odd.

Therefore, the orthogonal complement to the set of even functions in
$L^2(\R)$ is the set of odd functions in $L^2(\R)$.

\subsection*{3.8.50}

Let $S$ be a subset of an inner product space. Show: $\Sp=\Snp$.

\begin{description}
\item $\subseteq$ Assume $\vx\in\Sp$.

  $\forall\,\vy\in S,\vx\perp\vy$. \\
  Assume $\vy\in\spn(S)$. \\
  $\exists\,\{\vy_1,\ldots,\vy_n\}\subseteq S$ and scalars $\a_1,\ldots,\a_n$
  such that $\vy=\sum_{k=1}^n\a_k\vy_k$. \\
  $\inner{\vy,\vx}=\inner{\sum_{k=1}^n\a_k\vy_k,\vx}=
  \sum_{k=1}^n\a_k\inner{\vy_k,\vx}=0$, since $\vx\perp\vy_k$.

  Therefore $\vx\perp\vy$ and thus $\vx\in\Snp$.

\item $\supseteq$ Assume $\vx\in\Snp$.

  $\forall\,\vy\in\spn(S),\vx\perp\vy$. \\
  But $\forall\,\vy\in S,\vy\in\spn(S)$. \\
  And so $\forall\,\vy\in S,\vx\perp\vy$.

  Therefore $\vx\in\Sp$.
\end{description}

\subsection*{3.8.52}

Let $E$ be the Banach space $\R^2$ with norm
$\norm{(x,y)}=\max(\abs{x},\abs{y})$. Show that $E$ does not have the closest
point property.

Let $S=\{(x,0)\mid x\in[-1,1]\}$, which is a closed and convex subset of $E$.
Also, let $w=(0,1)\in E$:

\begin{minipage}{3.5in}
  \begin{tikzpicture}
    \draw (-4,0) -- (4,0);
    \draw (0,-4) -- (0,4);
    \node [draw,circle,fill,scale=0.5] (a) at (-3,0) {};
    \node [below] at (a) {$(-1,0)$};
    \node [draw,circle,fill,scale=0.5] (b) at (3,0) {};
    \node [below] at (b) {$(1,0)$};
    \node [draw,circle,fill,scale=0.5] (c) at (0,3) {};
    \node [above left] at (c) {$w(0,1)$};
    \draw [line width=1mm] (a) to node [above,pos=0.75] {$S$} (b);
    \node [draw,circle,fill,scale=0.5] (d) at (-1,0) {};
    \node [below] at (d) {$p(x,0)$};
    \draw [dashed] (c) to (d);
  \end{tikzpicture}
\end{minipage}
\begin{minipage}{3in}
  Assume $p=(x,0)\in S$:
  
  \begin{eqnarray*}
    d(w,p) &=& \norm{w-p} \\
    &=& \norm{(0,1)-(x,0)} \\
    &=& \norm{(-x,1)} \\
    &=&\max(\abs{-x},\abs{1}) \\
    &=& 1
  \end{eqnarray*}
  
  Since $-1\le x\le1$.
\end{minipage}

Thus, $\forall\,p\in S,d(w,p)=1$ and so the closest point is not unique.

Therefore, $E$ does not have the closest point property with the given norm.

\subsection*{3.8.53}

Let $S$ be a closed subspace of a Hilbert space $H$ and let
$(\ve_1,\ve_2,\ldots)$ be a complete orthonormal sequence in $S$. For an
arbitrary $\vx\in H$ there exists $\vy\in S$ such that
$\norm{\vx-\vy}=\inf_{\vz\in S}\norm{\vx-\vz}$. Define $\vy$ in terms of
$(\ve_1,\ve_2,\ldots)$.

Assume $\vx\in H$. \\
Thus, $\exists\,\vy\in S,\norm{\vx-\vy}=\inf_{\vz\in S}\norm{\vx-\vz}$. \\
By definition, this means that $d(x,S)=\norm{\vx-\vy}$. \\
By Theorem done is class, we can conclude that $\vx-\vy\perp S$. \\
Thus, $\vx-\vy\perp\ve_n$. \\
Now, since $(\ve_n)$ is complete:
$\vy=\sum_{n=1}^{\infty}\inner{\vy,\ve_n}\ve_n$.

$\inner{\vx,\ve_n}-\inner{\vy,\ve_n}=\inner{\vx-\vy,\ve_n}=0$, since
${\vx-\vy\perp\ve_n}$, and so $\inner{\vx,\ve_n}=\inner{\vy,\ve_n}$.

$\therefore\vy=\sum_{n=1}^{\infty}\inner{\vx,\ve_n}\ve_n$
\end{document}
