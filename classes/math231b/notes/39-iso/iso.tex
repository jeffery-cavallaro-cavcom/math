\documentclass[letterpaper,12pt,fleqn]{article}
\usepackage{matharticle}
\pagestyle{empty}
\newcommand{\vx}{\vec{x}}
\newcommand{\vy}{\vec{y}}
\newcommand{\vo}{\vec{0}}
\renewcommand{\a}{\alpha}
\renewcommand{\b}{\beta}
\newcommand{\inner}[1]{\left<#1\right>}
\newcommand{\conj}[1]{\overline{#1}}
\begin{document}
\section*{Isomorphisms}

\begin{definition}[Isomorphism]
  Let $H_1$ and $H_2$ be inner product (or Hilbert) spaces. To say that $H_1$
  is \emph{isomorphic} to $H_2$ means there exists a mapping $T:H_1\to H_2$,
  called an inner product (or Hilbert) space isomorphism, such that:
  \begin{itemize}
  \item $T$ is a bijection.
  \item $\forall\,\vx,\vy\in H_1,\inner{T\vx,T\vy}=\inner{\vx,\vy}$.
  \end{itemize}
\end{definition}

\begin{theorem}
  Every finite dimensional inner product (and hence Hilbert) space $H$ is
  isomorphic to $\C^N$.
\end{theorem}

\begin{theproof}
  Assume $\dim H=N$. \\
  Assume $\{\vx_1,\ldots,\vx_N\}$ is an orthonormal basis for $H$. \\
  Let $T:H\to\C^N$ be defined by:
  \[T\vx=T\left(\sum_{k=1}^N\a_k\vx_k\right)=\sum_{k=1}^N\a_ke_k=y\]
  Clearly $T$ is bijective.
  
  Assume $\vx,\vy\in H$. \\
  $\exists\,\a,\b\in\C$ such that $\vx=\sum_{k=1}^N\a_k\vx_k$ and
  $\vy=\sum_{k=1}^N\b_k\vx_k$. \\
  $\inner{T\vx,T\vy}=\inner{\sum_{k=1}^N\a_ke_k,\sum_{j=1}^N\b_je_j}=
  \sum_{k=1}^N\a_k\conj{\b_k}$ \\
  Similarly:
  $\inner{\vx,\vy}=\inner{\sum_{k=1}^N\a_k\vx_k,\sum_{j=1}^N\b_j\vx_j}=
  \sum_{k=1}^N\a_k\conj{\b_k}$

  $\therefore\inner{T\vx,T\vy}=\inner{\vx,\vy}$

  $\therefore T$ is an isomorphism and thus $H\sim\C^N$.
\end{theproof}

\begin{theorem}
  Every infinite dimensional separable Hilbert space is isomorphic to $\ell^2$.
\end{theorem}

\begin{theproof}
  Since $H$ is separable, $H$ contains a complete orthonormal sequence
  $(\vx_n)$. \\
  Assume $\vx=\sum_{n=1}^{\infty}\inner{\vx,\vx_n}\vx_n\in H$. \\
  Define $T:H\to\ell^2$ by $T\vx=(\inner{\vx,\vx_n})$, which
  converges (Bessel). \\

  $T$ is linear due to the linearity of the inner product.

  Assume $T\vx=0$. \\
  So $\forall\,n\in\N,\inner{\vx,\vx_n}=0$. \\
  Thus $\vx=\vo$, and so the kernel of linear $T=\{\vo\}$.

  Therefore $T$ is injective.

  Assume $(\a_n)\in\ell^2$. \\
  Let $\vx=\sum_{n=1}^{\infty}\a_n\vx_n$, which converges since
  $(\a_n)\in\ell^2$. \\
  But $(\vx_n)$ is complete, so $\a_n=\inner{\vx,\vx_n}$. \\
  And so $T\vx=(\a_n)$.

  Therefore $T$ is surjective.

  Therefore $T$ is a bijection.

  Finally, assume $\vx,\vy\in H$:
  \begin{eqnarray*}
    \inner{T\vx,T\vy} &=& \inner{(\inner{\vx,\vx_n}),(\inner{\vy,\vx_n})} \\
    &=& \sum_{n=1}^{\infty}\inner{\vx,\vx_n}\conj{\inner{\vy,\vx_n}} \\
    &=& \inner{\sum_{n=1}^{\infty}\inner{\vx,\vx_n}\vx_n,
      \sum_{m=1}^{\infty}\inner{\vy,\vx_m}\vx_m} \\
    &=& \inner{\vx,\vy}
  \end{eqnarray*}

  Therefore $T$ is an isomorphism and thus $H\sim\ell^2$.
\end{theproof}

\end{document}
