\documentclass[letterpaper,12pt,fleqn]{article}
\usepackage{matharticle}
\pagestyle{empty}
\newcommand{\mb}{\mathcal{B}}
\newcommand{\md}{\mathcal{D}}
\newcommand{\mr}{\mathcal{R}}
\newcommand{\vx}{\vec{x}}
\newcommand{\vy}{\vec{y}}
\newcommand{\vo}{\vec{0}}
\newcommand{\F}{\mathbb{F}}
\renewcommand{\a}{\alpha}
\renewcommand{\b}{\beta}
\newcommand{\norm}[1]{\left\|#1\right\|}
\begin{document}
\section*{Invertible Operators}

\begin{definition}[Inverse]
  Let $E$ be a vector space and let $A$ be an operator on some subspace of
  $A$. To say that an operator $B$ on $\mr(A)$ is an \emph{inverse} of $A$
  means:
  \begin{itemize}
  \item $\forall\vx\in\md(A), (BA)\vx=x$
  \item $\forall\vy\in\mr(A), (AB)\vy=y$
  \end{itemize}
  If $B$ exists then $A$ is said to be \emph{invertible} and $B$ can be
  denoted by $B=A^{-1}$.
\end{definition}

\begin{properties}
  Let $E$ be a vector space and let $A$ and $B$ be linear operators on some
  subspace of $E$.
  \begin{enumerate}
  \item $A$ is invertible $\implies A^{-1}$ is unique.
  \item $A$ is invertible $\implies A^{-1}$ linear.
  \item $A$ is invertible $\iff\ker(A)=\{\vo\}$.
  \item $A$ is invertible and $\{\vx_1,\ldots,\vx_n\}$ is a linearly
    independent set $\implies\{A\vx_1,\ldots,A\vx_n\}$ is a linearly
    independent set.
  \item $A$ and $B$ invertible $\implies AB$ is invertible and
    $(AB)^{-1}=B^{-1}A^{-1}$.
  \end{enumerate}
\end{properties}

\begin{theproof}
  \listbreak
  \begin{enumerate}
  \item Assume $A$ is invertible.
    
    Assume $S$ and $T$ are inverses of $A$.

    $S=SI=S(AT)=(SA)T=IT=I$

  \item Assume $A$ is invertible.

    Assume $\vy_1,\vy_2\in\mr(A)$.
    
    $\exists\,\vx_1,\vx_2\in\md(A)$ such that $\vy_1=A\vx_1$ and
    $\vy_2=A\vx_2$.
    
    $A^{-1}\vy_1=\vx_1$ and $A^{-1}\vy_2=\vx_2$.
    
    Assume $\a,\b\in\F$:
    \begin{eqnarray*}
      A^{-1}(\a\vy_1+\b\vy_2) &=& A^{-1}(\a A\vx_1+\b A\vx_2) \\
      &=& A^{-1}A(\a\vx_1+\b\vx_2) \\
      &=& \a\vx_1+\b\vx_2 \\
      &=& (\a A^{-1}\vy_1+\b A^{-1}\vy_2)
    \end{eqnarray*}
    Therefore $A^{-1}$ is linear.

  \item
    \begin{description}
    \item $\implies$ Assume $A$ is invertible.

      $\vx\in\ker(A)\iff A\vx=\vo\iff A^{-1}A\vx=A^{-1}\vo\iff I\vx=\vo\iff
      \vx=\vo$

    \item $\impliedby$ Assume $\ker(A)=\{\vo\}$.

      $A$ is one-to-one. \\
      But $A$ is also onto $\mr(A)$.
      
      Therefore $A$ is bijective and thus invertible.
    \end{description}

  \item Assume $A$ is invertible.

    Assume $\sum_{k=1}^n\a_kA\vx_k=\vo$:
    \begin{eqnarray*}
      A\sum_{k=1}^n\a_k\vx &=& \vo \\
      A^{-1}A\sum_{k=1}^n\a_k\vx &=& A^{-1}\vo \\
      \sum_{k=1}^n\a_k\vx &=& \vo
    \end{eqnarray*}
    But $\{\vx_1,\ldots,\vx_n\}$ is a linearly independent set.

    Therefore $\a_k=0$ and thus $\{A\vx_1,\ldots,A\vx_n\}$ is a linearly
    independent set.

  \item Assume $A$ and $B$ are invertible.

    $(AB)(B^{-1}A^{-1})=A(BB^{-1})A^{-1}=AIA^{-1}=AA^{-1}=I$

    $(B^{-1}A^{-1})(AB)=B^{-1}(A^{-1}A)B=B^{-1}IB=BB^{-1}=I$

    Therefore, since the inverse is unique, $AB$ is invertible and
    $(AB)^{-1}=B^{-1}A^{-1}$.
  \end{enumerate}
\end{theproof}

\begin{corollary}
  Let $E$ be a finite dimensional vector space and let $A$ be a linear
  operator on $E$. If $A$ is invertible then $A$ is onto.
\end{corollary}

\begin{theproof}
  Assume $A$ is invertible. \\
  Assume $\dim(E)=n$. \\
  Assume $\{\vx_1,\ldots,\vx_n\}$ is a basis for $E$. \\
  $\{\vx_1,\ldots,\vx_n\}$ is a linearly independent set. \\
  So $\{A\vx_1,\ldots,A\vx_n\}$ is a linearly independent set. \\
  Thus $\{A\vx_1,\ldots,A\vx_n\}$ is also a basis for $E$. \\
  Now assume $\vy\in E$. \\
  $\exists\,\a_k\in\F$ such that $\vy=\sum_{k=1}^n\a_kA\vx_k=
  A\sum_{k=1}^n\a_kA\vx_k$. \\
  But $\vx=\sum_{k=1}^n\a_kA\vx_k\in \md(A)=E$ and so $A\vx=\vy$.

  Therefore $A$ is onto.
\end{theproof}

\begin{example}
  Let $E=\ell^2$ and $A(z_1,z_2,z_3,\ldots)=(0,z_1,z_2,z_3,\ldots)$.

  $A$ is linear, one-to-one, and invertible; however, $A$ is not onto.
\end{example}

Note that the inverse of a bounded linear operator need not be bounded:

\begin{example}
  Consider $T:\ell^2\to\ell^2$ defined by
  $T(z_1,z_2,z_3,\ldots)=\left(z_1,\frac{z_2}{2},\frac{z_3}{3},\ldots\right)$.

  Claim: $T$ is bounded.

  Assume $z=(z_n)\in\ell^2$:
  \[\norm{Tz}^2=\sum_{k=1}^{\infty}\abs{Tz_k}^2=
  \sum_{k=1}^{\infty}\abs{\frac{z_k}{k}}^2=
  \sum_{k=1}^{\infty}\frac{1}{k^2}\abs{z_k}^2\le
  \sum_{k=1}^{\infty}\abs{z_k}^2=\norm{z}^2\]
  Thus $\norm{Tz}\le\norm{z}$ and so $\norm{T}\le1$ (bounded).

  Claim: $T$ is invertible.

  Let $T^{-1}(z_1,z_2,z_3,\ldots)=(z_1,2z_2,3z_3,\ldots)$. \\
  $T(T^{-1}z)=T^{-1}(Tz)=z$

  Therefore $T$ and $T^{-1}$ are inverses.

  Claim: $T^{-1}$ is unbounded.

  $\norm{T^{-1}e_n}=\norm{ne_n}=n\norm{e_n}=n\cdot1=n$

  So $\norm{T^{-1}e_n}\to\infty$ as $n\to\infty$. \\
  But $\norm{e_n}=1$.

  Therefore $T^{-1}$ is unbounded.
\end{example}

If $E$ is finite dimensional and $A$ on $E$ is invertible, then both $A$ and
$A^{-1}$ are bounded because all operators on finite dimensional vector spaces
are bounded.

\begin{theorem}
  Let $H$ be a Hilbert space and let $A\in\mb(H)$ such that $\exists\,m>0$
  such that $\forall\,\vx\in H, \norm{A\vx}\ge m\norm{\vx}$:

  \qquad$A$ is invertible and $A^{-1}$ is bounded.
\end{theorem}

\begin{theproof}
  $A\vx=\vo\iff\vx=\vo$

  Therefore $A$ is invertible.

  Assume $\vy\in H$. \\
  Since $A^{-1}$ is onto, $\exists\,\vx\in H$ such that $A^{-1}\vy=\vx$. \\
  And so $A\vx=\vy$. \\
  By assumption, $m\norm{\vx}\le\norm{A\vx}$.

  Therefore $\norm{A^{-1}\vy}\le\frac{1}{m}\norm{\vy}$ and thus $A^{-1}$ is
  bounded.
\end{theproof}

The number $m(A)=\inf_{\norm{\vx}=1}\norm{A\vx}$ is called the \emph{conorm} of
$A$. Thus, if $m(A)>0$ then $A$ is invertible and
$\norm{A^{-1}}=\frac{1}{m(A)}$

Note that the previous example failed because $\norm{Te_n}=\frac{1}{n}\to0$.

\begin{theorem}
  Let $H$ be a Hilbert space and let $T\in\mb(H)$ be invertible such that
  $T^{-1}\in\mb(H)$:
  \[(T^{-1})^*=(T^*)^{-1}\]
\end{theorem}

\begin{theproof}
  $(T^{-1})^*T^*=(TT^{-1})^*=I^*=I$

  $T^*(T^{-1})^*=(T^{-1}T)^*=I^*=I$

  Therefore, since inverses are unique, $(T^{-1})^*=(T^*)^{-1}$.
\end{theproof}

\begin{corollary}
  Let $H$ be a Hilbert space and let $T\in\mb(H)$ be invertible such that
  $T$ is self-adjoint and $T^{-1}\in\mb(H)$:
  
  \qquad$T^{-1}$ is self-adjoint.
\end{corollary}

\begin{theproof}
  $(T^{-1})^*=(T^*)^{-1}=T^{-1}$
\end{theproof}

\end{document}
