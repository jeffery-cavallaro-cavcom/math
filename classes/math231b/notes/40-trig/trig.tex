\documentclass[letterpaper,12pt,fleqn]{article}
\usepackage{matharticle}
\pagestyle{empty}
\newcommand{\p}{\varphi}
\newcommand{\norm}[1]{\left\|#1\right\|}
\newcommand{\inner}[1]{\left<#1\right>}
\begin{document}
\section*{Trigonometric Fourier Series}

\begin{theorem}
  The orthonormal sequence $\p_n(t)=\frac{1}{\sqrt{2\pi}}e^{int}$ for $n\in\Z$
  is complete in $L^2[-\pi.\pi]$. Thus, $\forall\,f\in L^2[-\pi.\pi]$, $f$ can
  be written as:
  \[f\sim\sum_{n=-\infty}^{\infty}\inner{f,\p_n}\p_n\]
\end{theorem}

Proving completeness is non-trivial and requires Fej\'{e}r kernels.

Note that this is convergence in the $L^2$ sense:
\[\norm{f-\sum_{n=-N}^N\inner{f,\p_n}\p_n}\to0\]
\[\int_{-\pi}^{\pi}\abs{f(t)-\sum_{n=-N}^N\inner{f,\p_n}e^{int}}^2dt\to0\]
Meaning $S_n\overset{L^2}{\to}f$.

However, $L^2$ convergence does not guarantee pointwise convergence.

\begin{theorem}[Carleson]
  Fourier series of $L^2$ periodic functions converge pointwise a.e.
\end{theorem}

\end{document}
