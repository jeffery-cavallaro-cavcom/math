\documentclass[letterpaper,12pt,fleqn]{article}
\usepackage{matharticle}
\pagestyle{empty}
\newcommand{\mb}{\mathcal{B}}
\newcommand{\mr}{\mathcal{R}}
\newcommand{\norm}[1]{\left\|#1\right\|}
\newcommand{\inner}[1]{\left<#1\right>}
\newcommand{\conj}[1]{\overline{#1}}
\begin{document}
\section*{Unitary Operators}

\begin{definition}[Unitary]
  Let $H$ be a Hilbert space and let $T\in\mb(H)$. To say that $T$ is a
  \emph{unitary} operator means:
  \[T^*T=TT^*=I\]
\end{definition}

\begin{theorem}
  Let $H$ be a Hilbert space and let $T\in\mb(H)$:

  \qquad$T$ is unitary $\iff T$ is invertible and $T^{-1}=T^*$.
\end{theorem}

\begin{theproof}
  $T$ is unitary $\iff T^*T=TT^*=I\iff T$ is invertible and $T^{-1}=T^*$.
\end{theproof}

\begin{properties}
  Let $H$ be a Hilbert space and let $T\in\mb(H)$ such that $T$ is unitary:
  \begin{enumerate}
  \item $\mr(T)=H$
  \item $T$ is isometric.
  \item $T$ is a Hilbert space isomorphism on $H$.
  \end{enumerate}
  In fact, $T$ unitary $\implies T$ isometric; however, $T$ isometric and onto
  $\implies T$ unitary.
\end{properties}

\begin{theorem}
  Let $H$ be a Hilbert space and let $T\in\mb(H)$ such that $T$ is unitary:

  \qquad$T^{-1}$ and $T^*$ are unitary.
\end{theorem}

\begin{theproof}
  Since $T^{-1}=T^*$, it is sufficient to show that $T^*$ is unitary.

  $(T^*)^*T^*=TT^*=I$ and $T^*(T^*)^*=T^*T=I$

  Therefore $T^*$, and hence $T^{-1}$, are unitary.
\end{theproof}

\begin{examples}
  \listbreak
  \begin{enumerate}
  \item $H=\ell^2(\Z)$ (bi-infinite)

    $\ell^2(Z)=\left\{(z_n)_{n\in\Z}\middle|\sum_{-\infty}^{\infty}\abs{z_n}<\infty
    \right\}$

    $\inner{x,y}=\sum_{-\infty}^{\infty}x_n\conj{y_n}$

    Let $S$ be shift right by one position: $S(x_n)=(x_{n-1})$

    $\inner{Sx,y}=\sum_{-\infty}^{\infty}x_{n-1}\conj{y_n}=
    \sum_{-\infty}^{\infty}x_n\conj{y_{n+1}}=\inner{x,S^*y}$

    So $S^*(y_n)=(y_{n+1})$, or a left shift by one position.

    $SS^*=S^*S=I$ and therefore $S$ is unitary.

  \item $H=L^2[0,1]$ and $(Tf)(t)=f(1-t)$
    \begin{eqnarray*}
      \norm{Tf}^2 &=& \int_0^1\abs{(Tf)(t)}^2dt \\
      &=& \int_0^1\abs{f(1-t)}^2dt \\
      &=& \int_1^0\abs{f(s)}^2(-ds) \\
      &=& \int_0^1\abs{f(s)}ds \\
      &=& \norm{f}
    \end{eqnarray*}
    Therefore $T$ is isometric.

    Now, assume $g\in H$. \\
    Let $f\in H$ such that $f(t)=g(1-t)$. \\
    $(Tf)(t)=(Tg)(1-t)=g(1-(1-t))=g(t)$

    Therefore $T$ is onto.

    Therefore $T$ is unitary.
  \end{enumerate}
\end{examples}

\end{document}
