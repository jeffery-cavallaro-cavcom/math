\documentclass[letterpaper,12pt,fleqn]{article}
\usepackage{matharticle}
\newcommand{\F}{\mathbb{F}}
\newcommand{\cl}[1]{\overline{#1}}
\renewcommand{\O}{\Omega}
\DeclareMathOperator{\spn}{Span}
\pagestyle{empty}
\begin{document}
\section*{Bases}

\begin{definition}[Finite Basis]
  Let $E$ be a vector space over a scalar field $\F$ and let $B$ be a
  non-empty, finite subset of $E$. To say that $B$ is a \emph{finite basis} of
  $E$ means:
  \begin{enumerate}
  \item $B$ is a linearly independent set.
  \item $\spn{B}=E$.
  \end{enumerate}
  A vector space with such a finite basis is called a \emph{finite dimensional}
  vector space. If no such finite basis exists then the vector space is called
  an \emph{infinite dimensional} vector space.
\end{definition}

$\R^n$ and $C^n$ are finite-dimensional vector spaces. In fact, all other
finite-dimensional vector spaces are isomorphic to these.

Now, consider a space like $\ell^2$ and the set $S=\{e_k|k\in\N\}$. Since
linear combinations are of a finite number of elements, the span of $S$
consists of sequences with a finite number of non-zero elements. This is
clearly not all of $\ell^2$.

\begin{definition}[Closure]
  Let $E$ be a vector space over a scalar field $\F$ and let $S\subset E$. The
  \emph{closure} of the span of $S$, denoted $\cl{S}$, is the union of the
  $\spn(S)$ and the limits of all partial sums in $\spn(S)$.
\end{definition}

\begin{definition}[Basis]
  Let $E$ be a vector space over a scalar field $\F$ and let $B$ be a non-empty
  subset of $E$. To say that $B$ is a \emph{basis} of $E$ means:
  \begin{enumerate}
  \item $B$ is a linearly independent set.
  \item $\cl{\spn(B)}=E$.
  \end{enumerate}
\end{definition}

Examples of infinite-dimensional vector spaces are: $C^k(\O)$, $P(\O)$, and
$\ell^p$.

\end{document}
