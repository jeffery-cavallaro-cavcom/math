\documentclass[letterpaper,12pt,fleqn]{article}
\usepackage{matharticle}
\pagestyle{empty}
\newcommand{\F}{\mathbb{F}}
\renewcommand{\a}{\alpha}
\renewcommand{\b}{\beta}
\renewcommand{\l}{\lambda}
\renewcommand{\o}{\omega}
\renewcommand{\O}{\Omega}
\newcommand{\vx}{\vec{x}}
\newcommand{\vy}{\vec{y}}
\newcommand{\vz}{\vec{z}}
\newcommand{\vu}{\vec{u}}
\newcommand{\vo}{\vec{0}}
\begin{document}
\section*{Vector Spaces}

\begin{definition}[Vector Space]
  Let $\F$ be a field (usually either $\R$ or $\C$) called \emph{scalars} and
  let $E$ be a set of objects called \emph{vectors} that is equipped with two
  binary operators:

  \begin{figure}[h]
    \setlength{\leftskip}{0.5in}
    \begin{tabular}{lrl}
      Vector Addition: & $+:E\times E\to E$ & where
      $(\vx,\vy)\mapsto \vx+\vy$ \\
      Scalar Multiplication: & $\cdot\,:\F\times E\to E$ & where
      $(\a,\vx)\mapsto \a\vx$
    \end{tabular}
  \end{figure}

  To say that $E$ is a \emph{vector space} (over $\F$) means that the following
  axioms hold $\forall\,\vx,\vy\in E$ and $\forall\,\a,\b\in\F$:
  \begin{enumerate}
  \item $(E,+)$ is an abelian group
    \begin{enumerate}
    \item Closure
    \item Commutative
    \item Associative
    \item $\exists\,\vz,\vx+\vz=\vy$
    \end{enumerate}

  \item Left Distributive: $\a(\vx+\vy)=\a\vx+\a\vy$

  \item Right Distributive: $(\a+\b)\vx=\a\vx+\b\vx$

  \item Associative Multiplication: $\a(\b\vx)=(\a\b)\vx$

  \item Multiplicative Identity: $1\vx=\vx$
  \end{enumerate}
\end{definition}

\begin{theorem}[Additive Identity]
  Let $E$ be a vector space over a scalar field $\F$:
  \[\exists\,\vo\in E,\forall\,\vx\in E,\vx+\vo=\vx\]
  Moreover, $\vo$ is unique.

  This $\vo$ is called the \emph{additive identity} for $E$.
\end{theorem}

\begin{theproof}
  Assume $\vx\in E$. \\
  Since $(E,+)$ is an abelian group, $\exists\,\vo\in E,\vx+\vo=\vx$. \\
  Assume $\vy\in E$. \\
  $\exists\,\vz\in E,\vx+\vz=\vy$ \\
  $\vy+\vo=(\vx+\vz)+\vo=\vx+(\vz+\vo)=\vx+(\vo+\vz)=(\vx+\vo)+\vz=
  \vx+\vz=\vy$

  Now, assume $\vo,\vo\,'\in E$ are both additive identities. \\
  $\vo+\vo\,'=\vo$ \\
  $\vo+\vo\,'=\vo\,'+\vo=\vo\,'$ \\
  $\therefore\vo=\vo\,'$
\end{theproof}

\begin{theorem}
  Let $E$ be a vector space over a scalar field $\F$:
  \[\forall\,\vx,\vy\in E,\exists!\,\vz\in E,\vx+\vz=\vy\]
  (i.e., the $\vz$ is unique)
\end{theorem}

\begin{theproof}
  Assume $\vx,\vy\in E$. \\
  Since $E$ is an abelian group, $\exists\,\vz\in E, \vx+\vz=\vy$. \\
  Assume $\exists\,\vz\,'\in E, \vx+\vz\,'=\vy$. \\
  $\exists\,\vu\in E,\vz\,'+\vu=\vz$ \\
  $\vy=\vx+\vz=\vx+(\vz\,'+\vu)=(\vx+\vz\,')+\vu=\vy+\vu$ \\
  Thus $\vu$ is the unique additive identity and: \\
  $\therefore\vz=\vz\,'+\vu=\vz\,'+\vo=\vz\,'$
\end{theproof}

\begin{theorem}[Additive Inverses]
  Let $E$ be a vector space over a scalar field $\F$: \\
  \[\forall\,\vx\in E,\exists(-\vx)\in E,\vx+(-\vx)=\vz\]
  Moreover, $(-\vx)$ is unique.

  $(-\vx)$ is called the \emph{additive inverse} for $\vx$.
\end{theorem}

\begin{theproof}
  Assume $\vx\in E$. \\
  Since $E$ is an abelian group, $\exists\,(-\vx)\in E,\vx+(-\vx)=\vo$. \\
  By previous theorem, $(-\vx)$ is unique.
\end{theproof}

\begin{notation}
$\vx+(-\vy)=\vx-\vy$
\end{notation}

\begin{theorem}[Vector Space Properties]
  Let $E$ be a vector space over a scalar field $\F$.
  $\forall\,\vx\in E$ and $\forall\,\l\in\F$:
  \begin{enumerate}
  \item $\vo=-\vo$
  \item $0\vx=\vo$
  \item $\l\vo=\vo$
  \item $0-\vx=-\vx$
  \item $(-1)x=-x$
  \item $\l\vx=0\implies\l=0$ or $\vx=\vo$
  \end{enumerate}
\end{theorem}

\begin{theproof}
  \listbreak
  \begin{enumerate}
  \item
    $-\vo=-\vo+\vo=\vo+(-\vo)=\vo$

  \item
    $0\vx=(0+0)\vx=0\vx+0\vx$ \\
    Thus $0\vx$ is the unique additive identity, and: \\
    $\therefore0\vx=\vo$

  \item
    $\l\vo=\l(\vo+\vo)=\l\vo+\l\vo$ \\
    Thus $\l\vo$ is the unique additive identity, and: \\
    $\therefore\l\vo=\vo$

  \item
    $0-\vx=0+(-\vx)=(-\vx)+\vo=(-\vx)$

  \item
    $\vx+(-1)\vx=1\vx+(-1)\vx=[1+(-1)]\vx=0\vx=\vo$ \\
    This $(-1)\vx$ is the unique additive inverse for $\vx$, and: \\
    $\therefore(-1)\vx=(-\vx)$

  \item Assume $\l\vx=\vo$. \\
    Trivial if $\vx=0$, so AWLOG $\vx\ne0$ \\
    ABC: $\l\ne0$
    \begin{eqnarray*}
      \l\vx &=& \vo \\
      \frac{1}{\l}(\l\vx) &=& \frac{1}{\l}\cdot\vo \\
      (\frac{1}{\l}\cdot\l)\vx &=& \vo \\
      1\vx &=& \vo \\
      \vx &=& \vo
    \end{eqnarray*}
    CONTRADICTION! \\
    $\therefore\l=0$
  \end{enumerate}
\end{theproof}

\begin{examples}
  \listbreak
  \begin{enumerate}
  \item $E=\{\vo\}$ is called the trivial vector space.

  \item $E=\R^n$ with $\F=\R$ equipped with component-wise addition and scalar
    multiplication.
  
  \item $E=\C^n$ with $\F=\R$ or $\F=\C$ equipped with component-wise addition
    and scalar multiplication.

  \item Let $E$ be a vector space over a field $F$ and let $X$ be a non-empty
    set. The set of functions:
    \[\mathcal{F}=\{f:X\to E\}\]
    equipped with the standard operations:
    \[(f+g)(x)=f(x)+g(x)\]
    \[(\l f)(x)=\l f(x)\]
    is a vector space.
  \end{enumerate}
\end{examples}

\end{document}
