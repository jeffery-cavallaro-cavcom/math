\documentclass[letterpaper,12pt,fleqn]{article}
\usepackage{matharticle}
\pagestyle{empty}
\newcommand{\ve}{\vec{e}}
\newcommand{\vx}{\vec{x}}
\newcommand{\vy}{\vec{y}}
\newcommand{\vo}{\vec{0}}
\newcommand{\norm}[1]{\left\|#1\right\|}
\newcommand{\inner}[1]{\left<#1\right>}
\newcommand{\conj}[1]{\overline{#1}}
\newcommand{\mn}{\mathcal{N}}
\renewcommand{\a}{\alpha}
\DeclareMathOperator{\proj}{proj}
\begin{document}
\section*{Riesz Representation Theorem}

\begin{lemma}
  Let $H$ be a Hilbert space and let $f\in H'$ such that $f\not\equiv0$:
  \[\dim \mn(f)^{\perp}=1\]
\end{lemma}

\begin{tikzpicture}
  \draw (0,0) -- (5,0) -- (6,2) -- (1,2) -- cycle;
  \node at (1,0.5) {$\mn(f)$};
  \node [draw,circle,fill,scale=0.5] (z) at (3,1) {};
  \node [left] at (z) {$\vo$};
  \draw [->] (3,1) -- node [right] {$\mn(f)^{\perp}$} (3,4);
  \draw [dashed] (3,1) -- (3,0);
  \draw [->] (3,0) -- (3,-2);
\end{tikzpicture}

\begin{theproof}
  Since $f\not\equiv0$, $\mn(f)$ is a proper subset of $H$ and thus
  $\mn(f)^{\perp}$ is not trivial, so fix a $\vx\in\mn(f)^{\perp}$ such that
  $\vx\ne\vo$. \\
  Assume $\vy\in\mn(f)^{\perp}$ such that $\vy\ne\vo$. \\
  Thus $f(\vx),f(\vy)\ne0$; otherwise $\vx,\vy\in\mn(f)$. \\
  So $\exists\,\a\in\C$ such that $f(\vy)=\a f(\vx)$. \\
  $f(\vy)-\a f(\vx)=0$ \\
  $f(\vy-\a\vx)=0$ \\
  Thus $\vy-\a\vx\in\mn(f)$. \\
  But $\mn(f)^{\perp}$ is subspace of $H$, and so $\vy-\a\vx\in\mn(f)^{\perp}$
  also. \\
  Therefore $\vy-\a\vx=0$ and thus $\vy=\a\vx$.
\end{theproof}

\begin{theorem}[Riesz]
  Let $H$ be a Hilbert space and let $f\in H'$. There exists a unique element
  $\vy\in H$ such that $\forall\,\vx\in H$:
  \[f(\vx)=\inner{\vx,\vy}\]
  Moreover, $\norm{f}=\norm{\vy}$.
\end{theorem}

\newpage

\begin{theproof}
  \begin{minipage}{3in}
    \begin{tikzpicture}
      \draw (0,0) -- (5,0) -- (6,2) -- (1,2) -- cycle;
      \node at (1,0.5) {$\mn(f)$};
      \node [draw,circle,fill,scale=0.5] (z) at (3,1) {};
      \node [left] at (z) {$\vo$};
      \draw [->] (3,1) -- node [left] {$\mn(f)^{\perp}$} (3,4);
      \draw [dashed] (3,1) -- (3,0);
      \draw [->] (3,0) -- (3,-2);
      \node [draw,circle,fill,scale=0.5] (x) at (5,3) {};
      \draw [->] (z) to node [left,pos=0.75] {$\vx$} (x);
      \draw [dashed] (3,3) to (x);
      \draw [dashed] (5,1) to (x);
      \draw [->] (z) to node [below] {$\vx-\vx_0$} (5,1);
      \draw [->] (z) to node [right,pos=0.75] {$\vx_0$} (3,3);
      \draw [->] (z) to node [left] {$\ve$} (3,1.75);
    \end{tikzpicture}
  \end{minipage}
  \begin{minipage}{3.5in}
    Let $\ve\in\mn(f)$ such that $\norm{\ve}=1$. \\
    Assume $\vx\in H$. \\
    Let $\vx_0=\proj_{N(f)^{\perp}}\vx$. \\
    By lemma, $\dim N(f)^{\perp}=1$. \\
    And so $\ve$ is an orthonormal basis for $\dim N(f)^{\perp}$. \\
    Thus $\vx_0=\inner{\vx,\ve}\ve$.
  \end{minipage}

  But $\vx-\vx_0\in\mn{f}$ and so $f(\vx-\vx_0)=0$. \\
  $f(\vx)-f(\vx_0)=0$ \\
  $f(\vx)=f(\vx_0)=f(\inner{\vx,\ve}\ve)=\inner{\vx,\ve}f(\ve)=
  \inner{\vx,\conj{f(\ve)}\ve}$

  Therefore $f(x)=\inner{\vx,\vy}$ where $\vy=\conj{f(\ve)}\ve$.

  The fact that $\norm{f}=\norm{y}$ was previously proven.
\end{theproof}

\end{document}
