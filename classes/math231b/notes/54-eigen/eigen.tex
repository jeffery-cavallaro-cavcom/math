\documentclass[letterpaper,12pt,fleqn]{article}
\usepackage{matharticle}
\pagestyle{empty}
\renewcommand{\l}{\lambda}
\newcommand{\vx}{\vec{x}}
\newcommand{\vo}{\vec{0}}
\renewcommand{\a}{\alpha}
\renewcommand{\b}{\beta}
\DeclareMathOperator{\spn}{Span}
\allowdisplaybreaks
\begin{document}
\section*{Eigenvalues and Eigenvectors}

\begin{definition}[Eigenvalue]
  Let $E$ be a complex vector space and let $A$ be an operator on $E$. To say
  that $\l\in\C$ is an \emph{eigenvalue} of $A$ means $\exists\,\vx\in E$ such
  that $\vx\ne\vo$ and:
  \[A\vx=\l\vx\]
  All such $\vx$ are called the \emph{eigenvectors} of $A$ corresponding to
  $\l$.

  When $E$ is a function space the eigenvectors are referred to as
  \emph{eigenfunctions}.

  The \emph{eigenspace} corresponding to $\l$, denoted $E_{\l}$, is given by:
  \[E_{\l}=\{\vx\in H-\{\vo\}\mid A\vx=\l\vx\}\]
\end{definition}

\begin{theorem}
  Let $E$ be a complex vector space and let $A$ be an operator on $E$ with
  an eigenvalue $\l$:

  \qquad$E_{\l}$ is a vector space.
\end{theorem}

\begin{theproof}
  Assume $\vx\in E_{\l}$. \\
  $A\vx=\l\vx$ \\
  $A\vx-\l\vx=\vo$ \\
  $(A-\l I)\vx=\vo$
  
  Therefore $E_{\l}=\ker(A-\l I)$, which is a subspace of $H$.
\end{theproof}

Thus, $\l$ is an eigenvalue of $A$ iff $\ker(A-\l I)$ is nontrivial.

\begin{definition}[Multiplicity]
  Let $E$ be a complex vector space and let $A$ be an operator on $E$ with
  an eigenvalue $\l$. The \emph{multiplicity} of $\l$ is the dimension of
  the corresponding eigenspace $E_{\l}$.

  An eigenvalue with a multiplicity of $1$ is called \emph{simple}.
\end{definition}

\begin{example}
  Let $E=L^2[0,2\pi]$ and let $A$ be an operator on $E$ defined by:
  \[Au=\cos\star u\]
  and so:
  \[(Au)(t)\int_0^{2\pi}\cos(t-x)u(x)dx\]
  First, assume $\l\ne0$:

  $Au=\l u$
  
  $\int_0^{2\pi}\cos(t-x)u(x)dx=\l u(t)$

  $\int_0^{2\pi}[cos(t)cos(x)+sin(t)sin(x)]u(x)dx=\l u(t)$
  
  $\left[\int_0^{2\pi}cos(x)u(x)dx\right]cos(t)+
  \left[\int_0^{2\pi}sin(x)u(x)dx\right]sin(t)=\l u(t)$
  
  And so $u(t)=\a\cos(t)+\b\sin(t)\in\spn\{\cos,\sin\}$ and $\dim E_{\l}=2$.
  
  Let:
  \[a=\int_0^{2\pi}cos(x)u(x)dx\]
  \[b=\int_0^{2\pi}sin(x)u(x)dx\]
  Now solve for $a$ and $b$:
  \begin{eqnarray*}
    a &=& \int_0^{2\pi}cos(x)[\a\cos(x)+\b\sin(x)]dx \\
    &=& \a\int_0^{2\pi}cos^2(x)dx+\b\int_0^{2\pi}\cos(x)\sin(x)dx \\
    &=& \frac{\a}{2}\int_0^{2\pi}[1+cos(2x)]dx+
    \frac{\b}{2}\int_0^{2\pi}\sin(2x)dx \\
    &=& \frac{\a}{2}\int_0^{2\pi}[1+cos(2x)]dx+0 \\
    &=& \frac{\a}{2}\left[\int_0^{2\pi}dx+\int_0^{2\pi}cos(2x)dx\right] \\
    &=& \frac{\a}{2}(2\pi+0) \\
    &=& \a\pi
  \end{eqnarray*}
  \begin{eqnarray*}
    b &=& \int_0^{2\pi}sin(x)[\a\cos(x)+\b\sin(x)]dx \\
    &=& \a\int_0^{2\pi}sin(x)cos(x)dx+\b\int_0^{2\pi}\sin^2(x)dx \\
    &=& \frac{\a}{2}\int_0^{2\pi}\sin(2x)dx+
    \frac{\b}{2}\int_0^{2\pi}[1-\cos(2x)]dx \\
    &=& 0+\frac{\b}{2}\int_0^{2\pi}[1-cos(2x)]dx \\
    &=& \frac{\b}{2}\left[\int_0^{2\pi}dx-\int_0^{2\pi}cos(2x)dx\right] \\
    &=& \frac{\b}{2}(2\pi+0) \\
    &=& \b\pi
  \end{eqnarray*}
  And so:
  \[\a\pi\cos(t)+\b\pi\sin(t)=\l(\a\cos(t)+\b\sin(t)\]
  and:
  $\a\pi=\l\a$ and $\b\pi=\l\b$
  And therefore $\l=\pi$.

  Now assume $\l=0$.

  $a\cos(t)+b\sin(t)=0\iff a=b=0$ and so:
  \[E_0=\{u\in E\mid u\perp\cos$ and $u\perp\sin\}=\{\cos,\sin\}^{\perp}\]
  and $\dim E_0=\infty$.

  And so $L^2[0,2\pi]=E_{\pi}\oplus E_0$.
\end{example}

\end{document}
