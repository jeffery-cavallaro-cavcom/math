\documentclass[letterpaper,12pt,fleqn]{article}
\usepackage{matharticle}
\pagestyle{empty}
\newcommand{\ve}{\vec{e}}
\newcommand{\vx}{\vec{x}}
\newcommand{\vy}{\vec{y}}
\newcommand{\vz}{\vec{z}}
\newcommand{\vo}{\vec{0}}
\newcommand{\norm}[1]{\left\|#1\right\|}
\newcommand{\inner}[1]{\left<#1\right>}
\renewcommand{\a}{\alpha}
\renewcommand{\d}{\delta}
\newcommand{\e}{\epsilon}
\newcommand{\mb}{\mathcal{B}}
\newcommand{\mk}{\mathcal{K}}
\newcommand{\mr}{\mathcal{R}}
\newcommand{\weak}{\overset{w}{\longrightarrow}}
\begin{document}
\section*{Compact Operators}

\begin{definition}
  Let $H$ be a Hilbert space and let $A$ be an operator on $H$. To say that
  $A$ is \emph{compact} (\emph{completely continuous}) means for all bounded
  subsequences $(\vx_n)$, the sequence $(A\vx_n)$ in $H$ has a convergent
  subsequence.
\end{definition}

\begin{examples}
  \listbreak
  \begin{enumerate}
  \item $I$ is not compact.

    Consider $(\ve_n)$ be an orthonormal sequence in $H$.

    $\forall\,n\in\N,\norm{\ve_n}=1$ and so $(\ve_n)$ is bounded.

    But $\forall\,j,k,\norm{e_j-e_k}=\sqrt{2}\not\to0$

    Therefore $(I\ve_n)$ does not have a convergent subsequence.

  \item Let $H$ be finite dimensional and let $T$ be a linear operator on $H$.

    Claim: $T$ is compact.

    $H$ is isomorphic to $\C^n$. \\
    Since $H$ is finite dimensional, $T$ is bounded. \\
    Assume $(\vx_n)$ in $H$ is bounded and let $\norm{\vx_n}\le M$. \\
    $\norm{T\vx_n}\le\norm{T}\norm{\vx_n}\le M\norm{T}<\infty$ \\
    And so $(T\vx_n)$ is a bounded sequence in $\C^n$.

    Therefore, by the Bolzano-Weierstrass theorem, $(T\vx_n)$ has a
    convergent subsequence and thus $T$ is compact.

  \item Let $H$ be a Hilbert space and let $T$ be a linear operator on $H$
    defined by:
    \[T\vx=\inner{\vx,\vy_0}\vz_0\]
    for fixed $\vy_0,\vz_0\in H$.

    Claim: $T$ is compact.

    Assume $(\vx_n)$ in $H$ is bounded. \\
    Let $\norm{\vx_n}\le M$. \\
    $\abs{\inner{\vx_n,\vy_0}}\le\norm{\vx_n}\norm{\vy_0}\le M\norm{\vy_0}$ \\
    Thus $\inner{\vx_n,\vy_0}$ is a bounded sequence in $\C$. \\
    And so by Bolzano-Weierstrass, $\exists\,(\vx_{n_k})$ such that
    $\inner{\vx_{n_k},\vy_0}\to\a\in\C$ \\
    And so $T\vx_{n_k}=\inner{\vx_{n_k},\vy_0}\vz_0\to\a\vz_0$.

    Therefore $T$ is compact.
  \end{enumerate}
\end{examples}

\newpage

\begin{definition}[Equicontinuous]
  Let $f\in L^2[a,b]$. To say that $f$ is \emph{equicontinuous} means:
  \[\forall\,\e>0,\exists\,\d>0,\forall\,n\ge1,\forall\,x,y\in[a,b],
  \abs{x-y}<\d\implies\abs{f(x)-f(y)}<\e\]
\end{definition}

\begin{theorem}[Arzel\'{a}-Ascoli]
  Let $H$ be a Hilbert space and let $(q_n)$ be a sequence in $H$:

  \qquad$(q_n)$ uniformly bounded and equicontinuous $\implies(q_n)$ has a
  convergent subsequence.
\end{theorem}

\begin{theorem}
  Let $H=L^2[a,b]$ and let $T$ be a linear operator on $H$ defined by:
  \[(Tf)(s)=\int_a^bK(s,t)f(t)dt\]
  $K$ continuous $\implies T$ compact.
\end{theorem}

\begin{theproof}
  Assume $K$ is continuous on $Q=[a,b]\times[a,b]$. \\
  Assume $(f_n)$ is bounded.

  Applying H\"{o}lder:
  \begin{eqnarray*}
    \abs{(Tf_n)(s)} &=& \abs{\int_a^bK(s,t)f_n(t)dt} \\
    &\le& \int_a^b\abs{K(s,t)f_n(t)}dt \\
    &=& \int_a^b\abs{K(s,t)}\abs{f_n(t)}dt \\
    &\le& \left(\int_a^b\abs{K(s,t)}^2\right)^{\frac{1}{2}}
    \left(\int_a^b\abs{f_n(t)}^2\right)^{\frac{1}{2}}dt \\
    &=& \left(\int_a^b\abs{K(s,t)}^2\right)^{\frac{1}{2}}\norm{f_n}
  \end{eqnarray*}
  But $K$ is continuous on compact $Q$ and thus is bounded so $K(s,t)\le C$. \\
  Furthermore, $(f_n$) is bounded and so $\norm{f_n}\le M$.
  And so:
  \[\abs{(Tf_n)(s)}\le CM\left(\int_a^bdt\right)=CM\sqrt{b-a}\]
  For all $s\in[a,b]$ and all $n\in\N$.

  Therefore $(Tf_n)$ is uniformly bounded.

  Assume $\e>0$. \\
  Since $Q$ is compact, $Q$ is uniformly continuous on $Q$. \\
  So $\exists\,\d>0,\forall\,x,y\in[a,b],\abs{x-y}<\d\implies
  \abs{K(x,t)-K(y,t)}<\frac{\e}{M\sqrt{b-a}}$. \\
  And so, once again applying H\"{o}lder:
  \begin{eqnarray*}
    \abs{(Tf_n)(x)-(Tf_n)(y)} &=& \abs{\int_a^bK(x,t)f_n(t)dt-
      \int_a^bK(y,t)f_n(t)dt} \\
    &=& \abs{\int_a^b[K(x,t)-K(y,t)]f_n(t)dt} \\
    &\le& \int_a^b\abs{[K(x,t)-K(y,t)]f_n(t)dt} \\
    &=& \int_a^b\abs{K(x,t)-K(y,t)}\abs{f_n(t)}dt \\
    &\le& \left[\int_a^b\abs{K(x,t)-K(y,t)}^2dt\right]^{\frac{1}{2}}
    \left(\int_a^b\abs{f_n(t)}^2dt\right)^{\frac{1}{2}} \\
    &=& \left[\int_a^b\abs{K(x,t)-K(y,t)}^2dt\right]^{\frac{1}{2}}\norm{f_n} \\
    &<& M\left[\int_a^b\left(\frac{\e}{M\sqrt{b=a}}\right)dt
      \right]^{\frac{1}{2}} \\
    &=& \frac{\e}{\sqrt{b-a}}\sqrt{b-a} \\
    &=& \e
  \end{eqnarray*}
  Therefore $T$ is equicontinuous.

  Therefore, by Arzel\'{a}-Ascoli, $(Tf_n)$ has a convergent subsequence
  and thus $T$ is compact.
\end{theproof}

Note that $T$ is compact iff it maps bounded sets to sets with compact
closure:

\qquad$\forall\,B\subset H$ where $B$ is bounded, $\overline{T[B]}$ is compact.

\begin{notation}
  Let $H$ be a Hilbert space. The set of all compact operators on $H$ is
  denoted by $\mk(H)$.
\end{notation}

\begin{theorem}
  Let $H$ be a Hilbert space:
  \[\mk(H)\subset\mb(H)\]
  In other words, all compact operators on $H$ are also bounded.
\end{theorem}

\newpage

\begin{theproof}
  Assume $T$ is compact. \\
  ABC: $T$ is not bounded. \\
  $\forall\,M>0,\exists\,\vx\in H,\norm{\vx}=1,\norm{T\vx}>M$ \\
  Let $(\vx_n)$ in $H$ such that $\norm{\vx_n}=1$. \\
  Let $M=n$. \\
  $\norm{T\vx_n}>n\to\infty$ \\
  And so $(T\vx_n)$ does not have a convergent subsequence. \\
  CONTRADICTION!

  Therefore $T$ is bounded.
\end{theproof}

\begin{theorem}
  Let $H$ be a Hilbert space:
  
  \qquad$\mk(H)$ is an ideal in $\mb(H)$.
\end{theorem}

\begin{theproof}
  Clearly, $\mk(H)$ is a subspace of $\mb(H)$.

  Assume $T\in\mk(H)$. \\
  Assume $(\vx_n)$ is a bounded sequence in $H$. \\
  $(T\vx_n)$ has a convergent subsequence $(T\vx_{n_k})\to\vy$. \\
  Assume $B\in\mb(H)$. \\
  $B(T\vx_n)\to B(\vy)$ and so $BT\in\mk(H)$. \\
  Also, $(\vx_n)$ bounded $\implies(B\vx_n)$ bounded. \\
  Thus $TB(\vx_n)$ has a convergent subsequence and thus $TB\in\mk(H)$.

  Therefore $\mk(H)$ is an ideal of $\mb(H)$.
\end{theproof}

\begin{definition}[Finite Rank]
  Let $H$ be a Hilbert space and let $T$ be an operator on $H$. To sat that $T$
  is a \emph{finite rank} operator means $\dim\mr(T)<\infty$.
\end{definition}

\begin{theorem}
  Finite rank bounded operators are compact.
\end{theorem}

\begin{theproof}
  Assume $H$ is a Hilbert space and assume $T\in\mb(H)$. \\
  Let $\mr(T)=S$ such that $\dim S=n$. \\
  Assume $\{\ve_1,\ldots,\ve_n\}$ is an orthonormal basis for $S$. \\
  Define: $T_n\vx=\inner{T\vx,\ve_n}\ve_n$.
  But $T_n\vx=\inner{\vx,T^*\ve_n}\ve_n$, which is compact by above example.
  Furthermore, $\sum_{k=1}^n\inner{T\vx,\ve_k}\ve_k=T\vx\in S$. \\
  Also $\mr(T)$ is a vector space.

  Therefore $T$ is compact.
\end{theproof}

\begin{theorem}
  Let $H$ be a Hilbert space:

  \qquad$\mk(H)$ is closed in $\mb(H)$.
\end{theorem}

\begin{theproof}
  Assume $(T_n)$ is a sequence in $\mk(H)$ such that $\norm{T_n-T}\to0$. \\
  Assume $(\vx_n)$ is a bounded sequence in $H$ such that
  $\norm{\vx_n}\le M$. \\
  Since $T_1$ is compact, there exists a subsequence $(\vx_{1_n})$ of $(\vx_n)$
  such that $(T_1\vx_{1_n})$ converges. \\
  Similarly, the sequence $(T_2\vx_{1_n})$ contains a convergent subsequence
  $(T_2\vx_{2_n})$. \\
  In general, for $k\ge2$, let $(\vx_{k_n})$ be a subsequence of
  $(\vx_{(k-1)_n})$ such that $(T_k\vx_{k_n})$ converges. \\
  Let $(\vx_{n_n})=(\vx_{p_n})$, where $(p_n)$ is increasing positive. \\
  $\forall\,k\in\N,(T_k\vx_{p_n})$ converges. \\
  Assume $\e>0$. \\
  $\exists\,N>0,k>N\implies\norm{T_n-T}<\frac{\e}{3}$ \\
  Assume $n,m>N$.
  \begin{eqnarray*}
    \norm{T\vx_{p_n}-T\vx_{p_m}} &=&
    \norm{T\vx_{p_n}-T_k\vx_{p_n}+T_k\vx_{p_n}-T_k\vx_{p_m}+T_k\vx_{p_m}
      -T\vx_{p_m}} \\
    &\le& \norm{T\vx_{p_n}-T_k\vx_{p_n}}+\norm{T_k\vx_{p_n}-T_k\vx_{p_m}}+
    \norm{T_k\vx_{p_m}-T\vx_{p_m}} \\
    &<& \frac{\e}{3}+\frac{\e}{3}+\frac{\e}{3} \\
    &=& \e
  \end{eqnarray*}
  Thus $(T\vx_{p_n})$ is Cauchy, and by completeness of $H$, converges.

  Therefore $T$ is compact and $T\in\mk(H)$, and thus $\mk(H)$ is closed.
\end{theproof}

\begin{theorem}
  The adjoint of a compact operator is compact.
\end{theorem}

\begin{theproof}
  Assume $H$ is a Hilbert space. \\
  Assume $T\in\mk(H)$, and thus $T\in\mb(H)$. \\
  Hence $T^*\in\mb(H)$. \\
  Assume $(\vx_n)$ is a bounded sequence in $H$ such that
  $\norm{\vx_n}\le M$. \\
  Let $\vy_n=T^*\vx_n$. \\
  Since $T^*$ is bounded, $\vy_n$ is bounded. \\
  Thus $T\vy_n$ has a convergent subsequence $(T\vy_{n_k})$.
  \begin{eqnarray*}
    \norm{\vy_m-\vy_n}^2 &=& \inner{\vy_m-\vy_n,\vy_m-\vy_n} \\
    &=& \inner{T^*\vx_m-T^*\vx_n,T^*\vx_m-T^*\vx_n} \\
    &=& \inner{T^*(\vx_m-\vx_n),T^*(\vx_m-\vx_n)} \\
    &=& \inner{TT^*(\vx_m-\vx_n),\vx_m-\vx_n} \\
    &\le& \norm{TT^*(\vx_m-\vx_n)}\norm{\vx_m-\vx_n} \\
    &\le& \norm{T(T^*\vx_m-T^*\vx_n)}(\norm{\vx_m}+\norm{\vx_n}) \\
    &=& \norm{T(\vy_m-\vy_n)}(M+M) \\
    &=& 2M\norm{T\vy_m-T\vy_n}
  \end{eqnarray*}
  Now, apply this to subsequences:
  \[\norm{\vy_{n_j}-\vy_{n_k}}\le2M\norm{T\vy_{n_j}-T\vy_{n_k}}\to0\]
  Therefore $(\vy_{n_k})=(T^*\vx_{n_k})$ converges and thus $T^*\in\mk(H)$.
\end{theproof}

\begin{theorem}
  $T$ is compact iff $T$ maps weakly-convergent sequences to strongly-
  convergent sequences.
\end{theorem}

\begin{theproof}
  Assume $H$ is a Hilbert space. \\
  Assume $T$ is an operator on $H$.
  \begin{description}
  \item $\implies$ Assume $T$ is compact.

    Assume $(\vx_n)$ is a sequence in $H$ such that $\vx_n\weak\vx$. \\
    ABC: $T\vx_n\not\to T\vx$. \\
    $\exists\,\e>0,\forall\,N>0,\exists\,n>N,\norm{T\vx_n-T\vx}\ge\e$ \\
    Similarly, $\norm{T\vx_{p_n}-T\vx}\ge\e$.

    Since $\vx_n\weak\vx$, $(\vx_n)$ is bounded. \\
    And since $T$ is compact, $(T\vx)$ has a convergent subsequence
    $(T\vx_{p_n})$.

    Assume $\vy\in H$. \\
    $\inner{T\vx_n,\vy}=\inner{\vx_n,T^*\vy}\to\inner{\vx,T^*\vy}=
    \inner{T\vx,\vy}$ \\
    Thus $T\vx_n\weak T\vx$. \\
    Likewise, $T\vx_{p_n}\weak T\vx$. \\
    But $\vx_n\to\vx\implies\vx_n\weak\vx$, and so $T\vx_{p_n}\to T\vx$. \\
    CONTRADICTION!

    Therefore $T\vx_n\to T\vx$.

  \item $\impliedby$ Assume $\vx_n\weak\vx\implies T\vx_n\to T\vx$.

  \end{description}
\end{theproof}

\begin{corollary}
  Compact operators map orthonormal sequences into sequences that strongly
  converge to 0.
\end{corollary}

\begin{theproof}
  Assume $H$ is a Hilbert space. \\
  Assume $T\in\mk(H)$. \\
  Assume $(\ve_n)$ is an orthonormal sequence in $H$. \\
  $\ve\weak\vo$ \\
  Therefore $T\ve\to0$.
\end{theproof}

\begin{corollary}
  Let $H$ be a Hilbert space and $T\in\mk(H)$ such that $T$ is invertible:

  \qquad$T^{-1}$ is unbounded.
\end{corollary}

\begin{theproof}
  Assume $H$ is a Hilbert space. \\
  Assume $T\in\mk(H)$. \\
  Assume $(\ve_n)$ is an orthonormal sequence in $H$. \\
  $T\ve_n\to\vo$ \\
  $T^{-1}(T\ve_n)=\ve_n\not\to0$.

  Therefore $T^{-1}$ is discontinuous, and thus unbounded.
\end{theproof}

\end{document}
