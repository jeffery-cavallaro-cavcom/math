\documentclass[letterpaper,12pt,fleqn]{article}
\usepackage{matharticle}
\pagestyle{empty}
\renewcommand{\l}{\lambda}
\renewcommand{\o}{\sigma}
\newcommand{\p}{\rho}
\newcommand{\mb}{\mathcal{B}}
\newcommand{\norm}[1]{\left\|#1\right\|}
\begin{document}
\section*{Resolvent and Spectrum}

\begin{definition}[Resolvent and Spectrum]
  Let $E$ be a normed space and let $A$ be an operator on $E$. The
  \emph{resolvent} of $A$, denoted $A_{\l}$, is given by:
  \[A_{\l}=(A-\l I)^{-1}\]
  Note that $\l$ is an eigenvalue of $A$ iff $A_{\l}$ is not defined.

  To say that $\l$ is a \emph{regular value} of $A$ means $A_{\l}\in\mb(E)$.

  The \emph{resolvent set} of $A$, denoted $\p(A)$, is the set of all regular
  values of $A$.

  The \emph{spectrum} of $A$, denoted $\o(A)$, is given by:
  \[\o(A)=\C\setminus\p(A)\]

  The \emph{spectral radius} of $A$, denoted $r(A)$, is given by:
  \[r(A)=\sup\{\abs{\l}\mid\l\in\o(A)\}\]
\end{definition}

If $\l$ is an eigenvalue of $A$ then $\l\in\o(A)$. But note that the spectrum
can contain non-eigenvalues and no eigenvalues.

\begin{lemma}
  Let $E$ be a Banach space and let $A\in\mb(E)$ such that $\norm{A}<1$:
  \begin{enumerate}
  \item $I-A$ is invertible.
  \item $(I-A)^{-1}=\sum_{n=0}^{\infty}A^n$
  \end{enumerate}
\end{lemma}

\begin{theproof}
  Let $S_n=\sum_{k=0}^nA^k$. \\
  AWLOG: $n>m$.
  \[\norm{S_n-S_m}=\norm{\sum_{k=m+1}^nA^k}\le\sum_{k=m+1}^n\norm{A^k}=
  \sum_{k=m+1}^n\norm{A}^k\to0\]
  And so $(S_n)$ is Cauchy. \\
  But $E$ Banach $\implies\mb(E)$ Banach. \\
  And $A\in\mb(E)\implies S_n\in\mb(E)$. \\
  Thus $(S_n)$ converges to $S\in\mb(E)$ where $S=\sum_{n=0}^{\infty}A^n$.

  $S(I-A)=\left(\sum_{n=0}^{\infty}A^n\right)(I-A)=
  \sum_{n=0}^{\infty}A^n-\sum_{n=0}^{\infty}A^{n+1}=A^0=I$

  $(I-A)S=(I-A)\left(\sum_{n=0}^{\infty}A^n\right)=
  \sum_{n=0}^{\infty}A^n-\sum_{n=0}^{\infty}A^{n+1}=A^0=I$

  $\therefore S=(I-A)^{-1}=\sum_{n=0}^{\infty}A^n$
\end{theproof}

\begin{corollary}
  Let $E$ be a Banach space and $T\in\mb(E)$ such that $\norm{I-T}<1$:
  \begin{enumerate}
  \item $T$ is invertible.
  \item $T^{-1}=\sum_{n=0}^{\infty}(I-T)^n$
  \end{enumerate}
\end{corollary}

\begin{theproof}
  $T^{-1}=[I-(I-T)]^{-1}=\sum_{n=0}^{\infty}(I-T)^n$
\end{theproof}

\begin{theorem}
  Let $E$ be a Banach space and let $A\in\mb(E)$ and $\norm{A}\le\abs{\l}$:
  \begin{enumerate}
  \item $A_{\l}=-\sum_{n=0}^{\infty}\frac{1}{\l^{n+1}}A^n$
  \item $\norm{A_{\l}}\le\frac{1}{\abs{\l}-\norm{A}}$
  \end{enumerate}
  Thus, $A_{\l}\in\mb(E)$.
\end{theorem}

\begin{theproof}
  Let $B=\frac{A}{\l}$. \\
  $\norm{B}=\frac{\norm{A}}{\abs{\l}}<1$ \\
  $\sum_{n=0}^{\infty}B^n=(I-B)^{-1}$ \\
  $\sum_{n=1}^{\infty}\left(\frac{A}{\l}\right)^n=
  \left(I-\frac{A}{\l}\right)^{-1}=-\frac{1}{\l}(A-\l I)^{-1}$
  
  $\therefore A_{\l}=(A-\l I)^{-1}=-\sum_{n=0}^{\infty}\frac{1}{\l^{n+1}}A^n$

  $\norm{A_{\l}}=\norm{-\sum_{n=0}^{\infty}\frac{1}{\l^{n+1}}A^n}\le
  \sum_{n=0}^{\infty}\norm{\frac{1}{\l^{n+1}}A^n}=
  \frac{1}{\abs{\l}}\sum_{n=1}^{\infty}\left(\frac{\norm{A}}{\abs{\l}}\right)^n=
  \frac{1}{\abs{\l}}\left(\frac{1}{1-\frac{\norm{A}}{\abs{\l}}}\right)$

  $\therefore\norm{A_{\l}}\le\frac{1}{\abs{\l}-\norm{A}}$
\end{theproof}

Thus, if $\l$ is an eigenvalue of $A$, then $\abs{\l}<\norm{A}$ and
$r(A)\le\norm{A}$ and $\o(A)$ is an open set.

\begin{figure}[h]
  \setlength{\leftskip}{1in}
  \begin{tikzpicture}
    \draw [dashed] (0,0) circle [radius=2];
    \draw [dashed] (0,0) -- node [above] {$\norm{A}$} (2,0);
    \node at (0,-1) {$\o(A)$};
    \node at (2,2) {$\p(A)$};
    \node at (-3/4,3/4) {$\l$};
  \end{tikzpicture}
\end{figure}

\end{document}
