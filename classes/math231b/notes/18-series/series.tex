\documentclass[letterpaper,12pt,fleqn]{article}
\usepackage{matharticle}
\pagestyle{empty}
\newcommand{\norm}[1]{\left\|#1\right\|}
\newcommand{\vx}{\vec{x}}
\renewcommand{\mp}{\mathcal{P}}
\begin{document}
\section*{Series in a Normed Space}

\begin{definition}[Infinite Series]
  Let $E$ be a normed space and let $S=\sum_{n=1}^{\infty}\vx_n$ be an
  \emph{infinite series} in $E$. To say that $S$ \emph{converges} in $E$ means
  the sequence of partial sums $(S_N)$ where $S_N=\sum_{n=1}^N\vx_n$ converges
  in the norm to some value $\vx\in E$:
  \[\norm{S_n-S}=\norm{\sum_{n=1}^N\vx_n-\vx}\to0\]

  To say that $S$ converges \emph{absolutely} in $E$ means:
  \[\sum_{n=1}^{\infty}\norm{\vx_n}<\infty\]
\end{definition}

\begin{examples}
  \listbreak
  \begin{enumerate}
  \item $E=\R$ and $\sum_{n=1}^{\infty}\frac{(-1)^n}{n}$ converges, but not
    absolutely.

    $\sum_{n=1}^{\infty}\frac{(-1)^n}{n}=\ln{2}$

    $\sum_{n=1}^{\infty}\abs{\frac{(-1)^n}{n}}=
    \sum_{n=1}^{\infty}\frac{1}{n}=\infty$

  \item $E=\mp[0,1]$ and $f_n(t)=\sum_{n=1}^{\infty}\frac{t^n}{n!}$
    converges absolutely but does not converge in the norm.

    $f_n\to f=e^t\notin\mp[0,1]$

    $\sum_{n=1}^{\infty}\norm{\frac{t^n}{n!}}=
    \sum_{n=1}^{\infty}\max_{t\in[0,1]}\abs{\frac{t^n}{n!}}=
    \sum_{n=1}^{\infty}\frac{1}{n!}=e<\infty$
  \end{enumerate}
\end{examples}

\begin{theorem}
  Let $E$ be a normed space. $E$ is Banach iff every ACV series in $E$
  converges in $E$.
\end{theorem}

\begin{theproof}
  \listbreak
  \begin{description}
  \item $\implies$ Assume $E$ is Banach.

    Assume $\sum_{n=1}^{\infty}\vx_n$ is ACV in $E$. \\
    Thus, $\sum_{n=1}^{\infty}\norm{\vx_n}<\infty$. \\
    Let $S_N=\sum_{n=1}^N\norm{\vx_n}$. \\
    Let $s_N=\sum_{n=1}^N\vx_n$. \\
    AWLOG: $N<M$. \\
    $\norm{s_M-s_N}=\norm{\sum_{n=1}^M\vx_n-\sum_{n=1}^N\vx_n}=
    \norm{\sum_{n=N+1}^M\vx_n}\le\sum_{n=N+1}^M\norm{\vx_n}=S_M-S_N\to0$

    Therefore, $(s_n)$ is Cauchy in $E$. \\
    But $E$ is complete, therefore $(s_n)$ converges in $E$.

  \item $\impliedby$ Assume every ACV series in $E$ converges in $E$.

    Assume $(\vx_n)$ is Cauchy in $E$. \\
    $\forall\,k\in\N,\exists\,N_k>0,m,n>N_k\implies\norm{\vx_n-\vx_m}<
    \frac{1}{2^k}$ \\
    Let $(n_k)$ be a strictly increasing sequence in $\N$. \\
    Thus, for all $n_k>N_k$:
    \[\sum_{k=1}^{\infty}\norm{\vx_{n_{k+1}}-\vx_{n_k}}\le
    \sum_{k=1}^{\infty}\left(\frac{1}{2}\right)^k<\infty\]
    Thus $\sum_{k=1}^{\infty}(\vx_{n_{k+1}}-\vx_{n_k})$ is ACV, and by assumption,
    converges to some element $\vx\in E$.

    Let $S_N=\sum_{k=1}^N(\vx_{n_{k+1}}-\vx_{n_k})$.
    
    Note that this sum is telescoping, so:
    \[S_N=\vx_{n_{N+1}}-\vx_{n_1}\to\vx\]
    And so $\vx_{n_{N+1}}\to\vx+\vx_{n_1}$.
    
    This means that $(\vx_{n_k})$ is a convergent subsequence of $(\vx_n)$
    that converges to $\vx\in E$.

    Therefore, by previous lemma, $\vx_n\to\vx\in E$ and thus $E$ is complete.
  \end{description}
\end{theproof}

\end{document}
