\documentclass[letterpaper,12pt,fleqn]{article}
\usepackage{matharticle}
\pagestyle{empty}
\renewcommand{\l}{\lambda}
\renewcommand{\O}{\Omega}
\newcommand{\F}{\mathbb{F}}
\newcommand{\norm}[1]{\|#1\|}
\newcommand{\vx}{\vec{x}}
\newcommand{\vy}{\vec{y}}
\newcommand{\vz}{\vec{z}}
\newcommand{\vo}{\vec{0}}
\newcommand{\lp}{\ell^{\,p}}
\begin{document}
\section*{Norms}

\begin{definition}[Norm]
  Let $E$ be a vector space over a scalar field $\F$. A \emph{norm} on $E$ is a
  function $\norm{\cdot}:E\to\R$ such that the following axioms are satisfied
  $\forall\,\vx\in E$ and $\forall\l\in\F$:
  \begin{enumerate}
  \item Definiteness: $\norm{\vx}=0\iff\vx=\vo$
  \item Homogeneity: $\norm{\l\vx}=\abs{\l}\norm{\vx}$
  \item Subadditive: $\norm{\vx+\vy}\le\norm{\vx}+\norm{\vy}$\qquad(triangle
    inequality)
  \end{enumerate}
  A vector space equipped with such a norm is called a \emph{normed} space.
\end{definition}

\begin{theorem}
  Let $E$ be a vector space over a scalar field $\F$ and let $\norm{\cdot}$ be
  a norm on $E$. $\forall,\vx\in E$:
  \[\norm{\vx}=\norm{-\vx}\]
\end{theorem}

\begin{theproof}
  Assume $\vx\in E$.
  \[\norm{-\vx}=\norm{(-1)\vx}=\abs{-1}\norm{\vx}=1\cdot\norm{\vx}=\norm{\vx}\]
\end{theproof}

\begin{theorem}
  Let $E$ be a vector space over a scalar field $\F$ and let $\norm{\cdot}$ be
  a norm on $E$. $\forall,\vx\in E$:
  \[\norm{\vx}\ge0\]
  Thus, the norm is positive-definite.
\end{theorem}

\begin{theproof}
  Assume $\vx\in E$.
  \[0=\norm{\vo}=\norm{\vx+(-\vx)}\le\norm{\vx}+\norm{-\vx}=
  \norm{\vx}+\norm{\vx}=2\norm{\vx}\]
  $\therefore\norm{\vx}\ge0$
\end{theproof}

\begin{theorem}
  Let $E$ be a vector space over a scalar field $\F$ and let $\norm{\cdot}$ be
  a norm on $E$. $\forall,\vx,\vy\in E$:
  \[\abs{\norm{\vx}-\norm{\vy}}\le\norm{\vx-\vy}\]
\end{theorem}

\newpage

\begin{theproof}
  Assume $\vx,\vy\in E$.

  $\norm{\vx}=\norm{(\vx-\vy)+\vy}\le\norm{\vx-\vy}+\norm{\vy}$ \\
  $\norm{\vx}-\norm{\vy}\le\norm{\vx-\vy}$

  $\norm{\vy}=\norm{(\vy-\vx)+\vx}\le\norm{\vy-\vx}+\norm{\vx}$ \\
  $\norm{\vy}-\norm{\vx}\le\norm{\vy-\vx}$ \\
  $-(\norm{\vx}-\norm{\vy})\le\norm{\vx-\vy}$

  $\therefore\abs{\norm{\vx}-\norm{\vy}}\le\norm{\vx-\vy}$
\end{theproof}

\begin{examples}
  \listbreak
  \begin{enumerate}
  \item $E=\R^n$ or $\C^n$:
    \[\norm{\vz}=\sqrt{\sum_{k=1}^n\abs{z_n}^2}\]
    is called the \emph{Euclidean} norm.

  \item $E=\R^n$ or $\C^n$:
    \[\norm{\vz}=\max_{1\le k\le n}\abs{z_n}\]
    is called the \emph{sup} norm.

  \item Let $\O$ be a closed, bounded subset of $\R^n$ and $E=C(\O)$:
    \[\norm{f}_p=\left(\int_{\O}\abs{f(t)}^pdt\right)^{\frac{1}{p}}\]
    is a norm for $1\le p<\infty$. The integral form of Minkowski is needed
    for this:
    \[\left(\int_{\O}\abs{f+g}^p\right)^{\frac{1}{p}}\le
    \left(\int_{\O}\abs{f}^p\right)^{\frac{1}{p}}+
    \left(\int_{\O}\abs{g}^p\right)^{\frac{1}{p}}\]

  \item Let $\O$ be a closed, bounded subset of $\R^n$ and $E=C(\O)$:
    \[\norm{f}_{\infty}=\max_{t\in\O}\abs{f(t)}\]
    is a sup norm for $E$.

  \item Let $E=\lp$:
    \[\norm{x}_p=\left(\sum_{n=1}^{\infty}\abs{x_n}^p\right)^{\frac{1}{p}}\]
    is a norm for $1\le p<\infty$. In fact, Minkowski serves as the triangle
    inequality for this norm.

  \item Let $E=\lp$:
    \[\norm{x}_{\infty}=\sup\abs{x_n}\]
    is a sup norm for $\lp$.
  \end{enumerate}
\end{examples}

\begin{definition}[Metric]
  Let $E$ be a vector space and define function $d:E\times E\to\R$ such that
  $d$ satisfies the following axioms $\forall\vx,\vy,\vz\in E$:
  \begin{enumerate}
  \item $d(\vx,\vy)=0\iff\vx=\vy$
  \item $d(\vx,\vy)=d(\vy,\vx)$
  \item $d(\vx,\vz)\le d(\vx,\vy)+d(\vy,\vz)$
  \end{enumerate}
  The function $d$ is called a \emph{metric} for $E$.

  A vector space equipped with a metric is called a \emph{metric} space.
\end{definition}

\begin{theorem}
  Let $E$ be a normed space. $E$ is a metric space with metric
  $d(\vx,\vy)=\norm{\vx-\vy}$.
\end{theorem}

\begin{theproof}
  Assume $\vx,\vy,\vz\in E$

  $d(\vx,\vy)=0\iff\norm{\vx-\vy}=0\iff\vx-\vy=\vo\iff\vx=\vy$.

  $d(\vx,\vy)=\norm{\vx-\vy}=\norm{\vy-\vx}=d(\vy,\vx)$

  $d(\vx,\vz)=\norm{\vx-\vz}=\norm{(\vx-\vy)+(\vy-\vz)}\le
  \norm{\vx-\vy}+\norm{\vy-\vz}=d(\vx,\vy)+d(\vy,\vz)$
\end{theproof}

\end{document}
