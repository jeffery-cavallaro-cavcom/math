\documentclass[letterpaper,12pt,fleqn]{article}
\usepackage{matharticle}
\pagestyle{empty}
\newcommand{\vw}{\vec{w}}
\newcommand{\vx}{\vec{x}}
\newcommand{\vy}{\vec{y}}
\newcommand{\vyp}{\vec{y}\,'}
\newcommand{\vz}{\vec{z}}
\newcommand{\vzp}{\vec{z}\,'}
\newcommand{\vo}{\vec{0}}
\newcommand{\zs}{\{\vo\}}
\newcommand{\Sp}{S^{\perp}}
\newcommand{\Spp}{S^{\perp\perp}}
\newcommand{\mc}{\mathcal{C}}
\renewcommand{\a}{\alpha}
\renewcommand{\b}{\beta}
\newcommand{\e}{\epsilon}
\renewcommand{\mp}{\mathcal{P}}
\newcommand{\norm}[1]{\left\|#1\right\|}
\newcommand{\inner}[1]{\left<#1\right>}
\newcommand{\conj}[1]{\overline{#1}}
\DeclareMathOperator{\Real}{Re}
\DeclareMathOperator{\Imag}{Im}
\DeclareMathOperator{\proj}{proj}
\begin{document}
\section*{Orthogonal Complement}

\begin{definition}[Orthogonal Complement]
  Let $E$ be an inner product space and let $S$ be a non-empty subset of $E$.
  To say that an $\vx\in E$ is \emph{orthogonal} to $S$, denoted $\vx\perp S$,
  means $\forall\,\vy\in S,\vx\perp\vy$.

  The set of all elements of $E$ that are orthogonal to $S$, denoted $\Sp$, is
  called the \emph{orthogonal complement} of $S$:
  \[\Sp=\{\vx\in E\mid\vx\perp S\}\]

  To say that two non-empty subsets $A$ and $B$ of $E$ are orthogonal, denoted
  $A\perp B$, means $\forall\,\vx\in A,\forall\,\vy\in B,\vx\perp\vy$.
\end{definition}

\begin{examples}
  Let $E$ be an inner product space:
  \begin{enumerate}
  \item $E^{\perp}=\{\vo\}$

  \item $\zs^{\perp}=E$

  \item Let $(\vx_n)$ be a complete orthonormal sequence is $E$:
    \[\{\vx_n\mid n\in\N\}^{\perp}=\{\vo\}\]

  \item $E=\mc[0,1]$ with $\inner{f,g}=\int_0^1f\conj{g}$.

    Let $S=\mp[0,1]$, all polynomials. \\
    Assume $f\in\Sp$. \\
    But there exists $(p_n)$ in $\mp[0,1]$ such that $p_n\rightrightarrows f$,
    and thus $p_n\overset{L_2}{\rightarrow}f$. \\
    Now, by continuity of the inner product, $\forall\,n\in\N$:
    \[\norm{f}^2=\inner{f,f}=\inner{f,\lim_{n\to\infty}p_n}=
    \lim_{n\to\infty}\inner{f,p_n}=0\]
    $\therefore\Sp=\zs$
  \end{enumerate}
\end{examples}

\begin{theorem}
  Let $E$ be an inner product space and $S\subseteq E$:

  \qquad$\Sp$ is a closed subspace of $E$.
\end{theorem}

\begin{theproof}
  Assume $\vx,\vy\in\Sp$. \\
  Assume $\a,\b\in\C$. \\
  Assume $\vz\in S$. \\
  $\inner{\a\vx+\b\vy,\vz}=\a\inner{\vx,\vz}+\b\inner{\vy,\vz}=0$ \\
  Thus $\a\vx+\b\vy\perp\vz$ and so $\a\vx+\b\vy\in\Sp$.

  Therefore, $\Sp$ is a subspace of $E$.

  Assume $(\vx_n)$ is a sequence in $\Sp$ such that $\vx_n\to\vx\in E$. \\
  Assume $\vy\in S$. \\
  So $\forall\,n\in\N,\vy\perp\vx_n$. \\
  $\inner{\vx,\vy}=\inner{\lim_{n\to\infty}\vx_n,\vy}=
  \lim_{n\to\infty}\inner{\vx_n,\vy}=0$ \\
  Thus $\vx\perp\vy$ and so $\vx\in\Sp$.

  Therefore $\Sp$ is closed.
\end{theproof}

\begin{theorem}
  Let $H$ be a Hilbert space and let $S$ be a convex subset of $H$. Let
  $\vx\in H\setminus S$ and let $\vy\in S$ such that $d(\vx,S)=\norm{\vx-\vy}$:
  \[\vx-\vy\perp S\]
  Thus, $\forall\,\vz\in S,\vx-\vy\perp\vz$.
\end{theorem}

\begin{theproof}
  Assume $\vz\in S$ such that $\vz\ne\vo$. \\
  Consider the perturbation $\vy+\e\vz$ for some $\e\in\R$. \\
  Let $d=d(\vx,S)$.
  \begin{eqnarray*}
    d^2 &\le& \norm{\vx-(\vy+\e\vz}^2 \\
    &=& \norm{(\vx-\vy)+\e\vz}^2 \\
    &=& \inner{(\vx-\vy)+\e\vz,(\vx-\vy)+\e\vz} \\
    &=& \norm{\vx-\vy}^2-\inner{\vx-\vy,\e\vz}-\inner{\e\vz,\vx-\vy}+
    \e^2\norm{\vz}^2 \\
    &=& \norm{\vx-\vy}^2-\e[\inner{\vx-\vy,\vz}-\conj{\inner{\vx-\vy,\vz}}]+
    \e^2\norm{\vz}^2 \\
    &=& \norm{\vx-\vy}^2-2\e\Real(\inner{\vx-\vy,\vz})+\e^2\norm{\vz}^2 \\
    &=& d^2-2\e\Real(\inner{\vx-\vy,\vz})+\e^2\norm{\vz}^2 \\
    0 &\le& \e^2\norm{\vz}^2-2\e\Real(\inner{\vx-\vy,\vz}) \\
  \end{eqnarray*}
  Now, let $a=\norm{\vz}^2$ and $b=2\Real(\inner{\vx-\vy,\vz})$:  
  \begin{eqnarray*}
    a\e^2-b\e &\ge& 0 \\
    a\e\left(\e-\frac{b}{a}\right) &\ge& 0
  \end{eqnarray*}
  But for this to always be true it must be the case that $b=0$.

  And so $\Real(\inner{\vx-\vy,\vz})=0$.

  By similar argument for $\vy+i\e\vz$, $\Imag(\inner{\vx-\vy,\vz})=0$.

  Thus $\inner{\vx-\vy,\vz}=0$ and so $\vx-\vy\perp\vz$.

  $\therefore\vx-\vy\perp S$.
\end{theproof}

Such a $\vy$ is called the \emph{orthogonal projection} of $\vx$ onto $S$:
\[y=\proj_S\vx\]
Resulting in a mapping: $\proj_S:H\to S$.

\begin{corollary}
  Let $H$ be Hilbert space and let $S$ be a closed subspace of $H$. Every
  $\vx\in H$ can be written uniquely as $\vx=\vy+\vz$ where $\vy\in S$ and
  $\vz\in\Sp$.

  Thus, $H=S\oplus\Sp$.
\end{corollary}

\begin{theproof}
  Assume $\vx\in H$. \\
  Since $S$ is a closed subspace, $S$ is closed and convex. \\
  Thus a nearest point $\vy\in S$ to $\vx$ exists. \\
  But $\vz=\vx-\vy\perp S$ and so $\vz\in\Sp$.

  Therefore $\vx=\vy+\vz$, thus proving existence.

  Now, assume $\vx=\vy+\vz=\vyp+\vzp$ where $\vy,\vyp\in S$ and
  $\vz,\vzp\in\Sp$. \\
  Let $\vw=\vy-\vyp=\vzp-\vz$. \\
  But $\vy-\vyp\in S$ and $\vzp-\vz\in\Sp$. \\
  And so $\vw\in S$ and $\vw\in\Sp$.

  $\therefore\vw=\vo$ and thus $\vy=\vyp$ and $\vz=\vzp$, thus proving
  uniqueness.
\end{theproof}

\begin{theorem}
  Let $H$ be a Hilbert space and let $S$ be a closed subspaceof $H$:
  \[\Spp=S\]
\end{theorem}

\begin{theproof}
  \listbreak
  \begin{description}
  \item $\subseteq$ Assume $\vx\in\Spp$

    $\exists\,\vy\in S$ and $\vz\in\Sp$ such that $\vx=\vy+\vz$. \\
    So $\vy\in\Spp$. \\
    $\vx-\vy=\vz$. \\
    So $\vz\in\Spp$. \\
    So $\vz\in\Sp$ and $\vz\in\Spp$, and so $\vz=\vo$.

    $\therefore\vx=\vy\in S$

    $\therefore\Spp=\Sp$

  \item $\supseteq$ Assume $\vx\in S$.

    $\vx\perp\Sp$

    $\therefore\vx\in\Spp$
  \end{description}
\end{theproof}

In general, in a Hilbert space $H$, $\Spp$ is the closure of $S$.
\end{document}
