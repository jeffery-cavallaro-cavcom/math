\documentclass[letterpaper,12pt,fleqn]{article}
\usepackage{matharticle}
\pagestyle{empty}
\renewcommand{\l}{\lambda}
\renewcommand{\L}{\Lambda}
\renewcommand{\o}{\sigma}
\newcommand{\norm}[1]{\left\|#1\right\|}
\newcommand{\inner}[1]{\left<#1\right>}
\newcommand{\weak}{\overset{w}{\longrightarrow}}
\newcommand{\vx}{\vec{x}}
\newcommand{\vy}{\vec{y}}
\begin{document}
\section*{Eigenvalues of Compact, Self-adjoint Operators}

\begin{theorem}
  Let $A$ be a compact, self-adjoint operator on a Hilbert space $H$:

  \qquad There exists and eigenvalue $\l$ of $A$ such that
  $\abs{\l}=\norm{A}$.
\end{theorem}

\begin{theproof}
  From a previous theorem, there exists a bounded sequence $(\vx_n)$ in $H$
  such that:
  \begin{itemize}
  \item $\norm{\vx}=1$
  \item $\exists\l\in\o(A)$ such that $A\vx_n-\l\vx_n\to0$
  \item $\abs{\l}=\norm{A}$
  \end{itemize}
  Since $A$ is compact, $(A\vx_n)$ has a convergent subsequence
  $(A\vx{n_k})$. \\
  Let $(A\vx_{n_k})\to\vy\in H$. \\
  And so $A\vx_{n_k}-\l\vx_{n_k}\to0\implies\vy-\l\vx_{n_k}\to0\implies
  \vx_{n_k}\to\frac{1}{\l}\vx_y$. \\
  Thus $\vx_n$ contains a convergent subsequence
  $\vx_{n_k}\to\frac{1}{\l}\vx_y=\vx\in H$. \\
  And so $A\vx-\l\vx=0$ or $A\vx=\l\vx$ where $\vx\ne0$.

  Therefore $\l$ is an eigenvalue of $A$ and $\abs{\l}=\norm{A}$.
\end{theproof}

\begin{corollary}
  Let $A$ be a compact, self-adjoint operator on a Hilbert space $H$.
  $\exists\,\vx_0\in H$ such that $\norm{\vx_0}=1$ and:
  \[\abs{\inner{A\vx_0,\vx_0}}=\sup_{\norm{\vx}=1}\abs{\inner{A\vx,\vx}}\]
\end{corollary}

\begin{theproof}
  By previous theorem, there exists an eigenvector $\l$ of $A$ such that
  $\abs{\l}=\norm{A}$. \\
  Also by previous theorem,
  $\sup_{\norm{\vx}=1}\abs{\inner{A\vx,\vx}}=\norm{A}$.

  Assume $\vx_0$ is an eigenvalue of $A$ such that $\norm{\vx_0}=1$:
  \[\abs{\inner{A\vx_0,\vx_0}}=\abs{\inner{\l\vx_0,\vx_0}}=
  \abs{\l\inner{\vx_0,\vx_0}}=\abs{\l}\norm{\vx_0}^2=\abs{\l}=\norm{A}\]
  $\therefore
  \abs{\inner{A\vx_0,\vx_0}}=\sup_{\norm{\vx}=1}\abs{\inner{A\vx,\vx}}$
\end{theproof}

\begin{theorem}
  Let $A$ be a compact operator on a Hilbert space $H$.
  For all eigenvalues $\l$ of $A$:
  \[\l\ne0\implies\dim E_{\l}<\infty\]
  In other words, eigenspaces for non-zero eigenvalues are finite-dimensional.
\end{theorem}

\begin{theproof}
  Assume $\l$ is an eigenvalue of $A$ such that $\l\ne0$. \\
  ABC: $E_{\l}$ is infinite-dimensional. \\
  Since $E_{\l}=\ker(A-\l I)$, $E_{\l}$ is a closed subspace of $H$ and is thus
  also Hilbert (and separable). \\
  So there exists a complete orthonormal sequence $(\vx_n)$ in $E_{\l}$,
  where $\vx_n\weak0$. \\
  Thus, since $A$ is compact, $A\vx_n\to0$. \\
  And so $A\vx_n=\l\vx_n\to0$. \\
  But $\l\ne0$ and so $\vx_n\to0$. \\
  CONTRADICTION!

  Therefore $E_{\l}$ is finite-dimensional.
\end{theproof}

\begin{definition}
  Let $H$ be a Hilbert space and let $A$ be an operator on $H$:

  \qquad$\L=\{\l\in\C\mid\l$ is an eigenvalue of $A\}$
\end{definition}

\begin{theorem}
  Let $A$ be a compact, self-adjoint operator on a Hilbert space $H$. One
  of the following is true:
  \begin{enumerate}
  \item $\L$ is finite.
  \item $L$ is infinitely-countable such that $\l_n\to0$.
  \end{enumerate}
\end{theorem}

\begin{theproof}
  First, Assume $A=0$. \\
  Thus, $\l=0$ is the only eigenvalue for $A$ and therefore $\L$ is finite.

  So, assume $\L$ is infinite. \\
  Since $A$ is self-adjoint and compact, eigenvectors for distinct eigenvalues
  are orthogonal. \\
  And so there exists an orthonormal sequence $(\vx_n)$ where $\vx_n$ is
  an eigenvector of $\l_n$. \\
  But $H$ is separable and so $(\vx_n)$ is countable.

  Therefore $\L$ is countable.

  Now, since $(\vx_n)$ is orthonormal, $\vx_n\weak0$. \\
  But $A$ is compact and so $A\vx\to0$.

  $\norm{A\vx}=\norm{\l\vx}=\abs{\l}\norm{\vx}=\abs{\l}\to0$
\end{theproof}

\end{document}
