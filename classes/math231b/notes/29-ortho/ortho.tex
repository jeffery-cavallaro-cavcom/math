\documentclass[letterpaper,12pt,fleqn]{article}
\usepackage{matharticle}
\pagestyle{empty}
\newcommand{\vx}{\vec{x}}
\newcommand{\vy}{\vec{y}}
\newcommand{\vo}{\vec{0}}
\newcommand{\norm}[1]{\left\|#1\right\|}
\newcommand{\inner}[2]{\left<#1,#2\right>}
\newcommand{\conj}[1]{\overline{#1}}
\renewcommand{\l}{\lambda}
\renewcommand{\o}{\theta}
\newcommand{\F}{\mathbb{F}}
\DeclareMathOperator{\spn}{Span}
\begin{document}
\section*{Orthogonal}

\begin{definition}[Orthogonal]
  Let $E$ be an inner product space and let $\vx,\vy\in E$. To say that $\vx$
  is \emph{orthogonal} to $\vy$, denoted $\vx\perp\vy$, means:
  \[\inner{\vx}{\vy}=0\]
\end{definition}

\begin{properties}
  Let $E$ be an inner product space. $\forall\,\vx,\vy\in E$:
  \begin{enumerate}
  \item $\vx\perp\vo$
  \item $\vx\perp\vy\iff\vy\perp\vx$
  \end{enumerate}
\end{properties}

\begin{theproof}
  Assume $\vx,\vy\in E$:
  \begin{enumerate}
  \item $\inner{\vx}{\vo}=0$

  \item
    \begin{description}
    \item Assume $\vx\perp\vy$

      $\inner{\vx}{\vy}=0$ \\
      $\inner{\vy}{\vx}=\conj{\inner{\vx}{\vy}}=\conj{0}=0$

      $\therefore\vy\perp\vx$
      
    \item Assume $\vy\perp\vx$

      $\inner{\vy}{\vx}=0$ \\
      $\inner{\vx}{\vy}=\conj{\inner{\vy}{\vx}}=\conj{0}=0$

      $\therefore\vx\perp\vy$
    \end{description}
  \end{enumerate}
\end{theproof}

\begin{theorem}[Pythagorean]
  Let $E$ be an inner product space and $\vx,\vy\in E$:
  \[\vx\perp\vy\implies\norm{\vx+\vy}^2=\norm{\vx}^2+\norm{\vy}^2\]
\end{theorem}

\begin{theproof}
  Assume $\vx\perp\vy$.

  Also $\vy\perp\vx$ and so $\inner{\vx}{\vy}=\inner{\vy}{\vx}=0$.
    
  $\norm{\vx+\vy}^2=\inner{\vx+\vy}{\vx+\vy}=
  \inner{\vx}{\vx}+\inner{\vx}{\vy}+\inner{\vy}{\vx}+\inner{\vy}{\vy}=
  \norm{\vx}^2+0+0+\norm{\vy}^2$

  $\therefore\norm{\vx+\vy}^2=\norm{\vx}^2+\norm{\vy}^2$
\end{theproof}

\begin{definition}[Mutually Orthogonal]
  Let $E$ be an inner product space and let $S=\{\vx_1,\ldots,\vx_n\}$ be a
  non-empty subset of $E$. To say that $S$ is a \emph{mutually orthogonal} set
  means:
  \[\forall\,i\ne j,\vx_i\perp\vx_j\]
\end{definition}

\begin{lemma}
  Let $E$ be an inner product space over a field $\F$ and let
  $S=\{\vx_1,\ldots,\vx_n\}$ be a mutually orthogonal subset of $E$.
  $\forall\,1\le r\le n$:
  \[S'=\{\vx,\vx_{r+1},\ldots,\vx_n\}\]
  where $\vx\in\spn\{\vx_1,\ldots,\vx_r\}$ is also a mutually orthogonal set.
\end{lemma}

\begin{theproof}
  Assume $1\le r\le n$. \\
  Assume $\vx\in\spn\{\vx_1,\ldots,\vx_r\}$. \\
  $\exists\,\l_k\in\F$ such that $\vx=\sum_{k=1}^r\l_k\vx_k$.

  Assume $r+1\le j\le n$.

  $\inner{\vx}{\vx_j}=\inner{\sum_{k=1}^r\l_k\vx_k}{\vx_j}=
  \l_k\sum_{k=1}^r\inner{\vx_k}{\vx_j}=\l_k\sum_{k=1}^r0=0$

  Thus, $\vx$ is orthogonal to all the $\vx_j\mid r+1\le j\le n$. \\
  Furthermore, all the $\vx_j$ are orthogonal.

  Therefore $S'$ is a mutually orthogonal set.
\end{theproof}

\begin{corollary}
  Let $E$ be an inner product space and let $\{\vx_1,\ldots,\vx_n\}$ be a
  non-empty, mutually orthogonal subset of $E$:
  \[\norm{\sum_{k=1}^n\vx_k}^2=\sum_{k=1}^n\norm{\vx_k}^2\]
\end{corollary}

\begin{theproof}
  By induction on $n$:
  \begin{description}
  \item Base case: $n=1$

    Trivial.

  \item Assume for mutually orthogonal set $\{\vx_1,\ldots,\vx_n\}\subset E$:
    \[\norm{\sum_{k=1}^n\vx_k}^2=\sum_{k=1}^n\norm{\vx_k}^2\]

  \item Consider the mutually orthogonal set
    $\{\vx_1,\ldots,\vx_n,\vx_{n+1}\}\subset E$:

    $\norm{\sum_{k=1}^{n+1}\vx_k}^2=\norm{\sum_{k=1}^n\vx_k+\vx_{n+1}}^2=
    \norm{\sum_{k=1}^n\vx_k}^2+\norm{\vx_{n+1}}^2=
    \sum_{k=1}^n\norm{\vx_k}^2+\norm{\vx_{n+1}}^2=
    \sum_{k=1}^{n+1}\norm{\vx_k}^2$
  \end{description}
\end{theproof}

\bigskip

\begin{definition}[Euclidean]
  A \emph{Euclidean} space is a finite-dimensional real inner product space.
  In other words, $\inner{\cdot}{\cdot}\in\R$.
\end{definition}

Applying the Cauchy-Schwarz inequality to a Euclidean space:
\[\abs{\inner{\vx}{\vy}}\le\norm{\vx}\norm{\vy}\]
For $\vx,\vy\ne\vo$:
\[\frac{\abs{\inner{\vx}{\vy}}}{\norm{\vx}\norm{\vy}}\le1\]
\[-1\le\frac{\inner{\vx}{\vy}}{\norm{\vx}\norm{\vy}}\le1\]
So let $\cos\o=\frac{\inner{\vx}{\vy}}{\norm{\vx}\norm{\vy}}$:
\[\inner{\vx}{\vy}=\norm{\vx}\norm{\vy}\cos\o\]

\end{document}
