\documentclass[letterpaper,12pt,fleqn]{article}
\usepackage{matharticle}
\usepackage{enumitem}
\pagestyle{plain}
\newcommand{\ve}{\vec{e}}
\newcommand{\vx}{\vec{x}}
\newcommand{\vy}{\vec{y}}
\newcommand{\norm}[1]{\left\|#1\right\|}
\newcommand{\inner}[1]{\left<#1\right>}
\newcommand{\conj}[1]{\overline{#1}}
\newcommand{\mb}{\mathcal{B}}
\renewcommand{\a}{\alpha}
\renewcommand{\b}{\beta}
\renewcommand{\l}{\lambda}
\begin{document}
Cavallaro, Jeffery \\
Math 231b \\
Homework \#6

\subsection*{4.12.20}

Let $(\ve_n)$ be a complete orthonormal sequence in a Hilbert space $H$. Show
that a bounded operator $A$ on $H$ is unitary if and only if $(A\ve_n)$ is a
complete orthonormal sequence in $H$.

\begin{description}
\item $\implies$ Assume $A$ is unitary.

  Since $A\in\mb(H)$ it is also the case that $A^*\in\mb(H)$.

  $A$ is isometric and thus preserves the norm and inner product.
  
  $\inner{\ve_i,\ve_j}=\inner{A\ve_i,A\ve_j}$ \\
  But $\ve_i\perp\ve_j$ and so $\inner{\ve_i,\ve_j}=\inner{A\ve_i,A\ve_j}=0$ \\
  $\therefore A\ve_i\perp A\ve_j$

  Also $\norm{A\vx}=\norm{\vx}=1$.

  Therefore $(A\ve_n)$ is an orthonormal sequence.

  Now, assume $\vy\in H$. \\
  Since $A$ is onto, $\exists\,\vx\in H$ such that $\vy=A\vx$. \\
  Since $(\ve_n)$ is complete orthonormal and since $A$ is linear and
  isometric:
  \[\vy=A\vx=A\sum_{k=1}^{\infty}\inner{\vx,\ve_k}\ve_k=
  \sum_{k=1}^{\infty}\inner{A\vx,A\ve_k}A\ve_k\]
  Therefore $(A\ve_n)$ is complete.

\item $\impliedby$ Assume $(A\ve_n)$ is a complete orthonormal sequence.

  Assume $\vx\in H$.

  $\vx=\sum_{k=1}^{\infty}\inner{\vx,A\ve_k}A\ve_k=
  A\sum_{k=1}^{\infty}\inner{A^*\vx,\ve_k}\ve_k$

  But $(\ve_n)$ is complete orthonormal, and so:

  $\vx=AA^*\vx$ for all $\vx\in H$ and thus $AA^*=I$

  Now, note that:
  \[A^*A\vx=\sum_{k=1}^{\infty}\inner{A^*A\vx,\ve_k}\ve_k=
  \sum_{k=1}^{\infty}\inner{A\vx,A\ve_k}\ve_k\]
  Let $\vx=\ve_j$:
  \[A^*A\ve_j=\sum_{k=1}^{\infty}\inner{A\ve_j,A\ve_k}\ve_k=
  \norm{A\ve_j}^2\ve_j=1\cdot\ve_j=\ve_j\]
  Thus $(A^*A\ve_n)$ is also a complete orthonormal sequence.

  Assume $\vx\in H$:
  \[\vx=\sum_{k=1}^{\infty}\inner{\vx,A^*A\ve_k}A^*A\ve_k=
  A^*\sum_{k=1}^{\infty}\inner{A\vx,A\ve_k}A\ve_k=A^*A\vx\]
  And so $\vx=A^*A\vx$ for all $\vx\in H$ and thus $A^*A=I$

  Therefore $AA^*=A^*A=I$ and thus $A$ is unitary.
\end{description}

\subsection*{4.12.23}

Let $A$ be a bounded operator on a Hilbert space. Define the exponential
operator:
\[e^A=\sum_{n=0}^{\infty}\frac{A^n}{n!}\]
where $A^0=I$.

Show that $e^A$ is a well-defined operator.

$\sum_{k=1}^{\infty}\norm{\frac{A^n}{n!}}=
\sum_{k=1}^{\infty}\frac{\norm{A^n}}{n!}\le
\sum_{k=1}^{\infty}\frac{\norm{A}^n}{n!}$

which converges to $e^{\norm{A}}$ for all $\norm{A}\in\R$.

So $e^A$ converges absolutely. But $H$ is complete, so $e^A$ converges.

Therefore $e^A$ is well-defined.

Prove the following:
\begin{enumerate}[label=(\alph*)]
\item $(e^A)^n=e^{nA}$

  Proof by induction on $n$:
  \begin{description}
  \item Base case: $n=1$

    Trivial.

  \item Assume $(e^A)^n=e^{nA}$

  \item Consider $(e^A)^{n+1}$.

    $(e^A)^{n+1}=\left[\sum_{k=0}^{\infty}\frac{A^k}{k!}\right]^{n+1}=
    \left[\sum_{k=0}^{\infty}\frac{A^k}{k!}\right]^n
    \left[\sum_{k=0}^{\infty}\frac{A^k}{k!}\right]=
    (e^A)^ne^A=e^{nA}e^A$

    Note that $nA$ and $A$ commute, so applying part (d):
    \[(e^A)^{n+1}=e^{nA}e^A=e^{nA+A}=e^{(n+1)A}\]
  \end{description}
  
\item $e^0=I$

  $e^A=\sum_{n=0}^{\infty}\frac{A^n}{n!}=
  \frac{A^0}{0!}+\sum_{n=1}^{\infty}\frac{A^n}{n!}=
  I+\sum_{n=1}^{\infty}\frac{A^n}{n!}$

  Now let $A=0$:

  $e^0=I+\sum_{n=1}^{\infty}\frac{0^n}{n!}=I+0=I$
  
\item $e^A$ is invertible (even if $A$ is not) and its inverse is $e^{-A}$.

  Note that $A$ and $-A$ commute, so applying part (d):

  $e^Ae^{-A}=e^{A-A}=e^0=I$

  $e^{-A}e^A=e^{-A+A}=e^0=I$

  Therefore $e^A$ is invertible with inverse $e^{-A}$.
  
\item $e^Ae^B=e^{A+B}$ for any commuting operators $A$ and $B$.
  \begin{eqnarray*}
    e^{A+B} &=& \sum_{n=0}^{\infty}\frac{(A+B)^n}{n!} \\
    &=& \sum_{n=0}^{\infty}\frac{1}{n!}\sum_{k=0}^n\binom{n}{k}A^kB^{n-k} \\
    &=& \sum_{n=0}^{\infty}\frac{1}{n!}\sum_{k=0}^n
    \frac{n!}{k!(n-k)!}A^kB^{n-k} \\
    &=& \sum_{n=0}^{\infty}\sum_{k=0}^n\frac{A^k}{k!}\frac{B^{n-k}}{(n-k)!} \\
    &=& \left[\sum_{n=0}^{\infty}\frac{A^k}{k!}\right]
    \left[\sum_{n=0}^{\infty}\frac{B^k}{k!}\right] \\
    &=& e^Ae^B
  \end{eqnarray*}
  (by the Cauchy product of two infinite, absolutely converging series).
    
\item If $A$ is self-adjoint then $e^{iA}$ is unitary.

  Assume $A$ is self-adjoint. \\
  $A=A^*$
  
  \begin{eqnarray*}
    (e^{iA})^* &=& \left[\sum_{n=1}^{\infty}\frac{(iA)^n}{n!}\right]^* \\
    &=& \lim_{N\to\infty}\left[\sum_{n=1}^N\frac{(iA)^n}{n!}\right]^* \\
    &=& \lim_{N\to\infty}\sum_{n=1}^N\frac{[(iA)^n]^*}{n!} \\
    &=& \lim_{N\to\infty}\sum_{n=1}^N\frac{[(iA)^*]^n}{n!} \\
    &=& \lim_{N\to\infty}\sum_{n=1}^N\frac{[-iA^*]^n}{n!} \\
    &=& \lim_{N\to\infty}\sum_{n=1}^N\frac{[-iA]^n}{n!} \\
    &=& e^{-iA}
  \end{eqnarray*}

  Therefore, by part (c), $e^{iA}$ is unitary.
\end{enumerate}

\subsection*{4.12.28}

If $T^*T=I$, is it true that $TT^*=I$?

No. Let $H=\ell^2$ and let $T$ be the right-shift operator:
\[T(z_1,z_2,z_3,\ldots)=(0,z_1,z_2,z_3,\ldots)\]
Claim: $\forall\,z\in\ell^2,Tz\in\ell^2$

Assume $z\in\ell^2$.

$\sum_{n=1}^{\infty}\abs{(Tz)_n}^2=0+\sum_{n=1}^{\infty}\abs{z_n}^2=
\sum_{n=1}^{\infty}\abs{z_n}^2<\infty$

Claim: $T$ is linear.

Assume $x,y\in\ell^2$ and $\a,\b\in\C$.
\begin{eqnarray*}
  T(\a x+\b y) &=& T(\a x_1+\b y_1,\a x_2+\b y_2,\a x_3+\b y_3,\ldots) \\
  &=& (0,\a x_1+\b y_1,\a x_2+\b y_2,\a x_3+\b y_3,\ldots) \\
  &=& \a(0,x_1,x_2,x_3,\ldots)+\b(0,y_1,y_2,y_3,\ldots) \\
  &=& \a Tx+\b Ty
\end{eqnarray*}
Claim: $T$ is bounded.
\[\norm{Tz}^2=\sum_{n=1}^{\infty}\abs{(Tz)_n}^2=0+\sum_{n=1}^{\infty}\abs{z_n}^2
=\norm{z}^2\]
Therefore $\norm{T}\le1$ and thus $T$ is bounded.

$\therefore T\in\mb(\ell^2)$

Now, assume $T^*T=I$. \\
Thus, $T^*$ is a left-inverse of $T$ and must therefore be the left-shift
operator:
\[T^*(z_1,z_2,z_3,z_4\ldots)=(z_2,z_3,z_4,\ldots)\]
And so:
\[T^*T(z_1,z_2,z_3,\ldots)=T^*(0,z_1,z_2,z_3,\ldots)=(z_1,z_2,z_3,\ldots)\]
However:
\[TT^*(z_1,z_2,z_3,\ldots)=T(z_2,z_3,\ldots)=(0,z_2,z_3,\ldots)\ne
(z_1,z_2,z_3,\ldots)\]
And therefore $TT^*\ne I$.

\subsection*{4.12.31}

If $A$ and $B$ are positive operators and $A+B=0$, show that $A=B=0$.

Assume $A$ and $B$ are positive operators and $A+B=0$.

$\inner{(A+B)\vx,\vx}=\inner{A\vx+B\vx,\vx}=
\inner{A\vx,\vx}+\inner{B\vx,\vx}=0$

But $A$ and $B$ are positive, so $\inner{A\vx,\vx}\ge0$ and
$\inner{B\vx,\vx}\ge0$.

And so $\inner{A\vx,\vx}=\inner{B\vx,\vx}=0$, for all $\vx$.

$\therefore A=B=0$

\subsection*{4.12.54}

Give an example of a self-adjoint operator that has no eigenvalues.

Let $H=L^2[a,b]$ and let $f_0\in H$ be a real-valued, continuous, non-constant
(and hence non-zero), and bounded function. For example: $f_0(x)=2+\sin x$.

Define $Tf=f_0f$.

Claim: $T$ is linear.

$T(\a f+\b g)=f_0(\a f+\b g)=\a f_0f+\b f_0g=\a Tf+\b Tg$

Claim: $T$ is bounded.

$\norm{Tf}=\norm{f_0f}\le\norm{f_0}\norm{f}$

Therefore $\norm{T}\le\norm{f_0}$ and thus $T$ is bounded.

$\therefore T\in\mb(H)$

Claim: $T$ is self-adjoint.
\begin{eqnarray*}
  \inner{Tf,g} &=& \int_a^b((Tf)(x))\conj{g(x)}dx \\
  &=& \int_a^b(f_0f)(x)\conj{g(x)}dx \\
  &=& \int_a^bf_0(x)f(x)\conj{g(x)}dx \\
  &=& \int_a^bf(x)\conj{f_0(x)g(x)}dx \\
  &=& \int_a^bf(x)\conj{(Tg)(x)}dx \\
  &=& \inner{f,Tg}
\end{eqnarray*}

Claim: $T$ has no eigenvalues.

If it did, then $(Tf)(x)=(f_0f)x=f_0(x)f(x)=\l f(x)$ for all $x\in[a,b]$.
But this is only true for $f(x)\equiv0$.

Therefore $T$ has no eigenvalues.

\end{document}
