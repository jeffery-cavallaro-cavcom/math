\documentclass[letterpaper,12pt,fleqn]{article}
\usepackage{matharticle}
\pagestyle{plain}
\DeclareMathOperator{\End}{End}
\begin{document}
Cavallaro, Jeffery \\
Math 221b \\
Homework \#1

\bigskip

\begin{enumerate}
\item Let $A$ be an abelian group. Prove that $\End(A)$ is a ring with pointwise
  addition and composition as multiplication.

  Assume $\phi,\mu,\gamma\in\End(A)$ \\
  $\phi$, $\mu$, and $\gamma$ are functions on $A$ \\
  Assume $a\in A$

  $\phi(a)\in A$ \\
  $\mu(a)\in A$ \\
  But $A$ is a group, so by closure: \\
  $(\phi+\mu)(a)=\phi(a)+\mu(a)\in A$

  $\therefore\End(A)$ is closed under addition.

  $(\phi\mu)(a)=\phi(\mu(a))\in A$

  $\therefore\End(A)$ is closed under multiplication (composition).

  $A$ is a group and is thus associative under addition:
  \begin{eqnarray*}
    ((\phi+\mu)+\gamma)(a) &=& (\phi+\mu)(a)+\gamma)(a) \\
    &=& (\phi(a)+\mu(a))+\gamma(a) \\
    &=& \phi(a)+(\mu(a)+\gamma(a)) \\
    &=& \phi(a)+(\mu+\gamma)(a) \\
    &=& (\phi+(\mu+\gamma))(a)
  \end{eqnarray*}
  $\therefore\End(A)$ is associative under addition.

  And likewise for multiplication (composition):
  \begin{eqnarray*}
    ((\phi\mu)\gamma)(a) &=& (\phi\mu)(\gamma(a)) \\
    &=& \phi(\mu(\gamma(a))) \\
    &=& \phi((\mu\gamma)(a)) \\
    &=& (\phi(\mu\gamma))(a)
  \end{eqnarray*}
  $\therefore\End(A)$ is associative under multiplication (composition).

  $A$ is a group, so $0\in A$ is a two-sided additive identity for $A$ \\
  Let $0_A$ be the zero (trivial) endomorphism \\
  $0_A\in\End(A)$ \\
  $(\phi+0_A)(a)=\phi(a)+0_A(a)=\phi(a)+0=\phi(a)$ \\
  $(0_A+\phi)(a)=0_A(a)+\phi(a)=0+\phi(a)=\phi(a)$

  Therefore $0_A$ is a two-sided additive identity for $\End(A)$.

  Let $\phi'=-\phi$ \\
  Since $A$ is a group it is closed under additive inverses, so: \\
  $\phi'(a)=-\phi(a)\in A$ \\
  Assume $b\in A$ \\
  $\phi'(a+b)=-\phi(a+b)=-(\phi(a)+\phi(b))=-\phi(a)+(-\phi(b))=\phi'(a)+\phi'(b)$ \\
  $\phi'$ is a homomorphism, and hence an endomorphism \\
  $\phi'\in\End(A)$ \\
  $(\phi'+\phi)(a)=\phi'(a)+\phi(a)=-\phi(a)+\phi(a)=0=0_A(a)$ \\
  $(\phi+\phi')(a)=\phi(a)+\phi'(a)=\phi(a)+(-\phi(a))=0=0_A(a)$ \\
  So $\phi'$ is a two-sided additive inverse for $\phi$

  $\therefore\End(A)$ is closed under additive inverses.

  $\therefore\End(A)$ is a group.

  $A$ is an abelian (commutative) group: \\
  \[(\phi+\mu)(a)=\phi(a)+\mu(a)=\mu(a)+\phi(a)=(\mu+\phi)(a)\]

  $\therefore\End(A)$ is an abelian group.

  $\phi$ is a group homomorphism, so:
  \begin{eqnarray*}
    (\phi(\mu+\gamma))(a) &=& \phi((\mu+\gamma)(a)) \\
    &=& \phi(\mu(a)+\gamma(a)) \\
    &=& \phi(\mu(a))+\phi(\gamma(a)) \\
    &=& (\phi\mu)(a)+(\phi\gamma)(a) \\
    &=& (\phi\mu+\phi\gamma)(a)
  \end{eqnarray*}
  $\therefore$ left distributivity holds.

  Likewise:
  \begin{eqnarray*}
    ((\mu+\gamma)\phi)(a) &=& (\mu+\gamma)(\phi(a)) \\
    &=& \mu(\phi(a))+\gamma(\phi(a)) \\
    &=& (\mu\phi)(a)+(\gamma\phi)(a) \\
    &=& (\mu\phi+\gamma\phi)(a)
  \end{eqnarray*}
  $\therefore$ right distributivity holds.

  So $\End(A)$ is an additive abelian group, is associative under multiplication
  (composition), and the distributive properties hold

  $\therefore\End(A)$ is a ring.
\newpage
  \newcommand{\Rx}{R^{\times}}

\item
  \begin{enumerate}
  \item Let $R$ be a ring with $1\ne 0$. Prove: $\Rx$ is a group.

    $R$ is ring and thus is associative under multiplication \\
    $\Rx\subset R$
    
    $\therefore \Rx$ inherits multiplicative associativity.

    $1\in R$ \\
    $1\cdot1=1$ \\
    $1$ is a unit \\
    $1\in \Rx$
    
    $\therefore \Rx\ne\emptyset$

    Assume $r,s\in \Rx$ \\
    By construction: $r^{-1},s^{-1}\in \Rx$ \\
    $r,s,r^{-1},s^{-1}\in R$ \\
    By closure, $rs,s^{-1}r^{-1}\in R$ \\
    $1$ is a two-sided identity for $R$ \\
    $(rs)(s^{-1}r^{-1})=r(ss^{-1})r^{-1}=r1r^{-1}=rr^{-1}=1$ \\
    $(s^{-1}r^{-1})(rs)=s^{-1}(r^{-1}r)s=s^{-1}1s=s^{-1}s=1$ \\
    So $s^{-1}r^{-1}$ is a two-sided multiplicative inverse for $rs$ in $R$ \\
    $rs$ is a unit \\
    $rs\in \Rx$

    $\therefore \Rx$ is closed under multiplication.

    $r1=1r=r$ \\
    $\therefore 1$ is a two-sided identity for $\Rx$.

    By construction, $\Rx$ is closed under multiplicative inverses.

    $\therefore \Rx$ is a multiplicative group.

    \newcommand{\M}{M_2(\Z)}
    \newcommand{\Mx}{\M^{\times}}
    \newcommand{\MA}{\{A\in\M\mid\det(A)=\pm1\}}

    \item Prove: $\Mx=\MA$

      It is known that $\Z$ is a commutative ring with unity $1$ \\
      It is also known that $\M$ is a ring with unity $I_2$
      
      Assume $B\in\M$ \\
      Let $B=\begin{bmatrix}a & b \\ c & d\end{bmatrix}$, $a,b,c,d\in\Z$ \\
      $\det(B)=ad-bc\in\Z$ (closure)

      \begin{description}
      \item $\implies$ Assume $B\in\Mx$

        By construction, $B$ is a unit \\
        So $B$ is invertible and $B^{-1}\in\Mx$ \\
        $BB^{-1}=I_2$ \\
        $\det(BB^{-1})=\det(I_2)=1$ \\
        $\det(B)\det(B^{-1})=1$ \\
        Thus, $\det(B)$ and $\det(B^{-1})$ must be units in $\Z$ \\
        But $\Z^{\times}=\{\pm1\}$ \\
        So $\det(B)=\pm1$

        $\therefore B\in\MA$

      \item $\impliedby$ Assume $B\in\MA$

        $\det(B)=ad-bc=\pm1\ne0$ \\
        So $B$ is invertible and $B^{-1}$ exists \\
        $B^{-1}=\frac{1}{ad-bc}\begin{bmatrix} d & -b \\ -c & a\end{bmatrix}$ \\
        But $ad-bc=\pm1$ and $a,(-b),(-c),d\in\Z$, so $B^{-1}\in\M$ \\
        So $B$ and $B^{-1}$ are multiplicative inverses in $\M$ \\
        $B$ is a unit in $\M$
        
        $\therefore B\in\Mx$
      \end{description}

      $\therefore\Mx=\MA$

      \newcommand{\ZnZ}{\Z/n\Z}
      \newcommand{\ZnZx}{(\ZnZ)^{\times}}
      \newcommand{\anZ}{\{a+n\Z\mid(a,n)=1\}}

    \item Prove: $\forall\,n\in\Z^+,\ZnZx=\anZ$

      Assume $n\in\Z^+$

      It is known that $\ZnZx$ is ring with unity $1+nZ$
      \begin{eqnarray*}
        a+n\Z\in\ZnZx &\iff& \exists\,b+n\Z\in\ZnZx,(a+n\Z)(b+n\Z)=ab+n\Z=1+n\Z \\
        &\iff& ab\equiv1\pmod{n} \\
        &\iff& \exists\,k\in\Z,ab-1=kn \\
        &\iff& ba+(-k)n=1\ \mbox{has solutions in}\ \Z \\
        &\iff& (a,n)=1\hspace{0.25in}\mbox{(B\'{e}zout)} \\
        &\iff& a+n\Z\in\anZ
      \end{eqnarray*}

      \newcommand{\EZ}{\Z[i]}
      \newcommand{\EZx}{\EZ^{\times}}
      \newcommand{\U}{\{\pm1,\pm i\}}

    \item Prove: $\EZx=\U$

      It is known that $\Z$ is a ring with unity $1$ \\
      It is also known that $\EZ$ is a ring with unity $1+i0=1$ \\
      $\EZ=\{a+ib\in\C\mid a,b\in\Z\}$ \\
      $\abs{a+ib}^2=a^2+b^2\in\Z$ (closure)

      \begin{description}
      \item $\implies$ Assume $z\in\EZx$

        $\exists z'\in\EZx,zz'=1$ \\
        $\abs{zz'}=1$ \\
        $\abs{zz'}^2=1$ \\
        $\abs{z}^2\abs{z'}^2=1$ \\
        But $\abs{z}^2,\abs{z'}^2\in\Z$ \\
        So $\abs{z}^2$ and $\abs{z'}^2$ are units in $\Z$ \\
        But both are $\ge0$ \\
        So $\abs{z}^2=\abs{z'}^2=1$ \\
        But $\abs{z}\in\R$ and $\abs{z}\ge0$ \\
        So $\abs{z}=1$, the unit circle \\
        But the only lattice points on the unit circle are $\U$

        $\therefore z\in\U$

      \item $\impliedby$ Assume $a+ib\in\U$

        $1=1+i0\in\EZ$ \\
        $1\cdot1=1$
        
        $-1=-1+i0\in\EZ$ \\
        $(-1)\cdot(-1)=1$

        $i=0+i1\in\EZ$ \\
        $-i=0+i(-1)\in\EZ$ \\
        $i\cdot(-i)=1$

        $\therefore\U\subseteq\EZx$
      \end{description}

      $\therefore\EZx=\U$
  \end{enumerate}

\item Prove: Every finite integral domain is a field.

  Assume $F$ is a finite integral domain \\
  $F$ is a commutative ring with unity $1\ne0$

  Assume $a\in F,a\ne0$ \\
  Let $L_a:F\to F$ be defined by $L_a(x)=ax$

  Assume $L_a(x)=L_a(y)$ \\
  $ax=ay$ \\
  But $F$ is an integral domain, so the cancellation laws hold \\
  $x=y$ \\
  $\therefore L_a$ is one-to-one.

  But $F$ is finite, so $L_a$ is also onto \\
  $\therefore L_a$ is a bijection on $F$.

  $1\in F$ \\
  $\exists\,x\in F,L_a(x)=1$ \\
  $ax=1$ \\
  But $F$ is cummutative so $xa=1$ \\
  So $x$ is a multiplicative inverse for $a$ \\
  Thus every non-zero element of $F$ has a multiplicative inverse

  $\therefore F$ is a field.
\end{enumerate}

\end{document}
