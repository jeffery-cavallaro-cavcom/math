\documentclass[letterpaper,12pt,fleqn]{article}
\usepackage{matharticle}
\pagestyle{plain}
\newcommand{\ide}{\trianglelefteq}
\newcommand{\w}{\omega}
\begin{document}
Cavallaro, Jeffery \\
Math 221b \\
Homework \#5

\bigskip

\begin{enumerate}
\item Show that every non-zero prime ideal in a PID $R$ is maximal.

  Assume $R$ is a PID and $a\in R$ such that $a\ne0$ and $(a)$ is prime. \\
  Since $(a)$ is prime, by definition it must be a proper ideal. \\
  ABC: $(a)$ is not maximal. \\
  Since a PID is an integral domain, there exists $b\in R$ such that $(a)$ is a
  proper subset of $(b)$ and $(b)$ is maximal in $R$. \\
  To contain is to divide, so $b\mid a$. \\
  So $\exists\,c\in R$ such that $a=cb\in(a)$. \\
  But $(a)$ is prime, so $b\in(a)$ or $c\in(a)$.
  \begin{description}
  \item Case 1: $b\in(a)$

    Let $b=as$ for some $s\in R$ \\
    Assume $d\in(b)$ \\
    Let $d=br$ for some $r\in R$. \\
    Note that by closure, $sr\in R$, and so: \\
    $d=(as)r=a(sr)\in(a)$ \\
    Thus $(b)\subseteq(a)$ \\
    Contradiction.

  \item Case 2: $c\in(a)$

    Then $\exists\,d\in R$ such that $c=ad$. \\
    So $a=(ad)b=a(db)$ and so $db=1$. \\
    Thus $b$ is a unit and $(b)=R$. \\
    Contradiction.
  \end{description}
  Therefore $(a)$ is maximal.

  A shorter proof might be:

  Assume $R$ is a PID and $a\in R$ such that $a\ne0$ and $(a)$ is prime. \\
  Since $(a)$ is prime, by definition it must be a proper ideal. \\
  Since $R$ is a PID, prime and irreducible are the same thing. \\
  So $a$ is prime and irreducible and thus has no non-trivial factorization. \\
  To divide is to contain, so since $a$ has no non-trivial divisors there is
  no containing proper ideal other than $R$. \\
  Therefore $(a)$ is maximal.
\newpage  
\item Let $R$ be a ring and $I_1\subseteq I_2\subseteq I_3\subseteq\ldots$ be
  a chain of ideals in $R$. Prove:
  \[I=\bigcup_{k=1}^{\infty}I_k\ide R\]

  Clearly, $I$ is a non-empty subset of $R$

  Assume $a,b\in I$ \\
  $\exists\,i,j\in\Z^+$ such that $a\in I_i$ and $b\in I_j$ \\
  AWLOG: $I_i\subseteq I_j$ \\
  So also $a\in I_j$ \\
  $I_j$ is a group so $(-b)\in I_j$ and by closure $a-b\in I_j$ \\
  But $I_j\subseteq I$, so $a-b\in I$

  Therefore, by the subgroup test, $I$ is a subgroup of $R$. Furthermore,
  since $R$ is an additive abelian group, so is $I$.

  Since $a\in I_i$ and $I_i\subseteq I$ we have $a\in I$ as well \\
  Assume $r\in R$ \\
  $I_i$ is an ideal, and so $ar\in I_i$ and thus $ar\in I$ \\
  Likewise, $ra\in I$

  Therefore, $I\ide R$.
\newpage
\item Show that an integral domain $R$ is Noetherian iff every ideal is
  finitely-generated.
  \begin{description}
  \item $\implies$ Assume $R$ is Noetherian:

    Assume $I\ide R$. \\
    Assume $a_1\in I$. \\
    If $(a_1)=I$ then $I$ is finitely-generated, so done. \\
    Otherwise, choose $a_2\in I\setminus(a_1)$. \\
    If $(a_1,a_2)=I$ then $I$ is finitely-generated, so done. \\
    Continue in this fashion as long as the generated ideal does not equal $I$,
    which creates the chain:
    \[(a_1)\subset(a_1,a_2)\subset(a_1,a_2,a_3)\subset\cdots\]
    But $R$ is Noetherian, so there exists $k\in\Z^+$ such that the chain
    stabilizes after $k$ steps. At that point, $(a_1,\ldots,a_k)=I$,
    otherwise, another step could be performed.

    Therefore, $I$ is finitely-generated with $k$ generators.

  \item $\impliedby$ Assume every ideal in $R$ is finitely-generated:

    Assume $\mathcal{C}$ is an ascending chain of ideals in $R$:
    \[I_1\subseteq I_2\subseteq I_3\subseteq\cdots\]
    By the result of problem (2):
    \[I=\bigcup_{k=1}^{\infty}I_k\ide R\]
    Furthermore, by assumption, $I$ is finitely-generated, so let
    $I=(a_1,a_2,\ldots,a_n)$. \\
    For each $a_i$ in the generating set, pick a ideal in the chain where it
    occurs and identify that ideal by $I_{k_i}$. \\
    Let $k=\max\{k_i\}$ \\
    So by the $I_k$ ideal, all generators have been included and the chain must
    thus stabilize.

    Therefore $R$ is Noetherian.
  \end{description}
\newpage
\item Show that $\Z[\w]=R_{-3}$ (the ring of Eisenstein integers) is a
  Euclidean domain with Euclidean function:
  \[N(a+b\w)=a^2-ab+b^2\]

  Let $a,b\in\Z[\w]$. By the division algorithm and working in $\Q[\w]$ (the
  field of fractions), we have:
  \[\frac{a}{b}=q+\frac{r}{b}\]
  \[\frac{a}{b}-q=\frac{r}{b}\]
  We want $q$ to be close to $\frac{a}{b}$ such that:
  \[N(\frac{a}{b}-q)=N(\frac{r}{b})<1\]
  so that we get the desired condition for $N(r)<N(b)$. So, let
  $\frac{a}{b}=n_1+n_2\w$ and $q=q_1+q_2\w$ and try the condition:
  \[\abs{n_1-q_1}\le\frac{1}{2}\ \mbox{and}\ \abs{n_2-q_2}\le\frac{1}{2}\]
  Now, calculate the resulting norm:
  \begin{eqnarray*}
    N(\frac{a}{b}-q) &=& N((n_1+n_2\w)-(q_1+q_2\w)) \\
    &=& N((n_1-q_1)+(n_2-q_2)\w) \\
    &=& (n_1-q_1)^2+(n_2-q_2)^2-(n_1-q_1)(n_2-q_2) \\
    &\le& \left(\frac{1}{2}\right)^2+\left(\frac{1}{2}\right)^2-
    \left(\frac{1}{2}\right)\left(-\frac{1}{2}\right) \\
    &=& \frac{1}{4}+\frac{1}{4}+\frac{1}{4} \\
    &=& \frac{3}{4} \\
    &<& 1
  \end{eqnarray*}
  Thus resulting in the desired condition.

  Therefore, $\Z[\w]$ is a Euclidean domain under the norm function.
\newpage
\item Let $R$ be a Euclidean domain with a nice Euclidean function
  $d:R^*\to\N_0$. Show that if $a\mid b$ and $d(a)=d(b)$ for some $a,b\in R$
  then $a$ and $b$ are associates.

  Since $a$ divides $b$, let $\frac{a}{b}=q\in R$. Now, since $d$ is
  multiplicative:
  \[d\left(\frac{a}{b}\right)=\frac{d(a)}{d(b)}=1\]
  So $d(q)=1$ and thus $q$ is a unit in $R$. So:
  \[a=qb\ \mbox{and}\ b=q^{-1}a\]
  Therefore $a$ and $b$ are associates.
\end{enumerate}

\end{document}
