\documentclass[letterpaper,12pt,fleqn]{article}
\usepackage{matharticle}
\pagestyle{plain}
\begin{document}
Cavallaro, Jeffery \\
Math 221b \\
Homework \#3

\begin{enumerate}
\item Let $R$ be a ring and let $I$ and $J$ be ideals in $R$:
  \begin{enumerate}
  \item Prove: $I\cap J$ and $I+J$ are both ideals in $R$.

    From group theory, we already know that $I\cap J$ is an additive subgroup
    of $R$. Furthermore, since $R$ is an additive abelian group, $I\cap J$ is
    an additive abelian subgroup of $R$.

    Assume $a\in I\cap J$ \\
    $a\in I$ and $a\in J$ \\
    Assume $b\in R$ \\
    But $I$ is an ideal in $R$, so $ab,ba\in I$ \\
    Similarly, $ab,ba\in J$ \\
    So $ab,ba\in I\cap J$

    Therefore, by the ideal test, $I\cap J$ is an ideal in $R$

    From group theory, we know that $I+J=I\vee J$ (join) when either subgroup is
    normal in $R$. But since $R$ is abelian, all subgroups are normal.
    Therefore $I+J$ is an additive subgroup of $R$. Furthermore, since $R$ is
    an additive abelian group, $I+J$ is an additive abelian subgroup of $R$.

    Now, assume $a\in I+J$ \\
    By definition, there exists $i\in I$ and $j\in J$ such that $a=i+j$ \\
    Assume $b\in R$ \\
    $ab=(i+j)b=ib+jb$ \\
    But $I$ is an ideal, so $ib\in I$ \\
    Similarly, $jb\in J$ \\
    Thus,  $ab=ib+jb\in I+J$

    $ba=b(i+j)=bi+bj$ \\
    But $I$ is an ideal, so $bi\in I$ \\
    Similarly, $bj\in J$ \\
    Thus $ba=bi+bj\in I+J$

    Therefore, by the ideal test, $I+J$ is an ideal in $R$.

  \item Prove that there is an isomorphism of rings:
    \[I/(I\cap J)\simeq(I+J)/J\]

    From part (a) we know that $I+J$ is a ring \\
    Since $0\in I$ we have $J\subseteq I+J$ \\
    $J$ is a ring and is thus an additive abelian subgroup of $I+J$

    Assume $a\in J$ \\
    Assume $b\in I+J$ \\
    There exists $i\in I$ and $j\in J$ such that $b=i+j$ \\
    $ab=a(i+j)=ai+aj$ \\
    But $J$ is an ideal in $R$, so $ai,aj\in J$ \\
    So by closure, $ab=ai+aj\in J$

    $ba=(i+j)a=ia+ja$ \\
    But $J$ is an ideal in $R$, so $ia,ja\in J$ \\
    So by closure, $ba=ia+ja\in J$

    Thus, by the ideal test, $J$ is an ideal in $I+J$, and therefore $(I+J)/J$
    is a factor ring.

    Now, consider $\phi:I\to(I+J)/J$ defined by $\phi(i)=i+J$.

    Assume $i,i'\in I$ \\
    $\phi(i+i')=(i+i')+J=(i+J)+(i'+J)=\phi(i)+\phi(i')$ \\
    $\phi(ii')=(ii')+J=(i+J)(i'+J)=\phi(i)\phi(i')$

    Therefore $\phi$ is a ring homomorphism.

    Now, assume $a\in(I+J)/J$ \\
    There exists $b\in(I+J)$ such that $a=b+J$ \\
    But, there exists $i\in I$ and $j\in J$ such that $b=i+j$ \\
    So, $a=(i+j)+J$ \\
    Now, since $J$ is the additive identity for $(I+J)/J$: \\
    $\phi(i)=i+J=(i+J)+J$ \\
    And since $j\in J$: \\
    $\phi(i)=(i+J)+(j+J)=(i+j)+J$

    Therefore, $\phi$ is surjective.

    Now, consider $i\in I$ such that $\phi(i)=i+J=J$ \\
    This means that $i\in J$ as well, so $\ker(\phi)=I\cap J$

    Therefore, by the first fundamental theorem:
    \[I/(I\cap J)\simeq(I+J)/J\]
  \end{enumerate}
  
\item Let $R$ be a commutative ring with $1\ne0$ and suppose $S$ is a
  multiplicatively-closed subset of $R\setminus\{0\}$ containing no zero
  divisors. Define $\sim$ on $R\times S$ by $(a,b)\sim(c,d)\iff ad=bc$.
  \begin{enumerate}
  \item Prove: $\sim$ is an equivalence relation.
    \begin{description}
    \item R: Assume $(a,b)\in R\times S$

      $ab=ab$ and so, by definition, $(a,b)\sim(a,b)$

      Therefore $\sim$ is reflexive.

    \item S: Assume $(a,b)\sim(c,d)$

      $ad=bc$ \\
      But $R$ is commutative, so $da=cb$ \\
      Furthermore, equality is symmetric, so $cb=da$ \\
      Thus, by definition, $(c,d)\sim(a,b)$

      Therefore $\sim$ is symmetric.

    \item T: Assume $(a,b)\sim(c,d)$ and $(c,d)\sim(e,f)$

      $ad=bc$ and $cf=de$ \\
      $R$ is a ring, so using ring properties: \\
      $adcf=bcde$ \\
      $adcf-bcde=0$ \\
      $(af)(dc)-(be)(cd)=0$ \\
      $(af)(cd)-(be)(cd)=0$ \\
      $(af-be)(cd)=0$ \\
      But since $R$ has no zero-divisors: $af-be=0$ or $cd=0$
      \begin{description}
      \item Case 1: $cd=0$

        By construction, $d\ne 0$, and so $c=0$ \\
        $ad=b0=0$, and thus $a=0$ \\
        Similarly, $0f=de=0$, and thus $e=0$ \\
        So $af=0f=0$ and $be=b0=0$ \\
        $af=be$ \\
        Thus, by definition, $(a,b)\sim(e,f)$

      \item Case 2: $af-be=0$

        $af=be$ \\
        Thus, by definition, $(a,b)=(e,f)$

      \end{description}

      Therefore, $\sim$ is transitive.
    \end{description}

    Therefore, $\sim$ is an equivalence relation.

  \item Let $R_S$ denote the set of equivalence classes $\frac{a}{b}$ of
    $(a,b)$. Prove that addition in $R_S$:
    \[\frac{a}{b}+\frac{c}{d}=\frac{ad+bc}{bd}\]
    is well-defined. (Given: multiplication is well-defined)

    Assume $(a,b)$ and $(c,d)\in R_S$ \\
    Assume $(a,b)\sim(a',b')$ and $(c,d)\sim(c',d')$ \\
    By definition: $ab'=ba'$ and $cd'=dc'$

    Consider $(1,1)\in R_S$:
    \[(a,b)(1,1)=(a1,b1)=(a,b)\]
    \[(1,1)(a,b)=(1a,1b)=(a,b)\]
    Thus $(1,1)$ is a multiplicative identity for $R_S$
 
    By construction, $b,d\in S$ are non-zero and $S$ has no zero divisors, so
    $bd\ne0$ \\
    Thus, $\frac{bd}{bd}\in R_S$ \\
    Furthermore: $(bd)1=1(bd)$, so $(bd,bd)\sim(1,1)$ \\
    Similarly: $(b'd',b'd')\sim(1,1)$

    Adding the two alternate representatives we get:
    \[\frac{a'}{b'}+\frac{c'}{d'}=\frac{a'd'+b'c'}{b'd'}\]
    Since multiplication is assumed to be well-defined:
    \[\frac{a'}{b'}+\frac{c'}{d'}=\frac{1}{1}\cdot\frac{a'd'+b'c'}{b'd'}=
    \frac{bd}{bd}\cdot\frac{a'd'+b'c'}{b'd'}=
    \frac{(bd)(a'd'+b'c')}{(bd)(b'd')}\]
    But $R$ is a commutative ring, so using ring properties:
    \begin{eqnarray*}
      \frac{a'}{b'}+\frac{c'}{d'} &=& \frac{bda'd'+bdb'c'}{bdb'd'} \\
      &=& \frac{(ba')(dd')+(dc')(bb')}{(b'd')(bd)} \\
      &=& \frac{(ab')(dd')+(cd')(bb')}{(b'd')(bd)} \\
      &=& \frac{(b'd')(ad)+(b'd')(bc)}{(b'd')(bd)} \\
      &=& \frac{(b'd')(ad+bc)}{(b'd')(bd)} \\
      &=& \frac{b'd'}{b'd'}\cdot\frac{ad+bc}{db} \\
      &=& \frac{1}{1}\cdot\frac{ad+bc}{db} \\
      &=& \frac{ad+bc}{db}
    \end{eqnarray*}

    Therefore, addition in $R_S$ is well-defined.

    \newcommand{\Zp}{\Z_{(p)}}

  \item Prove that there are exactly two prime ideals in $\Zp$: one
    corresponding to the zero ideal and one corresponding to the prime $p$.

    We know that the ideals of $\Zp$ are of the form $p\Zp$ and that they form
    a chain:
    \[\{0\}\subset\ldots\subset p^3\Zp\subset p^2\Zp\subset p\Zp\subset\Zp\]

    Assume $x,y\in\Zp$

    Since $\Zp$ is an integral domain and thus has no zero divisors:
    \[xy=0\implies x=0\ \mbox{or}\ y=0\]
    Thus, the zero ideal is prime.

    Now, assume $k\in\Z^+$
    \begin{description}
    \item Case 1: $k>1$:

      $p^{k-1}\in p^{k-1}\Zp$ \\
      $p\in p\Zp$ \\
      $p^{k-1}p=p^k\in p^k\Zp$ \\
      But $p^{k-1},p\notin p^k$

      Therefore $p^k\Zp$ is not prime.

    \item Case 2: $k=1$

      Let $x=\frac{a}{b}$ and $y=\frac{c}{d}$ \\
      Since $p$ divides neither $b$ nor $d$, $xy\in p\Zp$ means that one of $x,y$ must
      be in $p\Zp$ and the other must be in $\Zp$. Otherwise, $xy$ would fall in one of
      the other ideals.

      Therefore, $p\Zp$ is prime.
    \end{description}
  \end{enumerate}
\end{enumerate}

\end{document}
