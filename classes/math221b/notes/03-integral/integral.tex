\documentclass[letterpaper,12pt,fleqn]{article}
\usepackage{matharticle}
\pagestyle{empty}
\begin{document}
\section*{Integral Domain}

\begin{definition}[Zero Divisor]
  Let $R$ be a ring and $r,s\in\R^*$ such that $r,s\ne0$ and  $rs=0$. $r$ is
  called a \emph{left zero divisor} of $s$ and $s$ is called a
  \emph{right zero divisor} of $r$.
\end{definition}

\begin{example}
  \listbreak
  \begin{enumerate}
  \item $\Z\times\Z$

    $(0,a)(b,0)=(0,0)$

  \item $\Z_6$

    $2\cdot3=0$

  \item $M_2(Z)$

    $\begin{bmatrix} 1 & -2 \\ -2 & 4 \end{bmatrix}
    \begin{bmatrix} 6 & 2 \\ 3 & 1 \end{bmatrix}=
    \begin{bmatrix} 0 & 0 \\ 0 & 0 \end{bmatrix}$
  \end{enumerate}
\end{example}

\begin{definition}[Integral Domain]
  Let $R$ be a commutative ring with $1\ne0$. To say that $R$ is an
  \emph{integral domain} means that $R$ has no zero divisors.
\end{definition}

\begin{theorem}
  Let $R$ be a commutative ring with $1\ne0$. $R$ is an integral domain iff
  the cancellation laws hold.
\end{theorem}

\begin{theproof}
  \listbreak
  \begin{description}
  \item $\implies$ Assume $R$ is an integral domain

    Assume $rs=rt$ for $r,s,t\in R$ and $r\ne0$ \\
    $rs-rt=0$ \\
    $r(s-t)=0$ \\
    But $r\ne0$ by assumption, so $s-t=0$ and $s=t$

    Therefore, the left cancellation law holds.

    Similarly, $sr=tr$ \\
    $sr-tr=0$ \\
    $(s-t)r=0$ \\
    and thus $s=t$

    Therefore the right cancellation law holds.
\newpage
  \item $\impliedby$ Assume that the cancellation laws hold

    Assume $r,s\in R$ such that $r\ne0$ and $rs=0$ \\
    $r0=0$ \\
    $rs=r0$ \\
    So by left cancellation, $s=0$

    Therefore $R$ contains no left zero divisors.

    Similarly, assume $t\in R$ such that $tr=0$ \\
    $0r=0$ \\
    $tr=0r$ \\
    So by right cancellation, $t=0$
    
    Therefore $R$ contains no right zero divisors.

    Therefore $R$ is an integral domain.
  \end{description}
\end{theproof}

\begin{example}
  \listbreak
  \begin{enumerate}
  \item $\Z$, $\Q$, $\R$, $\C$
  \item $\Z[x]$
  \item $\Z[x,y]$
  \item $\Z[i]$
  \item $\Z[\omega]$
  \end{enumerate}
\end{example}

Note that $M_n(R)$ is not an integral domain due to lack of multiplicative
commutativity.

\begin{definition}[Field]
  Let $F$ be an integral domain. To say that $F$ is a field means:
  \[F^{\times}=F^*\]
  In other words, every non-zero element in $F$ is a unit.
\end{definition}

\begin{theorem}
  Let $F$ be a finite integral domain. $F$ is a field.
\end{theorem}

\begin{theproof}
  By definition, $F$ is a commutative ring with unity $1\ne0$

  Assume $a\in F,a\ne0$ \\
  Let $L_a:F\to F$ be defined by $L_a(x)=ax$
\newpage
  Assume $L_a(x)=L_a(y)$ \\
  $ax=ay$ \\
  But $F$ is an integral domain, so the cancellation laws hold \\
  $x=y$ \\
  $\therefore L_a$ is one-to-one.

  But $F$ is finite, so $L_a$ is also onto \\
  $\therefore L_a$ is a bijection on $F$.

  $1\in F$ \\
  $\exists\,x\in F,L_a(x)=1$ \\
  $ax=1$ \\
  But $F$ is commutative so $xa=1$ \\
  So $x$ is a multiplicative inverse for $a$ \\
  Thus every non-zero element of $F$ has a multiplicative inverse

  $\therefore F$ is a field.
\end{theproof}

\end{document}
