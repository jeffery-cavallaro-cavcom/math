\documentclass[letterpaper,12pt,fleqn]{article}
\usepackage{matharticle}
\pagestyle{empty}
\newcommand{\vp}{\varphi}
\renewcommand{\a}{\alpha}
\renewcommand{\i}{\iota}
\DeclareMathOperator{\Aut}{Aut}
\DeclareMathOperator{\id}{id}
\begin{document}
\section*{Automorphism Groups}

\begin{definition}[Automorphism]
  Given a field extension $K/F$, to say that $\vp$ is an \emph{F-automorphism}
  of $K/F$ means that $\vp:K\to K$ is a bijective ring homomorphism such that
  $\forall\,\a\in F,\vp(\a)=\a$.

  In other words, $\vp$ fixes (acts trivially on) $F$ --- it may fix more
  that $F$, but at least $F$ is guaranteed.
\end{definition}

\begin{theorem}
  Let $G=G(F)=\Aut(K/F)$ be the set of $F$-automorphisms of $K/F$:

  $G$ is a group under the operation of function composition.
\end{theorem}

\begin{theproof}
  Assume $\vp_1,\vp_2\in G$ \\
  Assume $\a\in F$ \\
  $\vp_1(\a)=\a$ and $\vp_2(\a)=\a$ \\
  $(\vp_1\vp_2)(\a)=\vp_1(\vp_2(\a))=\vp_1(\a)=\a$ \\
  So $\vp_1\vp_2\in G$

  Therefore $G$ is closed under the operation.

  Function composition is associative.

  Assume $\vp\in G$ and $\a\in F$ \\
  $\i_K(\a)=\a$, so $\i_K\in G$ \\
  $\i_K\vp=\vp\i_K=\vp$

  Therefore $G$ has identity $\i_K$.

  Assume $\vp\in G$ and $\a\in F$ \\
  $\vp(\a)=\a$ \\
  $\vp$ is bijective, so $\vp^{-1}$ exists \\
  $\vp^{-1}(\a)=\vp^{-1}(\vp(\a))=(\vp^{-1}\vp)(\a)=\i_K(\a)=\a$ \\
  So $\vp^{-1}\in G$

  Therefore $G$ is closed under inverses.

  Therefore $G$ is a group under the operation of function composition.
\end{theproof}

\begin{theorem}
  Let $F\subseteq L\subseteq K$ be an inclusion of fields:
  \[G(L)=\Aut(K/L)\le\Aut(K/F)=G(F)\]
\end{theorem}
\newpage
\begin{theproof}
  Assume $\a\in L$ \\
  $\i_L(\a)=\a$ \\
  $\i_L\in G(L)$

  $\therefore G(L)\ne\emptyset$

  Assume $\vp\in G(L)$ \\
  Assume $\a\in F$, and thus $\a\in L$ \\
  $\vp(\a)=\a$, so $\vp\in G(F)$

  $\therefore G(L)\subseteq G(F)$

  Assume $\vp_1,\vp_2\in G(L)$ \\
  Assume $\a\in L$
  \begin{eqnarray*}
    (\vp_1\vp_2^{-1})(\a) &=& \vp_1(\vp_2^{-1}(\a)) \\
    &=& \vp_1(\vp_2^{-1}(\vp_2(\a))) \\
    &=& \vp_1((\vp_2^{-1}\vp_2)(\a))) \\
    &=& \vp_1(\i_L(\a)) \\
    &=& \vp_1(\a) \\
    &=& \a
  \end{eqnarray*}
  So $\vp_1\vp^{-1}\in G(L)$

  Therefore, by the subgroup test, $G(L)\le G(F)$.
\end{theproof}

The result is a so-called ``reverse inclusion'':

\bigskip

\begin{tikzpicture}
  \node (F) at (0,0) {$F$};
  \node (L) at (0,1) {$L$};
  \node (K) at (0,2) {$K$};
  \draw (F) to (L);
  \draw (L) to (K);
  \node (GF) at (2,0) {} node at (GF) [right] {$G(F)=\Aut(K/F)$};
  \node (GL) at (2,1) {} node at (GL) [right] {$G(L)=\Aut(K/L)$};
  \node (GK) at (2,2) {} node at (GK) [right] {$G(K)=\Aut(K/K)=\{\id\}$};
  \draw [dashed] (F) to (GF);
  \draw [dashed] (L) to (GL);
  \draw [dashed] (K) to (GK);
  \node at (2.5,0.5) {\rotatebox{-90}{$\subseteq$}};
  \node at (2.5,1.5) {\rotatebox{-90}{$\subseteq$}};
\end{tikzpicture}

Note that the larger the field, the smaller the group. This makes sense
because if there are more group elements then the number of automorphisms that
fix all of the group elements is less.

\end{document}
