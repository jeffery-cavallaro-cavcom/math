\documentclass[letterpaper,12pt,fleqn]{article}
\usepackage{matharticle}
\pagestyle{empty}
\renewcommand{\a}{\alpha}
\newcommand{\w}{\omega}
\begin{document}
\section*{Algebraic Numbers}

\begin{definition}[Algebraic Number]
  To say that $\a\in\C$ is an \emph{algebraic number} means that it is the
  zero of some monic polynomial with rational coefficients:
  \[f(x)=\sum_{k=0}^na_kx^k\]
  where $a_k\in\Q$ and $f(\a)=0$.

  Otherwise, $\a$ is called a \emph{transcendental number}.
\end{definition}

\begin{theorem}
  Every algebraic number has a unique minimal monic polynomial, which is a
  polynomial of minimal degree that divides all other polynomials with
  rational coefficients that have $\a$ as a zero.
\end{theorem}

\begin{example}
  \begin{tabular}{c|c}
    $\a$ & $f(x)$ \\
    \hline
    $r\in\Q$ & $x-r$ \\
    $i$ & $x^2+1$ \\
    $\w$ & $x^2+x+1$ \\
    $\sqrt[3]{2}$ & $x^3-2$ \\
    $\frac{1}{\sqrt[3]{2}}$ & $x^3-\frac{1}{2}$ \\
  \end{tabular}

  Transcendental: $\pi$, $e$, $e^{\pi}$
\end{example}

\begin{theorem}
  $\Q[x]$ is a PID.
\end{theorem}

\begin{theorem}
  Let $\a\in\C$. The set of polynomials for which $\a$ is a zero is an ideal in
  $\Q[x]$.
\end{theorem}

\begin{theorem}
  Let $\overline{\Q}$ be the set of algebraic numbers:

  $\overline{\Q}$ is a field.
\end{theorem}

Thus, sums and products of algebraic numbers are also algebraic.

\end{document}
