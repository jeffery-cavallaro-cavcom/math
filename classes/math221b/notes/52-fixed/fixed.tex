\documentclass[letterpaper,12pt,fleqn]{article}
\usepackage{matharticle}
\pagestyle{empty}
\renewcommand{\a}{\alpha}
\newcommand{\vp}{\varphi}
\DeclareMathOperator{\Aut}{Aut}
\DeclareMathOperator{\id}{id}
\begin{document}
\section*{Fixed Fields}

\begin{definition}[Fixed Field]
  Let $K/F$ be a field extension and let $H\le\Aut(K/F)$. The \emph{fixed field} of $H$,
  denoted $F(H)$, is given by:
  \[F(H)=\{\a\in K\mid\forall\,\vp\in H,\vp(\a)=\a\}\]
\end{definition}

\begin{tikzpicture}
  \node (F) at (0,0) {$F$};
  \node (K) at (0,1) {$K$};
  \draw (F) to (K);
  \node [right] (GF) at (2,0) {$G(F)$};
  \node [right] (GK) at (2,1) {$G(K)$};
  \draw [dashed] (F) to (GF);
  \draw [dashed] (K) to (GK);
  \node [right] (FGF) at (4,0) {$F(G(F))\supseteq F$};
  \node [right] (FGK) at (4,1) {$F(\{\id\})=K$};
  \draw [dashed] (GF) to (FGF);
  \draw [dashed] (GK) to (FGK);
  \node at (2.5,0.5) {\rotatebox{-90}{$\subseteq$}};
  \node at (4.5,0.5) {\rotatebox{90}{$\subseteq$}};
\end{tikzpicture}

\begin{example}
  Recall that for $K=\Q(\sqrt[3]{2})$ and $F=Q$, $G(F)$ is trivial because $K$ is
  not a splitting field for $x^3-2$. Thus:
  \[F(G(F))=K\supset F\]
\end{example}

\begin{definition}
  To say that a field extension $K/F$ is \emph{Galois} means:
  \[F(G(F))=F\]
  There is no slippage - $G(F)$ only fixes $F$ and nothing else.
\end{definition}

\begin{example}[Quadratic Extensions]
  Let $[K:\Q]=2$ \\
  Assume $\a\in K$:
  \[m_{\a,\Q}(x)=x^2+bx+c\]
  for some $b,c\in\Q$.
  \[x=\frac{-b\pm\sqrt{b^2-4c}}{2}\]
  If $\sqrt{b^2-4c}\in\Q$ then $[K:\Q]=1$, so assume not. \\
  Let $b=\frac{p}{q}$ and $c=\frac{h}{k}$ where $p,q,h,k\in\Z$ and $q,k\ne0$:
  \begin{eqnarray*}
    x &=& \frac{-\frac{p}{q}\pm\sqrt{\left(\frac{p}{q}\right)^2-\frac{4h}{k}}}{2} \\
    &=& -\frac{p}{2q}\pm\frac{1}{2}\sqrt{\frac{p^2k-4qh}{q^2k}} \\
    &=& -\frac{p}{2q}\pm\frac{1}{2q^2k}\sqrt{q^2k(p^2k-4qh)}
  \end{eqnarray*}
  Note that $q^2k(p^2k-4qh)\in\Z$, so factor out any perfect square part, calling it
  $n^2$, and whatever squarefree integer is left call it $d$:
  \[x=-\frac{p}{2q}\pm\frac{1}{2q^2k}\sqrt{n^2d}=
  -\frac{p}{2q}\pm\frac{n}{2q^2k}\sqrt{d}\]
  Now let $r=-\frac{p}{2q}\in\Q$ and $s=\frac{n}{2q^2k}\in\Q$:
  \[x=r\pm s\sqrt{d}\]
  And so $K=\Q(\sqrt{d})$

  Now assume $\vp\in G(\Q)$ \\
  Since $\vp$ is a ring homomorphism that fixes $\Q$:
  \[\vp(x)=\phi(r\pm s\sqrt{d})=\phi(r)\pm\phi(s\sqrt{d})=r\pm\phi(s)\phi(\sqrt{d})=
  r\pm s\phi(\sqrt{d})\]
  And so $\vp$ is completely determined by what it does to $\sqrt{d}$.
  
  Thus, there are only two $\Q$-automorphisms:
  \begin{enumerate}
  \item $\id$
  \item $\sqrt{d}\mapsto -\sqrt{d}$
  \end{enumerate}
  In other words, the identity and a two-cycle.

  Therefore, $\Aut(\Q(\sqrt{d})/\Q)\cong \Z/2\Z$

  Also note that since $\vp$ only moves $\pm\sqrt{d}$:
  \[F(G(\Q))=\Q\]
  Thus, there there are no proper subfields of $L$ such that
  $\Q\subset L\subset \Q(\sqrt{d})$.
\end{example}

\end{document}
