\documentclass[letterpaper,12pt,fleqn]{article}
\usepackage{matharticle}
\pagestyle{empty}
\newcommand{\x}{\times}
\begin{document}
\section*{Polynomial Rings}

\begin{definition}[Polynomial Ring]
  Let $R$ be a commutative ring with $1\ne0$. The \emph{ring of polynomials}, denoted
  $R[x]$, is given by:
  \[R[x]=\left\{\sum_{k=0}^na_kx^k\mid n\in\N_0\ \mbox{and}\ a_k\in R\right\}\]
  In other words, $R[x]$ consists of polynomials with coefficients from $R$.

  $a_0$ is called the \emph{constant} coefficient/term.

  $a_nx^n$ for largest $n$ such that $a_n\ne0$ is called the \emph{leading} term and
  $a_n$ is called the \emph{leading} coefficient.
\end{definition}

\begin{theorem}
  Let $R$ be a ring. $R[x]$ is a ring under the standard definitions of polynomial
  addition and multiplication:
  \[\sum_{k=0}^na_kx^k+\sum_{k=0}^nb_kx^k=\sum_{k=0}^n(a_k+b_l)x^k\]
  \[\left(\sum_{k=0}^na_kx^k\right)\left(\sum_{k=0}^nb_kx^k\right)=\sum_{k=0}^nc_kx^k,
  \hspace{2ex}c_k=\sum_{j=0}^ka_kb_{k-j}\]
  Addition is component-wise and multiplication is based on the distributive property.
\end{theorem}

\begin{theproof}
  Addition is component-wise and is thus based on the additive properties of $R$. Thus,
  $R[x]$ is an additive abelian group. Likewise, multiplication is based on the
  associative and distributive properties of $R$. Therefore, $R[x]$ is a ring.
\end{theproof}

\begin{definition}[Degree]
  Let $R[x]$ be a polynomial ring over a ring $R$. The \emph{degree} function:
  \[\deg:R[x]\to\N_0\cup\{-\infty\}\]
  is defined by:
  \[\deg(f(x))=\begin{cases}
  -\infty & f(x)\equiv0 \\
  n & a_nx^n\ \mbox{is the leading term of}\ f(x)
  \end{cases}\]
\end{definition}
\newpage
\begin{definition}[Equality]
  Let $R$ be a ring and $f(x),g(x)\in R[x]$ where:
  \[f(x)=\sum_{k=0}^na_kx^k\]
  \[g(x)=\sum_{k=0}^mb_kx^k\]
  To say that $f(x)=g(x)$ means $m=n$ and $a_k=b_k$ for all $0\le k\le n$.
\end{definition}

\begin{properties}[Polynomial Rings]
  Assume $R$ is an integral domain:
  \begin{enumerate}
  \item $R[x]$ is an integral domain.
  \item $\deg(f(x)g(x))=\deg(f(x))+\deg(g(x))$
  \item $R[x]^{\x}=R^{\x}$
  \end{enumerate}
\end{properties}

\begin{theorem}[Division Algorithm]
  Let $R$ be an integral domain and $f(x),g(x)\in R[x]$ such that $g(x)\ne0$.
  There exists $k\in\N_0$ and $q(x),r(x)\in R[x]$ such that:
  \begin{enumerate}
  \item $b^kf(x)=q(x)g(x)+r(x)$
  \item $deg(r(x))<\deg(g(x))$
  \item $b$ is the leading coefficient of $g(x)$
  \end{enumerate}
  Note that if $b$ is a unit then we can take $k=0$.

  It is OK to select the minimum $k$ that works.

  For a fixed $k$, $q(x)$ and $r(x)$ are unique.
\end{theorem}

\begin{example}
  Let $R=\Z$ and:
  \begin{eqnarray*}
    f(x) &=& 2x^2+1 \\
    g(x) &=& 3x-1
  \end{eqnarray*}

  $3^k(2x^2+1)=(ax+b)(3x-1)+c$ \\
  $3^k\cdot2x^2+3^k=3ax^2+(3b-a)x+(c-b)$

  $3a=2\cdot3^k$ \\
  $3b-a=0$ \\
  $c-b=3^k$

  $a=2\cdot3^{k-1}$, so $k\ne0$ \\
  $b=2\cdot3^{k-2}$, so $k\ne1$ \\
  $c=3^k+2\cdot3^{k-2}$, so $k\ge 2$

  For $k=2: a=6, b=2, c=11$:

  $3^2(2x^2+1)=(6x+2)(3x-1)+11$

  Note that $\deg(3x-1)=1$ and $\deg(11)=0$, and indeed: $0\le0<1$.
\end{example}

\begin{theproof}
  Let $\deg(f(x))=m$ and $\deg(g(x))=n$
  
  If $m<n$ then we can take $k=0, q(x)\equiv0$, and $r(x)=f(x)$:
  \[b^0f(x)=0\cdot g(x)+f(x)\]
  So, AWLOG: $m\ge n\ge 0$

  Let $a$ be the leading coefficient for $f(x)$

  Proof by induction on $m$ for a given $n$
  \begin{description}
  \item Base: $m=0$

    Since $m\ge n$ it must be the case that $n=0$ \\
    $f(x)=a$ and $g(x)=b$ \\
    $bf(x)=ba=ab+0=ag(x)+0$ and $-\infty<0$

  \item Assume the statement is true for $\deg(f(x))<m$.

  \item Consider $\deg(f(x))=m$

    $ax^m$ is the leading term of $f(x)$ \\
    $bx^n$ is the leading term of $g(x)$ \\
    Let $f_1(x)=bf(x)-a^{m-n}g(x)$ \\
    Consider the leading term of $f_1(x): bax^m-abx^m=0$ \\
    So $\deg(f_1(x))<m$ and thus by the inductive assumption, there exists
    $k_1\in N_0$ and $q_1(x),r_1(x)\in R[x]$ such that:
    \[b^{k_1}f_1(x)=q_1(x)g(x)+r_1(x)\]
    where $\deg(r_1(x))<\deg(g(x))=n$. Now:
    \begin{eqnarray*}
      b^{k_1}f_1(x) &=& b^{k_1+1}f(x)-b^{k_1}a^{m-n}g(x) \\
      b^{k_1+1}f(x) &=& b^{k_1}f_1(x)+b^{k_1}a^{m-n}g(x) \\
      &=& q_1(x)g(x)+r_1(x)+b^{k_1}a^{m-n}g(x) \\
      &=& [q_1(x)+b^{k_1}a^{m-n}]g(x)+r_1(x)
    \end{eqnarray*}
    Let $k=k_1+1$, $q(x)=q_1(x)+b^{k_1}a^{m-n}$, and $r_1(x)=r(x)$:
    \[b^kf(x)=q(x)g(x)+r(x)\hspace{2ex}\deg(r(x))<\deg(g(x))\]
  \end{description}
\end{theproof}

Note that if $b$ is a unit then:
\[f(x)=[b^{-k}q(x)]g(x)+b^{-k}r(x)\]
which does not affect the various degrees.

\begin{corollary}[Remainder Theorem]
  Let $R[x]$ be a ring of polynomials over a ring $R$, $f(x),g(x)\in R[x]$, and
  $g(x)=x-a$. The remainder upon division of $f(x)$ by $g(x)$ is the constant $f(a)$.
\end{corollary}

\begin{theproof}
  $f(x)=q(x)(x-a)+r$
  
  $f(a)=q(a)(a-a)+r=0+r=r$
\end{theproof}

We can also consider rings of multiple, independent, commuting variables:
\[R[x,y]=(R[x])[y]\]

\end{document}
