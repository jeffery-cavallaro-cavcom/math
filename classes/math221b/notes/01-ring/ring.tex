\documentclass[letterpaper,12pt,fleqn]{article}
\usepackage{matharticle}
\usepackage{stmaryrd}
\pagestyle{empty}
\renewcommand{\H}{\mathbb{H}}
\newcommand{\hs}{\hspace{2ex}}
\newcommand{\w}{\omega}
\begin{document}
\section*{Rings}

\begin{definition}[Ring]
  A \emph{ring} $R$ is a non-empty set equipped with the binary operations of
  addition ($+$) and multiplication ($\cdot$) such that:
  \begin{enumerate}
  \item $R$ is an additive abelian group.
  \item Multiplication is associative.
  \item The distributive laws hold. $\forall\,r,s,t\in R$:
    \begin{description}
    \item Left: $r(s+t)=rs+rt$
    \item Right: $(s+t)r=sr+tr$
    \end{description}
  \end{enumerate}

  A ring with \emph{unity} is a ring with multiplicative identity element
  $1\in\R$ such that $\forall\,r\in R$:
  \[1r=r1=r\]
  We normally assert that $1\ne0$ in order to exclude $R=\{0\}$.

  A commutative ring is a ring with commutative multiplication.
\end{definition}

\begin{example}
  \listbreak
  \begin{enumerate}
  \item $\Z$, $\Q$, $\R$, $\C$
  \item $R[x]$\hs(formal polynomial ring)
  \item $R[x,y]$\hs(algebraically independent, commuting variables)
  \item $R\llbracket x\rrbracket$\hs(formal power series ring)
  \item $M_n(R)$
  \item $\Z/n\Z$
  \item $\Z\times\Z$
  \item $\Z[i]=\{a+bi\mid a,b\in\Z\}$\hs(Gaussian Integers)
  \item $\Z[\w]=\{a+b\frac{-1+\sqrt{3}}{2}\mid a,b\in\Z\}$\hs(Eisenstein Integers)
  \end{enumerate}
\end{example}

\begin{definition}[Quaternion Group]
  The \emph{quaternion group}, denoted by $Q_8$, is given by:
  \[Q_8=\{\pm1,\pm i,\pm j,\pm k\}\]
  where $i^2=j^2=k^2=ijk=(-1)$
\end{definition}

Note that $ij=k$ and $ji=-k$, so $Q_8$ is not commutative.

\begin{definition}[Hamilton Ring of Quaternions]
  The \emph{Hamilton Ring of Quaternions}, denoted by $\H$, is given by:
  \[\H=\{a+ib+cj+dk\mid a,b,c,d\in\R\}\]
  Note that $\Z\subset\Q\subset\R\subset\C\subset\H$.
\end{definition}

\begin{theorem}
  Let $R$ be a ring. $\forall\,r\in R$:
  \[0r=r0=0\]
\end{theorem}

\begin{theproof}
  $0r=(0+0)r=0r+0r$ \\
  $0r=0r+0$ \\
  $0r+0r=0r+0$

  $\therefore 0r=0$

  $r0=r(0+0)=r0+r0$ \\
  $r0=r0+0$ \\
  $r0+r0=r0+0$

  $\therefore r0=0$
\end{theproof}

\begin{definition}[Subring]
  To say that $S$ is a \emph{subring} of a ring $R$, denoted $S\le R$, means
  that $S\subseteq R$ and $S$ is also a ring using the same operations as $R$.
\end{definition}

\begin{theorem}[Subring Test]
  Let $R$ be a ring and $S$ a non-empty subset of $R$.
  $S\le R\iff\forall\,x,y\in S$:
  \begin{enumerate}
  \item $x+(-y)\in S$
  \item $xy\in S$
  \end{enumerate}
\end{theorem}

\begin{theproof}
  Assume $x,y\in S$
  \begin{description}
  \item $\implies$ Assume $S\le R$

    $S$ is a ring, so $-y\in S$ \\
    By additive closure: $x+(-y)\in S$ \\
    By multiplicative closure: $xy\in S$
\newpage
  \item $\impliedby$ Assume that the two closure conditions hold

    Since $x+(-y)\in S$, by the subgroup test, $S$ is an additive subgroup of
    $R$. Moreover, $S$ inherits additive commutativity from $R$, so $S$ is an
    additive abelian subgroup of $R$.

    Since $xy\in S$, $S$ is closed under multiplication. Moreover, $S$ inherits
    multiplicative associativity and the distributive laws from $R$.

    $\therefore S\le R$
  \end{description}
\end{theproof}

\end{document}
