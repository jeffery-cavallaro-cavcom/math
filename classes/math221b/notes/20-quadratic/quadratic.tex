\documentclass[letterpaper,12pt,fleqn]{article}
\usepackage{matharticle}
\pagestyle{empty}
\newcommand{\sd}{\sqrt{d}}
\newcommand{\Qd}{\Q(\sd)}
\renewcommand{\a}{\alpha}
\renewcommand{\b}{\beta}
\begin{document}
\section*{Quadratic Number Fields}

\begin{definition}[Squarefree]
  To say that and integer $d\ne1$ is \emph{squarefree} means no prime $p$
  exists such that $p^2\mid d$.
\end{definition}

\begin{definition}[Quadratic Number Field]
  Let $d\ne1$ be a squarefree integer. The \emph{quadratic number field}
  associated to $d$ is given by:
  \[\Qd=\{r+s\sd\mid r,s\in\Q\}\]

  Note that since $d$ is squarefree it is irrational.

  When $d>0$ then $\Qd$ is said to be real.
  
  When $d<0$ then $\Qd$ is said to be imaginary.
\end{definition}

Note that $\Qd=\Q[\sqrt{d}]$.

\begin{theorem}
  $\Qd$ is a field.
\end{theorem}

\begin{theproof}
  Assume $\a,\b\in\Qd$ \\
  Let $\a=r+s\sd$ and $\b=u+v\sd$ for $r,s,u,v\in\Q$

  $\a+\b=(r+s\sd)+(u+v\sd)=(r+u)+(s+v)\sd\in\Qd$

  $\a\b=(r+s\sd)(u+v\sd)=(ru+svd)+(rv+su)\sd\in\Qd$

  So, $\Qd$ is closed under the operations and is thus a subring of $\C$, and
  thus an integral domain.

  Now, assume $\a\ne0$ \\
  $\frac{1}{\a}=\frac{1}{r+s\sd}=\frac{r-s\sd}{r^2-ds^2}=
  \frac{r}{r^2-ds^2}-\frac{s}{r^2-ds^2}\sd\in\Qd$
  
  So, $\Qd$ is closed under inverses

  Therefore $\Qd$ is a field.
\end{theproof}

\end{document}
