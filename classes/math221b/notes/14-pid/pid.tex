\documentclass[letterpaper,12pt,fleqn]{article}
\usepackage{matharticle}
\pagestyle{empty}
\newcommand{\ide}{\trianglelefteq}
\begin{document}
\section*{Principle Ideal Domain}

\begin{definition}[PID]
  Let $R$ be an integral domain. To say that $R$ is a
  \emph{principal ideal domain} means that every $I\ide R$ is ideal - there
  exists $a\in R$ such that:
  \[I=(a)=aR\]
\end{definition}

\begin{example}
  The following are PIDs:
  \begin{enumerate}
  \item $\Z$
  \item $\Z[i]$
  \item $\Z[\omega]$
  \item $F[x]$ for a field $F$
  \end{enumerate}

  The following are not PIDs:
  \begin{enumerate}
  \item $\Z[x]$ because $(2,x)$ is not principal
  \item $F[x,y]$ because $(x,y)$ is not principal
  \item $\Z[\sqrt{-5}]$ because it contains irreducibles that are not prime
  \end{enumerate}
\end{example}

\begin{lemma}
  Let $R$ be a PID and $a,b\in R$. $(a,b)\in R$ and $\exists\,x,y\in R$ such
  that:
  \[(a,b)=xa+yb\]
  In other words, the GCD can be represented as a linear combination of $a$ and
  $b$ in $R$.
\end{lemma}

\begin{theproof}
  Since $R$ is a PID, $(a,b)=(a)+(b)$ should still be principal \\
  So $\exists\,d\in R$ such that $(a,b)=(d)$ \\
  $(d)=(a)+(b)\supseteq(a)$, and so $d\mid a$ (to contain is to divide) \\
  Likewise, $d\mid b$ \\
  Thus, $d$ is a common divisor of $a$ and $b$

  Assume $c$ is some other common divisor of $a$ and $b$ \\
  $(c)\supseteq(a)$ and $(c)\supseteq(b)$ \\
  But $(d)$ is the smallest ideal containing $(a,b)$
  
  So $(c)\supseteq(a,b)=(d)$ and therefore $c\mid d$.

  Moreover, since $(d)=(a)+(b)$, there exists $x,y\in R$ such that
  $d=xa+yb$.
\end{theproof}

\begin{theorem}
  Let $R$ be a PID and $p\in R$:
  \[p\ \mbox{irreducible}\implies p\ \mbox{prime}\]
\end{theorem}

\begin{theproof}
  Assume $p$ is irreducible \\
  Assume $p\mid ab$ for some $a,b\in R$ \\
  AWLOG: $p\nmid a$ \\
  Since $p$ is irreducible, the only divisors of $p$ are associates of $p$ and
  $1$, and hence the only common divisors of $p$ and $a$ are units, which are
  associates of $1$ \\
  In particular, $(p,a)=1$ \\
  So, by the previous lemma, $\exists\,x,y\in R$ such that $1=ax+py$ \\
  $b=bax+bpy=abx+pby$ \\
  But $p\mid ab$ and $p\mid pby$, so $p\mid b$

  Therefore, $p$ is prime.
\end{theproof}

Thus, is a PID, prime and irreducible are the same thing.

\end{document}
