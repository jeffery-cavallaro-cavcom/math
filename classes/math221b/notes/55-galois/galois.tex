\documentclass[letterpaper,12pt,fleqn]{article}
\usepackage{matharticle}
\pagestyle{empty}
\renewcommand{\a}{\alpha}
\newcommand{\p}{\varphi}
\newcommand{\s}{\psi}
\newcommand{\rest}[2]{\left.#1\right|_{#2}}
\newcommand{\n}{\trianglelefteq}
\DeclareMathOperator{\Aut}{Aut}
\DeclareMathOperator{\Gal}{Gal}
\DeclareMathOperator{\id}{id}
\begin{document}
\section*{Fundamental Theorem of Galois Theory}

\begin{theorem}
  Let $K/F$ be a field extension. There exists an inclusion-reversing
  bijection between all closed intermediate fields in $K/F$ and all closed
  subgroups of $\Aut(K/F)$.
\end{theorem}

$\begin{array}{ccc}
  K & \mapsto & \id \\
  | & & \rotatebox[origin=c]{-90}{$\le$} \\
  L & \mapsto & G(L) \\
  | & & \rotatebox[origin=c]{-90}{$\le$} \\
  E & \mapsto & G(E) \\
  | & & \rotatebox[origin=c]{-90}{$\le$} \\
  F & \mapsto & G(F)
\end{array}$

\begin{theproof}
  Consider the bijection given by:
  \[L\mapsto G(L)\]
  By the previous theorem behind the above diagram it is clear that this
  inclusion-reversing and G(L) is closed.

  Consider the inverse:
  \[H\mapsto F(H)\]
  By the previous theorem, $F(H)$ is also closed.

  Assume $L$ is closed. By composition:
  \[L\mapsto G(L) \mapsto F(G(L))=L\]
  Assume $H$ is closed. By composition:
  \[H\mapsto F(H)\mapsto G(F(H))=H\]
\end{theproof}

Note that if there is an intermediate field $L$ that is not closed then
$F(G(L))\supset L$ and the composition chain:
\[L\mapsto G(L)\mapsto F(G(L))\mapsto E\subset L\]
is broken.

\begin{theorem}
  Let $F\subseteq E\subseteq L\subseteq K$ be an inclusion of fields such that
  $[L:E]<\infty$:
  \[[G(E):G(L)]\le[L:E]\]
\end{theorem}

\begin{theproof}
  Proof by induction on $n=[L:E]$

  \begin{description}
  \item Base Case: $n=1$

    $L=E$ and $[G(E):G(L)]=[L:E]=1$

  \item Assume $[G(E):G(L)]\le[L:E]$ for extension of degree $<n$

  \item Assume $[L:E]=n$

    \begin{description}
    \item Case 1: There exists a proper extension $M$ such that
      $E\subset M\subset L$

      $[G(E):G(L)]=[G(E):G(M)][G(M):G(L)]\le[E:M][M:L]=[E:L]$

    \item Case 2: No such $M$ exists

      Assume $\a\in L\setminus E$ \\
      $L=E(\a)$ \\
      Since $L/E$ is finite, $\a$ is algebraic \\
      Hence, $[L:E]=[E(\a):E]=\deg(m_{\a,E}(x))=n$ \\

      Assume $\p,\s\in G(E)$
      \begin{eqnarray*}
        \p G(L)=\s G(L) &\iff& \p\s^{-1}\in G(L) \\
        &\iff& \rest{\p\s^{-1}}{L}=\id_L \\
        &\iff& (\p\s^{-1})(\a)=\a \\
        &\iff& \p(\a)=\s(\a)
      \end{eqnarray*}
      But $\p$ and $\s$ permute the roots of $m_{\a,E}(x)$, so $[G(E):G(L)]$ is
      the number of distinct roots of $m_{\a,E}(x)$ which equals $n$
    \end{description}

    $\therefore[G(E):G(L)]=[L:E]$
  \end{description}
\end{theproof}

Similarly:

\begin{theorem}
  Let $K/F$ be an extension of fields and $G=\Aut(K/F)$ with subgroups
  $1\le J\le H\le G$ such that $[H:J]<\infty$:
  \[[F(J):F(H)]\le[H:J]\]
\end{theorem}

\begin{theorem}
  Let $F\subseteq E\subseteq L\subseteq L$ be an inclusion of fields such that
  $E$ is closed and $[L:E]<\infty$:
  \[L\ \mbox{is closed and}\ [G(E):G(L)]=[L:E]\]
\end{theorem}

\newpage

\begin{theproof}
  Since $E$ is closed:
  \[[L:E]=[L:F(G(E))]\le[F(G(L)):F(G(E))]\le[G(E):G(L)]\le[L:E]\]
  Therefore $L=F(G(L))$ and so $L$ is closed, and $[G(E):G(L)]=[L:E]$.
\end{theproof}

Similarly:

\begin{theorem}
  Let $K/F$ be an extension of fields and $G=\Aut(K/F)$ with subgroups
  $1\le J\le H\le G$ such that $J$ is closed and $[H:J]<\infty$:
  \[H\ \mbox{is closed and}\ [F(J):F(H)]=[H:J]\]
\end{theorem}

\begin{theorem}
  Let $F\subseteq L\subseteq K$ be an inclusion of groups such that $L$ is
  stable:
  \[G/G(L)\cong G(L/F)\]
\end{theorem}

\begin{theproof}
  Consider the homomorphism from $G$ to $G(L/F)$ given by:
  \[\p\mapsto\rest{\p}{L}\]
  which is well-defined because $L$ is stable \\
  The kernel of this homomorphism is $G(L)$ \\
  Thus, by the FIT, $G/G(L)$ is isomorphic to some subgroup of $G(L/F)$ \\
  \[\abs{G/G(L)}=[G:G(L)]=[L:F]=\abs{G(L/F)}\]
  Thus, since the extensions are finite, the homomorphism is an isomorphism

  $\therefore G/G(L)\cong G(L/F)$
\end{theproof}

\begin{notation}
  When $K/F$ is Galois then $G=\Aut(K/F)=\Gal(K/F)$
\end{notation}

\begin{theorem}[Fundamental Theorem of Galois Theory]
  Let $K/F$ be a Galois extension with $G=\Gal(K/F)$:
  \begin{enumerate}
  \item There exists a bijection between intermediate field $L$ and subgroups
    of $G$.
  \item For $F\subseteq E\subseteq L\subseteq K$, $[L:E]=[G(E),G(L)]$.
  \item For $1\le J\le H\le G$, $[H:J]=[F(J):F(H)]$
  \item $H\n G\iff L=F(H)$, in which case $G/G(L)\cong G(L/F)$.
  \end{enumerate}
\end{theorem}

\end{document}
