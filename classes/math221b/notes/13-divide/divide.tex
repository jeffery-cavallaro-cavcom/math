\documentclass[letterpaper,12pt,fleqn]{article}
\usepackage{matharticle}
\pagestyle{empty}
\newcommand{\x}{\times}
\begin{document}
\section*{Divisibility}

\begin{definition}[Divides]
  Let $R$ be an integral domain and $a,b\in R$. To say that $a$ \emph{divides} $b$,
  denoted $a\mid b$, means there exists $c\in R$ such that $b=ca$.
\end{definition}

\begin{definition}[Associate]
  To say that $a$ and $b$ are \emph{associates} means $a\mid b$ and $b\mid a$.
\end{definition}

\begin{theorem}
  Let $R$ be a ring and $a,b\in R$ be associates. $\exists\,u\in R^{\x}$ such that
  $b=ua$ and $a=u^{-1}b$.
\end{theorem}

\begin{theproof}
  $\exists\,c\in R,b=ca$ \\
  $\exists\,d\in R,a=db$ \\
  $b=ca=(cd)b$ \\
  So $cd=1$, and thus $c$ and $d$ are units in $R$ \\
  Let $c=u$ and $d=u^{-1}$
  
  $\therefore b=ua$ and $a=u^{-1}b$, where $u\in R^{\x}$.
\end{theproof}

\begin{definition}[Irreducible]
  Let $R$ be an integral domain and $r\in R$. To say that $r$ is \emph{irreducible} in
  $R$ mean $r$ is non-zero, is not a unit in $R$, and if $r=ab$ for $a,b\in R$ then
  either $a$ or $b$ is a unit in $R$. Such a factorization of $p$ is called
  \emph{trivial}.
\end{definition}

\begin{definition}[Prime]
  Let $R$ be an integral domain and $p\in R$. To say that $p$ is \emph{prime} in $R$
  means that $p$ is non-zero, $p$ is not a unit in $R$, and if $p\mid ab$ for $a,b\in R$
  then $p\mid a$ or $p\mid b$.

  Note that in $Z$, prime and irreducible are the same thing; however, this is not true
  in general.
\end{definition}

\begin{theorem}
  Let $R$ be an integral domain and $p\in R$:
  \[p\ \mbox{prime}\implies p\ \mbox{irreducible}\]
\end{theorem}

\begin{theproof}
  Assume $p$ is prime in $R$ \\
  Assume $p=ab$ for some $a,b\in R$ \\
  $p\mid p$, so $p\mid ab$, and thus $p\mid a$ or $p\mid b$ \\
  AWLOG: $p\mid a$ \\
  $\exists\,c\in R,a=cp=pc$ \\
  $p=ab=pc(b)=p(bc)$ \\
  So $bc=1$ and $b$ is a unit, and thus the factorization of $p$ is trivial

  Therefore $p$ is irreducible.
\end{theproof}

\begin{definition}[GCD]
  Let $R$ be an integral domain and $a,b\in R$. To say that $d\in R$ is a
  \emph{common divisor} of $a$ and $b$ means $d\mid a$ and $d\mid b$.

  To say that $d$ is a \emph{greatest common divisor} (GCD) of $a$ and $b$, denoted
  $(a,b)$ or $\gcd(a,b)$, means that $d$ is a divisor of $a$ and $b$, and every other
  divisor of $a$ and $b$ also divides $d$.

  Note that GCD is unique up to associates.
\end{definition}

\begin{example}
  $(12,30)=\pm6$, but $6$ and $-6$ are associates.
\end{example}

\end{document}
