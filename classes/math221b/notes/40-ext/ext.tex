\documentclass[letterpaper,12pt,fleqn]{article}
\usepackage{matharticle}
\pagestyle{empty}
\renewcommand{\a}{\alpha}
\renewcommand{\b}{\beta}
\begin{document}
\section*{Extension Fields}

\begin{definition}
  Let $F$ and $K$ be fields such that $F$ is a subring of $K$. $F$ is called a
  \emph{subfield} of $K$ and $K$ is called an \emph{extension field} of $F$.

  Note that $K$ (vectors) is a vector space over $F$ (scalars), called an
  \emph{F-vector space}, and denoted denoted $K/F$. A basis for $K/F$ is called an
  $\emph{F-basis}$ for $K$. The dimension of $K/F$ is denoted by $[K:F]=\dim_F(K)$ and
  represents the cardinality of an $F$-basis for $K$.

  If $[K:F]$ is finite then $K$ is called a finite extension of $F$.
\end{definition}

\begin{example}
  \begin{tabular}{cccl}
    $F$ & $K$ & $[F:K]$ & BASIS \\
    \hline
    $\Q$ & $\C$ & $\infty$ \\
    $\R$ & $\C$ & $2$ & $\{1,i\}$ \\
    $Q$ & $\Q(\sqrt{d})$ & $2$ & $\{1,\sqrt{d}\}$ \\
    $Q$ & $\Q(\sqrt[n]{d})$ & $n$ &
    $\{1,\sqrt[n]{d},\sqrt[n]{d^2}\ldots,\sqrt[n]{d^{n-1}}\}$
  \end{tabular}
\end{example}

\begin{notation}
  $K/F$ is sometimes denoted as follows:

  \begin{table}[h]
    \setlength{\leftskip}{1in}
    \begin{tabular}{c}
      $K$ \\
      \vline \\
      $F$
    \end{tabular}
  \end{table}

  The reason for this is that there may be a tree of subfields of interest:

  \begin{figure}[h]
    \setlength{\leftskip}{1in}
    \begin{tikzpicture}
      \node (K) at (2,3) {$K$};
      \node (F1) at (0,2) {$F_1$};
      \node (F2) at (2,2) {$F_2$};
      \node (F3) at (4,2) {$F_3$};
      \node (F4) at (3,1) {$F_4$};
      \node (F) at (2,0) {$F$};
      \draw (F) to (F1);
      \draw (F) to (F4);
      \draw (F4) to (F2);
      \draw (F4) to (F3);
      \draw (F1) to (K);
      \draw (F2) to (K);
      \draw (F3) to (K);
    \end{tikzpicture}
  \end{figure}
\end{notation}

\begin{definition}[Generated Extension]
  Let $K/F$ and $S\subseteq K$. The smallest subfield of $K$ containing both
  $F$ and $S$, denoted $F(S)$, is called the extension of $F$ \emph{generated}
  by $S$ and is the intersection of all extended fields $L$ of $F$ such that
  $S\subseteq L\subseteq K$.
\end{definition}

\begin{definition}[Simple Extension]
  Let $K/F$ and $\a\in K$. The field extension generated by $\{\a\}$, denoted
  $F(\a)$, is called the \emph{simple} field extension of $F$ generated by $\a$,
  and $\a$ is called a primitive element for $F(\a)/F$.

  Note that $F(\a)$ is the field of fractions for the ring $F[x]$ with
  polynomials evaluated at $\a$:
  \[F(\a)=
  \left\{\frac{f(\a)}{g(\a)}\mid f(x),g(x)\in F[x]\ \mbox{and}\ g(\a)\ne 0
  \right\}\]

  When the $\a$ is algebraic then $g(\a)$ can be eliminated by a technique
  such as rationalization. Thus, $\Q(\sqrt{d})=\Q[\sqrt{d}]$; however
  $\Q(\pi)\ne\Q[\pi]$ because $\frac{1}{\pi}\notin\Q[\pi]$.
\end{definition}

\begin{theorem}
  Let $K/L$ and $L/F$ be field extensions:
  \[[K:F]=[K:L][L:F]\]
  Furthermore, if $A$ is an $F$-basis for $L$ and $B$ is an $L$-basis for $K$
  then:
  \[AB=\{ab\mid a\in A\ \mbox{and}\ b\in B\}\]
  is an $F$-basis for $K$.
\end{theorem}

\begin{theproof}
  Let $n=[K:L]$ and $m=[L:F]$ \\
  Assume $c\in K$ \\
  $c=\sum_{i=1}^n\ell_ib_i$, where $\ell_i\in L$ and $b_i\in B$ \\
  But each $\ell_i$ can be written as $\ell_i=\sum_{j=1}^mf_ja_j$, where
  $f_j\in F$ and $a_j\in A$ \\
  So $c=\sum_{i=1}^n\left(\sum_{j=1}^mf_ja_j\right)b_i=\sum_{i,j}f_{ji}(a_jb_i)$

  Therefore $AB$ spans $K$.

  Now assume $\sum_{i,j}f_{ji}(a_jb_i)=0$ for some finite
  $\{a_ib_i\}\subseteq AB$ \\
  For a given $i$, let $\ell_i=\sum_jf_{ji}a_j$ \\
  $\sum_{i}\ell_ib_i=0$ \\
  But the $b_i$ are linearly independent and so each $\ell_i0$ \\
  So for each $i$, $\sum_jf_{ji}a_j=0$ \\
  But the $a_i$ are linearly independent and so each $f_{ji}=0$

  Therefore the $a_jb_i$ are linearly independent.

  Therefore $AB$ is an $F$-basis for $K$ and $[K:L][L:F]=[K:F]$.
\end{theproof}

\end{document}
