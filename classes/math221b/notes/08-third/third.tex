\documentclass[letterpaper,12pt,fleqn]{article}
\usepackage{matharticle}
\pagestyle{empty}
\newcommand{\ide}{\trianglelefteq}
\begin{document}
\section*{Third Ring Isomorphism Theorem}

\begin{definition}
  Let $R$ be a ring and $I,J\ide R$:
  \[I+J=\{i+j\mid i\in I\ \mbox{and}\ j\in J\}\]
\end{definition}

\begin{theorem}
  Let $R$ be a ring and $I,J\ide R$:
  \[I+J\ide R\]
  In fact, $I+J$ is the smallest ideal in $R$ containing $I$ and $J$.
\end{theorem}

\begin{theproof}
  From group theory, we know that $I+J=I\vee J$ (join) when either subgroup is
  normal in $R$. But since $R$ is an additive abelian group, all subgroups are
  normal. Therefore $I+J$ is an additive abelian subgroup of $R$.

  Now, assume $a\in I+J$ \\
  By definition, there exists $i\in I$ and $j\in J$ such that $a=i+j$ \\
  Assume $b\in R$ \\
  $ab=(i+j)b=ib+jb$ \\
  But $ib\in I$ and $jb\in J$ \\
  Thus, $ab\in I+J$

  Likewise, $ba=b(i+j)=bi+bj$ \\
  But $bi\in I$ and $bj\in J$ \\
  Thus, $ba\in I+J$
  Thus $ba\in I+J$

  Therefore, by the ideal test, $I+J$ is an ideal in $R$.
\end{theproof}

It has already been proven that any intersection of ideals of $R$ is also an
ideal of $R$.

\begin{definition}
  Let $R$ be a ring and $S\subseteq R$. The ideal:
  \[\bigcap_{\tiny\begin{array}{c}I\ide R \\ S\subseteq I\end{array}}I\]
  is called the ideal generated by $S$ and is the smallest ideal of $R$
  containing $S$.

  When $S=\{r\}$ for some $r\in R$ then the ideal generated by $r$, denoted
  $(r)$, is called a \emph{principal} ideal.
\end{definition}

\begin{properties}[Principle ideals]
  Let $R$ be a ring and $r_k\in R$:
  \begin{enumerate}
  \item If $R$ is commutative then $(r)=\{r\alpha\mid\alpha\in R\}$
  \item $(r_1,\ldots,r_n)=(r_1)+\cdots+(r_n)$
  \end{enumerate}
\end{properties}

\begin{theorem}[Third Ring Isomorphism Theorem]
  Let $R$ be a ring and $I,J\ide R$:
  \[(I+J)/I\simeq J/(I\cap J)\]
\end{theorem}

\begin{theproof}
  From the previous theorem: $I+J\ide R$ \\
  But $J\ide R$ and $j\subseteq I+J$, so $J\ide I+J$ \\
  Thus $(I+J)/J$ is a factor ring
  
  Now, consider $\phi:I\to(I+J)/J$ defined by $\phi(i)=i+J$.

  Assume $i,i'\in I$ \\
  $\phi(i+i')=(i+i')+J=(i+J)+(i'+J)=\phi(i)+\phi(i')$ \\
  $\phi(ii')=(ii')+J=(i+J)(i'+J)=\phi(i)\phi(i')$

  Therefore $\phi$ is a ring homomorphism.

  Now, assume $a\in(I+J)/J$ \\
  There exists $b\in(I+J)$ such that $a=b+J$ \\
  But, there exists $i\in I$ and $j\in J$ such that $b=i+j$ \\
  So, $a=(i+j)+J$ \\
  Now, since $J$ is the additive identity for $(I+J)/J$: \\
  $\phi(i)=i+J=(i+j)+J$ \\
  And since $j\in J$: \\
  $\phi(i)=(i+J)+(j+J)=(i+j)+J$

  Therefore, $\phi$ is surjective.

  Now, consider $i\in I$ such that $\phi(i)=i+J=J$ \\
  This means that $i\in J$ as well, so $\ker(\phi)=I\cap J$

  Therefore, by the first fundamental ring theorem:
  \[I/(I\cap J)\simeq(I+J)/J\]
\end{theproof}

\end{document}
