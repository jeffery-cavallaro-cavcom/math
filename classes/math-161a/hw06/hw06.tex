\documentclass[letterpaper,12pt,fleqn]{article}
\usepackage{matharticle}
\usepackage{tikz}
\usepackage{siunitx}
\pagestyle{plain}
\renewcommand{\l}{\lambda}
\renewcommand{\o}{\sigma}
\renewcommand{\O}{\Phi}
\renewcommand{\a}{\alpha}
\newcommand{\m}{\mu}
\DeclareMathOperator{\nord}{N}
\DeclareMathOperator{\expd}{Exp}
\DeclareMathOperator{\bind}{B}
\begin{document}
Cavallaro, Jeffery \\
Math 161A \\
Homework \#6

\bigskip

\section*{4.28}

Let \(Z\) be a standard normal random variable and calculate the following probabilities, drawing pictures wherever
appropriate.
\begin{enumerate}[label={\alph*)},start=5]
\item \(P(Z\le1.37)\)

  \begin{tikzpicture}
    \draw [<->] (-5,0) -- (5,0);
    \draw [domain=-5:5] plot ({\x},{2*exp(-(\x)^2/2)});
    \pgfmathsetmacro{\a}{2*exp(-(-5)^2/2)}
    \pgfmathsetmacro{\b}{2*exp(-1.37^2/2)}
    \node [below] at (0,0) {\(0\)};
    \node [below] at (1.37,0) {\(1.37\)};
    \draw [fill=lightgray]
    (-5,0) --
    (-5,{\a}) --
    plot [domain=-5:1.37] ({\x},{2*exp(-(\x)^2/2)}) --
    (1.37,0) --
    cycle;
    \draw [->] (0,0) -- (0,3);
  \end{tikzpicture}
  \[P(Z\le1.37)=\O(1.37)=0.9147\]
\item \(P(-1.75\le Z)\)

  \begin{tikzpicture}
    \draw [<->] (-5,0) -- (5,0);
    \draw [domain=-5:5] plot ({\x},{2*exp(-(\x)^2/2)});
    \pgfmathsetmacro{\a}{2*exp(-(5)^2/2)}
    \pgfmathsetmacro{\b}{2*exp(-(-1.75)^2/2)}
    \node [below] at (0,0) {\(0\)};
    \node [below] at (-1.75,0) {\(-1.75\)};
    \draw [fill=lightgray]
    (-1.75,0) --
    (-1.75,{\b}) --
    plot [domain=-1.75:5] ({\x},{2*exp(-(\x)^2/2)}) --
    (5,0) --
    cycle;
    \draw [->] (0,0) -- (0,3);
  \end{tikzpicture}
  \[P(-1.75\le Z)=1-\O(-1.75)=1-0.0401=0.9599\]
\item (not assigned)
\item \(P(1.37\le Z\le2.50)\)

  \begin{tikzpicture}
    \draw [<->] (-5,0) -- (5,0);
    \draw [domain=-5:5] plot ({\x},{2*exp(-(\x)^2/2)});
    \pgfmathsetmacro{\a}{2*exp(-(1.37)^2/2)}
    \pgfmathsetmacro{\b}{2*exp(-(2.50)^2/2)}
    \node [below] at (0,0) {\(0\)};
    \node [below] at (1.37,0) {\(1.37\)};
    \node [below] at (2.50,0) {\(2.50\)};
    \draw [fill=lightgray]
    (1.37,0) --
    (1.37,{\a}) --
    plot [domain=1.37:2.50] ({\x},{2*exp(-(\x)^2/2)}) --
    (2.50,0) --
    cycle;
    \draw [->] (0,0) -- (0,3);
  \end{tikzpicture}
  \[P(1.37\le Z\le2.50)=\O(2.50)=\O(1.37)=0.9938-0.9147=0.0791\]
\item (not assigned)
\item \(P(\abs{Z}\le2.50)\)

  \begin{tikzpicture}
    \draw [<->] (-5,0) -- (5,0);
    \draw [domain=-5:5] plot ({\x},{2*exp(-(\x)^2/2)});
    \pgfmathsetmacro{\a}{2*exp(-(-2.50)^2/2)}
    \pgfmathsetmacro{\b}{2*exp(-(2.50)^2/2)}
    \node [below] at (0,0) {\(0\)};
    \node [below] at (-2.50,0) {\(-2.50\)};
    \node [below] at (2.50,0) {\(2.50\)};
    \draw [fill=lightgray]
    (-2.50,0) --
    (-2.50,{\a}) --
    plot [domain=-2.50:2.50] ({\x},{2*exp(-(\x)^2/2)}) --
    (2.50,0) --
    cycle;
    \draw [->] (0,0) -- (0,3);
  \end{tikzpicture}
  \begin{align*}
    P(\abs{Z}\le2.50) &= P(-2.50\le Z\le2.50) \\
    &= 2P(0\le Z\le 2.50) \\
    &= 2(\O(2.50)-\O(0)) \\
    &= 2(0.9938-0.5000) \\
    &= 2(0.4938) \\
    &= 0.9876
  \end{align*}
\end{enumerate}

\section*{4.29}

In each case, determine the value of the constant \(c\) that makes the probability statement correct.
\begin{enumerate}[label={\alph*)}]
\item \(\O(c)=0.9838\)
  \[c=2.14\]
\item \(P(0\le Z\le c)=0.2910\)
  \begin{gather*}
    P(0\le Z\le c)=0.2910 \\
    \O(c)-\O(0)=0.2910 \\
    \O(c)-0.5000=0.2910 \\
    \O(c)=0.7910 \\
    c=0.81
  \end{gather*}
\item \(P(c\le Z)=0.1210\)
  \begin{gather*}
    P(c\le Z)=0.1210 \\
    1-\O(c)=0.1210 \\
    \O(c)=0.8790 \\
    c=1.17
  \end{gather*}
\item \(P(-c\le Z\le c)=0.6680\)
  \begin{gather*}
    P(-c\le Z\le c)=0.6680 \\
    2P(0\le Z\le c)=0.6680 \\
    P(0\le Z\le c)=0.3340 \\
    \O(c)-\O(0)=0.3340 \\
    \O(c)-0.5000=0.3340 \\
    \O(c)=0.8340 \\
    c=0.97
  \end{gather*}
\item \(P(c\le\abs{Z})=0.0160\)
  \begin{gather*}
    P(c\le\abs{Z})=0.0160 \\
    2P(Z\le(-c))=0.0160 \\
    2\O(-c)=0.0160 \\
    \O(-c)=0.0080 \\
    -c=-2.41 \\
    c=2.41
  \end{gather*}
\end{enumerate}

\section*{4.30}

Find the following percentiles for the standard normal distribution.  Interpolate where appropriate.
\begin{enumerate}[label={\alph*)}]
\item \(91^{st}\)
  \begin{gather*}
    \O(c)=0.9100 \\
    c=1.34
  \end{gather*}
\item \(9^{th}\)
  \begin{gather*}
    \O(c)=0.0900 \\
    c=-1.34
  \end{gather*}
\item \(75^{th}\)
  \begin{gather*}
    \O(c)=0.7500 \\
    c=0.675
  \end{gather*}
\item \(25^{th}\)
  \begin{gather*}
    \O(c)=0.2500 \\
    c=-0.675
  \end{gather*}
\item \(6^{th}\)
  \begin{gather*}
    \O(c)=0.0600 \\
    c=-1.555
  \end{gather*}
\end{enumerate}

\section*{4.31}

Determine \(z_{\a}\) for the following values of \(\a\).
\begin{enumerate}[label={\alph*)}]
\item \(\a=0.0055\)
  \begin{gather*}
    \O(z_{\a})=1-0.0055=0.9945 \\
    z_{\a}=2.54
  \end{gather*}
\item \(\a=0.0900\)
  \begin{gather*}
    \O(z_{\a})=1-0.0900=0.9100 \\
    z_{\a}=1.34
  \end{gather*}
\item \(\a=0.6630\)
  \begin{gather*}
    \O(z_{\a})=1-0.6630=0.3370 \\
    z_{\a}=-0.42
  \end{gather*}
\end{enumerate}

\section*{4.45}

A machine that produces ball bearings has initially been set so that the true average diameter of the bearings it produces is
0.500 in.  A bearing is acceptable if its diameter is within 0.004 in of this target value.  Suppose, however, that the
setting has changed during the course of production, so that the bearings have normally distributed diameters with mean value
0.499 in and standard deviation 0.002 in.  What percentage of the bearings produced will not be accepted?

Let \(X=\) diameter of ball bearing and normalize \(X\):
\begin{align*}
  P(\text{rejected}) &= 1-P(0.496\le X\le5.004) \\
  &= 1-P\left(\frac{0.496-0.499}{0.002}\le Z\le\frac{0.504-0.499}{0.002}\right) \\
  &= 1-P(-1.5\le Z\le2.5) \\
  &= 1-(\O(2.5)-\O(-1.5)) \\
  &= 1-(0.9938-0.0668) \\
  &= 1-0.9270 \\
  &= 0.0730 \\
  &= 7.3\%
\end{align*}

\section*{4.49}

Consider babies born in the ``normal'' range of 37--43 weeks gestation age.  Extensive data supports the assumption that for
such babies born in the United States, birth weight is normally distributed with mean 3432 g and standard deviation 482 g.
[The article ``Are Babies Normal?'' (The American Statistician, 1999:298--302) analyzed data from a particular year; for a
  sensible choice of class intervals, a histogram did not look all that normal, but after further investigations it was
  determined that this was due to some hospitals measuring weight in grams and others measuring to the nearest ounce and
  then converting to grams.  A modified choice of class intervals that allowed for this gave a histogram that was well
  described by a normal distribution.]
\begin{enumerate}[label={\alph*)}]
\item What is the probability that the birth weight of a randomly selected baby of this type exceeds 4000 g? Is between
  3000 and 4000 g?
  \begin{align*}
    P(4000<X) &= 1-P(X\le4000) \\
    &= 1-P\left(Z\le\frac{4000-3432}{482}\right) \\
    &= 1-P(Z\le1.18) \\
    &= 1-\O(1.18) \\
    &= 1-0.8810 \\
    &= 0.1190
  \end{align*}
  \begin{align*}
    P(3000\le X\le4000) &= P\left(\frac{3000-3432}{482}\le Z\le\frac{4000-3432}{482}\right) \\
    &= P(-0.90\le Z\le1.18) \\
    &= \O(1.18)-\O(-0.90) \\
    &= 0.8810-0.1841 \\
    &= 0.6969
  \end{align*}
\item What is the probability that the birth weight of a randomly selected baby of this type is either less than 2000 g or
  greater than 5000 g?
  \begin{align*}
    P(X<2000\ \text{or}\ X>5000) &= 1-P(2000\le X\le5000) \\
    &= 1-P\left(\frac{2000-3432}{482}\le Z\le\frac{5000-3432}{482}\right) \\
    &= 1-P(-2.97\le Z\le3.25) \\
    &= 1-(\O(3.25)-\O(-2.97)) \\
    &= 1-(0.9994-0.0015) \\
    &= 1-0.9979 \\
    &= 0.0021
  \end{align*}

\item What is the probability that the birth weight of a randomly selected baby of this type exceeds 7 lb?
  \[(\SI{7}{lb})\left(\frac{\SI{453.592}{g}}{\SI{1}{lb}}\right)\approx\SI{3175}{g}\]
  \begin{align*}
    P(3175<X) &= 1-P(X\le3175) \\
    &= 1-P\left(Z\le\frac{3175-3432}{482}\right) \\
    &= 1-P(Z\le-0.53) \\
    &= 1-\O(-0.53) \\
    &= 1-0.2981 \\
    &= 0.7019
  \end{align*}

\item How would you characterize the extreme 0.1\% of all birth weights?
  Since the 0.1\% is split equally at both ends, first find \(z_{0.0005}\):
  \[\O(z_{0.0005})=1-0.0005=0.9995\]
  In the table, the corresponding values range from 3.27 to 3.32, so take the average of the endpoints:
  \[z_{0.0005}=\frac{3.27+3.32}{2}=3.295\]
  Now find the two extremes:
  \begin{gather*}
    -3.295=\frac{x_1-3432}{482} \\
    -1588=x_1-3482 \\
    x_1=1844 \\
    \\
    3.295=\frac{x_2-3432}{482} \\
    1588=x_2-3432 \\
    x_2=5020
  \end{gather*}
  And so the extreme 0.1\% is characterized by \(X<\SI{1844}{g}\) or \(X>\SI{5020}{g}\).

\item If \(X\) is a random variable with a normal distribution and \(a\) is a numerical constant (\(a\ne0\)), then \(Y=aX\)
  also has a normal distribution.  Use this to determine the distribution of birth weight expressed in pounds (shape, mean,
  and standard deviation), and then recalculate the probability from part (c).  How does this compare with your previous
  answer?

  Consider the exponent of the exponential and replace \(x\) with \(ax\):
  \[\frac{(ax-\m)^2}{2\o^2}=\frac{a^2\left(x-\frac{\m}{a}\right)^2}{2\o^2}=\frac{\left(x-\frac{\m}{a}\right)^2}{
    2\left(\frac{\o}{a}\right)^2}\]
  and so:
  \begin{gather*}
    \m_Y=\frac{3432}{453.592}=7.566 \\
    \\
    \o_Y=\frac{482}{453.592}=1.063 \\
    \\
    Y=aX\sim\nord(7.566,1.063^2)
  \end{gather*}
  \begin{align*}
    P(7.000<Y) &= 1-P(Y\le7.000) \\
    &= 1-P\left(Z\le\frac{7.000-7.566}{1.063}\right) \\
    &= 1-P(Z\le-0.53) \\
    &= 1-\O(-0.53) \\
    &= 1-0.2981 \\
    &= 0.7019
  \end{align*}
  This matches the answer from part (c).
\end{enumerate}

\section*{4.50}

In response to concerns about nutritional contents of fast foods, McDonald's has announced that it will use a new cooking oil
for its french fries that will decrease substantially trans fatty acid levels and increase the amount of more beneficial
ployunsaturated fat.  The company claims that 97 out of 100 people cannot detect a difference in taste between the new and old
oils.  Assuming that this figure is correct (as a long-run proportion), what is the approximate probability that in a random
sample of 1000 individuals who have purchased fries at McDonald's:

Let \(X=\) number of people who can taste the difference:
\[X\sim\bind(1000,0.03)\]
which can be approximated by:
\[X\sim\nord(1000\cdot0.03,1000\cdot0.03\cdot0.97)=\nord(30,29.1)\]
\begin{enumerate}[label={\alph*)}]
\item At least 40 can taste the difference between the two oils?
  \begin{align*}
    P(40\le X) &= 1-P(X<40) \\
    &= 1-P(X\le39.5)\qquad\text{(with continuity correction)} \\
    &= 1-P\left(Z\le\frac{39.5-30}{\sqrt{29.1}}\right) \\
    &= 1-P(Z\le1.76) \\
    &= 1-\O(1.76) \\
    &= 1-0.9608 \\
    &= 0.0392
  \end{align*}
\item At most 5\% can taste the difference between the two oils?
  \begin{align*}
    P(X\le50) &= P(X\le50.5)\qquad\text{(with continuity correction)} \\
    &= P\left(Z\le\frac{50.5-30}{\sqrt{29.1}}\right) \\
    &= P(Z\le3.80) \\
    &= \O(3.80) \\
    &\approx1
  \end{align*}
\end{enumerate}

\section*{4.55}

Suppose that only 75\% of all drivers in a certain state regularly wear a seat belt.  A random sample of 500 drivers is
selected.  What is the probability that:

Let \(X=\) number of drivers who regularly wear seat belts.
\[X\sim\bind(500,0.75)\]
which can be approximated by:
\[X\sim\nord(500\cdot0.75,500\cdot0.75\cdot0.25)=\nord(375,93.75)\]
\begin{enumerate}[label={\alph*)}]
\item Between 360 and 400 (inclusive) of the drivers in the sample regularly wear a seat belt?
  \begin{align*}
    P(360\le X\le 400) &= P(359.5<X\le400.5)\qquad\text{(with continuity correction)} \\
    &= P\left(\frac{359.5-375}{\sqrt{93.75}}<Z\le\frac{400.5-375}{\sqrt{93.75}}\right) \\
    &= P(-1.60<Z\le2.63) \\
    &= \O(2.63)-\O(-1.60) \\
    &= 0.9957-0.0548 \\
    &= 0.9409
  \end{align*}
\item Fewer than 400 of those in the sample regularly wear a seat belt?
  \begin{align*}
    P(X<400) &= P(X\le399.5)\qquad\text{(with continuity correction)} \\
    &= P\left(Z\le\frac{399.5-375}{\sqrt{93.75}}\right) \\
    &= P(Z\le2.53) \\
    &= \O(2.53) \\
    &= 0.9943
  \end{align*}
\end{enumerate}

\section*{4.58}

There is no nice formula for the standard normal cdf \(\O(x)\), but several good approximations have been published in
articles.  The following is from ``Approximations for Hand Calculators Using Small Integer Coefficients'' (Mathematics
of Computation, 1977:214--222).  For \(0<z\le5.5\):
\[P(Z\ge z)=1-\O(z)\approx0.5\exp\left\{-\left[\frac{(83z+351)z+562}{\frac{703}{z}+165}\right]\right\}\]
The relative error of this approximation is less than 0.042\%.  Use this to calculate approximations to the following
probabilities, and compare whenever possible to the probabilities obtained from Appendix Table A.3.
\begin{enumerate}[label={\alph*)}]
\item \(P(Z\ge1)\)
  
  Formula:
  \[P(Z\ge1)=0.1587\]
  Table:
  \[P(Z\ge1)=P(Z\le-1)=0.1587\]
\item \(P(Z<-3)\)
  
  Formula:
  \[P(Z<-3)=P(Z\ge3)=0.0013\]
  Table:
  \[P(Z<-3)=0.0013\]
\item \(P(-4<Z<4)\)

  Formula:
  \[P(-4<Z<-4)=1-2P(Z\ge4)=1-2(0.000032)\approx1\]
  Table:
  \[P(-4<Z<4)\approx1\]
\item \(P(Z>5)\)

  Formula:
  \[P(Z>5)=0.00000029\approx0\]
  Table:
  \[P(Z>5)\approx0\]
\end{enumerate}

\section*{4.59}

Let \(X=\) the time between two successive arrivals at the drive-up window of a local bank.  If \(X\) has an exponential
distribution with \(\l=1\), compute the following:
\begin{enumerate}[label={\alph*)}]
\item The expected time between two successive arrivals.
  \[E(X)=\frac{1}{\l}=\frac{1}{1}=1\]
\item The standard deviation of the time between successive arrivals.
  \[\o=\frac{1}{\l}=\frac{1}{1}=1\]
\item \(P(X\le4\)
  \[P(X\le4)=F(4)=1-e^{-4}=0.9819\]
\item \(P(2\le X\le5)\)
  \[P(2\le X\le5)=F(5)-F(2)=e^{-2}-e^{-5}=0.1286\]
\end{enumerate}

\section*{4.69}

A system consists of five identical components connected in series as shown:

\bigskip

\begin{tikzpicture}[every node/.style={draw,minimum width=2cm,minimum height=1cm,fill=lightgray}]
  \node (1) at (2,0) {1};
  \node (2) at (5,0) {1};
  \node (3) at (8,0) {1};
  \node (4) at (11,0) {1};
  \node (5) at (14,0) {1};
  \draw (0,0) to (1) to (2) to (3) to (4) to (5) to (16,0);
\end{tikzpicture}

\bigskip

As soon as one component fails, the entire system will fail.  Suppose each component has a lifetime that is exponentially
distributed with \(\l=0.01\) and that components fail independently of one another.  Define events:
\[A_i=\setb{i^{th}\ \text{component lasts at least}\ t\ \text{hours}}{i=1,\ldots5}\]
so that the \(A_i\)s are independent events.  Let \(X=\) the time at which the system fails --- that is, the shortest (minimum)
lifetime among the five components.
\begin{enumerate}[label={\alph*)}]
\item The event \(\set{X\ge t}\) is equivalent to what event involving \(A_1,\ldots,A_5\)?
  \[\set{X\ge t}=\bigcap_{i=1}^5A_i\]
\item Using the independence of the \(A_i\)s, compute \(P(X\ge t)\).  Then obtain \(F(t)=P(X\ge t)\) and the pdf of \(X\).
  What type of distribution does \(X\) have?
  \begin{gather*}
    P(X\ge t)=\prod_{i=1}^5P(A_i)=\prod_{i=1}^5e^{-0.01t}=e^{-0.05t}=\bar{F}(t) \\
    F(t)=1-\bar{F}(t)=1-e^{-0.05t} \\
    \\
    f(x)=\begin{cases}
    0.05e^{-0.05x} & x\ge 0 \\
    0 & \text{otherwise}
    \end{cases} \\
    \\
    X\sim\expd(0.05)
  \end{gather*}
\item Suppose there are \(n\) components, each having exponential lifetime with parameter \(\l\).  What type of distribution
  does \(X\) have?
  \[X\sim\expd(n\l)\]
\end{enumerate}

\section*{4.70}

If \(X\) has an exponential distribution with parameter \(\l\), derive a general expression for the \((100p)^{th}\)
percentile of the distribution.  Then specialize to obtain the median.
\begin{gather*}
  p=F(x)=1-e^{-\l x} \\
  e^{-\l x}=1-p \\
  -\l x=\ln(1-p) \\
  x=-\frac{1}{\l}\ln(1-p) \\
  \\
  \text{median}=-\frac{1}{\l}\ln\left(1-\frac{1}{2}\right)=-\frac{1}{\l}\ln\left(\frac{1}{2}\right)=
  \frac{1}{\l}\ln(2) \\
\end{gather*}

\end{document}
