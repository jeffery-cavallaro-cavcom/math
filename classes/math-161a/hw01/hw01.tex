\documentclass[letterpaper,12pt,fleqn]{article}
\usepackage{matharticle}
\usepackage{tikz}
\usepackage{venndiagram}
\usepackage{siunitx}
\pagestyle{plain}
\renewcommand{\S}{\mathcal{S}}
\begin{document}
Cavallaro, Jeffery \\
Math 161A \\
Homework \#1

\bigskip

\section*{2.2}

Suppose that vehicles taking a particular freeway exit can turn right (\(R\)), turn left (\(L\)), or go straight (\(S\)).
Consider observing the direction for each of three successive vehicles.
\begin{enumerate}[label=\alph*)]
\item List all outcomes in the event \(A\) that all three vehicles go in the same direction.
  \[A=\set{RRR,LLL,SSS}\]
\item List all outcomes in the event \(B\) that all three vehicles take different directions.
  \[B=\set{RLS,RSL,LRS,LSR,SRL,SLR}\]
\item List all outcomes in the event \(C\) that exactly two of the three vehicles turn right.
  \[C=\set{RRL,RRS,RLR,RSR,LRR,SRR}\]
\item List all outcomes in the event \(D\) that exactly two vehicles go in the same direction.
  \begin{align*}
    D=\{ & RRL,RRS,RLR,RSR,LRR,SRR, \\
    & LLR,LLS,LRL,LSL,RLL,SLL, \\
    & SSR,SSL,SRS,SLS,RSS,LSS \}
  \end{align*}
\item List outcomes in \(D'\), \(C\cup D\), and \(C\cap D\).
  \begin{align*}
    D' &= \set{\text{all vehicles go in the same direction or different directions}} \\
    &= A\cup B \\
    &= \set{RLS,RSL,LRS,LSR,SRL,SLR,RRR,LLL,SSS}
  \end{align*}
  Note that \(C\subset D\):
  \begin{align*}
    C\cup D=D=\{ & RRL,RRS,RLR,RSR,LRR,SRR, \\
    & LLR,LLS,LRL,LSL,RLL,SLL, \\
    & SSR,SSL,SRS,SLS,RSS,LSS \}
  \end{align*}
  \[C\cap D=C=\set{RRL,RRS,RLR,RSR,LRR,SRR}\]
\end{enumerate}

\bigskip

\section*{2.3}

Three components are connected to form a system as shown in the accompanying diagram.  Because the components in the \(2-3\)
subsystem are connected in parallel, that subsystem will function if at least one of the two individual components function.
For the entire system to function, component \(1\) must function and so must the \(2-3\) subsystem.

\begin{tikzpicture}[every node/.style={draw,rectangle}]
  \node (1) at (1,0) {\(1\)};
  \node (2) at (4,1) {\(2\)};
  \node (3) at (4,-1) {\(3\)};
  \draw (0,0) to (1) to (2,0) to (3,1) to (2) to (5,1) to (6,0) to (7,0);
  \draw (2,0) to (3,-1) to (3) to (5,-1) to (6,0);
\end{tikzpicture}

The experiment consists of determining the condition of each component [\(S\) (success) for a functioning component and
  \(F\) (failure) for a nonfunctioning component].
\begin{enumerate}[label=\alph*)]
\item Which outcomes are contained in the event \(A\) that exactly two out of the three components function?
  \[A=\set{SSF,SFS,FSS}\]
\item Which outcomes are contained in the event \(B\) that at least two of the components function?
  \[B=A\cup\set{SSS}=\set{SSF,SFS,FSS,SSS}\]
\item Which outcomes are contained in event \(C\) that the system functions?
  \[C=\set{SSF,SFS,SSS}\]
\item List outcomes in \(C'\), \(A\cup C\), \(A\cap C\), \(B\cup C\), and \(B\cap C\).
  \[C'=\set{\text{the system does not function}}=\set{FFF,FFS,FSF,FSS,SFF}\]
  \[A\cup C=\set{SSF,SFS,FSS,SSS}\]
  \[A\cap C=\set{SSF,SFS}\]
  Note \(C\subset B\)
  \[B\cup C=B=\set{SSF,SFS,FSS,SSS}\]
  \[B\cap C=C=\set{SSF,SFS,SSS}\]
\end{enumerate}

\bigskip

\section*{2.6}

A college library has five copies of a certain text on reserve.  Two copies (\(1\) and \(2\)) are first printings, and the
other three (\(3\), \(4\), and \(5\)) are second printings.  A student examines these books in random order, stopping only
when a second printing has been selected.  One possible outcome is \(5\), and another is \(213\).
\begin{enumerate}[label=\alph*)]
\item List the outcomes in \(\S\).
  \[\S=\set{3,4,5,13,14,15,23,24,25,123,124,125,213,214,215}\]
\item Let \(A\) denote the event that exactly one book must be examined.  What outcomes are in \(A\)?
  \[A=\set{3,4,5}\]
\item Let \(B\) be the event that book \(5\) is the one selected.  What outcomes are in \(B\)?
  \[B=\set{5,15,25,125,215}\]
\item Let \(C\) be the event that book \(1\) is not examined.  What outcomes are in \(C\)?
  \[C=\set{3,4,5,23,24,25}\]
\end{enumerate}

\bigskip

\section*{2.12}

Consider randomly selecting a student at a large university, and let \(A\) be the event that the selected student has a
VISA card and \(B\) be the analagous event for MasterCard.  Suppose \(P(A)=0.6\) and \(P(B)=0.4\).
\begin{enumerate}[label=\alph*)]
\item Could it be the case that \(P(A\cap B)=0.5\)?  Why or why not?

  No, because:
  \begin{gather*}
    A\cap B\subseteq B \\
    P(A\cap B)\le P(B) \\
    0.5\not\le0.4
  \end{gather*}

\item From now on, suppose \(P(A\cap B)=0.3\).  What is the probability that the selected student has at least one
  of these cards?
  \[P(A\cup B)=P(A)+P(B)-P(A\cap B)=0.6+0.4-0.3=0.7\]

\item What is the probability that the selected student has neither type of card?
  \[P\left(A^C\cap B^C\right)=P\left((A\cup B)^C\right)=1-P(A\cup B)=1.0-0.7=0.3\]

\item Describe, in terms of \(A\) and \(B\), the event that the selected student has a VISA card but not a MasterCard, and
  then calculate the probability of this event.

  \begin{minipage}{2.25in}
    \begin{venndiagram2sets}
      \fillOnlyA
    \end{venndiagram2sets}
  \end{minipage}
  \begin{minipage}{3.5in}
    \(\displaystyle P\left(A\cap B^C\right)=P(A)-P(A\cap B)=0.6-0.3=0.3\)
  \end{minipage}

\item Calculate the probability that the selected student has exactly one of the two types of cards.
  
  \begin{minipage}{2.25in}
    \begin{venndiagram2sets}
      \fillOnlyA
      \fillOnlyB
    \end{venndiagram2sets}
  \end{minipage}
  \begin{minipage}{3.5in}
    \begin{gather*}
      P\left(A\cap B^C\right)=0.3 \\
      P\left(B\cap A^C\right)=P(B)-P(A\cap B)=0.4-0.3=0.1 \\
      P\left(\left(A\cap B^C\right)\cup\left(B\cap A^C\right)\right)=0.3+0.1=0.4
    \end{gather*}
  \end{minipage}
\end{enumerate}

\bigskip

\section*{2.16}

An individual is presented with three different glasses of cola, labeled \(C\), \(D\), and \(P\).  He is asked to taste all
three and then list them in order of precedence.  Suppose the same cola has actually been put into all three glasses.
\begin{enumerate}[label=\alph*)]
\item What are the simple events in this ranking experiment, and what probability would you assign to each?

  Since the three glasses are filled with the \emph{same} cola, we can assume that each outcome is a random permutation of
  the three choices:
  \[\S=\set{CDP,CPD,DCP,DPC,PCD,PDC}\]
  \[P(CDP)=P(CPD)=P(DCP)=P(DPC)=P(PCD)=P(PDC)=\frac{1}{6}\]

\item What is the probability that \(C\) is ranked first?
  \[P(Cxx)=\frac{2}{6}=\frac{1}{3}\]

\item What is the probability that \(C\) is ranked first and \(D\) is ranked last?
  \[P(CxD)=\frac{1}{6}\]
\end{enumerate}

\bigskip

\section*{2.26}

A certain system can experience three different types of defects.  Let \(A_i(i=1,2,3)\) denote the event that the system
has a defect of type \(i\). Suppose that
\begin{gather*}
  P(A_1)=0.12 \\
  P(A_2)=0.07 \\
  P(A_3)=0.05 \\
  P(A_1\cup A_2)=0.13 \\
  P(A_1\cup A_3)=0.14 \\
  P(A_2\cup A_3)=0.10 \\
  P(A_1\cap A_2\cap A_3)=0.01
\end{gather*}
\begin{enumerate}[label=\alph*)]
\item What is the probability that the system does not have a type 1 defect?

  \begin{minipage}{2.25in}
    \begin{venndiagram3sets}[labelA=\(A_1\),labelB=\(A_2\),labelC=\(A_3\)]
      \fillNotA
    \end{venndiagram3sets}
  \end{minipage}
  \begin{minipage}{3.5in}
    \(\displaystyle P\left(A_1^C\right)=1-P(A_1)=1-0.12=0.88\)
  \end{minipage}

\item What is the probability that the system has both type 1 and type 2 defects?

  \begin{minipage}{2.25in}
    \begin{venndiagram3sets}[labelA=\(A_1\),labelB=\(A_2\),labelC=\(A_3\)]
      \fillACapB
    \end{venndiagram3sets}
  \end{minipage}
  \begin{minipage}{3.5in}
    \begin{align*}
      P(A_1\cap A_2) &= P(A_1)+P(A_2)-P(A_1\cup A_2) \\
      &= 0.12+0.07-0.13 \\
      &= 0.06
    \end{align*}
  \end{minipage}

\item What is the probability that the system has both type 1 and type 2 defects but not a type 3 defect?
  
  \begin{minipage}{2.25in}
    \begin{venndiagram3sets}[labelA=\(A_1\),labelB=\(A_2\),labelC=\(A_3\)]
      \fillACapBNotC
    \end{venndiagram3sets}
  \end{minipage}
  \begin{minipage}{3.5in}
    \begin{align*}
      P\left(A_1\cap A_2\cap A_3^C\right) &= P(A_1\cap A_2)-P(A_1\cap A_2\cap A_3) \\
      &= 0.06-0.01 \\
      &= 0.05
    \end{align*}
  \end{minipage}

\item What is the probability that the system has at most two of these defects?
  
  \begin{minipage}{2.25in}
    \begin{venndiagram3sets}[labelA=\(A_1\),labelB=\(A_2\),labelC=\(A_3\)]
      \fillNotABC
      \fillOnlyA
      \fillOnlyB
      \fillOnlyC
      \fillACapBNotC
      \fillACapCNotB
      \fillBCapCNotA
    \end{venndiagram3sets}
  \end{minipage}
  \begin{minipage}{3.5in}
    \begin{align*}
      P\left(\left(A_1\cap A_2\cap A_3\right)^C\right) &= 1-P(A_1\cap A_2\cap A_3) \\
      &= 1-0.01 \\
      &= 0.99
    \end{align*}
  \end{minipage}
\end{enumerate}

\bigskip

\section*{2.31}

The composer Beethoven wrote 9 symphonies, 5 piano concertos (music for piano and orchestra), and 32 piano sonatas (music
for solo piano).
\begin{enumerate}[label=\alph*)]
\item How many ways are there to play first a Beethoven symphony and then a Beethoven piano concerto?
  \[9\cdot5=45\]

\item The manager of a radio station decides that on each successive evening (7 days per week), a Beethoven symphony will be
  played followed by a Beethoven piano concerto followed by a Beethoven piano sonata.  For how many years could this policy
  be continued before exactly the same program would have to be repeated?
  \[9\cdot5\cdot7=\SI{1440}{days}\sim\SI{4}{years}\]
\end{enumerate}

\bigskip

\section*{2.34}

Computer keyboard failures can be attributed to electrical defects or mechanical defects.  A repair facility currently has
25 failed keyboards, 6 of which have electrical defects and 19 of which have mechanical defects.
\begin{enumerate}[label=\alph*)]
\item How many ways are there to randomly select 5 of these keyboards for a thorough inspection (without regard to order)?
  \[\binom{25}{5}=53130\]

\item In how many ways can a sample of 5 keyboards be selected so that exactly two have an electrical defect?
  \[\binom{6}{2}\binom{19}{3}=15\cdot969=14535\]

\item If a sample of 5 keyboards is randomly selected, what is the probability that at least 4 of these will have a
  mechanical defect?
  \[\frac{\binom{19}{4}\binom{6}{1}+\binom{19}{5}\binom{6}{0}}{\binom{25}{5}}=
  \frac{3876\cdot6+11628\cdot1}{53130}\approx0.66\]
\end{enumerate}

\bigskip

\section*{2.36}

An academic department with five faculty members narrowed its choice for department head to either candidate \(A\) or
candidate \(B\).  Each member then voted on a slip of paper for one of the candidates.  Supposed there are actually three
votes for \(A\) and two for \(B\).  If the slips are selected for tallying in random order, what is the probability that
\(A\) remains ahead of \(B\) throughout the vote count (e.g., this event occurs if the selected ordering is \(AABAB\), but
not \(ABBAA\)?

For this problem, the three \(A\) votes are considered identical, as are the two \(B\) votes.  To determine the total
number of possible tallying sequences, place the three \(A\) votes in the five slots:
\[\binom{5}{3}=10\]
The only two outcomes for \(A\) always ahead are \(AAABB\) and \(AABAB\).  Thus, the probability that \(A\) remains ahead is:
\[\frac{2}{10}=\frac{1}{5}=0.2\]

\end{document}
