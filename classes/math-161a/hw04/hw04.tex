\documentclass[letterpaper,12pt,fleqn]{article}
\usepackage{matharticle}
\pagestyle{plain}
\DeclareMathOperator{\bin}{B}
\DeclareMathOperator{\hg}{HyperGeom}
\DeclareMathOperator{\nb}{NB}
\DeclareMathOperator{\pois}{Pois}
\renewcommand{\o}{\sigma}
\begin{document}
Cavallaro, Jeffery \\
Math 161A \\
Homework \#4

\bigskip

\section*{3.48}

NBC News reported on May 2, 2013 that 1 in 20 children in the United States have a food allergy of some sort.  Consider
selecting a random sample of 25 children and let \(X\) be the number in the sample who have a food allergy.  Then
\(X\sim B(25,0.05)\).

\begin{enumerate}[label={\alph*)}]
\item Determine both \(P(X\le3\) and \(P(X<3)\).
  
  From the table:
  \begin{gather*}
    P(X\le3)=0.966 \\
    P(X<3)=P(X\le2)=0.873
  \end{gather*}
  By hand:
  \begin{gather*}
    P(X=0)=\binom{25}{0}(0.05)^0(0.95)^{25}=0.277 \\
    P(X=1)=\binom{25}{1}(0.05)^1(0.95)^{24}=0.365 \\
    P(X=2)=\binom{25}{2}(0.05)^2(0.95)^{23}=0.231 \\
    P(X=3)=\binom{25}{3}(0.05)^3(0.95)^{22}=0.093 \\
    \\
    P(X<3)=0.277+0.365+0.231=0.873 \\
    P(X\le3)=0.873+0.093=0.966
  \end{gather*}

\item Determine \(P(X\ge4)\).
  \[P(X\ge4)=1-P(X\le3)=1-0.966=0.034\]

\item Determine \(P(1\le X\le 3)\).
  \[P(1\le X\le 3)=P(X\le3)-P(X=0)=0.966-0.277=0.693\]

\item What are \(E(X)\) and \(\o_X\)?
  \begin{gather*}
    E(X)=np=25\cdot0.05=1.25 \\
    V(X)=np(1-p)=25\cdot0.05\cdot0.95=1.1875 \\
    \o_X=\sqrt{1.1875}\approx1.090
  \end{gather*}

\item In a sample of 50 children, what is the probability that none has a food allergy?
  \begin{gather*}
    X\sim\bin(50,0.05) \\
    \\
    P(X=0)=0.95^{50}\approx0.077
  \end{gather*}
\end{enumerate}

\section*{3.60}

A toll bridge charges \$1.00 for passenger cars and \$2.50 for other vehicles.  Suppose that during the daytime hours, 60\%
of all vehicles are passenger cars.  If 25 vehicles cross the bridge during a particular daytime period, what is the resulting
expected toll revenue? [Hint: Let \(X=\) the number of passenger cars; then the toll revenue \(h(X)\) is a linear function of
  \(X\).]
\begin{gather*}
  X\sim\bin(25,0.6) \\
  \\
  h(X)=1.00X+2.50(25-X)=X+62.5-2.5X=62.5-1.5X
\end{gather*}
\begin{align*}
  E(h(X)) &= E(62.5-1.5X) \\
  &= 62.5-1.5E(X) \\
  &= 62.5-1.5np \\
  &= 62.5-1.5(25)(0.6) \\
  &= \$40.00
\end{align*}

\section*{3.66}

An airport limousine can accommodate up to four passengers on any one trip.  The company will accept a maximum of six
reservations for a trip, and a passenger must have a reservation.  From previous records, 20\% of all those making reservations
do not appear for the trip.  Answer the following questions, assuming independence wherever appropriate.

\begin{enumerate}[label={\alph*)}]
\item If six reservations are made, what is the probability that at least one individual with a reservation cannot be
  accommodated on the trip?

  Let \(X=\) the number of passengers who do not show up.  Then \(X\sim\bin(6,0.2)\).
  \begin{gather*}
    P(X=0)=\binom{6}{0}0.2^00.8^{6-0}=0.8^6=0.262 \\
    P(X=1)=\binom{6}{1}0.2^10.8^{6-1}=6\cdot0.2\cdot0.8^5=0.393 \\
    \\
    P(X<2)=0.262+0.393=0.655
  \end{gather*}

\item If six reservations are made, what is the expected number of available places when the limousine departs?
  \[E(X)=np=6\cdot0.2=1.2\]
  On average, 1.2 people do not show up.  Thus, the limousine is always full (i.e., no available places).
\end{enumerate}

\section*{3.68}

Eighteen individuals are scheduled to take a driving test at a particular DMV office on a certain day, eight of whom will be
taking the test for the first time.  Suppose that six of these individuals are randomly assigned to a particular examiner,
and let \(X\) be the number among the six who are taking the test for the first time.

\begin{enumerate}[label={\alph*)}]
\item What kind of distribution does \(X\) have (name and values of all parameters)?
  \[X\sim\hg(18,8,6)\]
\item Compute \(P(X=2)\), \(P(X\le2)\), and \(P(X\ge2)\).
  \begin{gather*}
    P(X=x)=\frac{\binom{8}{x}\binom{10}{6-x}}{\binom{18}{6}} \\
    \\
    P(X=0)=\frac{\binom{8}{0}\binom{10}{6}}{\binom{18}{6}}=\frac{1\cdot210}{18564}=0.0113 \\
    P(X=1)=\frac{\binom{8}{1}\binom{10}{5}}{\binom{18}{6}}=\frac{8\cdot252}{18564}=0.1086 \\
    \\
    P(X=2)=\frac{\binom{8}{2}\binom{10}{4}}{\binom{18}{6}}=\frac{28\cdot210}{18564}=0.3167 \\
    \\
    P(X\le2)=P(X=0)+P(X=1)+P(X=2)=0.0113+0.1086+0.3167=0.4366 \\
    \\
    P(X\ge2)=1-P(X=0)-P(X=1)=1-0.0113-0.1086=0.8801
  \end{gather*}
\end{enumerate}

\section*{3.75}

The probability that a randomly selected box of a certain type of cereal has a particular prize is 0.2.  Suppose you purchase
box after box until you have obtained two of these prizes.

Let \(X=\) number of boxes purchased:
\[X\sim\nb(0.2,2)\]

\begin{enumerate}[label={\alph*)}]
\item\label{a} What is the probability that you purchase \(x\) boxes that do not contain the desired prize?

  I am interpreting this question as meaning that \(x+2\) total boxes are purchased.
  \[P(X=x+2)=\binom{(x+2)-1}{2-1}(0.2)^2(1-0.2)^{(x+2)-2}=\binom{x+1}{1}(0.04)(0.8)^x=0.04(x+1)(0.8)^x\]
\item What is the probability that you purchase four boxes?

  From part (\ref{a} with \(x=2\):
  \[P(X=4)=0.04(2+1)(0.8)^2=0.04(3)(0.64)=0.0768\]
\item What is the probability that you purchase at most four boxes?
  \begin{gather*}
    P(X=2)=0.04(0+1)(0.8)^0=0.04(1)(1)=0.04 \\
    P(X=3)=0.04(1+1)(0.8)^1=0.04(2)(0.8)=0.064 \\
    \\
    P(X\le4)=P(X=2)+P(X=3)+P(X=4)=0.04+0.064+0.0768=0.1808
  \end{gather*}
\item How many boxes without the desired prize do you expect to purchase?  How many boxes do you expect to purchase?
  \[E(X)=\frac{r}{p}=\frac{2}{0.2}=10\]
  Therefore, we would expect to purchase a total of 10 boxes, 8 of which do not have the desired prize.
\end{enumerate}

\section*{3.76}
A family decides to have children until it has three children of the same gender.  Assuming \(P(B)=P(G)=0.5\), what is the
pmf of \(X=\) the number of children in the family.

The possibilities are as follows:

\begin{tabular}{|c|c|c|}
  \hline
  BOYS & GIRLS & TOTAL \\
  \hline
  3 & 0 & 3 \\
  \hline
  3 & 1 & 4 \\
  \hline
  3 & 2 & 5 \\
  \hline
  0 & 3 & 3 \\
  \hline
  1 & 3 & 4 \\
  \hline
  2 & 3 & 5 \\
  \hline
\end{tabular}

Thus, the range for \(X\) is \(\set{3,4,5}\) and each value appears twice: once for boys and once for girls.  Since each
gender is \(\sim\nb(0.5,3)\):
\begin{gather*}
  f_X(x)=2\binom{x-1}{3-1}(0.5)^3(1-0.5)^{x-3}=2\binom{x-1}{2}(0.5)^3(0.5)^{x-3}=2\binom{x-1}{2}(0.5)^x \\
  \\
  f_X(3)=2\binom{3-1}{2}(0.5)^3=2\binom{2}{2}(0.5)^3=2(1)(0.125)=0.250 \\
  f_X(4)=2\binom{4-1}{2}(0.5)^4=2\binom{3}{2}(0.5)^4=2(3)(0.0625)=0.375 \\
  f_X(5)=2\binom{5-1}{2}(0.5)^5=2\binom{4}{2}(0.5)^5=2(6)(0.03125)=0.375 \\
\end{gather*}
And therefore:
\[f_X(x)=\begin{cases}
0.250 & x=3 \\
0.375 & x=4 \\
0.375 & x=5 \\
0 & \text{otherwise}
\end{cases}\]
As expected, the total probability \(=1\).

\section*{3.79}

The article \emph{``Expectation Analysis of the Probability of Failure for Water Supply Pipes''} (J. of Pipeline Systems
Engr. and Practice, May 2012: 36--46) proposed using the Poisson distribution to model the number of failures in pipelines
of various types.  Suppose that for cast-iron pipe of a particular length, the expected number of failures is 1 (very close
to one of the cases considered in the article).  Then \(X\), the number of failures, has a Poisson distribution with
\(\mu=1\).

\begin{enumerate}[label={\alph*)}]
\item Obtain \(P(X\le5\) by using Appendix Table A.2.
  \[P(X\le5)=0.999\]
\item Determine P(X=2) first from the pmf formula and then from Appendix Table A.2.
  \begin{gather*}
    P(X=2)=\frac{1^2}{2!}e^{-1}=\frac{1}{2e}=0.184 \\
    \\
    P(X=2)=P(X\le2)-P(X\le1)=0.920-0.736=0.184
  \end{gather*}
\item Determine \(P(2\le X\le4)\).
  \[P(2\le X\le4)=P(X\le4)-P(X\le1)=0.996-0.736=0.260\qquad\text{(from table)}\]
\item What is the probability that \(X\) exceeds its mean value by more than one standard deviation?

  Since \(\mu=1\) it is the case that \(E(X)=V(X)=1\) and \(\o_X=\sqrt{1}=1\). So, we want (from the table):
  \[P(X>2)=1-P(X\le2)=1-0.920=0.080\]
\end{enumerate}

\section*{3.83}

An article in the \emph{Los Angeles Times} (Dec. 3, 1993) reports that 1 in 200 people carry the defective gene that causes
inherited colon cancer.  In a sample of 1000 individuals, what is the approximate distribution of the number who carry this
gene?

Let \(X=\) the number of people who carry the gene.
\[X\sim\pois(\mu=1000\cdot0.005)=\pois(5)\]
Use this distribution to calculate the approximate probability that:
\begin{enumerate}[label={\alph*)}]
\item Between 5 and 8 (inclusive) carry the gene.

  From the table:
  \[P(5\le X\le8)=P(X\le8)-P(X\le4)=0.932-0.440=0.492\]

\item At least 8 carry the gene.
  
  From the table:
  \[P(X\ge8)=1-P(X\le7)=1-0.867=0.133\]
\end{enumerate}

\section*{3.84}

The \emph{Centers for Disease Control and Prevention} reported in 2012 that 1 in 88 American children had been diagnosed
with an autism spectrum disorder (ASD).

\begin{enumerate}[label={\alph*)}]
\item If a random sample of 200 American children is selected, what are the expected value and standard deviation of the
  number who have been diagnosed with ASD?
  \begin{gather*}
    \mu=200\cdot\frac{1}{88}\approx2.27 \\
    \\
    E(X)=\mu=2.27 \\
    \\
    V(X)=\mu=2.27 \\
    \o_X=\sqrt{2.27}\approx1.51
  \end{gather*}
\item Referring back to (a), calculate the approximate probability that at least 2 children in the sample have been
  diagnosed with ASD.
  \begin{gather*}
    P(X=0)=\frac{2.27^0}{0!}e^{-2.27}=e^{-2.27}=0.1033 \\
    P(X=1)=\frac{2.27^1}{1!}e^{-2.27}=2.27e^{-2.27}=0.2345 \\
    P(X=2)=\frac{2.27^2}{2!}e^{-2.27}=\frac{2.27^2}{2}e^{-2.27}=0.2662 \\
    \\
    P(X\le2)=P(X=0)+P(X=1)+P(X=2)=0.1033+0.2345+0.2662=0.6040
  \end{gather*}
\item If the sample size is 352, what is the approximate probability that fewer than 5 of the selected children have been
  diagnosed with ASD?
  \[\mu=352\cdot\frac{1}{88}=4\]
  From the table:
  \[P(X\le4)=0.440\]
\end{enumerate}

\end{document}
