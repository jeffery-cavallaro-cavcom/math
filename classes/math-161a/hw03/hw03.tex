\documentclass[letterpaper,12pt,fleqn]{article}
\usepackage{matharticle}
\usepackage{pgfplots}
\pgfplotsset{compat=1.16}
\usepackage{siunitx}
\pagestyle{plain}
\DeclareMathOperator{\range}{Range}
\tikzset{
  dot/.style={mark=*,solid},
  opendot/.append style={dot,fill=white},
  filldot/.append style={dot,fill=black}
}
\begin{document}
Cavallaro, Jeffery \\
Math 161A \\
Homework \#3

\bigskip

\section*{3.1}

A concrete beam may fail either from shear (\(S\)) or flexure (\(F\)).  Suppose that three failed beams are randomly selected
and the type of failure is determined for each one.  Let \(X=\) the number of beams among the three selected that failed by
shear.  List each outcome in the sample space along with the associated value of \(X\).

\bigskip

\begin{center}
  \begin{tabular}{|c|c|}
    \hline
    \(\omega\) & \(X(\omega)\) \\
    \hline
    SSS & 3 \\
    \hline
    SSF & 2 \\
    \hline
    SFS & 2 \\
    \hline
    SFF & 1 \\
    \hline
    FSS & 2 \\
    \hline
    FSF & 1 \\
    \hline
    FFS & 1 \\
    \hline
    FFF & 0 \\
    \hline
  \end{tabular}
\end{center}

\section*{3.6}

Starting at a fixed time, each car entering an intersection is observed to see whether it turns left (\(L\)), right (\(R\)),
or goes straight ahead (\(A\)).  The experiment terminates as soon as a car is observed to turn left.  Let \(X=\) the number
of cars observed.  What are the possible \(X\) values?
\[\range(X)=\mathbb{N}\]
List five outcomes and their associated \(X\) values.

\bigskip

\begin{center}
  \begin{tabular}{|c|c|}
    \hline
    \(\omega\) & \(X(\omega)\) \\
    \hline
    \(L\) & 1 \\
    \hline
    \(RL\) & 2 \\
    \hline
    \(AL\) & 2 \\
    \hline
    \(RRL\) & 3 \\
    \hline
    \(AAL\) & 3 \\
    \hline
  \end{tabular}
\end{center}

\section*{3.7}

For each random variable defined here, describe the set of possible values for the variable, and state whether the value is
discrete.

\begin{enumerate}[label={\alph*)}]
\item \(X=\) the number of unbroken eggs in a randomly chosen standard egg carton.
  \[\range(X)=\setb{n\in\mathbb{Z}}{0\le n\le12}\]
  Discrete

  \setcounter{enumi}{3}
\item \(X=\) the length of a randomly selected rattlesnake.
  
  Theoretically:
  \[\range(X)=(0,\infty)\]
  However, according to wikipedia, baby rattlesnakes are about 6 inches in length and the longest rattlesnake on record is
  8 feet and 3 inches, so more realistically:
  \[\range(X)=(5,100)\]
  in inches.

  Continuous, in either case.
\end{enumerate}

\section*{3.8}

Each time a component is tested, the trial is a success (\(S\)) or failure (\(F\)).  Suppose the component is tested
repeatedly until a success occurs on three \emph{consecutive} trials.  Let \(Y\) denote the number of trials necessary
to achieve this.  List all outcomes corresponding to the five smallest possible values of \(Y\), and state which \(Y\)
value is associated with each one.

\bigskip

\begin{center}
  \begin{tabular}{|c|c|}
    \hline
    \(\omega\) & \(Y(\omega)\) \\
    \hline
    SSS & 3 \\
    \hline
    FSSS & 4 \\
    \hline
    FFSSS & 5 \\
    \hline
    SFSSS & 5 \\
    \hline
    FFFSSS & 6 \\
    \hline
    SFFSSS & 6 \\
    \hline
    FSFSSS & 6 \\
    \hline
    SSFSSS & 6 \\
    \hline
    FFFFSSS & 7 \\
    \hline
    SFFFSSS & 7 \\
    \hline
    FSFFSSS & 7 \\
    \hline
    SSFFSSS & 7 \\
    \hline
    FFSFSSS & 7 \\
    \hline
    SFSFSSS & 7 \\
    \hline
    FSSFSSS & 7 \\
    \hline
  \end{tabular}
\end{center}

\section*{3.12}

Airlines sometimes overbook flights.  Suppose that for a plane with 50 seats, 55 passengers have tickets.  Define the random
variable \(Y\) as the number of ticketed passengers who actually show up for the flight.  The probability mass function of
\(Y\) appears in the accompanying table.

\begin{tabular}{c|ccccccccccc}
  \(y\) & 45 & 46 & 47 & 48 & 49 & 50 & 51 & 52 & 53 & 54 & 55 \\
  \hline
  \(p(y)\) & 0.05 & 0.10 & 0.12 & 0.14 & 0.25 & 0.17 & 0.06 & 0.05 & 0.03 & 0.02 & 0.01
\end{tabular}

\begin{enumerate}[label={\alph*)}]
\item What is the probability that the plight will accommodate all ticketed passengers who show up?
  \begin{align*}
    P(Y\le50) &= p(45)+p(46)+p(47)+p(48)+p(49)+p(50) \\
    &= 0.05+0.10+0.12+0.14+0.25+0.17 \\
    &= 0.83
  \end{align*}

\item What is the probability that not all ticketed passengers who show up can be accommodated?
  \[P(Y>50)=1-P(Y\le50)=1-0.83=0.17\]

\item If you are the first person on the standby list (which means you will be the first one to get on the plane if there
  are any seats available after all ticketed passengers have been accommodated), what is the probability that you will be
  able to get on the flight?
  \[P(Y<50)=P(Y\le50)-P(Y=50)=0.83-0.17=0.66\]
  What is this probability if you are the third person on the standby list?
  \[P(Y\le47)=p(45)+p(46)+p(47)=0.05+0.10+0.12=0.27\]
\end{enumerate}

\section*{3.18}

Two fair six-sided dice are tossed independently.  Let \(M=\) the maximum of the two tosses (so \(M(1,5)=5\), \(M(3,3)=3\),
etc.).
\begin{enumerate}[label={\alph*)}]
\item What is the pmf of \(M\)? [Hint: first determine \(p(1)\), the \(p(2)\), and so on.]
  \begin{align*}
    p(1) &= P(\set{(1,1)})=\frac{1}{36} \\
    p(2) &= P(\setb{(2,x)}{x=1,2})+P(\set{(1,2)})=\frac{1}{6}\cdot\frac{2}{6}+\frac{1}{36}=\frac{3}{36} \\
    p(3) &= P(\setb{(3,x)}{x=1,2,3})+P(\setb{(x,3)}{x=1,2})=\frac{1}{6}\cdot\frac{3}{6}+\frac{2}{6}\cdot\frac{1}{6}=
    \frac{5}{36} \\
    p(4) &= P(\setb{(4,x)}{x=1,2,3,4})+P(\setb{(x,4)}{x=1,2,3})=\frac{1}{6}\cdot\frac{4}{6}+\frac{3}{6}\cdot\frac{1}{6}=
    \frac{7}{36} \\
    p(5) &= P(\setb{(5,x)}{x=1,2,3,4,5})+P(\setb{(x,5)}{x=1,2,3,4})=\frac{1}{6}\cdot\frac{5}{6}+\frac{4}{6}\cdot\frac{1}{6}=
    \frac{9}{36} \\
    p(6) &= P(\setb{(6,x)}{x=1,2,3,4,5,6})+P(\setb{(x,6)}{x=1,2,3,4,5})=\frac{1}{6}\cdot1+\frac{5}{6}\cdot\frac{1}{6}=
    \frac{11}{36} \\
  \end{align*}

  \begin{tabular}{c|cccccc}
    \(m\) & 1 & 2 & 3 & 4 & 5 & 6 \\
    \hline
    \(p(m)\) & \(\frac{1}{36}\) & \(\frac{3}{36}\) & \(\frac{5}{36}\) & \(\frac{7}{36}\) & \(\frac{9}{36}\) &
    \(\frac{11}{36}\)
  \end{tabular}

  \bigskip

  \begin{tikzpicture}
    \begin{axis}[axis x line=bottom, axis y line=left,
        xtick distance=1, xmin=0, xmax=7, xtick={1,2,3,4,5,6},
        ytick distance={1/36}, ymin=0, ymax=12/36, ytick={1/36,3/36,5/36,7/36,9/36,11/36},
        yticklabels={\(\frac{1}{36}\),
          \(\frac{3}{36}\),
          \(\frac{5}{36}\),
          \(\frac{7}{36}\),
          \(\frac{9}{36}\),
          \(\frac{11}{36}\)}
      ]
      \addplot [black] coordinates {(1,0) (1,1/36)};
      \addplot [only marks] coordinates {(1,1/36)};
      \addplot [black] coordinates {(2,0) (2,3/36)};
      \addplot [only marks] coordinates {(2,3/36)};
      \addplot [black] coordinates {(3,0) (3,5/36)};
      \addplot [only marks] coordinates {(3,5/36)};
      \addplot [black] coordinates {(4,0) (4,7/36)};
      \addplot [only marks] coordinates {(4,7/36)};
      \addplot [black] coordinates {(5,0) (5,9/36)};
      \addplot [only marks] coordinates {(5,9/36)};
      \addplot [black] coordinates {(6,0) (6,11/36)};
      \addplot [only marks] coordinates {(6,11/36)};
    \end{axis}
  \end{tikzpicture}

  \bigskip

\item Determine the cdf of \(M\) and graph it.
  \begin{align*}
    P(M\le1) &= p(1)=\frac{1}{36} \\
    P(M\le2) &= P(M\le1)+p(2)=\frac{1}{36}+\frac{3}{36}=\frac{4}{36} \\
    P(M\le3) &= P(M\le2)+p(3)=\frac{4}{36}+\frac{5}{36}=\frac{9}{36} \\
    P(M\le4) &= P(M\le3)+p(4)=\frac{9}{36}+\frac{7}{36}=\frac{16}{36} \\
    P(M\le5) &= P(M\le4)+p(5)=\frac{16}{36}+\frac{9}{36}=\frac{25}{36} \\
    P(M\le6) &= P(M\le5)+p(6)=\frac{25}{36}+\frac{11}{36}=\frac{36}{36}=1 \\
  \end{align*}
  
  \bigskip

  \begin{tikzpicture}
    \begin{axis}[axis x line=bottom, axis y line=left,
        xtick distance=1, xmin=0, xmax=7, xtick={1,2,3,4,5,6},
        ytick distance={1/36}, ymin=0, ymax=1.1, ytick={1/36,4/36,9/36,16/36,25/36,1},
        yticklabels={\(\frac{1}{36}\),
          \(\frac{4}{36}\),
          \(\frac{9}{36}\),
          \(\frac{16}{36}\),
          \(\frac{25}{36}\),
          \(1\)}
      ]
      \addplot [black] coordinates {(1,1/36) (2,1/36)};
      \addplot [only marks,filldot] coordinates {(1,1/36)};
      \addplot [only marks,opendot] coordinates {(2,1/36)};
      \addplot [black] coordinates {(2,4/36) (3,4/36)};
      \addplot [only marks,filldot] coordinates {(2,4/36)};
      \addplot [only marks,opendot] coordinates {(3,4/36)};
      \addplot [black] coordinates {(3,9/36) (4,9/36)};
      \addplot [only marks,filldot] coordinates {(3,9/36)};
      \addplot [only marks,opendot] coordinates {(4,9/36)};
      \addplot [black] coordinates {(4,16/36) (5,16/36)};
      \addplot [only marks,filldot] coordinates {(4,16/36)};
      \addplot [only marks,opendot] coordinates {(5,16/36)};
      \addplot [black] coordinates {(5,25/36) (6,25/36)};
      \addplot [only marks,filldot] coordinates {(5,25/36)};
      \addplot [only marks,opendot] coordinates {(6,25/36)};
      \addplot [only marks,filldot] coordinates {(6,1)};
      \draw [->] (axis cs:6,1) -- (axis cs:7,1);
    \end{axis}
  \end{tikzpicture}
\end{enumerate}

\section*{3.23}

A branch of a certain bank in New York City has six ATMs.  Let \(X\) represent the number of machines in use at a particular
time of the day.  The cdf of \(X\) is as follows:
\[F(x)=\begin{cases}
0 & x < 0 \\
0.06 & 0\le x<1 \\
0.19 & 1\le x<2 \\
0.39 & 2\le x<3 \\
0.67 & 3\le x<4 \\
0.92 & 4\le x<5 \\
0.97 & 5\le x<6 \\
1 & 6\le x
\end{cases}\]
Calculate the following probabilities directly from the cdf:
\begin{enumerate}[label={\alph*)}]
\item \(p(2)\), that is, \(P(X=2)\)
  \[P(X=2)=F(2)-F(1)=0.39-0.19=0.20\]
\item \(P(X>3)\)
  \[P(X>3)=F(6)-F(3)=1-0.67=0.33\]
\item \(P(2\le X\le5)\)
  \[P(2\le X\le5)=F(5)-F(1)=0.97-0.19=0.78\]
\item \(P(2<X<5)\)
  \[P(2<X<5)=F(4)-F(2)=0.92-0.39=0.53\]
\end{enumerate}

\section*{3.29}

The pmf of the amount of memory \(X\) (Gb) in a purchased flash drive was given in Example 3.13 as

\bigskip

\begin{tabular}{c|ccccc}
  \(x\) & 1 & 2 & 4 & 8 & 16 \\
  \hline
  \(p(x)\) & 0.05 & 0.10 & 0.35 & 0.40 & 0.10
\end{tabular}

\bigskip

Compute the following:

\bigskip

\begin{tabular}{c|ccccc}
  \(x\) & 1 & 2 & 4 & 8 & 16 \\
  \hline
  \(p(x)\) & 0.05 & 0.10 & 0.35 & 0.40 & 0.10 \\
  \hline
  \(x\cdot p(x)\) & 0.05 & 0.20 & 1.40 & 3.20 & 1.60 \\
  \hline
  \((x-\pi)^2\cdot p(x)\) & 1.49 & 1.98 & 2.10 & 0.96 & 9.12 \\
  \hline
  \(x^2\cdot p(x)\) & 0.05 & 0.40 & 5.60 & 25.60 & 25.60
\end{tabular}

\bigskip

\begin{enumerate}[label={\alph*)}]
\item \(E(X)\)
  \[E(X)=\pi=0.05+0.20+1.40+3.20+1.60=6.45\]
\item \(V(X)\) directly from the definition
  \[V(X)=1.49+1.98+2.10+0.96+9.12=15.65\]
\item The standard deviation of \(X\)
  \[\sigma=\sqrt{V(X)}=\sqrt{15.65}=3.96\]
\item \(V(X)\) using the shortcut formula
  \begin{gather*}
    E(X^2)=0.05+0.40+5.60+25.60+25.60=57.25 \\
    V(X)=E(X^2)-E(X)^2=57.25-6.45^2=15.65
  \end{gather*}
\end{enumerate}

\section*{3.32}

A certain brand of upright freezer is available in three different rated capacities: \SI{16}{ft^3}, \SI{18}{ft^3}, and
\SI{20}{ft^3}.  Let \(X=\) the rated capacity of a freezer of this brand sold at a certain store.  Suppose that \(X\) has
pmf

\bigskip

\begin{tabular}{c|ccc}
  \(x\) & 16 & 18 & 20 \\
  \hline
  \(p(x)\) & 0.2 & 0.5 & 0.3
\end{tabular}

\bigskip

\begin{enumerate}[label={\alph*)}]
\item Compute \(E(X)\), \(E(X^2)\), and \(V(X)\).

  \bigskip
  
  \begin{tabular}{c|ccc}
    \(x\) & 16 & 18 & 20 \\
    \hline
    \(p(x)\) & 0.2 & 0.5 & 0.3 \\
    \hline
    \(x\cdot p(x)\) & 3.2 & 9.0 & 6.0 \\
    \(x^2\cdot p(x)\) & 51.2 & 162.0 & 120.0 \\
  \end{tabular}

  \begin{gather*}
    E(X)=3.2+9.0+6.0=\SI{18.2}{ft^3} \\
    E(X^2)=51.2+162.0+120.0=\SI{333.2}{(ft^3)^2} \\
    V(X)=E(X^2)-E(X)^2=333.2-18.2^2=\SI{1.96}{(ft^3)^2} \\
  \end{gather*}

\item If the price of a freezer having capacity \(X\) is \(70X-650\), what is the expected price paid by the next
  customer to buy a freezer?

  Let \(Y=\) the price of the next freezer purchased:
  \[E(Y)=E(70X-650)=70E(X)-650=70(18.2)-650=\$624\]

\item What is the variance of the price paid by the next customer?
  \[V(Y)=70^2V(X)=4900\cdot1.96=\SI{9604}{dollars^2}\]

\item Suppose that although the rated capacity of a freezer is \(X\), the actual capacity is \(h(x)=X-0.008X^2\).  What is
  the expected actual capacity of the freezer purchased by the next customer?
  \[E(H)=E(X-0.008X^2)=E(X)-0.008E(X^2)=18.2-0.008\cdot333.2=\SI{15.5}{ft^3}\]
\end{enumerate}

\section*{3.34}

Suppose that the number of plants of a particular type found in a rectangular sampling region (called a quadrat by
ecologists) in a certain geographic area is a rv \(X\) with pmf
\[p(x)=\begin{cases}
\frac{c}{x^3} & x=1,2,3,\ldots \\
0 & \text{otherwise}
\end{cases}\]
Is \(E(X)\) finite? Justify your answer (this is another distribution that statisticians would call heavy-tailed).
\[E(X)=\sum_{x=1}^{\infty}xp(x)=\sum_{x=1}^{\infty}x\frac{c}{x^3}=c\sum_{x=1}^{\infty}\frac{1}{x^2}\]
By the so-called \(p\)-rule, this sum converges for \(p=2>1\).  In fact, we know:
\[E(X)=\frac{\pi^2}{6}<\infty\]

\section*{3.37}

The \(n\) candidates for a job have been ranked \(1,2,3,\ldots,n\).  Let \(X=\) the rank of a randomly selected candidate,
so \(X\) has pmf
\[p(x)=\begin{cases}
\frac{1}{n} & x=1,2,3,\ldots,n \\
0 & \text{otherwise}
\end{cases}\]
(this is called the \emph{discrete uniform distribution}).  Compute \(E(X)\) and \(V(X)\) using the shortcut formula.
   [Hint: The sum of the first \(n\) positive integers is \(n(n+1)/2\), whereas the sum of their squares is
     \(n(n+1)(2n+1)/6\).]
\begin{gather*}
  E(X)=\sum_{x=1}^nxp(x)=\sum_{x=1}^nx\cdot\frac{1}{n}=\frac{1}{n}\sum_{x=1}^nx=\frac{1}{n}\cdot\frac{n(n+1)}{2}=\frac{n+1}{2} \\
  E(X^2)=\sum_{x=1}^nx^2p(x)=\sum_{x=1}^nx^2\cdot\frac{1}{n}=\frac{1}{n}\sum_{x=1}^nx^2=
  \frac{1}{n}\cdot\frac{n(n+1)(2n+1)}{6}=\frac{(n+1)(2n+1)}{6} \\
\end{gather*}
\begin{align*}
  V(X) &= E(X^2)-E(X)^2 \\
  &= \frac{(n+1)(2n+1)}{6} - \left(\frac{n+1}{2}\right)^2 \\
  &= \frac{2(2n^2+3n+1)-3(n^2+2n+1)}{12} \\
  &= \frac{n^2-1}{12}
\end{align*}
\end{document}
