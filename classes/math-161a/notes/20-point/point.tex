\documentclass[letterpaper,12pt,fleqn]{article}
\usepackage{matharticle}
\pagestyle{empty}
\renewcommand{\O}{\theta}
\newcommand{\xb}{\bar{X}}
\newcommand{\m}{\mu}
\DeclareMathOperator{\ud}{Unif}
\begin{document}
\section*{Point Estimation}

Up to this point, both the population and its parameters have been known.  Now, suppose that only the population is known (or
suspected); however, the parameters are not known.  The use of sample statistics to make educated guesses of the parameters is
known as \emph{statistical inference}.

\begin{definition}[Point Estimate]
  A \emph{point estimate} of a population parameter \(\O\) is a single number that can be regarded as a sensible value for
  \(\O\).  It is obtained by selecting a suitable statistic and computing its value from the given sample data.  The selected
  statistic is called the \emph{point estimator} of \(\O\).

  The notation \(\hat{\O}\) is used to represent either the estimator or the estimate, depending on context.
\end{definition}

\begin{example}
  Suppose that the weight of 12 eggs in a selected carton are as follows:
  \begin{align*}
    x_1 &= 63.3 \\
    x_2 &= 63.4 \\
    x_3 &= 64.0 \\
    x_4 &= 63.0 \\
    x_5 &= 70.4 \\
    x_6 &= 65.7 \\
    x_7 &= 63.7 \\
    x_8 &= 65.8 \\
    x_9 &= 67.5 \\
    x_{10} &= 66.4 \\
    x_{11} &= 66.8 \\
    x_{12} &= 66.0
  \end{align*}
  Let \(\xb\) be a point estimator for \(\m\):
  \[\hat{\m}=65.5\]
\end{example}

Note that since \(\hat{\O}\) is a random variable, different random samples will yield different point estimates.

There may also be multiple choices of estimator for \(\O\).

\begin{example}
  Suppose we draw a random sample \(X_i\) from the uniform distribution \(\ud(0,\O)\).  Then the sample maximum:
  \[\hat{\O}=\max_{1\le i\le n}X_i\]
  can be used as a point estimator for \(\O\).

  Alternatively,
  \[\hat{\O}=2\xb\]
  is another possible estimator for \(\O\).
\end{example}

The question of which estimator is best for a given \(\O\) involves the question of \emph{bias}.

\end{document}
