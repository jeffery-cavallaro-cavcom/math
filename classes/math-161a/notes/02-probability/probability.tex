\documentclass[letterpaper,12pt,fleqn]{article}
\usepackage{matharticle}
\pagestyle{plain}
\renewcommand{\S}{\mathcal{S}}
\begin{document}
\section*{Probability}

\begin{definition}[Probability]
  Probability is a function defined on a space of events that satisfies the following axioms:
  \begin{enumerate}
  \item \(\forall\,E\subseteq S,P(E)\ge0\)
  \item \(P(\S)=1\)
  \item Let \(\set{E_1,E_2,\ldots}\) be a countably-infinite set of pairwise disjoint events:
    \[P\left(\bigcup_{i=1}^{\infty}E_i\right)=\sum_{i=1}^{\infty}P(E_i)\]
  \end{enumerate}
\end{definition}

\begin{definition}[Relative Frequency]
  Assume that an experiment is repeated \(n\) times and event \(E\) occurred \(n(E)\) times.  The \emph{relative frequency}
  of \(E\) is given by:
  \[\frac{n(E)}{n}\]
  The probability of \(E\) can then be interpreted as:
  \[P(E)=\lim_{n->\infty}\frac{n(E)}{n}\]
\end{definition}

\begin{theorem}
\(P(\emptyset)=0\)
\end{theorem}

\begin{proof}
  Let \(\set{E_1,E_2,\ldots}\) be a countably-infinite set of events such that all the \(E_i=\emptyset\).  Since
  \(E_i\cap E_j=\emptyset\cap\emptyset=\emptyset\), the \(E_i\) are pairwise disjoint.  So, from the third axiom:
  \begin{gather*}
    P(\bigcup_{i=1}^{\infty}E_i)=\sum_{i=1}^{\infty}P(E_i) \\
    P(\bigcup_{i=1}^{\infty}\emptyset)=\sum_{i=1}^{\infty}P(\emptyset) \\
    P(\emptyset)=\sum_{i=1}^{\infty}P(\emptyset)
  \end{gather*}
  But this can only happen when \(P(\emptyset)=0\).
\end{proof}

\begin{theorem}
  Let \(\set{E_1,\ldots,E_k}\) be a finite set of pairwise disjoint events:
  \[P\left(\bigcup_{i=1}^kE_i\right)=\sum_{i=1}^kP(E_i)\]
\end{theorem}

\begin{proof}
  Let \(\set{E_{k+1},E_{k+2},\ldots}\) be a countably-infinite set of events such that all the \(E_i=\emptyset\).  By the
  third axiom and previous theorem:
  \[P(\bigcup_{i=1}^kE_i)=P(\bigcup_{i=1}^{\infty}E_i)=\sum_{i=1}^{\infty}E_i=\sum_{i=1}^kE_i\]
\end{proof}

\begin{theorem}
\(P(E)=1-P(E^c)\)
\end{theorem}

\begin{proof}
  By definition, \(E\) and \(E^c\) are disjoint.  So, by the previous theorem:
  \begin{gather*}
    P(E\cup E^c)=P(E)+P(E^c) \\
    P(\S)=P(E)+P(E^c)
  \end{gather*}
  But by the second axiom, \(P(\S)=1\), and therefore:
  \[P(E)+P(E^c)=1\]
  and:
  \[P(E)=1-P(E^c)\]
\end{proof}

\begin{corollary}
  \(0\le P(E)\le1\)
\end{corollary}

\begin{theorem}
  \(A\subseteq B\implies P(A)\le P(B)\)
\end{theorem}

\begin{proof}
  Let \(C=B-A\), and thus \(B=A\cup C\) and \(A\cap C=\emptyset\).  By the previous theorem:
  \[P(B)=P(A\cup C)=P(A)+P(C)\]
  But, by the first axiom, \(P(C)\ge0\).  Therefore:
  \[P(A)\le P(B)\]
\end{proof}

\begin{theorem}[PIE]
  \(P(A\cup B)=P(A)+P(B)-P(A\cap B)\)
\end{theorem}

\begin{proof}
  First, decompose \(B\) into two disjoint events:
  \begin{gather*}
    B=(A\cap B)\cup(A^c\cap B) \\
    P(B)=P((A\cap B)\cup(A^c\cap B))=P(A\cap B)+P(A^c\cap B) \\
    P(A^c\cap B)=P(B)-P(A\cap B)
  \end{gather*}
  Now, decompose \(A\cup B\) into two disjoint events:
  \begin{gather*}
    A\cup B=A\cup(A^c\cap B) \\
    P(A\cup B)=P\left(A\cup(A^c\cap B)\right)=P(A)+P(A^c\cap B)=P(A)+P(B)-P(A\cap B)
  \end{gather*}
\end{proof}

\begin{example}
  In a large discrete math class, 55\% of the students are math majors, 35\% are CS majors, and 5\% are dual majors (in
  math and CS).  What percentage of the class majors in neither of them?

  Let:
  \begin{gather*}
    A=\set{\text{majors in math}} \\
    B=\set{\text{majors in CS}}
  \end{gather*}
  \begin{align*}
    P(A^c\cap B^c) &= P\left((A\cap B)^c\right) \\
    &= 1-P(A\cap B) \\
    &= 1-\left(P(A)+P(B)-P(A\cap B)\right) \\
    &= 1-(0.55+0.35-0.05) \\
    &= 1-0.985 \\
    &= 0.15
  \end{align*}
\end{example}

This theorem can be expanded to three events:
\[P(A\cup B\cup C)=P(A)+P(B)+P(C)-P(A\cap B)-P(A\cap C)-P(B\cap C)+P(A\cap B\cap C)\]

\end{document}
