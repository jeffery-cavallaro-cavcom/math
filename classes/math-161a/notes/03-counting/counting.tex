\documentclass[letterpaper,12pt,fleqn]{article}
\usepackage{matharticle}
\pagestyle{plain}
\newcommand{\w}{\omega}
\renewcommand{\S}{\mathcal{S}}
\begin{document}
\section*{Counting}

Since simple events are by definition mutually exclusive, the following corollary follows from the previous axioms and
theorems:

\begin{corollary}
  Let \(E\) be an event composed of a countable number of outcomes \(\w\):
  \[P(E)=\sum_{\w\in E}P(\w)\]
\end{corollary}

\begin{theorem}
  Let \(\S\) be a sample space consisting of a finite number of equally likely outcomes.  The probability of each outcome
  \(\w\) is given by:
  \[P(\w)=\frac{1}{\abs{\S}}\]
\end{theorem}

\begin{proof}
  Let \(p=\) the equally likely probability of each \(\w\):
  \[1=\sum_{i=1}^{\abs{\S}}P(\w)=\sum_{i=1}^{\abs{\S}}p=p\abs{\S}\]
  Therefore:
  \[p=\frac{1}{\abs{\S}}\]
\end{proof}

\begin{corollary}
  Let \(\S\) be a sample space consisting of a finite number of equally likely outcomes and let \(E\subseteq S\):
  \[P(E)=\frac{\abs{E}}{\abs{\S}}\]
\end{corollary}

\begin{examples}
  \begin{itemize}
  \item[]
  \item Flip a fair coin: \(P(H)=P(T)=\frac{1}{2}\)
  \item Toss a fair die:
    \begin{itemize}
    \item Simple events: \(P(1)=P(2)=P(3)=P(4)=P(5)=P(6)=\frac{1}{6}\)
    \item \(P\left(\set{\text{an even number}}\right)=\frac{3}{6}=\frac{1}{2}\)
    \item \(P\left(\set{\text{at least 5}}\right)=\frac{2}{6}=\frac{1}{3}\)
    \item \(P\left(\set{\text{not 3}}\right)=\frac{5}{6}\)
    \end{itemize}
  \item Toss a fair coin 5 times:
    \begin{itemize}
    \item \(P\left(\set{\text{at least one head}}\right)=1-P\left(\set{\text{no heads}}\right)=1-\frac{1}{2^5}=
      \frac{31}{32}\)
    \end{itemize}
  \end{itemize}
\end{examples}

When events are large and complicated, combinatorics can help.

\begin{theorem}[Fundamental Counting Principle]
  Suppose an experiment can be performed in a sequence of \(k\) steps such that there are \(n_i\) ways to perform the
  \(i^{th}\) step.  The total number of outcomes for the experiment is given by:
  \[N=\prod_{i=1}^kn_i\]
\end{theorem}

\begin{examples}
  \begin{enumerate}
  \item[]
  \item A subway restautant provides 5 kinds of bread, 4 kinds of cheese, 4 kinds of meat, and 6 kinds of sauces.  In how
    many ways can you order a sandwich:
    \[N=5\cdot4\cdot4\cdot6=480\ \text{ways}\]

  \item How many different CA driver's licenses are there (1 capital letter followed by 7 numbers)?  How many driver's
    licenses have all repeated digits?  All distinct digits?

    total: \(26\cdot10^7=260,000,000\)

    repeated: \(26\cdot10=260\)

    distinct: \(26\cdot P(10,7)=26\cdot\frac{10!}{(10-3)!}=15,724,800\)

  \item How many ordered lists of size 3 can be made from a set \(S=\set{a,b,c,d}\)?
    \begin{enumerate}
    \item with repetition allowed: \(4^3\)
    \item with repetition not allowed: \(4\cdot3\cdot2=24\)
    \end{enumerate}
  \end{enumerate}
\end{examples}

\begin{definition}[Permutation]
  A \emph{permutation} of \(k\) elements from a set with \(n\) elements is a set of ordered lists with:
  \[P(n,k)=\frac{n!}{(n-k)!}\]
  elements.
\end{definition}

\begin{example}
  List all permutations of size \(r=3\) chosen from the set \(S=\set{a,b,c,d}\).  How many are there?  What if \(r=4\)?

  \begin{tabular}{cccc}
    abc & bac & cab & dab \\
    abd & bad & cad & dac \\
    acb & bca & cba & dba \\
    acd & bcd & cbd & dbc \\
    adb & bda & cda & dca \\
    adc & bdc & cdb & dcb
  \end{tabular}
  \[P(4,3)=\frac{4!}{(4-3)!}=24\]
  \[P(4,4)=\frac{4!}{(4-4)!}=24\]
\end{example}

\begin{example}
  In how many different ways can 5 people be arranged in a row?  Along a circle?
  \[5!=120\]

  First, seat one person, then the other four.  Each pattern is simply repeated depending on where the first person sits:
  \[1\cdot4!=24\]
\end{example}

\begin{example}
  How many 3-digit numbers are divisible by 5?

  The number cannot start with a 0 and must end with a 0 or 5:
  \[9\cdot10\cdot2=180\]
\end{example}

\begin{example}[The Birthday Problem]
  Find the probability \(p\) that no two people in a class of 35 have a common birthday (i.e., all have different birthdays).
  Assume that people's birthdays are equally likely to occur among the 365 days of the year and ignore leap years.
  \[\frac{P(365,35)}{365^{35}}=0.1856\]
\end{example}

\begin{definition}[Combination]
  A \emph{combination} of \(k\) elements from a set with \(n\) elements is a set of subsets (unordered lists) with:
  \[C(n,k)=\binom{n}{k}=\frac{n!}{k!(n-k)!}\]
  elements.
\end{definition}

\begin{example}
  List all combinations of size 3 chosen from the set \(S=\set{a,b,c,d}\):
  \[\set{a,b,c}, \set{a,b,d}, \set{a,c,d}, \set{b,c,d}\]
  \[\binom{4}{3}=\frac{4!}{3!(4-3)!}=4\]
\end{example}

\begin{example}
  Consider the problem of choosing 4 members from a group of 10 to work on a special project.
  \begin{enumerate}[label={\alph*)}]
  \item Suppose two people \(A\) and \(B\) really like each other, so that they must be simultaneously chosen or skipped.  How
    many distinct four-person teams can be chosen?
    \[\binom{8}{2}+\binom{8}{2}\]

  \item Suppose two people \(A\) and \(B\) really hate each other, so that they cannot both be selected for the project.  How
    many distinct four-person teams can be chosen?
    \[\binom{8}{4}+\binom{8}{3}+\binom{8}{3}\]
  \end{enumerate}
\end{example}

\begin{example}
  An urn has 5 red balls and 7 blue balls.  Suppose you randomly select 5 balls from the urn.  What is the probability that
  your hand has exactly 3 red balls?
  \[\frac{\binom{5}{3}\binom{7}{2}}{\binom{12}{5}}\]
\end{example}

\begin{example}
  An ordinary deck of 52 cards is divided into 4 suits (hearts, diamonds, spades, and clubs) and 13 ranks
  (2,3,4,5,6,7,8,9,10,J,Q,K,A).  Suppose you randomly draw 5 cards from a deck of 52.  What is the probability that you have a:
  \begin{enumerate}[label={\alph*)}]
  \item four of a kind (4 cards of the same rank, and one side card)?
    \[\frac{\binom{13}{1}\binom{48}{1}}{\binom{52}{5}}\]

  \item flush (5 cards of the same suit)?
    \[\frac{\binom{4}{1}\binom{13}{5}}{\binom{52}{5}}\]
  \end{enumerate}
\end{example}

\end{document}
