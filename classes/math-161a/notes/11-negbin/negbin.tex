\documentclass[letterpaper,12pt,fleqn]{article}
\usepackage{matharticle}
\pagestyle{empty}
\DeclareMathOperator{\nb}{NB}
\DeclareMathOperator{\geom}{Geom}
\renewcommand{\o}{\sigma}
\begin{document}
\section*{Negative Binomial Distribution}

\begin{definition}[Negative Binomial Distribution]
  To say that a random variable \(X\) has a \emph{Negative Binomial} distribution with parameters \(p\) and \(r\), denoted:
  \[X\sim\nb(p,r)\]
  means that:
  \begin{enumerate}
  \item The underlying experiment is composed of repeated Bernoulli trials until \(r\) successes occur.
  \item The trials are independent.
  \item Each of the trials has fixed probability \(p\) for success.
  \item \(X\) counts the number of trials up to and including the last success.
  \end{enumerate}
\end{definition}

\begin{examples}[Negative Binomial Distributions]
  \begin{enumerate}
  \item[]
  \item Flip a fair coin until 5 heads occur: \(X=\) the number trials.
    \[X\sim\nb\left(\frac{1}{2},5\right)\]
  \item Select (with replacement) balls from an urn that has 30 red balls and 20 blue balls until 3 red balls are
    selected: \(Y=\) the number of balls selected.
    \[Y\sim\nb\left(0.6,3\right)\]
  \end{enumerate}
\end{examples}

\begin{theorem}
  Let \(X\) be a random variable with a Negative Binomial distribution with parameters \(r\) and \(p\):
  \begin{itemize}
  \item \(f_X(x)=\begin{cases}
    \binom{x-1}{r-1}p^r(1-p)^{x-r} & x=r,r+1,r+2,\ldots \\
    0 & \text{otherwise}
  \end{cases}\)
  \item \(E(X)=\frac{r}{p}\)
  \item \(V(X)=\frac{r(1-p)}{p^2}\)
  \end{itemize}
\end{theorem}

\begin{proof}
  For \(P(X=x)\), fix a success in the \(x^{th}\) trial with probability \(p\).  The remaining \(x-1\) trials contain \(r-1\)
  successes with probability \(p^{r-1}\) and \((x-1)-(r-1)=x-r\) failures with probability \((1-p)^{x-r}\).  Therefore:
  \[f_X(x)=\binom{x-1}{r-1}p^{r-1}(1-p)^{x-r}p=\binom{x-1}{r-1}p^r(1-p)^{x-r}\]
  Now, let \(X_i=\) the number of trials since \(i^{th}-1\) success until the \(i^{th}\) success, with \(X_1\) measured from
  the first trial.  Note that each of these \(X_i\) are independent and \(X_i\sim\geom(p)\).  Thus:
  \begin{gather*}
    E(X)=E(\sum_{i=1}^rX_i)=\sum_{i=1}^rE(X_i)=\sum_{i=1}^r\frac{1}{p}=\frac{r}{p} \\
    V(X)=V(\sum_{i=1}^rX_i)=\sum_{i=1}^rV(X_i)=\sum_{i=1}^r\frac{1-p}{p^2}=\frac{r(1-p)}{p^2}
  \end{gather*}
\end{proof}

\begin{example}
  Suppose \(X\) has a Negative Binomial distribution with \(p=\frac{1}{2}\) and \(r=3\).
  \begin{gather*}
    X\sim\nb\left(\frac{1}{2},3\right) \\
    \\
    P(X=2)=0 \\
    P(X=3)=\binom{3-1}{3-1}\left(\frac{1}{2}\right)^3\left(1-\frac{1}{2}\right)^{3-3}=\left(\frac{1}{2}\right)^3=\frac{1}{8} \\
    P(X=4)=\binom{4-1}{3-1}\left(\frac{1}{2}\right)^3\left(1-\frac{1}{2}\right)^{4-3}=
    \binom{3}{2}\left(\frac{1}{2}\right)^4=3\cdot\frac{1}{16}=\frac{3}{16} \\
    P(X=5)=\binom{5-1}{3-1}\left(\frac{1}{2}\right)^3\left(1-\frac{1}{2}\right)^{5-3}=
    \binom{4}{2}\left(\frac{1}{2}\right)^5=6\cdot\frac{1}{32}=\frac{3}{16} \\
    P(X\ge5)=1-P(X=3)-p(X=4)=1-\frac{1}{8}-\frac{3}{16}=\frac{11}{16} \\
    \\
    E(X)=\frac{r}{p}=\frac{3}{\frac{1}{2}}=6 \\
    \\
    V(X)=\frac{r(1-p)}{p^2}=\frac{3(1-\frac{1}{2})}{\frac{1}{4}}=6 \\
    \\
    \o=\sqrt{6}\approx2.45
  \end{gather*}
\end{example}

\end{document}
