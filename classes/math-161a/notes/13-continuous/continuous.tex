\documentclass[letterpaper,12pt,fleqn]{article}
\usepackage{matharticle}
\usepackage{tikz}
\usepackage{pgfplots}
\pgfplotsset{compat=1.16}
\pagestyle{empty}
\DeclareMathOperator{\range}{Range}
\newcommand{\med}{\tilde{\mu}}
\renewcommand{\o}{\sigma}
\begin{document}
\section*{Continuous Random Variables}

\begin{definition}[Continuous Random Variable]
  To say that a random variable \(X\) is \emph{continuous} means that its range is an interval or a union of intervals.
\end{definition}

\begin{definition}[Probability Density Function]
  The \emph{Probability Density Function} (pdf) of a continuous random variable \(X\) is a function \({f:\R\to\R}\) such that:
  \begin{enumerate}
  \item \(\forall\,x\in\R, f(x)\ge0\)
  \item \(\displaystyle\int_{-\infty}^{\infty}f(x)dx=1\)
  \item \(\forall\,a,b\in\R\) such that \(a\le b\):
    \[P(a\le X\le b)=\int_a^bf(x)dx\]
  \end{enumerate}
\end{definition}

\begin{properties}[Probability Density Function]
  Let \(X\) be a continuous random variable:
  \begin{itemize}
  \item \(\range(X)=\setb{x}{f(x)>0}\)
  \item \(x\notin\range(X)\iff f(x)=0\)
  \item \(P(X=c)=\int_c^cf(x)dx=0\)
  \end{itemize}
\end{properties}
Since the probability of a single value is 0, the endpoints of an interval do not matter:
\[P(a\le X\le b)=P(a<X\le b)=P(a\le X<b)=P(a<X<b)\]

\begin{definition}[Cumulative Distribution Function]
  The \emph{Cumulative Distribution Function} (cdf) of a continuous random variable \(X\) is a function \({F:\R\to\R}\) such
  that:
  \[F(x)=P(X\le x)=\int_{-\infty}^xf(t)dt\]
  The \emph{complementary cdf}, denoted \(\bar{F}(x)\), is given by:
  \[\bar{F}(x)=1-F(x)\]
\end{definition}

\begin{properties}[Cumulative Distribution Function]
  Let \(X\) be a continuous random variable with pdf \(f(x)\) and cdf \(F(x)\):
  \begin{itemize}
  \item \(\displaystyle\lim_{x\to-\infty}F(x)=0\)
  \item \(\displaystyle\lim_{x\to\infty}F(x)=1\)
  \item \(F(x)\) is non-decreasing and continuous on \(\R\).
  \item \(P(a<X)=1-F(a)\)
  \item \(P(a<X<b)=F(b)-F(a)\)
  \item \(F'(x)=f(x)\)
  \end{itemize}
\end{properties}

\begin{definition}[Median]
  The \emph{median} of a continuous random variable \(X\) with pdf \(f(x)\) and cdf \(F(x)\), denoted \(\med\), is
  given by:
  \[\frac{1}{2}=F(\med)=\int_{-\infty}^{\med}f(x)dx\]
\end{definition}

\begin{definition}[Expected Value]
  The \emph{expected value (mean)} of a continuous random variable \(X\) with pdf \(f(x)\), denoted \(\mu_X\) or \(E(X)\), is
  given by:
  \[\mu_X=E(X)=\int_{-\infty}^{\infty}xf(x)dx\]
  Similarly, for some function \(h(x)\):
  \[\mu_{h(X)}=E(h(X))=\int_{-\infty}^{\infty}h(x)f(x)dx\]
\end{definition}

\begin{definition}[Variance]
  The \emph{variance} of a continuous random variable \(X\) with pdf \(f(x)\) and expected value \(\mu\), denoted \(\o^2\) or
  \(V(X)\), is given by:
  \[\o^2=V(x)=E((X-\mu)^2)=\int_{-\infty}^{\infty}(x-\mu)^2f(x)dx=E(X^2)-\mu^2=E(X^2)-E(X)^2\]
  The \emph{standard deviation}, denoted \(\o\), is given by:
  \[\o=\sqrt{\o^2}=\sqrt{V(X)}\]
\end{definition}

\bigskip

\begin{example}
  Let \(f(x)=1, 0\le x\le1\) be a pdf for a continuous random variable \(X\):

  \begin{tikzpicture}[scale=0.75]
    \begin{axis}[
        xlabel={\(x\)},
        ylabel={\(f(x)\)},
        axis lines=middle,
        xmin={-1/2},
        xmax={3/2},
        ymin={-1/4},
        ymax={3/2},
        xtick={1},
        ytick={1},
        every axis x label/.style={
          at={(ticklabel* cs:1)},
          anchor=west
        },
        every axis y label/.style={
          at={(ticklabel* cs:1)},
          anchor=south,
        }
      ]
      \node [draw,circle,scale=0.5] at (0,0) {};
      \node [draw,circle,fill=black,scale=0.5] at (0,1) {};
      \node [draw,circle,fill=black,scale=0.5] at (1,1) {};
      \node [draw,circle,scale=0.5] at (1,0) {};
      \addplot [domain=0:1] {1};
      \addplot [domain=-1/2:0, very thick] {0};
      \addplot [domain=1:1.5, very thick] {0};
      \draw [dashed] (1,1) to (1,0);
    \end{axis}
  \end{tikzpicture}

  Since this pdf is simple, we can use area to calculate probabilities:

  \begin{enumerate}
  \item \(P(X<-1)=0\)
  \item \(P(X=0.2)=0\)
  \item \(P(X<0.2)=0.2\cdot1=0.2\)
  \item \(P(0.2<X<0.5)=(0.5-0.2)\cdot1=0.3\)
  \item \(P(X>0.6)=(1-0.6)\cdot1=0.5\)
  \end{enumerate}

  To find the cdf:
  \[F(x)=\int_0^xdt=\left.t\right|_0^x=x\]
  \begin{tikzpicture}[scale=0.75]
    \begin{axis}[
        xlabel={\(x\)},
        ylabel={\(F(x)\)},
        axis lines=middle,
        xmin={-1/2},
        xmax={3/2},
        ymin={-1/4},
        ymax={3/2},
        xtick={1},
        ytick={1},
        every axis x label/.style={
          at={(ticklabel* cs:1)},
          anchor=west
        },
        every axis y label/.style={
          at={(ticklabel* cs:1)},
          anchor=south,
        }
      ]
      \node [draw,circle,fill=black,scale=0.5] at (0,0) {};
      \node [draw,circle,fill=black,scale=0.5] at (1,1) {};
      \addplot [domain=0:1] {x};
      \addplot [domain=-1/2:0, very thick] {0};
      \addplot [domain=1:1.5, very thick] {1};
      \draw [dashed] (0,1) to (1,1);
      \draw [dashed] (1,0) to (1,1);
    \end{axis}
  \end{tikzpicture}

  To find the median:
  \[\frac{1}{2}=F(\med)=\med\]
  To find the expected value:
  \[E(X)=\int_0^1x\cdot1dx=\int_0^1xdx=\left.\frac{1}{2}x^2\right|_0^1=\frac{1}{2}\]
  To find the variance and standard deviation:
  \begin{gather*}
    E(X^2)=\int_0^1x^2\cdot1dx=\int_0^1x^2dx=\left.\frac{1}{3}x^2\right|_0^1=\frac{1}{3} \\
    \\
    \o^2=\frac{1}{3}-\left(\frac{1}{2}\right)^2=\frac{1}{3}-\frac{1}{4}=\frac{1}{12} \\
    \\
    \o=\sqrt{\frac{1}{12}}=\frac{1}{2\sqrt{3}}=\frac{1}{6}\sqrt{3}
  \end{gather*}
\end{example}

\bigskip

\begin{example}
  Let \(f(x)=c(1-x),0<x<1\) be the pdf for a continuous random variable \(X\).  We first need to find an appropriate value
  for \(c\):
  \begin{gather*}
    \int_0^1c(1-x)dx=\left.c\left(x-\frac{1}{2}x^2\right)\right|_0^1=\frac{1}{2}c=1 \\
    c=2
  \end{gather*}

  And so \(f(x)=2(1-x)\):

  \begin{tikzpicture}[scale=0.75]
    \begin{axis}[
        xlabel={\(x\)},
        ylabel={\(f(x)\)},
        axis lines=middle,
        xmin={-1/2},
        xmax={3/2},
        ymin={-1/4},
        ymax={9/4},
        xtick={1},
        ytick={2},
        every axis x label/.style={
          at={(ticklabel* cs:1)},
          anchor=west
        },
        every axis y label/.style={
          at={(ticklabel* cs:1)},
          anchor=south,
        }
      ]
      \node [draw,circle,scale=0.5] at (0,0) {};
      \node [draw,circle,fill=black,scale=0.5] at (0,2) {};
      \node [draw,circle,fill=black,scale=0.5] at (1,0) {};
      \addplot [domain=0:1] {2*(1-x)};
      \addplot [domain=-1/2:0, very thick] {0};
      \addplot [domain=1:1.5, very thick] {0};
    \end{axis}
  \end{tikzpicture}

  To find the cdf:
  \[F(x)=\int_0^x2(1-t)dt=\int_0^x(2-2t)dt=\left.(2t-t^2)\right|_0^x=2x-x^2=x(2-x)\]
  \begin{tikzpicture}[scale=0.75]
    \begin{axis}[
        xlabel={\(x\)},
        ylabel={\(F(x)\)},
        axis lines=middle,
        xmin={-1/2},
        xmax={3/2},
        ymin={-1/4},
        ymax={3/2},
        xtick={1},
        ytick={1},
        every axis x label/.style={
          at={(ticklabel* cs:1)},
          anchor=west
        },
        every axis y label/.style={
          at={(ticklabel* cs:1)},
          anchor=south,
        }
      ]
      \node [draw,circle,fill=black,scale=0.5] at (0,0) {};
      \node [draw,circle,fill=black,scale=0.5] at (1,1) {};
      \addplot [domain=0:1] {x*(2-x)};
      \addplot [domain=-1/2:0, very thick] {0};
      \addplot [domain=1:1.5, very thick] {1};
      \draw [dashed] (0,1) to (1,1);
      \draw [dashed] (1,0) to (1,1);
    \end{axis}
  \end{tikzpicture}

  To find the median:
  \begin{gather*}
    \frac{1}{2}=F(\med)=2\med-\med^2 \\
    \med^2-2\med+\frac{1}{2}=0 \\
    2\med^2-4\med+1=0 \\
    \med=\frac{4\pm\sqrt{16-8}}{4}=1\pm\frac{1}{2}\sqrt{2}=1\pm\frac{1}{\sqrt{2}}
  \end{gather*}
  Honoring the domain:
  \[\med=1-\frac{1}{\sqrt{2}}=0.2929\]
  To find the expected value:
  \[E(X)=\int_0^1x\cdot2(1-x)dx=\int_0^1(2x-2x^2)dx=\left.\left(x^2-\frac{2}{3}x^2\right)\right|_0^1=1-\frac{2}{3}=
  \frac{1}{3}\]
  To find the expected value:
  \[E(X)=\int_0^1x\cdot2(1-x)dx=\int_0^1(2x-2x^2)dx=\left.\left(x^2-\frac{2}{3}x^3\right)\right|_0^1=1-\frac{2}{3}=
  \frac{1}{3}\]
  To find the variance and standard deviation:
  \begin{gather*}
    E(X^2)=\int_0^1x^2\cdot2(1-x)dx=\int_0^1(2x^2-2x^3)dx=\left.\left(\frac{2}{3}x^3-\frac{1}{2}x^4\right)\right|_0^1=
    \frac{2}{3}-\frac{1}{2}=\frac{1}{6} \\
    \\
    \o^2=\frac{1}{6}-\left(\frac{1}{3}\right)^2=\frac{1}{6}-\frac{1}{9}=\frac{1}{18} \\
    \\
    \o=\sqrt{\frac{1}{18}}=\frac{1}{3\sqrt{2}}=\frac{1}{6}\sqrt{2}
  \end{gather*}
\end{example}

\end{document}
