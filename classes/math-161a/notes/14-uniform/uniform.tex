\documentclass[letterpaper,12pt,fleqn]{article}
\usepackage{matharticle}
\usepackage{tikz}
\usepackage{pgfplots}
\pgfplotsset{compat=1.16}
\usepgfplotslibrary{groupplots}
\DeclareMathOperator{\unif}{Unif}
\pagestyle{empty}
\begin{document}
\section*{Uniform Distribution}

\begin{definition}[Uniform PDF]
  To say that a continuous random variable \(X\) has a \emph{uniform distribution} with parameters \(a\) and \(b\), denoted
  \(X\sim\unif(a,b)\), means that it has the following pdf:
  \[f(x)=\begin{cases}
  \frac{1}{a-b} & a\le x\le b \\
  0 & \text{otherwise}
  \end{cases}\]
\end{definition}

Note that:
\[\int_{-\infty}^{\infty}f(x)dx=\frac{1}{b-a}\int_a^bdx=\left.\frac{1}{b-a}x\right|_a^b=\frac{1}{b-a}(b-a)=1\]

\begin{theorem}[Uniform CDF]
  Let \(X\) be a continuous random variable such that \(X\sim\unif(a,b)\).  The cdf of \(X\) is given by:
  \[F(x)=\begin{cases}
  0 & x<a \\
  \frac{x-a}{b-a} & a\le x\le b \\
  1 & b<x
  \end{cases}\]
\end{theorem}

\begin{proof}
  \[F(x)=\int_{-\infty}^xf(x)dx=\frac{1}{b-a}\int_a^xdx=\left.\frac{1}{b-a}x\right|_a^x=\frac{x-a}{b-a}\]
\end{proof}

\begin{center}
  \begin{tikzpicture}[scale=0.75]
    \begin{groupplot}[
        group style={
          group size=1 by 2,
          horizontal sep=0pt,
          vertical sep=0pt,
          x descriptions at=edge bottom,
          y descriptions at=edge left,
        },
        xmin=0,
        xmax=4,
        axis x line=middle,
        axis y line=left,
        y axis line style=-,
        title style={at={(0.5,0)}}
      ]
      \nextgroupplot[
        ymin=-1/4,
        ymax=3/4,
        xtick={0,1,3},
        xticklabels={\(0\),\(a\),\(b\)},
        ytick={0,1/2},
        yticklabels={\(0\),\(\frac{1}{a-b}\)},
        title={pdf}
      ]
      \addplot [domain=0:1, very thick] {0};
      \addplot [domain=1:3] {1/2};
      \addplot [domain=3:4, very thick] {0};
      \node [draw,circle,scale=0.5] at (1,0) {};
      \node [draw,circle,fill,scale=0.5] at (1,1/2) {};
      \node [draw,circle,fill,scale=0.5] at (3,1/2) {};
      \node [draw,circle,scale=0.5] at (3,0) {};
      \draw [dashed] (1,0) -- (1,1/2);
      \draw [dashed] (3,0) -- (3,1/2);
      \draw [dashed] (0,1/2) -- (1,1/2);
      \nextgroupplot[
        ymin=-1/2,
        ymax=3/2,
        xtick={0,1,3},
        xticklabels={\(0\),\(a\),\(b\)},
        ytick={0,1},
        title={cdf}
      ]
      \addplot [domain=0:1, very thick] {0};
      \addplot [domain=1:3] {(1/2)*(x-1)};
      \addplot [domain=3:4] {1};
      \draw [dashed] (3,0) -- (3,1);
      \draw [dashed] (0,1) -- (3,1);
    \end{groupplot}
  \end{tikzpicture}
\end{center}

\begin{theorem}[Uniform Mean and Variance]
  Let \(X\) be a continuous random variable such that \(X\sim\unif(a,b)\):
  \begin{itemize}
  \item \(\displaystyle E(X)=\frac{a+b}{2}\)
  \item \(\displaystyle V(X)=\frac{(b-a)^2}{12}\)
  \end{itemize}
\end{theorem}

\begin{proof}
  \[E(X)=\int_{-\infty}^{\infty}xf(x)dx=\frac{1}{b-a}\int_a^bxdx=\left.\frac{1}{2(b-a)}x^2\right|_a^b=\frac{b^2-a^2}{2(b-a)}=
  \frac{a+b}{2}\]
  \[E(X^2)=\int_{-\infty}^{\infty}x^2f(x)dx=\frac{1}{b-a}\int_a^bx^2dx=\left.\frac{1}{3(b-a)}x^3\right|_a^b=\frac{b^3-a^3}{3(b-a)}=
  \frac{a^2+ab+b^2}{3}\]
  \begin{align*}
    V(X) &= E(X^2)-E(X)^2 \\
    &= \frac{a^2+ab+b^2}{3}-\frac{(a+b)^2}{4} \\
    &= \frac{4(a^2+ab+b^2)-3(a^2+2ab+b^2)}{12} \\
    &= \frac{4a^2+4ab+4b^2-3a^2-6ab-3b^2}{12} \\
    &= \frac{a^2-2ab+b^2}{12} \\
    &= \frac{(b-a)^2}{12}
  \end{align*}
\end{proof}

\begin{example}
  Let \(X\sim\unif(2,4)\):
  \begin{gather*}
    f(x)=\frac{1}{4-2}=\frac{1}{2} \\
    \\
    F(x)=\frac{x-2}{4-2}=\frac{x-2}{2}=\frac{1}{2}x-1
    \\
    E(X)=\frac{2+4}{2}=3 \\
    \\
    V(X)=\frac{(4-2)^2}{12}=\frac{4}{12}=\frac{1}{3}
  \end{gather*}
\end{example}

\begin{example}
  A bus arrives at a stop uniformly random between noon and 12:15pm.  A passenger arrives at the bus stop exactly at noon.
  What is the probability that the passenger will wait no more than 5 minutes, between 5 and 10 minutes, or more than 10
  minutes for the bus?

  Let \(X=\) minutes waited.  \(X\sim\unif(0,15)\).
  \begin{gather*}
    f(x)=\frac{1}{15} \\
    \\
    F(x)=\frac{x}{15} \\
    \\
    P(0\le X\le5)=F(5)=\frac{5}{15}=\frac{1}{3} \\
    P(5\le X\le 10)=F(10)-F(5)=\frac{10}{15}-\frac{5}{15}=\frac{5}{15}=\frac{1}{3} \\
    P(10\le X\le15)=F(15)-F(10)=1-\frac{10}{15}=\frac{5}{15}=\frac{1}{3}
  \end{gather*}
\end{example}

\end{document}
