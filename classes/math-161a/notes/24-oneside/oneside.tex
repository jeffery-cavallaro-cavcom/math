\documentclass[letterpaper,12pt,fleqn]{article}
\usepackage{matharticle}
\usepackage{tikz}
\pagestyle{empty}
\renewcommand{\a}{\alpha}
\newcommand{\m}{\mu}
\renewcommand{\o}{\sigma}
\renewcommand{\P}{\Phi}
\newcommand{\iid}{\overset{\text{iid}}{\sim}}
\DeclareMathOperator{\nd}{N}
\begin{document}
\section*{One-sided Confidence Intervals}

Sometimes only a one-sided confidence interval is needed:
\begin{itemize}
\item Lower confidence bound
  \[1-\a=P(\bar{X}<\m)=P\left(\m\in(\bar{X}-m,\infty)\right)\]
  \begin{quote}
    \begin{tikzpicture}
      \draw (0,0) -- (10,0);
      \node (x) at (5,0) {\scalebox{1.5}{\(\times\)}};
      \node [below=5pt] at (x) {\(\bar{X}\)};
      \node (l) at (3,0) {\scalebox{1.5}{\((\)}};
      \node [below=5pt] at (l) {\(\bar{X}-m\)};
    \end{tikzpicture}
  \end{quote}

\item Upper confidence bound
  \[1-\a=P(\m<\bar{X})=P\left(\m\in(-\infty,\bar{X}+m)\right)\]
  \begin{quote}
    \begin{tikzpicture}
      \draw (0,0) -- (10,0);
      \node (x) at (5,0) {\scalebox{1.5}{\(\times\)}};
      \node [below=5pt] at (x) {\(\bar{X}\)};
      \node (r) at (7,0) {\scalebox{1.5}{\()\)}};
      \node [below=5pt] at (r) {\(\bar{X}+m\)};
    \end{tikzpicture}
  \end{quote}
\end{itemize}

\begin{theorem}
  Let \(X_i\iid\nd(\m,\o^2)\) such that \(\m\) is unknown.  If \(\o\) is known then the one-sided \(1-\a\) confidence
  intervals are given by:
  \begin{itemize}
  \item Upper
    \[\m<\bar{x}+z_{\a}\frac{\o}{\sqrt{n}}\]
  \item Lower
    \[\m>\bar{x}-z_{\a}\frac{\o}{\sqrt{n}}\]
  \end{itemize}
  If \(\o\) is unknown then the one-sided \(1-a\) confidence intervals are given by:
  \begin{itemize}
  \item Upper
    \[\m<\bar{x}+t_{\a,n-1}\frac{s}{\sqrt{n}}\]
  \item Lower
    \[\m>\bar{x}-t_{\a,n-1}\frac{s}{\sqrt{n}}\]
  \end{itemize}
\end{theorem}

\begin{example}
  A sample carton of brown eggs from a farm has \(\bar{x}=65.5\).  Assuming a normal population with \(\o^2=4\), obtain 95\%
  upper and lower confidence intervals for \(\m\).
  \begin{gather*}
    z_{0.05}=\P(0.95)=1.645 \\
    \\
    z_{0.05}\frac{\o}{\sqrt{n}}=1.645\cdot\frac{2}{\sqrt{12}}=0.95
    \\
    65.5+0.95=66.45 \\
    65.5-0.95=64.55
  \end{gather*}
\end{example}

\end{document}
