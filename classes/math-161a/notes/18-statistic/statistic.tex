\documentclass[letterpaper,12pt,fleqn]{article}
\usepackage{matharticle}
\pagestyle{empty}
\newcommand{\iid}{\overset{\text{iid}}{\sim}}
\newcommand{\m}{\mu}
\renewcommand{\o}{\sigma}
\DeclareMathOperator{\nd}{N}
\DeclareMathOperator{\bd}{Bernoulli}
\begin{document}
\section*{Statistics}

\begin{definition}[Sample]
  \begin{itemize}[leftmargin=*]
  \item[]
  \item A \emph{sample} is a subset of randomly selected members from a population.
  \item A \emph{physical sample} is an actual random selection of members from a population where each member's measurement is
    denoted by \(x_i\).
  \item A \emph{random sample} is a theoretical collection of random variables \(X_i\) such that the \(X_i\) are:
    \begin{enumerate}
    \item independent
    \item identically distributed according to some pmf \(f(x)\)
    \end{enumerate}
    denoted by:
    \[X_i\iid f(x)\]
  \end{itemize}
\end{definition}

Samples are used when it is not feasible to address an entire population, either because the population is too large or only
exists theoretically.

\begin{example}
  Consider a farm that produces brown eggs.  All of the eggs produced by the farm is a population.  A carton of a dozen
  randomly selected eggs is a sample.  Since the weight of each egg is iid to a normal distribution:
  \[X_i\iid\nd(\m,\o^2)\]
\end{example}

\begin{example}
  Consider tossing a coin with the probability of heads \(p\) independently for \(N\) times, where \(X_i\) denotes the
  numerical value of the \(i^{th}\) toss:
  \[X_i\iid\bd(p)\]
\end{example}

\begin{definition}[Statistic]
  A \emph{statistic} is any value that can be calculated from a sample of a population.
\end{definition}

Note that since the measurement of each value in a sample has uncertainly and is thus a random variable described by a
distribution, a statistic also exhibits uncertainty and is therefore also a random variable with a distribution.

\bigskip

\begin{definition}[Mean]
  The \emph{sample mean} of a sample \(X\) of size \(N\), denoted \(\bar{X}\), is given by:
  \[\bar{X}=\frac{1}{N}\sum_{i=1}^NX_i\]
\end{definition}

\bigskip

\begin{definition}[Median]
  The \emph{sample median} of a sample \(X\) of size \(N\) ordered by increasing value, denoted \(\tilde{X}\), is given by:
  \[\tilde{X}=x_{\left(\frac{N+1}{2}\right)}\]
  for \(N\) odd, and:
  \[\tilde{X}=\frac{x_{\left(\frac{N}{2}\right)}+x_{\left(\frac{N}{2}+1\right)}}{2}\]
  for \(N\) even.
\end{definition}

Note that both the mean and median are measures of center; however, the mean is more sensitive to outliers.  They are equal for
symmetric distributions, but the median will move towards the side of the mean with the most values.

\begin{definition}[Variance]
  The \emph{sample variance} of a sample \(X\) of size \(N\), denoted \(s^2\), is given by:
  \[S^2=\frac{1}{n-1}\sum_{i=1}^N(X_i-\bar{X})^2\]
  Also, the \emph{sample standard deviation} is given by:
  \[S=\sqrt{S^2}\]
\end{definition}

\begin{definition}[Sampling Distribution]
  The distribution of a statistic random variable is called the \emph{sampling distribution} of the statistic.
\end{definition}

\end{document}
