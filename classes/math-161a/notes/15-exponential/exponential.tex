\documentclass[letterpaper,12pt,fleqn]{article}
\usepackage{matharticle}
\usepackage{tikz}
\usepackage{pgfplots}
\pgfplotsset{compat=1.16}
\usepgfplotslibrary{groupplots}
\DeclareMathOperator{\expd}{Exp}
\renewcommand{\l}{\lambda}
\pagestyle{empty}
\allowdisplaybreaks
\begin{document}
\section*{Exponential Distribution}

\begin{definition}[Exponential PDF]
  To say that a continuous random variable \(X\) has an \emph{exponential distribution} with parameter \(\l>0\), denoted
  \(X\sim\expd(\l)\), means that it has the following pdf:
  \[f(x)=\begin{cases}
  \l e^{-\l x} & x\ge0 \\
  0 & \text{otherwise}
  \end{cases}\]
\end{definition}

Note that:
\begin{align*}
  \int_{-\infty}^{\infty}f(x)dx &= \l\int_0^{\infty}e^{-\l x}dx \\
  &= \left.\l\left(-\frac{1}{\l}\right)e^{-\l x}\right|_0^{\infty} \\
  &= \left.-e^{-\l x}\right|_0^{\infty} \\
  &= \left.e^{-\l x}\right|_{\infty}^0 \\
  &= 1-0 \\
  &= 1
\end{align*}

\begin{theorem}[Exponential CDF]
  Let \(X\) be a continuous random variable such that \(X\sim\expd(\l)\).  The cdf of \(X\) is given by:
  \[F(x)=\begin{cases}
  1-e^{-\l x} & x\ge0 \\
  0 & \text{otherwise}
  \end{cases}\]
\end{theorem}

\begin{proof}
  \begin{align*}
    F(x) &= \int_{-\infty}^xf(x)dx \\
    &= \l\int_0^xe^{-\l x}dx \\
    &= \left.\l\left(-\frac{1}{\l}\right)e^{-\l x}\right|_0^x \\
    &= \left.-e^{-\l x}\right|_0^x \\
    &= \left.e^{-\l x}\right|_x^0 \\
    &= 1-e^{-\l x}
  \end{align*}
\end{proof}

\begin{center}
  \begin{tikzpicture}[scale=0.75]
    \begin{groupplot}[
        group style={
          group size=1 by 2,
          horizontal sep=0pt,
          vertical sep=0pt,
          x descriptions at=edge bottom,
          y descriptions at=edge left,
        },
        xmin=-1,
        xmax=4,
        axis x line=middle,
        axis y line=middle,
        y axis line style=-,
        xmajorticks=false,
        title style={at={(0.5,0)}}
      ]
      \nextgroupplot[
        ymin=-1/2,
        ymax=5/4,
        ytick={1},
        yticklabels={\(\l\)},
        title={pdf}
      ]
      \addplot [domain=-1:0, very thick] {0};
      \addplot [domain=0:4] {exp(-x)};
      \node [draw,circle,scale=0.5] at (0,0) {};
      \node [draw,circle,fill,scale=0.5] at (0,1) {};
      \nextgroupplot[
        ymin=-1/2,
        ymax=3/2,
        ytick={1},
        title={cdf}
      ]
      \addplot [domain=-1:0, very thick] {0};
      \addplot [domain=0:4] {1-exp(-x)};
      \draw [dashed] (0,1) -- (4,1);
    \end{groupplot}
  \end{tikzpicture}
\end{center}

\begin{corollary}
  Let \(X\) be a continuous random variable such that \(X\sim\expd(\l)\):
  \[P(a\le X\le b)=e^{-\l a}-e^{-\l b}\]
\end{corollary}

\begin{corollary}
  Let \(X\) be a continuous random variable such that \(X\sim\expd(\l)\):
  \[\bar{F}(x)=e^{-\l x}\]
\end{corollary}

\begin{theorem}[Exponential Mean and Variance]
  Let \(X\) be a continuous random variable such that \(X\sim\expd(\l)\):
  \begin{itemize}
  \item \(\displaystyle E(X)=\frac{1}{\l}\)
  \item \(\displaystyle V(X)=\frac{1}{\l^2}\)
  \end{itemize}
\end{theorem}

\begin{proof}
  \[E(X)=\int_{-\infty}^{\infty}xf(x)dx=\l\int_0^{\infty}xe^{-\l x}dx\]
  \begin{quote}
    \begin{tabular}{cc}
      \(u=x\) & \(dv=e^{-\l x}dx\) \\
      \(du=dx\) & \(v=-\frac{1}{\l}e^{-\l x}\)
    \end{tabular}
  \end{quote}
  \begin{align*}
    E(X) &= \l\left(\left.-\frac{1}{\l}xe^{-\l x}\right|_0^{\infty}+\frac{1}{\l}\int_0^{\infty}e^{-\l x}dx\right) \\
    &= \l\left(0-\left.\frac{1}{\l^2}e^{-\l x}\right|_0^{\infty}\right) \\
    &= -\left.\frac{1}{\l}e^{-\l x}\right|_0^{\infty} \\
    &= \left.\frac{1}{\l}e^{-\l x}\right|_{\infty}^0 \\
    &= \frac{1}{\l}(1-0) \\
    &= \frac{1}{\l}
  \end{align*}
  \[E(X^2)=\int_{-\infty}^{\infty}x^2f(x)dx=\l\int_0^{\infty}x^2e^{-\l x}dx\]
  \begin{quote}
    \begin{tabular}{cc}
      \(x^2\) & \(e^{-\l x}\) \\
      \(2x\) & \(-\frac{1}{\l}e^{-\l x}\) \\
      \(2\) & \(\frac{1}{\l^2}e^{-\l x}\) \\
      & \(-\frac{1}{\l^3}e^{-\l x}\) \\
    \end{tabular}
  \end{quote}
  \begin{align*}
    E(X^2) &= \left.\l\left(-\frac{1}{\l}x^2e^{-\l x}-\frac{2}{\l^2}xe^{-\l x}-\frac{2}{\l^3}e^{-\l x}\right)
    \right|_0^{\infty} \\
    &= \left.\left(-x^2e^{-\l x}-\frac{2}{\l}xe^{-\l x}-\frac{2}{\l^2}e^{-\l x}\right)\right|_0^{\infty} \\
    &= \left.\left(x^2e^{-\l x}+\frac{2}{\l}xe^{-\l x}+\frac{2}{\l^2}e^{-\l x}\right)\right|_{\infty}^0 \\
    &= \left(0+0+\frac{2}{\l^2}\right)-(0+0+0) \\
    &= \frac{2}{\l^2}
  \end{align*}
  \[V(X)=E(X^2)-E(X)^2=\frac{2}{\l^2}-\frac{1}{\l^2}=\frac{1}{\l^2}\]
\end{proof}

\begin{example}
  The lifetime of a certain brand of lightbulbs is exponentially distributed with an average of 1000 hours.  What is
  the probability that a new lightbulb will still be working after this amount of time?  Between 1000 and 2000 hours?
  \begin{gather*}
    \frac{1}{\l}=1000\implies\l=\frac{1}{1000} \\
    \\
    P(1000<X)=1-P(X\le1000)=1-F(1000)=1-\left(1-e^{-\frac{1000}{1000}}\right)=e^{-1}\approx0.3679 \\
    \\
    P(1000\le X\le2000)=e^{-\frac{1000}{1000}}-e^{-\frac{2000}{1000}}=e^{-1}-e^{-2}\approx0.2325
  \end{gather*}
\end{example}

\begin{theorem}[The Memoryless Property]
  Let \(X\) be a continuous random variable such that \(X\sim\expd(\l)\):
  \[\forall\,t_0,t\ge0,P\left(X\ge t_0+t\,\middle|\,X\ge t_0\right)=P(X\ge t)\]
\end{theorem}

\begin{proof}
  Assume \(t_0,t\ge0\):
  \begin{align*}
    P\left(X\ge t_0+t\,\middle|\,X\ge t_0\right) &= \frac{P((X\ge t_0+t)\cap(X\ge t_0))}{P(X\ge t_0)} \\
    &= \frac{P(X\ge t_0+t)}{P(X\ge t_0)} \\
    &= \frac{\bar{F}(t_0+t)}{\bar{F}(t_o)} \\
    &= \frac{e^{-\l(t_0+t)}}{e^{-\l t_0}} \\
    &= e^{\l t} \\
    &= \bar{F}(t) \\
    &= P(X\ge t)
  \end{align*}
  Note that the event \(X\ge t_0\) must have occurred if the event \(X\ge t_o+t\) has occurred.
\end{proof}
The exponential distribution is the only distribution that has the memoryless property.

\begin{example}
  Jones figures that the total number of thousands of miles an auto can be driven before it would need to be junked is an
  exponential random variable with parameter \(\l=\frac{1}{20}\).  Smith has a used car that he claims has been driven only
  10,000 miles.  If Jones purchases the car, what is the probability that he would get at least 20,000 additional miles out
  of it?  Repeat under the assumption that the lifetime mileage of the car is not exponentially distributed but rather is
  (in thousands of miles) uniformly distributed over \([0,40]\).

  Exponential:
  \[P\left(X\ge30\middle|X\ge10\right)=P(X\ge20)=\bar{F}(20)=e^{-\frac{20}{20}}=e^{-1}=0.3679\]
  Uniform:
  \[P\left(X\ge30\middle|X\ge10\right)=\frac{P((X\ge30)\cap(X\ge10))}{P(X\ge10)}=\frac{P(X\ge30)}{P(X\ge10)}=
  \frac{\bar{F}(30)}{\bar{F}(10)}=\frac{\left(1-\frac{30}{40}\right)}{\left(1-\frac{10}{40}\right)}=\frac{1}{3}\]
\end{example}

\end{document}
