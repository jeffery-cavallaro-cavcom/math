\documentclass[letterpaper,12pt,fleqn]{article}
\usepackage{matharticle}
\pagestyle{empty}
\DeclareMathOperator{\geom}{Geom}
\renewcommand{\o}{\sigma}
\allowdisplaybreaks
\begin{document}
\section*{Geometric Distribution}

\begin{definition}[Geometric Distribution]
  To say that a random variable \(X\) has a \emph{Geometric} distribution with parameter \(p\), denoted:
  \[X\sim\geom(p)\]
  means that:
  \begin{enumerate}
  \item The underlying experiment is composed of repeated Bernoulli trials until the first success occurs.
  \item The trials are independent.
  \item Each of the trials has fixed probability \(p\) for success.
  \item \(X\) counts the number of trials up to and including the first success.
  \end{enumerate}
\end{definition}

\begin{examples}[Geometric Distributions]
  \begin{enumerate}
  \item[]
  \item Flip a fair coin until the first head occurs: \(X=\) the number trials.
    \[X\sim\geom\left(\frac{1}{2}\right)\]
  \item Select (with replacement) balls from an urn that has \(N\) balls, \(r\) of which are red, until the first red ball
    is selected: \(Y=\) the number of balls selected.
    \[Y\sim\geom\left(\frac{r}{N}\right)\]
  \end{enumerate}
\end{examples}

\begin{lemma}
  Assume \(a\in\R\) such that \(0<\abs{a}<1\):
  \begin{gather*}
    \sum_{x=1}^{\infty}xa^{x-1}=\frac{1}{(1-a)^2} \\
    \sum_{x=2}^{\infty}x(x-1)a^{x-2}=\frac{2}{(1-a)^3}
  \end{gather*}
\end{lemma}

\begin{proof}
  \begin{gather*}
    \sum_{x=0}^{\infty}a^x=\frac{1}{1-a} \\
    \frac{d}{da}\left[\sum_{x=0}^{\infty}a^x\right]=\frac{d}{da}\left[\frac{1}{1-a}\right] \\
    \sum_{x=1}^{\infty}xa^{x-1}=\frac{1}{(1-a)^2} \\
    \frac{d}{da}\left[\sum_{x=1}^{\infty}xa^{x-1}\right]=\frac{d}{da}\left[\frac{1}{(1-a)^2}\right] \\
    \sum_{x=2}^{\infty}x(x-1)a^{x-2}=\frac{2}{(1-a)^3}
  \end{gather*}
\end{proof}

\begin{theorem}
  Let \(X\) be a random variable with a Geometric distribution with parameter \(p\):
  \begin{itemize}
  \item \(f_X(x)=\begin{cases}
    p(1-p)^{x-1} & x\in\N \\
    0 & \text{otherwise}
  \end{cases}\)
  \item \(E(X)=\frac{1}{p}\)
  \item \(V(X)=\frac{1-p}{p^2}\)
  \end{itemize}
\end{theorem}

\begin{proof}
  For \(P(X=x)\), the first \(x-1\) failures have probability \((1-p)^{x-1}\) and the final success has probability \(p\).
  Therefore:
  \[f_X(x)=p(1-p)^{x-1}\]
  To calculate the expected value:
  \[E(X)=\sum_{x=1}^{\infty}xp(1-p)^{x-1}=p\sum_{x=1}^{\infty}x(1-p)^{x-1}=p\cdot\frac{1}{[1-(1-p)]^2}=p\cdot\frac{1}{p^2}=
  \frac{1}{p}\]
  To calculate the variance:
  \begin{align*}
    V(X) &= E(X^2)-E(X)^2 \\
    &= E(X^2-X+X)-E(X)^2 \\
    &= E(X^2-X)+E(X)-E(X)^2 \\
    &= E(X(X-1))+E(X)-E(X)^2 \\
    &= \sum_{x=2}^{\infty}x(x-1)p(1-p)^{x-1}+\frac{1}{p}-\frac{1}{p^2} \\
    &= p(1-p)\sum_{x=2}^{\infty}x(x-1)(1-p)^{x-2}+\frac{p-1}{p^2} \\
    &= p(1-p)\frac{2}{[1-(1-p)]^3}+\frac{p-1}{p^2} \\
    &= \frac{2(1-p)}{p^2}+\frac{p-1}{p^2} \\
    &= \frac{2-2p+p-1}{p^2} \\
    V(X) &= \frac{1-p}{p^2}
  \end{align*}
\end{proof}

\begin{example}
  Suppose \(X\) has a Geometric distribution with \(p=\frac{1}{2}\).
  \begin{gather*}
    X\sim\geom\left(\frac{1}{2}\right) \\
    \\
    P(X=4)=\frac{1}{2}\left(1-\frac{1}{2}\right)^{4-1}=\left(\frac{1}{2}\right)^4=\frac{1}{16} \\
    P(X\ge4)=\sum_{x=4}^{\infty}\frac{1}{2}\left(1-\frac{1}{2}\right)^{x-1}=\sum_{x=4}^{\infty}\left(\frac{1}{2}\right)^x=
    \frac{1}{1-\frac{1}{2}}-1-\frac{1}{2}-\frac{1}{4}-\frac{1}{8}=\frac{1}{8}\\
    \\
    E(X)=\frac{1}{p}=\frac{1}{\frac{1}{2}}=2 \\
    \\
    V(X)=\frac{1-p}{p^2}=\frac{1-\frac{1}{2}}{\frac{1}{4}}=2 \\
    \\
    \o=\sqrt{2}\approx1.41
  \end{gather*}
\end{example}

\end{document}
