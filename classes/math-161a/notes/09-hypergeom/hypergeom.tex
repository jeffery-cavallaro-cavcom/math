\documentclass[letterpaper,12pt,fleqn]{article}
\usepackage{matharticle}
\pagestyle{empty}
\DeclareMathOperator{\hg}{HyperGeom}
\DeclareMathOperator{\bin}{Binomial}
\renewcommand{\o}{\sigma}
\begin{document}
\section*{Hypergeometric Distribution}

\begin{definition}[Binomial Distribution]
  To say that a random variable \(X\) has a \emph{Hypergeometric} distribution with parameters \(N\), \(r\), and \(n\),
  denoted:
  \[X\sim\hg(n,p)\]
  means that:
  \begin{enumerate}
  \item The underlying experiment is composed of \(n\) repeated Bernoulli trials of selecting elements from a population of
    finite size \(N\).
  \item The \(n\) trials are independent.
  \item There are \(r\) elements in the population that result in success when selected.
  \item Every subset of \(n\) elements from the population has an equal probability of being selected.
  \item \(X\) counts the number of successes resulting from the \(n\) trials.
  \end{enumerate}
\end{definition}

Note that a Hypergeometric distribution for selection from a population implies \emph{no} replacement.

\begin{example}[Hypergeometric Distributions]
  \begin{enumerate}
  \item[]
  \item Select (without replacement) 10 balls from an urn that has 30 red balls and 20 blue balls: \(X=\) the number of
    selected red balls.
    \[X\sim\hg(50,30,10)\]
  \item Poll \(n\) different voters at random from the whole pool of \(N\) registered voters, \(r\) of which support a
    certain presidential candidate: \(Y=\) the number of polled voters that support the candidate.
    \[X\sim\hg(N,r,n)\]
  \end{enumerate}
\end{example}

\begin{theorem}
  Let \(X\) be a random variable with a Hypergeometric distribution with parameters \(N\), \(r\), and \(n\) such that
  \(x\le r\) and \(n-x\le N-r\), and let \(p=\frac{r}{n}\):
  \begin{itemize}
  \item \(f_X(x)=\begin{cases}
    \frac{\binom{r}{x}\binom{N-r}{n-x}}{\binom{N}{n}} & x=0,1,2,\ldots,n \\
    0 & \text{otherwise}
  \end{cases}\)
  \item \(E(X)=\frac{nr}{N}=np\)
  \item \(V(X)=np(1-p)\left(\frac{N-n}{N-1}\right)\)
  \end{itemize}
\end{theorem}

\begin{proof}
  For \(P(X=x)\), select any \(x\) of \(r\): \(\binom{r}{x}\), then select any \(n-x\) of \(N-r\): \(\binom{N-r}{n-x}\).  The
  total number of possible selections is \(\binom{N}{n}\).  Therefore:
  \[f_X(x)=\frac{\binom{r}{x}\binom{N-r}{n-x}}{\binom{N}{n}}\]
\end{proof}

\begin{theorem}
  Let \(X\) be a random variable with a Hypergeometric distribution \(\hg(N,r,n)\). \\
  If \(N,r\gg n\):
  \[\hg(N,r,n)\approx\bin\left(n,p=\frac{r}{N}\right)\]
\end{theorem}

Note that when comparing a Hypergeometric distribution to its Binomial approximation, the expected values are the same;
however, the exact variance is always less than or equal to the approximation due to the extra correction factor.

\begin{example}
  Select 5 balls at random from an urn containing 300 red balls and 200 blue balls.  Let \(X=\) the number of selected red
  balls.
  \begin{gather*}
    X\sim\hg(500,200,5) \\
    \\
    P(X=3)=\frac{\binom{300}{3}\binom{500-300}{5-3}}{\binom{500}{5}}=
    \frac{\binom{300}{3}\binom{200}{2}}{\binom{500}{5}}=\frac{4455100\cdot19900}{255244687600}=0.3473 \\
    \\
    X\sim\hg(5,0.6) \\
    \\
    P(X=3)\approx\binom{5}{3}(0.6)^3(0.4)^2=10\cdot0.216\cdot0.16=0.3456 \\
    \\
    E(X)=np=5\cdot0.6=3 \\
    \\
    V(X)=np(1-p)\frac{N-n}{N-1}=5\cdot0.6\cdot0.4\cdot\frac{495}{499}=1.1904 \\
    \\
    \o=\sqrt{1.1904}\approx1.0910
  \end{gather*}
\end{example}

\end{document}
