\documentclass[letterpaper,12pt,fleqn]{article}
\usepackage{matharticle}
\usepackage{diagbox}
\newcommand{\cp}[3]{#1\left(#2\mathbin{\vert}#3\right)}
\pagestyle{empty}
\begin{document}
\section*{Joint Distributions}

\begin{definition}[Joint Distribution]
  Let \(X\) and \(Y\) be two discrete random variables associated with the same sample space.  The \emph{joint} pmf is
  a function \(F:\R^2\to\R\), denoted \(f(X,Y)\) where:
  \[f(x,y)=\begin{cases}
  P(X=x,Y=y) & \text{for all feasible}\ (x,y) \\
  0 & \text{otherwise}
  \end{cases}\]
  that satisfies the following:
  \begin{enumerate}[label={\arabic*)}]
  \item \(p(x,y)\ge0\)
  \item \(p(x,y)>0\) for a countable number of \((x,y)\) pairs.
  \item \(\displaystyle\sum_x\sum_yp(x,y)=1\)
  \item The probability that \((x,y)\) occurs in a set \(A\) is given by:
    \[P[(X,Y)\in A]=\mathop{\sum\sum}_{(x,y)\in A}p(x,y)\]
  \end{enumerate}
\end{definition}
A joint discrete distribution is easily represented by a table with the possible values of \(X\) and \(Y\) across the top
and left, respectively, and the \emph{marginal} pdfs \(f_X(x)\) and \(f_Y(y)\) across the bottom and right, respectively,
and the joint pmf \(f(x,y)\) in the middle.

\begin{example}
  Toss two fair dice.  Let \(X=\) their sum and let \(Y=\) the absolute value of their difference.

  \bigskip

  \setlength{\extrarowheight}{5pt}
  \begin{tabular}{|c|ccccccccccc||c|}
    \hline
    \diagbox{y}{x} & 2 & 3 & 4 & 5 & 6 & 7 & 8 & 9 & 10 & 11 & 12 & \(f_Y(y)\) \\
    \hline
    0 & \(\frac{1}{36}\) & & \(\frac{1}{36}\) & & \(\frac{1}{36}\) & & \(\frac{1}{36}\) & & \(\frac{1}{36}\) & &
    \(\frac{1}{36}\) & \(\frac{6}{36}\) \\
    1 & & \(\frac{2}{36}\) & & \(\frac{2}{36}\) & & \(\frac{2}{36}\) & & \(\frac{2}{36}\) & & \(\frac{2}{36}\) & &
    \(\frac{10}{36}\) \\
    2 & & & \(\frac{2}{36}\) & & \(\frac{2}{36}\) & & \(\frac{2}{36}\) & & \(\frac{2}{36}\) & & & \(\frac{8}{36}\) \\
    3 & & & & \(\frac{2}{36}\) & & \(\frac{2}{36}\) & & \(\frac{2}{36}\) & & & & \(\frac{6}{36}\) \\
    4 & & & & & \(\frac{2}{36}\) & & \(\frac{2}{36}\) & & & & & \(\frac{4}{36}\) \\
    5 & & & & & & \(\frac{2}{36}\) & & & & & & \(\frac{2}{36}\) \\
    & & & & & & & & & & & & \\
    \hline
    \hline
    \(f_X(x)\) & \(\frac{1}{36}\) & \(\frac{2}{36}\) & \(\frac{3}{36}\) & \(\frac{4}{36}\) & \(\frac{5}{36}\) &
    \(\frac{6}{36}\) &
    \(\frac{5}{36}\) & \(\frac{4}{36}\) & \(\frac{3}{36}\) & \(\frac{2}{36}\) & \(\frac{1}{36}\) & \\
    & & & & & & & & & & & & \\
    \hline
  \end{tabular}

  \bigskip

  \begin{itemize}
  \item \(P(X\le4,Y\le2)=\frac{6}{36}\)
  \item \(P(X\le5)=\frac{10}{36}\)
  \item \(P(X\ge11,Y\ge2)=0\)
  \item \(P(Y\le1)=\frac{16}{36}\)
  \end{itemize}
\end{example}

\begin{definition}[Conditional Probability]
  Let \(X\) and \(Y\) be two discrete random variables with joint pmf \(f(x,y)\).  The \emph{conditional} pmf of \(Y\) given
  \(X=x\) (with \(f_X(x)\ne0\)) is given by:
  \[\cp{f}{y}{x}=\frac{f(x,y)}{f_X(x)}\]
  for all feasible \(y\).
\end{definition}

\begin{example}
  From the previous example, the conditional pmfs of \(Y\) given \(X=x\):

  \bigskip

  \setlength{\extrarowheight}{5pt}
  \begin{tabular}{|c|ccccccccccc|}
    \hline
    \diagbox{y}{x} & 2 & 3 & 4 & 5 & 6 & 7 & 8 & 9 & 10 & 11 & 12 \\
    \hline
    \hline
    0 & 1 & & \(\frac{1}{3}\) & & \(\frac{1}{5}\) & & \(\frac{1}{5}\) & & \(\frac{1}{3}\) & & 1 \\
    1 & & 1 & & \(\frac{1}{2}\) & & \(\frac{1}{3}\) & & \(\frac{1}{2}\) & & 1 & \\
    2 & & & \(\frac{2}{3}\) & & \(\frac{2}{5}\) & & \(\frac{2}{5}\) & & \(\frac{2}{3}\) & & \\
    3 & & & & \(\frac{1}{2}\) & & \(\frac{1}{3}\) & & \(\frac{1}{2}\) & & & \\
    4 & & & & & \(\frac{2}{5}\) & & \(\frac{2}{5}\) & & & & \\
    5 & & & & & & \(\frac{1}{3}\) & & & & & \\
    \hline
  \end{tabular}

  \bigskip

  And the conditional pmfs of \(X\) given \(Y=y\):

  \bigskip

  \setlength{\extrarowheight}{5pt}
  \begin{tabular}{|c|ccccccccccc|}
    \hline
    \diagbox{y}{x} & 2 & 3 & 4 & 5 & 6 & 7 & 8 & 9 & 10 & 11 & 12 \\
    \hline
    \hline
    0 & \(\frac{1}{6}\) & & \(\frac{1}{6}\) & & \(\frac{1}{6}\) & & \(\frac{1}{6}\) & & \(\frac{1}{6}\) & & \(\frac{1}{6}\) \\
    1 & & 1 & & \(\frac{1}{5}\) & & \(\frac{1}{5}\) & & \(\frac{1}{5}\) & & \(\frac{1}{5}\) & \\
    2 & & & \(\frac{1}{4}\) & & \(\frac{1}{4}\) & & \(\frac{1}{4}\) & & \(\frac{1}{4}\) & & \\
    3 & & & & \(\frac{1}{3}\) & & \(\frac{1}{3}\) & & \(\frac{1}{3}\) & & & \\
    4 & & & & & \(\frac{1}{2}\) & & \(\frac{1}{2}\) & & & & \\
    5 & & & & & & 1 & & & & & \\
    \hline
  \end{tabular}

  \bigskip

  \begin{itemize}
  \item \(Y\) given \(X=6\)

    \begin{tabular}{c|ccc}
      \hline
      y & 0 & 2 & 4 \\
      \hline
      \(\cp{f}{y}{x=6}\) & \(\frac{1}{5}\) & \(\frac{2}{5}\) & \(\frac{2}{5}\) \\
      \hline
    \end{tabular}

    \bigskip
    
  \item \(Y\) given \(X=4\)

    \begin{tabular}{c|cc}
      \hline
      y & 0 & 2 \\
      \hline
      \(\cp{f}{y}{x=4}\) & \(\frac{1}{3}\) & \(\frac{2}{3}\) \\
      \hline
    \end{tabular}

    \bigskip
    
  \item \(X\) given \(Y=3\)

    \begin{tabular}{c|ccc}
      \hline
      x & 5 & 7 & 9 \\
      \hline
      \(\cp{f}{x}{y=3}\) & \(\frac{1}{3}\) & \(\frac{1}{3}\) & \(\frac{1}{3}\) \\
      \hline
    \end{tabular}

    \bigskip
    
  \item \(X\) given \(Y=0\)

    \begin{tabular}{c|cccccc}
      \hline
      x & 2 & 4 & 6 & 8 & 10 & 12 \\
      \hline
      \(\cp{f}{x}{y=0}\) & \(\frac{1}{6}\) & \(\frac{1}{6}\) & \(\frac{1}{6}\) & \(\frac{1}{6}\) & \(\frac{1}{6}\) &
      \(\frac{1}{6}\) \\
      \hline
    \end{tabular}

    \bigskip
  \end{itemize}
\end{example}

\begin{definition}{Independence}
  Let \(X\) and \(Y\) be two discrete random variables.  To say that \(X\) and \(Y\) are \emph{independent} means that:
  \[\forall\,x,y,f(x,y)=f_X(x)f_Y(y)\]
\end{definition}

\begin{example}
  In the previous example, \(X\) and \(Y\) are not independent because:
  \[f_X(2)f_Y(0)=\frac{1}{36}\cdot\frac{6}{36}=\frac{1}{216}\ne\frac{1}{36}=f(2,0)\]
  In fact, if the joint pmf has 0's then the variables are never independent.
\end{example}

\begin{example}
  The following joint and marginal pmf's demonstrate independent variables:

  \bigskip
  
  \setlength{\extrarowheight}{5pt}
  \begin{tabular}{|c|ccc|c|}
    \hline
    \diagbox{y}{x} & 0 & 1 & 2 & \(p_Y(y)\) \\
    \hline
    -1 & \(\frac{1}{6}\) & \(\frac{1}{12}\) & \(\frac{1}{12}\) & \(\frac{1}{3}\) \\
    1 & \(\frac{1}{3}\) & \(\frac{1}{6}\) & \(\frac{1}{6}\) & \(\frac{2}{3}\) \\
    \hline
    \(p_X(x)\) & \(\frac{1}{2}\) & \(\frac{1}{4}\) & \(\frac{1}{4}\) & \\
    \hline
  \end{tabular}

  \bigskip
\end{example}

\begin{theorem}
  Let \(X\) and \(Y\) be two discrete random variables.  If for all \(x\) and \(y\):
  \[\cp{f}{y}{x}=f_Y(y)\ \text{or}\ \cp{f}{x}{y}=f_X(x)\]
  then \(X\) and \(Y\) are independent.
\end{theorem}

\end{document}
