\documentclass[letterpaper,12pt,fleqn]{article}
\usepackage{matharticle}
\usepackage{tikz}
\usepackage{siunitx}
\pagestyle{empty}
\newcommand{\m}{\mu}
\renewcommand{\o}{\sigma}
\renewcommand{\O}{\theta}
\begin{document}
\section*{Hypothesis Testing}

\begin{definition}[Statistical Hypothesis]
  A \emph{statistical hypothesis} is a claim concerning the parameter(s) or form of a probability distribution.
\end{definition}

\begin{example}
  Consider a farm that produces brown eggs.  The farm claims that the weight of each egg is normally distributed with
  \(\m=\SI{65}{g}\) and \(\o=\SI{2}{g}\).
\end{example}

Suppose a carton of eggs is purchased from the farm in the example and the average weight of the sample is only
\SI{61.5}{g}.  Is the difference due to randomness or is this significant evidence against the farm's claim of
\SI{65}{g}?

\begin{definition}[Hypothesis Test]
  A \emph{hypothesis test} compares a favored statistical hypothesis, called the \emph{null hypothesis} and denoted \(H_0\),
  and a contradictory statistical hypothesis, called the \emph{alternate hypothesis} and denote \(H_a\).  A sample is used to
  obtain statistical information related to \(H_0\).  If the evidence against \(H_0\) is strong enough then \(H_0\) is
  rejected in favor of \(H_a\).  Otherwise, the test fails to reject \(H_0\).  Thus, the two outcomes of a hypothesis test
  are:
  \begin{enumerate}
  \item Reject \(H_0\)
  \item Fail to reject \(H_0\)
  \end{enumerate}
\end{definition}

\begin{example}
  In the above example, the two hypotheses are:
  \begin{description}[left=0.5in]
  \item[\(H_0\):] \(\m=65\)
  \item[\(H_a\):] \(\m\ne65\)
  \end{description}
\end{example}

\begin{definition}[Null Value]
  The \emph{null value} of a hypothesis test, denoted \(\O_0\), is the claimed parameter value in the null hypothesis.
\end{definition}

Thus, for a null hypothesis: \(H_0:\O=\O_0\) there is a two-sided and two one-sided alternate hypotheses:
\begin{enumerate}
\item \(H_a: \O\ne\O_0\)
\item \(H_a: \O<\O_0\)
\item \(H_a: \O>\O_0\)
\end{enumerate}
For the two one-sided alternates, the null hypothesis is assumed to be the fully contradictory \(\O\ge\O_0\) or \(\O\le\O_0\);
however, during calculations the equality form is simplier and deemed sufficient.

\begin{example}
  In the above example, the FDA wishes to enforce that \(\m=65\) by making sure that the true value just is not less than
  the claimed value - i.e., \(H_a:\m<65\).  A true value greater than 65 simply benefits the customer.
\end{example}

\subsection*{Hypothesis Test Procedure}

\begin{enumerate}
\item Construct \(H_0\) and \(H_a\) in terms of some null value \(\O_0\).
\item Select an unbiased point estimator \(\hat{\O}\) for \(\O_0\).
\item Collect sample data.
\item Compare the observation to the claim to see if there is sufficient reason to reject \(H_0\).
\end{enumerate}

\begin{example}
  In the above example, for \(H_0:\m=65\), select \(\bar{X}\) as a suitable (unbiased) point estimator for \(\m\).  A sample
  is then collected and \(\bar{x}\) is calculated.  The three possible alternatives would then be judged as follows:
  \begin{enumerate}
  \item \(H_a:\m\ne65\)

    Reject \(H_0\) if \(\bar{x}\) is either too small or too large:

    \begin{tikzpicture}
      \draw (0,0) -- (10,0) node [right] {\(\m\)};
      \node (X) at (5,0) {\scalebox{1.5}{\(\times\)}};
      \node [below=5pt] at (X) {\(65\)};
      \node [draw,circle,scale=0.5,red,fill=red] (B) at (2,0) {};
      \node [below=5pt] at (B) {\(\bar{x}\)};
      \node [draw,circle,scale=0.5,red,fill=red] (A) at (8,0) {};
      \node [below=5pt] at (A) {\(\bar{x}\)};
    \end{tikzpicture}

  \item \(H_a:\m<65\)

    Reject \(H_0\) if \(\bar{x}\) is too small.

    \begin{tikzpicture}
      \draw (0,0) -- (10,0) node [right] {\(\m\)};
      \node (X) at (5,0) {\scalebox{1.5}{\(\times\)}};
      \node [below=5pt] at (X) {\(65\)};
      \node [draw,circle,scale=0.5,red,fill=red] (B) at (2,0) {};
      \node [below=5pt] at (B) {\(\bar{x}\)};
    \end{tikzpicture}

  \item \(H_a:\m>65\)

    Reject \(H_0\) if \(\bar{x}\) is too large.

    \begin{tikzpicture}
      \draw (0,0) -- (10,0) node [right] {\(\m\)};
      \node (X) at (5,0) {\scalebox{1.5}{\(\times\)}};
      \node [below=5pt] at (X) {\(65\)};
      \node [draw,circle,scale=0.5,red,fill=red] (A) at (8,0) {};
      \node [below=5pt] at (A) {\(\bar{x}\)};
    \end{tikzpicture}
  \end{enumerate}
\end{example}

Notions of ``too small'' and ``too large'' will be developed by balancing test error and using \(p\)-values.

\end{document}
