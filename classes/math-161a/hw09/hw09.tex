\documentclass[letterpaper,12pt,fleqn]{article}
\usepackage{matharticle}
\usepackage{siunitx}
\pagestyle{plain}
\newcommand{\m}{\mu}
\renewcommand{\o}{\sigma}
\newcommand{\iid}{\overset{\text{iid}}{\sim}}
\DeclareMathOperator{\nd}{N}
\begin{document}
Cavallaro, Jeffery \\
Math 161A \\
Homework \#9

\bigskip

\section*{8.3}

For which of the given \(p\)-values would the null hypothesis be rejected when performing a level 0.05 test?
\begin{enumerate}[label={\alph*)}]
\item 0.001 (reject)
\item 0.021 (reject)
\item 0.078 (fail to reject)
\item 0.047 (reject)
\item 0.148 (fail to reject)
\end{enumerate}

\section*{8.9}

Water samples are taken from water used for cooling as it is being discharged from a power plant into a river.  It has been
determined that as long as the mean temperature of the discharged water is at most \(150^{\circ}F\), there will be no negative
effects on the river's ecosystem.  To investigate whether the plant is in compliance with regulations that prohibit a mean
discharge water temperature above \(150^{\circ}\), 50 water samples will be taken at randomly-selected times and the
temperature of each sample recorded.  The resulting data will be used to test the hypothesis \(H_0:\m=150^{\circ}\) versus
\(H_a:\m>150^{\circ}\).  In context of this situation, describe type I and type II errors.  Which type of error would you
consider more serious.

Type I: Concluding that the mean temperature is greater than \(150^{\circ}\) and hence the power plant is out of compliance
when the true mean is actually less than or equal to \(150^{\circ}\) and the power plant is in compliance.

Type II: Concluding that the mean temperature is less than or equal to \(150^{\circ}\) and hence the power plant is in
compliance when the true mean is greater than \(150^{\circ}\) and the power plant is not in compliance.

In this case, the type II error is more important because damage is being done to the river's ecosystem.

\section*{8.12}

A mixture of pulverized fuel ash and Portland cement to be used for grouting should have a compressive strength of more than
\SI{1300}{kN/m^2}.  The mixture will not be used unless experimental evidence indicates conclusively that the strength
specification has been met.  Suppose compressive strength for specimens of this mixture is normally distributed with
\(\o=60\).  Let \(\m\) denote the true average compressive strength.
\begin{enumerate}[label={\alph*)}]
\item What are the appropriate null and alternative hypotheses?
  \[H_0:\m=1300\qquad H_a:\m<1300\]
\item Let \(\bar{X}\) denote the sample average compressive strength for \(n=10\) randomly selected specimens.  Consider the
  test procedure with test statistic \(\bar{X}\) itself (not standardized).  If \(\bar{x}=1360\), should \(H_0\) be
  rejected using a significance level of 0.01?  [Hint: What is the probability distribution of the test statistic when
    \(H_0\) is true?]
\end{enumerate}

\section*{8.13}

\section*{8.14}

\section*{8.19}

\end{document}
