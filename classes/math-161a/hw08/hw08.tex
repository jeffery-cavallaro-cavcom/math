\documentclass[letterpaper,12pt,fleqn]{article}
\usepackage{matharticle}
\usepackage{siunitx}
\pagestyle{plain}
\renewcommand{\a}{\alpha}
\newcommand{\m}{\mu}
\renewcommand{\o}{\sigma}
\renewcommand{\P}{\Phi}
\newcommand{\zadt}{z_{\frac{\a}{2}}}
\newcommand{\td}[2]{t_{{#1},{#2}}}
\newcommand{\X}{\chi}
\newcommand{\Xs}{\X^2}
\newcommand{\xd}[2]{\Xs_{{#1},{#2}}}
\allowdisplaybreaks
\begin{document}
Cavallaro, Jeffery \\
Math 161A \\
Homework \#8

\bigskip

\section*{7.2}

Each of the following is a confidence interval for \(\m=\) true average (i.e., population mean) resonance frequency (Hz) for
all tennis rackets of a certain type.
\[(114.4, 115.6)\qquad(114.1,115.9)\]
\begin{enumerate}[label={\alph*)}]
\item What is the value of the sample mean resonance frequency?
  \[\bar{x}=\frac{114.4+115.6}{2}=\frac{230}{2}=\SI{115.0}{Hz}\]
\item Both intervals were calculated from the same sample data.  The confidence level for one of these intervals is 90\% and
  for the other is 99\%.  Which of the intervals has the 90\% confidence level, and why?

  For the same sample data, the less-confident interval will be narrower.  Thus, the 90\% confidence interval is
  \((114.4, 115.6)\).
\end{enumerate}

\section*{7.3}

Suppose that a random sample of 50 bottles of a particular brand of cough syrup is selected and the alcohol content of each
bottle is determined.  Let \(\m\) denote the average alcohol content for the population of all bottles of the brand under
study.  Suppose that the resulting 95\% confidence interval is \((7.8,9.4)\).
\begin{enumerate}[label={\alph*)}]
\item Would a 90\% confidence interval calculated from this same sample have been narrower or wider than the given
  interval?  Explain your reasoning.

  A less confident interval for the same sample data results in a smaller \(1-\a\), which results in a larger \(\a\), which
  results in a smaller \(\zadt\), which results in a smaller \(m\) (margin of error).  Thus, the less confident interval is
  narrower.

\item Consider the following statement: There is a 95\% chance that \(\m\) is between 7.8 and 9.4.  Is this statement
  correct?  Why or why not.

  The statement is incorrect.  The \(m\) is fixed and the interval is probabilistic, yet probability is being applied to the
  fixed \(\m\) value.  When addressing the fixed (or an already determined) value, confidence should be used instead of
  probability.

\item Consider the following statement: We can be highly confident that 95\% of all bottles of this type of cough syrup
  have an alcohol content that is between 7.8 and 9.4.  Is this statement correct?  Why or why not?

  The statement is incorrect.  The given confidence interval applies to the mean for a sample of 50, not a single bottle.

\item Consider the following statement: If the process of selecting a sample of size 50 and then computing the corresponding
  95\% interval is repeated 100 times, 95 of the resulting intervals will include \(\m\).  Is this statement correct?  Why or
  why not?

  The statement is not quite correct.  We are highly confident that 95 of the resulting intervals will include \(\m\), but
  this is not guaranteed.
\end{enumerate}

\section*{7.4}

A CI is desired for the true average stray-load loss \(\m\) (watts) for a certain type of induction motor when the line
current is held at \SI{10}{amps} for a speed of \SI{1500}{rpm}.  Assume that stray-load loss is normally distributed with
\(\o=3\).
\begin{enumerate}[label={\alph*)}]
\item Compute a 95\% CI for \(\m\) when \(n=25\) and \(\bar{x}=58.3\).
  \begin{gather*}
    1-\a=0.95 \\
    \a=1-0.95=0.05 \\
    \frac{a}{2}=0.025 \\
    z_{0.025}=\P^{-1}(1-0.025)=\P^{-1}(0.975)=1.96 \\
    \\
    m=1.96\frac{3}{\sqrt{25}}=1.176\approx1.2 \\
    \\
    58.3\pm1.2=(57.1,59.5)
  \end{gather*}
\item Compute a 95\% CI for \(\m\) when \(n=100\) and \(\bar{x}=58.3\).
  \begin{gather*}
    m=1.96\frac{3}{\sqrt{100}}=0.588\approx0.6 \\
    \\
    58.3\pm0.6=(57.7,58.9)
  \end{gather*}
\item Compute a 99\% CI for \(\m\) when \(n=100\) and \(\bar{x}=58.3\).
  \begin{gather*}
    1-\a=0.99 \\
    \a=1-0.99=0.01 \\
    \frac{a}{2}=0.005 \\
    z_{0.005}=\P^{-1}(1-0.005)=\P^{-1}(0.995)=2.575 \\
    \\
    m=2.575\frac{3}{\sqrt{100}}=0.7725\approx0.8 \\
    \\
    58.3\pm0.8=(57.5,59.1)
  \end{gather*}
\item Compute a 82\% CI for \(\m\) when \(n=100\) and \(\bar{x}=58.3\).
  \begin{gather*}
    1-\a=0.82 \\
    \a=1-0.82=0.18 \\
    \frac{a}{2}=0.09 \\
    z_{0.09}=\P^{-1}(1-0.09)=\P^{-1}(0.91)=1.34 \\
    \\
    m=1.34\frac{3}{\sqrt{100}}=0.402\approx0.4 \\
    \\
    58.3\pm0.4=(57.9,58.7)
  \end{gather*}
\item How large must \(n\) be if the width of the 99\% interval for \(\m\) is to be 1.0?
  \[n=\left(2\cdot2.575\cdot\frac{3}{1}\right)^2=238.7\]
  Thus, the minimum required size is \(n=239\).
\end{enumerate}

\section*{7.7}

By how much must the sample size \(n\) be increased if the width of the CI (7.5) is to be halved?
\[n=\left(2\zadt\frac{\o}{\frac{w}{2}}\right)^2=\left(2\zadt\frac{2\o}{w}\right)^2=4\left(2\zadt\frac{\o}{w}\right)^2\]
Thus, \(n\) must be increased by a factor of 4.

If the sample size is increased by a factor of 25, what effect will this have on the width of the interval?
\[w=2\zadt\frac{\o}{\sqrt{25n}}=2\zadt\frac{\o}{5\sqrt{n}}=\frac{1}{5}\left(2\zadt\frac{\o}{\sqrt{n}}\right)\]
Thus, \(w\) will be decreased by a factor of 5.

\section*{7.8}

Let \(\a_1>0\), \(\a_2>0\), with \(\a_1+\a_2=\a\).  Then:
\[P\left(-z_{\a_1}<\frac{\bar{X}-\m}{\frac{\o}{\sqrt{n}}}<z_{\a_2}\right)=1-\a\]
\begin{enumerate}[label={\alph*)}]
\item Use this equation to derive a more general expression for a \(100(1-\a)\%\) CI for \(\m\) of which the interval (7.5) is
  a special case.
  \begin{gather*}
    P\left(-z_{\a_1}<\frac{\bar{X}-\m}{\frac{\o}{\sqrt{n}}}<z_{\a_2}\right)=1-\a \\
    P\left(-z_{\a_1}\frac{\o}{\sqrt{n}}<\bar{X}-\m<z_{\a_2}\frac{\o}{\sqrt{n}}\right)=1-\a \\
    P\left(-z_{\a_2}\frac{\o}{\sqrt{n}}<\m-\bar{X}<z_{\a_1}\frac{\o}{\sqrt{n}}\right)=1-\a \\
    P\left(\bar{X}-z_{\a_2}\frac{\o}{\sqrt{n}}<\m<z_{\a_1}\bar{X}+\frac{\o}{\sqrt{n}}\right)=1-\a \\
    P\left(\m\in\left(\bar{X}-z_{\a_2}\frac{\o}{\sqrt{n}},z_{\a_1}\bar{X}+\frac{\o}{\sqrt{n}}\right)\right)=1-\a
  \end{gather*}
\item Let \(\a=0.05\) and \(\a_1=\frac{\a}{4}\), \(\a_2=\frac{3\a}{4}\).  Does this result in a narrower or wider interval
  than the interval (7.5)?
  \begin{gather*}
    \a_1=\frac{0.05}{4}=0.0125 \\
    z_{0.0125}=\P^{-1}(1-0.0125)=\P^{-1}(0.9875)=2.24 \\
    \\
    \a_2=\frac{3\cdot0.05}{4}=0.0375 \\
    z_{0.0375}=\P^{-1}(1-0.0375)=\P^{-1}(0.9625)=1.78 \\
    \\
    \frac{\a}{2}=\frac{0.05}{2}=0.025 \\
    \zadt=\P^{-1}(1-0.025)=P^{-1}(0.975)=1.96
    \\
    (2.24+1.78)\frac{\o}{\sqrt{n}}=4.02\frac{\o}{\sqrt{n}} \\
    (2\cdot2.24)\frac{\o}{\sqrt{n}}=4.48\frac{\o}{\sqrt{n}}
  \end{gather*}
  Therefore, the skewed interval is narrower than the standard centered interval.
\end{enumerate}

\section*{7.28}

Determine the values for the following quantities:
\begin{enumerate}[label={\alph*)}]
\item \(\td{0.1}{15}=1.341\)
\item \(\td{0.05}{15}=1.753\)
\item \(\td{0.05}{25}=1.708\)
\item \(\td{0.05}{40}=1.684\)
\item \(\td{0.005}{40}=2.704\)
\end{enumerate}

\section*{7.30}

Determine the \(t\) critical value for a two-sided confidence interval in each of the following situations:
\begin{enumerate}[label={\alph*)}]
\item Confidence level=95\%, df=10
  \[\td{0.025}{10}=2.228\]
\item Confidence level=95\%, df=15
  \[\td{0.025}{15}=2.131\]
\item Confidence level=99\%, df=15
  \[\td{0.005}{15}=2.947\]
\item Confidence level=99\%, n=5
  \[\td{0.005}{4}=4.604\]
\end{enumerate}

\section*{7.31}

Determine the \(t\) critical value for a lower or an upper confidence bound for each of the situations described in
Exercise 30.
\begin{enumerate}[label={\alph*)}]
\item Confidence level=95\%, df=10
  \[\td{0.05}{10}=1.812\]
\item Confidence level=95\%, df=15
  \[\td{0.05}{15}=1.753\]
\item Confidence level=99\%, df=15
  \[\td{0.01}{15}=2.602\]
\item Confidence level=99\%, n=5
  \[\td{0.01}{4}=3.747\]
\end{enumerate}

\section*{7.32}

According to the article ``Fatigue Testing of Condoms'' (Polymer Testing, 2009: 567--571), ``tests currently used for condoms
are surrogates for the challenges they face in use,'' including a test for holes, an inflation test, a package seal test, and
tests of dimensions and lubricant quality (all fertile territory for the use of statistical methodology!).  The investigators
developed a new test that adds cyclic strain to a level well below breakage and determines the number of cycles to break.  A
sample of 20 condoms of one particular type resulted in a sample mean number of 1584 and a sample standard deviation of 607.
Calculate and interpret a confidence interval at the 99\% confidence level for the true average number of cycles to
break. [Note: The article presented the results of hypothesis tests based on the \(t\) distribution; the validity of these
  depends on assuming normal population distributions.]
\begin{gather*}
  1-\a=0.99 \\
  \a=1-0.99=0.01 \\
  \frac{\a}{2}=\frac{0.01}{2}=0.005 \\
  \td{0.005}{19}=2.861 \\
  \\
  \bar{x}\pm\td{0.005}{19}\frac{s}{\sqrt{n}}=1584\pm2.861\cdot\frac{607}{\sqrt{20}}\approx1584\pm388=(1196,1972)
\end{gather*}
Thus, we are 99\% confident that the true value lies within the interval \((1196,1972)\).

\section*{7.42}

Determine the values of the following quantities:
\begin{enumerate}[label={\alph*)}]
\item \(\xd{0.1}{15}=22.307\)
\item \(\xd{0.1}{25}=34.381\)
\item \(\xd{0.01}{25}=44.313\)
\item \(\xd{0.005}{25}=46.925\)
\item \(\xd{0.99}{25}=11.523\)
\item \(\xd{0.995}{25}=10.519\)
\end{enumerate}

\end{document}
