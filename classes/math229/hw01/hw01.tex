\documentclass[letterpaper,12pt,fleqn]{article}
\usepackage{matharticle}
\pagestyle{plain}
\renewcommand{\l}{\lambda}
\newcommand{\vx}{\vec{x}}
\newcommand{\vy}{\vec{y}}
\newcommand{\vv}{\vec{v}}
\newcommand{\vz}{\vec{0}}
\DeclareMathOperator{\Sp}{Sp}
\DeclareMathOperator{\Null}{Null}
\DeclareMathOperator{\Eig}{Eig}
\DeclareMathOperator{\rnk}{rank}
\DeclareMathOperator{\nullity}{nullity}
\newcommand{\yb}{\overline{y}}
\begin{document}
Cavallaro, Jeffery \\
Math 229 \\
Homework \#1

\bigskip

\subsection*{1.1.5}

Let $A\in M_n$ be idempotent. Show that each eigenvalue of $A$ is either
0 or 1. Explain why $I_n$ is the only non-singular idempotent matrix.

Assume $\l$ is an eigenvalue of $A$ \\
$\exists\,\vx\ne\vz,A\vx=\l\vx$ \\
$A(A\vx)=A(\l\vx)$ \\
$A^2\vx=\l(A\vx)$ \\
$A^2\vx=\l(\l\vx)$ \\
$A\vx=\l^2\vx$ \\
$\l^2\vx=\l\vx$ \\
$(\l^2-\l)\vx=\vz$ \\
$\l(\l-1)\vx=\vz$ \\
But $\vx$ is an eigenvector and thus $\vx\ne\vz$, so: \\
$\l(\l-1)=0$

$\therefore \l=0\ \mbox{or}\ 1$

We know that $(I_n)^2=I_n$ so $I_n$ is idempotent.
Also, $I_nI_n=I_n$, so $I_n$ is its own inverse and is thus non-singular.

Now, assume $A$ is idempotent. If $A$ is singular then done. So AWLOG that $A$
is non-singular (invertible):
\begin{eqnarray*}
  A^2 &=& A \\
  A^{-1}A^2 &=& A^{-1}A \\
  (A^{-1}A)A &=& I_n \\
  I_nA &=& I_n \\
  \therefore A &=& I_n
\end{eqnarray*}
  
Therefore $I_n$ is the only non-singular idempotent matrix.

\subsection*{1.2.21}

Let $A\in M_n$ and non-zero vectors $\vx,\vv\in\C^n$ be given. Suppose that
$c\in\C$, $\vv^{\,*}\vx=1$, $A\vx=\l\vx$, and $\Sp(A)=\{\l,\l_2,\ldots,\l_n\}$.
Define the Google matrix by:
\[A(c)=cA+(1-c)\l\vx\vv^{\,*}\]
Show that $\Sp(A(c))=\{\l,c\l_2,\ldots,c\l_n\}$.

$A\vx=\l\vx\iff(cA)\vx=(c\l)\vx$ \\
Therefore there is a one-to-one correspondence between the eigenvalues for
$A$ and those for $cA$, and so:
\[\Sp(cA)=\{c\l,c\l_2,\ldots,c\l_n\}\]
In class, we proved:
\[\Sp(A+\vx\vy^T)=\{\l+\vy^T\vx,\l_2,\ldots,\l_n\}\]
So taking $A=cA$, $\vx=(1-c)\l\vx$, and $y^T=\vv^{\,*}$, we have:
\[\Sp(A(c))=\Sp(cA+((1-c)\l\vx)(\vv^{\,*}))=
\{c\l+\vv^{\,*}(1-c)\l\vx,c\l_2,\ldots,c\l_n\}\]
Now, simplifying the expression for the first eigenvalue:
\begin{eqnarray*}
  c\l+\vv^{\,*}(1-c)\l\vx &=& c\l+(1-c)\l(\vv^{\,*}\vx) \\
  &=& c\l+(1-c)\l(1) \\
  &=& c\l+\l-c\l \\
  &=& \l
\end{eqnarray*}
Therefore:
\[\Sp(A(c))=\{\l,c\l_2,\ldots,c\l_n\}\]

\subsection*{1.3.7}

Show that every diagonalizable $B\in M_n$ has a square root.

Assume that $B$ is diagonalizable \\
Then $B$ is similar to some diagonal matrix $D$ such that for some invertible
matrix $S$:
\[B=SDS^{-1}\]
Now, let $C$ be a diagonal matrix such that $C_{kk}=\sqrt{D_{kk}}$:
\[C^2=D\]
And so:
\[B=SC^2S^{-1}=(SCS^{-1})(SCS^{-1})\]
Let $A=SCS^{-1}$

Therefore: $B=A^2$

Show that $B=\begin{bmatrix} 0 & 1 \\ 0 & 0 \end{bmatrix}$ does not have a
square root.

ABC: $B$ does have a square root
$A=\begin{bmatrix} a & b \\ c & d \end{bmatrix}$. Then:
\[A^2=\begin{bmatrix} a^2+bc & ab+bd \\ ca+dc & cb+d^2 \end{bmatrix}=
\begin{bmatrix} 0 & 1 \\ 0 & 0 \end{bmatrix}\]
Which results in the four equations:
\begin{eqnarray*}
  a^2+bc &=& 0 \\
  (a+d)b &=& 1 \\
  c(a+d) &=& 0 \\
  cb+d^2 &=& 0
\end{eqnarray*}
Assume $c=0$. Then $a=d=0$, which contradicts $(a+d)b=1$ \\
Assume $c\ne0$. Then $a+d=0$, which also contradicts $(a+d)b=1$

Therefore, no such $A$ exists.

\subsection*{1.3.13}

Let $A$ and $B$ be diagonalizable matrices. Prove:
\[A\sim B\iff p_A(t)=p_b(t)\]

Since $A$ is diagonalizable, there exists diagonal matrix $D_A$ and
invertible matrix $S_A$ such that $D_A=S_AAS_A^{-1}$. Similarly, there exists
diagonal matrix $D_B$ and invertible matrix $S_B$ such that $D_B=S_BBS_B^{-1}$.

\begin{description}
\item $\implies$ Assume $A\sim B$

  There exists invertible matrix $S$ such that $A=SBS^{-1}$
  \begin{eqnarray*}
    p_A(t) &=& \det(tI-A) \\
    &=& \det(tI-SBS^{-1}) \\
    &=& \det(tSS^{-1}I-SBS^{-1}) \\
    &=& \det(S(tI)S^{-1}-SBS^{-1}) \\
    &=& \det(S(tI-B)S^{-1}) \\
    &=& \det(S)\det(tI-B)\det(S^{-1}) \\
    &=& \det(S)\cdot\frac{1}{\det(S)}\cdot\det(tI-B) \\
    &=& \det(tI-B) \\
    &=& p_B(t)
  \end{eqnarray*}

\item $\impliedby$ Assume $p_A(t)=p_B(t)$
  \begin{eqnarray*}
    \det(tI-A) &=& \det(tI-B) \\
    \det(tI-S_A^{-1}D_AS_A) &=& \det(tI-S_B^{-1}D_BS_B) \\
    \det(tS_A^{-1}S_AI-S_A^{-1}D_AS_A) &=& \det(tS_B^{-1}S_BI-S_B^{-1}D_BS_B) \\
    \det(S_A^{-1}(tI)S_A-S_A^{-1}D_AS_A) &=& \det(S_B^{-1}(tI)S_B-S_B^{-1}D_BS_B) \\
    \det(S_A^{-1}(tI-D_A)S_A) &=& \det(S_B^{-1}(tI-D_B)S_B) \\
    \det(S_A^{-1})\det(tI-D_A)\det(S_A) &=& \det(S_B^{-1})\det(tI-D_B)\det(S_B) \\
    \frac{1}{\det(S_A^{-1})}\cdot\det(S_A)\cdot\det(tI-D_A) &=&
    \frac{1}{\det(S_B^{-1})}\cdot\det(S_B)\det(tI-D_B) \\
    \det(tI-D_A) &=& \det(tI-D_B)
  \end{eqnarray*}
  Thus, $\Sp(D_A)=\Sp(D_B)$; however, the matching eigenvalues may be in
  different positions on the diagonals of $D_A$ and $D_B$. But, using
  invertible permutation matrices, multiplication on the left (row) and
  right (column) can be used to reorder the diagonal of one to match the
  diagonal of the other. Also note that the product of invertible permutation
  matrices is still an invertible permutation matrix.

  So, AWLOG, there exists a product of permutation matrices $P$ such that:
  \[D_A=PD_BP^{-1}\]
  Substituting back we have:
  \begin{eqnarray*}
    S_AAS_A^{-1} &=& PS_BBS_B^{-1}P^{-1} \\
    A &=& (S_A^{-1}PS_B)B(S_B^{-1}P^{-1}S_A) \\
    A &=& (S_A^{-1}PS_B)B(S_A^{-1}PS_B)^{-1}
  \end{eqnarray*}
  Finally, let $C=S_A^{-1}PS_B$ \\
  $A=CBC^{-1}$

  $\therefore A\sim B$
\end{description}

The forward direction does not require diagonalizable; however, the reverse
direction does. For a counterexample, we need two matrices with the same characteristic
polynomial that are not similar.

Note that if two matrices are similar and if one is diagonalizable then the other must
also be diagonalizable. To prove this, assume that $A\sim B$ and AWLOG that $A$ is
diagonalizable:
\begin{eqnarray*}
  A &=& SBS^{-1} \\
  TDT^{-1} &=& SBS^{-1} \\
  D &=& T^{-1}SBS^{-1}T \\
  D &=& (T^{-1}S)B(T^{-1}S)^{-1} \\
\end{eqnarray*}
Therefore $B$ must also be diagonalizable.

Thus, we need to find two matrices with the same characteristic polynomial where only
one is diagonalizable. Consider:
\[A=\begin{bmatrix} 1 & 0 \\ 0 & 1 \end{bmatrix}\hspace{0.5in}
B=\begin{bmatrix} 1 & 1 \\ 0 & 1 \end{bmatrix}\]
Note that $\sigma(A)=\sigma(B)=\{1\}$ and $p_A(t)=p_B(t)=(t-1)^2$, so the characteristic
polynomials match. Furthermore, since $A=I_2$ and $I_2I_2I_2^{-1}=I_2$ it follows that
$A$ is diagonalizable.

To find $g_B(1)$, consider the nullity of $(B-I_2)$:
\[B-I_2=\begin{bmatrix} 0 & 1 \\ 0 & 0 \end{bmatrix}\]
So the nullity of $B-I_2$ is $1$ and thus $g_B(1)=1$. But, $a_B(1)=2$. Thus,
$g_B(1)\ne a_B(1)$ and therefore $B$ is not diagonalizable.

\subsection*{1.4.1}

Let non-zero vectors $\vx,\vy\in\C^n$ be given, let $A=\vx\vy^{\,*}$ and let
$\l=\vy^{\,*}\vx$.
\begin{enumerate}[label=\alph*)]
\item Show that $\l\in\sigma(A)$

  $A=\vx\vy^{\,*}$ \\
  $A\vx=\vx(\vy^{\,*}\vx)=\vx\begin{bmatrix}\l\end{bmatrix}=\l\vx$

  Therefore $\l\in\sigma(A)$.

\item Show that $\vx$ is a right eigenvector and $\vy$ is a left eigenvalue of
  $A$ associated with $\l$.

  By part (a), $\vx$ is a right eigenvector of $A$ associated with $\l$.

  $A=\vx\vy^{\,*}$ \\
  $\vy^{\,*}A=(\vy^{\,*}\vx)\vy^{\,*}=\l\vy^{\,*}$

  Therefore $\vy$ is a left eigenvector of $A$ associated with $\l$.

\item Show that if $\l\ne0$ then it is the only non-zero eigenvalue of $A$.

  Note that by the definition of matrix multiplication, we have:
  \[A=xy^{\,*}=\begin{bmatrix}\yb_1\vx\cdots\yb_n\vx\end{bmatrix}\]
  Thus, the columns of $A$ are all scalar multiples of $\vx$ and so $A$ is a
  rank-one matrix \\
  Next, note that $\Null(A)=\Eig_A(0)$, but by the dimension theorem we have:
  \[\nullity(A)=n-\rnk(A)=n-1\]
  and $\Eig_A(0)$ has $n-1$ independent vectors, or $g_A(0)=n-1$
  Finally, since $g_A(0)\le a_A(0)\le n$, it must be the case that:
  $a_A(0)=n-1$ or $a_A(0)=n$
  
  Therefore, if $\l\ne0$ then $a_A(\l)=1$.
\end{enumerate}

\end{document}
