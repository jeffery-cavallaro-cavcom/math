\documentclass[letterpaper,12pt,fleqn]{article}
\usepackage{matharticle}
\pagestyle{plain}
\renewcommand{\l}{\lambda}
\renewcommand{\o}{\sigma}
\newcommand{\w}{\omega}
\renewcommand{\a}{\alpha}
\DeclareMathOperator{\rnk}{rank}
\begin{document}
Cavallaro, Jeffery \\
Math 229 \\
Homework \#3

\bigskip

\subsection*{3.1.6}

$A=\begin{bmatrix} 1 & 1 \\ 1 & 1 \end{bmatrix}$

$\det(A-\l I)=\det\begin{bmatrix} 1-\l & 1 \\ 1 & 1-\l \end{bmatrix}=
(1-\l)^2-1=\l^2-2\l+1-1=\l^2-2\l=\l(\l-2)$

$\o(A)=\{0,2\}$

Start with $\l=0$:

$r_0(0)=\rnk(A^0)\rnk(I_2)=2$ \\
$r_1(0)=\rnk(A)=1$ \\
$r_2(0)=n-a(0)=2-1=1$
$r_3(0)=r_2(0)=1$

$b_1(0)=r_0-2r_1+r_2=2-2(1)+1=1$ \\
So $b_2(0)=0$ \\

Also, this means:

$b_1(2)=1$ \\
$b_2(2)=0$

And therefore:

$J_A=\begin{bmatrix} 0 & 0 \\ 0 & 2 \end{bmatrix}$

\bigskip

$B=\begin{bmatrix} 3 & 1 & 2 \\ 0 & 3 & 0 \\ 0 & 0 & 3 \end{bmatrix}$

$\o(B)=\{3\}$

$B-3I=\begin{bmatrix} 0 & 1 & 2 \\ 0 & 0 & 0 \\ 0 & 0 & 0 \end{bmatrix}$

$r_0(3)=\rnk(B-3I)^0=\rnk(I_3)=3$ \\
$r_1(3)=\rnk(B-3I)^1=1$ \\
$r_2(3)=\rnk(B-3I)^2=0$ \\
$r_3(3)=n-a(3)=3-3=0$ \\
$r_4(3)=r_3(3)=0$

$b_1(3)=r_0-2r_1+r_2=3-2(1)+0=1$ \\
So $b_2(3)=1$ and $b_3(3)=0$

$J_b=\begin{bmatrix} 3 & 1 & 0 \\ 0 & 3 & 0 & \\ 0 & 0 & 3 \end{bmatrix}$

\subsection*{3.2.7}

What are the possible Jordan forms of a matrix $A\in M_n$ such that $A^3=I$.

Since $A^3-I=0$, $t^3-1$ is an annihilating polynomial, and thus $q_A(t)$ must divide
it. Since:
\[t^3-1=(t-1)(t-\w)(t-\w^2)\]
Thus $\o(A)\subseteq\{1,\w,\w^2\}$ and the actual $q_A(t)$ must have some combination of
these linear factors, all with multiplicity of 1. Thus, the maximum Weyr index for any
eigenvalue is 1 and $J_A$ is diagonal with any combination of $1$, $\w$, and $\w^2$ as
diagonal values.

\subsection*{3.3.3}

Show that every protection (idempotent) matrix is diagonalizable. What is the minimum
polynomial of $A$? What can you say if $A$ is tripotent ($A^3=A$). What if $A^k=A$?

If $A^2=A$ then $t(t-1)$ is an annihilating polynomial for $A$ and thus $q_A(t)$ must be
some combination of these distinct linear factors with multiplicity 1. Thus, the
possibilities for $q_A(t)$ are $t$, $t-1$ and $t(t-1)$.

If $A^3=A$ then $t^3-t=t(t^2-1)=t(t-1)(t+1)$ and thus the possibilities for $q_A(t)$ are
any combination of the following linear factors with multiplicity 1: $t$, $t-1$, $t+1$.

To generalize, for $A^k=A$, $q_A(t)$ is any combination of linear factors with
multiplicity 1 from the set $\{t,t-\a\mid\a$ is a $(k-1)$-root of $1\}$.

\subsection*{3.3.9}

If $A\in M_5$ has $p_t(A)=(t-4)^3(t+6)^2$ and $q_A(t)=(t-4)^2(t+6)$, what is $J_A$?

$\o(A)=\{4,-6\}$ with $a_A(4)=3$ and $a_A(-6)=2$.

The highest non-zero Weyr index for $\l=4$ is $b_2$.

The highest non-zero Weyr index for $\l=-6$ is $b_1$.

And so:

$b_2(4)=1$ \\
$b_1(4)=1$ \\
$b_1(-6)=2$

$J_A=\begin{bmatrix}
4 & 1 & 0 & 0 & 0 \\
0 & 4 & 0 & 0 & 0 \\
0 & 0 & 4 & 0 & 0 \\
0 & 0 & 0 & -6 & 0 \\
0 & 0 & 0 & 0 & -6 \\
\end{bmatrix}$

\subsection*{3.3.31}

Show that there is no real $3\times3$ matrix whose minimal polynomial is $t^2+1$, but
that there is a real $2\times2$ matrix as well as a complex $3\times3$ matrix with this
property.

ABC: $A\in M_3(\R)$ is such a matrix.

Since $\deg(p_A(t))=3$ and since $q_A(t)$ divides $p_A(t)$, we know that $\pm i\in\o(A)$
and that:
\[p_A(t)=(t^2+1)(t-\a)=t^3-\a t^2+t-\a\]
But we know that $\a$ cannot be distinct from $\pm i$, otherwise, all three eigenvalues
would have to be present in $p_A(t)$ with linear factors. Thus, $a=\pm i$; however, that
would mean that $p_A(t)$ has complex coefficients, which cannot result from a matrix
with real components - CONTRADICTION!.

Therefore, no such $A$ exists.

Let $B=\begin{bmatrix} 1 & 2 \\ -1 & -1 \end{bmatrix}$

$B^2+I=0$, so $t^2+1$ is an annihilator polynomial for $B$. Furthermore, the only
possible linear cases would be $x+i$ and $x-i$, neither of which is an annihilator
for $B$, so $q_B(t)=t^2+1$.

Let $C=\begin{bmatrix} i & 0 & 0 \\ 0 & i & 0 \\ 0 & 0 & -i \end{bmatrix}$

$C^2+I=0$, so $t^2+1$ is an annihilator polynomial for $C$. Furthermore, the only
possible linear cases would be $x+i$ and $x-i$, neither of which is an annihilator
for $C$, so $q_C(t)=t^2+1$.
  
\end{document}
