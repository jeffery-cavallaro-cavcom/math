\documentclass[letterpaper,12pt,fleqn]{article}
\usepackage{matharticle}
\pagestyle{plain}
\renewcommand{\l}{\lambda}
\renewcommand{\o}{\sigma}
\newcommand{\conj}[1]{\overline{#1}}
\newcommand{\norm}[1]{\left\lVert#1\right\rVert}
\DeclareMathOperator{\rnk}{rank}
\DeclareMathOperator{\Sp}{Sp}
\begin{document}
Cavallaro, Jeffery \\
Math 229 \\
Homework \#6

\bigskip

\subsection*{7.1.1}

Let $A=[a_{ij}]\in M_n$ be positive semidefinite. Why is $a_{ii}a_{jj}\ge\abs{a_{ij}}^2$
for all distinct $i,j\in[n]$? If $A$ is positive definite, why is
$a_{ii}a_{jj}>\abs{a_{ij}}^2$? If there is a pair of distinct indices $i,j$ such that
$a_{ii}a_{jj}=\abs{a_{ij}}^2$, why is $A$ singular.

Assume $i,j\in[n]$ and AWLOG $i\le j$. Consider the $\{i,j\}$ principal submatrix for
$A$:
\[A_{ij}=\begin{bmatrix} a_{ii} & a_{ij} \\ a_{ji} & a_{jj} \end{bmatrix}\]
Note that $A_{ij}$ is also positive semidefinite and so $\det A_{ij}\ge0$. Hence:
\[a_{ii}a_{jj}-a_{ij}a_{ji}\ge0\]
But $A$ is also Hermitian, so $a_{ij}=\overline{a_{ji}}$, and so:
\[a_{ii}a_{jj}-\abs{a_{ij}}^2\ge0\]
and finally:
\[a_{ii}a_{jj}\ge\abs{a_{ij}}^2\]

For the positive definite case, use the same proof, only replace `$\ge$' with '$>$'.

Now assume $a_{ii}a_{jj}=\abs{a_{ij}}^2$. Thus, there exists a $2\times2$ principle
submatrix with a zero determinant. Use permutation matrices to make this submatrix
a leading principle submatrix (which does not affect the eigenvalues), and call it
$A_2$. Since $\det A_2=0$, we know that $0\in\o(A_2)$. Since $A_2$ is also positive
semidefinite, $\o(A_2)\subseteq[0,\infty)$ and thus we can conclude that $\l_1(A_2)=0$.

Now, using interlacing, it is the case that $0\le\l_1(A_3)\le\l_1(A_2)$ and so
$\l_1(A_3)=0$. This can be continued all the way to $A_n$ and thus $A$ has a 0
eigenvector.  Therefore, by the IMT, $A$ is singular.

\newpage

\subsection*{7.1.2}

Let $A$ be a positive semidefinite matrix. Prove:

$A$ has a zero entry on its main diagonal $\iff$ the corresponding entire row and
column are zero.

\begin{description}
\item $\implies$ Assume $a_{ii}=0$

  By the previous problem, for all $1\le j\le n$:
  \[a_{ii}a_{jj}=0\ge\abs{a_{ij}}^2\]
  And thus $a_{ij}=0$ for all $1\le j\le n$. But $A$ is Hermitian, so
  $a_{ij}=\conj{a_{ji}}=0$ and so $a_{ji}=0$ for all $1\le j\le n$. Therefore, the
  corresponding row and column are all zeros.

\item $\impliedby$ Assume for given $i$ and all $1\le j\le n$ that $a_{ij}=a_{ji}=0$

  Then clearly $a_{ii}=0$.
\end{description}

\subsection*{7.2.6}

Let $A\in M_n$ for $n\ge2$ be Hermitian and let $B\in M_{n-1}$ be a leading principal
submatrix of $A$. Prove: $B$ is positive semidefinite and $\rnk(B)=\rnk(A) \implies A$
is positive semidefinite.

Assume $B$ is positive semidefinite and $\rnk(B)=\rnk(A)$.

Since the ranks are equal but $\dim(A)=\dim(B)+1$, $A$ has one more zero singular
value than $B$. Since $B$ is positive semidefinite, and thus, Hermitian, and thus
normal, $s_k=\abs{\l_k(B)}$ and so $a_A(0)=a_B(0)+1$.

Now, by applying the interlacing theorem, we note that:
\[\l_1(A)\le\l_1(B)\le\l_2(A)\le\l_2(B)\le\cdots\le\l_{n-1}(B)\le\l_n(A)\]
However, since $B$ is positive semidefinite, all of the $\l_k(B)\ge0$, and thus
all of the $\l_k(A)\ge0$ for $2\le k\le n$. And, sinc e $\Sp(A)$ has an additional $0$,
we can conclude $\l_1=0$

Thus $A$ is Hermitian and $\Sp(A)\subseteq[0,\infty)$ and therefore $A$ is positive
semidefinite.

\subsection*{7.3.3}

Let $A=M_n$. Prove: $A$ has a zero singular value $\iff A$ has a zero eigenvector.

$A$ has a zero singular value $\iff\det(A)=0\iff A$ has a
zero eigenvector.

Alternately, consider the proof done in class. Let $S$ be the set of singular
matrices and use the operator norm to measure distance. Given a matrix $A$, the
distance between $A$ and $S$ is given by the smallest singular value. Thus, the
smallest singular values is zero $\iff$ $A\in S\iff A$ is singular $\iff A$ has a
zero eigenvalue.

\subsection*{7.4.13}

Let $\norm{\cdot}$ be a self-adjoint norm and let $H_n$ be the set of all $n\times n$
Hermitian matrices. Prove that the distance from a matrix $A$ to $H_n$ is given by:
\[d(A)=\frac{1}{2}\norm{A-A^*}\]

Assume $H\in H_n$:
\[\norm{A-H}=\frac{1}{2}\norm{A-H}+\frac{1}{2}\norm{A-H}=
\frac{1}{2}\norm{A-H}+\frac{1}{2}\norm{H-A}\]
But $\norm{\cdot}$ is self-adjoint by assumption, so:
\[\norm{A-H}=\frac{1}{2}\norm{A-H}+\frac{1}{2}\norm{(H-A)^*}=
\frac{1}{2}\norm{A-H}+\frac{1}{2}\norm{H^*-A^*}\]
But $H$ is Hermitian, so:
\[\norm{A-H}=\frac{1}{2}\norm{A-H}+\frac{1}{2}\norm{H-A^*}\]
Now, apply the triangle inequality:
\[\norm{A-H}\ge\norm{\frac{1}{2}(A-H)+\frac{1}{2}(H-A^*)}=\frac{1}{2}\norm{A-A^*}\]
So $d(A)\ge\frac{1}{2}\norm{A-A^*}$.

Now, consider $A+A^*$ and note that:
\[(A+A^*)^*=A^*+(A^*)^*=A^*+A=A+A^*\]
So $A+A^*$ is Hermitian, and so is $\frac{1}{2}(A+A^*)$, so let $H=\frac{1}{2}(A+A^*)$:
\[\norm{A-\frac{1}{2}(A+A^*)}=\norm{\frac{1}{2}(A-A^*)}=\frac{1}{2}\norm{A-A^*}\]
$\therefore d(A)=\frac{1}{2}\norm{A-A^*}$
\end{document}
