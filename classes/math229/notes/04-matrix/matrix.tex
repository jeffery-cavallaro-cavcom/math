\documentclass[letterpaper,12pt,fleqn]{article}
\usepackage{matharticle}
\pagestyle{empty}
\renewcommand{\a}{\alpha}
\renewcommand{\b}{\beta}
\newcommand{\g}{\gamma}
\begin{document}
\section*{Matrices}

\begin{definition}[Operations]
  \listbreak
  \begin{description}
  \item Addition: $A+B$
  \item Multiplication: $AB$
  \item Transpose: $A^T$
  \item Conjugate: $\bar{A}$
  \item Conjugate Transpose: $A^*=(\bar{A})^T$
  \end{description}
\end{definition}

\begin{example}
  $A=\begin{bmatrix} 1+i & 3 \\ 2i & 4 \end{bmatrix}$
  \hspace{4ex}
  $\bar{A}=\begin{bmatrix} 1-i & 3 \\ -2i & 4 \end{bmatrix}$
  \hspace{4ex}
  $A^*=\begin{bmatrix} 1-i & -2i \\ 3 & 4 \end{bmatrix}$
\end{example}

\subsection*{Submatrices}

\begin{definition}[Submatrix]
  Let $A\in M_{m,n}$. A \emph{submatrix} of $A$, denoted $A[\a.\b]$, is the
  matrix derived from $A$ by selecting elements row-indexed by $\a$ and
  column-indexed by $\b$, where $\a=\{i_1,i_2,\ldots,\i_r\}$ for
  $1\le r\le m$ and $\b=\{j_1,j_2,\ldots,\j_s\}$ for $1\le s\le n$.

  In particular, when $\a=\b$, $A[\a,\a]$ (also denoted $A[\a]$ is called a
  \emph{principal} submatrix of $A$.

  When $\a=\{1,2,\ldots,k\}$ for $1\le k\le m,n$ then $A[\a]$ is called a
  \emph{leading} principal submatrix of $A$.
\end{definition}

\begin{example}
  $A=\begin{bmatrix}
  1 & 2 & 3 & 4 & 5 \\
  6 & 7 & 8 & 9 & 10 \\
  11 & 12 & 13 & 14 & 15 \\
  16 & 17 & 18 & 19 & 20
  \end{bmatrix}$

  $\a=\{1,2,4\}$ and $\b=\{2,5\}$ and $\g=\{1,2,3\}$

  $A[\a,\b]=\begin{bmatrix}
  2 & 5 \\
  7 & 10 \\
  17 & 20
  \end{bmatrix}$
  \hspace{4ex}
  $A[\a]=\begin{bmatrix}
    1 & 2 & 4 \\
    6 & 7 & 9 \\
    16 & 17 & 19
  \end{bmatrix}$
  \hspace{4ex}
  $A[\g]=\begin{bmatrix}
    1 & 2 & 3 \\
    6 & 7 & 8 \\
    11 & 12 & 13
  \end{bmatrix}$
\end{example}
\newpage
\subsection*{Minors and Cofactors}

\begin{definition}[Minor]
  Let $A\in M_n$. A \emph{minor} of $A$ is the determinant of a submatrix:
  \[\det A[\a,\b]\]
  The minor generated by removing the $i^{th}$ row and $j^{th}$ column is
  denoted $A_{ij}$.

  A \emph{cofactor} of $A$ is a signed minor given by:
  \[(-1)^{i+j}\det A_{ij}\]
\end{definition}

\begin{example}
  $A=\begin{bmatrix}
  1 & 2 & 3 \\
  4 & 5 & 6 \\
  7 & 8 & 9
  \end{bmatrix}$

  $A_{1,2}=\begin{bmatrix}
  4 & 6 \\
  7 & 9
  \end{bmatrix}$

  $\det A_{1,2}=36-42=-6$

  $(-1)^{1+2}\det A_{i,j}=(-1)^3(-6)=6$
\end{example}

\end{document}
