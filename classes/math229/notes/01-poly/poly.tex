\documentclass[letterpaper,12pt,fleqn]{article}
\usepackage{matharticle}
\pagestyle{empty}
\newcommand{\conj}[1]{\overline{#1}}
\begin{document}
\section*{Polynomials}

For matrices, any field will work (e.g., $\Q$, $\R$, $\C$, any finite
field); however, the default scalar field for this class will be $\C$.

Give $z=a+ib\in\C$, there are two important operations:
\begin{enumerate}
\item Conjugation: $\conj{z}=a-ib$
\item Modulus: $\abs{z}=\sqrt{a^2+b^2}=\sqrt{z\conj{z}}$
\end{enumerate}

An alternate representation using Euler's formula:
\[z=\abs{z}e^{i\theta}=\abs{z}(\cos{\theta}+i\sin{\theta})\]

\begin{definition}
  A \emph{polynomial} in a complex variable $z$ is of the form:
  \[p(z)=\sum_{k=0}^na_kz^k,\hspace{4ex}a_k\in\C, a_n\ne0\]
  The \emph{degree} of $p(z)=n$.

  To say that $p(z)$ is \emph{monic} means $a_n=1$.
\end{definition}

\begin{definition}
  To say that $z_0$ is a \emph{zero} of the polynomial $p(z)$ means:
  \[p(z_0)=0\]
\end{definition}

\begin{theorem}[Fundamental Theorem of Algebra]
  Every non-constant polynomial has at least one zero.
\end{theorem}

\begin{corollary}
  A monic polynomial of degree $n\ge1$ has exactly $n$ zeros, including multiplicity:
  \[p(z)=\sum_{k=0}^na_kz^k=\prod_{k=1}^n(z-\lambda_k)\]
\end{corollary}

\begin{theorem}
  Let $p(z)=\sum_{k=0}^na_kz^k$ be a polynomial of degree $n\ge1$:
  \[a_{n-k}=(-1)^k\sum_{\mathcal{P}_k[n]}\prod_{i=1}^k\lambda_i\]
  where $\mathcal{P}_k[n]$ are all $k$ subsets of $\{1,2,\ldots,n\}$.

  In particular:
  \begin{eqnarray*}
    a_{n-1} &=& -\sum_{k=1}^n\lambda_k \\
    a_0 &=& (-1)^n\prod_{k=1}^n\lambda_k
  \end{eqnarray*}
\end{theorem}

\begin{corollary}
  Let $p(z)$ be a polynomial with real coefficients:
  \[p(z_0)=0\implies p(\conj{z_0})=0\]
  In other words, all complex zeros must occur in conjugate pairs.
\end{corollary}

\begin{theproof}
  Assume $p(z_0)=0$ \\
  $\conj{p(z_0)}=\conj{0}=0$

  $\therefore p(\conj{z_0})=0$
\end{theproof}

\end{document}
