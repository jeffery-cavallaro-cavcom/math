\documentclass[letterpaper,12pt,fleqn]{article}
\usepackage{matharticle}
\pagestyle{empty}
\newcommand{\vx}{\vec{x}}
\newcommand{\vy}{\vec{y}}
\newcommand{\vz}{\vec{0}}
\DeclareMathOperator{\Null}{Null}
\DeclareMathOperator{\nullity}{nullity}
\DeclareMathOperator{\range}{range}
\DeclareMathOperator{\rnk}{rank}
\begin{document}
\section*{Rank}

\begin{definition}[Nullity and Rank]
  Let $A\in M_{n,m}$:
  \begin{itemize}
  \item The \emph{null space} of $A$ is given by:
    \[\Null(A)=\{\vx\in\C^n\mid A\vx=\vz\}\]

  \item The \emph{nullity} of $A$ is given by:
    \[\nullity(A)=\dim\Null(A)\]

  \item The \emph{range} of $A$, denoted $\range(A)$, is the space spanned by
    the columns of $A$.

  \item The \emph{rank} of $A$ is given by:
    \[\rnk(A)=\dim\range(A)\]
  \end{itemize}
\end{definition}

\begin{theorem}[Dimension Theorem]
  Let $A\in M_{m,n}$:
  \[n=\rnk(A)+\nullity(A)\]
  where $\rnk(A)$ equals the number of pivots in the REF of $A$ and
  $\nullity(A)$ equals the number of free variables in the REF of $A$.
\end{theorem}

Note that the transformation $T:\C^n\to\C^m$ defined by $T(\vx)=A\vx$ is a
linear transformation.

\begin{theorem}
  Let $T:\C^n\to\C^n$ be the linear transformation defined by $T(\vx)=A\vx$ for
  some $A\in M_{m,n}$:
  \[T\ \mbox{is injective} \iff \Null(A)\ \mbox{is trivial}\]
\end{theorem}

\begin{theproof}
  \listbreak
  \begin{description}
  \item $\implies$ Assume $T$ is injective

    Assume $T(\vx)=T(\vy)$ \\
    $T(\vx)-T(\vy)=\vz$ \\
    $T(\vx-\vy)=\vz$ \\
    But $T$ is injective, so $\vx=\vy$ \\
    Thus, $T(\vz)=\vz$ \\
    Furthermore, since $T$ is injective, no other element in the domain may
    map to $\vz$

    Therefore $\Null(A)$ is trivial.

  \item $\impliedby$ Assume $\Null(A)$ is trivial

    Assume $T(\vx)=T(\vy)$ \\
    $T(\vx)-T(\vy)=\vz$ \\
    $T(\vx-\vy)=\vz$ \\
    But the null space is trivial, and so $\vx-\vy=\vz$ \\
    and so $\vx=\vy$

    Therefore $T$ is injective.
  \end{description}
\end{theproof}

\begin{theorem}
  Let $T:\C^n\to\C^n$ be the linear transformation defined by $T(\vx)=A\vx$ for
  some $A\in M_{m,n}$:
  \[T\ \mbox{is injective} \iff T\ \mbox{is surjective}\]
\end{theorem}

\begin{theproof}
  $T$ is injective \\
  $\iff$ the null space is trivial \\
  $\iff \nullity(A)=0$ \\
  $\iff \rnk(A)=n-0=n$ \\
  $\iff $ the column space of $A$ spans $C^n$ \\
  $\iff T$ is surjective.
\end{theproof}

\begin{lemma}
  Let $A\in M_{m,k}$ and $B\in M_{k,n}$:
  \[\rnk(AB)\le\rnk(A)\]
\end{lemma}

\begin{theproof}
  $\range(AB)=\{(AB)\vx|\vx\in\C^n\}=\{A(B\vx)|\vx\in\C^n\}\subseteq\range(A)$

  $\therefore \rnk(AB)\le\rnk(A)$
\end{theproof}

\begin{theorem}
  Let $A\in M_{m,n}$:
  \[\rnk(A)=\rnk(A^T)\]
  Thus, the dimension of the column space equals the dimension of the row
  space.
\end{theorem}
\newpage
\begin{theproof}
  Let $\rnk(A)=r\le n$ \\
  Thus, only $r$ of the $n$ columns of $A$ are linearly independent \\
  So construct $B\in M_{m,r}$ from the linearly independent columns of $A$ \\
  Assume $A=BX$ for some $X\in M_{r,n}$ \\
  $\rnk(A^T)=\rnk((BX)^T)=\rnk(X^TB^T)\le\rnk(X^T)$ \\
  But $X^T\in M_{n,r}$ and so $\rnk(X^T)\le r$ \\
  So $\rnk(A^T)\le r=\rnk(A)$ \\
  But since $(A^T)^T=A$, $\rnk(A)\le\rnk(A^T)$

  $\therefore \rnk(A)=\rnk(A^T)$
\end{theproof}

\end{document}
