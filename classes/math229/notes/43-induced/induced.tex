\documentclass[letterpaper,12pt,fleqn]{article}
\usepackage{matharticle}
\pagestyle{empty}
\newcommand{\inner}[2]{\left<#1,#2\right>}
\newcommand{\norm}[1]{\left\lVert#1\right\rVert}
\newcommand{\vx}{\vec{x}}
\newcommand{\vy}{\vec{y}}
\newcommand{\ve}{\vec{e}}
\begin{document}
\section*{Inner Product Induced Norm}

\begin{definition}[Inner Product Induced Norm]
  Let $V$ be a vector space equipped with inner product $\inner{\cdot}{\cdot}$
  and norm $\norm{\cdot}$. To say that the norm is an
  \emph{inner product-induced norm} means:
  \[\norm{\vx}=\inner{\vx}{\vx}^{\frac{1}{2}}\]
\end{definition}

\begin{example}
  The $\ell_2$ norm is an inner product-induced norm because:
  \[\norm{\vx}_2=\left(\sum_{k=1}^n\abs{x_k}^2\right)^{\frac{1}{2}}=
  (\vx^{\,*}\vx)^{\frac{1}{2}}=\inner{\vx}{\vx}^{\frac{1}{2}}\]
\end{example}

\begin{theorem}
  A norm is inner product-induced iff it satisfies the parallelogram identity:
  \[\norm{\vx+\vy}^2+\norm{\vx-\vy}^2=2(\norm{\vx}^2+\norm{\vy}^2)\]
\end{theorem}

\begin{theproof}
  \listbreak
  \begin{description}
  \item Assume $\norm{\cdot}$ is inner product-induced
    \begin{eqnarray*}
      \norm{\vx+\vy}^2+\norm{\vx-\vy} &=&
      \inner{\vx+\vy}{\vx+\vy}+\inner{\vx-\vy}{\vx-\vy} \\
      &=& \inner{\vx}{\vx}+\inner{\vx}{\vy}+\inner{\vy}{\vx}+\inner{\vy}{\vy}+
      \inner{\vx}{\vx}-\inner{\vx}{\vy}-\inner{\vy}{\vx}+\inner{\vy}{\vy} \\
      &=& 2\inner{\vx}{\vx}+2\inner{\vy}{\vy} \\
      &=& 2(\inner{\vx}{\vx}+\inner{\vy}{\vy}) \\
      &=& 2(\norm{\vx}^2+\norm{\vy}^2) \\
    \end{eqnarray*}
  \end{description}
\end{theproof}

Note that the $\ell_1$ normal fails the parallelogram identity and thus is not
inner product-induced:

Let $\vx=\ve_1$ and $\vy=\ve_2$:
\[\norm{\ve_1+\ve_2}_1=
\norm{\begin{bmatrix} 1 \\ 1 \\ 0 \\ \vdots \\ 0 \end{bmatrix}}^2+
\norm{\begin{bmatrix} 1 \\ -1 \\ 0 \\ \vdots \\ 0 \end{bmatrix}}^2=
2^2+2^2=4+4=8\]
\[2(\norm{\ve_1}_1^2+\norm{\ve_2}_1^2)=2(1^2+1^2)=2(1+1)=2(2)=4\]

\end{document}
