\documentclass[letterpaper,12pt,fleqn]{article}
\usepackage{matharticle}
\pagestyle{empty}
\renewcommand{\o}{\sigma}
\renewcommand{\l}{\lambda}
\newcommand{\m}{\mu}
\newcommand{\vx}{\vec{x}}
\newcommand{\vz}{\vec{z}}
\DeclareMathOperator{\Eig}{Eig}
\begin{document}
\section*{Commute Conditions}

\begin{definition}[Simultaneously Triangularizable]
  Let $A,B\in M_n$. To say that $A$ and $B$ are simultaneously triangularizable means
  there exists unitary matrix $U$ such that:
  \[U^*AU=T_A\ \mbox{and} \ U^*BU=T_B\]
  where $T_A,T_B\in UT(n)$.
\end{definition}

\begin{theorem}
  Let $A,B\in M_n$:

  $AB=BA\implies A$ and $B$ are simultaneously triangularizable.
\end{theorem}

\begin{definition}
  Let $S,T\subseteq\C$:
  \[S-T=\{s-t\mid s\in S\ \mbox{and}\ t\in T\}\]
\end{definition}

\begin{theorem}
  Let $A,B\in M_n$:
  \[AB=BA\implies \o(A-B)\subseteq\o(A)-\o(B)\]
\end{theorem}

\begin{example}
  \newcommand{\ea}{\begin{bmatrix} 3 & 1 \\ -2 & 0 \end{bmatrix}}
  \newcommand{\eb}{\begin{bmatrix} 1 & 0 \\ 0 & 0 \end{bmatrix}}

  Thus, $\o(A-B)\not\subseteq\o(A)-\o(B)\implies AB\ne BA$

  $A=\ea$ and $B=\eb$

  $\o(A)=\{1,2\}$ and $\o(B)={0,1}$

  $\o(A)-\o(B)=\{1-0,1-1,2-0,2-1\}=\{0,1,1,2\}=\{0,1,2\}$

  $A-B=\begin{bmatrix} 2 & 1 \\ -2 & 0 \end{bmatrix}$

  $p_{A-B}(t)=\det\begin{bmatrix} t-2 & 1 \\ -2 & t \end{bmatrix}=t(t-2)+2=t^2-2t+2$

  $\o(A-B)=\{1\pm2i\}\not\subseteq\{0,1,2\}=\o(A)-\o(B)$

  So $A$ and $B$ do not commute:

  $\ea\eb=\begin{bmatrix} 3 & 0 \\ -2 & 0 \end{bmatrix}$
  
  $\eb\ea=\begin{bmatrix} 3 & 1 \\ 0 & 0 \end{bmatrix}$

  $AB\ne BA$
\end{example}

\begin{theproof}
  Assume $AB=BA$ \\
  $A$ and $B$ are simultaneously triangularizable \\
  Let $U$ be the necessary unitary matrix \\
  $U^*(A-B)U=U^*AU-U^*BU=T_A-T_B$ \\
  So $\o(A-B)$ is from the diagonal entries of $T_A-T_B$, which is a subset of all
  possible differences, represented by $\o(A)-\o(B)$

  $\therefore\o(A-B)\subseteq\o(A)-\o(B)$
\end{theproof}

\begin{theorem}
  Let $A,B\in M_n$:

  $AB=BA\implies A$ and $B$ have a common eigenvector.
\end{theorem}

\begin{theproof}
  Let $\l\in\o(A)$ with eigenvector $\vx$ \\
  $A\vx=\l\vx$ \\
  Consider $B\vx$
  \begin{description}
  \item Case 1: $B\vx=0$

    $\vx\in\Eig_B(0)$

    Therefore $\vx$ is a common eigenvector.

  \item Case 2: $B\vx\ne0$

    $A(B\vx)=BA\vx=B(\l\vx)=\l(B\vx)$ \\
    So $B\vx\in\Eig_A(\l)$ \\
    Thus $\{B\vx\mid\vx\in\Eig_A(\l)\}\subseteq\Eig_A(\l)$ \\
    Let $\{\vx_1,\ldots,\vx_r\}$ be a basis for $\Eig_A(\l)$ \\
    Extend the set to a basis for $\C^n: \{\vx_1,\ldots,\vx_r,\vx_{r+1},\ldots,\vx_n\}$ \\
    AWLOG that this is an orthonormal basis (otherwise use Gram-Schmidt) \\
    Let $U=\begin{bmatrix} \vx_1 & \cdots & \vx_r & \vx_{r+1} & \ldots & \vx_n
    \end{bmatrix}$, thus a unitary matrix
    \[BU=\begin{bmatrix} B\vx_1 & \cdots & B\vx_r & B\vx_{r+1} & \cdots & B\vx_n
    \end{bmatrix}\]
    But note that:
    \[B\vx_k\in\begin{cases}
    \Eig_A(\l) & 1\le k\le r \\
    \C^n & r+1\le k\le n
    \end{cases}\]
    and so:
    \begin{eqnarray*}
      BU &=&
      \begin{bmatrix} \vx_1 & \cdots & \vx_r & \vx_{r+1} & \ldots & \vx_n\end{bmatrix}
        \begin{bmatrix}
          k_{1,1} & k_{1,2} & \cdots & k_{1,r} & k_{1,r+1} & \cdots & k_{1,n} \\
          k_{2,1} & k_{2,2} & \cdots & k_{2,r} & k_{2,r+1} & \cdots & k_{2,n} \\
          & & \vdots & & & \vdots & \\
          k_{r,1} & k_{r,2} & \cdots & k_{r,r} & k_{r,r+1} & \cdots & k_{r,n} \\
          0 & 0 & \cdots & 0 & k_{r+1,r+1} & \cdots & k_{r+1,n} \\
          & & \vdots & & & \vdots & \\
          0 & 0 & \cdots & 0 & k_{n,r+1} & \cdots & k_{n,n}
        \end{bmatrix} \\
        &=& U\begin{bmatrix} C_{11} & C_{12} \\ 0 & C_{22} \end{bmatrix}
    \end{eqnarray*}
    where $C_{11},C_{12}\in M_r$ and $C_{22}\in M_{n-(r+1)}$\\
    Select an eigenvector $\vz$ of $C_{11}$ with respect to eigenvalue $\m$ \\
    $C_{11}\vz=\m\vz$ \\
    Consider $U\begin{bmatrix} \vz \\ 0 \end{bmatrix}$ \\
    $U\begin{bmatrix} \vz \\ 0 \end{bmatrix}$ is a linear combination of
    $\{\vx_1,\ldots,\vx_r\}$ and is thus in $\Eig_A(\l)$ \\
    But also:
    \begin{eqnarray*}
      B\left(U\begin{bmatrix} \vz \\ 0 \end{bmatrix}\right) &=&
      U\begin{bmatrix} C_{11} & C_{12} \\ 0 & C_{22} \end{bmatrix}
      \begin{bmatrix} \vz \\ 0 \end{bmatrix} \\
      &=& U\begin{bmatrix} C_{11}\vz \\ 0 \end{bmatrix} \\
      &=& U\begin{bmatrix} \m\vz \\ 0 \end{bmatrix} \\
      &=& \m\left(U\begin{bmatrix} \vz \\ 0 \end{bmatrix}\right) \\
    \end{eqnarray*}
    Therefore $U\begin{bmatrix} \vz \\ 0 \end{bmatrix}\in\Eig_B(\m)$
  \end{description}
\end{theproof}

\end{document}
