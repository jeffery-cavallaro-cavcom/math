\documentclass[letterpaper,12pt,fleqn]{article}
\usepackage{matharticle}
\pagestyle{empty}
\newcommand{\vx}{\vec{x}}
\newcommand{\vxct}{\vx^{\,*}}
\newcommand{\vy}{\vec{y}}
\newcommand{\vyct}{\vy^{\,*}}
\newcommand{\vu}{\vec{u}}
\newcommand{\vv}{\vec{v}}
\newcommand{\vvct}{\vv^{\,*}}
\newcommand{\vw}{\vec{w}}
\newcommand{\vz}{\vec{0}}
\renewcommand{\l}{\lambda}
\newcommand{\m}{\mu}
\DeclareMathOperator{\Sp}{Sp}
\DeclareMathOperator{\spn}{span}
\begin{document}
\section*{Interlacing Inequalities}

\begin{theorem}[Rank-One Interlace]
  Let $A,B\in M_n$ such that $B=\vx\vxct$ for some $\vx\in\C^n$:
  \[\l_1(A)\le\l_1(A+B)\le\l_2(A)\le\l_2(A+B)\le\cdots\le\l_n(A)\le\l_n(A+B)\]
\end{theorem}

\begin{theproof}
  Recall that $\Sp(B)=\Sp(\vx\vxct)=\{0^{(n-1)},\vxct\vx\}$.
  
  From Weyl with $j=i$ and $k=1$:
  \[\l_i(A)+\l_1(B)\le\l_i(A+B)\]
  But $\l_1(B)=0$
  
  $\therefore\l_i(A)\le\l_i(A+B)$

  Now, from Weyl with $j=i+1$ and $k=n-1$:
  \[\l_i(A+B)\le\l_{i+1}(A)+\l_{n-1}(B)\]
  But $\l_{n-1}(B)=0$
  
  $\therefore\l_i(A+B)\le\l_{i+1}(A)$
\end{theproof}

Note that the converse is also true: Given interlacing sets of $n$ numbers $\{\l_k\}$ and
$\{\mu_k\}$, there exists Hermitian matrices $A$ and $B=\vx\vxct$ with these interlacing
sets as their respective eigenvalues (proof omitted). To find such a $\vx$, solve the
following SOLE:
\[\Sp\left\{\begin{bmatrix}\l_1 & & 0 \\
& \ddots & \\
0 & & \l_n
\end{bmatrix}+\vx\vxct\right\}=\{\mu_k\mid 1\le k\le n\}\]

\begin{theorem}
  Let $A\in M_n$ be Hermitian and let $B$ be the leading principal submatrix of $A$
  (also Hermitian):
  \[\l_1(A)\le\l_1(B)\le\l_2(A)\le\l_2(B)\le\cdots\le\l_{n-1}(A)\le\l_{n-1}(B)\le\l_n(A)\]
\end{theorem}

\begin{example}
  \begin{minipage}{1.5in}
    $A=\begin{bmatrix} 1 & 2 \\ 2 & 4 \end{bmatrix}$
  \end{minipage}
  \begin{minipage}{1in}
    $B=\begin{bmatrix} 1 \end{bmatrix}$
  \end{minipage}
  \begin{minipage}{2in}
    \begin{eqnarray*}
      (1-\l)(4-\l) &=& 2(2) \\
      4-5\l+\l^2 &=& 4 \\
      \l^2-5\l &=& 0 \\
      \l(\l-5) &=& 0 \\
      \l &=& 0,5
    \end{eqnarray*}
  \end{minipage}
  
  $0\le1\le5$
\end{example}

\begin{theproof}
  Let $A=\begin{bmatrix} B & \vy \\ \vyct & a \end{bmatrix}$ where $B$ is the leading
  principal submatrix of $A$, $\vy\in\C^{n-1}$, and $a\in\C$.

  Let $\{\vu_1,\ldots,\vu_n\}$ be eigenvectors of $A$ corresponding to
  $\l_1(A)\le\cdots\le\l_n(A)$. \\
  Let $\{\vv_1,\ldots,\vv_{n-1}\}$ be eigenvectors of $B$ corresponding to
  $\l_1(B)\le\cdots\le\l_{n-1}(B)$. \\
  Let $\vw_k=\begin{bmatrix} \vv_k \\ 0 \end{bmatrix}$ for $1\le k\le n-1$.

  Let $S_1=\spn\{\vu_i,\ldots\vu_n\}$ \\
  Let $S_2=\spn\{\vw_1,\ldots\vw_i\}$ \\
  $\dim(S_1\cap S_2)\ge\dim(S_1)+\dim(S_2)-n=(n-i+1)+i-n=1$ \\
  Thus, there exists $\vx\in S_1\cap S_2$ such that $\vx\ne\vz$

  Let $\vx\in S_1\cap S_2$ such that $\vx=\begin{bmatrix} \vv \\ 0 \end{bmatrix}$ where
  $\vv\in\Sp\{\vv_1,\ldots,\vv_i\}$. \\
  Since $\vx\in S_1$, by the key lemma:
  \[\l_i(A)\le\frac{\vxct A\vx}{\vxct\vx}\]
  Since $\vv\in\Sp\{\vv_1,\ldots,\vv_i\}$, by the key lemma:
  \[\frac{\vvct B\vv}{\vvct\vv}\le\l_i(B)\]
  But:
  \[\frac{\vxct A\vx}{\vxct\vx}=
  \frac{\begin{bmatrix} \vvct & 0 \end{bmatrix}
    \begin{bmatrix} B & \vy \\ \vyct & a \end{bmatrix}
    \begin{bmatrix} \vv \\ 0 \end{bmatrix}
  }{\begin{bmatrix} \vvct & 0 \end{bmatrix}\begin{bmatrix} \vv \\ 0 \end{bmatrix}}=
  \frac{\begin{bmatrix} \vvct B & \vvct\vy \end{bmatrix}
    \begin{bmatrix} \vv \\ 0 \end{bmatrix}
  }{\vvct\vx}=
  \frac{\vvct B\vv}{\vvct\vx}
  \]
  Therefore:
  \[\l_i(A)\le\frac{\vxct A\vx}{\vxct\vx}=\frac{\vvct B\vv}{\vvct\vx}\le\l_i(B)\]

  Let $S_3=\spn\{\vu_1,\ldots,\vu_{i+1}\}$ \\
  Let $S_4=\spn\{\vw_i,\ldots,\vw_{n-1}\}$ \\
  $\dim(S_3\cap S_4)\ge\dim(S_3)+\dim(S_4)-n=(i+1)+[(n-1)-i+1]-n=1$ \\
  Thus, there exists $\vx\in S_3\cap S_4$ such that $\vx\ne\vz$

  Let $\vx\in S_3\cap S_4$ such that $\vx=\begin{bmatrix} \vv \\ 0 \end{bmatrix}$ where
  $\vv\in\Sp\{\vv_i,\ldots,\vv_{n-1}\}$. \\
  Since $\vx\in S_3$, by the key lemma:
  \[\frac{\vxct A\vx}{\vxct\vx}\le\l_{i+1}(A)\]
  Since $\vv\in\Sp\{\vv_i,\ldots,\vv_{n-1}\}$, by the key lemma:
  \[\l_i(B)\le\frac{\vvct B\vv}{\vvct\vv}\]
  Therefore:
  \[\l_i(B)\le\frac{\vvct B\vv}{\vvct\vx}=\frac{\vxct A\vx}{\vxct\vx}\le\l_{i+1}(A)\]
\end{theproof}

Note that the converse is also true: Given interlacing sets of $n$ numbers
$\{\l_k\mid 1\le k\le n\}$ and $\{\mu_k\mid 1\le k\le n-1\}$, there exists Hermitian
matrix $A$ with leading princple submatrix $B$ with these interlacing sets as their
respective eigenvalues (proof omitted). To find such an $A$, solve the
following SOLE for a suitable $\vy\in\C^{n-1}$:
\[\Sp\left\{\left[\begin{array}{ccc|c}
    \mu_1 & & 0 & y_1 \\
    & \ddots & & \vdots \\
    0 & & \mu_{n-1} & y_{n-1} \\
    \hline
    \bar{y}_1 & \cdots & \bar{y}_{n-1} & \mu'
  \end{array}\right]\right\}=\{\l_1,\ldots,\l_n\]
where:
\[\mu'=\sum_{k=1}^n\l_k-\sum_{k=1}^{n-1}\m_k\]
\end{document}
