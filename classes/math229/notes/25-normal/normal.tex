\documentclass[letterpaper,12pt,fleqn]{article}
\usepackage{matharticle}
\pagestyle{empty}
\newcommand{\vx}{\vec{x}}
\newcommand{\vxct}{\vx^{\,*}}
\newcommand{\vz}{\vec{0}}
\renewcommand{\l}{\lambda}
\newcommand{\conj}[1]{\overline{#1}}
\DeclareMathOperator{\tr}{tr}
\DeclareMathOperator{\Sp}{Sp}
\begin{document}
\section*{Normal Matrices}

\begin{definition}
  Let $A\in M_n$. To say that $A$ is \emph{normal} means:
  \[AA^*=A^*A\]
\end{definition}

Examples:
\begin{enumerate}
\item Unitary ($UU^*=I$)
\item Hermitian ($H^*=H$)
\item Skew-Hermitian ($H^*=-H$)
\item Positive Definite ($\forall\vx\in\C^n-\{\vx\},x^*Ax>0$)
\item Positive Semidefinite ($\forall\vx\in\C^n,x^*Ax\ge0$)
\end{enumerate}

\begin{lemma}
  Let $T\in UT(n)$:
  \[T\ \mbox{normal}\implies T\ \mbox{diagonal}\]
\end{lemma}

\begin{theproof}
  Proof by induction on $n$

  \begin{description}
  \item Base Case: $n=1$

    $T=\begin{bmatrix}\l\end{bmatrix}$ is diagonal.

  \item Assume that $T\in UT(n-1)$ normal $\implies T$ diagonal.

  \item Assume $T\in UT(n)$ is normal

    Let $T=\left[\begin{array}{c|c}
        S & \vx \\
        \hline
        0 & a
      \end{array}\right]$ where $S\in UT(n-1)$, $\vx\in\C^{n-1}$, and $a\in\C$

    Now, since $T$ is normal:
    \begin{eqnarray*}
      TT^* &=& T^*T \\
      \left[\begin{array}{c|c} S & \vx \\ \hline 0 & a \end{array}\right]
      \left[\begin{array}{c|c} S^* & 0 \\ \hline \vxct & \bar{a} \end{array}
        \right]
      &=&
      \left[\begin{array}{c|c} S^* & 0 \\ \hline \vxct & \bar{a} \end{array}
        \right]
      \left[\begin{array}{c|c} S & \vx \\ \hline 0 & a \end{array}\right] \\
      \left[\begin{array}{c|c} SS^*+\vx\vxct & \bar{a}\vx \\ \hline
          a\vxct & \abs{a}^2 \end{array}\right] &=&
      \left[\begin{array}{c|c} S^*S & S^*\vx \\ \hline
        \vxct S & \vxct\vx+\abs{a}^2 \end{array}\right]
    \end{eqnarray*}
    From the lower right quadrant we get:
    \[\abs{a}^2=\vxct\vx+\abs{a}^2\]
    And so $\vxct\vx=0$, and thus $\vx=0$:
    \[T=\left[\begin{array}{c|c} S & 0 \\ \hline 0 & a \end{array}\right]\]
    Now, from the upper left quadrant we get:
    \[SS^*+\vx\vxct=S^*S\]
    and so $SS^*=S^*S$, indicating that $S$ is normal. Thus, by the inductive
    assumption, $S$ is also diagonal.

    Therefore, $T$ is diagonal.
  \end{description}
\end{theproof}

\begin{theorem}
  Let $A\in M_n$. TFAE:
  \begin{enumerate}
  \item $A$ is normal ($AA^*=A^*A$)
  \item $A$ is unitary diagonalizable ($A=UDU^*$)
  \item $\tr(A^*A)=\sum_{k=1}^n\abs{\l_k}^2$, where $\l_k\in\Sp(A)$
  \end{enumerate}
\end{theorem}

\begin{theproof}
  \listbreak
  \begin{description}
  \item $1\implies2$: Assume $A$ is normal

    There exists unitary $U$ such that $A=UTU^*$ for some $T\in UT(n)$
    (Schur) \\
    $T=U^*AU$ \\
    Now, using the fact that $A$ is normal:
    \begin{eqnarray*}
      AA^* &=& A^*A \\
      U^*AUU^*A^*U &=& U^*A^*UU^*AU \\
      (U^*AU)(U^*AU)^* &=& (UAU^*)^*(U^*AU) \\
      TT^* &=& T^*T
    \end{eqnarray*}
    $T$ is triangular and normal and thus, by the above lemma, $T$ is diagonal

    Therefore $A$ is unitary diagonalizable.

  \item $2\implies1$: Assume $A$ is unitary diagonalizable

    $A=UDU^*$ for some diagonal matrix $D$
    \begin{eqnarray*}
      AA^* &=& (UDU^*)(UDU^*)^* \\
      &=& UDU^*UD^*U^* \\
      &=& UDD^*U^* \\
      &=& UD^*DU^* \\
      &=& UD^*U^*UDU^* \\
      &=& (UDU^*)^*(UDU^*) \\
      &=& A^*A
    \end{eqnarray*}
    Therefore $A$ is normal.

  \item $2\implies3$: Assume $A$ is unitary diagonalizable

    \newcommand{\md}{\begin{bmatrix}
        \l_1 & & 0 \\ & \ddots & \\ 0 & & \l_n
    \end{bmatrix}}

    \newcommand{\mdct}{\begin{bmatrix}
        \conj{\l_1} & & 0 \\ & \ddots & \\ 0 & & \conj{\l_n}
    \end{bmatrix}}

    \newcommand{\mdn}{\begin{bmatrix}
        \abs{\l_1}^2 & & 0 \\ & \ddots & \\ 0 & & \abs{\l_n}^2
    \end{bmatrix}}
    
    There exists unitary $U$ such that:
    \[A=U\md U^*\]
    \begin{eqnarray*}
      \tr(A^*A) &=&
      \tr\left(\left(U\md U^*\right)^*\left(U\md U^*\right)\right) \\
      &=& \tr\left(U\md^*U^*U\md U^*\right) \\
      &=& \tr\left(U\mdct\md U^*\right) \\
      &=& \tr\left(U^*U\mdct\md\right) \\
      &=& \tr\left(\mdct\md\right) \\
      &=& \tr\left(\mdn\right) \\
      &=& \sum_{k=1}^n\abs{\l_k}^2
    \end{eqnarray*}

  \item $3\implies2$: Assume $\tr(A^*A)=\sum_{k=1}^n\abs{\l_k}^2$

    There exists unitary $U$ and $T\in UT(n)$ such that:
    \[A=UTU^*\]
    such that $T=\begin{bmatrix} \l_1 & & t_{ij} \\ & \ddots & \\ 0 & & \l_n
    \end{bmatrix}$
    \begin{eqnarray*}
      \tr(A^*A) &=& \tr((UTU^*)^*(UTU^*)) \\
      &=& \tr(UT^*U^*UTU^*) \\
      &=& \tr(UT^*TU^*) \\
      &=& \tr(U^*UT^*T) \\
      &=& \tr(T^*T) \\
      \sum_{k=1}^n\abs{\l_k}^2 &=&
      \sum_{k=1}^n\abs{\l_k}^2+\sum_{i<j}\abs{t_{ij}}^2
    \end{eqnarray*}
    So $\sum\abs{t_{ij}}^2=0$ and thus $t_{ij}=0$

    Therefore, $T$ is diagonal and $A$ is unitary diagonalizable.
  \end{description}
\end{theproof}

\end{document}
