\documentclass[letterpaper,12pt,fleqn]{article}
\usepackage{matharticle}
\pagestyle{empty}
\newcommand{\norm}[1]{\left\lVert#1\right\rVert}
\newcommand{\nc}{\norm{\cdot}}
\newcommand{\vx}{\vec{x}}
\newcommand{\e}{\epsilon}
\renewcommand{\a}{\alpha}
\renewcommand{\b}{\beta}
\begin{document}
\section*{Convergence in a Vector Norm}

\begin{definition}[Convergence]
  Let $\nc$ be a vector norm on $\C^n$. To say that a sequence of vectors
  $\{\vx_k\}$ in $\C^n$ \emph{converges} with respect to the norm means
  $\exists\,\vx_0\in\C^n$ such that:
  \[\lim_{k\to\infty}\norm{\vx_k-\vx_0}=0\]
\end{definition}

\begin{definition}[Cauchy]
  Let $\nc$ be a vector norm on $\C^n$. To say that a sequence of vectors
  $\{\vx_k\}$ in $\C^n$ is \emph{Cauchy} with respect to the norm means:
  \[\forall\,\e>0,\exists\,N_{\e}>0,\forall\,i,j>N_{\e},\norm{\vx_i-\vx_j}<\e\]
\end{definition}

\begin{definition}[Complete]
  Let $\nc$ be a vector norm on $\C^n$. To say that $\C^n$ is \emph{complete}
  with respect to the norm means that every Cauchy sequence in $\C^n$ converges
  to to some $\vx_0\in\C^n$.
\end{definition}

\begin{theorem}
  $\C^n$ is complete with respect to $\ell_{\infty}$ \\
\end{theorem}

\begin{theproof}
  Assume $\{\vx_k\}$ in $\C^n$ is Cauchy with respect to $\ell_{\infty}$ \\
  Let $\vx_k(i)$ refer to the $i^{th}$ component of $\vx_k$ \\
  Assume $1\le i\le n$ \\
  Assume $\e_1>0$ \\
  $\exists\,N_{\e_1}>0,\forall\,j,k>N_{\e_1},\norm{\vx_j-\vx_k}<\e_1$ \\
  Assume $j,k>N_{\e_1}$
  \[\abs{\vx_j(i)-\vx_k(i)}\le
  \max_{1\le i\le n}\abs{\vx_j(i)-\vx_k(i)}=\norm{\vx_j-\vx_k}_{\infty}<\e_1\]
  So $\{\vx_k(i)\}$ is Cauchy in $\C$ \\
  But $\C$ is complete, so $\{\vx_k(i)\}\to\vx_0(i)$ as $k\to\infty$ \\
  Assume $\e>0$ \\
  $\exists\,N_{\e}(i)>0,\forall\,k>N_{\e}(i),\abs{\vx_k(i)-\vx_0(i)}<\e$ \\
  Let:
  \[N_{\e}=\max_{1\le i\le n}N_{\e}(i)\]
  Assume $k>N_{\e}$
  \[\norm{\vx_k-\vx_0}=\max_{1\le i\le n}\abs{\vx_k(i)-\vx_0(i)}<\e\]
  Therefore:
  \[\lim_{k\to\infty}\norm{\vx_k-\vx_0}=0\]
  and thus $\{\vx_k\}$ converges with respect to the norm to some
  $\vx_0\in\C^n$.
\end{theproof}

\begin{theorem}
  Let $\nc_{\a}$ and $\nc_{\b}$ be two norms on $\C^n$. There exists
  $c_m,c_M\in\R$ such that $\forall\,\vx\in\C^n$:
  \[c_m\norm{\vx}_{\a}\le\norm{\vx}_{\b}\le c_M\norm{\vx}_{\a}\]
\end{theorem}

\begin{theproof}
  Consider $S=\{\vx\in\C^n\mid\norm{\vx}_2=1\}$ \\
  $S$ is compact \\
  Let $h(\vx)=\frac{\norm{\vx}_{\b}}{\norm{\vx}_{\a}}$ \\
  $h(\vx)$ is continuous \\
  $h[S]$ is compact in $\R$ \\
  Let $h[S]=[c_m,c_M]$ \\
  Assume $\vx\in S$ \\
  $c_m\le\frac{\norm{\vx}_{\b}}{\norm{\vx}_{\a}}\le c_M$ \\
  $c_m\norm{\vx}_{\a}\le\norm{\vx}_{\b}\le c_M\norm{\vx}_{\a}$ \\
  Now, assume $\vx\in\C^n$ \\
  $\frac{\vx}{\norm{\vx}_2}\in S$ \\
  $c_m\norm{\frac{\vx}{\norm{\vx}_2}}_{\a}\le
  \norm{\frac{\vx}{\norm{\vx}_2}}_{\b}\le
  c_M\norm{\frac{\vx}{\norm{\vx}_2}}_{\a}$

  $\therefore c_m\norm{\vx}_{\a}\le\norm{\vx}_{\b}\le c_M\norm{\vx}_{\a}$
\end{theproof}

\begin{example}
  $c_m\norm{\vx}_2\le\norm{\vx}_1\le c_M\norm{\vx}_2$

  Clearly, $c_m=1$

  $\sum_{k=1}^n\abs{x_k}\le c_M(\sum_{k=1}^n\abs{x_k}^2)^{\frac{1}{2}}$

  But by C-S:

  $\sum_{k=1}^n1\cdot\abs{x_k}\le
  (\sum_{k=1}^n1^2)^{\frac{1}{2}}(\sum_{k=1}^n\abs{x_k}^2)^{\frac{1}{2}}$

  So $c_M=(\sum_{k=1}^n1^2)^{\frac{1}{2}}=\sqrt{n}$
\end{example}

\begin{theorem}
  Let $\nc$ be a vector norm on $\C^n$ and let $\{\vx_k\}$ be a sequence in
  $\C^n$:

  The sequence converges iff the sequence is Cauchy.
\end{theorem}
\newpage
\begin{theproof}
  The statement holds for $\ell_{\infty}$ \\
  Assume $\vx\in\C^n$ \\
  There exists some $c_M$ such that
  $0\le\norm{\vx}\le c_M\norm{\vx}_{\infty}$ \\
  So, by the squeeze theorem, the statement also holds for $\nc$.
\end{theproof}

Consequences:
\begin{enumerate}
\item $\{\vx_k\}$ Cauchy wrt $\nc_{\a}\implies\{\vx_k\}$ Cauchy wrt
  $\nc_{\b}$.
\item $\{\vx_k\}$ converges wrt $\nc_{\a}\implies\{\vx_k\}$ converges wrt
  $\nc_{\b}$.
\item All $\nc$ on $\C^n$ are complete.
\end{enumerate}

\end{document}
