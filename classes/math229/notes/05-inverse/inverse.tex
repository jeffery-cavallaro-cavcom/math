\documentclass[letterpaper,12pt,fleqn]{article}
\usepackage{matharticle}
\pagestyle{empty}
\begin{document}
\section*{Inverses}

\begin{definition}[Inverses]
  Let $A\in M_n$. To say that $B\in M_n$ is an \emph{inverse} of $A$ means:
  \[AB=BA=I_n\]
  $B$ is often denoted by $A^{-1}$.
\end{definition}

\begin{properties}
  Let $A\in M_n$:
  \begin{enumerate}
  \item $A^{-1}$ is unique (if it exists).
  \item $(A^{-1})^{-1}=A$
  \item $(A^T)^{-1}=(A^{-1})^T$
  \item $(\bar{A})^{-1}=\overline{A^{-1}}$
  \item $(A^*)^{-1}=(A^{-1})^*$
  \item $(AB)^{-1}=B^{-1}A^{-1}$
  \end{enumerate}
\end{properties}

\begin{theorem}[Sherman-Morrison-Woodbury]
  Let $A\in M_n$, $X\in M_{n,h}$, $R\in M_{h,k}$, and $Y\in M_{k,n}$:
  \[(A+XRY)^{-1}=A^{-1}-A^{-1}X(R^{-1}+YA^{-1}X)^{-1}YA^{-1}\]
\end{theorem}

Special Cases:
\begin{enumerate}
\item $R=\begin{bmatrix} 1 \end{bmatrix}$, $X\in M_{n,1}$ (a row matrix), and
  $Y\in M_{1,n}$ (a column matrix):
  \[(A+xy^T)^{-1}=A^{-1}-A^{-1}x(1+y^TA^{-1}x)y^TA^{-1}\]

\item Same conditions, but $A=I$:
  \[(I+xy^T)^{-1}=I-x(1+y^Tx)y^T=I-\frac{xy^T}{1+y^Tx}\]

\item $A=I_n$, $R=I_m$, and $X,Y\in M_{n,m}$ where $n>>m$:
  \[(I+XY^T)^{-1}=I-X(1+y^Tx)^{-1}y^T\]
  This formula results in the inversion of a much smaller $m\times m$ matrix.
\end{enumerate}

\begin{example}
  \listbreak
  \begin{eqnarray*}
    \begin{bmatrix}
      2 & 1 & 1 & 1 \\
      1 & 2 & 1 & 1 \\
      1 & 1 & 2 & 1 \\
      1 & 1 & 1 & 2 \\
    \end{bmatrix}^{-1} &=& \left(I_4+\begin{bmatrix}
    1 & 1 & 1 & 1 \\
    1 & 1 & 1 & 1 \\
    1 & 1 & 1 & 1 \\
    1 & 1 & 1 & 1
    \end{bmatrix}\right)^{-1} \\
    &=& \left(I_4+\begin{bmatrix} 1 \\ 1 \\ 1 \\ 1 \end{bmatrix}
    \begin{bmatrix} 1 & 1 & 1 & 1 \end{bmatrix}\right)^{-1} \\
    &=& I_4-\frac{1}{1+4}
    \begin{bmatrix}
      1 & 1 & 1 & 1 \\
      1 & 1 & 1 & 1 \\
      1 & 1 & 1 & 1 \\
      1 & 1 & 1 & 1
    \end{bmatrix} \\
    &=& \frac{1}{5}\left(5I_4-\begin{bmatrix}
      1 & 1 & 1 & 1 \\
      1 & 1 & 1 & 1 \\
      1 & 1 & 1 & 1 \\
      1 & 1 & 1 & 1
    \end{bmatrix}\right) \\
    &=& \frac{1}{5}\begin{bmatrix}
      4 & -1 & -1 & -1 \\
      -1 & 4 & -1 & -1 \\
      -1 & -1 & 4 & -1 \\
      -1 & -1 & -1 & 4
    \end{bmatrix}
  \end{eqnarray*}
\end{example}

\end{document}
