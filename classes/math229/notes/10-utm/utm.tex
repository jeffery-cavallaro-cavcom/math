\documentclass[letterpaper,12pt,fleqn]{article}
\usepackage{matharticle}
\pagestyle{empty}
\DeclareMathOperator{\adj}{adj}
\begin{document}
\section*{Upper Triangular Matrices}

\begin{definition}[Upper Triangular Matrix]
  To say that a matrix $A$ is \emph{upper triangular} means:
  \[i>j\implies a_{ij}=0\]
  In particular, a diagonal matrix is upper triangular.

  The set of all $n\times n$ upper triangular matrices is denoted by $UT(n)$.
\end{definition}

\begin{properties}[Upper Triangular]
  Let $A,B\in UT(n)$:
  \begin{enumerate}
  \item $cA\in UT(n)$
  \item $A+B\in UT(n)$
  \item $AB\in UT(n)$
  \item $\adj(A)\in UT(n)$
  \item $A$ invertible$\implies A^{-1}\in UT(n)$
  \end{enumerate}
\end{properties}

\begin{theorem}
  Let $T\in UT(n)$ such that $T$ has distinct diagonal entries and $AT=TA$:
  \[A\in UT(n)\]
\end{theorem}

\begin{theproof}
  Proof by induction on $n$.
  \begin{description}
  \item Base case: $n=1$

    $1\times1$ matrices are by definition UT, so nothing to prove.

  \item Assume true for $n-1$

  \item Let $A=\begin{bmatrix} A_{11} & A_{12} \\ A_{21} & A_{22} \end{bmatrix}$, where
    $A_{11}\in M_{n-1}$ and let
    $T=\begin{bmatrix} T_{11} & T_{12} \\ 0 & T_{22} \end{bmatrix}$, where
    $T_{11}\in UT(n-1)$ and $T_{11}$ and $T_{22}$ have distinct diagonal entries.
    \begin{eqnarray*}
      AT &=& TA \\
      \begin{bmatrix} A_{11} & A_{12} \\ A_{21} & A_{22} \end{bmatrix}
      \begin{bmatrix} T_{11} & T_{12} \\ 0 & T_{22} \end{bmatrix} &=&
      \begin{bmatrix} T_{11} & T_{12} \\ 0 & T_{22} \end{bmatrix} 
      \begin{bmatrix} A_{11} & A_{12} \\ A_{21} & A_{22} \end{bmatrix} \\
      \begin{bmatrix} * & * \\ A_{21}T_{11} & * \end{bmatrix} &=&
      \begin{bmatrix} * & * \\ A_{21}T_{22} & * \end{bmatrix} \\
      A_{21}T_{11} &=& T_{22}A_{21} \\
    \end{eqnarray*}

  \end{description}
\end{theproof}

\end{document}
