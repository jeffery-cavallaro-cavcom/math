\documentclass[letterpaper,12pt,fleqn]{article}
\usepackage{matharticle}
\pagestyle{empty}
\newcommand{\inner}[2]{\left<#1,#2\right>}
\newcommand{\conj}[1]{\overline{#1}}
\newcommand{\norm}[1]{\|#1\|}
\newcommand{\vx}{\vec{x}}
\newcommand{\vy}{\vec{y}}
\newcommand{\vz}{\vec{z}}
\newcommand{\vu}{\vec{u}}
\begin{document}
\section*{Inner Product}

\begin{definition}[Inner Product]
  Let $\vx,\vy\in\C^n$. The \emph{inner product} of $\vx$ and $\vy$,
  denoted $\inner{\vx}{\vy}$, is given by:
  \[\inner{\vx}{\vy}=\vy^{\,*}\vx=\sum_{k=1}^n\conj{y_k}x_k\]
\end{definition}

\begin{theorem}
  Let $\vx,\vy,\vz,\vu\in\C^n$ and $c\in\C$:
  \begin{enumerate}
  \item $\inner{\vx}{\vy}=\conj{\inner{\vy}{\vx}}$
  \item $\inner{c\vx}{\vy}=c\inner{\vx}{\vy}$
  \item $\inner{\vx}{c\vy}=\conj{c}\inner{\vx}{\vy}$
  \item $\inner{\vx+\vy}{\vz+\vu}=\inner{\vx}{\vz}+\inner{\vx}{\vu}+
    \inner{\vy}{\vz}+\inner{\vy}{\vu}$
  \end{enumerate}
\end{theorem}

\begin{theproof}
  \listbreak
  \begin{enumerate}
  \item
    \[\inner{\vx}{\vy}=\vy^{\,*}\vx=\left(\vx^{\,*}\vy\right)^*=
    \conj{\vx^{\,*}\vy}=\conj{\inner{\vy}{\vx}}\]

  \item
    \[\inner{c\vx}{\vy}=\vy^{\,*}(c\vx)=c(\vy^{\,*}\vx)=c\inner{\vx}{\vy}\]

  \item
    \[\inner{\vx}{c\vy}=(c\vy)^*\vx=\conj{c}(\vy^{\,*}\vx)=
    \conj{c}\inner{\vx}{\vy}\]

  \item
    \begin{eqnarray*}
      \inner{\vx+\vy}{\vz+\vu} &=& (\vz+\vu)^*(\vx+\vy) \\
      &=& (\vz^{\,*}+\vu^{\,*})(\vx+\vy) \\
      &=& \vz^{\,*}\vx+\vu^{\,*}\vx+\vz^{\,*}\vy+\vu^{\,*}\vy \\
      &=& \inner{\vx}{\vz}+\inner{\vx}{\vu}+\inner{\vy}{\vz}+\inner{\vy}{\vu}
    \end{eqnarray*}
  \end{enumerate}
\end{theproof}

\newpage

\subsection*{Norm}

\begin{definition}[Norm]
  Let $\vx\in\C^n$. The \emph{norm} of $\vx$, denoted $\norm{\vx}$, is given by:
  \[\norm{\vx}=\sqrt{\inner{\vx}{\vx}}=\sqrt{\sum_{k=1}^n\conj{x_k}x_k}=
  \sqrt{\sum_{k=1}^n\abs{x_k}^2}\]
\end{definition}

\begin{definition}[Orthogonal]
  Let $\vx,\vy\in\C^n$. To say that $\vx$ and $\vy$ are \emph{orthogonal}
  means:
  \[\inner{\vx}{\vy}=0\]
\end{definition}

\begin{theorem}[Pythagorean Theorem]
  Let $\vx,\vy\in\C^n$:

  $\vx,\vy$ orthogonal $\implies\norm{\vx+\vy}=\norm{\vx}+\norm{\vy}$
\end{theorem}

\begin{theproof}
  Assume $\vx,\vy$ orthogonal
  \begin{eqnarray*}
    \norm{\vx+\vy} &=& \inner{\vx+\vy}{\vx+\vy} \\
    &=& \inner{\vx}{\vx}+\inner{\vx}{\vy}+\inner{\vy}{\vx}+\inner{\vy}{\vy} \\
    &=& \norm{\vx}+0+0+\norm{\vy} \\
    &=& \norm{\vx}+\norm{\vy}
  \end{eqnarray*}
\end{theproof}

Note that the converse is only true when $\inner{\vx}{\vy}+\inner{\vy}{\vx}=0$,
or when $\inner{\vx}{\vy}=-\conj{\inner{\vx}{\vy}}$, but this can only occur
if the real part is zero. Thus, the converse only holds when $\inner{\vx}{\vy}$
is imaginary.

\begin{theorem}
  Let $\vx$ and $\vy$ be non-zero vectors in $\C^n$:

  $\vx$ and $\vy$ are orthogonal $\implies$ $\vx$ and $\vy$ are linearly
  independent.
\end{theorem}

\begin{theproof}
  Assume $\vx$ and $\vy$ are orthogonal \\
  $\inner{\vx}{\vy}=0$ \\
  ABC: $\vx$ and $\vy$ are linearly dependent \\
  There exists non-zero $c\in\C$ such that $\vx=c\vy$ \\
  $\inner{\vx}{\vy}=\inner{c\vy}{\vy}=c\norm{\vy}^2\ne0$ \\
  CONTRADICTION!
  
  Therefore, $\vx$ and $\vy$ must be linearly independent.
\end{theproof}

\newpage

\subsection*{Inequalities}

\begin{theorem}[Cauchy-Schwarz]
  Let $\vx,\vy\in\C^n$:
  \[\abs{\inner{\vx}{\vy}}\le\norm{\vx}\norm{\vy}\]
\end{theorem}

\begin{theproof}
  Note that when $\vx$ and $\vy$ are dependent (including one or both zero)
  then equality holds, so AWLOG: $\vx$ and $\vy$ are independent (and thus
  non-zero).

  Let $\vz=\vx-\frac{\inner{\vx}{\vy}}{\inner{\vy}{\vy}}\vy$ \\
  $\inner{\vy}{\vz}=\inner{\vz}{\vy}=0$ \\
  $\vx=\vz+\frac{\inner{\vx}{\vy}}{\inner{\vy}{\vy}}\vy$ \\
  $\norm{\vx}^2=\norm{\vz}^2+
  \norm{\frac{\inner{\vx}{\vy}}{\inner{\vy}{\vy}}\vy}^2=
  \norm{\vz}^2+\frac{\abs{\inner{\vx}{\vy}}^2}{\norm{\vy}^2}$ \\
  $\norm{\vx}^2\ge\frac{\abs{\inner{\vx}{\vy}}^2}{\norm{\vy}^2}$
  
  $\therefore \abs{\inner{\vx}{\vy}}\le\norm{\vx}\norm{\vy}$
\end{theproof}

\begin{theorem}[Triangle Inequality]
  Let $\vx,\vy\in\C^n$:
  \[\norm{\vx+\vy}\le\norm{\vx}+\norm{\vy}\]
\end{theorem}

\begin{theproof}
  Note that when $\vx$ and $\vy$ are dependent (including one or both zero)
  then equality holds.
  \begin{eqnarray*}
    \norm{\vx+\vy}^2 &=& \inner{\vx+\vy}{\vx+\vy} \\
    &=& \norm{\vx}^2+\norm{\vy}^2+\inner{\vx}{\vy}+\inner{\vy}{\vx} \\
    &\le& \norm{\vx}^2+\norm{\vy}^2+2\abs{\inner{\vx}{\vy}} \\
    &\le& \norm{\vx}^2+\norm{\vy}^2+2\norm{\vx}\norm{\vy} \\
    &=& (\norm{\vx}+\norm{\vy})^2 \\
    \therefore \norm{\vx+\vy} &\le& \norm{\vx}+\norm{\vy}
  \end{eqnarray*}
\end{theproof}
  
\end{document}
