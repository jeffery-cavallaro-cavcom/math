\documentclass[letterpaper,12pt,fleqn]{article}
\usepackage{matharticle}
\pagestyle{empty}
\DeclareMathOperator{\tr}{tr}
\renewcommand{\a}{\alpha}
\begin{document}
\section*{Trace}

\begin{definition}[Trace]
  Let $A\in M_n$. The \emph{trace} of $A$ is given by:
  \[\tr(A)=\sum_{k=1}^na_{kk}\]
  In other words, the trace is the sum of the diagonal entries.
\end{definition}

\begin{properties}[Trace]
  \listbreak
  \begin{enumerate}
  \item $\tr(A+B)=\tr(A)+\tr(B)$
  \item $\tr(A)=\tr(A^T)$
  \item $\tr(cA)=c\tr(A)$
  \end{enumerate}
\end{properties}

\begin{theorem}
  Let $f:M_n\to\C$ be a linear transformation such that $f(AB)=f(BA)$:
  \[f(I_n)=n\iff f\ \mbox{is the trace}\]
\end{theorem}

\begin{theproof}
  \listbreak
  \begin{description}
  \item Assume $f(I_n)=n$

    Assume $A\in M_n$ \\
    Let $f(A)=f\left(\sum_{i,j}a_{ij}E_{ij}\right)=\sum_{i,j}a_{ij}f(E_{ij})$
    \begin{description}
    \item Case 1: $i\ne j$

      $f(E_{ij})=f(E_{i1}E_{1j})=f(E_{1j}E_{i1})=f(0)=0$

    \item Case 2: $i=j$

      $f(E_{ii})=f(E_{i1}E_{1i})=f(E_{1i}E_{i1})=f(E_{11})=\a$
    \end{description}

    So, after discarding the zero $f(E_{ij})$ entries and replacing the $f(E_{ii})$
    entries with $\a$:
    \[f(A)=\a\sum_{i=1}^na_{ii}\]
    But $f(I)=\a n=n$ \\
    So $\a=1$

    $\therefore f(A)=\sum_{i=1}^na_{ii}=\tr(A)$

  \item Assume $f$ is the trace

    $f(I_n)=tr(I_n)=n\cdot1=n$
  \end{description}
\end{theproof}

\end{document}
