\documentclass[letterpaper,12pt,fleqn]{article}
\usepackage{matharticle}
\pagestyle{empty}
\DeclareMathOperator{\adj}{adj}
\newcommand{\At}{\tilde{A}}
\newcommand{\at}{\tilde{a}}
\newcommand{\e}{\epsilon}
\newcommand{\vx}{\vec{x}}
\newcommand{\vy}{\vec{y}}
\begin{document}
\section*{Adjugate}

\begin{definition}
  Let $A\in M_n$. The \emph{adjugate} of $A$ is the matrix given by:
  \[\adj(A)=[(-1)^{i+j}\det(A_{ji})]\]
  Thus, each entry in the adjugate of $A$ is the corresponding cofactor of $A$.
\end{definition}

\begin{lemma}
  Let $A\in M_n$:
  \[A(\adj(A))=(\adj(A))A=(\det(A))I\]
\end{lemma}

\begin{theproof}
  $[A(\adj(A))]_{ii}=\sum_{k=1}^na_{ik}[\adj(A)]_{ki}=
  \sum_{k=1}^na_{ik}(-1)^{k+i}\det(A_{ik})=\det(A)$

  Now, consider a matrix $\At$ where the $i^{th}$ row is replaced by the $j^{th}$ row
  and expand along the $j^{th}$ row. Since $\At$ has two dependent rows, its determinant
  is 0:
  \[\det(\At)=\sum_{k=1}^n(-1)^{j+k}\at_{jk}\det(\At_{jk})=0\]
  But $\at_{jk}=a_{ik}$ and $\det(\At_jk)=\det(A_{jk})$, so:
  \[\sum_{k=1}^n(-1)^{j+k}a_{ik}\det(A_{jk})=[A\adj(A)]_{ij}=0\]

  $\therefore A(\adj(A))=(\det(A))I$

  $A^T(\adj(A^T))=(\det(A^T))I$ \\
  $A^T(\adj(A)^T)=(\det(A^T))I$ \\
  $(\adj(A)A)^T=(\det(A^T))I$

  $\therefore (\adj(A))A=(\det(A))I$
\end{theproof}

\begin{corollary}
  Let $A\in M_n$ be invertible:
  \[A^{-1}=\frac{1}{\det(A)}\adj(A)\]
\end{corollary}

\begin{theproof}
  $A(\adj(A))=(\det(A))I$ \\
  But $A$ is invertible and so $\det(A)\ne0$

  $\therefore A^{-1}=\frac{1}{\det(A)}\adj(A)$
\end{theproof}

\begin{lemma}
  There exists invertible matrices $A_{\e}$ such that:
  \[\lim_{\e\to0}A_{\e}=A\]
\end{lemma}

\begin{theorem}
  Let $A\in M_n$, and $x,y\in M_{n,1}$:
  \[\det(A+\vx\vy^T)=\det(A)+\vy^T(\adj(A))\vx\]
\end{theorem}

\begin{example}
  $A=\begin{bmatrix}
  1 & 4 & 5 \\
  0 & 2 & 6 \\
  0 & 0 & 0
  \end{bmatrix}
  \hspace{4ex}
  \vx=\begin{bmatrix} 1 \\ 2 \\ 3 \end{bmatrix}
  \hspace{4ex}
  \vy=\begin{bmatrix} 1 \\ 1 \\ 2 \end{bmatrix}$
  \begin{eqnarray*}
    \det(A+\vx\vy^T) &=& \det\left(\begin{bmatrix}
    1 & 4 & 5 \\
    0 & 2 & 6 \\
    0 & 0 & 0
    \end{bmatrix}+
    \begin{bmatrix} 1 \\ 2 \\ 3 \end{bmatrix}\begin{bmatrix} 1 & 1 & 2 \end{bmatrix}
    \right) \\
    &=& \det\left(\begin{bmatrix}
    1 & 4 & 5 \\
    0 & 2 & 6 \\
    0 & 0 & 0
    \end{bmatrix}+\begin{bmatrix}
    1 & 1 & 2 \\
    2 & 2 & 4 \\
    3 & 3 & 6
    \end{bmatrix}\right) \\
    &=& \det\left(\begin{bmatrix}
      2 & 5 & 7 \\
      2 & 4 & 10 \\
      3 & 3 & 6
    \end{bmatrix}\right) \\
    &=& 2(24-30)-5(12-30)+7(6-12) \\
    &=& -12+90-42 \\
    &=& 36
  \end{eqnarray*}
  \begin{eqnarray*}
    \det(A)+\vy^T(\adj(A))\vx &=& \det\left(\begin{bmatrix}
      1 & 4 & 5 \\
      0 & 2 & 6 \\
      0 & 0 & 0
    \end{bmatrix}\right)+\begin{bmatrix} 1 & 1 & 2 \end{bmatrix}\adj(a)
    \begin{bmatrix} 1 \\ 2 \\ 3 \end{bmatrix} \\
    &=& 0 + \begin{bmatrix} 1 & 1 & 2 \end{bmatrix}
    \begin{bmatrix}
      0 & 0 & 14 \\
      0 & 0 & -6 \\
      0 & 0 & 2
    \end{bmatrix}\begin{bmatrix} 1 \\ 2 \\ 3 \end{bmatrix} \\
    &=& \begin{bmatrix} 0 & 0 & 12 \end{bmatrix}
    \begin{bmatrix} 1 \\ 2 \\ 3 \end{bmatrix} \\
    &=& 36
  \end{eqnarray*}
\end{example}
\newpage
\begin{theproof}
  \newcommand{\mm}{\begin{bmatrix}
      A & -\vx \\ \vy^T & 1
  \end{bmatrix}}
  \newcommand{\mn}{\begin{bmatrix}
      I_n & 0_{n\times 1} \\ -\vy^T & 1
  \end{bmatrix}}
  \newcommand{\mo}{\begin{bmatrix}
      A+\vx\vy^T & -\vx \\ 0 & 1
  \end{bmatrix}}

  Claim: $\det(x+\vx\vy^T)=\det\mm$

  $\det\mn=\det(I_n)\det(\begin{bmatrix} 1 \end{bmatrix})=1\cdot1=1$

  $\det\mo=\det(A+\vx\vy^T)\det(\begin{bmatrix} 1 \end{bmatrix})=
  \det(A+\vx\vy^T)\cdot1=\det(A+\vx\vy^T)$

  \begin{eqnarray*}
    \det\left(\mm\mn\right) &=& \det\mo \\
    \left(\det\mm\right)\left(\det\mn\right) &=& \det\mo \\
    \left(\det\mm\right)(1) &=& \det(A+\vx\vy^T) \\
    \therefore \det(A+\vx\vy^T) &=& \det\mm
  \end{eqnarray*}

  Claim: $\det\mm=\det(A)+\vy^T(\adj(A))\vx$

  \begin{description}
  \item Case 1: $A$ is invertible
    \begin{eqnarray*}
      \begin{bmatrix} I & 0 \\ -\vy^TA^{-1} & 1 \end{bmatrix}\mm &=&
      \begin{bmatrix} A & -\vx \\ 0 & y^TA^{-1}\vx+1 \end{bmatrix} \\
      \det\begin{bmatrix} I & 0 \\ -\vy^TA^{-1} & 1 \end{bmatrix}\det\mm &=&
      \det\begin{bmatrix} A & -\vx \\ 0 & \vy^TA^{-1}\vx+1 \end{bmatrix} \\
      1\cdot\det\mm &=& \det(A)(\vy^TA^{-1}\vx+1) \\
      \cdot\det\mm &=& \det(A)+\vy^T(\det(A))A^{-1}\vx \\
      &=& \det(A)+\vy^T\adj(A)\vx
    \end{eqnarray*}

  \item Case 2: $A$ is not invertible

    There exists invertible matrices $A_{\e}$ such that by case 1:
    \[\det\begin{bmatrix} A_{\e} & -\vx \\ \vy^T & 1 \end{bmatrix}=
    \det(A_{\e})+\vy^T(\adj(A_{\e}))\vx\]
    Take the limit as $\e\to0$.
  \end{description}
  $\therefore \det(A+\vx\vy^T)=\det(A)+\vy^T(\adj(A))\vx$
\end{theproof}

\end{document}
