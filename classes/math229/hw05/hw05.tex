\documentclass[letterpaper,12pt,fleqn]{article}
\usepackage{matharticle}
\pagestyle{plain}
\renewcommand{\l}{\lambda}
\newcommand{\p}{\rho}
\newcommand{\vx}{\vec{x}}
\newcommand{\vxct}{\vx^{\,*}}
\newcommand{\rr}{\frac{\vxct A\vx}{\vxct\vx}}
\newcommand{\ve}{\vec{e}}
\newcommand{\vz}{\vec{0}}
\newcommand{\va}{\vec{a}}
\DeclareMathOperator{\Sp}{Sp}
\DeclareMathOperator{\tr}{tr}
\DeclareMathOperator{\Eig}{Eig}
\DeclareMathOperator{\Diag}{Diag}
\begin{document}
Cavallaro, Jeffery \\
Math 229 \\
Homework \#5

\bigskip

\subsection*{4.1.3}

Let $A,B\in M_n$ be Hermitian. Show that $A$ and $B$ are similar iff $A$ and $B$ are
unitary similar.

\begin{description}
\item $\implies$ Assume $A$ and $B$ are similar.

  Since $A$ and $B$ are similar we have $\Sp(A)=\Sp(B)$. Now, since $A$ and $B$ are
  Hermitian, they are unitary diagonalizable. Let:

  \newcommand{\dm}{\begin{bmatrix}
      \l_1 & & 0 \\
      & \ddots & \\
      0 & & \l_n
  \end{bmatrix}}
  
  Let $A=U\dm U^*$ and $B=V\dm V^*$ for unitary matrices $U$ and $V$.

  $\dm=V^*BV$

  $A=U(V^*BV)U^*=(UV^*)B(VU^*)=(UV^*)B(UV^*)^*$

  But the product of unitary matrices is unitary, so $UV^*$ is unitary.

  Therefore, $A$ and $B$ are unitary similar.

\item $\impliedby$ Assume $A$ and $B$ are unitary similar.

  Let $A=UBU^*$ for unitary matrix $U$. \\
  $UU^*=U^*U=I$, so unitary matrices are invertible with $U^{-1}=U^*$. \\
  Thus, $A=UBU^{-1}$

  Therefore, $A$ and $B$ are similar.
\end{description}

\subsection*{4.1.11}

Let $A,B\in M_n$ be Hermitian. Explain why $AB-BA$ is skew-Hermitian and deduce from
(4.1.P10) that $\tr(AB)^2\le\tr(A^2B^2)$ with equality iff $AB=BA$.

$(AB-BA)^*=(AB)^*-(BA)^*=B^*A^*-A^*B^*=BA-AB=-(AB-BA)$

Therefore $AB-BA$ is skew-Hermitian.
\newpage
4.1.P10 shows that if $C\in M_n$ is skew-Hermitian then the eigenvalues of $C$ are
pure imaginary and the eigenvalues of $B^2$ are real and non-positive, and all zero iff
$B=0$.
with equality iff $B=0$.
\begin{eqnarray*}
  \tr(AB-BA)^2 &=& \tr(ABAB-ABBA-BAAB+BABA) \\
  &=& \tr(ABAB)+\tr(BABA)-\tr(ABBA)-\tr(BAAB) \\
  &=& \tr(ABAB)+\tr((BAB)A)-\tr((ABB)A)-\tr(B(AAB)) \\
  &=& \tr(ABAB)+\tr(A(BAB))-\tr(A(ABB))-\tr((AAB)B) \\
  &=& \tr(ABAB)+\tr(ABAB)-\tr(AABB)-\tr(AABB) \\
  &=& 2\tr(ABAB)-2\tr(AABB) \\
  &=& 2\tr(AB)^2-2\tr(A^2B^2)
\end{eqnarray*}

But $AB-BA$ is skew Hermitian, so the eigenvalues of $(AB-BA)^2$ are real and
non-positive, so:
\begin{eqnarray*}
  2\tr(AB)^2-2\tr(A^2B^2) &\le& 0 \\
  \tr(AB)^2-\tr(A^2B^2) &\le& 0 \\
  \therefore\tr(AB)^2 &\le& \tr(A^2B^2)
\end{eqnarray*}
Furthermore, the eigenvalues of $AB-BA$ are all zero iff $AB-BA=0$. Therefore,
$\tr(AB)^2\le\tr(A^2B^2)$ iff $AB-BA=0$, or $AB=BA$.

\subsection*{4.2.3}

Let $A\in M_n$ be Hermitian with eigenvalues $\l_1\le\cdots\le\l_n$. Use the key
lemma to show that $\l_1\le a_{ii}\le\l_n$ for all $1\le i\le n$ with equality in
one of the inequalities for some $i$ only if $a_{ij}=a_{ji}=0$ for all $j\ne i$.
Consider $A=\Diag(1,2,3)$ and explain why the condition $a_{ij}=a_{ji}=0$ for all
$j\ne i$ does not imply that $a_{ii}=\l_1$ or $a_{ii}=\l_n$.

By the key lemma we have $\forall\,\vx\in\C^n$ such that $\vx\ne\vz$:
\[\l_1\le\rr\le\l_n\]
Let $\vx=\ve_i$:
\[\frac{\ve_i^{\,*}A\ve_i}{\ve_i^{\,*}\ve_i}=\frac{a_{ii}}{1}=a_{ii}\]
Therefore:
\[\l_1\le a_{ii}\le\l_n\]

\newpage

Claim: $\forall\vx\in\C^n$ such that $\vx\ne\vz$:
\[\l_i=\rr\iff\vx\in\Eig_A(\l_i)\]

\begin{description}
\item $\implies$ Assume $\l_i=\rr$
  \begin{eqnarray*}
    \vxct A\vx &=& \l_i\vxct\vx \\
    \vxct A\vx &=& \vxct\l_i\vx \\
    A\vx &=& \l_i\vx
  \end{eqnarray*}
  Therefore, since $\vx\ne\vz$, $\vx\in\Eig_A(\l_i)$

\item $\impliedby$ Assume $\vx\in\Eig_A(\l_i)$
  \[\rr=\frac{\vxct\l_i\vx}{\vxct\vx}=\l_i\frac{\vxct\vx}{\vxct\vx}=\l_i\]
\end{description}

Now assume that $a_{ii}=\l_1$ or $a_{ii}=\l_n$ for some $i$. Since:
\[a_{ii}=\frac{\ve_i^{\,*}A\ve_i}{\ve_i^{\,*}\ve_i}\]
It must be the case that $\ve_i\in\Eig_A(a_{ii})$, and so:
\begin{eqnarray*}
  A\ve_i &=& a_{ii}\ve_i \\
  \va_i &=& a_{ii}\ve_i \\
  a_{ij} &=& \begin{cases} a_{ii}, & i=j \\ 0, & i\ne j \end{cases}
\end{eqnarray*}
But $A$ is Hermitian, so $a_{ji}=0$ as well.

Now, consider $A=\begin{bmatrix} 1 & 0 & 0 \\ 0 & 2 & 0 \\ 0 & 0 & 3 \end{bmatrix}$
We have $\l_1=1$ and $\l_n=3$. Consider $i=2$. $a_{ii}=2$, which is neither $\l_1$
nor $\l_n$.

\subsection*{4.3.1}

Let $A,B\in M_n$ be Hermitian. Show that:
\[\l_1(B)\le\l_i(A+B)-\l_i(A)\le\l_n(B)\]
Conclude that $\abs{\l_i(A+B)-\l_i(A)}\le\p(B)$.

Start with Weyl's inequalities:
\[\l_{i+j-n}(A+B)\le\l_i(A)+\l_j(B)\le\l_{i+j-1}(A+B)\]
First, let j=1:
\[\l_{i+1-1}(A+B)=\l_i(A+B)\ge\l_i(A)+\l_1(B)\]
Now, for the same $i$, let $j=n$:
\[\l_{i+n-n}(A+B)=\l_i(A+B)\le\l_i(A)+\l_n(B)\]
Putting these two together we have:
\[\l_i(A)+\l_1(B)\le\l_i(A+B)\le\l_i(A)+\l_n(B)\]
and finally:
\[\l_1(B)\le\l_i(A+B)-\l_i(A)\le\l_n(B)\]

Note that due to the assumed ordering for the eigenvalues of B:
\[\l_1(B)\le\cdots\le\l_n(B)\]
It is the case that:
\[\p(B)=\max\{\abs{\l_1(B)},\abs{\l_n(B)}\}\]
\begin{description}
\item Case 1: $\p(B)=\abs{\l_1(B)}$

  It must be the case that $\l_1(B)\le0$ and thus:
  \[-\abs{\l_1(B)}\le\l_i(A+B)-\l_i(A)\le\abs{\l_n(B)}\le\abs{\l_1(B)}\]
  and thus:
  \[\abs{\l_i(A+B)-\l_i(A)}\le\abs{\l_1(B)}=\p(B)\]

\item Case 2: $\p(B)=\abs{\l_n(B)}$

  It must be the case that $\l_n(B)\ge0$ and thus:
  \[-\abs{\l_n(B)}\le\abs{\l_1(B)}\le\l_i(A+B)-\l_i(A)\le\abs{\l_n(B)}\]
  and thus:
  \[\abs{\l_i(A+B)-\l_i(A)}\le\abs{\l_n(B)}=\p(B)\]
\end{description}

$\therefore\abs{\l_i(A+B)-\l_i(A)}\le\p(B)$

\subsection*{4.3.3}

Let $A,B\in M_n$ be Hermitian. Explain why:
\[\l_i(A+B)\le\min_{j+k=i+n}\{\l_j(A)+\l_k(B)\}\]

Starting with the first part of Weyl's inequalities:
\[\l_{j+k-n}(A+B)\le\l_j(A)+\l_k(B)\]
For $j+k=i+n$ we have:
\[\l_i(A+B)\le\l_j(A)+\l_k(B)\]
and therefore:
\[\l_i(A+B)\le\min_{j+k=i+n}\{\l_j(A)+\l_k(B)\}\]
\end{document}
