\documentclass[letterpaper,12pt,fleqn]{article}
\usepackage{matharticle}
\usepackage{enumitem}
\pagestyle{plain}
\begin{document}
Cavallaro, Jeffery \\
Math 275A \\
Homework \#0

\subsection*{3.1}

Define two points $(x_0,y_0)$ and $(x_1,y_1)$ of the plane to be equivalent if:
\[y_0-x_0^2=y_1-x_1^2\]
Check that this is an equivalence relation and describe the equivalence
classes.

\bigskip

\begin{description}
\item (R) Assume $(x_0,y_0)\in\R^2$.

  $y_0-x_0^2=y_0-x_0^2$

  $\therefore(x_0,y_0)\sim(x_0,y_0)$

\item (S) Assume $(x_0,y_0)\sim(x_1,y_1)$.

  $y_0-x_0^2=y_1-x_1^2$ \\
  $y_1-x_1^2=y_0-x_0^2$

  $\therefore(x_1,y_1)\sim(x_0,y_0)$

\item (T) Assume $(x_0,y_0)\sim(x_1,y_1)$ and $(x_1,y_1)\sim(x_2,y_2)$.
  
  $y_0-x_0^2=y_1-x_1^2$ and $y_1-x_1^2=y_2-x_2^2$ \\
  $y_0-x_0^2=y_2-x_2^2$

  $\therefore(x_0,y_0)\sim(x_2,y_2)$
\end{description}

The equivalence classes are the parabolas $y=x^2+c$.

\subsection*{3.3}

Here is a ``proof'' that every relation $\sim$ that is both symmetric and
transitive is also reflexive: ``Since $\sim$ is symmetric,
$a\sim b\implies b\sim a$. Since $\sim$ is transitive,
$a\sim b$ and $b\sim a\implies a\sim $, as desired. Find the flaw in this
argument.

\bigskip

Suppose $a$ is not related to any $b$? For example, consider the set
$X=\{1,2,3\}$ and define a relation on $X$ as follows:
\[\left\{(2,2),(2,3),(3,2),(3,3)\right\}\]
There no way to show $1\sim1$ without an explicit statement of reflexivity.

\subsection*{3.4}

Let $f:A\to B$ be a surjective function. Let us define a relation on $A$ by
setting $a_0\sim a_1$ if:
\[f(a_0)=f(a_1)\]
\begin{enumerate}[label=(\alph*)]
\item Show that this is an equivalence relation.
  \begin{description}
  \item (R) Assume $a_0\in A$.

    $f$ is well-defined. \\
    And so $f(a_0)=f(a_0)$
    
    $\therefore a_0\sim a_0$

  \item (S) Assume $a_0\sim a_1$.

    $f(a_0)=f(a_1)$ \\
    $f(a_1)=f(a_0)$

    $\therefore a_1\sim a_0$

  \item (T) Assume $a_0\sim a_1$ and $a_1\sim a_2$.

    $f(a_0)=f(a_1)$ and $f(a_1)=f(a_2)$ \\
    $f(a_0)=f(a_2)$

    $\therefore a_0\sim a_2$
  \end{description}

\item Let $A^*$ be the set of equivalence classes. Show that there is a
  bijective correspondence of $A^*$ with $B$.

  Define $g:A^*\to B$ by $g(a^*)=b$ where $\forall\,a\in a^*,f(a)=b$.

  Assume $g(a_0^*)=g(a_1^*)$. \\
  Assume $a_0\in a_0^*$ and $a_1\in a_1^*$. \\
  $f(a_0)=f(a_1)$ and so $a_0\sim a_1$.
  
  $\therefore a_0^*=a_1^*$ and thus $g$ is injective.

  Assume $b\in B$. \\
  Since $f$ is surjective, $\exists a\in A,f(a)=b$. \\
  But $a\in a^*$.

  $\therefore g(a^*)=b$ and thus $g$ is surjective.

  $\therefore g$ is bijective.
\end{enumerate}

\subsection*{3.6}

\subsection*{3.9}

\subsection*{3.10}

\subsection*{3.12}

\end{document}
