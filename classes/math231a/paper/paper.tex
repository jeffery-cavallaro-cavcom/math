%%%%%%%%%%%%%%%%%%%%%%%%%%%%%%%%%%%%%%%%%%%%%%%%%%
%%%            TITLE:                             
%%%
%%%            DATE:
%%%
%%%            AUTHOR(S):
%%%
%%%%%%%%%%%%%%%%%%%%%%%%%%%%%%%%%%%%%%%%%%%%%%%%%%

\documentclass[12pt,letterpaper,reqno]{amsart}                       


%%%%%%%%%%%      PACKAGES      %%%%%%%%%%%%%%%

\usepackage{amsmath}
\usepackage{amsfonts}
\usepackage{amssymb}
\usepackage{amsthm}
%\usepackage{amsrefs}
% \usepackage[alphabetic,author-year,sorted]{amsrefs}
\usepackage{amscd}

\usepackage[colorlinks=true]{hyperref}
%\hypersetup{urlcolor=blue, citecolor=red}

%\usepackage[varg]{pxfonts}   % Nice fonts! The pxfonts package must 
                             % be loaded AFTER amssymb, amsmath etc

\usepackage{latexsym}
\usepackage{mathrsfs}
\usepackage{psfrag}
\usepackage[dvips]{epsfig}
\usepackage{epsfig}
\usepackage[all]{xy}         % Commutative diagrams


%-------------       SPACING      -------------------%

%\setlength{\textwidth}{6.5in}                    % Change to 1-inch margins.
%\setlength{\textheight}{9in}
%\setlength{\evensidemargin}{0in}
%\setlength{\oddsidemargin}{0in}
%\setlength{\topmargin}{+.4in}

\setlength{\oddsidemargin}{0.in}                 % Alternative formatting
\setlength{\evensidemargin}{0.in}
\setlength{\textwidth}{6.46in}
\setlength{\textheight}{8.4in}

%\usepackage[margin=2.5cm]{geometry}

% \sloppy


%%%%%%%%%%%    THEOREM STYLES      %%%%%%%%%%%%%%%

\swapnumbers 

\theoremstyle{plain}                              % Italicized                
\newtheorem{thm}{Theorem}[section]
\newtheorem{prop}[thm]{Proposition}
\newtheorem{defn}[thm]{Definition}
\newtheorem{lem}[thm]{Lemma}
\newtheorem{cor}[thm]{Corollary}
\newtheorem*{conj}{Conjecture}
\newtheorem*{mainthm}{Main Theorem}               % Or Theorem A, etc.

\theoremstyle{definition}                         % Non-italicized
\newtheorem{example}[thm]{Example}
\newtheorem*{remark}{Remark} 

\theoremstyle{remark}                             % Non-italicized &
\newtheorem*{rem}{Remark}                         % non-bold heading
\newtheorem{case}{Case}
\newtheorem*{note}{Note}

\numberwithin{equation}{section}

\renewcommand{\baselinestretch}{1.09}


%%%%%%%%%%%      PAGE STYLE    %%%%%%%%%%%%%%%%

% \pagestyle{empty}
% \pagestyle{plain}
% \pagestyle{headings}

%%%%%%%%%%%      COMMANDS      %%%%%%%%%%%%%%%%

\newcommand{\R}{\mathbb{R}}                     % Reals
\newcommand{\Z}{\mathbb{Z}}                     % Integers
\newcommand{\C}{\mathbb{C}}                     % Complex numbers
\newcommand{\N}{\mathbb{N}}                     % Natural numbers

\newcommand{\ms}[1]{\mathscr{#1}}               % Script

\newcommand{\too}{\longrightarrow}

\newcommand{\vect}[1]{\mathbf{#1}}              % Vectors

\newcommand{\noi}{\noindent} 

\newcommand{\veps}{\varepsilon}        

\newcommand{\del}{\partial}
         


%%%%%%%%%%%    OPERATORS     %%%%%%%%%%%%%%%%%

\DeclareMathOperator{\rank}{rank}

\DeclareMathOperator{\codim}{codim}

\DeclareMathOperator{\vol}{vol}

\providecommand{\norm}[1]{\left\lVert#1\right\rVert}       % Norm 

\providecommand{\abs}[1]{\left\lvert#1\right\rvert}        % Absolute value



\newcommand{\aperta}{\setlength{\itemsep}{-1pt}} % This shrinks enumerate 
                   % and itemize. see Lamport, p. 185

%%%%%%%%%%%%%%%%%%%%%%%%%%%%%%%%%%%%%%%%%%%%%%%%%%%%%%%%%
%%%%%%%%%%%           DOCUMENT BODY      %%%%%%%%%%%%%%%%
%%%%%%%%%%%%%%%%%%%%%%%%%%%%%%%%%%%%%%%%%%%%%%%%%%%%%%%%%

\begin{document}

\title[short title]{Title}

\author{Your name here}

\address{Department of Mathematics, San Jos\'e State University, San
  Jos\'e, CA 95192-0103}

\email{your email address here}

\thanks{\textsc{Math 231A: Real Analysis I}}

\date{\today}

\begin{abstract}
  Write a short abstract here.
\end{abstract}

\maketitle


\section{Introduction}       \label{sec:intro}

Introduce the topic here, talk about the motivation, intuitive
understanding of whatever you are writing about, give some basic
examples, etc. For instance:

\textsf{Ergodic theory} is a branch of mathematics that studies
statistical behavior of orbits of a dynamical system, that is, a map
$T : X \to X$, where $X$ is a space equipped with some probability
measure $\mu$ and $T$ is a transformation which preserves $\mu$:
%
\begin{displaymath}
  \mu(T^{-1}(E)) = \mu(E),
\end{displaymath}
%
for all measurable sets $E$ in $X$. We call $\mu$ a probability
measure if $\mu(X) = 1$. 

\section{Basic concepts}

Define things, introduce the notation and terminology, describe
examples (in more detail than in Section \ref{sec:intro}. For instance:

\begin{defn}
  A map $T : X \to X$ is called \textsf{ergodic} with respect to a
  measure $\mu$ on $X$ if $T$ preserves $\mu$ (in the sense described
  above) and every $T$-invariant measurable set $E$ has either measure
  zero or measure one. That is,
%
  \begin{displaymath}
    T^{-1}(E) = E \quad \Rightarrow \quad \mu(E) \in \{ 0, 1 \}.
  \end{displaymath}

\end{defn}

Our main reference is \cite{walters}.


\section{Main results}      \label{sec:properties}

Talk about the most important properties of your subject, state the
main results of the field, etc. For instance:

\begin{thm}[Birkhoff's Ergodic Theorem]
%
  Suppose that $T : X \to X$ is ergodic with respect to $\mu$ and let
  $f : X \to \R$ be an integrable function. Then for $\mu$-a.e. $x \in
  X$, 
%
  \begin{displaymath}
    \lim_{n \to \infty} \frac{1}{n} \sum_{i=0}^{n-1} f(T^ix) = \int_X f \: d\mu.
  \end{displaymath}
%
\end{thm}

This can be summarized as: \textit{the time averages of $f$ converge to the
space average of $f$.}
%  



\section{Applications}     \label{sec:applications}

Write about some applications, if any. For instance:

Ergodic theory is closely related to physics, where $f$ is called an
observable. Birkhoff's Ergodic Theorem allows one to compute the space
average of an observable $f$ by taking sufficiently many measurements
and averaging them.


%%%%%%% old reference style %%%%%%%

\bibliographystyle{amsplain} 
\bibliography{ref}    


%%%%%%%   amsrefs  %%%%%%%

%\begin{bibdiv}
%\begin{biblist}

% \bibselect{master}                           % Use master.ltb file         

%\end{biblist}
%\end{bibdiv}


\end{document}

%%% Local Variables: 
%%% mode: latex
%%% TeX-master: t
%%% End: 
