\documentclass[letterpaper,12pt,fleqn]{article}
\usepackage{matharticle}
\usepackage{tikz}
\pagestyle{plain}
\newcommand{\T}{\mathscr{T}}
\newcommand{\U}{\mathcal{U}}
\newcommand{\e}{\epsilon}
\renewcommand{\a}{\alpha}
\renewcommand{\l}{\lambda}
\renewcommand{\C}{\mathcal{C}}
\DeclareMathOperator{\Int}{Int}
\DeclareMathOperator{\Bd}{Bd}
\begin{document}
Cavallaro, Jeffery \\
Math 275A \\
Homework \#3

\bigskip

\begin{lemma}
  Let \((X,\T)\) be a topological space, \(A\subset X\), and \(p\in X\):
  \[p\in\bar{A}\iff\forall\,U\in\U_p,U\cap A\ne\emptyset\]
\end{lemma}

\begin{proof}
  By definition, \(p\in\bar{A}\) iff \(p\in A\) or \(\forall\,U\in\U_p,(U-\set{p})\cap A\ne\emptyset\).  Assume that
  \(U\in\U_p\).  If \(p\in A\) then \(p\in U\cap A\ne\emptyset\).  If \(p\notin A\) then
  \((U-\set{p})\cap A=U\cap A\).  In either case: \(p\in A\) or \(\forall\,U\in\U_p,(U-\set{p})\cap A\ne\emptyset\)
  is logically equivalent to \(\forall\,U\in\U_p,U\cap A\ne\emptyset\).
\end{proof}

\begin{theorem}[2.16]
  Let \((X,\T)\) be a topological space:
  \begin{enumerate}
  \item \(\emptyset\) is closed.
  \item \(X\) is closed.
  \item The union of finitely many closed sets is closed.
  \item Let \(\set{A_{\a}:\a\in\l}\) be a family of closed sets.  \(\bigcap_{\a\in\l}A_{\a}\) is closed.
  \end{enumerate}
\end{theorem}

\begin{proof}
  \begin{enumerate}
  \item[]
  \item \(X\) is open, so \(X-X=\emptyset\) is closed.
  \item \(\emptyset\) is open, so \(X-\emptyset=X\) is closed.
  \item \(X-\bigcup_{i=1}^nA_i=\bigcap_{i=1}^n(X-A_i)\).

    But the \(X-A_i\) are open and thus \(X-\bigcup_{i=1}^nA_i\) is open.

    Therefore \(\bigcup_{i=1}^nA_i\) is closed.

  \item \(X-\bigcap_{\a\in\l}A_{\a}=\bigcup_{\a\in\l}(X-A_{\a})\).

    But the \(X-A_{\a}\) are open and thus \(X-\bigcap_{\a\in\l}A_{\a}\) is open.

    Therefore \(\bigcup_{\a\in\l}A_{\a}\) is closed.
  \end{enumerate}
\end{proof}

\begin{notation}
  Let \((X,\T)\) be a topological space and \(A\subset X\):
  \begin{gather*}
    \C=\setb{B\subset X}{B\ \text{is closed}} \\
    \C_A=\setb{B\in\C}{A\subset B}
  \end{gather*}
\end{notation}

\begin{theorem}[2.20]
  Let \((X,\T)\) be a topological space and \(A\subset X\).  The closure of \(A\) equals the intersection of all
  closed sets containing \(A\):
  \[\bar{A}=\bigcap\C_A\]
  Thus, \(\bar{A}\) is the smallest closed set containing \(A\).
\end{theorem}

\begin{proof}
  Since \(A\subset\bar{A}\) and \(\bar{A}\) is closed, \(\bar{A}\in\C_A\) and so:
  \[\bar{A}\supset\bigcap\C_A\]
  ABC:
  \[\bar{A}\supsetneq\bigcap\C_A\]
  This means that there exists some \(B'\in\C_A\) such that:
  \[\bar{A}\supsetneq\bar{A}\cap B'\supset A\]
  where \(\bar{A}\cap B'\in\C\).

  This would imply that there exists some closed set containing \(A\) with less limit points of \(A\) than
  \(\bar{A}\), which contradicts the definition of \(\bar{A}\).

  Therefore, \(\displaystyle\bar{A}=\bigcap\C_A\).
\end{proof}

\begin{theorem}[2.22]
  Let \((X,\T)\) be a topological space and \(A,B\subset X\):
  \begin{enumerate}
  \item \(A\subset B\implies \bar{A}\subset\bar{B}\)
  \item \(\overline{A\cup B}=\bar{A}\cup\bar{B}\)
  \end{enumerate}
\end{theorem}

\begin{proof}
  \begin{enumerate}
  \item[]
  \item Assume \(A\subset B\).

    Assume \(p\in\bar{A}\).  This means that:
    \[\forall\,U\in\U_p,U\cap A\ne\emptyset\]
    But \(A\subset B\) and so
    \[\forall\,U\in\U_p,U\cap B\ne\emptyset\]
    meaning that \(p\in\bar{B}\) as well.

    Therefore \(\bar{A}\subset\bar{B}\).

  \item
    \begin{description}
    \item[\((\subset)\)] Since \(A\subset\bar{A}\) and \(B\subset\bar{B}\):
      \[A\cup B\subset\bar{A}\cap\bar{B}\]
      But \(\bar{A}\cap\bar{B}\) is closed and the smallest closed set containing \(A\cup B\) is
      \(\overline{A\cup B}\).  Therefore:
      \[A\cup B\subset\overline{A\cup B}\subset\bar{A}\cup\bar{B}\]
    \item[\((\supset)\)] Since \(A\subset A\cup B\):
      \[\bar{A}\subset\overline{A\cup B}\]
      and similarly:
      \[\bar{B}\subset\overline{A\cup B}\]
      Therefore:
      \[\bar{A}\cup\bar{B}\subset\overline{A\cup B}\]
    \end{description}
  \end{enumerate}
\end{proof}

\begin{example}[Exercise 2.24]
  Let \((R^2,\T)\):

  \begin{enumerate}
    \item Topologist's Sine Curve
      \[S=\setb{\left(x,\sin\left(\frac{1}{x}\right)\right)}{x\in(0,1)}\]

      \begin{tikzpicture}
        \draw [help lines] (-1,0) -- (4,0);
        \draw [help lines] (0,-2) -- (0,2);
        \draw [samples=10000,domain=0.04:4] plot ({\x},{sin(deg(4/(\x)))});
      \end{tikzpicture}
      \[\bar{S}=S\cup\set{(1,\sin(1))}\cup\setb{(0,y)}{y\in[-1,1]}\]

    \item Topologists Comb
      \[C=\setb{(x,0)}{x\in[0,1]}\cap\bigcup_{n=1}^{\infty}\setb{\left(\frac{1}{n},1)\right)}{y\in[0,1]}\]

      \begin{tikzpicture}
        \draw [help lines] (-1,0) -- (4,0);
        \draw [help lines] (0,-1) -- (0,2);
        \draw [very thick] (0,0) -- (3,0);
        \foreach \n in {1,...,100}{
          \draw ({3/(\n)},0) -- ({3/(\n)},1);
        }
      \end{tikzpicture}
      \[\bar{C}=C\cup\setb{(0,y)}{y\in[0,1]}\]
  \end{enumerate}
\end{example}

\begin{theorem}[2.26]
  Let \((X,\T)\) be a topological space, \(A\subset X\), and \(p\in X\).  \(p\) is an interior point of \(A\) iff
  there exists \(U\in\T\) such that \(p\in U\subset A\).
\end{theorem}

\begin{proof}
  \(p\in\Int(A)\iff p\in\bigcup\U_A\iff\exists\,U\in\U_A,p\in U\subset A\)
\end{proof}

\begin{theorem}[2.28]
  Let \((X,\T)\) be a topological space and let \(A\subset X\).  \(\Int(A)\), \(\Bd(A)\), and \(\Int(X-A)\) are
  disjoint sets whose union is \(X\).
\end{theorem}

\begin{proof}
  Assume that \(p\in\Int(A)\).  This means that there exists \(U\in\U_A\) such that \(p\in U\subset A\).  Now ABC
  that \(p\in\Bd(A)\).  This means that \(p\in\overline{X-A}\) and so for all \(U\in\U_p, U\cap(X-A)\ne\emptyset\).
  This contradicts the fact that there exists a \(U\in\U_p\) that is a subset of \(A\).

  Therefore \(\Int(A)\cap\Bd(A)=\emptyset\).

  Similarly, assume that \(p\in\Int(X-A)\).  This means that there exists \(U\in\U_{X-A}\) such that \(p\in
  U\subset(X-A)\).  Now ABC that \(p\in\Bd(A)\).  This means that \(p\in\bar{A}\) and so for all \(U\in\U_p, U\cap
  A\ne\emptyset\).  This contradicts the fact that there exists a \(U\in\U_p\) that is a subset of \(X-A\).

  Therefore \(\Int(X-A)\cap\Bd(A)=\emptyset\).

  Finally, note that for all \(U\in\U_p\), \(U\) cannot be a subset of both \(A\) and \(X-A\).

  Therefore \(\Int(A)\cap\Int(X-A)=\emptyset\).

  Clearly, \(\Int(A)\cup\Int(X-A)\cup\Bd(A)\subset X\).  Assume that \(p\in X\).  If \(p\in\Int(A)\) or
  \(p\in\Int(X-A)\) then done, so assume that \(X\) is in neither.  This means that for all \(U\in\U_p\),
  \(U\cap A\ne\emptyset\) and \(U\cap(X-A)\ne\emptyset\), and thus \(p\in\bar{A}\) and \(p\in\overline{X-A}\).

  Therefore, \(p\in\Bd(A)\).
\end{proof}

\begin{theorem}[2.30]
  Let \((X,\T)\) be a topological space, \(A\subset X\), and \(p\in X\):
  \[\setb{x_i}{i\in\N}\subset A\ \text{and}\ x_i\to p\implies p\in\bar{A}\]
\end{theorem}

\begin{proof}
  Assume that \(\setb{x_i}{i\in\N}\subset A\ \text{and}\ x_i\to p\).  Assume that \(U\in\U_p\).  This means that
  there exists some \(N\in\N\) such that for all \(i>N\) it is the case that \(x_i\in U\).  But \(x_i\in A\) also,
  and so \(U\cap A\ne\emptyset\).

  Therefore \(p\in\bar{A}\).
\end{proof}

\begin{theorem}[2.31]
  Let \((\R^n,\T)\) be the standard topology, \(A\subset\R^n\), and \(p\in X\) be a limit point of \(A\).  There
  exists a sequence of points in \(A\) that converge to \(p\).
\end{theorem}

\begin{proof}
  Select \(x_1\in A\) such that \(x_1\ne p\).  Note that \(x_1\in B(p,2\abs{x_1-p})\).  Now select \(x_{i+1}\in
  B(p,\abs{x_i-p})\cap A\), which cannot be empty since \(p\) is a limit point of \(A\).  These \(x_i\) fulfill the
  requirements for a sequence \((x_i)\) in \(A\) that converges to \(p\).
\end{proof}

\begin{example}[Exercise 2.32]
  Find an example of a topological space and a convergent sequence in that space for which the limit of the
  sequence is not unique.

  Consider \((\R,\T)\) with the indiscrete topology and consider any random sequence of points \(x_i\).  Since
  \(\R\) is the only non-empty open set, any \(p\in\R\) is a suitable limit for \((x_i)\).
\end{example}

\end{document}
