\documentclass[letterpaper,12pt,fleqn]{article}
\usepackage{matharticle}
\newcommand{\T}{\mathscr{T}}
\renewcommand{\a}{\alpha}
\renewcommand{\b}{\beta}
\renewcommand{\l}{\lambda}
\newcommand{\e}{\epsilon}
\begin{document}
Cavallaro, Jeffery \\
Math 275A \\
Homework \#13

\bigskip

\begin{theorem}[8.3]
  \(\R_{\text{std}}\) is connected.
\end{theorem}

\begin{proof}
  Since \(\R\) is homeomorphic to \((0,1)\), it is sufficient to show that \((0,1)\) is connected.  So ABC that
  \((0,1)\) is disconnected.  This means that there exists \(A\subset(0,1)\) such that \(A\ne\emptyset,(0,1)\) and
  \(A\) is clopen.  Since \(A\) is bounded, it has a \(\sup\), so let \(a=\sup A\).  But \(A\) is closed, so
  \(a\in A\).  But \(A\) is also open, so there exists \(\e>0\) such that \(B(a,\e)\subset A\), violating the fact
  that \(a=\sup A\).  Therefore \((0,1)\) is connected, and so \(\R\) is connected.
\end{proof}

\begin{theorem}[Exercise 8.7]
  The closure of the topologist's sine curve in \(\R^2\) is connected.
\end{theorem}

\begin{proof}
  Let:
  \begin{gather*}
    S=\setb{\left(x,\sin\left(\frac{1}{x}\right)\right)}{x\in(0,1)} \\
    \\
    \bar{S}=S\cup\set{(1,\sin(1))}\cup\setb{(0,y)}{y\in[-1,1]}
  \end{gather*}
  ABC that \(S\) is not connected.  This means that there exists \(g:S\to\set{0,1}\) such that \(g\) is continuous
  and surjective.  But \(f:(0,1)\to S\) defined by \(f(x)=(x,\sin\frac{1}{x})\) is also continuous and surjective.
  This means that \(g\circ f:(0,1)\to\set{0,1}\) is also continuous and surjective, indicating that \((0,1)\)
  is not connected, contradicting the connectedness of the interval.  Therefore \(S\) is connected, and by
  previous corollary, \(\bar{S}\) is connected.
\end{proof}

\end{document}
