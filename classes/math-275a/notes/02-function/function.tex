\documentclass[letterpaper,12pt,fleqn]{article}
\usepackage{matharticle}
\pagestyle{empty}
\begin{document}
\section*{Functions}

\begin{definition}[Function]
  A \emph{function} (\emph{map}) from a set \(X\) to a set \(Y\), denoted by \(f:X\to Y\), is a rule that assigns to
  each \(x\in X\) a corresponding \(f(x)\in Y\).  \(X\) is called the \emph{domain} of \(f\) and \(Y\) is called
  the \emph{codomain} of \(f\).
\end{definition}

\begin{definition}[Image]
  Let \(f:X\to Y\) be a function and let \(A\subset X\).  The \emph{image} of \(A\) under \(f\) is given by:
  \[f(A)=\setb{f(a)\in Y}{a\in A}\]
\end{definition}

\begin{definition}[Preimage]
  Let \(f:X\to Y\) be a function and let \(B\subset Y\).  The \emph{preimage} of \(B\) under \(f\) is given by:
  \[f^{-1}(B)=\setb{x\in X}{f(x)\in B}\]
  When \(B=\set{y}\) (a single point) then the alternate notation \(f^{-1}(y)\) is often used.
\end{definition}

\begin{theorem}
  Let \(f:X\to Y\) be a function and let \(A,B\subset Y\):
  \begin{gather*}
    f^{-1}(A\cup B)=f^{-1}(A)\cup f^{-1}(B) \\
    f^{-1}(A\cap B)=f^{-1}(A)\cap f^{-1}(B)
  \end{gather*}
\end{theorem}

\begin{proof}
  \begin{align*}
    x\in f^{-1}(A\cup B) &\iff f(x)\in A\cup B \\
    &\iff f(x)\in A\ \text{or}\ f(x)\in B \\
    &\iff x\in f^{-1}(A)\ \text{or}\ x\in f^{-1}(B) \\
    &\iff x\in f^{-1}(A)\cup f^{-1}(B) \\
    \therefore f^{-1}(A\cup B)=f^{-1}(A)\cup f^{-1}(B) \\
    \\
    x\in f^{-1}(A\cap B) &\iff f(x)\in A\cap B \\
    &\iff f(x)\in A\ \text{and}\ f(x)\in B \\
    &\iff x\in f^{-1}(A)\ \text{and}\ x\in f^{-1}(B) \\
    &\iff x\in f^{-1}(A)\cap f^{-1}(B) \\
    \therefore f^{-1}(A\cap B)=f^{-1}(A)\cap f^{-1}(B)
  \end{align*}
\end{proof}

\begin{definition}[Injection]
  Let \(f:X\to Y\) be a function.  To say that \(f\) is an \emph{injection} (\emph{one-to-one}) means that:
  \[\forall\,a,b\in X,f(a)=f(b)\implies a=b\]
\end{definition}

\begin{definition}[Surjection]
  Let \(f:X\to Y\) be a function.  To say that \(f\) is a \emph{surjection} (\emph{onto}) means that:
  \[\forall\,b\in Y,\exists\,a\in X,f(a)=b\]
\end{definition}

\begin{definition}[Bijection]
  Let \(f:X\to Y\) be a function.  To say that \(f\) is a \emph{bijection} (\emph{one-to-one correspondence}) means
  that \(f\) is both an injection and a surjection.
\end{definition}

\begin{theorem}
  Let \(f:X\to Y\) be a function and let \(y\in Y\).  If \(f\) is injective then \(f^{-1}(y)\) contains at most
  one point.
\end{theorem}

\begin{proof}
  Assume \(a,b\in f^{-1}(y)\).  By definition: \(f(a)=f(b)=y\).  But \(f\) is injective and therefore \(a=b\).
\end{proof}

\begin{theorem}
  Let \(f:X\to Y\) be a function and let \(y\in Y\).  If \(f\) is surjective then \(f^{-1}(y)\) contains at least
  one point.
\end{theorem}

\begin{proof}
  Since \(f\) is surjective, for all \(y\in Y\), there exists \(x\in X\) such that \(f(x)=y\).  Therefore, by
  definition, \(x\in f^{-1}(y)\).
\end{proof}

\end{document}
