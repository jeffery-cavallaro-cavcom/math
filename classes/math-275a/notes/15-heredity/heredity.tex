\documentclass[letterpaper,12pt,fleqn]{article}
\usepackage{matharticle}
\pagestyle{empty}
\newcommand{\T}{\mathscr{T}}
\begin{document}
\section*{Heredity}

\begin{definition}[Heredity]
  Let \(P\) be a topological property.  To say that a topological space \(X\) is hereditarily \(P\) means that each
  subspace \(Y\) of \(X\) has property \(P\) when \(Y\) is given the relative topology from \(X\).
\end{definition}

\begin{theorem}
  Every \(T_2\) space is hereditarily \(T_2\).
\end{theorem}

\begin{proof}
  Assume that \(X\) is a \(T_2\) topological space and assume that \(Y\subset X\).  Now assume that \(a,b\in Y\).
  Thus \(a,b\in X\) and, since \(X\) is \(T_2\), there exists \(U,V\in\T_X\) such that \(a\in U\), \(b\in V\), and
  \(U\cap V=\emptyset\).  Furthermore, \(a\in U\cap Y\in\T_Y\) and \(b\in V\cap Y\in\T_Y\).  And so:
  \[(Y\cap U)\cap(Y\cap V)=Y\cap(U\cap V)=Y\cap\emptyset=\emptyset\]
  Therefore \(Y\) is also \(T_2\).
\end{proof}

\begin{theorem}
  Every regular space is hereditarily regular.
\end{theorem}

\begin{proof}
  Assume that \(X\) is a regular topological space and assume that \(Y\subset X\).  Assume that \(p\in Y\).  This
  means that there exists some \(U_Y\in\T_Y\) such that \(p\in U_Y\), and hence there exists \(U_X\in\T_X\) such
  that \(U_Y=U_X\cap Y\) and so \(p\in U_X\).  Now, since \(X\) is regular, there exists \(V_X\in\T_X\) such that
  \(p\in V_X\subset\overline{V_X}\subset U_X\), and hence \(p\in V_X\cap Y=V_Y\in\T_Y\).  Furthermore, since
  \(\overline{V_X}\) is closed in \(X\), \(\overline{V_X}\cap Y=W_Y\) is closed in \(Y\).  Finally, since
  \(\overline{V_Y}\) is the smallest closed set in \(Y\) containing \(V_Y\):
  \[p\in V_Y\subset\overline{V_Y}\subset W_Y\subset U_Y\]
  Therefore \(Y\) is regular.
\end{proof}

\begin{lemma}
  Let \(X\) be a normal topological space and let \(Y\subset X\) such that \(Y\) is closed in \(X\).  For all
  \(A\subset Y\), if \(A\) is closed in \(Y\) then \(A\) is closed in \(X\).
\end{lemma}

\begin{proof}
  Assume \(A\subset Y\) such that \(A\) is closed in \(Y\).  This means that \(Y-A\in\T_Y\), and so there exists
  \(W\in\T_X\) such that \(W\cap Y=Y-A\).  Furthermore, \(X-W\) is closed in \(X\).  Now:
  \[(X-W)\cap Y=(X\cap Y)-(W\cap Y)=Y-(Y-A)=A\]
  But \(X-W\) and \(Y\) are closed in \(X\) and therefore \(A\) is also closed in \(X\).
\end{proof}

\begin{theorem}
  Let \(X\) be a normal topological space and let \(Y\subset X\) such that \(Y\) is closed in \(X\).  \(Y\) is
  normal when given the relative topology.
\end{theorem}

\begin{proof}
  Assume \(A,B\subset Y\) such that \(A\) and \(B\) are closed in \(Y\) and \(A\cap B=\emptyset\).  This means that
  \(A\) and \(B\) are also closed in \(X\).  Since \(X\) is normal, there exists \(U,V\in\T_X\) such that \(A\in U\),
  \(B\in V\), and \(U\cap V=\emptyset\).  Finally, since \(A\subset(U\cap Y)\in\T_Y\) and \(B\subset(V\cap Y)\in\T_Y\):
  \[(U\cap Y)\cap(V\cap Y)=(U\cap V)\cap Y=\emptyset\cap Y=\emptyset\]
  Therefore \(Y\) is normal.
\end{proof}

\end{document}
