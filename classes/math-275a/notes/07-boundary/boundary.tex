\documentclass[letterpaper,12pt,fleqn]{article}
\usepackage{matharticle}
\pagestyle{empty}
\newcommand{\T}{\mathscr{T}}
\newcommand{\U}{\mathcal{U}}
\DeclareMathOperator{\Int}{Int}
\DeclareMathOperator{\Bd}{Bd}
\begin{document}
\section*{Boundaries}

\begin{notation}
  Let \((X,\T)\) be a topological space and let \(A\subset X\):
  \[\U_A=\setb{U\in\T}{U\subset A}\]
\end{notation}

\begin{definition}[Interior]
  Let \((X,\T)\) be a topological space and let \(A\subset X\).  The \emph{interior} of \(A\), denoted by
  \(\Int(A)\), is given by:
  \[\Int(A)=\bigcup\U_A\]
  To say that a \(p\in X\) is an \emph{interior point} of \(A\) means that \(p\in\Int(A)\).
\end{definition}

\begin{theorem}
  Let \((X,\T)\) be a topological space, \(A\subset X\), and \(p\in X\).  \(p\) is an interior point of \(A\) iff
  there exists \(U\in\T\) such that \(p\in U\subset A\).
\end{theorem}

\begin{proof}
  \(p\in\Int(A)\iff p\in\bigcup\U_A\iff\exists\,U\in\U_A,p\in U\subset A\)
\end{proof}

\begin{theorem}
  Let \((X,\T)\) be a topological space and \(U\subset\T\):
  \[U\in\T\iff\forall\,p\in U,p\in\Int(U)\]
\end{theorem}\

\begin{proof}
  \(\U\in T\iff\forall\,p\in U,\exists U_p\in\T,p\in U_p\subset U\iff\forall\,p\in U,p\in\Int(U)\)
\end{proof}

\begin{definition}[Boundary]
  Let \((X,\T)\) be a topological space and let \(A\subset X\).  The \emph{boundary} or \(A\), denoted by
  \(\Bd(A)\), is given by:
  \[\Bd(A)=\bar{A}\cap\overline{X-A}\]
\end{definition}

\begin{theorem}
  Let \((X,\T)\) be a topological space and let \(A\subset X\).  \(\Int(A)\), \(\Bd(A)\), and \(\Int(X-A)\) are
  disjoint sets whose union is \(X\).
\end{theorem}

\begin{proof}
  Assume that \(p\in\Int(A)\).  This means that there exists \(U\in\U_A\) such that \(p\in U\subset A\).  Now ABC
  that \(p\in\Bd(A)\).  This means that \(p\in\overline{X-A}\) and so for all \(U\in\U_p, U\cap(X-A)\ne\emptyset\).
  This contradicts the fact that there exists a \(U\in\U_p\) that is a subset of \(A\).

  Therefore \(\Int(A)\cap\Bd(A)=\emptyset\).

  Similarly, assume that \(p\in\Int(X-A)\).  This means that there exists \(U\in\U_{X-A}\) such that \(p\in
  U\subset(X-A)\).  Now ABC that \(p\in\Bd(A)\).  This means that \(p\in\bar{A}\) and so for all \(U\in\U_p, U\cap
  A\ne\emptyset\).  This contradicts the fact that there exists a \(U\in\U_p\) that is a subset of \(X-A\).

  Therefore \(\Int(X-A)\cap\Bd(A)=\emptyset\).

  Finally, note that for all \(U\in\U_p\), \(U\) cannot be a subset of both \(A\) and \(X-A\).

  Therefore \(\Int(A)\cap\Int(X-A)=\emptyset\).

  Clearly, \(\Int(A)\cup\Int(X-A)\cup\Bd(A)\subset X\).  Assume that \(p\in X\).  If \(p\in\Int(A)\) or
  \(p\in\Int(X-A)\) then done, so assume that \(X\) is in neither.  This means that for all \(U\in\U_p\),
  \(U\cap A\ne\emptyset\) and \(U\cap(X-A)\ne\emptyset\), and thus \(p\in\bar{A}\) and \(p\in\overline{X-A}\).

  Therefore, \(p\in\Bd(A)\).
\end{proof}

\begin{example}
  Pick several different subsets \(A\) of \(\R\) and for each one find its interior and boundary using:
  \begin{enumerate}
  \item The discrete topology.

    Since \(A=\bar{A}\) and \(\R-A=\overline{\R-A}\), \(\Bd(A)=A\cap(\R-A)=\emptyset\).  Therefore \(\Int(A)=A\).

  \item The indiscrete topology.
    \[\Int(A)=\begin{cases}
    \emptyset, & A\ne\R \\
    \R, & A=\R
    \end{cases}\]
    \[\bar{A}=\begin{cases}
    \emptyset, & A=\emptyset \\
    \R, & A\ne\emptyset
    \end{cases}\]
    \[\overline{\R-A}=\begin{cases}
    \emptyset, & A=\R \\
    \R, & A\ne\R
    \end{cases}\]
    \[\Bd(A)=\begin{cases}
    \emptyset\cap\R=\emptyset, & A=\emptyset \\
    \R\cap\R=\R, & A\ne\emptyset,\R \\
    \R\cap\emptyset=\emptyset, & A=\R
    \end{cases}\]

  \item The cofinite topology.

    Assume that \(A\) is finite (closed):
    \begin{gather*}
      \Int(A)=\emptyset \\
      \bar{A}=A \\
      \Int(\R-A)=\R-A \\
      \overline{\R-A}=\R \\
      \Bd(A)=A\cap\R=A
    \end{gather*}

    Assume that \(A=\R-F\) where \(F\) is finite (thus \(A\) is open and \(F\) is closed):
    \begin{gather*}
      \Int(A)=A \\
      \bar{A}=\R \\
      \Int(\R-A)=\emptyset \\
      \overline{\R-A}=F \\
      \Bd(A)=\R\cap F=F
    \end{gather*}

    Assume that \(A=\Z\):
    \begin{gather*}
      \Int(\Z)=\emptyset\\
      \bar{\Z}=\R \\
      \Int(\R-\Z)=\emptyset \\
      \overline{\R-\Z}=\R \\
      \Bd(Z)=\R\cap\R=\R
    \end{gather*}

  \item The standard topology.

    Assume that \(A=(a,b)\):
    \begin{gather*}
      \Int(A)=A \\
      \bar{A}=[a,b] \\
      \Int(\R-A)=(-\infty,a)\cup(b,\infty) \\
      \overline(\R-A)=(-\infty,a]\cup[b,\infty) \\
      \Bd(A)=[a,b]\cap(-\infty,a]\cup[b,\infty)=\set{a,b} \\
    \end{gather*}

    Assume that \(A=[a,b]\):
    \begin{gather*}
      \Int(A)=(a,b) \\
      \bar{A}=A \\
      \Int(\R-A)=(-\infty,a)\cup(b,\infty) \\
      \overline(\R-A)=(-\infty,a]\cup[b,\infty) \\
      \Bd(A)=[a,b]\cap(-\infty,a]\cup[b,\infty)=\set{a,b} \\
    \end{gather*}

    Assume that \(A=\Z\):
    \begin{gather*}
      \Int(\Z)=\emptyset \\
      \bar{\Z}=\Z \\
      \Int(\R-\Z)=\R-\Z \\
      \overline{\R-\Z}=\R \\
      \Bd(Z)=\R\cap\Z=\Z
    \end{gather*}

    Assume that \(A=\Q\):
    \begin{gather*}
      \Int(\Q)=\emptyset \\
      \bar{\Q}=\R \\
      \Int(\R-\Q)=\R-\Q \\
      \overline{\R-\Q}=\R \\
      \Bd(\Q)=\R\cap\R=\R
    \end{gather*}
  \end{enumerate}
\end{example}

\end{document}
