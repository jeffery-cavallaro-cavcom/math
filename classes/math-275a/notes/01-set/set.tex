\documentclass[letterpaper,12pt,fleqn]{article}
\usepackage{matharticle}
\pagestyle{empty}
\renewcommand{\a}{\alpha}
\renewcommand{\l}{\lambda}
\begin{document}
\section*{Sets}

\begin{notation}[Element]
  Let \(A\) be a set.  The notation \(a\in A\) indicates that \emph{element} \(a\) is in set \(A\).
\end{notation}

\begin{definition}[Subset]
  Let \(A\) and \(B\) be sets.  To say that \(A\) is a \emph{subset} of \(B\), denoted by \(A\subset B\), means
  that:
  \[a\in A\implies a\in B\]
  In particular, a set is a subset of itself (\(A\subset A\)) and the empty set \(\emptyset\) is a subset of every
  other set.
\end{definition}

\begin{definition}[Equality]
  Let \(A\) and \(B\) be sets.  To say that \(A\) is \emph{equal} to \(B\), denoted by \(A=B\), means that:
  \[a\in A\iff a\in B\]
  or alternatively:
  \[A\subset B\ \text{and}\ B\subset A\]
\end{definition}

\begin{definition}[Proper]
  Let \(A\) and \(B\) be sets.  To say that \(A\) is a \emph{proper} subset of \(B\), denoted by \(A\subsetneq B\),
  means that \(A\subset B\) but \(A\ne B\).  Thus, \(B\not\subset A\), meaning \(\exists\,b\in B,b\notin A\).
\end{definition}

\begin{definition}[Operations]
  Let \(A\), \(B\), and \(X\) be sets such that \(A,B\subset X\):
  \begin{description}
  \item[Union:] \(A\cup B=\setb{x\in X}{x\in A\ \text{or}\ x\in B}\)
  \item[Intersection:] \(A\cap B=\setb{x\in X}{x\in A\ \text{and}\ x\in B}\)
  \item[Complement:] \(X-A=\setb{x\in X}{x\notin A}\)
  \end{description}
  When \(X\) is understood, \(X-A\) can be denoted by \(A^C\).
\end{definition}

\begin{theorem}[DeMorgan]
  Let \(A_1\), \(A_2\), and \(X\) be sets such that \(A_1,A_2\subset X\):
  \begin{gather*}
  (A_1\cup A_2)^C=A_1^C\cap A_2^C \\
  (A_1\cap A_2)^C=A_1^C\cup A_2^C
  \end{gather*}
\end{theorem}

\begin{proof}
  Assume \(x\in X\):
  \begin{align*}
    x\in(A_1\cup A_2)^C &\iff x\notin A_1\cup A_2 \\
    &\iff x\notin A_1\ \text{and}\ x\notin A_2 \\
    &\iff x\in A_1^C\ \text{and}\ x\in A_2^C \\
    &\iff x\in A_1^C\cap A_2^C \\
    \therefore(A_1\cup A_2)^C=A_1^C\cap A_2^C \\
    \\
    x\in(A_1\cap A_2)^C &\iff x\notin A_1\cap A_2 \\
    &\iff x\notin A_1\ \text{or}\ x\notin A_2 \\
    &\iff x\in A_1^C\ \text{or}\ x\in A_2^C \\
    &\iff x\in A_1^C\cup A_2^C \\
    \therefore(A_1\cap A_2)^C=A_1^C\cup A_2^C
  \end{align*}
\end{proof}

\begin{notation}
  Let \(X\) be a set and let \(\set{A_{\a}:\a\in\l}\) be a family of sets such that \(A_{\a}\subset X\):
  \begin{gather*}
    \bigcup_{\a\in\l}A_{\a}=\setb{x\in X}{\exists\,\a\in\l,x\in A_{\a}} \\
    \bigcap_{\a\in\l}A_{\a}=\setb{x\in X}{\forall\,\a\in\l,x\in A_{\a}}
  \end{gather*}
\end{notation}

\begin{theorem}[General DeMorgan]
  Let \(X\) be a set and let \(\set{A_{\a}:\a\in\l}\) be a family of sets such that \(A_{\a}\subset X\):
  \begin{gather*}
    \left(\bigcup_{\a\in\l}A_{\a}\right)^C=\bigcap_{\a\in\l}A_{\a}^C \\
    \left(\bigcap_{\a\in\l}A_{\a}\right)^C=\bigcup_{\a\in\l}A_{\a}^C
  \end{gather*}
\end{theorem}

\begin{proof}
  Assume \(x\in X\):
  \begin{align*}
    x\in\left(\bigcup_{\a\in\l}A_{\a}\right)^C &\iff x\notin\bigcup_{\a\in\l}A_{\a} \\
    &\iff \forall\,\a\in\l,x\notin A_{\a} \\
    &\iff \forall\,\a\in\l,x\in A_{\a}^C \\
    &\iff x\in\bigcap_{\a\in\l}A_{\a}^C \\
    \therefore\left(\bigcup_{\a\in\l}A_{\a}\right)^C=\bigcap_{\a\in\l}A_{\a}^C \\
    \\
    x\in\left(\bigcap_{\a\in\l}A_{\a}\right)^C &\iff x\notin\bigcap_{\a\in\l}A_{\a} \\
    &\iff \exists\,\a\in\l,x\notin A_{\a} \\
    &\iff \exists\,\a\in\l,x\in A_{\a}^C \\
    &\iff x\in\bigcup_{\a\in\l}A_{\a}^C \\
    \therefore\left(\bigcap_{\a\in\l}A_{\a}\right)^C=\bigcup_{\a\in\l}A_{\a}^C
  \end{align*}
\end{proof}

\end{document}
