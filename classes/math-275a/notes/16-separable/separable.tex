\documentclass[letterpaper,12pt,fleqn]{article}
\usepackage{matharticle}
\pagestyle{empty}
\newcommand{\T}{\mathscr{T}}
\newcommand{\Rs}{\R_{\text{std}}}
\newcommand{\RL}{\R_{LL}}
\newcommand{\RZ}{\R_{+00}}
\begin{document}
\section*{Separable Spaces}

\begin{definition}[Dense]
  Let \(X\) be a topological space and let \(A\subset X\).  To say that \(A\) is \emph{dense} in \(X\) means that
  \(\bar{A}=X\).
\end{definition}

\begin{theorem}
  Let \(X\) be a topological space and let \(A\subset X\).  \(A\) is dense in \(X\) iff for all \(U\in\T\),
  \(U\ne\emptyset\implies U\cap A\ne\emptyset\).
\end{theorem}

\begin{proof}
  \begin{description}
  \item[]
  \item[\(\implies\)] Assume that \(A\) is dense in \(X\), and hence \(\bar{A}=X\).

    Assume that \(U\in\T\) and assume that \(U\ne\emptyset\).  Since \(\bar{A}=X\) it must be the case that
    \(U\cap\bar{A}\ne\emptyset\).  So assume that \(x\in U\cap\bar{A}\), meaning that \(x\in U\) and
    \(x\in\bar{A}\).  Therefore, since \(x\in\bar{A}\), it must be the case that \(U\cap A\ne\emptyset\).

  \item[\(\impliedby\)] Assume that \(\forall\,U\in\T,U\ne\emptyset\implies U\cap A\ne\emptyset\).

    Clearly, \(\bar{A}\subset X\).  So assume that \(x\in X\).  But by the assumption, \(x\in\bar{A}\).
    Therefore \(\bar{A}=X\) and hence \(A\) is dense in \(X\).
  \end{description}
\end{proof}

\begin{definition}[Separable]
  Let \(X\) be a topological space.  To say that \(X\) is \emph{separable} means that \(X\) has a countable dense
  subset.
\end{definition}

\begin{example}
  Show that \(\Rs\) is separable.  Which of the previously investigated topologies on \(\R\) are not separable?

  Consider \(\Q\subset\R\) and assume that \(x\in\R-\Q\).  But \(x\) is the limit of some sequence in \(\Q\) and
  hence \(x\in\bar{\Q}\).  This means that \(\bar{\Q}=\R\) and thus \(\Q\) is a countable dense subset of \(\R\).
  Therefore \(\Rs\) is separable.

  For \(\RL\), also consider \(\Q\subset\R\).  Assume \(U\in\T\).  This means that there exists some \(a,b\in\R\)
  such that \([a,b)\subset U\).  If \(a\in\Q\) then done, so assume \(a\in\R-\Q\).  Since \(\Q\) is dense in
  \(\Rs\), there exists \(x\in\Q\) such that \(x\in(a,b)\subset[a,b)\subset U\).  Therefore \(\Q\) is dense in
  \(\RL\) as well and so \(\RL\) is separable.

  For \(\RZ\), consider \(A=\set{0',0''}\cup\Q^+\).  Assume \(U\in\T\).  If \(0'\in U\) or \(0''\in U\) then done,
  so assume that neither is in \(U\).  This means that there exists some \(a,b\in\R^+\) such that \((a,b)\subset U\).
  Since \(\Q\) is dense in \(\Rs\), there exists \(x\in\Q+\) such that \(x\in(a,b)\).  Therefore \(A\) is dense
  and countable in \(\RZ\) and so \(\RZ\) is separable.
\end{example}

\begin{lemma}
  \((A\cap X)\times(B\cap Y)=(A\times B)\cap(X\times Y)\)
\end{lemma}

\begin{proof}
  \begin{align*}
    (a,b)\in(A\cap X)\times(B\cap Y) &\iff a\in A\cap X\ \text{and}\ b\in B\cap Y \\
    &\iff a\in X\ \text{and}\ a\in X\ \text{and}\ b\in B\ \text{and}\ b\in Y \\
    &\iff (a,b)\in A\times B\ \text{and}\ (a,b)\in X\times Y \\
    &\iff (a,b)\in (A\times B)\cap(X\times Y)
  \end{align*}
\end{proof}

\begin{theorem}
  Let \(X\) and \(Y\) be topological spaces.  If \(X\) and \(Y\) are separable then \(X\times Y\) is separable.
\end{theorem}

\begin{proof}
  Assume that \(X\) and \(Y\) are separable.  This means that there exists a countable dense \(A\subset X\) and a
  countable dense \(B\subset Y\).

  Claim: \(A\times B\) is countable and dense in \(X\times Y\).

  Since \(A\) and \(B\) are countable, \(A\times B\) is countable.

  Now, assume \(W\in\T_{X\times Y}\).  This means that there exists \(U\in\T_X\) and \(V\in\T_Y\) such that
  \(U\times V\subset W\).  But \(A\) is dense in \(X\) and so \(U\cap A\ne\emptyset\).  Likewise, \(B\) is dense in
  \(Y\) and so \(V\cap B\ne\emptyset\).  And so:
  \[(U\cap A)\times(V\cap B)=(U\times V)\cap(A\times B)\ne\emptyset\]
  Thus, \(W\cap (A\times B)\ne\emptyset\) and so \(A\times B\) is dense in \(X\times Y\).

  Therefore \(A\times B\) is countable and dense in \(X\times Y\) and hence \(X\times Y\) is separable.
\end{proof}

\end{document}
