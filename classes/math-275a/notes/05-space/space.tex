\documentclass[letterpaper,12pt,fleqn]{article}
\usepackage{matharticle}
\pagestyle{empty}
\newcommand{\T}{\mathscr{T}}
\renewcommand{\a}{\alpha}
\renewcommand{\l}{\lambda}
\newcommand{\e}{\epsilon}
\newcommand{\Tstd}{\T_{\text{std}}}
\begin{document}
\section*{Topological Spaces}

\begin{definition}[Topology]
  Let \(X\) be a set.  To say that \(\T\) is a \emph{topology} on \(X\) means that \(\T\) is a collection of
  subsets of \(X\) such that:
  \begin{enumerate}
  \item \(\emptyset\in\T\)
  \item \(X\in\T\)
  \item \(U,V\in\T\implies U\cap V\in\T\)
  \item \(\set{U_{\a}:\a\in\l}\) such that \(U_{\a}\in\T\implies \bigcup_{\a\in\l}U_{\a}\in\T\).
  \end{enumerate}
  A \emph{topological space} is a tuple \((X,\T)\) where \(X\) is a set and \(\T\) is a topology on \(X\).
\end{definition}

\begin{definition}[Open Set]
  Let \((X,\T)\) be a topological space and let \(U\subset X\).  To say that \(U\) is an \emph{open set} in
  \((X,\T)\) means that \(U\in\T\).
\end{definition}

\begin{theorem}
  The intersection of a finite number of open sets is open.
\end{theorem}

\begin{proof}
  Let \((X,\T)\) be a topological space and Let \(\set{U:1\le i\le n}\) be a finite collection of open sets in
  \((X,\T)\).

  Induction on \(n\).

  \begin{description}
  \item[Base Case:] \(n=1\)
    \[\bigcap_{i=1}^1U_i=U_1\in\T\]

  \item[Inductive Hypothesis:] Assume \(\bigcap_{i=1}^nU_i\in\T\).

  \item[Inductive Step:] Consider \(n+1\).
    \[\bigcap_{i=1}^{n+1}U_i=\bigcap_{i=1}^nU_i\cap U_{n+1}\]
    But \(\bigcap_{i=1}^nU_i\in\T\) (inductive assumption) and \(U_{n+1}\in\T\) (assumption).

    \(\therefore\bigcap_{i=1}^{n+1}U_i\in\T\) (property 3).
  \end{description}

  Therefore, by the principle of induction, \(\bigcap_{i=1}^nU_i\in\T\).
\end{proof}

Note that this proof does not work for an infinite number of open sets because induction is only valid for a finite
number of steps.

\begin{theorem}
  Let \(X,\T\) be a topological space and let \(U\subset X\):
  \[U\in\T\iff\forall\,x\in U,\exists\,U_x\in\T,x\in U_x\subset U\]
\end{theorem}

\begin{proof}
  \begin{description}
  \item[]

  \item[\(\implies\)] Assume \(U\in\T\).

    Assume \(x\in U\).  So \(x\in U\subset U\).

  \item[\(\impliedby\)] Assume \(\forall\,x\in U,\exists U_x\in\T,x\in U_x\subset U\).

    Claim: \(\bigcup_{x\in U}U_x=U\)

    First, assume that \(y\in \bigcup_{x\in U}U_x\).  Therefore \(\exists\,x\in U\) such that \(y\in U_x\subset U\).

    Next, assume that \(y\in U\).  This means that \(\exists\,U_y\in\T\) such that \(y\in U_y\subset U\), and
    therefore \(y\in\bigcup_{x\in U}U_x\).

    Thus, \(U\) is an arbitrary union of open sets and hence is open.

    \(\therefore U\in\T\)
  \end{description}
\end{proof}

\begin{definition}
  Let \((X,\T)\) be a topological space and let \(x\in X\).  To say that a set \(U\subset X\) is a \emph{neighborhood}
  of \(x\) means that \(x\in U\) and \(U\in\T\).
\end{definition}

\begin{corollary}
  Let \((X,\T)\) be a topological space and let \(U\subset X\).  \(U\) is an open set iff every point in \(U\) has
  a neighborhood that lies within \(U\).
\end{corollary}

\begin{definition}[Open Ball]
  Let \(p\in\R^n\).  The \emph{open ball} around \(p\) of radius \(\e>0\) is given by:
  \[B(p,\e)=\setb{x\in X}{d(p,x)<\e}\]
  where \(d(p,x)\) is the Euclidean distance from \(p\) to \(x\) given by:
  \[d(p,x)=\sqrt{\sum_{i=1}^n(p_i-x_i)^2}\]
\end{definition}

\begin{definition}[Topologies]
  \begin{description}
  \item[]
  \item[Standard:] \((\R^n,\Tstd)\)
    \[U\in\T\iff\forall\,x\in U,\exists\,\e_x>0,B(x,\e_x)\subset U\]
  \item[Discrete:] \((X,2^X)\)
    \[\forall\,U\subset X,U\in\T\]
  \item[Indiscrete:] \((X,\set{\emptyset,X})\)
    \[\T=\set{\emptyset,X}\]
  \item[Cofinite:] \((X,\T)\)
    \[U\in\T\iff U=\emptyset\ \text{or}\ X-U\ \text{is finite}\]
  \item[Cocountable:] \((X,\T)\)
    \[U\in\T\iff U=\emptyset\ \text{or}\ X-U\ \text{is countable}\]
  \end{description}
\end{definition}

\begin{example}
  Verify that \(\Tstd\) is a topology on \(R^n\).

  \begin{enumerate}
  \item \(\emptyset\in\Tstd\) (vacuously).

  \item Assume \(x\in R^n\).

    Let \(e_x=1\).  Since \(R^n\) includes everything it must be the case that \(B(x,1)\subset R^n\).

    Therefore \(R^n\in\Tstd\).

  \item Assume \(U,V\in\Tstd\).

    Assume \(x\in U\cap V\).  This means that \(x\in U\) and \(x\in V\).  So there exists \(\e_U,\e_V\) such that
    \(B(x,\e_U)\subset U\) and \(B(x,\e_V)\subset V\).  Let \(\e=\min\set{\e_U,\e_V}\).  Thus
    \(B(x,\e)\subset U\cap V\).

    Therefore \(U\cap V\in\Tstd\).

  \item Assume \(\set{U_{\a}:\a\in\l}\) is a family of sets such that \(U_{\a}\in\Tstd\).

    Let \(U=\bigcup_{\a\in\l}U_{\a}\) and assume \(x\in U\).  This means that there exists \(\a\in\l\) such that
    \(x\in U_{\a}\).  Furthermore, there exists \(\e_x>0\) such that \(B(x,\e_x)\subset U_{\a}\subset U\).

    Therefore \(U\in\Tstd\).
  \end{enumerate}
\end{example}

\begin{example}
  Verify that the discrete, indiscrete, cofinite, and cocountable topologies are indeed topologies for any set \(X\).

  \begin{description}
  \item[Discrete]
    \begin{enumerate}
    \item[]
    \item \(\emptyset\subset X\) and so \(\emptyset\in\T\).

    \item \(X\subset X\) and so \(X\in\T\).

    \item Assume \(U,V\in\T\).

      \(U,V\subset X\) and so \(U\cap V\subset X\).

      Therefore \(U\cap V\in\T\).

    \item Assume \(\set{U_{\a}:\a\in\l}\) such that \(U_{\a}\in\T\).

      \(\forall\,\a\in\l,U_{\a}\subset X\) and so \(\bigcup_{\a\in\l}U_{\a}\subset X\).

      Therefore \(\bigcup_{\a\in\l}U_{\a}\in\T\).
    \end{enumerate}

  \item[Indiscrete]
    \begin{enumerate}
    \item[]
    \item By definition, \(\emptyset\in\T\).

    \item By definition, \(X\in\T\).

    \item Assume \(U,V\in\T\).
      \begin{align*}
        \emptyset\cap\emptyset &= \emptyset\in\T \\
        \emptyset\cap X &= \emptyset\in\T \\
        X\cap\emptyset &= \emptyset\in\T \\
        X\cap X &= X\in\T
      \end{align*}

      Therefore \(U\cap V\in\T\).

    \item Assume \(\set{U_{\a}:\a\in\l}\) such that \(U_{\a}\in\T\).
      \begin{align*}
        \emptyset\cup\emptyset &= \emptyset\in\T \\
        \emptyset\cup X &= X\in\T \\
        X\cup\emptyset &= X\in\T \\
        X\cup X &= X\in\T
      \end{align*}

      Therefore \(\bigcup_{\a\in\l}U_{\a}\in\T\).
    \end{enumerate}

  \item[Cofinite]
    \begin{enumerate}
    \item[]
    \item By definition, \(\emptyset\in\T\).

    \item \(X-X=\emptyset\), which is finite.  Therefore \(X\in\T\).

    \item Assume \(U,V\in\T\).

      \(X-(U\cap V)=(X-U)\cup(X-V)\).  But \(X-U\) and \(X-V\) are both finite and so their union is finite.

      Therefore \(U\cap V\in\T\).

    \item Assume \(\set{U_{\a}:\a\in\l}\) such that \(U_{\a}\in\T\).

      \(X-\bigcup_{\a\in\l}U_{\a}=\bigcap_{\a\in\l}(X-U_{\a})\).  But for all \(\a\in\l\) it is the case that
      \(X-U_{\a}\) is finite and so their intersection is finite.

      Therefore \(\bigcup_{\a\in\l}U_{\a}\in\T\).
    \end{enumerate}

  \item[Cocountable]
    \begin{enumerate}
    \item[]
    \item By definition, \(\emptyset\in\T\).

    \item \(X-X=\emptyset\), which is finite and thus countable.  Therefore \(X\in\T\).

    \item Assume \(U,V\in\T\).

      \(X-(U\cap V)=(X-U)\cup(X-V)\).  But \(X-U\) and \(X-V\) are both countable and so their union is countable.

      Therefore \(U\cap V\in\T\).

    \item Assume \(\set{U_{\a}:\a\in\l}\) such that \(U_{\a}\in\T\).

      \(X-\bigcup_{\a\in\l}U_{\a}=\bigcap_{\a\in\l}(X-U_{\a})\).  But for all \(\a\in\l\) it is the case that
      \(X-U_{\a}\) is countable.  Now, for some \(\a\in\l\):
      \[\bigcap_{\a\in\l}(X-U_{\a})\subset(X-U_{\a})\]
      However, the subset of a countable set is countable.

      Therefore \(\bigcup_{\a\in\l}U_{\a}\in\T\).
    \end{enumerate}
  \end{description}
\end{example}

\begin{example}
  Describe some of the open sets you get if \(\R\) is endowed with the standard, discrete, indiscrete, cofinite,
  and cocountable topologies.  Specifically, identify sets that demonstrate the differences among these topologies,
  that is, find sets that are open in some topologies but not in others.  For each of the topologies, determine
  if the interval \((0,1)\) is an open set in that topology.

  Of course, \(\emptyset\) and \(\R\) are in \(\T\) for all topologies.

  \begin{description}
  \item[Standard:] All open intervals, but not closed intervals, are in \(\T\).  Therefore \((0,1)\in\T\).

  \item[Discrete:] All open and closed intervals are in \(\T\).  Therefore \((0,1)\in\T\) and \([0,1]\in\T\)

  \item[Indiscrete:] Nothing other than \(\emptyset\) and \(\R\).  Therefore \((0,1)\notin\T\).

  \item[Cofinite] All open sets are of the form \(\R-X\) where \(X\) is finite.  For example: \(\R-\set{1,2,3}\).
    Thus, the open sets are uncountable.  Such sets are also open sets in all the other topologies sans indiscrete.
    Therefore \((0,1)\notin\T\).

  \item[Cocountable] All open sets are of the form \(\R-X\) where \(X\) is countable.  Thus, all open sets must
    include all but a countable number of irrational numbers and are thus uncountable.  For example: \(\R-\Q\) or
    \((\R-\Q)\cup\set{\sqrt{2},\sqrt{3}}\) or \((\R-\Q)\cup\set{\sqrt{2},\sqrt{3},\sqrt{5},\ldots}\).  Such sets are
    also open in the standard and discrete topologies.  Since finite sets are countable, it is the case that
    \(\T_{cof}\subset\T_{coc}\).  Therefore \((0,1)\notin\T\).
  \end{description}
\end{example}

\begin{example}
  Give an example of a topological space and a collection of open sets in that topological space that show that the
  infinite intersection of open sets need not be open.

  Consider \((\R,\Tstd)\) and the open sets \(\setb{(-\frac{1}{n},\frac{1}{n})}{n\in\N}\).  Their infinite
  intersection is \(\set{0}\), which is not open.
\end{example}

\end{document}
