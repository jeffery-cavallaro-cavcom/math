\documentclass[letterpaper,12pt,fleqn]{article}
\usepackage{matharticle}
\pagestyle{empty}
\begin{document}
\section*{Uncountable Sets}

\begin{theorem}[Cantor Diagonalization]
  The set of real numbers \(\R\) is uncountable.
\end{theorem}

\begin{proof}
  ABC that \((0,1)\) is countable.  This means that there exists some bijection \(f:\N\to(0,1)\).  Let \(a_{ij}\)
  be \(j^{th}\) decimal digit of the \(i^{th}\) number:
  \begin{align*}
    f(1) &= 0.a_{11}a_{12}a_{13}a_{14}a_{15}\cdots \\
    f(2) &= 0.a_{21}a_{22}a_{23}a_{24}a_{25}\cdots \\
    f(3) &= 0.a_{31}a_{32}a_{33}a_{34}a_{35}\cdots \\
    f(4) &= 0.a_{41}a_{42}a_{43}a_{44}a_{45}\cdots \\
    f(5) &= 0.a_{51}a_{52}a_{53}a_{54}a_{55}\cdots \\
    \vdots &= \vdots
  \end{align*}
  If \(f(n)\) is rational with more than one representation, for example: \(0.4\bar{9}=0.5\bar{0}\), then the
  repeating \(0\) case is selected.

  Now, let \(b=b_1b_2b_3b_4b_5\cdots\) where:
  \[b_i=\begin{cases}
  1, & a_{ii}\ne1 \\
  2, & a_{ii}=1
  \end{cases}\]
  So \(b\) never contains a \(0\) or \(9\) digit and thus the non-unique cases are avoided.  This means that
  \(b\in(0,1)\) but \(b\notin f(\N)\), contradicting the bijectiveness of \(f\).  Thus, \((0,1)\) is uncountable.
  But \((0,1)\subset\R\).

  Therefore \(\R\) is uncountable.
\end{proof}

\begin{definition}[Power Set]
  Let \(A\) be a set.  The \emph{power set} of \(A\), denoted by \(2^A\), is the set of all subsets of \(A\).
\end{definition}

\begin{example}
  Let \(A=\set{a,b,c}\):
  \[2^A=\set{\emptyset,\set{a},\set{b},\set{c},\set{a,b},\set{a,c},\set{b,c},\set{a,b,c}}\]
\end{example}

\begin{theorem}
  Let \(A\) be a finite set:
  \[\abs{2^A}=2^{\abs{A}}\]
\end{theorem}

\begin{proof}
  For each \(B\in2^A\), for each \(a\in A\), either \(a\in B\) or \(a\notin B\): \(2\) possibilities.  Therefore,
  since there are \(\abs{A}\) elements in \(A\):
  \[\abs{2^A}=2^{\abs{A}}\]
\end{proof}

\begin{theorem}
  Let \(A\) be a set.  There exists an injection from \(A\) to \(2^A\).
\end{theorem}

\begin{proof}
  Consider \(f:A\to2^A\) defined by \(f(a)=\set{a}\subset A\).  This is an injection from \(A\) to \(2^A\).
\end{proof}

\begin{theorem}
  Let \(A\) be a set and let \(P\) be the set of all functions from \(A\) to the two-point set \(\set{0,1}\):
  \[\abs{P}=\abs{2^A}\]
\end{theorem}

\begin{proof}
  Consider the function \(f:P\to2^A\) defined by \(f(p)=B\) such that:
  \[p(a)=\begin{cases}
  0, & a\notin B \\
  1, & a\in B
  \end{cases}\]
  Claim: \(f\) is a bijection.

  Assume \(f(p_1)=f(p_2)=B\).  Assume \(a\in A\).  If \(a\notin B\) then \(p_1(a)=p_2(a)=0\).  If \(a\in B\) then
  \(p_1(a)=p_2(a)=1\).  So \(\forall\,a\in A,p_1(a)=p_2(a)\).  Thus, by definition, \(p_1=p_2\) and therefore
  \(f\) is injective.

  Now, assume \(B\in 2^A\).  Since \(B\subset A\), for each \(a\in A\), \(a\) is either not in \(B\) or in \(B\).
  So define \(p:A\to\set{0,1}\) as above.  Thus \(p\in P\) and \(f(p)=B\).  Therefore \(f\) is surjective.

  Therefore \(f\) is a bijection and thus \(\abs{P}=\abs{2^A}\).
\end{proof}

\begin{theorem}
  Let \(B\) be the set of all bit strings of infinite length.
  \[\abs{B}=\abs{2^{\N}}\]
\end{theorem}

\begin{proof}
  Let \(P\) be the set of all functions from \(\N\) to the two-point set \(\set{0,1}\).  Consider the function
  \(f:P\to B\) defined by \(f(p)=b\) such that \(b=b_1b_2b_3\cdots\) and \(p(i)=b_i\).

  Claim: \(f\) is a bijection.

  Assume \(f(p_1)=f(p_2)=b\).  Assume \(i\in\N\).  If \(b_i=0\) then \(p_1(i)=p_2(i)=0\).  If \(b_i=1\) then
  \(p_1(i)=p_2(i)=1\).  So \(\forall\,i\in\N,p_1(i)=p_2(i)\).  Thus, by definition, \(p_1=p_2\) and therefore \(f\)
  is injective.

  Now, assume \(b\in B\).  For each \(i\in\N\), \(b_i\) is either \(0\) or \(1\).  So define \(p:\N\to\set{0,1}\) as
  above.  Thus \(p\in P\) and \(f(p)=b\).  Therefore \(f\) is surjective.

  Thus \(f\) is a bijection and \(\abs{P}=\abs{B}\).  But, by the previous theorem, \(\abs{P}=\abs{2^{\N}}\).

  \(\therefore\abs{B}=\abs{2^{\N}}\)
\end{proof}

\begin{theorem}[Cantor Power Set]
  Let \(A\) be a set:
  \[\abs{A}\ne\abs{2^A}\]
\end{theorem}

\begin{proof}
  Let \(f:A\to2^A\) and ABC that \(f\) is bijective.  For all \(a\in A\) let \(f(a)=B_a\).  This means that either
  \(a\notin B_a\) or \(a\in B_a\).  Now, construct \(B\in2^A\) as follows:
  \[B=\setb{a\in A}{a\notin f(a)}\]
  Note that if \(a\notin B_a\) then \(a\in B\) and if \(a\in B_a\) then \(a\notin B\) and so \(\forall\,a\in
  A,B_a\ne B\).  Thus, \(B\in2^A\) but \(B\notin f(A)\), contradicting the bijectiveness of \(f\).

  \(\therefore\abs{A}\ne\abs{2^A}\)
\end{proof}

\end{document}
