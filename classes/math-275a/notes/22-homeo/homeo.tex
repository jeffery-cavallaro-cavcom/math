\documentclass[letterpaper,12pt,fleqn]{article}
\usepackage{matharticle}
\pagestyle{empty}
\newcommand{\T}{\mathscr{T}}
\begin{document}
\section*{Homeomorphisms}

\begin{definition}[Homeomorphism]
  Let \(X\) and \(Y\) be topological spaces and let \(f:X\to Y\).  To say that \(f\) is a \emph{homeomorphism}
  means that \(f\) is a continuous bijection and \(f^{-1}\) is continuous.  If such an \(f\) exists then \(X\)
  and \(Y\) are said to be \emph{homeomorphic} or \emph{topologically equivalent}.
\end{definition}

\begin{theorem}
  Homeomorphic is an equivalence relation.
\end{theorem}

\begin{proof}
  Assume that \(X\),\(Y\), and \(Z\) are topological spaces.
  \begin{description}
  \item[R:] Consider \(i_X=i_X^{-1}\), which is continuous.  Therefore \(X\) is homeomorphic to \(X\).

  \item[S:] Assume that \(X\) is homeomorphic to \(Y\).

    Then there exists a homeomorphism \(f:X\to Y\).  Since \(f\) is a homeomorphism, it is invertible and its
    inverse is continuous. Thus, \(f^{-1}:Y\to X\) is a continuous, invertible function and \((f^{-1})^{-1}=f\) is
    invertible.  Therefore \(Y\) is homeomorphic to \(X\).

  \item[T:] Assume that \(X\) is homeomorphic to \(Y\) and \(Y\) is homeomorphic to \(Z\).

    Then there exists homeomorphics \(f:X\to Y\) and \(g:Y\to Z\).  So consider \(g\circ f:X\to Z\).  Since \(f\)
    and \(g\) are continuous and invertible, \(g\circ f\) is continuous and invertible.  Furthermore, since
    \(f^{-1}\) and \(g^{-1}\) are continuous, \(f^{-1}\circ g^{-1}=(g\circ f)^{-1}\) is continuous.  Therefore
    \(X\) is homeomorphic to \(Z\).
  \end{description}
\end{proof}

\begin{lemma}
  For all \(a,b\in\R\) such that \(a<b\), \((a,b)\) is homeomorphic to \((0,1)\).
\end{lemma}

\begin{proof}
  Let \(f:(0,1)\to(a,b)\) be defined by \(f(t)=a+t(a-b)\).  \(f\) is linear, and thus continuous and invertible
  with \(f^{-1}(s)=\frac{s-a}{b-a}\) which is also linear and thus continuous.  Therefore \((a,b)\) is homeomorphic
  to \((0,1)\).
\end{proof}

\begin{corollary}
  All open intervals in \(\R\) are homeomorphic.
\end{corollary}

\begin{proof}
  Assume \((a,b),(c,d)\subset\R\).  \((a,b)\) is homeomorphic to \((0,1)\) and \((0,1)\) is homeomorphic to
  \((c,d)\).  Therefore, \((a,b)\) is homeomorphic to \((c,d)\).
\end{proof}

\begin{theorem}
  \((a,b)\subset\R\) is homeomorphic to \(R\).
\end{theorem}

\begin{proof}
  \((a,b)\) is homeomorphic to \((-\frac{\pi}{2},\frac{\pi}{2})\).  Now, consider
  \(f:(-\frac{\pi}{2},\frac{\pi}{2})\to\R\) defined by \(f(x)=\tan x\).  This is a continuous and invertible
  function whose inverse is also continuous.  Thus, \((-\frac{\pi}{2},\frac{\pi}{2})\) is \(\R\).  Therefore,
  \((a,b)\) is homeomorphic to \(R\).
\end{proof}

\begin{theorem}
  Let \(X\) and \(Y\) be topological spaces and let \(f:X\to Y\) be continuous.  TFAE:
  \begin{enumerate}
  \item \(f\) is a homeomorphism.
  \item \(f\) is a closed bijection.
  \item \(f\) is an open bijection.
  \end{enumerate}
\end{theorem}

\begin{proof}
  \begin{description}
  \item[]
  \item[\((1\implies 2)\)] Assume that \(f\) is a homeomorphism.

    This means that \(f\) is a bijection and its inverse is continuous.  So assume that \(A\subset X\) is closed in
    \(X\). Since \(f\) is bijective, \(f(A)=(f^{-1})^{-1}(A)\), and since \((f^{-1})^{-1}\) is continuous, \(f(A)\)
    is also closed.  Therefore \(f\) is a closed bijection.

  \item[\((2\implies 3)\)] Assume that \(f\) is a closed bijection.

    Assume that \(U\in\T_X\).  This means that \(X-U\) is closed in \(X\), and since \(f\) is closed, \(f(X-U)\)
    is closed in \(Y\) and so \(Y-f(X-U)\in\T_Y\).  But \(f\) is a bijection and so \(Y-f(X-U)=f(U)\in\T_Y\).
    Therefore, \(f\) is an open bijection.

  \item[\((3\implies 1)\)] Assume that \(f\) is an open bijection.

    Assume that \(U\in\T_Y\).  Since \(f\) is continuous, \(f^{-1}(U)\in\T_X\).  But \(f\) is open so
    \((f^{-1})^{-1}(U)\in\T_Y\).  Therefore \(f^{-1}\) is continuous and hence \(f\) is a homeomorphism.
  \end{description}
\end{proof}

\begin{theorem}
  Let \(X\) and \(Y\) be topological spaces such that \(X\) is compact and \(Y\) is Hausdorff and let \(f:X\to Y\)
  be a continuous bijection.  \(f\) is a homeomorphism.
\end{theorem}

\begin{proof}
  Since \(X\) is compact, \(Y\) is Hausdorff, and \(f\) is a bijection, \(f\) is closed.  Therefore, since \(f\) is
  a continuous closed bijection, \(f\) is a homeomorphism.
\end{proof}

\begin{example}
  Let \(X\) and \(Y\) be topological spaces and let \(f\) be continuous bijection.
  \begin{enumerate}
  \item \(Y\) is \(T_2\) but \(X\) is not compact.

    Consider \(X=[0,2\pi]\) and \(Y=S^1\) with \(f:X\to Y\) defined by \(f(t)=e^{it}\).  But \(f^{-1}(e^{it})=t\)
    is not continuous.

  \item \(X\) is compact but \(Y\) is not \(T_2\).

    Consider \(X=[0,1]_{\text{std}}\) and \(Y=[0,1]_{\text{ind}}\) with \(f(x)=x\).  But \(f\) is not open.
  \end{enumerate}
\end{example}

\end{document}
