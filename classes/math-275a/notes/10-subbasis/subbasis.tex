\documentclass[letterpaper,12pt,fleqn]{article}
\usepackage{matharticle}
\pagestyle{empty}
\newcommand{\T}{\mathscr{T}}
\renewcommand{\S}{\mathcal{S}}
\newcommand{\B}{\mathcal{B}}
\begin{document}
\section*{Subbases}

\begin{definition}[Subbasis]
  Let \((X,\T)\) be a topological space and let \(\S\subset2^X\).  To say that \(\S\) is a \emph{subbasis} for \(\T\)
  means that the set \(\B\) consisting of all finite intersections of subsets of \(\S\) is a basis for \(\T\).
\end{definition}

\begin{theorem}
  Let \(\T\) be the standard topology on \(\R\) and let:
  \[\S=\setb{(-\infty,b)}{b\in\R}\cup\setb{(a,\infty)}{a\in\R}\]
  \(\S\) is a subbasis for \(\T\).
\end{theorem}

\begin{proof}
  Since all of the sets in \(S\) are open, all finite intersections are also open.  In particular, for all
  \(a,b\in\R\) such that \(a<b\):
  \[(-\infty,b)\cap(a,\infty)=(a,b)\]
  But the \((a,b)\) are known to be a basis of \(\T\).  Furthermore, adding more open sets to a basis just results
  in a finer basis.

  Therefore, \(\S\) is a subbasis of \(\T\).
\end{proof}

\end{document}
