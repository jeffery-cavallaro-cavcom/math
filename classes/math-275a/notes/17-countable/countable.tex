\documentclass[letterpaper,12pt,fleqn]{article}
\usepackage{matharticle}
\pagestyle{empty}
\newcommand{\T}{\mathscr{T}}
\newcommand{\B}{\mathcal{B}}
\newcommand{\U}{\mathcal{U}}
\renewcommand{\sc}{\(2^{nd}\) countable}
\newcommand{\fc}{\(1^{st}\) countable}
\newcommand{\Rs}{\R_{\text{std}}}
\newcommand{\RL}{\R_{LL}}
\newcommand{\e}{\epsilon}
\renewcommand{\d}{\delta}
\renewcommand{\a}{\alpha}
\renewcommand{\l}{\lambda}
\begin{document}
\section*{Countable Spaces}

\begin{definition}[\(2^{nd}\) Countable]
  Let \(X\) be a topological space.  To say that \(X\) is \emph{\sc} means that \(X\) has a countable basis.
\end{definition}

\begin{theorem}
  Let \(X\) be a topological space.  If \(X\) is \sc\ then \(X\) is separable.
\end{theorem}

\begin{proof}
  Assume that \(X\) is \sc\ and let \(\B=\set{B_i:i\in\N}\) be a countable basis for \(X\).  From each \(B_i\),
  select a value \(x_i\) and construct the set \(A=\set{x_1,x_2,\ldots}\).  Thus \(x_i\mapsto B_i\) is one-to-one
  and so \(A\) is countable.  Now assume that \(U\in\T\).  Then there exists at least some \(B_i\subset U\) and
  hence \(U\cap A\ne\emptyset\), and so \(A\) is countable and dense in \(X\).

  Therefore \(X\) is separable.
\end{proof}

There are two ways to determine that a set is not \sc:
\begin{enumerate}
\item Show that all possible bases are uncountable.
\item Show that all countable subsets are not bases.
\end{enumerate}

\begin{example}
  \begin{enumerate}
    \item[]
    \item Show that \(\Rs\) is \sc\ (and hence separable).

      Consider the countable set \(\B=\setb{(a,b)}{a,b\in\Q}\).  Since \(\Q\) is countable, \(\Q\times\Q\) is
      countable and hence \(\B\) is countable.  Now assume that \(U\in\T\) and assume \(x\in U\).  Since \(x\) is
      an interior point of \(U\), there exists some \(\e>0\) such that \(x\in(x-\e,x+\e)\subset U\).  So there must
      exist \(\d\in\Q\) such that \(0<\d<\e\), and so \(x\in(x-\d,x+\d)\subset(x-\e,x+\e)\subset U\).  But
      \((x-\d,d+\d)\in\B\), and so \(\B\) is a countable basis for \(\Rs\).

      Therefore \(\Rs\) is \sc.

    \item Show that \(\RL\) is separable but not \sc.

      It was already shown that \(\RL\) is separable.  So assume that \(\B\) is a basis for \(\RL\) and consider
      \(\displaystyle U_a=[a,\infty)=\bigcup_{b>a}[a,b)\in\T\).  Then there exists some \(B_a\in\B\) such that
      \(a\in B_a\).  Now, assume \(x,y\in\R\) such that \(x<y\).  Since \(U_y\subsetneq U_x\), there exists
      \(B_x\subset U_x\) and \(B_y\subset U_y\) such that \(B_x\ne B_y\).  Thus, \(x\mapsto B_x\) is injective
      and hence \(\B\) is uncountable.

      Therefore \(\RL\) is not \sc.
  \end{enumerate}
\end{example}

\begin{theorem}
  Every uncountable set in a \sc\ space has a limit point.
\end{theorem}

\begin{proof}
  Assume that \(X\) is a \sc\ space and assume that \(A\subset X\) such that \(A\) is uncountable.  Now, ABC that
  \(A\) has no limit points.  This means that for all \(a\in A\) it is the case that there exists \(U\in\U_a\) such
  that \(U_a\cap A=\set{a}\) and hence every \(a\in A\) is an isolated point.  So assume that \(x,y\in A\) such
  that \(x\ne y\).  There exists \(U\in\U_x\) and \(V\in\U_y\) such that \(U\ne V\).  So for any basis \(\B\) of
  \(X\), there exists \(B_x,B_y\in B\) such that \(B_x\ne B_y\) and \(B_x\subset U\) and \(B_y\subset V\).  Thus,
  \(a\mapsto B_a\) is injective and hence \(\B\) is uncountable, contradicting the assumption that \(X\) is \sc.

  Therefore \(A\) contains a limit point.
\end{proof}

\begin{theorem}
  If \(X\) and \(Y\) are \sc\ spaces then \(X\times Y\) is \sc.
\end{theorem}

\begin{proof}
  Assume that \(\B_X\) is a countable basis for \(X\) and \(B_y\) is a countable basis for \(Y\).

  Claim: \(\B_X\times\B_Y\) is a countable basis for \(X\times Y\).

  \(\B_X\times\B_Y\) is countable.  So assume that \(U\in\T_{X\times Y}\) and assume \((a,b)\in U\).  This means that
  there exists \(U_a\in\T_X\) and \(V_b\in\T_Y\) such that \((a,b)\in U_a\times V_b\subset U\).  Furthermore, there
  exists \(B_a\in\B_X\) and \(B_b\in\B_Y\) such that \((a,b)\in B_a\times B_b\subset U_a\times V_b\subset U\) and
  so \(\B_X\times\B_Y\) is a countable basis for \(X\times Y\).

  Therefore \(X\times Y\) is \sc.
\end{proof}

\begin{definition}[Neighborhood Basis]
  Let \(X\) be a topological space and let \(p\in X\).  To say that a collection of sets
  \(\set{U_{\a}\in\U_p : \a\in\l}\) is a \emph{neighborhood basis} for \(p\) means that for all \(U\in\U_p\)
  there exists some \(U_{\a}\subset U\).
\end{definition}

\begin{definition}[\(1^{st}\) Countable]
  Let \(X\) be a topological space.  To say that \(X\) is \emph{\fc} means that every \(p\in X\) has a countable
  neighborhood basis.
\end{definition}

\begin{theorem}
  Let \(X\) be a topological space.  If \(X\) is \sc\ then \(X\) is \fc.
\end{theorem}

\begin{definition}[Souslin]
  Let \(X\) be a topological space.  To say that \(X\) has the \emph{Souslin property} means that \(X\) does not
  contain uncountable collection of disjoint open sets.
\end{definition}

\begin{theorem}
  \(\Rs\) is Souslin.
\end{theorem}

\begin{proof}
  ABC that \(\Rs\) is not Souslin, meaning it does contain an uncountable collection of disjoint open sets.  Let
  \(\U\) be such a set.  Since \(\Q\) is countable and dense in \(\Rs\), every \(U\in\U\) contains some
  \(r_U\in\Q\).  So select one value from each \(U\in\U\) to construct the set \(\setb{r_U\in Q}{U\in\U}\).
  Thus \(r_U\mapsto U\) is injective and hence \(\U\) is countable, contradicting the assumption.

  Therefore \(\Rs\) is Souslin.
\end{proof}

\end{document}
