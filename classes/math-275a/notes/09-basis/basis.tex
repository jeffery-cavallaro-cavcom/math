\documentclass[letterpaper,12pt,fleqn]{article}
\usepackage{matharticle}
\pagestyle{empty}
\newcommand{\T}{\mathscr{T}}
\newcommand{\B}{\mathcal{B}}
\newcommand{\e}{\epsilon}
\renewcommand{\a}{\alpha}
\newcommand{\A}{A}
\renewcommand{\l}{\lambda}
\renewcommand{\L}{\Lambda}
\begin{document}
\section*{Bases}

\begin{definition}[Basis]
  Let \((X,\T)\) be a topological space and let \(\B\subset\T\).  To say that \(\B\) is a \emph{basis} for \(\T\)
  means that:
  \[\forall\,U\in\T,\exists\,\B_U\subset\B,U=\bigcup\B_U\]
  The elements of \(\B\) are called \emph{basis elements} or \emph{basic open sets}.
\end{definition}

\begin{theorem}
  Let \((X,\T)\) be a topological space and let \(\B\subset\T\).  \(\B\) is a basis for \(\T\) iff:
  \[\forall\,U\in\T,\forall\,p\in U,\exists\,V\in\B,p\in V\subset U\]
\end{theorem}

\begin{proof}
  \begin{description}
  \item[]
  \item[\(\implies\)] Assume that \(\B\) is a basis for \(\T\).

    Assume \(U\in\T\).  This means that there exists \(\B_U\subset\B\) such that \(U=\bigcup\B_U\).  Now, assume
    that \(p\in U\).  Thus, \(p\in\bigcup\B_U\) and therefore there exists some \(V\in\B_U\) such that
    \(p\in V\subset U\).

  \item[\(\impliedby\)] Assume \(\forall\,U\in\T,\forall\,p\in U,\exists\,V\in\B,p\in V\subset U\)

    Assume that \(U\in\T\).  For each \(p\in U\), choose a set \(V_p\in\B\) such that \(p\in V_p\subset U\).  Thus
    \(U=\bigcup_{p\in U}V_p\) and so every \(U\in\T\) is generated by \(\B\).  Therefore \(\B\) is a basis for
    \(\T\).
  \end{description}
\end{proof}

\begin{example}
  Let \(\T\) be the standard topology on \(\R\).  Show that the following are bases for \(\T\):
  \begin{enumerate}
  \item \(\B_1=\setb{(a,b)\subset\R}{a,b\in\Q}\)

    All the \((a,b)\) are open and hence \(\B_1\subset\T\).  So assume \(U\in\T\) and assume \(p\in U\).  Since
    \(U\) is open, there exists an open ball \(B(p,\e)\subset U\).  Since there exists an infinite number of
    rationals between any two reals, select two rationals \(a\in(p-\e,p)\) and \(b\in(p,p+\e)\).  Thus
    \((a,b)\in\B_1\) and \(p\in(a,b)\subset U\).  Therefore, by the previous theorem, \(\B_1\) is a basis for
    \(\T\).

  \item \(\B_2=\setb{(a,b)\cup(c,d)\subset\R}{a,b,c,d\in\R-\Q\ \text{and}\ a<b<c<d}\)

    All the \((a,b)\cup(c,d)\) are unions of open sets, so they are open as well and hence \(\B_2\subset\T\).  So
    assume \(U\in\T\) and assume \(p\in U\).  Since \(U\) is open, there exists an open ball \(B(p,\e)\in U\).
    Since there exists an infinite number of irrationals between any two real numbers, select four irrationals as
    follows:
    \begin{gather*}
      a\in(p-\e,p) \\
      b\in(p,p+\e) \\
      c\in(b,p+\e) \\
      d\in(c,p+\e)
    \end{gather*}
    Thus \(a<b<c<d\) and so \((a,b)\cup(c,d)\in\B_2\). Furthermore, \(p\in(a,b)\cup(c,d)\subset U\).  Therefore, by
    the previous theorem, \(\B_2\) is a basis for \(\T\).
  \end{enumerate}
\end{example}

\begin{theorem}
  Let \(X\) be a set and let \(\B\subset2^X\).  \(\B\) is a basis for some topology \(\T\) on \(X\) iff:
  \begin{enumerate}
  \item \(\forall\,p\in X,\exists\,V_p\in\B,p\in V_p\)
  \item \(\forall\,U,V\in\B,\forall\,p\in U\cap V,\exists\,W\in\B,p\in W\subset U\cap V\)
  \end{enumerate}
\end{theorem}

\begin{proof}
  \begin{description}
  \item[]
  \item[\(\implies\)] Assume \(\B\) is a basis for some topology \(\T\) on \(X\).

    Since \(X\in\T\), (1) must hold by the previous theorem.  So assume \(U,V\in\B\).  If \(U\cap V=\emptyset\)
    then (2) is vacuously true, so assume that \(U\cap V\ne\emptyset\) and assume \(p\in U\cap V\).  Since
    \(U,V\in\B\subset\T\), both \(U\) and \(V\) are open and hence \(U\cap V\) is open.  So there exists some
    \(U_p\in\T\) such that \(p\in U_p\subset U\cap V\).  But \(U_p\) is generated by \(\B\), and so there must
    exist some \(W\in\B\) such that \(p\in W\subset U\cap V\).  Therefore (2) holds as well.

  \item[\(\impliedby\)] Assume that properties (1) and (2) hold.

    WTS: \(\T=\setb{U\subset2^X}{\exists\,\B_U\subset\B,U=\bigcup\B_U}\) is a topology on \(X\).

    First, consider \(\emptyset\).  Since \(\emptyset\subset\B\) and \(\emptyset\) is the union of no sets,
    \(\emptyset\in\T\).

    Next, consider \(X\).  By (1), \(X\) is generated by \(\B\) and hence \(X\in\T\).

    Next, assume \(U,V\in\T\) and let \(U\) be generated by \(\B_U\subset\B\) and let \(V\) be generated by
    \(\B_V\subset\B\), specifically:
    \begin{gather*}
      U=\bigcup\B_U=\bigcup_{\a\in\A}B_{\a} \\
      V=\bigcup\B_V=\bigcup_{\l\in\L}B_{\l}
    \end{gather*}
    Then:
    \[U\cap V=\bigcup\B_U\cap\bigcup\B_V=\bigcup_{\a\in\A}B_{\a}\cap\bigcup_{\l\in\L}B_{\l}=
    \bigcup_{\a\in\A,\l\in\L}(B_{\a}\cap B_{\l})\]
    But by (2), each of the \(B_{\a}\cap B_{\l}\) is generated by some subset of \(\B\), and hence \(U\cap V\) is
    generated by some subset of \(\B\).  Therefore \(U\cap V\in\T\).

    Finally, assume that \(\set{U_{\a}:\a\in\A}\subset\T\) and let \(U=\bigcup_{\a\in\A}U_{\a}\).  But each \(U_{\a}\)
    is generated by some subset of \(\B\) and hence \(U\) is generated by some subset of \(\B\).  Therefore
    \(U\in\T\).

    Therefore \(\T\) is a topology on \(X\).
  \end{description}
\end{proof}

\begin{theorem}[Lower Limit Topology]
  Let \(\B=\setb{[a,b)}{a,b\in\R}\).  \(\B\) is a basis for a topology on \(\R\) called the \emph{lower limit}
    topology and denoted by \(\R_{LL}\).  It is also known as the \emph{Sorgenfrey line}, denoted by
    \(\R_{bad}^1\).
\end{theorem}

\begin{proof}
  Assume that \(p\in\R\).  There exists \(\e>0\) such that \(B(p,\e)\subset\R\).  Let \(a=p-\e\) and \(b=p+\e\).
  Thus \(p\in[a,b)\in\B\).

  Now, assume that \(U,V\in\B\).  Let \(U=[a,b)\) and \(V=[c,d)\):
  \begin{description}
  \item[Case 1:] \(b\le c\) or \(a\ge d\).

    The \(U\cap V=\emptyset\) and property (2) holds vacuously.

  \item[Case 2:] \(U\subset V\) or \(V\subset U\).

    AWLOG that \(U\subset V\).  Then \(U\cap V=U\) and so property (2) holds trivially.

  \item[Case 3:] Otherwise.

    AWLOG that \(a<c\).  Then \(U\cap V=[c,b)\in\B\) and so property (2) holds.
  \end{description}

  Therefore \(\R_{LL}\) is a topology on \(\R\).
\end{proof}

\begin{theorem}
  On \(\R\), \(\T_{std}\subseteq\T_{LL}\).
\end{theorem}

\begin{proof}
  Consider the basis \(B=\setb{(a,b)\subset\R}{a,b\in\R}\) for \(T_{std}\) and assume \((a,b)\in\B\). Then:
  \[(a,b)=\bigcup_{x\in(a,b)}[x,b)\]
  Thus everything in \(T_{std}\) is generated by \(B\), which is generated by \(\T_{LL}\).  Therefore,
  \(\T_{std}\subset\T_{LL}\).  However, \([a,b)\) is not open in \(T_{std}\).  Therefore
  \(\T_{std}\subsetneq\T_{LL}\).
\end{proof}

\begin{definition}[Finer]
  Let \(X\) be a set and let \(\T\) and \(\T'\) be two topologies on \(X\).  To say that \(\T'\) is \emph{finer}
  than \(\T\) means that \(\T\subset\T'\).  Also, \(T\) is \emph{coarser}.  Furthermore, if \(\T\ne\T'\) then the
  terms \emph{strictly} finer or coarser are used.
\end{definition}

\begin{example}
  Give an example of two topologies on \(R\) such that neither is finer than the other, that is, the two topologies
  are not comparable.

  Consider the standard and cocountable topologies on \(R\).  \((0,1)\) is open in the standard topology; however,
  \(\R-(0,1)\) is uncountable and hence \((0,1)\) is not in the cocountable topology.  Likewise, \(\R-\Q\) is open
  in the cocountable topology (since \(\Q\) is countable); however, since there are an infinite number of rationals
  between any two irrationals, it is impossible to draw an open ball around any irrational that is a subset of the
  irrationals and so \(\R-\Q\) is not in the standard topology.  Therefore the two topologies are not comparable.
\end{example}

\begin{theorem}[Double-headed Snake]
  Let \(X=\R_{+00}=\R^+\cup{0',0''}\) where \(0',0''\in\R-\R^+\) and \(0'\ne0''\), and let:
  \[\B=\setb{(a,b),\set{0'}\cup(0,c),\set{0''}\cup(0,d)}{a,b,c,d\in\R^+}\]
  \(\B\) forms a basis for a topology \(\T\) on \(X\) called the \emph{double-headed snake} topology.
\end{theorem}

\begin{proof}
  Assume that \(x\in\R_{+00}\):
  \begin{description}
  \item[Case 1:] \(x=0'\)
    \[x\in\set{0'}\cup(0,c)\in\B\]
  \item[Case 2:] \(x=0''\)
    \[x\in\set{0''}\cup(0,d)\in\B\]
  \item[Case 3:] \(x\in\R^+\)
    \[x\in\left(\frac{x}{2},x+1\right)\in\B\]
  \end{description}
  Therefore property (1) holds.

  Next, assume that \(U,V\in\B\).  If \(U\cap V=\emptyset\) or \(U\subset V\) or \(V\subset U\) then done, so
  assume otherwise:

  \begin{description}
  \item[Case 1:] \(U=(a,b)\) and \(V=(c,d)\)

    AWLOG that \(a\le c\). Then \(U\cap V=(c,d)\in\B\).

  \item[Case 2:] \(U=(a,b)\) and \(V=\set{0'}\cup(0,c)\) or \(V=\set{0''}\cup(0,c)\)

    \(U\cap V=(a,\min\set{b,c})\in\B\)

  \item[Case 3;] \(U=\set{0'}\cup(0,c)\) or \(V=\set{0''}\cup(0,d)\)

    \(U\cap V=(\min\set{c,d},\max\set{c,d})\in\B\)
  \end{description}

  Therefore property (2) holds.

  Therefore the double-headed snake is a topology on \(\R_{00+}\).
\end{proof}

\begin{theorem}
  Let \(\T\) be the double-headed snake topology on \(\R_{00+}\):
  \begin{enumerate}
  \item Every point in \(\R_{00+}\) is a closed set.
  \item It is impossible for find disjoint open sets \(U\) and \(V\) such that \(0'\in U\) and \(0''\in V\).
  \end{enumerate}
\end{theorem}

\begin{proof}
  \begin{enumerate}
  \item[]
  \item Assume \(x\in\R_{00+}\) and consider \(\set{x}\).  Let \(A=\R_{00+}-\set{x}\):

    \begin{description}
    \item[Case 1:] \(x=0'\)

      \(A=(\set{0''}\cup(0,2))\cup\bigcup_{1<b\in\R^+}(1,b)\in\T\)
      
    \item[Case 2:] \(x=0''\)

      \(A=(\set{0'}\cup(0,2))\cup\bigcup_{1<b\in\R^+}(1,b)\in\T\)

    \item[Case 3:] \(x\in\R^+\)

      \(A=(\set{0'}\cup(0,x))\cup(\set{0''}\cup(0,x))\cup\bigcup_{x<b\in\R^+}(x,b)\in\T\)
    \end{description}

    Therefore \(A\) is open and thus \(\set{x}\) is closed.

  \item WTS: \(\forall\,U,V\in\T,(0'\in U\ \text{and}\ 0''\in V\implies U\cap V\ne\emptyset\))

    Assume \(U,V\in\T\) and assume \(0'\in U\) and \(0''\in V\).  This means that \(\set{0'}\cup(0,c)\subset U\)
    and \(\set{0''}\cup(0,d)\subset V\) for some \(c,d\in\R+\).  Since \(U\) and \(V\) are generated by these and
    possibly other basis elements, it must be the case that:
    \[(\set{0'}\cup(0,c))\cap(\set{0''}\cup(0,d))=(0,\min\set{c,d})\subset{U\cap V}\]
    But \((0,\min\set{c,d})\ne\emptyset\).

    Therefore \(U\cap V\ne\emptyset\).
  \end{enumerate}
\end{proof}

\end{document}
