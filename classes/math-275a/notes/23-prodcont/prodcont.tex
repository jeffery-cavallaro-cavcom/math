\documentclass[letterpaper,12pt,fleqn]{article}
\usepackage{matharticle}
\pagestyle{empty}
\newcommand{\T}{\mathscr{T}}
\renewcommand{\a}{\alpha}
\renewcommand{\l}{\lambda}
\begin{document}
\section*{Product Spaces and Continuity}

\begin{definition}[Projection]
  Let \(X\) and \(Y\) be topological spaces. The \emph{projection maps} \(\pi_X:X\times Y\to X\) and
  \(\pi_Y:X\times Y\to Y\) are defined by \(\pi_X(x,y)=x\) and \(\pi_Y(x,y)=y\).
\end{definition}

\begin{theorem}
  Let \(X\) and \(Y\) be topological spaces.  The projection maps \(\pi_X\) and \(\pi_Y\) are continuous,
  surjective, and open.
\end{theorem}

\begin{proof}
  Assume \(U\in\T_X\).  \(\pi_X^{-1}(U)=U\times Y\in\T_{X\times Y}\).  Therefore \(\pi_X\) is continuous.

  Next, assume that \(x\in X\).  Now, assume that \(y\in Y\), and so \((x,y)\in X\times Y\).  Thus,
  \(\pi_X(x,y)=X\).  Therefore \(\pi_X\) is surjective.

  Assume \(W\in\T_{X\times Y}\).  Then \(W=\bigcup_{\a\in\l}U_{\a}\times V_{\a}\), where \(U_{\a}\in\T_X\) and
  \(V_{\a}\in\T_Y\).  Now:
  \[\pi_X(W)=\pi_X(\bigcup_{\a\in\l}U_{\a}\times V_{\a})=\bigcup_{\a\in\l}\pi_X(U_{\a}\times V_{\a})=
  \bigcup_{\a\in\l}U_{\a}\in\T_X\]
  Thus, \(\pi_X\) is open.

  A similar argument is used for \(\pi_Y\).

  Therefore, \(\pi_X\) and \(\pi_Y\) are continuous, surjective, and open.
\end{proof}

\begin{example}
  Let \(X\) and \(Y\) be topological spaces.  \(\pi_X:X\times Y\to X\) and \(\pi_Y:X\times Y\to Y\) need not be
  closed.  Consider \(X=Y=\R\) and \(A=\setb{(x,y)}{xy=1}\).  Since all points of \(X-A\) are interior points,
  \(X-A\) is open and so \(A\) is closed.  But \(\pi_X(A)=\pi_Y(A)=\R-\set{0}\), which is not closed.
\end{example}

\begin{theorem}
  Let \(X\), \(Y\), and \(Z\) be topological spaces.  A function \(g:Z\to X\times Y\) is continuous iff
  \(\pi_X\circ g\) and \(\pi_Y\circ g\) are both continuous.
\end{theorem}

\begin{proof}
  \begin{description}
  \item[]
  \item[\(\implies\)] Assume that \(g:Z\to X\times Y\) is continuous.

    Since \(\pi_X\) and \(\pi_Y\) are continuous, and since the composition of continuous functions is continuous,
    \(\pi_X\circ g\) and \(\pi_Y\circ g\) are both continuous.

  \item[\(\impliedby\)] Assume that \(\pi_X\circ g\) and \(\pi_Y\circ g\) are both continuous.

    Assume that \(W\in\T_{X\times Y}\).  So \(W=\bigcup_{\a\in\l}U_{\a}\times V_{\a}\) where \(U_{\a}\in\T_X\) and
    \(V_{\a}\in\T_Y\).  Then:
    \begin{align*}
      g^{-1}(W) &= g^{-1}(\bigcup_{\a\in\l}U_{\a}\times V_{\a}) \\
      &= g^{-1}(\bigcup_{\a\in\l}((U_{\a}\times Y)\cap(X\times V_{\a}))) \\
      &= g^{-1}(\pi_X^{-1}(\bigcup_{\a\in\l}U_{\a})\cap\pi_Y^{-1}(\bigcup_{\a\in\l}V_{\a})) \\
      &= g^{-1}(\pi_X^{-1}(\bigcup_{\a\in\l}U_{\a}))\cap g^{-1}(\pi_Y^{-1}(\bigcup_{\a\in\l}V_{\a})) \\
      &= (\pi_X^{-1}\circ g^{-1})(\bigcup_{\a\in\l}U_{\a})\cap (\pi_Y\circ g^{-1})(\bigcup_{\a\in\l}V_{\a})
    \end{align*}
    Now, since \(\pi_X^{-1}\circ g^{-1}\) is continuous and \(\bigcup_{\a\in\l}U_{\a}\in\T_X\),
    \((\pi_X^{-1}\circ g^{-1})(\bigcup_{\a\in\l}U_{\a})\in\T_X\).  Similarly,
    \((\pi_Y^{-1}\circ g^{-1})(\bigcup_{\a\in\l}V_{\a})\in\T_Y\).  Thus, \(g^{-1}(W)\in\T_Z\).

    Therefore \(g:Z\to X\times Y\) is continuous.
  \end{description}
\end{proof}

The previous theorem generalizes to arbitrary products.  In fact, \(X=\prod_{\a\in\l}X_{\a}\) is the smallest
topology that makes each \(\pi_{X_{\a}}\) continuous.

\begin{example}[The Cantor Set]
  \begin{gather*}
    C_0=[0,1] \\
    C_1=[0,\frac{1}{3}]\cup[\frac{2}{3},1] \\
    C_2=[0,\frac{1}{9}]\cup[\frac{2}{9},\frac{3}{9}]\cup[\frac{6}{9},\frac{7}{9}]\cup[\frac{8}{9},1] \\
    \qquad\vdots \\
    C=\bigcap_{n=0}^{\infty}C_n
  \end{gather*}
  The Cantor set is:
  \begin{itemize}
  \item Perfect (closed with no isolated points)
  \item Totally disconnected (no open intervals)
  \item Measure zero
  \item Uncountable
  \item Homeomorphic to \(\set{0,1}^N\) with the discrete topology
  \end{itemize}

  The last condition indicates \(C\) is isomorphic to trinary digit strings \(0.a_1a_2a_3\ldots\) such that
  \(a_k\ne2\).
\end{example}

\end{document}
