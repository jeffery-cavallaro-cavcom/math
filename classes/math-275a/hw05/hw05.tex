\documentclass[letterpaper,12pt,fleqn]{article}
\usepackage{matharticle}
\usepackage{tikz}
\pagestyle{plain}
\newcommand{\p}{\pi}
\newcommand{\e}{\epsilon}
\renewcommand{\d}{\delta}
\newcommand{\B}{\mathcal{B}}
\newcommand{\T}{\mathscr{T}}
\newcommand{\U}{\mathcal{U}}
\begin{document}
Cavallaro, Jeffery \\
Math 275A \\
Homework \#5

\bigskip

\begin{theorem}[Exercise 3.34]
  Let \(X,Y\) be topological spaces.  If \(A\subset X\) and \(B\subset Y\) are closed sets then \(A\times B\)
  is closed in \(X\times Y\).
\end{theorem}

\begin{proof}
  Since \(A\) and \(B\) are closed, \(X-A\) and \(Y-B\) are open.  And so:
  \[(X-A)\times(X-B)=(X\times Y)-(A\times B)\]
  is open.  Therefore \(A\times B\) is closed.
\end{proof}

\begin{theorem}[Exercise 3.35]
  Let \(X\) and \(Y\) be topological spaces.  The product topology on \(X\times Y\) is the same as the topology
  generated by the subbasis of inverse images of open sets under the projection functions, that is, the basis is
  given by:
  \[\B=\setb{\p_X^{-1}(U)}{U\in\T_X}\cup\setb{\p_Y^{-1}(V)}{V\in\T_Y}\]
\end{theorem}

\begin{proof}
  Assume \(U\in\T_X\) and \(V\in\T_y\):
  \begin{gather*}
    \p_X^{-1}(U)=\setb{(x,y)}{x\in U,y\in Y}=U\times Y \\
    \p_Y^{-1}(V)=\setb{(x,y)}{x\in X,y\in V}=X\times V \\
    \\
    \p_X^{-1}(U)\cap\p_Y^{-1}(V)=(U\times Y)\cap(X\times V)=(U\cap X,V\cap Y)=(U,V)
  \end{gather*}
\end{proof}

\begin{theorem}[4.1]
  Let \(X\) be a topological space.  \(X\) is \(T_1\) iff every point in \(X\) is a closed set.
\end{theorem}

\begin{proof}
  Assume \(x,y\in X\) such that \(x\ne y\).

  \begin{description}
  \item[\(\implies\)] Assume \(X\) is \(T_1\).

    So there exists \(U\in\T\) such that \(x\notin U\) and \(y\in U\).  This means that \(U\cap\set{x}=\emptyset\)
    and so \(y\) is not a limit point of \(\set{x}\).

    Therefore, \(\set{x}\) is closed.

  \item[\(\impliedby\)] Assume that every point in \(X\) is a closed set.

    So \(x\) is not a limit point of \(\set{y}\) and \(y\) is not a limit point of \(\set{x}\).  This means that
    there exists \(U,V\in\T\) such that \(x\in U\) and \(U\cap\set{y}=\emptyset\) and likewise \(y\in V\) and
    \(V\cap\set{x}=\emptyset\).  Hence \(x\in U\) but \(y\notin U\) and \(y\in V\) but \(x\notin V\).

    Therefore \(X\) is \(T_1\).
  \end{description}
\end{proof}

\begin{theorem}[Exercise 4.2]
  Let \(X\) be a topological space.  If \(X\) is cofinite then \(X\) is \(T_1\).
\end{theorem}

\begin{proof}
  Assume that \(X\) is cofinite and assume that \(x\in X\).  But \(X-\set{x}\) is open in the cofinite topology, and
  so \(\set{x}\) is closed.  Therefore, by the previous theorem, \(X\) is \(T_1\).
\end{proof}

\begin{example}[Exercise 4.6]
  Consider \(\R^2\) with the standard topology.

  \begin{enumerate}
  \item Let \(p\in\R^2\) and let \(A\subset\R^2\) be a closed set such that \(p\notin A\).  Show that:
    \[\inf\setb{d(a,p)}{a\in A}>0\]

    Since \(A\) is closed and \(p\notin A\), \(p\) is not a limit point of \(A\).  Thus, there exists \(\e>0\) such
    that \(B(p,\e)\cap A=\emptyset\) and so for all \(a\in A\) the distance from \(p\) to \(a\) is at least \(\e\).

    Therefore, \(\inf\setb{d(a,p)}{a\in A}>\e>0\).

  \item Show that \(\R^2\) with the standard topology is regular.

    Assume that \(p\in\R^2\) and \(A\subset\R^2\) such that \(p\notin A\) and \(A\) is closed.  By (1), there
    exists some \(\e>0\) such that for all \(a\in A\), \(d(p,a)>\e\).  Let \(\d=\frac{\e}{3}\) and consider
    \(U=B(p,\d)\) and open set \(V\) generated by \(\setb{B(a,\d_a)}{a\in A,\d_a<\d}\).  Thus, for every point
    \(x\in U\) and \(y\in V\), \(d(x,y)\ge\d\) and so \(U\cap V=\emptyset\).

    Therefore \(R^2\) is regular.

  \item Find two disjoint closed sets \(A,B\subset\R^2\) with the standard topology such that:
    \[\inf\setb{d(a,b)}{a\in A,b\in B}=0\]

    Any two asymptotic functions in \(R^2\) will do.  So let:
    \begin{align*}
      A &= \setb{(x,0)}{x\in[1,\infty)} \\
      B &= \setb{\left(x,\frac{1}{x}\right)}{x\in[1,\infty)}
    \end{align*}

  \item Show that \(\R^2\) with the standard topology is regular.

    Assume that \(A,B\in\R^2\) such that \(A\) and \(B\) are closed and \(A\cap B=\emptyset\).  By (2), for every
    \(a\in A\) there exists \(B(a,\e_a)\) such that \(B(a,\e_a)\cap B=\emptyset\).  Likewise, for every \(b\in B\)
    there exists \(B(b,\e_b)\) such that \(B(b,\e_b)\cap A=\emptyset\).  So let \(\d_a=\frac{\e_a}{3}\) and let
    \(\d_b=\frac{\e_b}{3}\) and consider the families of open sets \(U_a=B(a,\d_a)\) and \(V_b=B(b,\d_b)\).  Let:
    \begin{align*}
      U=\bigcup_{a\in A}U_a\supset A \\
      V=\bigcup_{b\in B}V_b\supset B
    \end{align*}
    Now, assume that \(a\in A\) and \(b\in B\):
    \[d(a,b)\ge\max\set{\e_a,\e_b}>\max\set{\d_a,\d_b}\]
    Thus \(U_a\cap V_b=\emptyset\) and hence \(U\cap V=\emptyset\).

    Therefore \(R^2\) is normal.
  \end{enumerate}
\end{example}

\begin{theorem}[4.7]
  \begin{enumerate}
  \item[]
  \item A \(T_2\)-space (Hausdorff) is a \(T_1\)-space.
  \item A \(T_3\)-space (regular and \(T_1\)) is a \(T_2\)-space (Hausdorff).
  \item A \(T_4\)-space (normal and \(T_1\)) is a \(T_3\)-space (regular and \(T_1\)).
  \end{enumerate}
\end{theorem}

\begin{proof}
  Let \(X\) be a topological space.
  \begin{enumerate}
  \item Assume that \(X\) is \(T_2\).

    Assume \(x,y\in X\) such that \(x\ne y\).  Since \(X\) is \(T_2\), there exists \(U,V\in\T\) such that
    \(x\in U\), \(y\in V\), and \(U\cap V=\emptyset\).  Thus, \(x\in U\), \(y\notin U\), \(x\notin V\), and
    \(y\in V\).

    Therefore \(X\) is \(T_1\).

  \item Assume that \(X\) is \(T_3\).

    Assume \(x,y\in X\) such that \(x\ne y\).  Since \(X\) is \(T_1\), \(\set{y}\) is closed, and since \(X\) is
    \(T_3\), there exists \(U,V\in\T\) such that \(x\in U\), \(\set{y}\subset V\) (\(y\in V\)), and
    \(U\cap V=\emptyset\).

    Therefore \(X\) is \(T_2\).

  \item Assume that \(X\) is \(T_4\).

    Assume \(x\in X\) and \(A\subset X\) such that \(A\) is closed and \(x\notin A\).  Assume \(y\in A\).  Since
    \(X\) is \(T_1\), \(\set{x}\) and \(\set{y}\) are closed, and since \(X\) is normal, there exists
    \(U_x,V_y\in\T\) such that \(\set{x}\subset U_x\) and \(\set{y}\subset V_y\) and \(U_x\cap V_y=\emptyset\).  So
    use the \(V_y\) to generate a set \(V_A\in\T\):
    \[V_A=\bigcup_{y\in A}V_y\supset A\]
    Since \(U_x\cap V_y=\emptyset\), it must be the case that \(U_x\cap V_A=\emptyset\).  Hence, \(x\in U_x\),
    \(A\subset V_A\) and closed, and \(U_x\cap V_A=\emptyset\).

    Therefore \(X\) is regular and \(T_1\) and hence \(T_3\).
  \end{enumerate}
\end{proof}

\begin{theorem}[4.8]
  Let \(X\) be a topological space.  \(X\) is regular iff for all \(p\in X\) and \(U\in\U_p\), there exists
  \(V\in\U_p\) such that \(\bar{V}\subset U\).
\end{theorem}

\begin{proof}
  \begin{description}
  \item[]
  \item[\(\implies\)] Assume that \(X\) is regular.

    Assume \(p\in X\) and assume \(U\in\U_p\).  Since \(U\) is open, \(X-U\) is closed.  So, since \(X\) is
    regular, there exists \(V,W\in\T\) such that \(p\in V\), \(X-U\subset W\), and \(V\cap W=\emptyset\).
    Now, since \(X-U\subset W\):
    \[X-(X-U)\supset X-W\]
    and so \(X-W\subset U\).  Next, since \(V\cap W=\emptyset\), it must be the case that \(V\subset X-W\).  But
    since \(W\) is open, \(X-W\) is closed.  Therefore:
    \[\bar{V}\subset\overline{X-W}=X-W\subset U\]

  \item[\(\impliedby\)] Assume that \(\forall\,p\in X,\forall\,U\in\U_p,\exists\,V\in\U_p,\bar{V}\subset U\).

    Assume \(p\in X\) and \(A\subset X\) such that \(A\) is closed and \(p\notin A\).  This means that \(p\) is
    not a limit point of \(A\) and so there exists \(U\in\U_p\) such that \(U\cap A=\emptyset\).  Furthermore,
    there exists \(V\in\U_p\) such that \(V\subset\bar{V}\subset U\), and so \(\bar{V}\cap A=\emptyset\).  This
    means that \(A\subset X-\bar{V}\), with \(X-\bar{V}\) open.  But \(V\cap X-\bar{V}=\emptyset\).

    Therefore \(X\) is regular.
  \end{description}
\end{proof}

\begin{theorem}[4.9]
  Let \(X\) be a topological space.  \(X\) is normal iff for all closed sets \(A\subset X\) and for all
  \(U\in\U_A\) there exists \(V\in\U_A\) such that \(\bar{V}\subset U\).
\end{theorem}

\begin{proof}
  \begin{description}
  \item[]
  \item[\(\implies\)] Assume that \(X\) is normal.

    Assume \(A\subset X\) and assume \(U\in\U_A\).  Since \(U\) is open, \(X-U\) is closed.  So, since \(X\) is
    normal, there exists \(V,W\in\T\) such that \(A\subset V\), \(X-U\subset W\), and \(V\cap W=\emptyset\).
    Now, since \(X-U\subset W\):
    \[X-(X-U)\supset X-W\]
    and so \(X-W\subset U\).  Next, since \(V\cap W=\emptyset\), it must be the case that \(V\subset X-W\).  But
    since \(W\) is open, \(X-W\) is closed.  Therefore:
    \[\bar{V}\subset\overline{X-W}=X-W\subset U\]

  \item[\(\impliedby\)] Assume that for all closed sets \(A\subset X\) and for all \(U\in\U_A\)
    there exists \(V\in\U_A\) such that \(\bar{V}\subset U\).

    Assume \(A,B\subset X\) such that \(A\) and \(B\) are closed and \(A\cap B=\emptyset\).  This means that for all
    \(p\in A\), \(p\) is not a limit point of \(B\) and so there exists \(U_p\in\T\) such that \(p\in U_p\) and
    \(U_p\cap B=\emptyset\).  Let \(U\supset A\) be the open set generated by these \(U_p\):
    \[U=\bigcup_{p\in A}U_p\supset A\]
    Now, since \(U_p\cap B=\emptyset\) for all \(p\in A\), it must be the case that \(U\cap B=\emptyset\).
    Furthermore, there exists \(V\in\U_A\) such that \(V\subset\bar{V}\subset U\), and so \(\bar{V}\cap
    B=\emptyset\).  This means that \(B\subset X-\bar{V}\), with \(X-\bar{V}\) open.  But \(V\cap
    X-\bar{V}=\emptyset\).

    Therefore \(X\) is normal.
  \end{description}
\end{proof}

\end{document}
