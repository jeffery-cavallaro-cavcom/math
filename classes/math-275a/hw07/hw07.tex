\documentclass[letterpaper,12pt,fleqn]{article}
\usepackage{matharticle}
\usepackage{array}
\usepackage{tikz}
\pagestyle{plain}
\newcommand{\T}{\mathscr{T}}
\newcommand{\B}{\mathcal{B}}
\newcommand{\U}{\mathcal{U}}
\renewcommand{\sc}{\(2^{nd}\) countable}
\newcommand{\Rs}{\R_{\text{std}}}
\newcommand{\RL}{\R_{LL}}
\newcommand{\RZ}{\R_{+00}}
\newcommand{\e}{\epsilon}
\renewcommand{\d}{\delta}
\begin{document}
Cavallaro, Jeffery \\
Math 275A \\
Homework \#7

\bigskip

\begin{theorem}[5.1]
  Let \(X\) be a topological space and let \(A\subset X\).  \(A\) is dense in \(X\) iff for all \(U\in\T\),
  \(U\ne\emptyset\implies U\cap A\ne\emptyset\).
\end{theorem}

\begin{proof}
  \begin{description}
  \item[]
  \item[\(\implies\)] Assume that \(A\) is dense in \(X\), and hence \(\bar{A}=X\).

    Assume that \(U\in\T\) and assume that \(U\ne\emptyset\).  Since \(\bar{A}=X\) it must be the case that
    \(U\cap\bar{A}\ne\emptyset\).  So assume that \(x\in U\cap\bar{A}\), meaning that \(x\in U\) and
    \(x\in\bar{A}\).  Therefore, since \(x\in\bar{A}\), it must be the case that \(U\cap A\ne\emptyset\).

  \item[\(\impliedby\)] Assume that \(\forall\,U\in\T,U\ne\emptyset\implies U\cap A\ne\emptyset\).

    Clearly, \(\bar{A}\subset X\).  So assume that \(x\in X\).  But by the assumption, \(x\in\bar{A}\).
    Therefore \(\bar{A}=X\) and hence \(A\) is dense in \(X\).
  \end{description}
\end{proof}

\begin{example}[Exercise 5.2]
  Show that \(\Rs\) is separable.  Which of the previously investigated topologies on \(\R\) are not separable?

  Consider \(\Q\subset\R\) and assume that \(x\in\R-\Q\).  But \(x\) is the limit of some sequence in \(\Q\) and
  hence \(x\in\bar{\Q}\).  This means that \(\bar{\Q}=\R\) and thus \(\Q\) is a countable dense subset of \(\R\).
  Therefore \(\Rs\) is separable.

  For \(\RL\), also consider \(\Q\subset\R\).  Assume \(U\in\T\).  This means that there exists some \(a,b\in\R\)
  such that \([a,b)\subset U\).  If \(a\in\Q\) then done, so assume \(a\in\R-\Q\).  Since \(\Q\) is dense in
  \(\Rs\), there exists \(x\in\Q\) such that \(x\in(a,b)\subset[a,b)\subset U\).  Therefore \(\Q\) is dense in
  \(\RL\) as well and so \(\RL\) is separable.

  For \(\RZ\), consider \(A=\set{0',0''}\cup\Q^+\).  Assume \(U\in\T\).  If \(0'\in U\) or \(0''\in U\) then done,
  so assume that neither is in \(U\).  This means that there exists some \(a,b\in\R^+\) such that \((a,b)\subset U\).
  Since \(\Q\) is dense in \(\Rs\), there exists \(x\in\Q+\) such that \(x\in(a,b)\).  Therefore \(A\) is dense
  and countable in \(\RZ\) and so \(\RZ\) is separable.
\end{example}

\begin{lemma}
  \((A\cap X)\times(B\cap Y)=(A\times B)\cap(X\times Y)\)
\end{lemma}

\begin{proof}
  \begin{align*}
    (a,b)\in(A\cap X)\times(B\cap Y) &\iff a\in A\cap X\ \text{and}\ b\in B\cap Y \\
    &\iff a\in X\ \text{and}\ a\in X\ \text{and}\ b\in B\ \text{and}\ b\in Y \\
    &\iff (a,b)\in A\times B\ \text{and}\ (a,b)\in X\times Y \\
    &\iff (a,b)\in (A\times B)\cap(X\times Y)
  \end{align*}
\end{proof}

\begin{theorem}[5.5]
  Let \(X\) and \(Y\) be topological spaces.  If \(X\) and \(Y\) are separable then \(X\times Y\) is separable.
\end{theorem}

\begin{proof}
  Assume that \(X\) and \(Y\) are separable.  This means that there exists a countable dense \(A\subset X\) and a
  countable dense \(B\subset Y\).

  Claim: \(A\times B\) is countable and dense in \(X\times Y\).

  Since \(A\) and \(B\) are countable, \(A\times B\) is countable.

  Now, assume \(W\in\T_{X\times Y}\).  This means that there exists \(U\in\T_X\) and \(V\in\T_Y\) such that
  \(U\times V\subset W\).  But \(A\) is dense in \(X\) and so \(U\cap A\ne\emptyset\).  Likewise, \(B\) is dense in
  \(Y\) and so \(V\cap B\ne\emptyset\).  And so:
  \[(U\cap A)\times(V\cap B)=(U\times V)\cap(A\times B)\ne\emptyset\]
  Thus, \(W\cap (A\times B)\ne\emptyset\) and so \(A\times B\) is dense in \(X\times Y\).

  Therefore \(A\times B\) is countable and dense in \(X\times Y\) and hence \(X\times Y\) is separable.
\end{proof}

\begin{theorem}[5.9]
  Let \(X\) be a topological space.  If \(X\) is \sc\ then \(X\) is separable.
\end{theorem}

\begin{proof}
  Assume that \(X\) is \sc\ and let \(\B=\set{B_i:i\in\N}\) be a countable basis for \(X\).  From each \(B_i\),
  select a value \(x_i\) and construct the set \(A=\set{x_1,x_2,\ldots}\).  Thus \(x_i\mapsto B_i\) is one-to-one
  and so \(A\) is countable.  Now assume that \(U\in\T\).  Then there exists at least some \(B_i\subset U\) and
  hence \(U\cap A\ne\emptyset\), and so \(A\) is countable and dense in \(X\).

  Therefore \(X\) is separable.
\end{proof}

\begin{example}[Exercise 5.10]
  \begin{enumerate}
    \item[]
    \item Show that \(\Rs\) is \sc\ (and hence separable).

      Consider the countable set \(\B=\setb{(a,b)}{a,b\in\Q}\).  Since \(\Q\) is countable, \(\Q\times\Q\) is
      countable and hence \(\B\) is countable.  Now assume that \(U\in\T\) and assume \(x\in U\).  Since \(x\) is
      an interior point of \(U\), there exists some \(\e>0\) such that \(x\in(x-\e,x+\e)\subset U\).  So there must
      exist \(\d\in\Q\) such that \(0<\d<\e\), and so \(x\in(x-\d,x+\d)\subset(x-\e,x+\e)\subset U\).  But
      \((x-\d,d+\d)\in\B\), and so \(\B\) is a countable basis for \(\Rs\).

      Therefore \(\Rs\) is \sc.

    \item Show that \(\RL\) is separable but not \sc.

      It was already shown that \(\RL\) is separable.  So assume that \(\B\) is a basis for \(\RL\) and consider
      \(\displaystyle U_a=[a,\infty)=\bigcup_{b>a}[a,b)\in\T\).  Then there exists some \(B_a\in\B\) such that
      \(a\in B_a\).  Now, assume \(x,y\in\R\) such that \(x<y\).  Since \(U_y\subsetneq U_x\), there exists
      \(B_x\subset U_x\) and \(B_y\subset U_y\) such that \(B_x\ne B_y\).  Thus, \(x\mapsto B_x\) is injective
      and hence \(\B\) is uncountable.

      Therefore \(\RL\) is not \sc.
  \end{enumerate}
\end{example}

\begin{theorem}[5.11]
  Every uncountable set in a \sc\ space has a limit point.
\end{theorem}

\begin{proof}
  Assume that \(X\) is a \sc\ space and assume that \(A\subset X\) such that \(A\) is uncountable.  Now, ABC that
  \(A\) has no limit points.  This means that for all \(a\in A\) it is the case that there exists \(U\in\U_a\)
  such that \(U_a\cap A=\set{a}\) and hence every \(a\in A\) is an isolated point.  So assume that \(x,y\in A\)
  such that \(x\ne y\).  There exists \(U\in\U_x\) and \(V\in\U_y\) such that \(U\ne V\).  So for any basis
  \(\B\) of \(X\), there exists \(B_x\subset U\) and \(B_y\subset V\) such that \(B_x\ne B_y\).  Thus,
  \(a\mapsto B_a\) is injective and hence \(\B\) is uncountable, contradicting the assumption that \(X\)
  is \sc.

  Therefore \(A\) contains a limit point.
\end{proof}

\begin{theorem}[5.13]
  If \(X\) and \(Y\) are \sc\ spaces then \(X\times Y\) is \sc.
\end{theorem}

\begin{proof}
  Assume that \(\B_X\) is a countable basis for \(X\) and \(B_y\) is a countable basis for \(Y\).

  Claim: \(\B_X\times\B_Y\) is a countable basis for \(X\times Y\).

  \(\B_X\times\B_Y\) is countable.  So assume that \(U\in\T_{X\times Y}\) and assume \((a,b)\in U\).  This means that
  there exists \(U_a\in\T_X\) and \(V_b\in\T_Y\) such that \((a,b)\in U_a\times V_b\subset U\).  Furthermore, there
  exists \(B_a\in\B_X\) and \(B_b\in\B_Y\) such that \((a,b)\in B_a\times B_b\subset U_a\times V_b\subset U\) and
  so \(\B_X\times\B_Y\) is a countable basis for \(X\times Y\).

  Therefore \(X\times Y\) is \sc.
\end{proof}

\begin{theorem}[5.21]
  \(\Rs\) is Souslin.
\end{theorem}

\begin{proof}
  ABC that \(\Rs\) is not Souslin, meaning it does contain an uncountable collection of disjoint open sets.  Let
  \(\U\) be such a set.  Since \(\Q\) is countable and dense in \(\Rs\), every \(U\in\U\) contains some
  \(r_U\in\Q\).  So select one value from each \(U\in\U\) to construct the set \(\setb{r_U\in Q}{U\in\U}\).
  Thus \(r_U\mapsto U\) is injective and hence \(\U\) is countable, contradicting the assumption.

  Therefore \(\Rs\) is Souslin.
\end{proof}

\end{document}
