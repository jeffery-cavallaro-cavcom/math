\documentclass[letterpaper,12pt,fleqn]{article}
\usepackage{matharticle}
\newcommand{\norm}[1]{\left\lVert{#1}\right\rVert}
\newcommand{\B}{\mathcal{B}}
\renewcommand{\C}{\mathcal{C}}
\newcommand{\T}{\mathscr{T}}
\renewcommand{\a}{\alpha}
\renewcommand{\b}{\beta}
\newcommand{\e}{\epsilon}
\renewcommand{\d}{\delta}
\renewcommand{\l}{\lambda}
\DeclareMathOperator{\dist}{dist}
\begin{document}
Cavallaro, Jeffery \\
Math 275A \\
Homework \#14

\bigskip

\begin{theorem}[8.18]
  Let \(X\) be a topological space.  Each component of \(X\) is connected, closed, and not contained in any
  strictly larger connected subset of \(X\).
\end{theorem}

\begin{proof}
  Assume that \(U\) is a component of \(X\) and that \(p\in U\).  By definition, \(U=\bigcup_{\a\in\l}U_{\a}\)
  where \(U_{\a}\) is a connected subset of \(X\) containing \(p\).  Now, for each \(U_{\a}\),
  \(U_{\a}\cap\set{p}\ne\emptyset\) and \(\set{p}\) is trivially connected.  Therefore \(U\) is connected.

  Assume that \(V\) is a connected component of \(X\) such that \(U\subset V\).  This means that \(p\in V\) and
  so, by definition, \(U=U\cup V\).  But this is ony true if \(V\subset U\).  Therefore \(U=V\).

  Now, since \(U\) is connected, \(\bar{U}\) is connected.  But \(p\in\bar{U}\) and so, by definition,
  \(U=U\cup\bar{U}\).  But this is only true if \(\bar{U}\subset U\).  Therefore \(U=\bar{U}\) and hence \(U\) is
  closed.
\end{proof}

\begin{theorem}[8.35]
  A path connected topological space is connected.
\end{theorem}

\begin{proof}
  Assume that \(X\) is a path connected topological space and ABC that \(X\) is disconnected.  This means that
  there exists \(A,B\subset X\) such that \(A\sqcup B=X\) where \(A,B\) are open and non-empty.  So assume that
  \(x\in A\) and \(y\in B\).  Since \(X\) is path connected, there exists some continuous \(f:[0,1]\to X\) such
  that \(f(0)=x\in A\) and \(f(1)=y\in B\).  This mean that \([0,1]=f^{-1}(A)\cup f^{-1}(B)\) where neither
  \(f^{-1}(A)\) nor \(f^{-1}(B)\) are empty.  Furthermore, since \(A\) and \(B\) are disjoint, \(f^{-1}(A)\) and
  \(f^{-1}(B)\) must also be disjoint, contradicting the connectedness of \([0,1]\).  Therefore \(X\) is
  connected.
\end{proof}

\begin{example}[Exercise 8.37]
  The closure of the topologist's sine curve is connected but not path connected.

  The topologists sine curve is given by:
  \[S=\setb{\left(x,\sin\frac{1}{x}\right)}{x\in(0,1)}\]
  and its closure is given by:
  \[\bar{S}=S\cup\set{(1,\sin(1)}\cup\setb{(0,y)}{y\in[-1,1]}\]
  Note that \(\bar{S}\) was already shown to be connected.

  ABC that \(\bar{S}\) is path connected and assume that \(p\in S\).  This means that there exists a path in
  \(\bar{S}\) such that \(f(0)=p\) and \(f(1)=(0,0)\).  Let \(f(t)=(x(t),y(t))\).  Note that since \(f\) is
  continuous, \(x(t)\) and \(y(t)\) are also continuous.  Now, defined \(U=\setb{t\in[0,1]}{x(t)>0}\).  Thus,
  for all \(t\in U\), \(f(t)\in S\) and \(y(t)=\frac{1}{x(t)}\).

  Next, since \(U\subset[0,1]\), \(U\) is bounded and thus has a sup.  So let \(t_*=\sup U\).  Note that \(t_*\)
  is the final value of \(t\) at which the path jumps to the y-axis part of \(\bar{S}\) and stays there on the
  way to \((0,0)\).  So \(x(t_*)=0\).  Let \(b=y(t_*)\) and let select \(\e>0\) such that:
  \[\e<\begin{cases}
  1-b, & b<1 \\
  \frac{1}{2}, & b=1
  \end{cases}\]
  Now, since \(f\) is continuous, there exists \(\d>0\) such that for all \(t\in[0,1]\), if \(\abs{t-t_*}<\d\)
  then \(\norm{f(t)-f(t_*)}<\e\).  Note that \([t_*-\d,t_*]\) is connected and compact.  Furthermore, \(f\) is
  continuous.  Hence \(f[t_*-\d,t_*]\) is connected and compact, and thus must be an interval.  So let
  \(x([t_*-\d,t_*)=[0,x_0]\) for some \(x_0\in(0,1]\).  This means that for every \(x\in(0,x_0]\) there exists
  some \(t\in[t_*-\d,t_*]\) such that \(f(t)\in S\), meaning \(f(t)=(x,\sin\frac{1}{x})\).

  Define a sequence \(x_n\) in [0,1] by:
  \[x_n=\frac{1}{2n\pi+\frac{\pi}{2}}\]
  Note that \(x_n\to0\) and:
  \[\sin\frac{1}{x_n}=\sin\left(2n\pi+\frac{\pi}{2}\right)=\sin\frac{\pi}{2}=1\]
  But since \(x_n\to 0\), there exists \(N\in\N\) such that for all \(x_n<x_0\) for all \(n>N\).  And so there
  exists \(t_n\in[t_*-\d,t_*)\) such that:
  \[f(t_n)=\left(x_n,\sin\frac{1}{x_n}\right)=(x_n,1)\]
  Thus:
  \[\norm{f(t_n)-f(t_*)}=\norm{(x_n,1)-(0,b)}\ge1-b>\e\]
  This contradicts the continuity of \(f\).  Therefore \(\bar{S}\) is not path connected.
\end{example}

\begin{theorem}[8.38]
  Let \(X\) and \(Y\) be topological spaces.  If \(X\) and \(Y\) are path connected then \(X\times Y\) is path
  connected.
\end{theorem}

\begin{proof}
  Assume that \(X\) and \(Y\) are path connected and assume that \((x_1,y_1),(x_2,y_2)\in X\times Y\).  This means
  that there must exist a path \(f\) from \(x_1\) to \(x_2\) and a path \(g\) from \(y_1\) to \(y_2\).  Now,
  defined \(h:[0,1]\to X\times Y\) as \(h(t)=(f(t),g(t))\).  But \(\pi_X\circ h=f\) and \(\pi_Y\circ h=g\) are
  by definition continuous, and thus \(h\) is continuous.  Furthermore, \(h(0)=(f(0),g(0))=(x_1,y_1)\) and
  \(h(1)=(f(1),g(1))=(x_2,y_2)\), and so \(h\) is a path between \((x_1,y_1)\) and \((x_2,y_2)\).  Therefore
  \(X\times Y\) is path connected.
\end{proof}

\begin{examples}[Exercise 9.1]
  Show that the following are all metrics on \(\R^n\):
  \begin{enumerate}
  \item The \emph{Euclidean metric} defined by:
    \[d(x,y)=\norm{x-y}=\sqrt{\sum_{k=1}^n(x_k-y_k)^2}\]

    \begin{description}
    \item[Positive Definition:]
      \begin{gather*}
        (x_k-y_k)^2\ge0 \\
        \sum(x_k-y_k)^2\ge0 \\
        \sqrt{\sum(x_k-y_k)^2}\ge0 \\
        d(x,y)\ge0
      \end{gather*}
      \begin{align*}
        d(x,y)=0 &\iff \sqrt{\sum(x_k-y_k)^2}=0 \\
        &\iff \sum(x_k-y_k)^2=0 \\
        &\iff (x_k-y_k)^2=0 \\
        &\iff x_k-y_k=0 \\
        &\iff x_k=y_k \\
        &\iff x=y
      \end{align*}
    \item[Symmetric:]
      \[d(x,y)=\sqrt{\sum(x_k-y_k)^2}=\sqrt{\sum(y_k-x_k)^2}=d(y,x)\]
    \item[Triangle Inequality:]
      \begin{align*}
        [d(x,y)]^2 &= \sum(x_k-y_k)^2 \\
        &= \sum\abs{(x_k-z_k)+(z_k-y_k)}^2 \\
        &\le \sum(\abs{x_k-z_k}+\abs{z_k-y_k})^2 \\
        &= \sum(\abs{x_k-z_k}^2+\abs{z_k-y_k}^2+2\abs{x_k-z_k}\abs{z_k-y_k}) \\
        &= \sum\abs{x_k-z_k}^2+\sum\abs{z_k-y_k}^2+2\sum\abs{x_k-z_k}\abs{z_k-y_k}
      \end{align*}
      Now, by the Cauchy-Schwarz inequality:
      \begin{align*}
        \sum\abs{x_k-z_k}\abs{z_k-y_k} &\le \sqrt{\left(\sum(x_k-z_k)^2\right)\left(\sum(z_k-y_k)^2\right)} \\
        &= \sqrt{[d(x,z)]^2[d(z,y)]^2} \\
        &= d(x,z)d(z,y)
      \end{align*}
      and so:
      \[[d(x,y)]^2\le[d(x,z)]^2+[d(z,y)]^2+2d(x,z)d(z,y)=[d(x,z)+d(z,y)]^2\]
      Therefore \(d(x,y)\le d(x,z)+d(z,y)\).
    \end{description}

  \item The \emph{box metric} defined by:
    \[d(x,y)=\max_{1\le k\le n}\set{\abs{x_k-y_k}}\]

    \begin{description}
    \item[Positive Definition:]
      \begin{gather*}
        \abs{x_k-y_k}\ge0 \\
        \max\set{x_k-y_k}\ge0 \\
        d(x,y)\ge0
      \end{gather*}
      \begin{align*}
        d(x,y)=0 &\iff \max\set{\abs{x_k-y_k}}=0 \\
        &\iff \abs{x_k-y_k}=0 \\
        &\iff x_k-y_k=0 \\
        &\iff x_k=y_k \\
        &\iff x=y
      \end{align*}
    \item[Symmetric:]
      \[d(x,y)=\max\set{\abs{x_k-y_k}}=\max\set{\abs{y_k-x_k}}=d(y,x)\]
    \item[Triangle Inequality:]
      \begin{align*}
        d(x,y) &= \max\set{\abs{x_k-y_k}} \\
        &= \max\set{\abs{(x_k-z_k)+(z_k-y_k)}} \\
        &\le \max\set{\abs{x_k-z_k}+\abs{z_k-y_k}} \\
        &\le \max\set{\abs{x_k-z_k}}+\max\set{\abs{z_k-y_k}} \\
        &= d(x,z)+d(z,y)
      \end{align*}
    \end{description}

  \item The \emph{taxi-cab metric} defined by:
    \[d(x,y)=\sum_{k=1}^n\abs{x_k-y_k}\]

    \begin{description}
    \item[Positive Definition:]
      \begin{gather*}
        \abs{x_k-y_k}\ge0 \\
        \sum\abs{x_k-y_k}\ge0 \\
        d(x,y)\ge0
      \end{gather*}
      \begin{align*}
        d(x,y)=0 &\iff \sum\abs{x_k-y_k}=0 \\
        &\iff \abs{x_k-y_y}=0 \\
        &\iff x_k-y_k=0 \\
        &\iff x_k=y_k \\
        &\iff x=y
      \end{align*}
    \item[Symmetric:]
      \[d(x,y)=\sum\abs{x_k-y_k}=\sum\abs{y_k-x_k}=d(y,x)\]
    \item[Triangle Inequality:]
      \begin{align*}
        d(x,y) &= \sum\abs{x_k-y_k} \\
        &= \sum\abs{(x_k-z_k)+(z_k-y_k)} \\
        &\le \sum(\abs{x_k-z_k}+\abs{z_k-y_k}) \\
        &= \sum\abs{x_k-z_k}+\sum\abs{z_k-y_k} \\
        &= d(x,z)+d(z,y)
      \end{align*}
    \end{description}
  \end{enumerate}

  Show that when \(n\ge2\), these metrics are different.

  Consider \((0,0),(3,4)\in\R^2\):
  \begin{gather*}
    d_E=\sqrt{(3-0)^2+(4-0)^2}=5 \\
    d_B=\max\set{(3-0),(4-0)}=4 \\
    d_T=(3-0)+(4-0)=7
  \end{gather*}
\end{examples}

\begin{example}[Exercise 9.2]
  Let \(X\) be a compact topological space and let \(\C(X)\) denote the set of continuous functions \(f:X\to\R\).  We
  can endow \(\C(X)\) with a metric:
  \[d(f,g)=\sup_{x\in X}\set{\abs{f(x)-g(x)}}\]
  This distance is sometimes denoted \(\norm{f-g}\).  Check that \(d\) is a well-defined metric on \(\C(X)\).

  Note that for any \(f\in\C(X)\), since \(X\) is compact and \(f:X\to f(X)\) is surjective, \(f(X)\) is compact
  and thus bounded.  Therefore, all sups are finite.

  \begin{description}
  \item[Positive Definition:]
    \begin{gather*}
      \abs{f(x)-g(x)}\ge0 \\
      \sup\set{\abs{f(x)-g(x)}}\ge0 \\
      d(f,g)\ge0
    \end{gather*}
    \begin{align*}
      d(f,g)=0 &\iff \sup\set{\abs{f(x)-g(x)}}=0 \\
      &\iff \abs{f(x)-g(x)}=0 \\
      &\iff f(x)-g(x)=0 \\
      &\iff f(x)=g(x) \\
      &\iff f=g
    \end{align*}
  \item[Symmetric:]
    \[d(f,g)=\sup\set{\abs{f(x)-g(x)}}=\sup\set{\abs{g(x)-f(x)}}=f(g,f)\]
  \item[Triangle Inequality:]
    \begin{align*}
      d(f,g) &= \sup\set{\abs{f(x)-g(x)}} \\
      &= \sup\set{\abs{(f(x)-h(x))+(h(x)-g(x))}} \\
      &\le \sup\set{\abs{f(x)-h(x)}+\abs{h(x)-g(x)}} \\
      &\le \sup\set{\abs{f(x)-h(x)}}+\sup\set{\abs{h(x)-g(x)}} \\
      &= d(f,h)+d(h,g)
    \end{align*}
  \end{description}
\end{example}

\begin{lemma}
  Let \(X\) be a metric space with metrics \(d_1\) and \(d_2\).  If there exists \(\a,\b>0\) such that for all
  \(x,y\in X\):
  \[\a d_1(x,y)\le d_2(x,y)\le\b d_1(x,y)\]
  then \(d_1\) and \(d_2\) generate the same topology.
\end{lemma}

\begin{proof}
  Let \(B_1\) denote a ball using \(d_1\) and let \(B_2\) denote a ball using \(d_2\).  Assume that \(x\in X\) and
  \(\e>0\).

  First, assume that \(y\in B_2(x,\e)\).  This means that \(d_2(x,y)<\e\), and so \(d_1(x,y)<\frac{\e}{\a}\).
  Hence \(y\in B_1\left(x,\frac{\e}{\a}\right)\), and so \(B_2(x,\e)\subset B_1\left(x,\frac{\e}{\a}\right)\).

  Next, assume that \(y\in B_1(x,\e)\).  This means that \(d_1(x,y)<\e\), and so \(d_2(x,y)<\b\e\).  Hence
  \(y\in B_2(x,\b\e)\), and so \(B_1(x,\e)\subset B_2(x,\b\e)\).

  Now, assume that \(B_1\in\T_1\).  For every \(x\in B_1\) there exists \(B_{2_x}\in\T_2\) such that
  \(B_{2_x}\subset B_1\).  Thus, \(\T_2\) generates \(\T_1\).  Likewise, assume that \(B_{2_x}\in\T_2\).  For every
  \(x\in B_2\) there exists \(B_{1_x}\in\T_1\) such that \(B_{1_x}\in\T_1\).  Thus \(\T_1\) generates \(\T_2\).

  Therefore \(\T_1=\T_2\).
\end{proof}

\begin{example}[Exercise 9.4]
  Show that the Euclidean metric, box metric, and taxicab metric generate the same topology as the product
  topology on \(n\) copies of \(\R\).
  \begin{align*}
    d_E(x,y) &= \sqrt{\sum(x_k-y_k)^2} \\
    &\le \sqrt{\sum\max\set{(x_k-y_k)^2}} \\
    &= \sqrt{n\cdot\max\set{(x_k-y_k)^2}} \\
    &=\sqrt{n}\max\set{\abs{x_k-y_k}} \\
    &= \sqrt{n}\cdot d_B(x,y) \\
  \end{align*}
  Also:
  \begin{align*}
    d_E(x,y) &= \sqrt{\sum(x_k-y_k)^2} \\
    &\ge \sqrt{\max\set{(x_k-y_k)^2}} \\
    &= \max\set{\sqrt{(x_k-y_k)^2}} \\
    &= \max\set{\abs{x_k-y_k}} \\
    &= d_B(x,y)
  \end{align*}
  So \(d_B(x,y)\le d_E(x,y)\le\sqrt{n}d_B(x,y)\) and thus \(\T_B=\T_E\).

  Similarly:
  \begin{align*}
    d_T(x,y) &= \sum\abs{x_k-y_k} \\
    &\le \sum\max\set{x_k-y_k} \\
    &= n\cdot\max\set{x_k-y_k} \\
    &= n\cdot d_B(x,y)
  \end{align*}
  Also:
  \begin{align*}
    d_T(x,y) &= \sum\abs{x_k-y_k} \\
    &\ge \max\set{x_k-y_k} \\
    &= d_B(x,y)
  \end{align*}
  So \(\frac{1}{n}d_T(x,y)\le d_B(x,y)\le d_T(x,y)\) and thus \(\T_T=\T_B\).

  Therefore \(\T_E=\T_B=\T_T\).

  Now, consider a basis element \(U=\prod_{k=1}^nU_k\in\R^n\) and assume that \(p\in U\).  Then there exists some
  \(\e>0\) such that \(p\in\prod_{k=1}^n(p-\e,p+\e)\).  But \(B(p,\e)\subset\prod_{k=1}^n(p-\e,p+\e)\) and so
  \(\T_E\) generates \(\T_{\R^n}\).  Similarly, consider a basis element \(B(p,r)\in\R^n\) and assume that
  \(a\in B(p,r)\).  Then there exists some \(\e>0\) such that \(B(a,\e)\in B(p,r)\).  But
  \(\prod_{k=1}^n\left(a-\frac{\e}{2},a+\frac{\e}{2}\right)\subset B(a,\e)\) and so \(\T_{\R^n}\) generates \(T_E\).

  Therefore \(\T_E=\T_B=\T_T=\T_{\R^n}\).
\end{example}

\begin{lemma}
  Let \((X,d)\) be a metric space and let \(p\in X\) and \(A\subset X\) such that \(p\notin A\) and \(A\) is
  closed:
  \[\dist(p,A)=\inf\setb{d(a,p)}{a\in A}>0\]
\end{lemma}

\begin{proof}
  Since \(A\) is closed and \(p\notin A\), \(p\) is not a limit point of \(A\).  Thus, there exists \(\e>0\) such
  that \(B(p,\e)\cap A=\emptyset\) and so for all \(a\in A\) the distance from \(p\) to \(a\) is at least \(\e\).

  Therefore, \(\inf\setb{d(a,p)}{a\in A}>\e>0\).
\end{proof}

\begin{theorem}[9.8]
  A metric space is Hausdorff, regular, and normal.
\end{theorem}

\begin{proof}
  Let \((X,d)\) be a metric space and let \(p\in X\) and \(A\subset X\) such that \(p\notin A\) and \(A\) is
  closed.  Then there exists some \(\e>0\) such that for all \(a\in A\), \(d(p,a)>\e\).  Let \(\d=\frac{\e}{3}\)
  and consider \(U=B(p,\d)\) and open set \(V\) generated by \(\setb{B(a,\d_a)}{a\in A,\d_a<\d}\).  Thus, for every
  point \(x\in U\) and \(y\in V\), \(d(x,y)\ge\d\) and so \(U\cap V=\emptyset\).

  Therefore \((X,d)\) is regular, and hence also Hausdorff.

  Now, assume that \(A,B\subset(X,d)\) such that \(A\) and \(B\) are closed and \(A\cap B=\emptyset\).  Then for
  every \(a\in A\) there exists \(B(a,\e_a)\) such that \(B(a,\e_a)\cap B=\emptyset\).  Likewise, for every \(b\in B\)
  there exists \(B(b,\e_b)\) such that \(B(b,\e_b)\cap A=\emptyset\).  So let \(\d_a=\frac{\e_a}{3}\) and let
  \(\d_b=\frac{\e_b}{3}\) and consider the families of open sets \(U_a=B(a,\d_a)\) and \(V_b=B(b,\d_b)\).  Let:
  \begin{align*}
    U=\bigcup_{a\in A}U_a\supset A \\
    V=\bigcup_{b\in B}V_b\supset B
  \end{align*}
  Now, assume that \(a\in A\) and \(b\in B\):
  \[d(a,b)\ge\max\set{\e_a,\e_b}>\max\set{\d_a,\d_b}\]
  Thus \(U_a\cap V_b=\emptyset\) and hence \(U\cap V=\emptyset\).

  Therefore \((X,d)\) is normal.
\end{proof}

\end{document}
