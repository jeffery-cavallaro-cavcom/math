\documentclass[letterpaper,12pt,fleqn]{article}
\usepackage{matharticle}
\usepackage{tikz}
\usetikzlibrary{positioning}
\pagestyle{plain}
\newcommand{\T}{\mathscr{T}}
\newcommand{\U}{\mathcal{U}}
\newcommand{\Tstd}{\T_{\text{std}}}
\renewcommand{\a}{\alpha}
\renewcommand{\l}{\lambda}
\newcommand{\e}{\epsilon}
\begin{document}
Cavallaro, Jeffery \\
Math 275A \\
Homework \#2

\bigskip

\begin{theorem}[1.20]
  Let \(A\) be a set and let \(P\) be the set of all functions from \(A\) to the two-point set \(\set{0,1}\):
  \[\abs{P}=\abs{2^A}\]
\end{theorem}

\begin{proof}
  Consider the function \(f:P\to2^A\) defined by \(f(p)=B\) such that:
  \[p(a)=\begin{cases}
  0, & a\notin B \\
  1, & a\in B
  \end{cases}\]
  Claim: \(f\) is a bijection.

  Assume \(f(p_1)=f(p_2)=B\).  Assume \(a\in A\).  If \(a\notin B\) then \(p_1(a)=p_2(a)=0\).  If \(a\in B\) then
  \(p_1(a)=p_2(a)=1\).  So \(\forall\,a\in A,p_1(a)=p_2(a)\).  Thus, by definition, \(p_1=p_2\) and therefore
  \(f\) is injective.

  Now, assume \(B\in 2^A\).  Since \(B\subset A\), for each \(a\in A\), \(a\) is either not in \(B\) or in \(B\).
  So define \(p:A\to\set{0,1}\) as above.  Thus \(p\in P\) and \(f(p)=B\).  Therefore \(f\) is surjective.

  Therefore \(f\) is a bijection and thus \(\abs{P}=\abs{2^A}\).
\end{proof}

\begin{theorem}[1.22]
  Let \(A\) be a set:
  \[\abs{A}\ne\abs{2^A}\]
\end{theorem}

\begin{proof}
  Let \(f:A\to2^A\) and ABC that \(f\) is bijective.  For all \(a\in A\) let \(f(a)=B_a\).  This means that either
  \(a\notin B_a\) or \(a\in B_a\).  Now, construct \(B\in2^A\) as follows:
  \[B=\setb{a\in A}{a\notin f(a)}\]
  Note that if \(a\notin B_a\) then \(a\in B\) and if \(a\in B_a\) then \(a\notin B\) and so \(\forall\,a\in
  A,B_a\ne B\).  Thus, \(B\in2^A\) but \(B\notin f(A)\), contradicting the bijectiveness of \(f\).

  \(\therefore\abs{A}\ne\abs{2^A}\)
\end{proof}

\begin{theorem}[2.3]
  Let \(X,\T\) be a topological space and let \(U\subset X\):
  \[U\in\T\iff\forall\,x\in U,\exists\,U_x\in\T,x\in U_x\subset U\]
\end{theorem}

\begin{proof}
  \begin{description}
  \item[]

  \item[\(\implies\)] Assume \(U\in\T\).

    Assume \(x\in U\).  So \(x\in U\subset U\).

  \item[\(\impliedby\)] Assume \(\forall\,x\in U,\exists U_x\in\T,x\in U_x\subset U\).

    Claim: \(\bigcup_{x\in U}U_x=U\)

    First, assume that \(y\in \bigcup_{x\in U}U_x\).  Therefore \(\exists\,x\in U\) such that \(y\in U_x\subset U\).

    Next, assume that \(y\in U\).  This means that \(\exists\,U_y\in\T\) such that \(y\in U_y\subset U\), and
    therefore \(y\in\bigcup_{x\in U}U_x\).

    Thus, \(U\) is an arbitrary union of open sets and hence is open.

    \(\therefore U\in\T\)
  \end{description}
\end{proof}

\begin{example}[Exercise 2.4]
  Verify that \(\Tstd\) is a topology on \(R^n\).

  \begin{enumerate}
  \item \(\emptyset\in\Tstd\) (vacuously).

  \item Assume \(x\in R^n\).

    Let \(e_x=1\).  Since \(R^n\) includes everything it must be the case that \(B(x,1)\subset R^n\).

    Therefore \(R^n\in\Tstd\).

  \item Assume \(U,V\in\Tstd\).

    Assume \(x\in U\cap V\).  This means that \(x\in U\) and \(x\in V\).  So there exists \(\e_U,\e_V\) such that
    \(B(x,\e_U)\subset U\) and \(B(x,\e_V)\subset V\).  Let \(\e=\min\set{\e_U,\e_V}\).  Thus
    \(B(x,\e)\subset U\cap V\).

    Therefore \(U\cap V\in\Tstd\).

  \item Assume \(\set{U_{\a}:\a\in\l}\) is a family of sets such that \(U_{\a}\in\Tstd\).

    Let \(U=\bigcup_{\a\in\l}U_{\a}\) and assume \(x\in U\).  This means that there exists \(\a\in\l\) such that
    \(x\in U_{\a}\).  Furthermore, there exists \(\e_x>0\) such that \(B(x,\e_x)\subset U_{\a}\subset U\).

    Therefore \(U\in\Tstd\).
  \end{enumerate}
\end{example}

\begin{example}[Exercise 2.5]
  Verify that the discrete, indiscrete, cofinite, and cocountable topologies are indeed topologies for any set \(X\).

  \begin{description}
  \item[Discrete]
    \begin{enumerate}
    \item[]
    \item \(\emptyset\subset X\) and so \(\emptyset\in\T\).

    \item \(X\subset X\) and so \(X\in\T\).

    \item Assume \(U,V\in\T\).

      \(U,V\subset X\) and so \(U\cap V\subset X\).

      Therefore \(U\cap V\in\T\).

    \item Assume \(\set{U_{\a}:\a\in\l}\) such that \(U_{\a}\in\T\).

      \(\forall\,\a\in\l,U_{\a}\subset X\) and so \(\bigcup_{\a\in\l}U_{\a}\subset X\).

      Therefore \(\bigcup_{\a\in\l}U_{\a}\in\T\).
    \end{enumerate}

  \item[Indiscrete]
    \begin{enumerate}
    \item[]
    \item By definition, \(\emptyset\in\T\).

    \item By definition, \(X\in\T\).

    \item Assume \(U,V\in\T\).
      \begin{align*}
        \emptyset\cap\emptyset &= \emptyset\in\T \\
        \emptyset\cap X &= \emptyset\in\T \\
        X\cap\emptyset &= \emptyset\in\T \\
        X\cap X &= X\in\T
      \end{align*}

      Therefore \(U\cap V\in\T\).

    \item Assume \(\set{U_{\a}:\a\in\l}\) such that \(U_{\a}\in\T\).
      \begin{align*}
        \emptyset\cup\emptyset &= \emptyset\in\T \\
        \emptyset\cup X &= X\in\T \\
        X\cup\emptyset &= X\in\T \\
        X\cup X &= X\in\T
      \end{align*}

      Therefore \(\bigcup_{\a\in\l}U_{\a}\in\T\).
    \end{enumerate}

  \item[Cofinite]
    \begin{enumerate}
    \item[]
    \item By definition, \(\emptyset\in\T\).

    \item \(X-X=\emptyset\), which is finite.  Therefore \(X\in\T\).

    \item Assume \(U,V\in\T\).

      \(X-(U\cap V)=(X-U)\cup(X-V)\).  But \(X-U\) and \(X-V\) are both finite and so their union is finite.

      Therefore \(U\cap V\in\T\).

    \item Assume \(\set{U_{\a}:\a\in\l}\) such that \(U_{\a}\in\T\).

      \(X-\bigcup_{\a\in\l}U_{\a}=\bigcap_{\a\in\l}(X-U_{\a})\).  But for all \(\a\in\l\) it is the case that
      \(X-U_{\a}\) is finite and so their intersection is finite.

      Therefore \(\bigcup_{\a\in\l}U_{\a}\in\T\).
    \end{enumerate}

  \item[Cocountable]
    \begin{enumerate}
    \item[]
    \item By definition, \(\emptyset\in\T\).

    \item \(X-X=\emptyset\), which is finite and thus countable.  Therefore \(X\in\T\).

    \item Assume \(U,V\in\T\).

      \(X-(U\cap V)=(X-U)\cup(X-V)\).  But \(X-U\) and \(X-V\) are both countable and so their union is countable.

      Therefore \(U\cap V\in\T\).

    \item Assume \(\set{U_{\a}:\a\in\l}\) such that \(U_{\a}\in\T\).

      \(X-\bigcup_{\a\in\l}U_{\a}=\bigcap_{\a\in\l}(X-U_{\a})\).  But for all \(\a\in\l\) it is the case that
      \(X-U_{\a}\) is countable.  Now, for some \(\a\in\l\):
      \[\bigcap_{\a\in\l}(X-U_{\a})\subset(X-U_{\a})\]
      However, the subset of a countable set is countable.

      Therefore \(\bigcup_{\a\in\l}U_{\a}\in\T\).
    \end{enumerate}
  \end{description}
\end{example}

\begin{example}[Exercise 2.6]
  Describe some of the open sets you get if \(\R\) is endowed with the standard, discrete, indiscrete, cofinite,
  and cocountable topologies.  Specifically, identify sets that demonstrate the differences among these topologies,
  that is, find sets that are open in some topologies but not in others.  For each of the topologies, determine
  if the interval \((0,1)\) is an open set in that topology.

  Of course, \(\emptyset\) and \(\R\) are in \(\T\) for all topologies.

  \begin{description}
  \item[Standard:] All open intervals, but not closed intervals, are in \(\T\).  Therefore \((0,1)\in\T\).

  \item[Discrete:] All open and closed intervals are in \(\T\).  Therefore \((0,1)\in\T\) and \([0,1]\in\T\)

  \item[Indiscrete:] Nothing other than \(\emptyset\) and \(\R\).  Therefore \((0,1)\notin\T\).

  \item[Cofinite] All open sets are of the form \(\R-X\) where \(X\) is finite.  For example: \(\R-\set{1,2,3}\).
    Thus, the open sets are uncountable.  Such sets are also open sets in all the other topologies sans indiscrete.
    Therefore \((0,1)\notin\T\).

  \item[Cocountable] All open sets are of the form \(\R-X\) where \(X\) is countable.  Thus, all open sets must
    include all but a countable number of irrational numbers and are thus uncountable.  For example: \(\R-\Q\) or
    \((\R-\Q)\cup\set{\sqrt{2},\sqrt{3}}\) or \((\R-\Q)\cup\set{\sqrt{2},\sqrt{3},\sqrt{5},\ldots}\).  Such sets are
    also open in the standard and discrete topologies.  Since finite sets are countable, it is the case that
    \(\T_{cof}\subset\T_{coc}\).  Therefore \((0,1)\notin\T\).
  \end{description}
\end{example}

\begin{theorem}[2.13]
  Let \((X,\T)\) be a topological space.  For all \(A\subset X\):
  \[\bar{\bar{A}}=\bar{A}\]
\end{theorem}

\begin{proof}
  \(\bar{\bar{\emptyset}}=\bar{\emptyset}=\emptyset\) is vacuously true, so assume \(A\ne\emptyset\).

  \(\bar{A}\subset\bar{\bar{A}}\) by definition, so assume \(p\in\bar{\bar{A}}\).  This means that for all
  \(U\in\U_p,U\cap\bar{A}\ne\emptyset\).  So assume that \(U\in\U_p\) and \(x\in U\cap\bar{A}\), meaning \(x\in U\)
  and \(x\in\bar{A}\).  But this is only true if \(U\cap A\ne\emptyset\) and so \(p\in\bar{A}\).

  Therefore \(\bar{\bar{A}}=\bar{A}\).
\end{proof}

\begin{theorem}[2.14]
  Let \((X,\T)\) be a topological space.  For all \(A\subset X\), \(A\) is closed iff \(X-A\) is open.
\end{theorem}

\begin{proof}
  \(X\) is closed iff \(X-X=\emptyset\) is open is true, so assume that \(A\ne X\).

  \begin{description}
  \item[\(\implies\)] Assume \(A\) is closed.

    Assume \(p\in X-A\).  Since \(p\notin A\), \(p\) is not a limit point of \(A\).  Thus, there exists a neighborhood
    \(U_p\) of \(p\) such \(U_p\cap A=\emptyset\).  But this means that \(U_p\subset X-A\).

    Therefore \(X-A\) is open.

  \item[\(\impliedby\)] Assume \(X-A\) is open.

    Assume \(p\in X-A\).  So there exists a neighborhood \(U_p\) of \(p\) such that \(U_p\subset X-A\).  But this
    means that \(U_p\cap A=\emptyset\) and hence \(p\) is not a limit point of \(A\).  Thus \(A\) contains all of
    its limit points.

    Therefore \(A\) is closed.
  \end{description}
\end{proof}

\end{document}
