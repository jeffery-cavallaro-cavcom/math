\documentclass[letterpaper,12pt,fleqn]{article}
\usepackage{matharticle}
\usepackage{array}
\usepackage{tikz}
\pagestyle{plain}
\newcommand{\T}{\mathscr{T}}
\newcommand{\U}{\mathcal{U}}
\renewcommand{\C}{\checkmark}
\newcommand{\X}{\(\bigtimes\)}
\newcommand{\A}{
  \begin{minipage}{1in}
    \(X\) finite: \C \\
    \(X\) infinite: \X
  \end{minipage}
}
\begin{document}
Cavallaro, Jeffery \\
Math 275A \\
Homework \#6

\bigskip

\begin{example}[Exercise 4.13]
  \setlength\extrarowheight{2ex}
  \begin{tabular}{|c|c|c|c|c|}
    \hline
    \(SPACE\) & \(T_1\) & \(T_2\) & REGULAR & NORMAL \\
    \hline
    \(R_{\text{std}}\) & \C & \C & \C & \C \\
    \hline
    \(R^n_{\text{std}}\) & \C & \C & \C & \C \\
    \hline
    indiscrete & \X & \X & \X & \X \\
    \hline
    discrete & \C & \C & \C & \C \\
    \hline
    cofinite & \C & \A & \A & \A \\
    \hline
    cocountable & \C & \A & \A & \A \\
    \hline
    \(R_{LL}\) & \C & \C & \C & \C \\
    \hline
    \(R_{+00}\) & \C & \X & \X & \X \\
    \hline
    LOS & \C & \C & \C & \C \\
    \hline
  \end{tabular}

  \bigskip

  \underline{\(R\) and \(\R^n\)}

  Since there is a finite distance between points and closed sets (not containing those points), there is always room
  for enclosing disjoint balls.

  \underline{indiscrete}

  Since the only non-empty set is the entire space, there is no separation.

  \underline{discrete}

  Since all disjoint subsets are both open and closed, they are self-enclosed.

  \underline{cofinite}

  All finite sets are closed.  Thus, single points can be viewed as closed sets.  Given any two closed sets \(A\) and
  \(B\), the sets \(X-A\) and \(X-B\) with \(A\notin X-A\), \(A\in X-B\), \(B\notin X-B\), and \(B\in X-A\).  Thus,
  cofinite is \(T_1\).  If \(X\) is finite then all disjoint subsets are both open and closed and hence
  self-enclosing.  Otherwise, enclosing open sets will always have some overlap.

  \underline{cocountable}

  Analagous to cofinite.

  \underline{\(\R_{LL}\)}

  Since \(R_{LL}\) is finer than \(\R\), it has the same separation properties.

  \underline{\(\R_{+00}\)}

  Any two points can be \(T_1\) separated using the basis elements; however, if one point or closed set contains
  \(0'\) and the other point or closed set contains \(0''\) then there is always overlap between the two containing
  basis elements.

  \underline{Lexigraphically Ordered Square}

  Use the alternate definitions.  For any point \(p\in X\), there exists some containing open set (strip), and it is
  always possible to use a smaller strip whose closure is contained in the original strip.  For any closed set
  \(A\in X\), \(X-A\) is an enclosing open set, and likewise, a smaller open set with contained closure is possible.
\end{example}

\begin{theorem}[4.16]
  \(X,Y\) are \(T_2\implies X\times Y\) is \(T_2\).
\end{theorem}

\begin{proof}
  Assume that \(X\) and \(Y\) are \(T_2\) and assume \(p_1,p_2\in X\times Y\) where \(p_1=(x_1,y_1)\) and
  \(p_2=(x_2,y_2)\).  Since \(X\) is \(T_2\), there exists \(U_1,U_2\in\T_X\) such that \(x_1\in U_1\) and
  \(x_2\in U_2\) and \(U_1\cap U_2=\emptyset\).  Likewise, since \(Y\) is \(T_2\), there exists \(V_1,V_2\in\T_Y\)
  such that \(y_1\in V_1\) and \(y_2\in V_2\) and \(V_1\cap V_2=\emptyset\).  So \(p_1\in U_1\times V_1\) and
  \(p_2\in U_2\times V_2\).  Furthermore, \(U_1\times V_1,U_2\times V_2\in\T_{X\times Y}\) and
  \[(U_1\times V_1)\cap(U_2\times V_2)=(U_1\cap U_2)\times(V_1\cap V_2)=\emptyset\]

  Therefore \(X\times Y\) is \(T_2\).
\end{proof}

\begin{lemma}
  Let \(X\) and \(Y\) be topological spaces and let \(A\subset X\) and \(B\subset Y\):
  \[\overline{A\times B}=\bar{A}\times\bar{B}\]
\end{lemma}

\begin{proof}
  Assume that \(p\in\overline{A\times B}\).  This means that for all \(U\in\T_{X\times B}\) such that \(p\in U\):
  \[U\cap(A\times B)\ne\emptyset\]
  Now assume \(U_1\in\T_X\) and \(U_2\in T_Y\) such that \(p\in U_1\times U_2\in\T_{A\times B}\).  Then it must be the
  case that \((U_1\times U_2)\cap(A\times B)=(U_1\cap A)\times(U_2\cap B)\ne\emptyset\).  This is only possible if
  \(U_1\cap A\ne\emptyset\) and \(U_2\cap B\ne\emptyset\).

  Therefore \(p\in\bar{A}\times\bar{B}\).

  Assume that \(p\in\bar{A}\times\bar{B}\).  This means that for all \(U_1\in\T_X\) and \(U_2\in\T_Y\) such that
  \(p\in U_1\times U_2\):
  \[(U_1\cap A)\times(U_2\cap B)\ne\emptyset\]
  Now assume \(U\in\T_{A\times B}\) such that \(p\in U\in\T_{A\times B}\).  Then there exists \(U_1\in\T_X\) and
  \(U_2\in T_Y\) such that \(p\in U_1\times U_2=U\).  So it must be the case that:
  \[U\cap(A\times B)=(U_1\times U_2)\cap(A\times B)=(U_1\cap A)\times(U_2\cap B)\ne\emptyset\]

  Therefore \(p\in\overline{A\times B}\).
\end{proof}

\begin{theorem}[4.17]
  \(X,Y\) are regular \(\implies X\times Y\) is regular.
\end{theorem}

\begin{proof}
  Assume that \(X\) and \(Y\) are regular and assume \(p\in X\times Y\) and \(U\in\U_p\).  Then there exists
  \(U_1\in\T_X\) and \(U_2\in\T_Y\) such that \(p\in U_1\times U_2\subset U\).  Now, since \(X\) and \(Y\) are
  regular, there exists \(V_1\in\T_X\) and \(V_2\in\T_y\) such that \(p\in V_1\times V_2\),
  \(V_1\subset\overline{V_1}\subset U_1\), and \(V_2\subset\overline{V_2}\subset U_2\).  Furthermore, since
  \(\overline{V_1}\) is closed in \(X\) and \(\overline{V_2}\) is closed in \(Y\),
  \(\overline{V_1}\times\overline{V_2}\) (and hence \(\overline{V_1\times V_2}\)) is closed in \(X\times Y\).
  And so:
  \[p\in V_1\times V_2\subset\overline{V_1\times V_2}=\overline{V_1}\times\overline{V_2}\subset U_1\times U_2\]
  Therefore \(X\times Y\) is regular.
\end{proof}

\begin{theorem}[4.19]
  Every \(T_2\) space is hereditarily \(T_2\).
\end{theorem}

\begin{proof}
  Assume that \(X\) is a \(T_2\) topological space and assume that \(Y\subset X\).  Now assume that \(a,b\in Y\).
  Thus \(a,b\in X\) and, since \(X\) is \(T_2\), there exists \(U,V\in\T_X\) such that \(a\in U\), \(b\in V\), and
  \(U\cap V=\emptyset\).  Furthermore, \(a\in U\cap Y\in\T_Y\) and \(b\in V\cap Y\in\T_Y\).  And so:
  \[(Y\cap U)\cap(Y\cap V)=Y\cap(U\cap V)=Y\cap\emptyset=\emptyset\]
  Therefore \(Y\) is also \(T_2\).
\end{proof}

\begin{theorem}[4.20]
  Every regular space is hereditarily regular.
\end{theorem}

\begin{proof}
  Assume that \(X\) is a regular topological space and assume that \(Y\subset X\).  Assume \(p\in Y\).  There
  exists \(U_X\in\T_X\) such that \(p\in U_X\) and so \(p\in U_X\cap Y=U_Y\in\T_Y\).  Now, since \(X\) is regular,
  there exists \(V_X\in\T_X\) such that \(p\in V_X\subset\overline{V_X}\subset U_X\), and hence
  \(p\in V_X\cap Y=V_Y\in\T_Y\).  Furthermore, since \(\overline{V_X}\) is closed in \(X\),
  \(\overline{V_X}\cap Y=W_Y\) is closed in \(Y\).  Finally, since \(\overline{V_Y}\) is the smallest closed set in
  \(Y\) containing \(V_Y\):
  \[p\in V_Y\subset\overline{V_Y}\subset W_Y\subset U_Y\]
  Therefore \(Y\) is regular.
\end{proof}

\begin{theorem}[4.23]
  Let \(X\) be a normal topological space and let \(Y\subset X\) such that \(Y\) is closed in \(X\).  \(Y\) is
  normal when given the relative topology.
\end{theorem}

\begin{proof}
  Assume \(A,B\subset Y\) such that \(A\) and \(B\) are closed in \(Y\) and \(A\cap B=\emptyset\).  Since \(A\) is
  closed in \(Y\), \(Y-A\in\T_Y\).  This means that there exists \(W\in\T_X\) such that \(W\cap Y=Y-A\).
  Furthermore, \(X-W\) is closed in \(X\).  Now:
  \[(X-W)\cap Y=(X\cap Y)-(W\cap Y)=Y-(Y-A)=A\]
  But \(X-W\) and \(Y\) are closed in \(X\) and hence \(A\) is also closed in \(X\).  By similar argument, \(B\) is
  also closed in \(X\).  And, since \(X\) is normal, there exists \(U,V\in\T_X\) such that \(A\in U\), \(B\in V\),
  and \(U\cap V=\emptyset\).  Finally, since \(A\subset(U\cap Y)\in\T_Y\) and \(B\subset(V\cap Y)\in\T_Y\):
  \[(U\cap Y)\cap(V\cap Y)=(U\cap V)\cap Y=\emptyset\cap Y=\emptyset\]
  Therefore \(Y\) is normal.
\end{proof}

\end{document}
