\documentclass[letterpaper,12pt,fleqn]{article}
\usepackage{matharticle}
\usepackage{array}
\usepackage{tikz}
\pagestyle{plain}
\newcommand{\U}{\mathcal{U}}
\newcommand{\B}{\mathcal{B}}
\renewcommand{\sc}{\(2^{nd}\) countable}
\begin{document}
Cavallaro, Jeffery \\
Math 275A \\
Homework \#7 Rewrite

\bigskip

\begin{theorem}[5.11]
  Every uncountable set in a \sc\ space has a limit point.
\end{theorem}

\begin{proof}
  Assume that \(X\) is a \sc\ space and assume that \(A\subset X\) such that \(A\) is uncountable.  Now, ABC that
  \(A\) has no limit points.  This means that for all \(a\in A\) it is the case that there exists \(U\in\U_a\)
  such that \(U_a\cap A=\set{a}\) and hence every \(a\in A\) is an isolated point.  So assume that \(x,y\in A\)
  such that \(x\ne y\).  There exists \(U\in\U_x\) and \(V\in\U_y\) such that \(U\ne V\).  So for any basis
  \(\B\) of \(X\), there exists \(B_x\subset U\) and \(B_y\subset V\) such that \(B_x\ne B_y\).  Thus,
  \(a\mapsto B_a\) is injective and hence \(\B\) is uncountable, contradicting the assumption that \(X\)
  is \sc.

  Therefore \(A\) contains a limit point.
\end{proof}

\end{document}
