\documentclass[letterpaper,12pt,fleqn]{article}
\usepackage{matharticle}
\newcommand{\T}{\mathscr{T}}
\renewcommand{\a}{\alpha}
\renewcommand{\l}{\lambda}
\begin{document}
Cavallaro, Jeffery \\
Math 275A \\
Homework \#11

\bigskip

\begin{theorem}[7.15]
  Let \(X\) and \(Y\) be topological spaces.  If \(X\) is compact and \(f:X\to Y\) is continuous and surjective then
  \(Y\) is compact.
\end{theorem}

\begin{proof}
  Assume that \(X\) is compact and \(f:X\to Y\) is continuous and surjective.  Assume that \(\set{V_{\a}:\a\in\l}\)
  is an open cover for \(Y\).  Since \(f\) is continuous, each \(f^{-1}(V_{\a})\in\T_X\).  Furthermore, since \(f\)
  is surjective, \(\displaystyle f^{-1}(\bigcup_{\a\in\l}V_{\a})=\bigcup_{\a\in\l}f^{-1}(V)\) is an open cover of \(X\).
  But \(X\) is compact, so there exists a finite subcover \(\set{f^{-1}(V_1),\ldots,f^{-1}(V_n)}\) of \(X\).  And
  since \(f\) is surjective \(\set{V_1,\ldots,V_n}\) is a finite subcover for \(Y\).  Therefore \(Y\) is compact.
\end{proof}

\begin{theorem}[7.18]
  Let \(X\) and \(Y\) be topological spaces.  If \(D\) is dense in \(X\) and \(f:X\to Y\) is continuous and surjective
  then \(f(D)\) dense in \(Y\).
\end{theorem}

\begin{proof}
  Assume that \(D\) is dense in \(X\) and \(f:X\to Y\) is continuous and surjective.  Assume that \(V\in\T_Y\) and
  \(V\ne\emptyset\).  Since \(f\) is continuous, \(f^{-1}(V)\in\T_Y\).  Furthermore, since \(f\) is surjective,
  \(f^{-1}(V)\ne\emptyset\), and since \(D\) is dense in \(X\), \(f^{-1}(V)\cap D\ne\emptyset\).  Therefore
  \(f(U)\cap f(D)\ne\emptyset\) and thus \(f(D)\) is dense in \(Y\).
\end{proof}

\begin{example}[Exercise 7.20]
  \begin{enumerate}
  \item[]
  \item An open function that is not continuous.

    Consider \(f:\R_{\text{cof}}\to\R_{\text{coc}}\) defined by \(f(x)=x\).  Since every open set in the cofinite
    topology is open in the cocountable topology, \(f\) is open.  However, \(\R-\Q\) is open in the cocountable
    topology but not in cofinite topology and so \(f\) is not continuous.

  \item A closed function that is not continuous.

    Consider \(f:\R_{\text{cof}}\to\R_{\text{coc}}\) defined by \(f(x)=x\).  Since every closed set in the cofinite
    topology is closed in the cocountable topology, \(f\) is closed.  However, \(\Q\) is closed in the cocountable
    topology but not in cofinite topology and so \(f\) is not continuous.

  \item A continuous function that is neither open nor closed.

    Consider \(f:\R_{\text{dis}}\to\R_{\text{ind}}\) defined by \(f(x)=x\).  Since the only open (and closed) sets
    in the indiscrete topology are \(\emptyset\) and \(R\), and these sets are also open in the indiscrete
    topology, \(f\) is continuous.  However, \([0,1]\) is open and closed in the discrete topology, but neither
    in the indiscrete topology, so \(f\) is neither open nor closed.

  \item A continuous function that is open but not closed.

    Consider \(f:\R\to\R\) defined by \(f(x)=e^x\), which is continuous.  Since \(f((a,b))=(e^a,e^b)\), open sets
    will always map to open sets.  However, \(\R\) is closed in \(\R\) and \(f(\R)=(0,\infty)\), which is not
    closed in \(R\).  Thus, \(f\) is open but not closed.

  \item A continuous function that is closed but not open.

    Consider \(f:\R\to\R\) defined by \(f(x)\to y_0\).  This was already shown to be continuous.  Note that
    \(\set{y}\) is closed in \(\R\) so closed sets will always map to closed sets; however, open sets will also
    map to the closed set and thus \(f\) is closed but not open.
  \end{enumerate}
\end{example}

\begin{lemma}
  Let \(X\) and \(Y\) be topological spaces and let \(f:X\to Y\) be continuous and closed.  For all \(A\subset X\),
  \(f(\bar{A})=\overline{f(A)}\).
\end{lemma}

\begin{proof}
  Assume that \(A\subset X\).  Since \(f\) is continuous, \(f(\bar{A})\subset\overline{f(A)}\).  Now, since
  \(A\subset\bar{A}\), \(f(A)\subset f(\bar{A})\).  Furthermore, \(\bar{A}\) is closed and \(f\) is closed, so
  \(f(\bar{A})\) is closed.  But \(\overline{f(A)}\) is the smallest closed set containing \(f(A)\), and so
  \(f(A)\subset\overline{f(A)}\subset f(\bar{A})\).  Therefore \(f(\bar{A})=\overline{f(A)}\).
\end{proof}

\begin{theorem}[7.21]
  Let \(X\) and \(Y\) be topological spaces.  If \(X\) is normal and \(f:X\to Y\) is continuous, surjective, and
  closed then \(Y\) is normal.
\end{theorem}

\begin{proof}
  Assume that \(X\) is normal and \(f:X\to Y\) is continuous, surjective, and closed.  Assume that \(B\) is closed
  in \(Y\) and assume that \(V\in\T_Y\) such that \(B\subset V\).  Since \(f\) is continuous, \(f^{-1}(B)\) is
  closed in \(X\) and \(f^{-1}(V)\in\T_X\) with \(f^{-1}(B)\subset f^{-1}(V)\).  Now, since \(X\) is normal, there
  exists \(U\in\T_X\) such that \(f^{-1}(B)\subset U\) and \(\bar{U}\subset f^{-1}(V)\).  Now, since \(U\in\T_X\),
  \(X-U\) is closed in \(X\).  Since \(f\) is closed, \(f(X-U)\) is closed in \(Y\) and thus
  \(Y-f(X-U)\subset f(U)\in\T_Y\).

  Claim: \(B\subset Y-f(X-U)\)

  Assume \(y\in B\).  Since \(f\) is surjective, \(y\) is mapped and all such \(x\in f^{-1}(B)\subset U\).  Thus,
  \(x\notin X-U\), so \(y=f(x)\notin f(X-U)\), and hence \(y\in Y-f(X-U)\).

  Claim: \(\overline{Y-f(X-U)}\subset V\)

  Since \(Y-f(X-U)\subset f(U)\), \(\overline{Y-f(X-U)}\subset\overline{f(U)}\).  Now, since
  \(\bar{U}\subset f^{-1}(V)\), \(f(\bar{U})\subset f(f^{-1}(V))\subset V\).  But \(f\) is continuous, so
  \(f(\bar{U})=\overline{f(U)}\), and so \(\overline{Y-f(X-U)}\subset\overline{f(U)}\subset V\).

  Therefore \(Y\) is normal.
\end{proof}

\begin{theorem}[7.24]
  Let \(X\) and \(Y\) be topological spaces such that \(X\) is compact and \(Y\) is Hausdorff.  For all \(f:X\to Y\),
  if \(f\) is continuous then \(f\) is closed.
\end{theorem}

\begin{proof}
  Assume that \(f\) is continuous and assume that \(A\subset X\) is closed in \(X\).  Since \(X\) is compact,
  \(A\) is also compact.  Now, consider \(f(A)\) as a subspace of \(Y\).  Since \(f|_A\) is surjective, \(f(A)\)
  is compact.  Finally, since \(Y\) is Hausdorff, \(f(A)\) is closed.  Therefore \(f\) is closed.
\end{proof}

\begin{theorem}[7.26]
  Homeomorphic is an equivalence relation.
\end{theorem}

\begin{proof}
  Assume that \(X\),\(Y\), and \(Z\) are topological spaces.
  \begin{description}
  \item[R:] Consider \(i_X=i_X^{-1}\), which is continuous.  Therefore \(X\) is homeomorphic to \(X\).

  \item[S:] Assume that \(X\) is homeomorphic to \(Y\).

    Then there exists a homeomorphism \(f:X\to Y\).  Since \(f\) is a homeomorphism, it is invertible and its
    inverse is continuous. Thus, \(f^{-1}:Y\to X\) is a continuous, invertible function and \((f^{-1})^{-1}=f\) is
    invertible.  Therefore \(Y\) is homeomorphic to \(X\).

  \item[T:] Assume that \(X\) is homeomorphic to \(Y\) and \(Y\) is homeomorphic to \(Z\).

    Then there exists homeomorphics \(f:X\to Y\) and \(g:Y\to Z\).  So consider \(g\circ f:X\to Z\).  Since \(f\)
    and \(g\) are continuous and invertible, \(g\circ f\) is continuous and invertible.  Furthermore, since
    \(f^{-1}\) and \(g^{-1}\) are continuous, \(f^{-1}\circ g^{-1}=(g\circ f)^{-1}\) is continuous.  Therefore
    \(X\) is homeomorphic to \(Z\).
  \end{description}
\end{proof}

\begin{lemma}
  For all \(a,b\in\R\) such that \(a<b\), \((a,b)\) is homeomorphic to \((0,1)\).
\end{lemma}

\begin{proof}
  Let \(f:(0,1)\to(a,b)\) be defined by \(f(t)=a+t(a-b)\).  \(f\) is linear, and thus continuous and invertible
  with \(f^{-1}(s)=\frac{s-a}{b-a}\) which is also linear and thus continuous.  Therefore \((a,b)\) is homeomorphic
  to \((0,1)\).
\end{proof}

\begin{corollary}
  All open intervals in \(\R\) are homeomorphic.
\end{corollary}

\begin{proof}
  Assume \((a,b),(c,d)\subset\R\).  \((a,b)\) is homeomorphic to \((0,1)\) and \((0,1)\) is homeomorphic to
  \((c,d)\).  Therefore, \((a,b)\) is homeomorphic to \((c,d)\).
\end{proof}

\begin{theorem}[7.27]
  \((a,b)\subset\R\) is homeomorphic to \(R\).
\end{theorem}

\begin{proof}
  \((a,b)\) is homeomorphic to \((-\frac{\pi}{2},\frac{\pi}{2})\).  Now, consider
  \(f:(-\frac{\pi}{2},\frac{\pi}{2})\to\R\) defined by \(f(x)=\tan x\).  This is a continuous and invertible
  function whose inverse is also continuous.  Thus, \((-\frac{\pi}{2},\frac{\pi}{2})\) is \(\R\).  Therefore,
  \((a,b)\) is homeomorphic to \(R\).
\end{proof}

\begin{lemma}
  Let \(X\) and \(Y\) be topological spaces and let \(f:X\to Y\) be bijective.  For all \(A\subset X\):
  \[f(A)=Y-f(X-A)\]
\end{lemma}

\begin{proof}
  Assume \(A\subset X\).
  \begin{description}
  \item[\((\subset)\)] Assume \(y\in f(A)\).

    Since \(f\) is injective, there exists one and only one \(x\in X\) such that \(y=f(x)\) and that \(x\in A\).
    Thus, \(x\notin X-A\) and so \(y=f(x)\notin f(X-A)\).  Therefore \(y\in Y-f(X-A)\).

  \item[\((\supset)\)] Assume \(y\in Y-f(X-A)\).

    Thus, \(y\notin f(X-A)\) and so there is no \(x\in X-A\) such that \(y=f(x)\).  But \(f\) is surjective, and
    so there is such an \(x\in X\) and that \(x\in A\).  Therefore \(y=f(x)\in f(A)\).
  \end{description}
\end{proof}

\begin{theorem}[7.28]
  Let \(X\) and \(Y\) be topological spaces and let \(f:X\to Y\) be continuous.  TFAE:
  \begin{enumerate}
  \item \(f\) is a homeomorphism.
  \item \(f\) is a closed bijection.
  \item \(f\) is an open bijection.
  \end{enumerate}
\end{theorem}

\begin{proof}
  \begin{description}
  \item[]
  \item[\((1\implies 2)\)] Assume that \(f\) is a homeomorphism.

    This means that \(f\) is a bijection and its inverse is continuous.  So assume that \(A\subset X\) is closed in
    \(X\). Since \(f\) is bijective, \(f(A)=(f^{-1})^{-1}(A)\), and since \((f^{-1})^{-1}\) is continuous, \(f(A)\)
    is also closed.  Therefore \(f\) is a closed bijection.

  \item[\((2\implies 3)\)] Assume that \(f\) is a closed bijection.

    Assume that \(U\in\T_X\).  This means that \(X-U\) is closed in \(X\), and since \(f\) is closed, \(f(X-U)\)
    is closed in \(Y\) and so \(Y-f(X-U)\in\T_Y\).  But \(f\) is a bijection and so \(Y-f(X-U)=f(U)\in\T_Y\).
    Therefore, \(f\) is an open bijection.

  \item[\((3\implies 1)\)] Assume that \(f\) is an open bijection.

    Assume that \(U\in\T_Y\).  Since \(f\) is continuous, \(f^{-1}(U)\in\T_X\).  But \(f\) is open so
    \((f^{-1})^{-1}(U)\in\T_Y\).  Therefore \(f^{-1}\) is continuous and hence \(f\) is a homeomorphism.
  \end{description}
\end{proof}

\begin{theorem}[7.29]
  Let \(X\) and \(Y\) be topological spaces such that \(X\) is compact and \(Y\) is Hausdorff and let \(f:X\to Y\)
  be a continuous bijection.  \(f\) is a homeomorphism.
\end{theorem}

\begin{proof}
  Since \(X\) is compact, \(Y\) is Hausdorff, and \(f\) is a bijection, \(f\) is closed.  Therefore, since \(f\) is
  a continuous closed bijection, \(f\) is a homeomorphism.
\end{proof}

\end{document}
