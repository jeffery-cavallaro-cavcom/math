\documentclass[letterpaper,12pt,fleqn]{article}
\usepackage{matharticle}
\newcommand{\U}{\mathcal{U}}
\renewcommand{\a}{\alpha}
\renewcommand{\l}{\lambda}
\newcommand{\Rs}{\R_{\text{std}}}
\newcommand{\A}{\mathcal{A}}
\newcommand{\T}{\mathscr{T}}
\newcommand{\K}{\mathcal{K}}
\begin{document}
Cavallaro, Jeffery \\
Math 275A \\
Homework \#8

\bigskip

\begin{theorem}[6.2]
  Let \(A\subset\Rs\).  If \(A\) is compact then \(A\) has a maximum point.
\end{theorem}

\begin{proof}
  If \(A\) is finite then trivial, so assume that \(A\) is infinite.  ABC that \(A\) has no maximum point.  This
  means that for all \(a\in A\) there exists \(b_a\in A\) such that \(b_a>a\).  So let \(\set{(-\infty,b_a):a\in A}\)
  be an open cover for \(A\).  Since \(A\) is compact, there exists a finite subcover
  \(U=\set{(-\infty,b_{a_k}):1\le k\le n}\).  Let \(c=\max\set{b_{a_k}}\), and so \(\bigcup U=(-\infty,c)\).  Thus
  \(c\in A\) but \(c\notin U\), contradicting the assumption that \(U\) is a finite subcover.

  Therefore \(A\) has a maximum point.
\end{proof}

\begin{theorem}[6.9]
  Every compact subspace of a Hausdorff space is closed.
\end{theorem}

\begin{proof}
  Assume that \(X\) is Hausdorff and \(A\) is a compact subspace of \(X\).  Assume that \(b\in A^C\).  Since \(X\)
  is Hausdorff, for every \(a\in A\) there exists \(U_a,V_a\in\T_X\) such that \(a\in U_a\), \(b\in V_a\), and
  \(U_a\cap V_a=\emptyset\).  So let the \(\set{U_a:a\in A}\) be an open cover of \(A\) in \(X\).  Thus
  \(\set{U_a\cap A:a\in A}\) for \(U_a\cap A\in\T_Y\) is an open cover of \(A\) in \(A\).  Now, since \(A\) is a
  compact subspace of \(X\), there exists a finite subcover \((U_{a_1}\cap A)\cup\cdots\cup(U_{a_n}\cap A)\) of
  \(A\) in \(A\), and hence a finite subcover \(U_{a_1}\cup\cdots\cup U_{a_n}\) of \(A\) in \(X\).  Let
  \(V=V_{a_1}\cap\cdots\cap V_{a_n}\).  Note that \(b\in V\) and \(V\in\T_X\).  Furthermore, since all the
  \(U_a\cap V_a=\emptyset\), it must be the case that \(V\cap(U_{a_1}\cup\cdots\cup U_{a_n})=\emptyset\).  But
  since \(U_{a_1}\cup\cdots\cup U_{a_n}\supset A\) it must be the case that \(V\subset A^C\).  So \(b\) is an
  interior point in \(A^C\), meaning that all the points in \(A^C\) are interior, and so \(A^C\in\T_X\).  Therefore
  \(A\) is closed in \(X\).
\end{proof}

\begin{lemma}
  Every compact, Hausdorff space is regular.
\end{lemma}

\begin{proof}
  Assume that \(X\) is compact and Hausdorff.  Assume that \(A\subset X\) is closed.  Thus, by previous theorem,
  \(A\) is also compact.  So assume \(p\in A^C\).  This means that \(p\notin A\) and so, by the previous proof,
  there exists \(U,V\in\T\) such that \(A\subset U\) and \(p\in V\) and \(U\cap V=\emptyset\).

  Therefore \(X\) is regular.
\end{proof}

\begin{theorem}[6.12]
  Every compact, Hausdorff space is normal.
\end{theorem}

\begin{proof}
  Assume \(A,B\subset X\) are closed.  Since \(X\) is regular (by the previous lemma), for all \(b\in B\) there
  exists \(U_b,V_b\in\T\) such that \(A\subset U_b\) and \(b\in V_b\) and \(U_b\cap V_b=\emptyset\).  So let
  \(V=\set{V_b:b\in B}\) be an open cover for \(B\).  But, by previous theorem, \(B\) is also compact, and so
  there exists a finite subcover \(V_{b_1}\cup\cdots\cup V_{b_n}\supset B\).  So let
  \(U=U_{b_1}\cap\cdots\cap U_{b_n}\in\T\).  Note that \(A\subset U\) and, since all the \(U_b\cap V_b=\emptyset\),
  \(U\cap V=\emptyset\).  Therefore, \(X\) is normal.
\end{proof}

\end{document}
