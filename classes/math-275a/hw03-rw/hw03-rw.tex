\documentclass[letterpaper,12pt,fleqn]{article}
\usepackage{matharticle}
\usepackage{tikz}
\pagestyle{plain}
\newcommand{\T}{\mathscr{T}}
\newcommand{\U}{\mathcal{U}}
\newcommand{\e}{\epsilon}
\begin{document}
Cavallaro, Jeffery \\
Math 275A \\
Homework \#3 Rewrite

\bigskip

\begin{theorem}[2.31]
  Let \((\R^n,\T)\) be the standard topology, \(A\subset\R^n\), and \(p\in X\) be a limit point of \(A\).  There
  exists a sequence of points in \(A\) that converge to \(p\).
\end{theorem}

\begin{proof}
  Let \(U_i=B\left(p,\e_i\right)\) where \(\e_i=\frac{1}{i}\) for \(i\in\N\).  Note that \(\e_i=\frac{1}{i}\to0\)
  as \(i\to\infty\).  Also note that \(U_i\cap A\ne\emptyset\) because \(p\) is a limit point of \(A\), so select
  \(x_i\in U_i\cap A\).  Thus, all of the \(x_i\in A\).

  Claim: \((x_i)_{i\in\N}\) is a sequence in \(A\) converging to \(p\).

  Assume \(U\in\U_p\).  Then there exists some \(\e>0\) such that \(B(p,\e)\subset U\).  Since the \(\e_i\to0\),
  there exists some \(\e_N<\e\).  Assume \(i>N\).  This means that \(e_i<e_N<e\) and so
  \(x_i\in U_i\subset U_N\subset U\) and therefore \(x_i\in U\).
\end{proof}

\end{document}
