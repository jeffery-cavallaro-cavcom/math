\documentclass[letterpaper,12pt,fleqn]{article}
\usepackage{matharticle}
\pagestyle{empty}
\newcommand{\vertex}[2]{\node [draw,circle,fill,scale=0.5] (#1) at #2 {}}

\newcommand{\gkfour}[1]{
  \begin{tikzpicture}[scale=#1]
    \vertex{a}{(0,0)};
    \vertex{b}{(1,0)};
    \vertex{c}{(1,1)};
    \vertex{d}{(0,1)};
    \draw (a) to (b) to (c) to (d) to (a);
    \draw (a) to (c);
    \draw (b) to (d);
  \end{tikzpicture}
}

\newcommand{\gpeter}[1]{
  \begin{tikzpicture}[scale=#1]
    \vertex{a}{({2*cos(72)},0)};
    \vertex{b}{({2*cos(72)+2},0)};
    \vertex{c}{({4*cos(72)+2},{2*sin(72)})};
    \vertex{d}{({2*cos(72)+1},{4*sin(72)})};
    \vertex{e}{(0,{2*sin(72)})};
    \vertex{f}{({2*cos(72)+cos(54)},0.75)};
    \vertex{g}{({2*cos(72)+2-cos(54)},0.75)};
    \vertex{h}{({4*cos(72)+2-0.75},{2*sin(72)})};
    \vertex{i}{({2*cos(72)+1},{4*sin(72)-1})};
    \vertex{j}{(0.75,{2*sin(72)})};
    \draw (a) to (b) to (c) to (d) to (e) to (a);
    \draw (f) to (i) to (g) to (j) to (h) to (f);
    \draw (a) to (f);
    \draw (b) to (g);
    \draw (c) to (h);
    \draw (d) to (i);
    \draw (e) to (j);
  \end{tikzpicture}
}

\DeclarePairedDelimiter{\ceil}{\lceil}{\rceil}
\DeclarePairedDelimiter{\floor}{\lfloor}{\rfloor}
\newcommand{\comp}[1]{\overline{#1}}
\begin{document}
\section*{Decomposition}

\begin{definition}[Decomposition]
  Let $G$ be a simple graph. A \emph{decomposition} of $G$ is a collection of
  simple graphs $G_1,G_2,\ldots G_k$ such that:
  \begin{enumerate}
  \item $E(G)=\bigcup_{k=1}^nE(G_k)$
  \item $\forall\,i,j,E(G_i)\cap E(G_j)=\emptyset$
  \end{enumerate}
\end{definition}

\begin{examples}
  \listbreak
  \begin{enumerate}
  \item $K_4$

    \begin{minipage}{1in}
      \centering
      \gkfour{2}
    \end{minipage}
    \begin{minipage}{0.5in}
      \centering
      $=$
    \end{minipage}
    \begin{minipage}{1in}
      \centering
      \begin{tikzpicture}[scale=2]
        \vertex{a}{(0,0)};
        \vertex{b}{(1,0)};
        \vertex{c}{(1,1)};
        \vertex{d}{(0,1)};
        \draw (a) to (c) to (d) to (a);
        \draw (b) to (c);
      \end{tikzpicture}
    \end{minipage}
    \begin{minipage}{0.5in}
      \centering
      $\bigcup$
    \end{minipage}
    \begin{minipage}{1in}
      \centering
      \begin{tikzpicture}[scale=2]
        \vertex{a}{(0,0)};
        \vertex{b}{(1,0)};
        \vertex{d}{(0,1)};
        \draw (a) to (b);
        \draw (b) to (d);
      \end{tikzpicture}
    \end{minipage}

    \bigskip

  \item $P=5P_4$

    \begin{minipage}{1.5in}
      \centering
      \gpeter{1}
    \end{minipage}
    \begin{minipage}{0.5in}
      \centering
      $=$
    \end{minipage}
    \begin{minipage}{1.5in}
      \centering
      \begin{tikzpicture}
        \vertex{d}{({2*cos(72)+1},{4*sin(72)})};
        \vertex{e}{(0,{2*sin(72)})};
        \vertex{g}{({2*cos(72)+2-cos(54)},0.75)};
        \vertex{i}{({2*cos(72)+1},{4*sin(72)-1})};
        \draw (g) to (i) to (d) to (e);
      \end{tikzpicture}
    \end{minipage}
    \begin{minipage}{0.5in}
      \centering
      $\bigcup$
    \end{minipage}
    \begin{minipage}{1.5in}
      \centering
      \begin{tikzpicture}
        \vertex{c}{({4*cos(72)+2},{2*sin(72)})};
        \vertex{d}{({2*cos(72)+1},{4*sin(72)})};
        \vertex{f}{({2*cos(72)+cos(54)},0.75)};
        \vertex{h}{({4*cos(72)+2-0.75},{2*sin(72)})};
        \draw (f) to (h) to (c) to (d);
      \end{tikzpicture}
    \end{minipage}
    \begin{minipage}{0.5in}
      \centering
      $\bigcup$
    \end{minipage}

    \vspace{0.5in}
    
    \begin{minipage}{1.5in}
      \centering
      \begin{tikzpicture}
        \vertex{b}{({2*cos(72)+2},0)};
        \vertex{c}{({4*cos(72)+2},{2*sin(72)})};
        \vertex{g}{({2*cos(72)+2-cos(54)},0.75)};
        \vertex{j}{(0.75,{2*sin(72)})};
        \draw (c) to (b) to (g) to (j);
      \end{tikzpicture}
    \end{minipage}
    \begin{minipage}{0.5in}
      \centering
      $\bigcup$
    \end{minipage}
    \begin{minipage}{1.5in}
      \centering
      \begin{tikzpicture}
        \vertex{a}{({2*cos(72)},0)};
        \vertex{b}{({2*cos(72)+2},0)};
        \vertex{f}{({2*cos(72)+cos(54)},0.75)};
        \vertex{i}{({2*cos(72)+1},{4*sin(72)-1})};
        \draw (b) to (a) to (f) to (i);
      \end{tikzpicture}
    \end{minipage}
    \begin{minipage}{0.5in}
      \centering
      $\bigcup$
    \end{minipage}
    \begin{minipage}{1.5in}
      \centering
      \begin{tikzpicture}
        \vertex{a}{({2*cos(72)},0)};
        \vertex{e}{(0,{2*sin(72)})};
        \vertex{h}{({4*cos(72)+2-0.75},{2*sin(72)})};
        \vertex{j}{(0.75,{2*sin(72)})};
        \draw (a) to (e) to (j) to (h);
      \end{tikzpicture}
    \end{minipage}
  \end{enumerate}
\end{examples}

\begin{definition}[Path Number]
  Let $G$ be a simple graph. The \emph{path number} for $G$, denoted $p(G)$, is
  the minimum number of paths in any decomposition of $G$.
\end{definition}

\newpage

\begin{conjecture}[Gallai]
  Let $G$ be a simple graph of order $n$. $G$ can be decomposed into at most
  $\ceil*{\frac{n}{2}}$ paths. In other words:
  \[p(G)\le\ceil*{\frac{n}{2}}\]
\end{conjecture}

\begin{example}
  Since $n(P)=10$, so $p(P)\le\ceil*{\frac{10}{2}}=5$, meaning the Petersen
  graph can be decomposed into at most $5$ paths --- indeed: $P=5P_4$ as shown
  above and $p(P)=5$, but we need a way to find a lower bound.
\end{example}

\begin{theorem}
  Let $G$ be a simple graph. If $G$ can be decomposed into $k$ paths then the
  number of odd vertices in $G\le2k$.
\end{theorem}

\begin{theproof}
  Assume $G$ can be decomposed into $k$ paths:
  \[G=\bigcup_{i=1}^kP_{n_i}\]
  Isolated vertices are treated as $P_1$ and are considered even. Each of the
  other paths consist of $2$ odd vertices and $n_i-2$ even vertices:

  \begin{tikzpicture}
    \vertex{a}{(0,0)} node at (a) [below] {$o$};
    \vertex{b}{(1,0)} node at (b) [below] {$e$};
    \node (c) at (2,0) {};
    \node (d) at (3,0) {};
    \vertex{e}{(4,0)} node at (e) [below] {$e$};
    \vertex{f}{(5,0)} node at (f) [below] {$o$};
    \draw (a) to (b) to (c);
    \draw (d) to (e) to (f);
    \node at (2.5,0) {$\cdots$};
  \end{tikzpicture}

  Let $G_i$ be the spanning subgraph of $G$ such that $E(G_i)=E(P_{n_i})$, in
  other words, $G_i$ contains the vertices and edges from $P_{n_i}$ and all of
  the remaining vertices in $G$ as isolated vertices. For example, if $G=K_4$,
  one such $G_i$ might be:

  \begin{tikzpicture}
    \vertex{a}{(0,0)};
    \vertex{b}{(2,0)};
    \vertex{c}{(2,2)};
    \vertex{d}{(0,2)};
    \draw (a) to (b) to (d);
  \end{tikzpicture}

  consisting of a $P_3$ and one isolated vertex.

  Let $e(G_i)$ be the number of even (including isolated) vertices in $G_i$ and
  $o(G_i)$ be the number of odd vertices in $G_i$:
  \[\bigcap_{i=1}^ke(G_i)\subseteq e(G)\]
  The LHS contains all isolated vertices in $G$ and all vertices that are
  either isolated or in the interior of the path in each $G_i$. Note that a
  vertex that is in the interior of multiple paths (never an end vertex of a
  path) remains even:

  \begin{tikzpicture}
    \node (a) at (0,0) {};
    \node (b) at (0,2) {};
    \node (c) at (2,2) {};
    \node (d) at (2,0) {};
    \vertex{e}{(1,1)};
    \draw (a) to (e) to (c);
    \draw (b) to (e) to (d);
  \end{tikzpicture}

  However, the LHS is a subset of the RHS, because two end (odd) vertices in
  different paths can be joined in $G$ to form an even vertex:
  
  \begin{tikzpicture}
    \node (a) at (0,0) {};
    \vertex{b}{(1,0)};
    \node (c) at (2,0) {};
    \draw (a) to (b) to (c);
  \end{tikzpicture}

  But no even vertex on the LHS can be converted to an odd vertex, because it
  would have to be an end vertex in at least one path and thus would not be in
  the LHS:

  \begin{tikzpicture}
    \node (a) at (0,2) {};
    \vertex{b}{(0,1)};
    \node (c) at (0,0) {};
    \node (d) at (1,1) {};
    \draw (a) to (b) to (c);
    \draw (b) to (d);
  \end{tikzpicture}
\end{theproof}

Now, take the complement of both sides:
\[\comp{\bigcap_{i=1}^ke(G_i)}\supseteq\comp{e(G)}=o(G)\]
To see why this is true, consider the following Venn diagram for $V(G)$:

\begin{tikzpicture}
  \draw [fill=lightgray] (0,0) rectangle (6,6);
  \draw (3,3) circle [radius=2];
  \draw [dashed,fill=white] (3,3) circle [radius=1];
  \node at (5,5) {$o(G)$};
  \node at (3,1.5) {$e(G)$};
  \node at (3,3) {$\bigcap_{i=1}^ke(G_i)$};
\end{tikzpicture}

Now, applying DeMorgan:
\[o(G)\subseteq\comp{\bigcap_{i=1}^ke(G_i)}=\bigcup_{i=1}^k\comp{e(G_i)}=
\bigcup_{i=1}^ko(G_i)\]
And then the triangle inequality:
\[\abs{o(G)}\le\abs{\sum_{i=1}^ko(G_i)}\le\sum_{i=1}^k\abs{o(G_i)}=
\sum_{i=1}^k2=2k\]

$\therefore\abs{o(G)}\le2k$

\begin{examples}
  \listbreak
  \begin{enumerate}
  \item P

    By Gallai: $p(P)\le5$. By the lemma: $10\le2p(P)$ and so $p(P)\ge5$.
    Therefore $p(P)=5$. A decomposition $P=5P_4$ is shown above.

  \item $C_n$

    $P(C_n)=P_n\cup P_2$ and so $p(C_n)=2$

  \item $ST_n$

    $ST_{2k+1}=\bigcup_{i=1}^kP_3$ and so $p(ST_{2k+1})=k$

    $ST_{2k}=\bigcup_{i=1}^{k-1}P_3\cup P_2$ and so $p(ST_{2k})=k$

    Therefore: $p(ST_n)=\floor*{\frac{n}{2}}$
  \end{enumerate}
\end{examples}

\end{document}
