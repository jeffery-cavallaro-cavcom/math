\documentclass[letterpaper,12pt,fleqn]{article}
\usepackage{matharticle}
\usepackage{enumitem}
\pagestyle{plain}
\newcommand{\p}{\phi}
\renewcommand{\r}{\sim}
\newcommand{\bas}[2]{\left<#1,#2\right>}
\newcommand{\cycle}[1]{\left<#1\right>}
\newcommand{\aut}[1]{\operatorname{Aut}{#1}}
\renewcommand{\o}{\sigma}
\renewcommand{\t}{\tau}
\newcommand{\iso}{\simeq}
\newcommand{\zstar}[2]{\bas{\Z_{#1}^*}{#2}}
\newcommand{\znx}{\Z_n^{\times}}
\begin{document}
Cavallaro, Jeffery \\
Math 221a \\
Homework \#2

\bigskip

\subsection*{1.2.1}

Let $\p:G\to H$ be a homomorphism of groups.
\begin{enumerate}[label=\alph*)]
\item Prove: $\p(e_G)=e_H$

  Assume $a\in G$

  \begin{minipage}[t]{3in}
    $\p(ae_G)=\p(a)$ \\
    $\p(ae_G)=\p(a)\p(e_G)$ \\
    $\p(a)\p(e_G)=\p(a)$
  \end{minipage}
  \begin{minipage}[t]{3in}
    $\p(e_Ga)=\p(a)$ \\
    $\p(e_Ga)=\p(e_G)\p(a)$ \\
    $\p(e_G)\p(a)=\p(a)$
  \end{minipage}

  \bigskip
  
  Thus $\p(e_G)$ is a two-sided identity for $H$; however, the identity is
  unique, \\
  $\therefore\p(e_G)=e_H$.

  \item Prove: $\p(a^{-1})=\p(a)^{-1}$

  Assume $a\in G$

  \begin{minipage}[t]{3in}
    $\p(aa^{-1})=\p(e_G)=e_H$ \\
    $\p(aa^{-1})=\p(a)\p(a^{-1})$ \\
    $\p(a)\p(a^{-1})=e_H$
  \end{minipage}
  \begin{minipage}[t]{3in}
    $\p(a^{-1}a)=\p(e_G)=e_H$ \\
    $\p(a^{-1}a)=\p(a^{-1})\p(a)$ \\
    $\p(a^{-1})\p(a)=e_H$
  \end{minipage}

  \bigskip
  
  Thus $\p(a^{-1})$ is a two-sided identity for $\p(a)$ in $H$; however,
  inverses are unique, \\
  $\therefore\p(a^{-1})=\p(a)^{-1}$.

\item Let $G=H=\bas{\Z_2}{\cdot}$ and define $\p:G\to H$ by $\p(x)=0$. $G$
  and $H$ are monoids with identity $1$ (0 has no inverse). Also:
  
  $\p(0x)=\p(0)=0=00=\p(0)\p(x)$ \\
  $\p(x0)=\p(0)=0=00=\p(x)\p(0)$ \\
  $\p(11)=\p(1)=0=00=\p(1)\p(1)$

  and so $\p$ is a homomorphism. However, $\p(1)=0\ne1$.
\end{enumerate}

\subsection*{1.2.2}

Let $G$ be a group and define $\p:G\to G$ by $\p(x)=x^{-1}$. \\
Prove: $G$ is abelian iff $\p$ is an automorphism.

\begin{description}
\item $\implies$ Assume $G$ is abelian

  Assume $\p(x)=\p(y)$ \\
  $x^{-1}=y^{-1}$ \\
  $x=y$ \\
  $\therefore\p$ is an injection

  Assume $y\in G$ \\
  $y^{-1}\in G$ \\
  Let $x=y^{-1}$ \\
  $\p(x)=\p(y^{-1})=(y^{-1})^{-1}=y$ \\
  $\therefore\p$ is a surjection and thus a bijection

  Assume $x,y\in G$ \\
  $\p(xy)=(xy)^{-1}=x^{-1}y^{-1}=\p(x)\p(y)$ \\
  $\therefore\p$ is a homomorphism and thus an isomorphism

  $\therefore \p$ is an automorphism
  
\item $\impliedby$ Assume $\p$ is an automorphism

  Assume $x,y\in G$ \\
  $\p$ is a homomorphism \\
  $\p(xy)=\p(x)\p(y)=x^{-1}y^{-1}=(yx)^{-1}=\p(yx)$ \\
  But $\p$ is a bijection and thus one-to-one, so $xy=yx$ \\
  $\therefore G$ is abelian
\end{description}

\subsection*{1.2.5}

Let $G$ be a group and $S\subseteq G$ such that $S\ne\emptyset$. Define a
relation $\r$ on $G$ by:
\[a\r b\iff ab^{-1}\in S\]
Prove: $\r$ is an equivalence relation $\iff S\le G$.

\begin{description}
\item $\implies$ Assume $\r$ is an equivalence relation

  $S$ is non-empty by assumption \\
  Assume $a\in S$ \\
  $ae^{-1}=ae=a\in S$ \\
  So $S=\bar{e}$

  Assume $a,b\in S$ \\
  $a\r e$ and $b\r e$ \\
  $e\r b$ (symmetry) \\
  $a\r b$ (transitivity) \\
  $ab^{-1}\in S$ \\
  $\therefore$ by the yucky subgroup test, $S\le G$.
  
\item $\impliedby$ Assume $S\le G$

  \begin{description}
  \item R: Assume $a\in G$

    $a^{-1}\in G$ \\
    $e\in S$ \\
    $aa^{-1}=e\in S$ \\
    $\therefore a\r a$

  \item S: Assume $a\r b$

    $ab^{-1}\in S$ \\
    $(ba^{-1})^{-1}\in S$ \\
    But $S$ is a group, so $ba^{-1}\in S$ \\
    $\therefore b\r a$

  \item T: Assume $a\r b$ and $b\r c$

    $ab^{-1}\in S$ and $bc^{-1}\in S$ \\
    But $S$ is a group, so by closure, $(ab^{-1})(bc^{-1})\in S$ \\
    $(ab^{-1})(bc^{-1})=a(b^{-1}b)c^{-1}=aec^{-1}=ac^{-1}\in S$ \\
    $\therefore a\r c$
  \end{description}
  $\therefore\,\r$ is an equivalence relation.
\end{description}

\subsection*{1.2.6}

Let $G$ be a group and let $S$ be a non-empty, finite, subset of $G$. \\
Prove: $S\le G\iff S$ is closed under the induced operation of $G$.

\begin{description}
\item $\implies$ Assume $S\le G$

  $S$ is a group under the induced operation of $G$ \\
  $\therefore S$ is closed under that operation.

\item $\impliedby$ Assume S is closed under the induced operation of $G$

  $S$ is a finite, non-empty subset of $G$ and closed by assumption \\
  $G$ is associative, and so $S$ is associative \\
  Thus, $S$ is a semigroup \\
  Assume $a,b,c\in S$ \\
  Assume $ac=bc$ \\
  $a,b,c\in G$ \\
  $c^{-1}\in G$ \\
  Thus, $a=b$, so right cancellation works \\
  Likewise, assume $ca=cb$ \\
  Thus, $a=b$, so left cancellation works \\
  So by HW 1.1.15, $S$ is a group \\
  $\therefore S\le G$
\end{description}

\subsection*{1.2.9}

Let $\p:G\to H$ be a homomorphism of groups and let $A\le G$ and $B\le H$.

\begin{enumerate}[label=\alph*)]
\item Prove:
  \begin{enumerate}[label=\arabic*)]
  \item $\ker\p\le G$

    $\ker\p=\{a\in G\mid\p(a)=e_H\}$

    Assume $a,b\in\ker\p$ \\
    $a,b\in G$ \\
    $ab\in G$ \\
    $\p(ab)=\p(a)\p(b)=e_He_H=e_H$ \\
    So $ab\in\ker\p$ \\
    $\therefore\ker\p$ is closed

    $\ker\p\subseteq G$ \\
    $G$ is associative \\
    $\therefore\ker\p$ is associative

    $\p(e_G)=e_H$ \\
    $\therefore e_G\in\ker\p$

    Assume $a\in\ker\p$ \\
    $a\in G$ and $a^{-1}\in G$ \\
    $\p(a^{-1})=\p(a)^{-1}=e_H^{-1}=e_H$ \\
    $\therefore a^{-1}\in\ker\p$

    $\therefore\ker\p\le G$

  \item $\p^{-1}[B]\le G$
    
    $\p^{-1}[B]=\{a\in G\mid\p(a)\in B\}$

    $B\le H$, so $e_H\in B$ \\
    $\p(e_G)=e_H\in B$ \\
    $\therefore e_G\in\p^{-1}[B]$ and $\p^{-1}[B]\ne\emptyset$

    Assume $a,b\in\p^{-1}[B]$ \\
    $\p(a)\in B$ and $\p(b)\in B$ and thus $\p(a)\p(b)\in B$\\
    $a,b\in G$ and thus $ab\in G$ \\
    $\p(ab)=\p(a)\p(b)\in B$ \\
    So $ab\in\p^{-1}[B]$ \\
    $\therefore\p^{-1}[B]$ is closed

    $\p^{-1}[B]\subseteq G$ \\
    $G$ is associative \\
    $\therefore\p^{-1}[B]$ is associative
\newpage
    Assume $a\in\p^{-1}[B]$ \\
    $\p(a)\in B$ \\
    But $B$ is a group, so $\p(a)^{-1}\in B$ \\
    $\p(a)^{-1}=\p(a^{-1})\in B$ \\
    $\therefore a^{-1}\in\p^{-1}[B]$

    $\therefore\p^{-1}[B]\le G$
  \end{enumerate}

\item Prove: $\p[A]\le H$

  $\p[A]=\{\p(a)\mid a\in A\}$

  $A\le G$, so $e_G\in A$ \\
  $\p(e_G)=e_H$ \\
  $\therefore e_H\in\p[A]$ and $\p[A]\ne\emptyset$
  
  Assume $a,b\in\p[A]$ \\
  $\exists\,x,y\in A,\p(x)=a$ and $\p(y)=b$ \\
  $xy\in A$ \\
  $\p(xy)\in\p[A]$\\
  $\p(xy)=\p(x)\p(y)=ab\in\p[A]$ \\
  $\therefore\p[A]$ is closed

  $\p[A]\subseteq H$ \\
  $H$ is associative \\
  $\therefore\p[A]$ is associative

  Assume $a\in\p[A]$ \\
  $\exists\,x\in A,\p(x)=a$ \\
  $x^{-1}\in A$ \\
  $\p(x^{-1})\in\p[A]$ \\
  $\p(x^{-1})=\p(x)^{-1}=a^{-1}$ \\
  $\therefore a^{-1}\in\p[A]$

  $\therefore\p[A]\le H$
\end{enumerate}

\subsection*{1.2.10}

Recall:

\bigskip

$\Z_2=\{0,1\}$

\bigskip

\begin{tabular}{c|cccc}
  $\bigoplus$ & $(0,0)$ & $(0,1)$ & $(1,0)$ & $(1,1)$ \\
  \hline
  $(0,0)$ & $(0,0)$ & $(0,1)$ & $(1,0)$ & $(1,1)$ \\
  $(0,1)$ & $(0,1)$ & $(0,0)$ & $(1,1)$ & $(1,0)$ \\
  $(1,0)$ & $(1,0)$ & $(1,1)$ & $(0,0)$ & $(0,1)$ \\
  $(1,1)$ & $(1,1)$ & $(1,0)$ & $(0,1)$ & $(0,0)$ \\
\end{tabular}
$\implies$
\begin{tabular}{c|cccc}
  $*$ & e & a & b & c \\
  \hline
  e & e & a & b & c \\
  a & a & e & c & b \\
  b & b & c & e & a \\
  c & c & b & a & e \\
\end{tabular}
$\implies K_4$

So, the possible subgroups are:

$\{e\}$ \\
$\{e,a\}$ \\
$\{e,b\}$ \\
$\{e,c\}$ \\
$\{e,a,b,c\}$

In order to be isomorphic to $\Z_4$, $K_4$ must be cyclic. However, $K_4$ has
only trivial (order 1) and proper (order 2) cyclic subgroups, and so
$K_4\not\iso\Z_4$.

\subsection*{1.2.11}

Let $G$ be group. Prove: $Z(G)$ is an abelian subgroup of $G$.

$Z(G)=\{a\in G\mid \forall\,x\in G,ax=xa\}$

$e\in G$ \\
$\forall\,x\in G,ex=xe$ \\
$\therefore e\in Z(G)$ and $Z(G)\ne\emptyset$

Assume $a,b\in Z(G)$ \\
$a,b\in G$ and so $ab\in G$ \\
Assume $x\in G$ \\
$(ab)x=axb=x(ab)$ \\
$ab\in Z(G)$ \\
$\therefore Z(G)$ is closed

$Z(G)\subseteq G$ \\
$G$ is associative \\
$\therefore Z(G)$ is associative

Assume $a\in Z(G)$ \\
$a\in G$ and $a^{-1}\in G$ \\
Assume $x\in G$ \\
$a^{-1}x=(x^{-1}a)^{-1}=(ax^{-1})^{-1}=xa^{-1}$ \\
$\therefore a^{-1}\in Z(G)$

$\therefore Z(G)\le G$

Assume $a,b\in Z(G)$ \\
$a,b\in G$ \\
By definition, $ab=ba$

$\therefore Z(G)$ is abelian.

\subsection*{1.2.13}

Let $\p:G\to H$ be a homomorphism of groups and $G=\cycle{a}$. \\
Prove: $\p$ is completely determined by $\p(a)$

\begin{lemma}
  Let $\p:G\to H$ be a homomorphism of groups:
  \[\forall\,a\in G,\forall\,n\in\Z,\p(a^n)=\p(a)^n\]
\end{lemma}

\begin{theproof}
  Assume $a\in G$ \\
  Assume $n\in\Z$
  
  \begin{description}
  \item Case 1: $n>0$

    Proof by induction on $n$

    \begin{description}
    \item Base: $n=1$

      $\p(a^1)=\p(a)=\p(a)^1$

    \item Assume $\p(a^n)=\p(a)^n$

    \item Consider $\p(a^{n+1})$

      $\p(a^{n+1})=\p(a^na)=\p(a^n)\p(a)=\p(a)^n\p(a)=\p(a)^{n+1}$
    \end{description}
    
  \item Case 2: $n=0$

    $\p(a^0)=\p(e_G)=e_H=\p(a)^0$

  \item Case 3: $n<0$

    Let $m=-n>0$ \\
    $\p(a^n)=\p(a^{-m})=\p((a^{-1})^m)=\p(a^{-1})^m=(\p(a)^{-1})^m=\p(a)^{-m}=
    \p(a)^n$
  \end{description}
\end{theproof}

\bigskip

Now, assume $b\in H$ \\
$\exists\,a^n\in G,\p(a^n)=\p(a)^n=b$ \\
$\therefore\p$ is completely determined by $\p(a)$

\subsection*{1.2.15}

Let $G$ be a group and $\aut{G}$ be the set of all automorphisms of $G$.
\begin{enumerate}[label=\alph*)]
\item Prove: $\aut{G}$ is a group under composition.

  Note that the identify function $i_G\in\aut{G}$ so $\aut{G}\ne\emptyset$.

  Assume $\o,\t\in\aut(G)$ \\
  $\o$ and $\t$ are bijections, so $\o\t$ is also a bijection \\
  $\o$ and $\t$ are homomorphisms \\
  Assume $x,y\in G$ \\
  $xy\in G$ \\
  $(\o\t)(xy)=\o[\t(xy)]=\o[\t(x)\t(y)]=\o[\t(x)]\o[\t(y)]=
  [(\o\t)(x)][(\o\t)(y)]$ \\
  So $\o\t$ is also a homomorphism, and thus an isomorphism, and thus an
  automorphism \\
  $\therefore\aut{G}$ is closed

  Composition is associative

  Assume $\o\in\aut{G}$ \\
  $\o$ is a bijection, and so $\o^{-1}$ exists such that
  $\o\o^{-1}=\o^{-1}\o=i_G$ \\
  $\therefore\aut{G}$ has inverses

  $\therefore\aut{G}$ is a group

\item Prove:
  \begin{enumerate}[label=\arabic*)]
  \item $\aut{\Z}\iso\Z_2$

    Since $\Z$ has two generators: $\pm1$, the possible automorphisms are
    defined by: $\p_1(1)=1$ (identity) and $\p_{-1}(1)=-1$:
    
    $\p_1(n)=n$ \\
    $\p_{-1}(n)=-n$ \\
    $(\p_{-1}\p_{-1})(n)=-(-n)=n=\p_1(n)$

    \begin{minipage}[t]{1.5in}
      \begin{tabular}{c|cc}
        $\circ$ & $\p_1$ & $\p_{-1}$ \\
        \hline
        $\p_1$ & $\p_1$ & $\p_{-1}$ \\
        $\p_{-1}$ & $\p_{-1}$ & $\p_1$ \\
      \end{tabular}
    \end{minipage}
    \begin{minipage}{0.4in}
      $\iso$
    \end{minipage}
    \begin{minipage}[t]{1in}
      \begin{tabular}{c|cc}
        $+$ & 0 & 1 \\
        \hline
        0 & 0 & 1 \\
        1 & 1 & 0 \\
      \end{tabular}
    \end{minipage}
    \begin{minipage}{0.4in}
      $=\Z_2$
    \end{minipage}

    \bigskip

  \item $\aut{\Z_6}\iso\Z_2$

    Since $\Z_6$ has two generators: $1,5$, the possible automorphisms are
    defined by: $\p_1(1)=1$ (identity) and $\p_5(1)=5$:
    
    $\p_1(n)=n$ \\
    $\p_5(n)=5n$ \\
    $(\p_5\p_5)(n)=5(5n)=n=\p_1(n)$

    \begin{minipage}[t]{1.5in}
      \begin{tabular}{c|cc}
        $\circ$ & $\p_1$ & $\p_5$ \\
        \hline
        $\p_1$ & $\p_1$ & $\p_5$ \\
        $\p_5$ & $\p_5$ & $\p_1$ \\
      \end{tabular}
    \end{minipage}
    \begin{minipage}{0.4in}
      $\iso$
    \end{minipage}
    \begin{minipage}[t]{1in}
      \begin{tabular}{c|cc}
        $+$ & 0 & 1 \\
        \hline
        0 & 0 & 1 \\
        1 & 1 & 0 \\
      \end{tabular}
    \end{minipage}
    \begin{minipage}{0.4in}
      $=\Z_2$
    \end{minipage}

    \bigskip

  \item $\aut{Z_8}\iso\Z_2\oplus\Z_2$

    Since $\Z_2\oplus\Z_2\iso K_4$, it is sufficient to show that
    $\aut{Z_8}\iso K_4$.

    Since $\Z_8$ has four generators: $1,3,5,7$, the possible automorphisms are
    defined by: $\p_1(1)=1$ (identity), $\p_3(1)=3$, $\p_5(1)=5$, and
    $\p_7(1)=7$:
    
    $\p_1(n)=n$ \\
    $\p_3(n)=3n$ \\
    $\p_5(n)=5n$ \\
    $\p_7(n)=7n$

    So $\aut{\Z_8}$ is a group of order 4, and thus must be isomorphic to
    either $\Z_4$ or $K_4$. But:
    
    $(\p_1\p_1)(n)=n=\p_1(n)$ \\
    $(\p_3\p_3)(n)=3(3n)=n=\p1(n)$ \\
    $(\p_5\p_5)(n)=5(5n)=n=\p1(n)$ \\
    $(\p_7\p_7)(n)=7(7n)=n=\p1(n)$

    So each element in the group is its own inverse, indicating that
    $\aut{\Z_8}\iso K_4$.

    \bigskip

  \item $\aut{Z_p}\iso\Z_{p-1}$, where $p$ is prime

    Since $p$ is prime, all non-zero elements of $\Z_p$ are relatively prime
    with $p$, and so $\Z_p$ has $p-1$ generators. Each generator corresponds
    to an automorphism determined by $\o_k(1)=k$ where $1\le k\le p-1$:

    $\o_k(n)=kn$

    So $\aut{Z_p}$ is a group under composition of order $p-1$.

    Let $\o_h,\o_k\in\aut{Z_p}$ \\
    $(o_ho_k)(n)=h(kn)=(hk)n=\o_{hk}(n)$

    Define $\p:\aut{Z_p}\to\zstar{p}{\cdot}$ by $\p(\o_k)=k$ \\
    $\p$ is clearly bijective \\
    Assume $\o_h,\o_k\in\aut{Z_p}$ \\
    $\p(\o_h\o_k)=\p(\o_{hk})=hk=\p(\o_h)\p(\o_k)$ \\
    So $p$ is a homomorphism and thus an isomorphism

    So $\aut{Z_p}\iso\zstar{p}{\cdot}$; however, $\zstar{p}{\cdot}$ is cyclic
    of order $p-1$ and is thus isomorphic to $\Z_{p-1}$.

    $\therefore\aut{Z_p}\iso\Z_{p-1}$
    
  \end{enumerate}

\item Describe $\aut{Z_n}$

  Let $\znx=\{k\in\N\mid1\le k<n\ \mbox{and}\ (k,n)=1\}$.

  Claim: $\bas{\znx}{\cdot}$ is a group:

  $(1,n)=1$ and thus $1\in\znx$ and $\znx\ne\emptyset$

  Assume $h,k\in\znx$ \\
  $(h,n)=1$ and $(k,n)=1$ \\
  ABC: $hk\notin\znx$ \\
  $n\mid hk$ \\
  But $n\nmid h$, so $n\mid k$ \\
  Contradiction! \\
  $\therefore hk\in\znx$ and $\znx$ is closed

  Multiplication $\pmod n$ is associative

  Assume $m\in\znx$ \\
  $(m,n)=1$ \\
  $\exists h,k\in\znx,hm+kn=1$ \\
  But $kn=0$, so $hm=1$ \\
  $h=m^{-1}\in\znx$ \\
  (Hmm: we did this in class, but how do we know that $h,k\in\znx$?)

  $\therefore\bas{\znx}{\cdot}$ is a group

  Now, consider $\aut{Z_n}$. It will have $\p(n)$ generators, where $\p(n)$ is
  the Euler totient function. Each generator is defined by $o_k(1)=k$, where
  $1\le k<n$ and $(k,n)=1$, and so the $k$ are indeed the set $\znx$, which is
  closed under multiplication.

  $\o_k(n)=kn$

  Assume $\o_h,\o_k\in\aut{Z_n}$ \\
  $h,k\in\znx$ \\
  $hk\in\znx$ \\
  $(\o_h\o_k)(n)=h(kn)=(hk)n=\o_{hk}(n)$

  Define $\p:\aut{Z_n}\to\znx$ by $\p(\o_k)=k$ \\
  $\p$ is clearly bijective \\
  $\p(\o_h\o_k)=\p(\o_{hk})=hk=\p(\o_h)\p(\o_k)$ \\
  Thus $\p$ is a homomorphism, and thus an isomorphism

  $\therefore\aut{Z_n}\iso\bas{\znx}{\cdot}$
\end{enumerate}

\end{document}
