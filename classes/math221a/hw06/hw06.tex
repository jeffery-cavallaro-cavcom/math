\documentclass[letterpaper,12pt,fleqn]{article}
\usepackage{matharticle}
\usepackage{enumitem}
\pagestyle{plain}
\begin{document}
Cavallaro, Jeffery \\
Math 221a \\
Homework \#6

\bigskip

\subsection*{2.2.1}

Prove: A finite abelian group $G$ that is not cyclic contains a subgroup which
is isomorphic to $\Z_p\times\Z_p$ for some prime $p$.

Let $\abs{G}=n$ \\
In order for finite abelian group $G$ to not be cyclic:
\begin{enumerate}
\item $n\ge4$
\item $n$ is not prime
\end{enumerate}
So, by Corollary II.2.4, $G$ contains a subgroup of order $p^rq^s$ where
$p$ and $q$ are primes \\
But if all $p$ and $q$ are distinct then $G\simeq\bigoplus_{k=1}^m\Z_{{p_k}^{r_k}}$,
which would make $G$ cyclic and thus a contradiction \\
Thus, $G$ must contain a subgroup isomorphic to $\Z_{p^r}\oplus\Z_{p^s}$ \\
But both $\Z_{p^r}$ and $\Z_{p^s}$ contain subgroups isomorphic to $\Z_p$ \\
So $\Z_{p^r}\oplus\Z_{p^s}$ contains a subgroup isomorphic to $\Z_p\oplus\Z_p$

Therefore, $G$ contains a subgroup isomorphic to $\Z_p\oplus\Z_p$.

\subsection*{2.2.12}

\begin{enumerate}[label={\alph*)}]
\item What are the invariant factors and elementary divisors of:
  \begin{enumerate}[label={\roman*)}]
  \item $\Z_2\oplus\Z_9\oplus\Z_{35}$

    $2=2$ \\
    $9=3^2$ \\
    $35=5\cdot7$

    \begin{tabular}{c}
      $2$ \\
      $3^2$ \\
      $5$ \\
      $7$
    \end{tabular}

    $2\cdot3^2\cdot5\cdot7=630$

    Invariant factors: $630$ \\
    Elementary divisors: $2,3^2,5,7$

  \item $\Z_{26}\oplus\Z_{42}\oplus\Z_{49}\oplus\Z_{200}\oplus\Z_{1000}$

    $26=2\cdot13$ \\
    $42=2\cdot3\cdot7$ \\
    $49=7^2$ \\
    $200=2^3\cdot5^2$ \\
    $1000=2^3\cdot5^3$

    \begin{tabular}{cccc}
      $2$ & $2$ & $2^3$ & $2^3$ \\
      & & & $3$ \\
      & & $5^2$ & $5^3$ \\
      & & $7$ & $7^2$ \\
      & & & $13$ \\
    \end{tabular}

    $2$ \\
    $2$ \\
    $2^3\cdot5^2\cdot7=1400$ \\
    $2^3\cdot3\cdot5^3\cdot7^2\cdot13=191100$

    Invariant factors: $2,2,1400,1911000$ \\
    Elementary divisors: $2,2,2^3,2^3,3,5^2,5^3,7,7^2,13$
  \end{enumerate}

\item Determine up to isomorphism all abelian groups of the following orders:
  \begin{enumerate}[label={\roman*)}]
  \item $64=2^6$
    \[\begin{array}{l}
    \Z_{64} \\
    \Z_{32}\oplus\Z_2 \\
    \Z_{16}\oplus\Z_{4} \\
    \Z_{16}\oplus\Z_{2}\oplus\Z_{2} \\
    \Z_8\oplus\Z_8 \\
    \Z_8\oplus\Z_4\oplus\Z_2 \\
    \Z_8\oplus\Z_2\oplus\Z_2\oplus\Z_2 \\
    \Z_4\oplus\Z_4\oplus\Z_4 \\
    \Z_4\oplus\Z_4\oplus\Z_2\oplus\Z_2 \\
    \Z_4\oplus\Z_2\oplus\Z_2\oplus\Z_2\oplus\Z_2 \\
    \Z_2\oplus\Z_2\oplus\Z_2\oplus\Z_2\oplus\Z_2\oplus\Z_2 \\
    \end{array}\]

  \item $96=2^5\cdot3$
    \[\begin{array}{l}
    \Z_{96} \\
    \Z_{32}\oplus\Z_3 \\
    \Z_{16}\oplus\Z_2\oplus\Z_3 \\
    \Z_8\oplus\Z_4\oplus\Z_3 \\
    \Z_8\oplus\Z_2\oplus\Z_2\oplus\Z_3 \\
    \Z_4\oplus\Z_4\oplus\Z_2\oplus\Z_3 \\
    \Z_4\oplus\Z_2\oplus\Z_2\oplus\Z_2\oplus\Z_3 \\
    \Z_2\oplus\Z_2\oplus\Z_2\oplus\Z_2\oplus\Z_2\oplus\Z_3 \\
    \end{array}\]
  \end{enumerate}

\item Determine all abelian groups of order $n$ for $n\le20$
  \[\begin{array}{llllll}
  1=1 & \{0\}\\
  2=2 & \Z_2 \\
  3=3 & \Z_3 \\
  4=2^2 & \Z_4 & \Z_2\oplus\Z_2 \\
  5=5 & \Z_5 \\
  6=2\cdot3 & \Z_6 & \Z_3\oplus\Z_2 \\
  7=7 & \Z_7 \\
  8=2^3 & \Z_8 & \Z_4\oplus\Z_2 & \Z_2\oplus\Z_2\oplus\Z_2 \\
  9=3^2 & \Z_9 & \Z_3\oplus\Z_3 \\
  10=2\cdot5 & \Z_{10} & \Z_5\oplus\Z_2 \\
  11=11 & \Z_{11} \\
  12=2^2\cdot3 & \Z_{12} & \Z_3\oplus\Z_4 & \Z_3\oplus\Z_2\oplus\Z_2 \\
  13=13 & \Z_{13} \\
  14=2\cdot7 & \Z_{14} & \Z_7\oplus\Z_2 \\
  15=3\cdot5 & \Z_{15} & \Z_5\oplus\Z_3 \\
  16=2^4 & \Z_{16} & \Z_8\oplus\Z_2 & \Z_4\oplus\Z_4 & \Z_4\oplus\Z_2\oplus\Z_2 &
  \Z_2\oplus\Z_2\oplus\Z_2\oplus\Z_2 \\
  17=17 & \Z_{17} \\
  18=2\cdot3^2 & \Z_{18} & \Z_2\oplus\Z_9 & \Z_2\oplus\Z_3\oplus\Z_3 \\
  19=19 & \Z_{19} \\
  20=2^2\cdot5 & \Z_{20} & \Z_5\oplus\Z_4 & \Z_5\oplus\Z_2\oplus\Z_2
  \end{array}\]
\end{enumerate}

\subsection*{2.2.13}

Show that the invariant factors of $\Z_m\oplus\Z_n$ are:
\begin{enumerate}[label={\alph*)}]
\item $mn$ if $(m,n)=1$

  Assume $(m,n)=1$ \\
  $m$ and $n$ have no common prime factors \\
  Let $m=\prod{p_i^{r_i}}$ and $n=\prod{q_j^{s_j}}$ where $p_i\ne q_j$ \\
  So no combinations of factors of $p_i$ and $q_j$ will result in a product
  that divides another product

  Therefore, the only invariant factor is $mn$
  
\item $(m,n)$ and $[m,n]$ if $(m,n)>1$

  Assume $(m,n)=d>1$ \\
  Let $m=m'd$ and $n=n'd$ \\
  Let $d=\prod{p_i^{a_i}}$, $m'=\prod{q_j^{b_j}}$, and $n'=\prod{r_k^{c_k}}$
  where $q_j\ne r_k$ \\
  Arranging the elementary factors:
  \[\begin{array}{cc}
  p_1^{a_1} & p_1^{a_1} \\
  p_2^{a_2} & p_2^{a_2} \\
  \vdots & \vdots \\
  & q_1^{b_1} \\
  & q_2^{b_2} \\
  & \vdots \\
  & r_1^{c_1} \\
  & r_2^{c_2} \\
  & \vdots
  \end{array}\]

  Therefore, there are two invariants, $(m,n)$ and $[m,n]$.
\end{enumerate}

\end{document}
